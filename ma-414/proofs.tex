\chapter{Proofs}
\section{Algebraic extensions}

\finextisalg*\label{prop:finextisalg2}
\begin{flushright}\hyperref[prop:finextisalg]{\upsym}\end{flushright}
\begin{proof}
    Let $\mathbb{K}/\mathbb{F}$ be a finite extension with $n \vcentcolon= \dim_{\mathbb{F}}(\mathbb{K}).$ Let $b \in \mathbb{K}$ be arbitrary. Consider the multiset $\{1, b, \ldots, b^{n}\}.$ It has $n + 1$ elements and thus, is linearly dependent. Thus, there exist $a_0, \ldots, a_{n} \in \mathbb{F}$ not all $0$ such that
    \begin{equation*} 
        a_0 + a_1b + \cdots + a_nb^n = 0.
    \end{equation*}
    Then, $f(x) \vcentcolon= a_0 + a_1b + \cdots + a_nx^n \in \mathbb{F}[x]$ is a non-zero polynomial such that $f(b) = 0.$
\end{proof}

\uniquemonicirred*\label{prop:uniquemonicirred2}
\begin{flushright}\hyperref[prop:uniquemonicirred]{\upsym}\end{flushright}
\begin{proof}
    Define $\psi : \mathbb{F}[x] \to \mathbb{K}$ by $p(x) \mapsto p(\alpha).$ Since $\alpha$ is algebraic, $I \vcentcolon= \ker(\psi)$ is non-zero.

    Since $\mathbb{F}[x]$ is a PID, we have $I = \langle f(x)\rangle$ for some $0 \neq f(x) \in \mathbb{F}[x].$ Since $\mathbb{F}[x]/I$ is isomorphic to a subring of $\mathbb{K},$ it is an integral domain and hence, $f(x)$ is irreducible. By scaling, we may assume that $f(x)$ is monic. Clearly, any other $g(x)$ as in the proposition is in the kernel and hence, $f(x) \mid g(x).$

    In particular, if $g(x)$ is irreducible and monic, then $f(x) \mid g(x) \implies g(x) = af(x)$ for some $a \in \mathbb{F}^\times.$ Since $g(x)$ is also monic, we have $a = 1.$
\end{proof}

\adjoiningalg*\label{prop:adjoiningalg2}
\begin{flushright}\hyperref[prop:adjoiningalg]{\upsym}\end{flushright}
\begin{proof}
    Consider the substitution homomorphism $\psi : \mathbb{F}[x] \to \mathbb{F}[\alpha]$ given by $p(x) \mapsto p(\alpha).$

    By \Cref{prop:uniquemonicirred}, we know that $\ker(\psi) = \langle f(x)\rangle.$ Since $f(x) \neq 0,$ the ideal $\langle f(x)\rangle$ is maximal. 

    Since $\psi$ is onto and $\ker(\psi)$ maximal, we see that $\mathbb{F}[\alpha]$ is in fact a field and hence, $\mathbb{F}[\alpha] = \mathbb{F}(\alpha).$

    Consider $B = \{1, \alpha, \ldots, \alpha^{n - 1}\}.$ \\
    Using $f(x),$ we may recursively write all higher powers of $\alpha$ as an $\mathbb{F}$-linear combination of elements of $B.$ Thus, $B$ spans $\mathbb{F}[\alpha].$ \\
    For linear independence, suppose that $a_0, \ldots, a_{n - 1} \in \mathbb{F}$satisfy
    \begin{equation*} 
        a_0 + a_1\alpha + \cdots + a_{n - 1}\alpha^{n - 1} = 0.
    \end{equation*}
    Then, we get a polynomial $g(x) = a_0 + a_1x + \cdots a_{n - 1}x^{n - 1} \in \mathbb{F}[x]$ satisfied by $\alpha.$ Since $\deg(g(x)) < \deg(f(x)),$ we see that $g(x) = 0,$ again by \Cref{prop:uniquemonicirred}.
\end{proof}

\isocarryingalphtobet*\label{prop:isocarryingalphtobet2}
\begin{flushright}\hyperref[prop:isocarryingalphtobet]{\upsym}\end{flushright}
\begin{proof}
    $(\implies)$ Let $\psi : \mathbb{F}(\alpha) \to \mathbb{F}(\beta)$ be as mentioned.\\
    Put $f(x) \vcentcolon= \irr(\alpha, \mathbb{F})$ and $g(x) \vcentcolon= \irr(\beta, \mathbb{F}).$ Then, 
    \[\begin{WithArrows}[displaystyle]
        0 &= \psi(0) \\
        &= \psi(f(\alpha)) \Arrow{$\psi$ is an $\mathbb{F}$-isomorphism} \\
        &= f(\psi(\alpha)) \\
        &= f(\beta).
    \end{WithArrows}\]
     Thus, $g(x) \mid f(x).$ Since both are irreducible and monic, $g(x) = f(x).$

     $(\impliedby)$ Let $f(x) \vcentcolon= \irr(\alpha, \mathbb{F}) = \irr(\beta, \mathbb{F}).$ \\
     The isomorphisms $\mathbb{F}(\alpha) \cong \mathbb{F}[x]/\langle f(x)\rangle \cong \mathbb{F}(\beta)$ are $\mathbb{F}$-isomorphisms and so is their composition.
\end{proof}

\towerlaw*\label{thm:towerlaw2}
\begin{flushright}\hyperref[thm:towerlaw]{\upsym}\end{flushright}
\begin{proof}
    If $\mathbb{K}/\mathbb{F}$ is a finite extension, then so are $\mathbb{K}/\mathbb{E}$ (pick a finite basis of $\mathbb{K}/\mathbb{F},$ it is a spanning set for $\mathbb{K}/\mathbb{E}$) and $\mathbb{E}/\mathbb{F}$ ($\mathbb{E}$ is an $\mathbb{F}$-subspace of $\mathbb{K}.$)

    Thus, if either of $\mathbb{K}/\mathbb{E}$ or $\mathbb{E}/\mathbb{F}$ is not a finite extension, then neither is $\mathbb{K}/\mathbb{F}.$

    Now, assume that both $n \vcentcolon= [\mathbb{K} : \mathbb{E}]$ and $m \vcentcolon= [\mathbb{E} : \mathbb{F}]$ are finite. Let $\{\alpha_i\}_{i = 1}^n \subset \mathbb{K}$ be an $\mathbb{E}$-basis and $\{\beta_j\}_{j = 1}^m \subset \mathbb{E}$ be an $\mathbb{F}$-basis.

    Put $B \vcentcolon= \{\alpha_i\beta_j : 1 \le i \le n,\; 1 \le j \le m\} \subset \mathbb{K}.$ We show that $B$ is an $\mathbb{F}$-basis of $\mathbb{K}.$

    \textbf{Spanning.} Let $a \in \mathbb{K}$ be arbitrary. Write 
    \begin{equation*} 
        a = \sum_{i = 1}^{n} a_i \alpha_i
    \end{equation*}
    for $a_i \in \mathbb{E}.$ For each $i = 1, \ldots, n,$ write
    \begin{equation*} 
        a_i = \sum_{j = 1}^{m} b_{ij} \beta_j
    \end{equation*}
    for $b_{ij} \in \mathbb{F}.$ Then,
    \begin{equation*} 
        a = \sum_{i = 1}^{n}\sum_{j = 1}^{m}b_{ij} (\alpha_i\beta_j)
    \end{equation*}
    is an $\mathbb{F}$-linear combination of elements of $B.$

    \textbf{Linear independence.} Let $\{b_{ij} : 1 \le i \le n,\; 1 \le j \le m\} \subset \mathbb{F}$ be such that
    \begin{equation*} 
        \sum_{\substack{1 \le i \le n \\ 1 \le j \le m}} b_{ij}\alpha_i\beta_j = 0.
    \end{equation*} 
    Group the above to get
    \begin{equation*} 
        \sum_{i = 1}^{n}\left[\sum_{j = 1}^{m}b_{ij} \alpha_i\right]\beta_j = 0.
    \end{equation*}
    Linear independence of $\{\beta_j\}$ forces $\sum_{j = 1}^{m}b_{ij} \alpha_i = 0$ for all $i.$ In turn, linear independence of $\{\alpha_i\}$ that forces each $b_{ij}$ to be $0.$

    Note that $B$ actually has cardinality $mn.$ (Why?) This finishes the proof.
\end{proof}

\adjoinalgsfinext*\label{prop:adjoinalgsfinext2}
\begin{flushright}\hyperref[prop:adjoinalgsfinext]{\upsym}\end{flushright}
\begin{proof}
    Consider the tower
    \begin{equation*} 
        \mathbb{F} \subset \mathbb{F}(\alpha_1) \subset \mathbb{F}(\alpha_1, \alpha_2) \subset \cdots \subset \mathbb{F}(\alpha_1, \ldots, \alpha_n).
    \end{equation*}
    At each stage, an element being adjoined is algebraic over the previous field. (\Cref{prop:decompalgisalg}.)

    Thus, each consecutive degree above is finite. (\Cref{cor:adjoinalgisfin}.)

    By the \nameref{thm:towerlaw}, so is the overall degree.
\end{proof}

\compalgisalg*\label{cor:compalgisalg2}
\begin{flushright}\hyperref[cor:compalgisalg]{\upsym}\end{flushright}
\begin{proof}
    Let $\alpha \in \mathbb{K}.$ Let $\irr(\alpha, \mathbb{E}) =\vcentcolon f(x) = a_0 + \cdots + a_{n - 1}x^{n - 1} + x^n.$

    Let $\mathbb{L} \vcentcolon= \mathbb{F}(a_0, \ldots, a_{n - 1}).$

    Then, $\mathbb{L}$ is finite over $\mathbb{F}$ since each $a_i \in \mathbb{E}$ is algebraic over $\mathbb{F}.$ Moreover, $0 \neq f(x) \in \mathbb{L}[x].$ Thus, $\alpha$ is algebraic over $\mathbb{L}$ and hence, $\mathbb{L}(\alpha)$ is finite over $\mathbb{L}.$

    By the \nameref{thm:towerlaw}, $\mathbb{L}/\mathbb{F}$ is finite and thus, $\alpha$ is algebraic over $\mathbb{F}.$ (\Cref{prop:finextisalg}.)
\end{proof}

\algclosureisfield*\label{cor:algclosureisfield2}
\begin{flushright}\hyperref[cor:algclosureisfield]{\upsym}\end{flushright}
\begin{proof}
    $\mathbb{F} \subset \mathbb{A}$ is clear. We show that $\mathbb{A}$ is a subfield. Let $\alpha, \beta \in \mathbb{A}$ with $\beta \neq 0.$ Then, $\mathbb{L} \vcentcolon= \mathbb{F}(\alpha, \beta)$ is a finite extension over $\mathbb{F}.$ \\
    Thus, all elements of $\mathbb{L}$ are algebraic over $\mathbb{F}.$ In particular, so are $\alpha \pm \beta,$ $\alpha\beta$ and $\alpha\beta^{-1}.$
\end{proof}

\intdomfinextfield*\label{prop:intdomfinextfield2}
\begin{flushright}\hyperref[prop:intdomfinextfield]{\upsym}\end{flushright}
\begin{proof}
    We only need to show that every non-zero element of $R$ has a multiplicative inverse (in $R$). Let $0 \neq a \in R$ be arbitrary. Since $\dim_{\mathbb{F}}(R) < \infty,$ there is a smallest $n \ge 1$ such that the set $\{1, a, \ldots, a^n\}$ is linearly dependent over $\mathbb{F}.$ Then, let $b_0, \ldots, b_{n} \in \mathbb{F}$ be not all zero such that
    \begin{equation*} 
        b_0 + b_1a + \cdots b_na^n = 0.
    \end{equation*} 
    If $b_n = 0,$ then the minimality of $n$ is contradicted. If $b_0 = 0,$ then we may cancel $a$ ($R$ is an integral domain and $a \neq 0$) and again contradict the minimality of $n.$ Thus, we get
    \begin{equation*} 
        a(b_1 + \cdots + b_na^{n - 1}) = -b_0.
    \end{equation*}
    This shows that the element
    \begin{equation*} 
        -\frac{1}{b_0}(b_1 + \cdots + b_na^{n - 1}) \in R
    \end{equation*}
    is a multiplicative inverse of $a.$
\end{proof}

\descofcompositum*\label{prop:descofcompositum2}
\begin{flushright}\hyperref[prop:descofcompositum]{\upsym}\end{flushright}
\begin{proof}
    Simple computations show that $\mathbb{L}$ is indeed a subring of $\mathbb{K}.$ If $\{\alpha_1, \ldots, \alpha_n\}$ and $\{\beta_1, \ldots, \beta_m\}$ are $\mathbb{F}$-bases for $\mathbb{E}_1$ and $\mathbb{E}_2,$ then clearly $\{\alpha_i\beta_j : 1 \le i \le n,\; 1 \le j \le m\}$ spans $\mathbb{L}$ over $\mathbb{F}.$ Thus, $\dim_{\mathbb{F}}(\mathbb{L}) \le mn = d.$ 

    Note that $\mathbb{L}$ is clearly the smallest subring of $\mathbb{K}$ containing $\mathbb{E}_1$ and $E_2.$ Since $\mathbb{L}$ is a subring of $\mathbb{K},$ it is an integral domain and hence, $\mathbb{L}$ is a field, by \Cref{prop:intdomfinextfield}. Thus, $\mathbb{L} = \mathbb{E}_1\mathbb{E}_2.$

    Lastly, note that $[\mathbb{E}_i : \mathbb{F}]$ divides $[\mathbb{L} : \mathbb{F}],$ in view of the \nameref{thm:towerlaw}. In particular, if $\gcd(m, n) = 1,$ then $mn \mid [\mathbb{L} : \mathbb{F}].$ Since $[\mathbb{L} : \mathbb{F}] \le mn,$ we are done.
\end{proof}

\rootcanbeadjoined*\label{thm:rootcanbeadjoined2}
\begin{flushright}\hyperref[thm:rootcanbeadjoined]{\upsym}\end{flushright}
\begin{proof}
    Let $g(x)$ be an irreducible factor of $f(x).$

    Put $\mathbb{K} = \mathbb{F}[x]/\langle g(x)\rangle.$ Since $g(x)$ is irreducible and non-zero, the quotient is indeed a field. Clearly, $\mathbb{F}$ is a subfield under the identification $a \mapsto \bar{a}.$ Moreover, $\bar{x}$ is a root of $g(x).$
\end{proof}

\splitfieldexists*\label{thm:splitfieldexists2}
\begin{flushright}\hyperref[thm:splitfieldexists]{\upsym}\end{flushright}
\begin{proof}
    Let $n \vcentcolon= \deg(f).$ By \Cref{thm:rootcanbeadjoined}, there exists a field $\mathbb{F}_1 \supset \mathbb{F}$ such that $f(x)$ has a root in $\mathbb{F}_1.$ Calling this root $a_1,$ we see that
    \begin{equation*} 
        f(x) = (x - a_1)f_1(x)
    \end{equation*}
    with $\deg(f_1) = n - 1.$ Continuing inductively, we get fields
    \begin{equation*} 
        \mathbb{F}_n \supset \cdots \supset \mathbb{F}_1 \supset \mathbb{F}
    \end{equation*}
    with $a_i \in \mathbb{F}_i,$ such that
    \begin{equation*} 
        f(x) = a(x - a_1) \cdots (x - a_n).
    \end{equation*}
    Then, $\mathbb{K} = \mathbb{F}(a_1, \ldots, a_n) \subset \mathbb{F}_n$ is a splitting field.
\end{proof}

\section{Symmetric Polynomials}
\FTSP*\label{thm:FTSP2}
\begin{flushright}\hyperref[thm:FTSP]{\upsym}\end{flushright}
\begin{proof}
    \textbf{Existence.} We apply induction on $n.$ The case $n = 1$ is clear since every polynomial is symmetric and $\sigma_1 = u_1.$ So, $g = f$ itself works\footnote{Being slightly sloppy since the indeterminates are different. We mean that you must take the same coefficients}.

    Suppose the theorem is true for $n - 1.$ Now, to prove the theorem for $n,$ apply induction on $\deg(f).$ If $f$ is constant, then again $g = f$ works. Suppose $\deg(f) \ge 1.$ Define
    \begin{equation*} 
        f^0 \vcentcolon= f(u_1, \ldots, u_{n - 1}, 0) \in R[u_1, \ldots, u_{n - 1}].
    \end{equation*}
    Then, $f^0$ is a symmetric polynomial in $n - 1$ variables. By induction hypothesis (on variables), there exists $g \in R[x_1, \ldots, x_{n - 1}]$ such that
    \begin{equation*} 
        f^0(u_1, \ldots, u_{n - 1}) = g(\sigma_1^0, \ldots, \sigma_{n - 1}^0).
    \end{equation*}
    Define $f_1 \in R[u_1, \ldots, u_n]$ by
    \begin{equation*} 
        f_1(u_1, \ldots, u_n) = f(u_1, \ldots, u_n) - g(\sigma_1, \ldots, \sigma_{n - 1}).
    \end{equation*}
    Then, $f_1(u_1, \ldots, u_{n - 1}, 0) = 0.$ Thus, $u_n \mid f_1.$ However, note that $f_1$ is symmetric and thus, $\sigma_n \mid f_1.$ Thus, we can write
    \begin{equation*} 
        f_1(u_1, \ldots, u_n) = \sigma_n h(u_1, \ldots, u_n)
    \end{equation*}
    for some $h \in R[u_1, \ldots, u_n].$ Since $\sigma_n$ is not a zero-divisor in $R[u_1, \ldots, u_n],$ we see that $h$ is also symmetric with $\deg(h) < \deg(f).$ Thus, by inductive hypothesis, $h$ is a polynomial in $\sigma_1, \ldots, \sigma_n$ and hence, $f$ is so.

    \textbf{Uniqueness.} It suffices to show that the elementary symmetric polynomials are algebraically independent. That is, to show that the map
    \begin{equation*} 
        \varphi : R[z_1, \ldots, z_n] \to R[u_1, \ldots, u_n]
    \end{equation*}
    defined by 
    \begin{equation*} 
        z_i \mapsto \sigma_i \andd \varphi|_R = \id_R
    \end{equation*}
    is an injection.

    We prove this by induction on $n.$ For $n = 1,$ it is clear since $\sigma_1 = u_1,$ an indeterminate. Assume that $n > 1$ and that the result is true for $n - 1.$ If $\varphi$ is not an injection, then we pick a nonzero polynomial $f(z_1, \ldots, z_n) \in \ker(\varphi)$ of least degree. Write $f$ as a polynomial in $z_n$ as
    \begin{equation*} 
        f(z_1, \ldots, z_n) = f_0(z_1, \ldots, z_{n - 1}) + \cdots + f_d(z_1, \ldots, z_{n - 1})z_n^d
    \end{equation*}
    with $f_d \neq 0.$ Minimality of $d$ (and the fact that $\sigma_n$ is not a zero-divisor) forces that $f_0 \neq 0.$ Since $f \in \ker(\varphi),$ we have
    \begin{equation*} 
        f_0(\sigma_1, \ldots, \sigma_{n - 1}) + \cdots + f_d(\sigma_1, \ldots, \sigma_{n - 1})\sigma_n^d = 0.
    \end{equation*}
    The above is an equality in $R[u_1, \ldots, u_n].$ Put $u_n = 0$ to get
    \begin{equation*} 
        f_0(\sigma_1^0, \ldots, \sigma_{n - 1}^0) = 0.
    \end{equation*}
    But the above shows that the corresponding $\varphi$ for $n - 1$ variables is not injective. A contradiction.
\end{proof}

\powersumformulae*\label{thm:powersumformulae2}
\begin{flushright}\hyperref[thm:powersumformulae]{\upsym}\end{flushright}
\begin{proof}
    Let $z$ be an indeterminate over $S \vcentcolon= R[u_1, \ldots, u_n].$ Note that 
    \begin{equation} \label{eq:001}
        (1 - u_1z) \cdots (1 - u_nz) = 1 - \sigma_1z + \cdots + (-1)^n \sigma_n z^n =\vcentcolon \sigma(z).
    \end{equation}
    Define $w(z) \in S[\![z]\!]$ as
    \begin{align*} 
        w(z) &= \sum_{k = 1}^{\infty} w_kz^k\\
        &= \sum_{k = 1}^{\infty}\left(\sum_{i = 1}^{n}u_i^k\right)z^k\\
        &= \sum_{i = 1}^{n} \left(\sum_{k = 1}^{\infty}(u_iz)^k\right)\\
        &= \sum_{i = 1}^{n} \frac{u_iz}{1 - u_iz}.
    \end{align*}
    Now, since $\sigma(z) = (1 - u_1z) \cdots (1 - u_nz),$ we get
    \begin{equation*} 
        \sigma'(z) = - \sum_{i = 1}^{n} \frac{u_i \sigma(z)}{1 - u_i z},
    \end{equation*}
    where we have taken the formal derivative in $S[\![z]\!].$ Rearranging the above gives
    \begin{equation*} 
        -\frac{z\sigma'(z)}{\sigma(z)} = \sum_{i = 1}^{n}\frac{u_i z}{1 - u_i z} = w(z)
    \end{equation*}
    and hence,
    \begin{equation*} 
        w(z)\sigma(z) = -z\sigma'(z).
    \end{equation*}
    Computing $\sigma'(z)$ from \Cref{eq:001} gives
    \begin{equation*} 
        w(z)\sigma(z) = \sigma_1z - 2\sigma_2z^2 + \cdots + (-1)^{n + 1}n\sigma_nz^n.
    \end{equation*}
    Comparing the coefficients of $z^k$ on both sides gives the result.
\end{proof}

\independencediscriminant*\label{prop:independencediscriminant2}
\begin{flushright}\hyperref[prop:independencediscriminant]{\upsym}\end{flushright}
\begin{proof}
    Let $r_1, \ldots, r_n \in \mathbb{K}$ be such that $f(x) = (x - r_1) \cdots (x - r_n).$

    Consider the Vandermonde matrix
    \begin{equation*} 
        M = \begin{bmatrix}
            1 & 1 & \cdots & 1\\
            r_1 & r_2 & \cdots & r_n\\
            r_1^2 & r_2^2 & \cdots & r_n^2\\
            \vdots & \vdots & \ddots & \vdots \\
            r_1^{n - 1} & r_2^{n - 1} & \cdots & r_n^{n - 1}\\
        \end{bmatrix}.
    \end{equation*}
    Then, $\disc_{\mathbb{K}}(f(x)) = (\det(M))^2 = \det(MM^{\mathsf{T}}).$ As before, let $\sigma_1, \ldots, \sigma_n \in \mathbb{F}[u_1, \ldots, u_n]$ be the elementary symmetric polynomials. Put
    \begin{equation*} 
        s_i \vcentcolon= \sigma_i(r_1, \ldots, r_n).
    \end{equation*}
    Then, note that
    \begin{equation*} 
        f(x) = x^n - s_1x^{n - 1} + \cdots + (-1)^ns_n
    \end{equation*}
    and hence, $s_i \in \mathbb{F}$ for all $i = 1, \ldots, n.$ Also, define
    \begin{equation*} 
        v_k \vcentcolon= r_1^k + \cdots + r_n^k
    \end{equation*}
    for all $k \ge 1.$ In view of \nameref{thm:powersumformulae}, we see that each $v_k \in \mathbb{F}$ as well. Moreover, note that
    \begin{equation*} 
        MM^{\mathsf{T}} = \begin{bmatrix}
            n & v_1 & \cdots & v_{n - 1}\\
            v_1 & v_2 & \cdots & v_n\\
            v_2 & v_3 & \cdots & v_{n + 1}\\
            \vdots & \vdots & \ddots & \vdots \\
            v_{n - 1} & v_n & \cdots & v_{2n - 2}\\
        \end{bmatrix}.
    \end{equation*} 
    Thus, $\disc_{\mathbb{K}}(f(x)) = \det(MM^{\mathsf{T}}) \in \mathbb{F}.$

    Note that $v_k$ can be calculated directly in terms of $s_i,$ the coefficients of $f(x).$ Thus, the discriminant does not depend on the choice of the splitting field.
\end{proof}

\discderivative*\label{prop:discderivative2}
\begin{flushright}\hyperref[prop:discderivative]{\upsym}\end{flushright}
\begin{proof}
    Note that
    \begin{equation*}
        f'(x) = \sum_{i = 1}^{n}\frac{f(x)}{x - r_i} = \sum_{i = 1}^{n}\prod_{\substack{j = 1 \\ j \neq i}}^{n}(x- r_j)
    \end{equation*}
    and thus,
    \begin{equation*} 
        f'(r_i) = \prod_{\substack{j = 1 \\ j \neq i}}^{n}(r_i - r_j).
    \end{equation*}
    The result now follows.
\end{proof}

\FTAprelim*\label{lem:FTAprelim2}
\begin{flushright}\hyperref[lem:FTAprelim]{\upsym}\end{flushright}
\begin{proof} 
    The first follows from intermediate value property. For the second, given $a + b \iota \in \mathbb{C}$ with $a, b \in \mathbb{R},$ define $c, d \in \mathbb{R}$ by
    \begin{equation*} 
        c \vcentcolon= \sqrt{\frac{1}{2}[a + \sqrt{a^2 + b^2}]} \andd d \vcentcolon= \sqrt{\frac{1}{2}[-a + \sqrt{a^2 + b^2}]}.
    \end{equation*}
    Then, $(c + d \iota)^2 = a + b\iota.$
\end{proof}
\FTA*\label{thm:FTA2}
\begin{flushright}\hyperref[thm:FTA]{\upsym}\end{flushright}
\begin{proof}
    Let $g(x) \in \mathbb{C}[x]$ be a non-constant polynomial. Then, $f(x) = g(x)\bar{g}(x)$ is a non-constant polynomial with real coefficients. Here, $\bar{g}(x)$ denotes the polynomial whose coefficients are complex conjugates of those of $g(x).$ Note that if $f(z) = 0$ for some $z \in \mathbb{C},$ then $g(z) = 0$ or $\bar{g}(z) = 0.$ If $\bar{g}(z) = 0,$ then $g(\bar{z}) = 0.$ In either case, $g$ has a complex root.

    Thus, it suffices to show that all non-constant real polynomials have a root in $\mathbb{C}.$ Given any $f(x) \in \mathbb{R}[x],$ we can write $\deg(f) = 2^nq$ for unique $n \ge 0$ and odd $q \in \mathbb{N}.$

    We prove the statement by induction on $n.$ If $n = 0,$ then $f$ has odd degree and hence, has a real root. \\
    Suppose $n \ge 1$ and the statement is true for $n - 1.$ Let $d \vcentcolon= \deg(f)$ and $\mathbb{K} = \mathbb{C}(\alpha_1, \ldots, \alpha_d)$ be a splitting field of $f(x)$ over $\mathbb{C},$ where the $\alpha_i$ are the roots of $f(x).$ For $r \in \mathbb{R},$ define
    \begin{equation*} 
        y_{ij}(r) = \alpha_i + \alpha_j + r\alpha_i\alpha_j
    \end{equation*}
    for $1 \le i \le j \le d.$ There are $\binom{d + 1}{2}$ such pairs $(i, j).$ Hence, the polynomial
    \begin{equation*} 
        h_r(x) \vcentcolon= \prod_{1 \le i \le j \le d} (x - y_{ij}(r))
    \end{equation*}
    has degree
    \begin{equation*} 
        \deg(h_r(x)) = \binom{d + 1}{2} = \frac{d}{2}(d + 1) = 2^{n - 1}\underbrace{q(d + 1)}_{\text{odd}}.
    \end{equation*}
    Note that the coefficients of $h_r(x)$ are elementary symmetric polynomials in $y_{ij}$s. Thus, they are symmetric polynomials in $\alpha_i, \ldots, \alpha_d.$ Hence, they are polynomials in the coefficients of $f(x).$ Thus, $h(x) \in \mathbb{R}[x].$ By inductive hypothesis (on $n$), we see that $h_r(x)$ has a root $z_r \in \mathbb{C} \subset \mathbb{K}.$ Thus, $z_r = y_{i(r)j(r)}(r)$ for some pair $(i(r), j(r))$ with $1 \le i(r) \le j(r) \le d.$

    Let $P = \{(i, j) : 1 \le i \le j \le d\}$ and define $\varphi : \mathbb{R} \to P$ by $r \mapsto (i(r), j(r)).$ Since $P$ is finite and $\mathbb{R}$ is not, $\varphi$ is not one-one and thus, there exist $c \neq d \in \mathbb{R}$ with
    \begin{equation*} 
        (i(c), j(c)) = (i(d), j(d)) =\vcentcolon (a, b) \in P.
    \end{equation*}
    Thus,
    \begin{equation*} 
        z_c = \alpha_a + \alpha_b + c\alpha_a\alpha_b \andd z_d = \alpha_a + \alpha_b + d\alpha_a\alpha_b.
    \end{equation*}
    Note that a priori, we only know that $\alpha_a, \alpha_b \in \mathbb{K}.$ But note that
    \begin{equation*} 
        \alpha_a\alpha_b = \frac{z_c - z_d}{d - c} \in \mathbb{C}
    \end{equation*}
    and consequently,
    \begin{equation*} 
        \alpha_a + \alpha_b = z_c - c\alpha_a\alpha_b \in \mathbb{C}.
    \end{equation*}
    Thus, $\alpha_a\alpha_b$ and $\alpha_a + \alpha_b \in \mathbb{C}.$ However, these are roots of the quadratic
    \begin{equation*} 
        x^2 - (\alpha_a + \alpha_b)x + \alpha_a\alpha_b \in \mathbb{C}[x].
    \end{equation*}
    Thus, $\alpha_a \in \mathbb{C}.$ But $\alpha_a$ was a root of $f(x),$ as desired.
\end{proof}

\section{Algebraic Closure of a Field}
\alglcosureinalgclosedisclosed*\label{prop:alglcosureinalgclosedisclosed2}
\begin{flushright}\hyperref[prop:alglcosureinalgclosedisclosed]{\upsym}\end{flushright}
\begin{proof}
    By \Cref{cor:algclosureisfield}, we already know that $\mathbb{A}/\mathbb{F}$ is actually an algebraic extension. We just need to show that $\mathbb{A}$ is algebraically closed. To this end, let $f(x) \in \mathbb{A}[x]$ be non-constant. Then, $f(x)$ has a root $\alpha \in \mathbb{K}.$ But then, $\alpha$ is algebraic over $\mathbb{A}$ and hence, over $\mathbb{F}.$ (\Cref{cor:compalgisalg}.) Thus, $\alpha \in \mathbb{A}.$
\end{proof}

\unionoffields*\label{lem:unionoffields2}
\begin{flushright}\hyperref[lem:unionoffields]{\upsym}\end{flushright}
\begin{proof}
    The operations are clearly well-defined. It is easy to see that the desired commutative and associative laws hold since they hold in each $\mathbb{F}_i.$ The $0$ and $1$ are those of each $\mathbb{F}_i.$ The appropriate inverses of any $a \in \mathbb{F}$ also exist in any $\mathbb{F}_i$ containing $a.$ The last sentence is also easy to check.
\end{proof}
\algclosedext*\label{thm:algclosedext2}
\begin{flushright}\hyperref[thm:algclosedext]{\upsym}\end{flushright}
\begin{proof}
    We first show that given any field $\mathbb{F},$ we can create a field $\mathbb{F}_1 \supset \mathbb{F}$ containing roots of any non-constant polynomial in $\mathbb{F}[x].$ Let $S$ be a set of indeterminates which are in one-to-one correspondence with set of all polynomials in $\mathbb{F}[x]$ with degree $\ge 1.$ Let $x_f \in S$ denote the indeterminate corresponding to $f.$

    Consider the (very large) polynomial ring $\mathbb{F}[S].$ Let 
    \begin{equation*} 
        I = \langle f(x_f)  : f \in \mathbb{F}[x],\;\deg(f) \ge 1\rangle
    \end{equation*}
    be the ideal generated by the polynomials $f(x_f) \in \mathbb{F}[S].$ We contend that $1 \notin I.$ Suppose the contrary. Then,
    \begin{equation*} 
        1 = g_1 f_1(x_{f_1}) + \cdots + g_n f_n(x_{f_n})
    \end{equation*}
    for some $g_1, \ldots, g_n \in \mathbb{F}[S].$ Note that these polynomials $g_j$ only involve finitely many variables. Let $x_i \vcentcolon= x_{f_i}$ for $i = 1, \ldots, n$ and let $x_{n + 1}, \ldots, x_m$ be the remaining variables in $g_1, \ldots, g_n.$ Then, we have
    \begin{equation*} 
        \sum_{i = 1}^{n} g_i(x_1, \ldots, x_n, x_{n + 1}, \ldots, x_m)f_i(x_i) = 1.
    \end{equation*}
    Now, let $\mathbb{E} \supset \mathbb{F}$ be an extension containing roots $\alpha_i$ of $f_i.$ (Note that $\deg(f_i) \ge 1$ and thus, we may use \Cref{thm:rootcanbeadjoined}.) Then, putting $x_i = \alpha_i$ for $i = 1, \ldots, n$ and putting $x_{n + 1} = \cdots = x_m = 0$ in the above equation gives a contradiction.

    Thus, $1 \notin I$ and hence, $I$ is a proper ideal of $\mathbb{F}[S].$ Thus, it is contained in some maximal ideal $\mathfrak{m} \subset \mathbb{F}[S].$ Put $\mathbb{F}_1 \vcentcolon= \mathbb{F}[S]/\mathfrak{m}.$ Then, $\mathbb{F}_1$ is a field extension of $\mathbb{F}.$ \\
    Note that $\overline{x_f} = x_f + \mathfrak{m} \in \mathbb{F}_1$ is a root of $f(x) \in \mathbb{F}[x].$ Thus, we have constructed a field $\mathbb{F}_1$ in which every non-constant polynomial of $\mathbb{F}[x]$ has a root.

    Repeating the procedure, we get fields 
    \begin{equation*} 
        \mathbb{F} = \mathbb{F}_0 \subset \mathbb{F}_1 \subset \mathbb{F}_2 \subset \mathbb{F}_3 \subset \cdots
    \end{equation*} 
    such that every non-constant polynomial in $\mathbb{F}_i$ has a root in $\mathbb{F}_{i + 1}.$

    Now, put $\mathbb{K} = \bigcup_{i \ge 0}\mathbb{F}_i.$ This is a field as per \Cref{lem:unionoffields}, having each $\mathbb{F}_i$ as a subfield. 

    Now, if $f(x) \in \mathbb{K}[x],$ then $f(x) \in \mathbb{F}_n[x]$ for some $n.$ This has a root in $\mathbb{F}_{n + 1} \subset \mathbb{K},$ as desired.
\end{proof}

\algclosure*\label{cor:algclosure2}
\begin{flushright}\hyperref[cor:algclosure]{\upsym}\end{flushright}
\begin{proof}
    Let $\mathbb{L} \supset \mathbb{F}$ be algebraically closed. (Existence given by \Cref{thm:algclosedext}.) Define
    \begin{equation*} 
        \mathbb{K} \vcentcolon= \{\alpha \in \mathbb{L} : \alpha \text{ is algebraic over } \mathbb{F}\}.
    \end{equation*}
    By \Cref{prop:alglcosureinalgclosedisclosed}, $\mathbb{K}$ is an algebraic closure of $\mathbb{F}.$
\end{proof}

\rootsandextensions*\label{prop:rootsandextensions2}
\begin{flushright}\hyperref[prop:rootsandextensions]{\upsym}\end{flushright}
\begin{proof}
    First, we note that the map is indeed well-defined. Let $\tau$ be an embedding extending $\sigma.$ Then,
    \begin{equation*} 
        \tau(p(\alpha)) = p^{\sigma}(\tau(\alpha)) = 0
    \end{equation*}
    and thus, $\tau(\alpha)$ is indeed a root of $p^{\sigma}.$ 

    Now, let $\beta \in L$ be such that $p^{\sigma}(\beta) = 0.$ Define $\tau_{\beta} : \mathbb{F}(\alpha) \to \mathbb{L}$ by $\tau_{\beta}(f(\alpha)) = f^{\sigma}(\beta)$ for $f(x) \in \mathbb{F}[x].$\footnote{Note that elements of $\mathbb{F}(\alpha)$ are precisely polynomials in $\alpha.$} We now show that $\tau_{\beta}$ is well-defined. 

    Suppose $f(\alpha) = g(\alpha).$ Then, $p(x) \mid f(x) - g(x)$ and hence, $p^{\sigma}(x) \mid f^{\sigma}(x) - g^{\sigma}(x).$ Thus, $f^{\sigma}(\beta) = g^{\sigma}(\beta).$ Thus, $\tau_{\beta}$ is well-defined. It is clearly a homomorphism (and hence, an embedding). Moreover, it extends $\sigma.$

    It is now easily seen that $\beta \mapsto \tau_{\beta}$ is a two-sided inverse of the map $\tau \mapsto \tau(\alpha).$
\end{proof}

\extendtoalgextension*\label{thm:extendtoalgextension2}
\begin{flushright}\hyperref[thm:extendtoalgextension]{\upsym}\end{flushright}
\begin{proof}
    Consider the set
    \begin{equation*} 
        \Sigma \vcentcolon= \{(\mathbb{E}, \tau) \mid \mathbb{F} \subset \mathbb{E} \subset \mathbb{K} \text{ are fields and } \tau : \mathbb{E} \to \mathbb{L} \text{ such that }\tau|_{\mathbb{F}} = \sigma\}.
    \end{equation*}
    Note that $\Sigma \neq \emptyset$ since $(\mathbb{F}, \sigma) \in \Sigma.$ Define the relation $\le$ on $\Sigma$ by
    \begin{equation*} 
        (\mathbb{E}, \tau) \le (\mathbb{E}', \tau') \iff \mathbb{E} \subset \mathbb{E}' \text{ and } \tau'|_{\mathbb{E}} = \tau.
    \end{equation*}
    Then, $(\Sigma, \le)$ is a partially ordered set. Moreover, if $\Lambda = \{(\mathbb{E}_\alpha, \tau_\alpha)\}_{\alpha \in I}$ is a chain in $\Sigma,$ then $\mathbb{E} \vcentcolon= \bigcup_{\alpha \in I}\mathbb{E}_\alpha$ is a subfield of $\mathbb{K}$ and $\tau : \mathbb{E} \to \mathbb{L}$ defined as $\tau(x) \vcentcolon= \tau_\alpha(x)$ for $x \in \mathbb{F}_\alpha$ is well-defined. (The proof is similar to that of \Cref{lem:unionoffields}.) Moreover, $(\mathbb{E}, \tau)$ is an upper bound of $\Lambda.$   

    Thus, by Zorn's lemma, there exists a maximal element $(\mathbb{E}, \tau) \in \Sigma.$ We contend that $\mathbb{E} = \mathbb{K}.$ If not, then pick $\alpha \in \mathbb{K} \setminus \mathbb{E}.$ By \Cref{prop:rootsandextensions}, we can extend $\tau$ to an embedding $\tau' : \mathbb{E}(\alpha) \to \mathbb{L}.$ But this contradicts maximality of $(\mathbb{E}, \tau).$

    Now, suppose that $\mathbb{K}$ is an algebraic closure of $\mathbb{F}$ and $\mathbb{L}$ of $\sigma(\mathbb{F}).$ We have
    \begin{equation*} 
        \sigma(\mathbb{F}) \subset \tau(\mathbb{K}) \subset \mathbb{L}
    \end{equation*}
    and thus, $L/\tau(\mathbb{K})$ is also algebraic. But $\tau(\mathbb{K})$ is also algebraically closed and thus, $\mathbb{L} = \tau(\mathbb{K}).$
\end{proof}

\isosplitting*\label{thm:isosplitting2}
\begin{flushright}\hyperref[thm:isosplitting]{\upsym}\end{flushright}
\begin{proof}
    Let $\overline{\mathbb{E}}$ be an algebraic closure of $\mathbb{E}.$ Then, it is also one of $\mathbb{F}.$ Thus, there exists an embedding $\tau : \mathbb{E}' \to \overline{\mathbb{E}}$ extending the inclusion $i : \mathbb{F} \hookrightarrow \overline{\mathbb{E}}.$

    Let $f(x) = a(x - \alpha_1) \cdots (x - \alpha_n)$ be a factorisation of $f(x)$ in $\mathbb{E}'[x].$ Then,
    \begin{equation*} 
        f^{\tau}(x) = a(x - \tau(\alpha_1)) \cdots (x - \tau(\alpha_n)) \in \overline{\mathbb{E}}[x].
    \end{equation*}
    (Note that $a \in \mathbb{F}^\times.$)
    Note that we have $\mathbb{E}' = \mathbb{F}(\alpha_1, \ldots, \alpha_n)$ and so, $\tau(\mathbb{E}') = \mathbb{F}(\tau(\alpha_1), \ldots, \tau(\alpha_n)).$ Thus, $\tau(\mathbb{E}')$ is a splitting field of $f^{\tau}.$ But $f^{\tau} = f$ since $f(x) \in \mathbb{F}[x]$ and $\tau$ extends the inclusion map. Thus, $\tau(\mathbb{E}') = \mathbb{E},$ since any algebraic closure contains a unique splitting field.
\end{proof}

\section{Separable extensions}
\multindepsplitting*\label{prop:multindepsplitting2}
\begin{flushright}\hyperref[prop:multindepsplitting]{\upsym}\end{flushright}
\begin{proof}
    Let $\mathbb{E}$ and $\mathbb{K}$ be splitting fields for $f(x)$ over $\mathbb{F}.$ By \Cref{thm:isosplitting}, there exists an $\mathbb{F}$-isomorphism $\tau : \mathbb{E} \to \mathbb{K}.$ In turn, we get an isomorphism
    \begin{align*} 
        \varphi_\tau : \mathbb{E}[x] &\to \mathbb{K}[x]\\
        \sum a_i x^i &\mapsto \sum \tau(a_i) x^i.
    \end{align*}
    Now, let $f(x) = \prod_{i = 1}^{g}(x - r_i)^{e_i}$ be the unique factorisation of $f(x)$ in $\mathbb{E}[x].$ The above isomorphism shows that 
    \begin{equation*} 
        f(x)= \prod_{i = 1}^{g}(x - \tau(r_i))^{e_i}
    \end{equation*}
    is the unique factorisation of $f(x)$ in $\mathbb{K}[x].$ The result follows.
\end{proof}

\derivcritreproot*\label{prop:derivcritreproot2}
\begin{flushright}\hyperref[prop:derivcritreproot]{\upsym}\end{flushright}
\begin{proof}
    \forward If $r$ is a repeated root, then write $f(x) = (x - r)^2g(x)$ for $g \in \mathbb{E}[x].$ Then, taking the derivative gives
    \begin{equation*} 
        f'(x) = 2(x - r)g(x) + (x - r)^2g'(x).
    \end{equation*}
    Thus, $f'(r) = 0.$

    \backward Write $f(x) = (x - r)g(x).$ Then,
    \begin{equation*} 
        0 = f'(r) = (r - r)g'(r) + g(r) = g(r).
    \end{equation*}
    Thus, $(x - r) \mid g(x)$ and hence, $(x - r)^2 \mid f(x).$
\end{proof}

\derivcritsep*\label{thm:derivcritsep2}
\begin{flushright}\hyperref[thm:derivcritsep]{\upsym}\end{flushright}
\begin{proof}
    Let $\mathbb{E}$ be a splitting field of $f(x).$
    \begin{enumerate}[leftmargin=*]
        \item Let $r \in \mathbb{E}$ be a root of $f(x).$ Then, $f'(r) = 0,$ by hypothesis and thus, $r$ is a repeated root, by \Cref{prop:derivcritreproot}.
        %
        \item Suppose $f'(x) \neq 0.$\\
        \forward Suppose $f(x)$ has simple roots. We need to show that $f(x)$ and $f'(x)$ have no common root. Let $r$ be a root of $f(x).$ Then $f'(r) \neq 0,$ by \Cref{prop:derivcritreproot}.

        \backward Suppose $\gcd(f(x), f'(x)) = 1$ and $r \in \mathbb{E}$ is an arbitrary root of $f(x).$ Then, $f'(r) \neq 0.$ Thus, $r$ is a simple root. \qedhere
    \end{enumerate}
\end{proof}

\irredsepderiv*\label{prop:irredsepderiv2}
\begin{flushright}\hyperref[prop:irredsepderiv]{\upsym}\end{flushright}
\begin{proof}
    Let $\mathbb{E}$ be a splitting field of $f(x)$ over $\mathbb{F}.$
    \begin{enumerate}[leftmargin=*]
        \item \forward $f(x)$ has no repeated roots and thus, $f'(x) \neq 0,$ by \Cref{thm:derivcritsep}.

        \backward Suppose $f'(x) \neq 0$ and $f(x)$ has a repeated root $r \in \mathbb{E}.$ Then, by \Cref{prop:derivcritreproot}, $f'(r) = 0.$ Thus, $g(x) \vcentcolon= \gcd(f(x), f'(x)) \neq 1.$ Irreducibility of $f(x)$ forces $f(x) = g(x).$ But then, $f(x) \mid f'(x),$ which is a contradiction since $\deg(f'(x)) < \deg(f(x)).$
        %
        \item If $f(x)$ is non-constant, then $f'(x) \neq 0.$ The previous part applies. \qedhere
    \end{enumerate} 
\end{proof}

\xppolyirredorroot*\label{prop:xppolyirredorroot2}
\begin{flushright}\hyperref[prop:xppolyirredorroot]{\upsym}\end{flushright}
\begin{proof}
    Suppose $f(x)$ is not irreducible. Write $f(x) = g(x)h(x)$ with $1 \le \deg(g(x)) =\vcentcolon m < p.$ Let $b \in \mathbb{E}$ be a root in a splitting field $\mathbb{E}$ of $f(x)$ over $\mathbb{F}.$ Then, $b^p = a.$ Thus, $f(x)$ factorises in $\mathbb{E}[x]$ as
    \begin{equation*} 
        f(x) = x^p - b^p = (x - b)^p.
    \end{equation*}
    Since $\mathbb{E}[x]$ is a UFD, we see that $g(x) = (x - b)^m.$ (We may assume that $g(x)$ is monic.) However, note that the coefficient of $x^{m - 1}$ is $mb.$ By assumption, $mb \in \mathbb{F}.$ Since $1 \le m < p,$ we see that $b \in \mathbb{F}.$ Thus, $a = b^p \in \mathbb{F}^p.$     
\end{proof}

\nonseppowerp*\label{prop:nonseppowerp2}
\begin{flushright}\hyperref[prop:nonseppowerp]{\upsym}\end{flushright}
\begin{proof}
    Since $f(x)$ is irreducible and not separable, we must have $f'(x) = 0.$ Write
    \begin{equation*} 
        f(x) = a_0 + a_1x + \cdots + a_nx^n
    \end{equation*}
    and note that
    \begin{equation*} 
        0 = f'(x) = a_1 + 2a_2x + \cdots + n a_n x^{n - 1}.
    \end{equation*}
    Thus, $ka_k = 0$ for all $k = 1, \ldots, n.$ If $\gcd(k, p) = 1,$ then we may cancel $k$ to see that $a_k = 0$ whenever $p \nmid k.$ Thus, $f(x)$ is of the form
    \begin{equation*} 
        f(x) = a_0 + a_px^p + \cdots + a_{mp} x^{mp}
    \end{equation*}
    for some $m \in \mathbb{N}.$ Thus, $g(x) = a_0 + a_p x + \cdots + a_{mp} x^m$ works.
\end{proof}

\perfectiffppower*\label{thm:perfectiffppower2}
\begin{flushright}\hyperref[thm:perfectiffppower]{\upsym}\end{flushright}
\begin{proof}
    \forward Suppose $\mathbb{F} \neq \mathbb{F}^p.$ Pick $\alpha \in \mathbb{F} \setminus \mathbb{F}^p.$ Then, $x^p - \alpha$ is irreducible (by \Cref{prop:xppolyirredorroot}) but not separable, by \Cref{prop:irredsepderiv}.

    \backward Suppose $\mathbb{F} = \mathbb{F}^p$ and $f(x) \in \mathbb{F}[x]$ is irreducible and not separable. By \Cref{prop:nonseppowerp}, we can write 
    \begin{equation*} 
        f(x) = \sum_{i = 0}^{m} a_i x^{ip}.
    \end{equation*} 
    Let $b_i \in \mathbb{F}$ be such that $a_i = b_i^p.$ Then,
    \begin{equation*} 
        f(x) = \sum_{i = 0}^{m} a_i x^{ip} = \sum_{i = 0}^{m} b_i^p x^{ip} = \left(\underbrace{\sum_{i = 0}^{m}b_i x^i}_{\in \mathbb{F}[x]}\right)^p,
    \end{equation*}
    contradicting the irreducibility of $f(x)$ in $\mathbb{F}[x].$
\end{proof}

\finitefieldperfect*\label{cor:finitefieldperfect2}
\begin{flushright}\hyperref[cor:finitefieldperfect]{\upsym}\end{flushright}
\begin{proof}
    Let $\mathbb{F}$ be a finite field of characteristic $p > 0.$ We show that $\mathbb{F} = \mathbb{F}^p.$ 

    Note that $\md{\mathbb{F}} = p^n$ for some $n \in \mathbb{N}.$ Thus, by Lagrange's theorem from group theory, we see that $\alpha^{p^n - 1} = 1$ for all $\alpha \in \mathbb{F}^\times.$ Thus, $\alpha^{p^n} = \alpha$ for all $\alpha \in \mathbb{F}.$ (This holds for $\alpha = 0$ as well.)

    Thus, given any arbitrary $\alpha \in \mathbb{F},$ put $\beta = \alpha^{p^{n - 1}}$ to get $\alpha = \beta^p \in \mathbb{F}^p.$
\end{proof}

\samemultirredpoly*\label{prop:samemultirredpoly2}
\begin{flushright}\hyperref[prop:samemultirredpoly]{\upsym}\end{flushright}
\begin{proof}
    Let $\overline{\mathbb{F}} \supset \mathbb{F}$ be an algebraic closure of $\mathbb{F}.$ Let $\alpha, \beta \in \overline{\mathbb{F}}$ be roots of $f.$ We have an $\mathbb{F}$-isomorphism $\sigma : \mathbb{F}(\alpha) \to \mathbb{F}(\beta)$ determined by $\alpha \mapsto \beta.$ 

    Thus, $\sigma$ can be extended to an automorphism $\tau$ of $\overline{\mathbb{F}}.$ Then, write $f(x) = (x - \alpha)^mh(x)$ where $m$ is the multiplicity of $\alpha$ and $h(x) \in \overline{\mathbb{F}}[x].$ Applying $\tau,$ we get
    \begin{equation*} 
        f(x) = f^\tau(x) = (x - \beta)^m h^\tau(x).
    \end{equation*}
    Thus, the multiplicity of $\beta$ is at least $m.$ By symmetry, we have equality.

    If $\chr(\mathbb{F}) = 0,$ then $f(x)$ is separable (\Cref{prop:irredsepderiv}) and thus, all roots are simple.

    Now, assume that $\chr(\mathbb{F}) =\vcentcolon p > 0.$ Let $n \in \mathbb{N}_0$ be the largest such that there exists a polynomial $g(x) \in \mathbb{F}[x]$ with $f(x) = g(x^{p^n}).$ (Note that we can take $g = f$ and $n = 0$ if no positive $n$ exists.)

    Then, $g$ is irreducible since $f$ is so. Moreover, $g$ must be separable. Indeed, if not, then we can write $g(x) = h(x^p)$ for some $h(x) \in \mathbb{F}[x],$ by \Cref{prop:nonseppowerp}. Then, $f(x) = h(x^{p^{n + 1}})$ contradicting maximality of $n.$

    Thus, $g(x)$ factors in $\overline{\mathbb{F}}$ as $g(x) = (x - r_1) \cdots (x - r_g)$ for distinct $r_g.$ Since $\overline{\mathbb{F}}$ is algebraically closed, we can find $s_1, \ldots, s_g$ necessarily distinct such that $s_i^{p^n} = r_i.$ Then, we have
    \begin{equation*} 
        f(x) = g(x^{p^n}) = (x - s_1)^{p^n} \cdots (x - s_g)^{p^n},
    \end{equation*}
    as desired.
\end{proof}

\separabledegreedef*\label{thm:separabledegreedef2}
\begin{flushright}\hyperref[thm:separabledegreedef]{\upsym}\end{flushright}
\begin{proof}
    If $\widetilde{\sigma} \in S_\sigma,$ then for any $x \in \mathbb{F},$ we have
    \begin{equation*} 
        (\lambda \circ \widetilde{\sigma})(x) = \lambda(\sigma(x)) = (\tau \circ \sigma^{-1})(\sigma(x)) = \tau(x).
    \end{equation*}
    Thus, $\psi$ actually maps into $S_\tau.$ Since $\lambda$ is an isomorphism, $\psi$ is easily seen to be a bijection. Explicitly, the inverse of $\psi$ can be seen to be $\widetilde{\tau} \mapsto \lambda^{-1} \circ \tau.$
\end{proof}

\towerlawsep*\label{thm:towerlawsep2}
\begin{flushright}\hyperref[thm:towerlawsep]{\upsym}\end{flushright}
\begin{proof}
    First, we show that the separable degree is multiplicative. Let $n \vcentcolon= [\mathbb{K} : \mathbb{E}]_s$ and $m \vcentcolon= [\mathbb{E} : \mathbb{F}]_s$ and $\sigma : \mathbb{F} \to \mathbb{L}$ be an embedding into an algebraically closed field $\mathbb{L}.$ 

    Let $\sigma_1, \ldots, \sigma_m : \mathbb{E} \to \mathbb{L}$ be extensions of $\sigma.$ Then, each $\sigma_i$ has extensions $\sigma_i^{(1)}, \ldots, \sigma_i^{(n)} : \mathbb{K} \to \mathbb{L}.$ Note that $\{\sigma_i^{(j)} : 1 \le i \le m,\; 1 \le j \le n\}$ has cardinality $mn.$ (All the extensions obtained are distinct.)

    Clearly, any embedding $\tau : \mathbb{K} \to \mathbb{L}$ extending $\sigma$ is obtained this way. ($\tau|_{\mathbb{E}}$ is $\sigma_i$ for some $i$ and thus, $\tau = \sigma_i^{(j)}$ for some $j.$) 

    Thus, $[\mathbb{K} : \mathbb{F}]_s = mn,$ as desired. 

    Now, since $\mathbb{E}/\mathbb{F}$ is finite, we can construct $\alpha_1, \ldots, \alpha_g$ such that $\mathbb{E} = \mathbb{F}(\alpha_1, \ldots, \alpha_g).$ We have the chain
    \begin{equation*} 
        \mathbb{F} \subset \mathbb{F}(\alpha_1) \subset \mathbb{F}(\alpha_1, \alpha_2) \subset \cdots \subset \mathbb{F}(\alpha_1, \ldots, \alpha_g).
    \end{equation*}
    Note that by \Cref{prop:sepdeglessthannordeg}, we know that 
    \begin{equation*} 
        [\mathbb{F}(\alpha_1, \ldots, \alpha_{i + 1}) : \mathbb{F}(\alpha_1, \ldots, \alpha_i)]_s \le [\mathbb{F}(\alpha_1, \ldots, \alpha_{i + 1}) : \mathbb{F}(\alpha_1, \ldots, \alpha_i)]
    \end{equation*}
    for all $i = 0, \ldots, g - 1.$ Since both degrees are multiplicative, we are done.
\end{proof}

\sepiffdegequal*\label{thm:sepiffdegequal2}
\begin{flushright}\hyperref[thm:sepiffdegequal]{\upsym}\end{flushright}
\begin{proof}
     Write $\mathbb{E} = \mathbb{F}(\alpha_1, \ldots, \alpha_n)$ for $\alpha_i \in \mathbb{E}.$ (Note that $\mathbb{E}/\mathbb{F}$ is a finite extension.)

    Put 
    \begin{equation*} 
        \mathbb{F}_0 \vcentcolon= \mathbb{F} \andd \mathbb{F}_i \vcentcolon= \mathbb{F}(\alpha_1, \ldots, \alpha_i),
    \end{equation*} 
    for $i = 1, \ldots, n.$

    \forward Assume $\mathbb{E}/\mathbb{F}$ is separable. Then, since each $\alpha_i$ is separable over $\mathbb{F},$ it follows that $\alpha_i$ is separable over $\mathbb{F}_i$ for $i = 1, \ldots, n.$ (Note that $\irr(\alpha_i, \mathbb{F}_i) \mid \irr(\alpha_i, \mathbb{F}).$) Thus, we see that 
    \begin{equation*} 
        [\mathbb{F}_{i} : \mathbb{F}_{i - 1}]_s = [\mathbb{F}_{i} : \mathbb{F}_{i - 1}]
    \end{equation*}
    for all $i = 1, \ldots, n.$ Multiplying gives $[\mathbb{E} : \mathbb{F}]_s = [\mathbb{E}:\mathbb{F}].$

    \backward Let $\alpha \in \mathbb{E}$ be arbitrary. Consider the tower
    \begin{equation*} 
        \mathbb{F} \subset \mathbb{F}(\alpha) \subset \mathbb{E}.
    \end{equation*}
    Since, we have the equality $[\mathbb{E} : \mathbb{F}]_s = [\mathbb{E} : \mathbb{F}],$ we also have the equality $[\mathbb{F}(\alpha) : \mathbb{F}]_s = [\mathbb{F}(\alpha) : \mathbb{F}],$ by the previous corollary. Thus, $\alpha$ is separable over $\mathbb{F},$ by \Cref{prop:sepdeglessthannordeg}.
\end{proof}

\compdecompsep*\label{prop:compdecompsep2}
\begin{flushright}\hyperref[prop:compdecompsep]{\upsym}\end{flushright}
\begin{proof}
    For both parts, we first note that if $\alpha \in \mathbb{K}$ is algebraic over $\mathbb{F},$ then it is also algebraic over $\mathbb{E}.$ Moreover, $\irr(\alpha, \mathbb{E}) \mid \irr(\alpha, \mathbb{F}).$ (The divisibility is in $\mathbb{E}[x].$)

    \forward Let $\alpha \in \mathbb{K}$ be arbitrary. Then, $\alpha$ is algebraic over $\mathbb{F}$ and hence, over $\mathbb{E}.$ Since $\irr(\alpha, \mathbb{F})$ has no repeated roots, neither does its factor $\irr(\alpha, \mathbb{E}).$ Thus, $\mathbb{K}/\mathbb{E}$ is separable. \\
    Now, let $\beta \in \mathbb{E}$ be arbitrary. Then, $\beta \in \mathbb{K}$ and thus, $\irr(\alpha, \mathbb{F})$ is separable. Thus, $\mathbb{E}/\mathbb{F}$ is separable.

    \backward Let $\alpha \in \mathbb{K}$ be arbitrary. Note that $\alpha$ is algebraic over $\mathbb{E},$ since it is separable over $\mathbb{E}.$ Let $\irr(\alpha, \mathbb{E}) = a_1 + \cdots + a_{n }x^{n - 1} + x^n \in \mathbb{E}[x].$ 

    Put 
    \begin{equation*} 
        \mathbb{F}_0 \vcentcolon= \mathbb{F} \andd \mathbb{F}_i \vcentcolon= \mathbb{F}(a_1, \ldots, a_i),
    \end{equation*} 
    for $i = 1, \ldots, n.$ By \forward, we see that $a_i$ is separable over $\mathbb{F}_{i - 1}$ and hence, 
    \begin{equation} \label{eq:002} \tag{$*$}
        [\mathbb{F}_i : \mathbb{F}_{i - 1}]_s = [\mathbb{F}_i : \mathbb{F}_{i - 1}]
    \end{equation} 
    for all $i = 1, \ldots, n.$

    Finally, put $\mathbb{F}_{n + 1} \vcentcolon= \mathbb{F}_n(\alpha).$ Then, \Cref{eq:002} holds for $i = n + 1$ as well, since $\alpha$ is separable over $\mathbb{F}_n.$ (Note that $\irr(\alpha, \mathbb{F}_n) = \irr(\alpha, \mathbb{E}),$ by our construction and the latter is separable by assumption.)

    Thus, upon multiplying, we get $[\mathbb{F}_{n + 1} : \mathbb{F}]_s = [\mathbb{F}_{n + 1} : \mathbb{F}]$ and hence, $\mathbb{F}_{n + 1}/\mathbb{F}$ is separable. Since $\alpha \in \mathbb{F}_{n + 1},$ we see that $\alpha$ is separable over $\mathbb{F}$ and hence, $\mathbb{K}/\mathbb{F}$ is separable.
\end{proof}

\sepdegdividesdeg*\label{prop:sepdegdividesdeg2}
\begin{flushright}\hyperref[prop:sepdegdividesdeg]{\upsym}\end{flushright}
\begin{proof}
    Clearly the statement is true if $\chr(\mathbb{F}) = 0$ since we have equality of degrees. Suppose $\chr(\mathbb{F}) =\vcentcolon p > 0.$

    First, suppose that $\mathbb{E} = \mathbb{F}(\alpha)$ for some $\alpha \in \mathbb{E}.$ Let $p(x) \vcentcolon= \irr(\alpha, \mathbb{F})$ and $d \vcentcolon= \deg(p(x)).$ By \Cref{prop:samemultirredpoly}, $p(x)$ factors in $\overline{\mathbb{F}}[x]$ as
    \begin{equation*} 
        p(x) = (x - \alpha)^{p^n} (x - \alpha_2)^{p^n} \cdots (x - \alpha_g)^{p^n},
    \end{equation*}
    where $\alpha_2, \ldots, \alpha_g \in \overline{\mathbb{F}}\setminus\{\alpha\}$ are distinct. Note that we have $gp^n = d.$ By \Cref{prop:rootsandextensions}, we know that $[\mathbb{F}(\alpha) : \mathbb{F}]_s = g.$ Thus, the statement is true.

    For a general finite extension $\mathbb{E}/\mathbb{F},$ write $\mathbb{E} = \mathbb{F}(\beta_1, \ldots, \beta_k)$ and use the fact that degrees are multiplicative.
\end{proof}

\section{Finite fields}
\uniquefinfields*\label{thm:uniquefinfields2}
\begin{flushright}\hyperref[thm:uniquefinfields]{\upsym}\end{flushright}
\begin{proof}
    Let $q \vcentcolon= \md{\mathbb{K}}$ and $p \vcentcolon= \chr(\mathbb{K}).$ Then, $q = p^n$ for some $n \in \mathbb{N}.$ Note that $\mathbb{K}^\times$ is a group of order $q - 1.$ By Lagrange's theorem, we have $a^{q - 1} = 1$ for all $a \in \mathbb{K}^\times.$ In turn, we get $a^q - a = 0$ for \emph{all} $a \in \mathbb{K}.$

    Hence, $\mathbb{K}$ is a splitting field of $x^q - x$ over $\mathbb{F}_p$ and so is $\mathbb{L}.$ By \Cref{thm:isosplitting}, $\mathbb{K}$ and $\mathbb{L}$ are isomorphic.
\end{proof}

\existencefinfields*\label{thm:existencefinfields2}
\begin{flushright}\hyperref[thm:existencefinfields]{\upsym}\end{flushright}
\begin{proof}
    Fix $n \in \mathbb{N}$ and let $q = p^n.$ $\overline{\mathbb{F}}_p$ contains a unique splitting field of $x^q - x =\vcentcolon f(x)$ over $\mathbb{F}_p.$ We show that this splitting field has $q$ elements. Consider
    \begin{equation*} 
        \mathbb{K} = \{\alpha \in \overline{\mathbb{F}}_p \mid f(\alpha) = 0\}.
    \end{equation*}
    Then, $\md{\mathbb{K}} = q$ since $f(x)$ is separable, by \Cref{thm:derivcritsep}. 

    Thus, $\mathbb{K}$ is the desired splitting field. Conversely any other field with $q$ elements would be the set of roots of $x^q - x$ and hence, we have uniqueness.

    We now show that $\overline{\mathbb{F}}_p = \bigcup_{k \ge 1}\mathbb{F}_{p^k}.$ Let $\alpha \in \overline{\mathbb{F}}_p$ and let $d \vcentcolon= \deg_{\mathbb{F}}(\alpha).$ Then, $[\mathbb{F}(\alpha) : \mathbb{F}_p] = d$ and hence, $\alpha \in \mathbb{F}(\alpha) = \mathbb{F}_{p^d}.$
\end{proof}


\xfourplusone*\label{prop:xfourplusone2}
\begin{flushright}\hyperref[prop:xfourplusone]{\upsym}\end{flushright}
\begin{proof}
    For irreducibility over $\mathbb{Z}[x],$ note that
    \begin{equation*} 
        f(x + 1) = x^4 + 4x^3 + 6x^2 + 4x + 2
    \end{equation*}
    is Eisenstein at the prime $2.$

    Now, let $p$ be a prime. If $p = 2,$ the we have $x^4 + 1 = (x + 1)^4.$ Let $p > 2$ be an odd prime. Then, $p^2 \equiv 1 \pmod{8}.$ Hence, we have
    \begin{equation*} 
        x^4 + 1 \mid x^8 - 1 \mid x^{p^2 - 1} - 1 \mid x^{p^2} - x.
    \end{equation*}
    For the sake of contradiction, assume that $x^4 + 1$ is irreducible and let $\alpha \in \overline{\mathbb{F}}_p$ be a root. Then, $[\mathbb{F}_p(\alpha) : \mathbb{F}_p] = \deg(x^4 + 1) = 4.$

    But $\alpha$ is clearly contained in the splitting of $x^{p^2} - x$ over $\mathbb{F}_p,$ which is $\mathbb{F}_{p^2} \subset \overline{\mathbb{F}}_p$ and so, $\alpha$ is contained in a degree $2$ extension. This is a contradiction.
\end{proof}

\xdxxnxdiv*\label{lem:xdxxnxdiv2}
\begin{flushright}\hyperref[lem:xdxxnxdiv]{\upsym}\end{flushright}
\begin{proof}
    Fix an algebraic closure $\overline{\mathbb{F}}_q.$ Since $f(x) \vcentcolon= x^{q^m} - x$ is separable, it suffices to show that every root of $f(x)$ is also a root of $x^{q^n} - x =\vcentcolon g(x).$ (Recall \Cref{prop:divisibilityofpoly}.)

    To this end, let $\alpha$ be a root of $f(x).$ We have 
    \begin{equation*} 
        \alpha^{q^m} = \alpha.
    \end{equation*}
    Now raise both sides to the power $q^m$ to obtain 
    \begin{equation*} 
        \alpha^{q^{2m}} = \alpha^{q^m} = \alpha.
    \end{equation*}
    Continue repeatedly to get 
    \begin{equation*} 
        \alpha^{q^{km}} = \alpha
    \end{equation*}
    for all $k \in \mathbb{N}.$ In particular, for $k = n/m,$ the above is true. This gives us that $g(\alpha) = 0,$ as desired.
\end{proof}

\irreddivsplitpoly*\label{lem:irreddivsplitpoly2}
\begin{flushright}\hyperref[lem:irreddivsplitpoly]{\upsym}\end{flushright}
\begin{proof}
    \forward Suppose $f(x) \mid x^{q^n} - x.$ Then, $\mathbb{F}_{q^n}$ contains all the roots of $f(x).$ Let $\alpha \in \overline{\mathbb{F}}_{q}$ be a root of $f(x).$ Thus, $\alpha \in \mathbb{F}_{q^n}.$ Considering the tower $\mathbb{F}_q \subset \mathbb{F}_q(\alpha) \subset \mathbb{F}_{q^n}$ shows that $\deg(f(x)) = [\mathbb{F}_q(\alpha) : \mathbb{F}_q]$ divides $[\mathbb{F}_{q^n} : \mathbb{F}_q] = n.$

    \backward Let $d \vcentcolon= \deg(f(x)) \mid n.$ Fix an algebraic closure $\overline{\mathbb{F}}_q$ of $\mathbb{F}_q.$ We show that every root of $f(x)$ in $\overline{\mathbb{F}}_q$ satisfies $x^{q^d} - x.$ Since this divides $x^{q^n} - x,$ we would be done.

    Let $\alpha \in \overline{\mathbb{F}}_q$ be a root of $f(x).$ Then, $[\mathbb{F}(\alpha) : \mathbb{F}] = d$ and thus, by \Cref{thm:existencefinfields}, we have that
    \begin{equation*} 
        \mathbb{F}(\alpha) = \mathbb{F}_{q^d} = \{\beta^{q^d} - \beta = 0 \mid \beta \in \overline{\mathbb{F}}_q\}.
    \end{equation*}
    (Note that any algebraic closure $\overline{\mathbb{F}}_q$ is also an algebraic closure of $\mathbb{F}_p \subset \mathbb{F}_q.$)

    Thus, $\alpha$ satisfies $x^{q^d} - x,$ as desired.
\end{proof}

\gaussnecklace*\label{thm:gaussnecklace2}
\begin{flushright}\hyperref[thm:gaussnecklace]{\upsym}\end{flushright}
\begin{proof}
    Note that $x^{q^n} - x$ is a separable polynomial. By \Cref{lem:irreddivsplitpoly}, we see that
    \begin{equation*} 
        x^{q^n} - x = \prod_{d \mid n} f_1^{(d)}(x) \cdots f_{N_q(d)}^{(d)}(x),
    \end{equation*}
    where $f_1^{(d)}(x), \ldots, f_{N_q(d)}^{(d)}(x)$ are all the irreducible monic polynomials of degree $d.$ 

    Equating the degrees of both sides gives
    \begin{equation*} 
        q^n = \sum_{d \mid n} d{N_q(d)}.
    \end{equation*}
    Thus, defining $f(n) \vcentcolon= q^n$ and $g(n) \vcentcolon= nN_q(n),$ we use \nameref{thm:mobiusinv} to conclude that
    \begin{equation*} 
        nN_q(n) = \sum_{d \mid n} \mu(d)q^{n/d}. \qedhere
    \end{equation*}
\end{proof}

\pet*\label{thm:pet2}
\begin{flushright}\hyperref[thm:pet]{\upsym}\end{flushright}
\begin{proof}
    If $\mathbb{F}$ is a finite, then $\mathbb{K}$ is also finite and hence, $\mathbb{K}^\times$ is cyclic by \Cref{thm:finsubgroupcyclic}. A generator of $\mathbb{K}^\times$ is clearly a primitive element of $\mathbb{K}$ over $\mathbb{F}.$ Clearly, there are only finitely many intermediate subfields as well. 

    Thus, we may assume that $\mathbb{F}$ is infinite.
    \begin{enumerate}[leftmargin=*]
        \item \forward Let $\mathbb{K} = \mathbb{F}(\alpha)$ for some $\alpha \in \mathbb{K}$ and let $f(x) \vcentcolon= \irr(\alpha, \mathbb{F}).$ Let $\mathbb{E}$ be an intermediate subfield. 

        Let $h_{\mathbb{E}}(x) \vcentcolon= \irr(\alpha, \mathbb{E}).$ Then, $h_{\mathbb{E}}(x) \mid f(x)$ for all intermediate subfields $\mathbb{E}.$

        Now, let $\mathbb{E}_0 \subset \mathbb{E}$ be the field obtained by adjoining the coefficients of $h(x)$ to $\mathbb{F}.$ Then, $\irr(\alpha, \mathbb{E}) = \irr(\alpha, \mathbb{E}_{0}).$ Note that we also have $\mathbb{K} = \mathbb{E}(\alpha) = \mathbb{E}_0(\alpha).$ Thus, we get that
        \begin{equation*} 
            [\mathbb{K} : \mathbb{E}] = \deg(\irr(\alpha, \mathbb{E})) = \deg(\irr(\alpha, \mathbb{E}_0)) = [\mathbb{K} : \mathbb{E}_0]
        \end{equation*}
        and hence, $\mathbb{E} = \mathbb{E}_0.$

        This shows that if $\mathbb{E}$ and $\mathbb{E}'$ are intermediate fields with $h_{\mathbb{E}} = h_{\mathbb{E}'},$ then $\mathbb{E} = \mathbb{E}'.$ Since $f(x)$ only has finitely many monic divisors, there are only finitely many intermediate subfields.

        \backward Suppose $\mathbb{K}/\mathbb{F}$ has finitely many intermediate subfields. Write $\mathbb{K} = \mathbb{F}(\alpha_1, \ldots, \alpha_n).$

        Assume that $n = 2.$ We show that $\mathbb{K}/\mathbb{F}$ has a primitive element. The general case then follows inductively. \\
        Thus, we have $\mathbb{K} = \mathbb{F}(\alpha_1, \alpha_2).$ 

        For each $c \in \mathbb{F},$ we have the subfield $\mathbb{F}(\alpha_1 + c\alpha_2).$ Since $\mathbb{F}$ is finite and there are only finitely many intermediate subfields, there exist $c \neq d \in \mathbb{F}$ such that 
        \begin{equation*} 
            \mathbb{F}(\alpha_1 + c\alpha_2) = \mathbb{F}(\alpha_1 + d\alpha_2) =\vcentcolon \mathbb{L}.
        \end{equation*} 
        We show that $\mathbb{L} = \mathbb{K}.$ (Note that $\mathbb{L}$ is primitive over $\mathbb{F}.$)

        By the above, we see that $(c - d)\alpha_2 \in \mathbb{L}$ and hence, $\alpha_2 \in \mathbb{L}.$ In turn, $\alpha_1 \in \mathbb{L}.$ Thus,
        \begin{equation*} 
            \mathbb{L} \subset \mathbb{K} = \mathbb{F}(\alpha_1, \alpha_2) \subset \mathbb{L}
        \end{equation*}
        and hence, we have equality.
        %
        %
        %
        \item Now, assume that $\mathbb{K}/\mathbb{F}$ is a finite separable extension. By the same inductive argument as earlier, it is sufficient to prove the existence of a primitive element when $\mathbb{K} = \mathbb{F}(\alpha, \beta)$ for some $\alpha, \beta \in \mathbb{K}.$ Fix an algebraic closure $\overline{\mathbb{F}}$ of $\mathbb{F}.$

        As earlier, we show that there exists $c \in \mathbb{F}$ such that
        \begin{equation} \label{eq:003} \tag{$*$}
            \mathbb{K} = \mathbb{F}(\alpha + c\beta).
        \end{equation}

        We now seek a condition on $c$ that implies \Cref{eq:003}. Let $n \vcentcolon= [\mathbb{K} : \mathbb{F}] = [\mathbb{K} : \mathbb{F}]_s.$ (Equality by \Cref{thm:sepiffdegequal}.) \\
        Then, by definition of separable degree, there exist $n$ embeddings $\sigma_1, \ldots, \sigma_n : \mathbb{K} \to \overline{\mathbb{F}}$ extending the natural inclusion. 

        Now, if $c \in \mathbb{F}$ is such that the \deff{conjugates} $\sigma_i(\alpha + c\beta)$ are distinct for $i = 1, \ldots, n,$ then this means that 
        \begin{equation*} 
            n = [\mathbb{K} : \mathbb{F}]_s \ge [\mathbb{F}(\alpha + c\beta) : \mathbb{F}]_s \ge n = [\mathbb{K} : \mathbb{F}]
        \end{equation*}
        and thus, \Cref{eq:003} holds. Our job now is to find such a $c \in \mathbb{F}$ for which the conjugates are distinct. 

        Let $c \in \mathbb{F}$ be arbitrary. Then, $\sigma_i(\alpha + c\beta) = \sigma_i(\alpha) + c\sigma_i(\beta).$ Consider the polynomial
        \begin{equation*} 
            f(x) \vcentcolon= \prod_{1 \le i < j \le n}\left[(\sigma_i(\alpha) - \sigma_j(\alpha)) + x(\sigma_i(\beta) - \sigma_j(\beta))\right] \in \mathbb{K}[x].
        \end{equation*}

        Thus, the conjugates of $c$ are distinct iff $f(c) \neq 0.$ Note that if $\sigma_i$ and $\sigma_j$ agree on $\alpha$ and $\beta,$ then $\sigma_i = \sigma_j$ since $\mathbb{K} = \mathbb{F}(\alpha, \beta).$ Thus, $f(x)$ above is not the zero polynomial. But since $\mathbb{F}$ is infinite, there exists $c \in \mathbb{F}$ such that $f(c) \neq 0$ and thus, we are done. \qedhere
    \end{enumerate}
\end{proof}

\section{Normal extensions}

\seppolysplittingfields*\label{prop:seppolysplittingfields2}
\begin{flushright}\hyperref[prop:seppolysplittingfields]{\upsym}\end{flushright}
\begin{proof}
    Let $a \in \mathbb{E} = \mathbb{F}(A)$ where $A$ is as in \Cref{rem:splitfamilyexists}. By \Cref{cor:FAdescfinite}, there is a finite set $\{a_1, \ldots, a_n\} \subset A$ such that $a \in \mathbb{F}(a_1, \ldots, a_n).$ Since each $a_i$ is a root of a separable, it is separable. By applying \Cref{cor:adjoiningsepissep} (repeatedly), we see that $\mathbb{F}(a_1, \ldots, a_n)/\mathbb{F}$ is a separable extension and thus, $a$ is separable over $\mathbb{F}.$
\end{proof}

\algebraicautomorphism*\label{lem:algebraicautomorphism2}
\begin{flushright}\hyperref[lem:algebraicautomorphism]{\upsym}\end{flushright}
\begin{proof}
    We only need to prove that $\sigma$ is onto. Let $\alpha \in \mathbb{E}$ be arbitrary. Put $p(x) \vcentcolon= \irr(\alpha, \mathbb{F}).$ Let $\mathbb{K} \subset \mathbb{E}$ be the subfield generated by the roots of $p(x)$ in $\mathbb{E}.$ Then, $\mathbb{K}$ is a finite dimensional vector space over $\mathbb{F}$ and $\alpha \in \mathbb{K}.$ Since $\sigma$ is an $\mathbb{F}$-embedding, it maps roots of $p(x)$ to roots of $p(x).$ Thus, $\sigma(\mathbb{K}) \subset \mathbb{K}.$

    But $\sigma$ is an $\mathbb{F}$-linear map and $\mathbb{K}$ is a finite dimensional $\mathbb{F}$-vector space. Thus, $\sigma|_{\mathbb{K}}$ is onto and contains $\alpha$ in its image.
\end{proof}

\normalequivalent*\label{thm:normalequivalent2}
\begin{flushright}\hyperref[thm:normalequivalent]{\upsym}\end{flushright}
\begin{proof}
    \ref{item:001} $\Rightarrow$ \ref{item:002}: Let $a \in E$ and $p_a(x) = \irr(a, \mathbb{F}).$ If $b \in \overline{\mathbb{F}}$ is a root of $p_a(x),$ then there exists an $\mathbb{F}$-isomorphism $\mathbb{F}(a) \to \overline{\mathbb{F}}$ with $a \mapsto b.$ Extend this to a map $\sigma : \mathbb{E} \to \overline{\mathbb{F}}.$ By hypothesis, we have $\mathbb{E} = \sigma(\mathbb{E}) \ni b.$ Thus, $\mathbb{E}$ is a splitting field of the family $\{p_a(x)\}_{a \in E}.$

    \ref{item:002} $\Rightarrow$ \ref{item:003}: Let $\mathbb{E}$ be a spitting field of $\{p_i(x)\}_{i \in I} \subset \mathbb{F}[x]$ over $\mathbb{F}.$ Let $f(x) \in \mathbb{F}[x]$ be an irreducible polynomial having a root $a \in \mathbb{E}.$ Let $b \in \overline{\mathbb{F}}$ be any root of $f(x).$ There exists an $\mathbb{F}$-embedding $\mathbb{F}(a) \to \overline{\mathbb{F}}$ with $a \mapsto b.$ Extend this to an $\mathbb{F}$-embedding $\sigma : \mathbb{E} \to \overline{\mathbb{F}}.$ Since $\sigma$ fixes $\mathbb{F},$ it maps roots of $p_i(x)$ to its roots for all $i \in I.$ Since $\mathbb{E}$ is generated by these roots, we see that $\sigma(\mathbb{E}) \subset \mathbb{E}$ and hence, $b \in \mathbb{E}.$

    \ref{item:003} $\Rightarrow$ \ref{item:001}: Let $\sigma : \mathbb{E} \to \overline{\mathbb{F}}$ be an $\mathbb{F}$-embedding. Let $a \in \mathbb{E}.$ Then, $p(x) \vcentcolon= \irr(\alpha, \mathbb{F})$ splits into linear factors in $\mathbb{E}.$ Since $\sigma(a)$ is a root of $p(x),$ we have $\sigma(a) \in \mathbb{E}.$ Thus, $\sigma(E) \subset E.$ By \Cref{lem:algebraicautomorphism}, we have that $\sigma$ is an automorphism. (Note that $\mathbb{E}/\mathbb{F}$ is indeed algebraic since $\mathbb{E} \subset \overline{\mathbb{F}}.$)
\end{proof}

\operationsonnormalexts*\label{prop:operationsonnormalexts2}
\begin{flushright}\hyperref[prop:operationsonnormalexts]{\upsym}\end{flushright}
\begin{proof}
    Fix an algebraic closure $\overline{\mathbb{F}} \supset \mathbb{K}.$

    Let $\sigma : \mathbb{E}_1\mathbb{E}_2 \to \overline{\mathbb{F}}$ be an $\mathbb{F}$-embedding. Then, $\sigma(\mathbb{E}_1\mathbb{E}_2) = \sigma(\mathbb{E}_1)\sigma(\mathbb{E}_2) = \mathbb{E}_1\mathbb{E}_2.$ Since this is true for all $\mathbb{F}$-embeddings, $\mathbb{E}_1\mathbb{E}_2/\mathbb{F}$ is normal, by \Cref{thm:normalequivalent}.

    Similar calculation shows the same for intersection as well.
\end{proof}

\section{Galois Extensions}
\orderofgalgroup*\label{prop:orderofgalgroup2}
\begin{flushright}\hyperref[prop:orderofgalgroup]{\upsym}\end{flushright}
\begin{proof}
    Fix an algebraic closure $\overline{\mathbb{F}} \supset \mathbb{E}.$ 

    Let $n \vcentcolon= [\mathbb{E} : \mathbb{F}]_s.$ Let $\sigma_1, \ldots, \sigma_n : \mathbb{E} \to \overline{\mathbb{F}}$ be $\mathbb{F}$-embeddings. Then, normality of $\mathbb{E}/\mathbb{F}$ implies that $\sigma_i \in \Gal(\mathbb{E}/\mathbb{F}).$ Thus, $\md{\Gal(\mathbb{E}/\mathbb{F})} \ge n.$

    On the other hand, if $\sigma \in \Gal(\mathbb{E}/\mathbb{F}),$ then $\sigma$ is an $\mathbb{F}$-embedding of $\mathbb{E}$ into $\overline{\mathbb{F}}$ upon composition by the inclusion. Thus, $\Gal(\mathbb{E}/\mathbb{F}) = \{\sigma_1, \ldots, \sigma_n\}.$
\end{proof}

\frobgenerates*\label{prop:frobgenerates2}
\begin{flushright}\hyperref[prop:frobgenerates]{\upsym}\end{flushright}
\begin{proof}
    Note that $\varphi$ does indeed fix $\mathbb{F}_q$ since any $a \in \mathbb{F}_q$ satisfies $x^q - x$ and thus, $\varphi \in \Gal(\mathbb{F}_{q^n}/\mathbb{F}_q).$

    By \Cref{prop:orderofgalgroup}, we know that $\md{\Gal(\mathbb{F}_{q^n}/\mathbb{F}_q)} = n.$ Thus, it suffices to show that $\varphi$ has order no less than $n.$ Let order of $\varphi$ be $d \le n.$ Note that
    \begin{equation*} 
        \varphi^d(a) = a^{q^d}.
    \end{equation*}
    Thus, if $\varphi^d = \id_{\mathbb{F}_{q^n}},$ then every element of $\mathbb{F}_{q^n}$ satisfies $x^{q^d} - x.$ Thus, the degree is at least $q^n.$ Thus, $q^d \ge q^n$ or $d \ge n.$
\end{proof}

\fixfieldinjectiveIG*\label{thm:fixfieldinjectiveIG2}
\begin{flushright}\hyperref[thm:fixfieldinjectiveIG]{\upsym}\end{flushright}
\begin{proof}
    \phantom{hi}
    \begin{enumerate}[leftmargin=*]
        \item Clearly, $\mathbb{F} \subset \mathbb{K}^G,$ by definition of the Galois group. Only the reverse inclusion needs to be shown.

        Let $a \in \mathbb{K}^G.$ Then, $a$ is separable over $\mathbb{F}$ and hence, $[\mathbb{F}(a) : \mathbb{F}]_s = [\mathbb{F}(a) : \mathbb{F}],$ by \Cref{cor:adjoiningsepissep} and \Cref{thm:sepiffdegequal}.

        Thus, if $a \notin \mathbb{F},$ then $[\mathbb{F}(a) : \mathbb{F}] > 1$ and so, there is one non-identity embedding $\mathbb{F}(a) \to \mathbb{K},$ which would necessarily move $a.$ Thus, we must have $a \in \mathbb{F}.$
        %
        \item The fact that $\mathbb{K}/\mathbb{E}$ is separable follows from \Cref{prop:compdecompsep} and that it is normal follows from \Cref{prop:decompnormal}. Thus, $\mathbb{K}/\mathbb{E}$ is Galois.

        Now, if $\mathbb{E}, \mathbb{E}' \in \mathcal{I}$ are such that 
        \begin{equation*} 
            H \vcentcolon= \Gal(\mathbb{K}/\mathbb{E}) = \Gal(\mathbb{K}/\mathbb{E}') =\vcentcolon H',
        \end{equation*}
        then the first part gives
        \begin{equation*} 
            \mathbb{E} = \mathbb{K}^H = \mathbb{K}^{H'} = \mathbb{E}'
        \end{equation*}
        and thus, the map is an injection. \qedhere
    \end{enumerate}
\end{proof}

\degboundedbyn*\label{lem:degboundedbyn2}
\begin{flushright}\hyperref[lem:degboundedbyn]{\upsym}\end{flushright}
\begin{proof}
    Let $\beta \in \mathbb{E}$ be such that $[\mathbb{F}(\beta) : \mathbb{F}]$ is maximal. Note that $[\mathbb{F}(\beta) : \mathbb{F}] \le n,$ by hypothesis. It suffices to show that $\mathbb{E} = \mathbb{F}(\beta).$

    Suppose that $\mathbb{E} \neq \mathbb{F}(\beta).$ Then, pick $\alpha \in \mathbb{E} \setminus \mathbb{F}(\beta).$ Then, $\mathbb{F}(\alpha, \beta)$ is a separable extension and thus, there exists $\eta \in \mathbb{F}(\alpha, \beta) \subset \mathbb{E}$ such that $\mathbb{F}(\alpha, \beta) = \mathbb{F}(\eta),$ by the \nameref{thm:pet}.

    But this is a contradiction since $\mathbb{F}(\beta) \subsetneq \mathbb{F}(\alpha, \beta) = \mathbb{F}(\eta)$ implies that $[\mathbb{F}(\eta) : \mathbb{F}] > [\mathbb{F}(\beta) : \mathbb{F}],$ contradicting the maximality of $\beta.$
\end{proof}

\artin*\label{thm:artin2}
\begin{flushright}\hyperref[thm:artin]{\upsym}\end{flushright}
\begin{proof}
    Let $G = \{\sigma_1, \ldots, \sigma_n\}$ and $\md{G} = n.$
    \begin{enumerate}[leftmargin=*]
        \item Let $\alpha \in \mathbb{E}.$ Consider $S = \{\sigma_1(\alpha), \ldots, \sigma_n(\alpha)\}.$ Note that the elements written need not all be distinct. Let $r \vcentcolon= \md{S}.$ Without loss of generality, assume that $S = \{\sigma_1(\alpha), \ldots, \sigma_r(\alpha)\}.$

        Let $\tau \in G.$ Then, $\tau(S) = S.$\footnote{Each $\tau\sigma_i$ is an element of $G$ and $\tau\sigma_i(\alpha)$ are distinct for $i = 1, \ldots, r.$} Thus, $\tau|_{S}$ is a permutation of $S.$ Consider the polynomial
        \begin{equation*} 
            f(x) \vcentcolon= (x - \sigma_1(\alpha)) \cdots (x - \sigma_r(\alpha)).
        \end{equation*}
        The coefficients of $f(x)$ are symmetric functions of $\sigma_1(\alpha), \ldots, \sigma_r(\alpha)$ and thus, are fixed by every $\tau \in G,$ by the previous observation. Thus, $f(x) \in \mathbb{E}^G[x].$

        Note that $f(\alpha) = 0$ since one of the $\sigma_i$ is the identity map. Thus, $\irr(\alpha, \mathbb{E}^G) \mid f(x).$ Note that $f(x)$ has distinct roots, by construction. In particular, $\alpha$ is separable over $\mathbb{E}^G.$ Since $\alpha \in \mathbb{E}$ was arbitrary, this tells us that $\mathbb{E}/\mathbb{E}^G$ is separable.

        Moreover, $f(x)$ splits completely in $\mathbb{E}[x]$ and thus, so does $\irr(\alpha, \mathbb{E}^G).$ Thus, $\mathbb{E}/\mathbb{E}^G$ is normal as well and hence, Galois.

        To see that it is finite, note that $[\mathbb{E}^G(\alpha) : \mathbb{E}^G] = r \le n$ and thus, $[\mathbb{E} : \mathbb{E}^G],$ by \Cref{thm:artin}.
        %
        \item Note that $G \subset \Gal(\mathbb{E}/\mathbb{E}^G).$ As we noted earlier, $[\mathbb{E} : \mathbb{E}^G] \le n = \md{G}.$ 

        By \Cref{prop:orderofgalgroup}, we have $\Gal(\mathbb{E}/\mathbb{E}^G) = [\mathbb{E} : \mathbb{E}^G].$ Thus, comparing cardinalities gives $G = \Gal(\mathbb{E}/\mathbb{E}^G).$
        %
        \item Follows from the second part. \qedhere
    \end{enumerate}
\end{proof}

\galoissubgroupscompositum*\label{thm:galoissubgroupscompositum2}
\begin{flushright}\hyperref[thm:galoissubgroupscompositum]{\upsym}\end{flushright}
\begin{proof}
    The third assertion about the inclusion is obvious since $H_1 \supset H_2$ implies that every element fixed by $H_2$ is also fixed by $H_1.$ Since the extensions are Galois, the fields fields are precisely the $\mathbb{E}_i,$ by \Cref{thm:fixfieldinjectiveIG}.

    Note that $\mathbb{K}/\mathbb{E}_i$ is Galois and thus, $\mathbb{E}_i = \mathbb{K}^{H_i} \subset \mathbb{K}^{H_1 \cap H_2}$ for $i = 1, 2.$ Thus, $\mathbb{E}_1\mathbb{E}_2 \subset \mathbb{K}^{H_1 \cap H_2}.$

    On the other hand, if $\sigma \in G$ fixes $\mathbb{E}_1\mathbb{E}_2,$ then it fixes both $\mathbb{E}_1$ and $\mathbb{E}_2.$ Thus, $\Gal(\mathbb{K}/\mathbb{E}_1\mathbb{E}_2) \subset H_1 \cap H_2$ and so, $\mathbb{E}_1\mathbb{E}_2 \supset \mathbb{K}^{H_1 \cap H_2}.$

    Let $H \vcentcolon= \Gal(\mathbb{K}/(\mathbb{E}_1 \cap \mathbb{E}_2)).$ Note that $H_1, H_2 \subset H$ since every $\sigma \in H_i$ fixes $\mathbb{E}_i$ and thus, fixes the intersection. Thus, $\langle H_1, H_2\rangle \subset H$ or $\mathbb{E}_1 \cap \mathbb{E}_2 \subset \mathbb{K}^{\langle H_1, H_2\rangle}.$

    On the other hand, 
    \begin{equation*} 
        \mathbb{K}^{\langle H_1, H_2\rangle} \subset \mathbb{K}^{H_i} = \mathbb{E}_i
    \end{equation*}
    and thus,
    \begin{equation*} 
        \mathbb{K}^{\langle H_1, H_2\rangle} \subset \mathbb{E}_1 \cap \mathbb{E}_2.
    \end{equation*}
\end{proof}

\isomorphismgalois*\label{prop:isomorphismgalois2}
\begin{flushright}\hyperref[prop:isomorphismgalois]{\upsym}\end{flushright}
\begin{proof}
    \phantom{hi}
    \begin{enumerate}[leftmargin=*]
        \item We use \Cref{thm:normalequivalent}. Since $\mathbb{K}/\mathbb{F}$ is Galois, $\mathbb{K}$ is the splitting field of a family of separable polynomials $\{f_i(x) : i \in I\}$ over $\mathbb{F}.$ Then, $\lambda(\mathbb{K})$ is the splitting field of the separable polynomials $\{f^{\lambda}_i(x) : i \in I\}$ over $\lambda(\mathbb{F}).$
        %
        \item Define $\psi : \Gal(\mathbb{K}/\mathbb{F}) \to \Gal(\lambda(\mathbb{K})/\lambda(\mathbb{F}))$ be $\sigma \mapsto \lambda\sigma\lambda^{-1}.$ Clearly, $\psi$ is a well-defined homomorphism. It is easy to see that $\tau \mapsto \lambda^{-1}\tau\lambda$ acts as an inverse. \qedhere
    \end{enumerate}
\end{proof}

\galoisiffnormal*\label{thm:galoisiffnormal2}
\begin{flushright}\hyperref[thm:galoisiffnormal]{\upsym}\end{flushright}
\begin{proof}
    Let $\mathbb{E}/\mathbb{F}$ be Galois. Define 
    \begin{align*} 
        \psi : \Gal(\mathbb{K}/\mathbb{F}) &\to \Gal(\mathbb{E}/\mathbb{F})\\
        \psi(\sigma) &= \sigma|_{\mathbb{E}}.
    \end{align*}
    Note that the above is well-defined since $\mathbb{E}$ is normal and so, $\sigma|_{\mathbb{E}}$ is indeed an automorphism of $\mathbb{F}.$ (That it fixes $\mathbb{F}$ is obvious since $\sigma$ did so.) Clearly, $\psi$ is a homomorphism. However, now note that
    \begin{equation*} 
        \ker(\psi) = \{\sigma \in \Gal(\mathbb{K}/\mathbb{F}) \mid \sigma|_{\mathbb{E}} = \id_{\mathbb{E}}\} = \Gal(\mathbb{K}/\mathbb{E}).
    \end{equation*}
    Thus, $\Gal(\mathbb{K}/\mathbb{E})$ is a normal subgroup of $\Gal(\mathbb{K}/\mathbb{F}).$

    Moreover, since $\mathbb{K}/\mathbb{E}$ is an algebraic and normal extension, every automorphism of $\mathbb{E}$ can indeed be extended to an automorphism of $\mathbb{K}.$\footnote{First extend it to a map $\mathbb{K} \to \overline{\mathbb{E}} \supset \mathbb{K}.$ Normality then forces the map to be an automorphism of $\mathbb{K}.$} Thus, $\psi$ is a surjective map and thus,
    \begin{equation*} 
        \Gal(\mathbb{E}/\mathbb{F}) \cong \frac{\Gal(\mathbb{K}/\mathbb{F})}{\Gal(\mathbb{K}/\mathbb{E})}.
    \end{equation*}
    This proves one direction of the first part as well as the second part.

    Conversely, suppose that $\Gal(\mathbb{K}/\mathbb{E}) \unlhd \Gal(\mathbb{K}/\mathbb{F}).$ Let $\lambda : \mathbb{K} \to \mathbb{K}$ be any $\mathbb{F}$-isomorphism. We first show that $\lambda(\mathbb{E}) = \mathbb{E}.$ By \Cref{prop:isomorphismgalois}, we have
    \begin{equation*} 
        \Gal(\mathbb{K}/\mathbb{E}) = \lambda\Gal(\mathbb{K}/\mathbb{E})\lambda^{-1} = \Gal(\lambda(\mathbb{K})/\lambda(\mathbb{E})) = \Gal(\mathbb{K}/\lambda(\mathbb{E})).
    \end{equation*}
    Thus, $\Gal(\mathbb{K}/\mathbb{E}) = \Gal(\mathbb{K}/\lambda(\mathbb{E})).$ By \Cref{thm:fixfieldinjectiveIG}, we get $\mathbb{E} = \lambda(\mathbb{E}).$

    Now, to show that $\mathbb{E}/\mathbb{F}$ is normal, let $\sigma : \mathbb{E} \to \overline{\mathbb{F}} \supset \mathbb{E}$ be an $\mathbb{F}$-embedding. Then, $\sigma$ can be extended to an $\mathbb{F}$-embedding $\lambda : \mathbb{K} \to \overline{\mathbb{F}}.$ Since $\mathbb{K}/\mathbb{F}$ is normal, we have $\lambda(\mathbb{K}) = \mathbb{K}.$ By the above, we have $\sigma(\mathbb{E}) = \lambda(\mathbb{E}) = \mathbb{E}.$
\end{proof}

\FTGT*\label{thm:FTGT2}
\begin{flushright}\hyperref[thm:FTGT]{\upsym}\end{flushright}
\begin{proof}
    Note that only the first part needs to be proven. We have proven the others (\Cref{thm:galoisiffnormal}, \Cref{prop:orderofgalgroup}, \Cref{thm:galoissubgroupscompositum}).

    Let $\Psi : \mathcal{I} \to \mathcal{G}$ be the map $\mathbb{E} \mapsto \Gal(\mathbb{K}/\mathbb{E}).$ Let $\Phi : \mathcal{G} \to \mathcal{I}$ denote the map $H \mapsto \mathbb{K}^H.$ The fact that these maps reverse inclusion is obvious.

    By \Cref{thm:fixfieldinjectiveIG}, we know that $\Psi$ is an injection.

    Let $H \in \mathcal{G}.$ Then, $H$ is finite and is the Galois group of $\mathbb{K}/\mathbb{K}^H,$ by \Cref{thm:artin}. Thus, $\Psi$ is onto.

    Hence, $\Psi$ is bijective. Therefore, to show that $\Phi = \Psi^{-1},$ it suffices to show only that $\Phi \circ \Psi = \id_{\mathcal{I}}.$

    To this end, let $\mathbb{E} \in \mathcal{I}$ be arbitrary. Then, $H \vcentcolon= \Psi(\mathbb{K}/\mathbb{E})$ is the Galois group of $\mathbb{K}/\mathbb{E}.$ Thus, $\mathbb{E} = \mathbb{K}^H,$ by \Cref{thm:fixfieldinjectiveIG}. In other words
    \begin{equation*} 
        \mathbb{E} = \Phi(\Psi(\mathbb{E})). \qedhere
    \end{equation*}
\end{proof}

\ftagalois*\label{thm:ftagalois2}
\begin{flushright}\hyperref[thm:ftagalois]{\upsym}\end{flushright}
\begin{proof}
    Let $g(x) \in \mathbb{C}[x]$ be a non-constant polynomial. Then, $f(x) = g(x)\bar{g}(x)$ is a non-constant polynomial with real coefficients. Here, $\bar{g}(x)$ denotes the polynomial whose coefficients are complex conjugates of those of $g(x).$ Note that if $f(z) = 0$ for some $z \in \mathbb{C},$ then $g(z) = 0$ or $\bar{g}(z) = 0.$ If $\bar{g}(z) = 0,$ then $g(\bar{z}) = 0.$ In either case, $g$ has a complex root. Thus, it suffices to show that $f(x)$ has a root in $\mathbb{C}.$

    Let $\mathbb{E}$ denote a splitting field of $f(x)$ over $\mathbb{C}.$ Then, it is a splitting of $(x^2 + 1)f(x)$ over $\mathbb{R}.$ It suffices to show that $\mathbb{E} = \mathbb{C}.$ 

    Since $\mathbb{R}$ has no proper odd degree extensions,\footnote{Every odd degree real polynomial has a root in $\mathbb{R}.$} we see that $2 \mid [\mathbb{E} : \mathbb{R}].$ Thus, $G = \Gal(\mathbb{E}/\mathbb{R})$ has a Sylow-$2$ subgroup, say $S.$

    Now, if $S \neq G,$ then $\mathbb{E} \supset \mathbb{E}^S \supsetneq \mathbb{R}.$ However, note that 
    \begin{equation*} 
        [\mathbb{E}^S : \mathbb{R}] = \frac{[\mathbb{E} : \mathbb{R}]}{[\mathbb{E} : \mathbb{E}^S]} = \frac{\md{G}}{\md{S}}
    \end{equation*} 
    is odd. But $\mathbb{R}$ has no proper odd degree extension and thus, $S = G.$

    Thus, $G$ is a $2$-group. (That is, $\md{G} = 2^n$ for some $n \in \mathbb{N}.$) If $\md{G} = 2,$ then $\mathbb{C} = \mathbb{E}$ are we are done.

    Thus, $\md{G} \ge 4.$ Then, $\md{\Gal(\mathbb{E}/\mathbb{C})} \ge 2.$ Let $H \le \Gal(\mathbb{E}/\mathbb{C})$ be a subgroup of index $2.$ Then, $[\mathbb{E}^H : \mathbb{C}] = 2,$ which is a contradiction, since $\mathbb{C}$ has no quadratic extensions. Thus, $\mathbb{C} = \mathbb{E}.$ 
\end{proof}

\section{Cyclotomic Extensions}
\Gfabeliansubgroup*\label{prop:Gfabeliansubgroup2}
\begin{flushright}\hyperref[prop:Gfabeliansubgroup]{\upsym}\end{flushright}
\begin{proof}
    As $f(x)$ is separable, it has $n$ distinct roots in $\overline{\mathbb{F}}.$ Let $Z = \{z_1, \ldots, z_n\}$ be the set of roots and $\mathbb{E} = \mathbb{F}(z_1, \ldots, z_n).$ By \Cref{thm:finsubgroupcyclic}, we know that $Z$ is cyclic. The map $\psi : \Gal(\mathbb{E}/\mathbb{F}) \to \Aut(Z)$ given as $\sigma \mapsto \sigma|_Z$ is an injective group homomorphism. Note that $\Aut(Z) \cong (\mathbb{Z}/n\mathbb{Z})^\times,$ which proves the result.
\end{proof}

\nthrootsnonunity*\label{prop:nthrootsnonunity2}
\begin{flushright}\hyperref[prop:nthrootsnonunity]{\upsym}\end{flushright}
\begin{proof}
    Let $Z = \{z_1, \ldots, z_n\} \subset \mathbb{F}^\times$ be the set of roots of $x^n - 1.$ Let $r$ be a root of $f(x)$ in a splitting field $\mathbb{E}$ of $f(x).$ Then, $rz_1, \ldots, rz_n$ are $n$ distinct roots of $f(x)$ and hence, all the roots. Thus, $\mathbb{E} = \mathbb{F}(r).$

    Let $\sigma, \tau \in \Gal(\mathbb{E}/\mathbb{F}).$ Then, $\sigma(r) = z_{\sigma}r$ and $\tau(r) = z_{\tau}r$ for some $z_\sigma, z_\tau \in Z.$ In turn, we see $\sigma\tau(r) = z_{\sigma}z_{\tau}r.$ Thus, the map
    \begin{equation*} 
        \psi : \Gal(\mathbb{E}/\mathbb{F}) \to Z
    \end{equation*}
    defined by $\psi(\sigma) = z_{\sigma}$ is a group homomorphism. Moreover it is injective since every $\mathbb{F}$-automorphism of $\mathbb{E} = \mathbb{F}(r)$ is uniquely determined by its action on $r.$ Thus, $\G_f$ is isomorphic to a subgroup of $Z$ and we are done.
\end{proof}

\cyclotomicQ*\label{thm:cyclotomicQ2}
\begin{flushright}\hyperref[thm:cyclotomicQ]{\upsym}\end{flushright}
\begin{proof}
    We have $x^n - 1 = \Phi_n(x) h(x),$ where $h(x) \in \mathbb{Q}[x]$ is monic. Thus, by Gauss' Lemma, we have $\Phi_n(x) \in \mathbb{Z}[x].$

    Now, suppose that $p$ is prime not dividing $n.$ We contend that $\Phi(\zeta_n^p) = 0.$ Indeed, suppose not. Then, $h(\zeta_n^p) = 0.$ Alternately, $\zeta_n$ is a root of $h(x^p) \in \mathbb{Q}[x].$ But note that $\Phi_n(x)$ is the minimal polynomial of $\zeta_n$ over $\mathbb{Q}.$ Thus, we can write
    \begin{equation*} 
        h(x^p) = \Phi_n(x) g(x)
    \end{equation*}
    for monic $g(x) \in \mathbb{Z}[x].$ (Again, by Gauss' Lemma.) Reduce the above equation $\mod p$ to get
    \begin{equation*} 
        (\bar{h}(x))^p = \bar{\Phi}_n(x) \bar{g}(x).
    \end{equation*}
    (Note that every element $a \in \mathbb{Z}/p\mathbb{Z}$ satisfies $a^p = a$ and so, $\bar{h}(x^p) = \bar{h}(x))^p.$)

    From the above, we see that $\bar{\Phi}_n(x)$ and $\bar{h}(x)$ have a common factor of $\mathbb{F}_p[x].$ ($\mathbb{F}_p[x]$ is a UFD. Factorise both sides of the above equation into primes.)

    But this, in turn, implies that
    \begin{equation*} 
        x^n - 1 = \bar{\Phi}_n(x) \bar{h}(x)
    \end{equation*}
    in $\mathbb{F}_p[x].$ In particular, $x^n - 1 \in \mathbb{F}_p[x]$ has repeated roots in $\overline{\mathbb{F}}_p.$ This is a contradiction since $x^n - 1$ is separable because $\gcd(n, p) = 1.$

    Thus, $\Phi_n(\zeta_n^p) = 0.$ Now, if $a \in \mathbb{N}$ is any integer such that $\gcd(a, n) = 1,$ we factorise $a = p_1 \cdots p_r$ where $p_1, \ldots, p_r$ are (not necessarily distinct) primes not dividing $n.$ Now, note that $\zeta_n^{p_1}$ is again a primitive root of unity satisfying $\Phi_n(x).$ Thus, the above argument applies and we get $\Phi_n\left((\zeta_n^{p_1})^{p_2}\right) = 0.$ Again, since $\gcd(n, p_1p_2) = 1,$ we see that $\zeta_n^{p_1p_2}$ is a primitive root and so on. Thus, 
    \begin{equation*} 
        \Phi_n(\zeta_n^a) = 0
    \end{equation*}
    for every $a \in \mathbb{N}$ with $\gcd(a, n) = 1.$ As $a$ varies over all such integers, we see that every primitive root of unity is a root of $\Phi_n(x).$

    In particular, $\Phi_n(x)$ has $\varphi(n)$ many distinct roots, each with multiplicity $1.$ Thus, $[\mathbb{Q}(\zeta_n) : \mathbb{Q}] = \varphi(n).$

    By \Cref{prop:Gfabeliansubgroup}, we already know that $\Gal(\mathbb{Q}(\zeta_n)/\mathbb{Q})$ is isomorphic to a subgroup of $(\mathbb{Z}/n\mathbb{Z})^\times.$ By comparing cardinalities, we see that the groups are isomorphic.
\end{proof}

\cycloreccurence*\label{thm:cycloreccurence2}
\begin{flushright}\hyperref[thm:cycloreccurence]{\upsym}\end{flushright}
\begin{proof}
    Clearly, $\Phi_1(x) = x - 1.$ Let $\zeta_n$ be a primitive $n$-th root of unity. By \Cref{thm:cyclotomicQ}, we know that the other roots of $\Phi_n(x)$ are $\zeta_n^i$ for $i \in \{1, \ldots, n\}$ with $\gcd(i, n) = 1.$ Thus,
    \begin{equation*} 
        \Phi_n(x) = \prod_{\substack{1 \le i \le n \\ \gcd(n, i) = 1}}(x - \zeta_n^i).
    \end{equation*}
    In turn, we have
    \begin{equation*} 
        x^n - 1 = \prod_{d \mid n} \Phi_d(x).
    \end{equation*}
    (Factor the above in $\overline{\mathbb{Q}}$ and note that every root of the left side is a primitive $d$-th root of unity for some unique $d.$ Since the $n$-th roots form a group of order $n,$ we must have $d \mid n.$ Conversely, every such $d$-th root is indeed a root of $x^n - 1$ and no two different cyclotomic polynomials have a common root.)

    Thus,
    \begin{equation*} 
        \Phi_n(x) = \frac{x^n - 1}{\displaystyle\prod_{\substack{d \mid n\\ d < n}} \Phi_d(x)}. \qedhere
    \end{equation*}
\end{proof}

\cyclocyclic*\label{prop:cyclocyclic2}
\begin{flushright}\hyperref[prop:cyclocyclic]{\upsym}\end{flushright}
\begin{proof}
    Note that $\Gal(\mathbb{Q}(\zeta_p)/\mathbb{Q}) \cong (\mathbb{Z}/p\mathbb{Z})^\times,$  by \Cref{thm:cyclotomicQ}. Since $\mathbb{Z}/p\mathbb{Z} = \mathbb{F}_p$ is a finite field, \Cref{thm:finsubgroupcyclic} tells us that $\mathbb{F}_p^\times$ is cyclic.

    Recall the general fact about finite cyclic groups: given a cyclic group $G$ of order $n,$ there is a unique subgroup of \emph{index} $d$ for every $d \mid n.$ 

    Using this with the Galois correspondence gives the last statement.
\end{proof}

\cyclodisc*\label{lem:cyclodisc2}
\begin{flushright}\hyperref[lem:cyclodisc]{\upsym}\end{flushright}
\begin{proof}
    We shall use \nameref{prop:discderivative}. First, we note that we have
    \begin{equation*} 
        x^p - 1 = \Phi_p(x)(x - 1)
    \end{equation*}
    and thus,
    \begin{equation*} 
        px^{p - 1} = \Phi_p'(x)(x - 1) + \Phi_p(x).
    \end{equation*}
    Substituting $\zeta_n^i$ above for $i = 1, \ldots, p - 1$ gives
    \begin{equation*} 
        \frac{p}{\zeta_p^i} = \Phi_p'(\zeta_p^i)(\zeta_p^i - 1).
    \end{equation*}
    (We have used $\zeta_p^{p - 1} = \zeta_p^{-1}$ to simplify the left hand side.)

    Thus, we have
    \begin{equation} \label{eq:004} \tag{$\prod$}
        \prod_{i = 1}^{p - 1} \Phi_p'(\zeta_p^i) = \prod_{i = 1}^{p - 1} \frac{p}{\zeta_p^i(\zeta_p^i - 1)}.
    \end{equation}
    Note that we have the following expressions for $\Phi_p(x).$
    \begin{align*} 
        \Phi_p(x) &= (x - \zeta_p)(x - \zeta_p^2) \cdots (x - \zeta_p^{p - 1})\\
        &= x^{p - 1} + \cdots + x + 1.
    \end{align*}
    Thus,
    \begin{equation*} 
        \prod_{i = 1}^{p - 1} \zeta_p^i = (-1)^{p - 1}\andd \prod_{i = 1}^{p - 1}(\zeta_p^i - 1) = (-1)^{p - 1}\Phi_p(1).
    \end{equation*}
    Since $p$ is odd, we have $(-1)^{p - 1} = 1$ and putting it back in \Cref{eq:004} gives
    \begin{equation*} 
        \prod_{i = 1}^{p - 1} \Phi_p'(\zeta_p^i) = \frac{p^{p - 1}}{1 \cdot \Phi_p(1)} = p^{p - 2}.
    \end{equation*}

    Now using the formula of discriminant in terms of derivatives, we get
    \begin{equation*} 
        \disc(\Phi_p(x)) = (-1)^{\binom{p - 1}{2}}p^{p - 2} = (-1)^{\binom{p}{2}}p^{p - 2}. \qedhere
    \end{equation*}
\end{proof}

\uniquequadraticcyclosubfield*\label{prop:uniquequadraticcyclosubfield2}
\begin{flushright}\hyperref[prop:uniquequadraticcyclosubfield]{\upsym}\end{flushright}
\begin{proof}
    The existence and uniqueness of quadratic subfield is given by \Cref{prop:cyclocyclic}, since $2 \mid p - 1.$ 

    Note that $\disc(\Phi_p(x))$ is not a perfect square in $\mathbb{Q}.$ On the other hand, by definition of $\disc(\Phi_p(x)),$ it is clear that $\disc(\Phi_p(x))$ has a square root in any splitting field of $\Phi_p(x).$ (Recall \Cref{rem:discrepeatedroots}.) Thus, $\sqrt{\disc(\Phi_p(x))} \in \mathbb{Q}(\zeta_p) \setminus \mathbb{Q}.$

    Hence, this generates the unique quadratic extension. Moreover note that
    \begin{equation*} 
        (-1)^{\binom{p}{2}} = (-1)^{\frac{p - 1}{2}}.
    \end{equation*}
    Thus, the square root is real iff $p \equiv 1 \pmod{4}.$
\end{proof}

\quadincyclo*\label{cor:quadincyclo2}
\begin{flushright}\hyperref[cor:quadincyclo]{\upsym}\end{flushright}
\begin{proof}
    Any quadratic extension of $\mathbb{Q}$ is of the form $\mathbb{Q}(\sqrt{d})$ for some square free integer $d.$ (Negative or positive.)

    Let $\zeta_n \vcentcolon= \exp\left(\dfrac{2 \pi \iota}{n}\right).$ Note that $\zeta_n$ is indeed a primitive $n$-th root of unity.

    Let $p$ be an odd prime and note that $\mathbb{Q}(\sqrt{-p}) \subset \mathbb{Q}(\zeta_p)$ if $p \equiv 3 \pmod{4}$ and $\mathbb{Q}(\sqrt{p}) \subset \mathbb{Q}(\zeta_p)$ if $p \equiv 1 \pmod{4}.$ Also, $\sqrt{2} \in \mathbb{Q}(\zeta_8)$.\footnote{Note that $(\zeta_8 + \zeta_8^{-1})^2 = 2.$} Lastly, $\iota \in \mathbb{Q}(\zeta_4)$ and $\mathbb{Q}(\zeta_4) \subset \mathbb{Q}(\zeta_8).$

    Armed with these facts, we note that if $d = \pm p_1 \cdots p_r$ where $p_i$ are distinct odd primes, then,
    \begin{equation*} 
        \mathbb{Q}(\sqrt{d}) \subset \mathbb{Q}(\zeta_{p_1}, \ldots, \zeta_{p_r}, \zeta_4) = \mathbb{Q}(\zeta_{4p_1 \cdots p_r}).
    \end{equation*}
    On the other hand, if $d = \pm 2 p_1 \cdots p_r$ where $p_i$ are distinct odd primes, then,
    \begin{equation*} 
        \mathbb{Q}(\sqrt{d}) \subset \mathbb{Q}(\zeta_{p_1}, \ldots, \zeta_{p_r}, \zeta_8) = \mathbb{Q}(\zeta_{8p_1 \cdots p_r}).
    \end{equation*}
    In both the above equations, the last equality follows from \Cref{ex:compositecyclo}.
\end{proof}

\quadgeneratorcyclo*\label{prop:quadgeneratorcyclo2}
\begin{flushright}\hyperref[prop:quadgeneratorcyclo]{\upsym}\end{flushright}
\begin{proof}
    Note that $\zeta_p$ is a root of the quadratic 
    \begin{equation*} 
        x^2 - (\zeta_p + \zeta_p^{-1})x + 1 \in \mathbb{Q}(\zeta_p + \zeta_p^{-1}).
    \end{equation*}
    Thus, $[\mathbb{Q}(\zeta_p) : \mathbb{Q}(\zeta_p + \zeta_p^{-1})] \le 2.$ The degree will be $1$ iff $\mathbb{Q}(\zeta_p) = \mathbb{Q}(\zeta_p + \zeta_p^{-1}).$ However, note that the latter is contained in $\mathbb{R}$ whereas the former is not. Thus, $[\mathbb{Q}(\zeta_p) : \mathbb{Q}(\zeta_p + \zeta_p^{-1})] = 2.$

    Now, by \Cref{prop:cyclocyclic}, there is a unique intermediate subfield $\mathbb{E}$ of $\mathbb{Q}(\zeta_p)/\mathbb{Q}$ satisfying $[\mathbb{Q}(\zeta_p) : \mathbb{E}] = 2.$ Thus, $\mathbb{E} = \mathbb{Q}(\zeta_p + \zeta_p^{-1}).$
\end{proof}

\fixedfieldcyclosubgroup*\label{prop:fixedfieldcyclosubgroup2}
\begin{flushright}\hyperref[prop:fixedfieldcyclosubgroup]{\upsym}\end{flushright}
\begin{proof}
    Fix $p$ and let $\zeta \vcentcolon= \zeta_p.$

    Clearly, $\beta_H \in \mathbb{Q}(\zeta)^H$ since given any $\tau \in H,$ we have
    \begin{equation*} 
        \tau(\beta_H) = \tau\left(\sum_{\sigma \in H} \sigma(\zeta)\right) = \sum_{\sigma \in H} \tau\sigma(\zeta) = \beta_H,
    \end{equation*}
    since the map $\sigma \mapsto \tau\sigma$ is a bijection from $H$ to itself.

    Thus, $\mathbb{Q}(\beta_H) \subset \mathbb{Q}(\zeta)^H.$ By the Galois correspondence, we know that there exists a subgroup $K$ with $H \le K \le G$ such that
    \begin{equation*} 
        \mathbb{Q}(\beta_H) = \mathbb{Q}(\zeta)^K.
    \end{equation*}
    (In fact, we know exactly what this subgroup is, $\Gal(\mathbb{Q}(\zeta)/\mathbb{Q}(\beta_h)).$)

    It suffices to prove that $H = K.$ Suppose not. Then, $H \subsetneq K$ and $\beta$ is fixed by every element of $K.$ Pick $\tau \in K \setminus H.$ We show that $\tau(\beta_H) \neq \beta_H$ and reach a contradiction. 

    Note that the set
    \begin{equation*} 
        B = \{\zeta, \zeta^2, \ldots, \zeta^{p - 1}\}
    \end{equation*}
    is a $\mathbb{Q}$-basis for $\mathbb{Q}(\zeta).$ Moreover, the above is the set of all roots of $\irr(\zeta, \mathbb{Q}).$ Thus, any $\sigma \in G$ permutes $B.$ Since any $\sigma \in G$ is determined by its action on $\zeta,$ we see that the elements $\sigma(\zeta)$ are distinct for distinct $\sigma \in G$ and hence, linearly independent.

    Thus, if $\tau(\beta_H) = \beta_H,$ then there is some $\sigma \in H$ such that $\tau\sigma = \id_{\mathbb{Q}(\zeta)}$ but then $\tau = \sigma^{-1} \in H,$ a contradiction. Thus, $\tau(\beta_H) \neq \beta_H$ but that contradicts the fact that $K$ fixes $\mathbb{Q}(\beta_H).$ Thus, $\mathbb{Q}(\beta_H) = \mathbb{Q}(\zeta)^H.$
\end{proof}

\section{Abelian and Cyclic extensions}

\pequivonemodn*\label{lem:pequivonemodn2}
\begin{flushright}\hyperref[lem:pequivonemodn]{\upsym}\end{flushright}
\begin{proof}
    Let $k \in \mathbb{Z}$ be such that $\bar{k} \in \mathbb{F}_p$ is a root of $\bar{\Phi}_n(x).$ Then, $p \mid \Phi_n(k)$ in $\mathbb{Z}.$ In turn, $p \mid k^n - 1$ or $k^n \equiv 1 \pmod{p}.$

    We contend that $o(\bar{k}) = n$ in $(\mathbb{F}_p)^\times.$ Suppose not. Then, $m \vcentcolon= o(\bar{k}) < n.$ Then, $m \mid n$ and so, we have
    \begin{align*} 
        x^n - 1 &= \prod_{d \mid n} \Phi_d(x)\\
        &= \Phi_n(x) \prod_{\substack{d \mid n \\ d \neq n}} \Phi_d(x) \\
        &= \Phi_n(x) \cdot \prod_{d \mid m} \Phi_d(x) \cdot \prod_{\substack{d \nmid m \\ d \neq n}} \Phi_d(x)\\
        &= \Phi_n(x) (x^m - 1) h(x)
    \end{align*}
    for some $h(x) \in \mathbb{Z}[x].$ We have used \Cref{thm:cycloreccurence} in the above.

    Going $\mod p$ gives
    \begin{equation*} 
        x^n - 1 = \bar{\Phi}_n(x) (x^m - 1) \bar{h}(x).
    \end{equation*}
    However, note that $\bar{k}$ is a root of both $\bar{\Phi}_n(x)$ and $x^m - 1$ and so, $x^n - 1$ has repeated roots in $\mathbb{F}_p.$ This is a contradiction since $p \nmid n.$

    Thus, $o(\bar{k}) = n$ and in particular, $n \mid (p - 1),$ as desired.
\end{proof}

\infprimesmodone*\label{thm:infprimesmodone2}
\begin{flushright}\hyperref[thm:infprimesmodone]{\upsym}\end{flushright}
\begin{proof}
    Suppose to the contrary that $p_1, \ldots, p_r$ are all such primes. Let $m = np_1 \cdots p_r.$ Consider the cyclotomic polynomial $\Phi_m(x).$ Since it is monic (and non-constant), we have
    \begin{equation*} 
        \lim_{x\to \infty} \Phi_m(mx) = \infty.
    \end{equation*}
    In particular, there exists $k \in \mathbb{N}$ such that $\Phi_m(mk) \ge 2.$ Thus, it has a prime factor $p.$ Then,
    \begin{equation*} 
        p \mid (mk)^m - 1
    \end{equation*}
    and thus, $p \nmid (mk).$ Hence, $\gcd(p, n) = 1.$ Consequently, $p \neq p_1, \ldots, p_r.$ But $\bar{\Phi}_m(\overline{mk}) = 0$ and so, $p \equiv 1 \pmod{mk}.$ In turn, we have
    \begin{equation*} 
        p \equiv 1 \pmod{n},
    \end{equation*}
    a contradiction.
\end{proof}

\fingroupQextension*\label{thm:fingroupQextension2}
\begin{flushright}\hyperref[thm:fingroupQextension]{\upsym}\end{flushright}
\begin{proof}
    We may assume that $\md{G} =\vcentcolon n \ge 2.$ For $m \in \mathbb{N},$ define $C_m \vcentcolon= \mathbb{Z}/m\mathbb{Z}$ and $U(m) \vcentcolon= (\mathbb{Z}/m\mathbb{Z})^\times.$ We have
    \begin{equation*} 
        G \cong C_{n_1} \times \cdots \times C_{n_k}
    \end{equation*}
    for some integers $n_1, \ldots, n_k \ge 2$ with
    \begin{equation*} 
        n = n_1 \cdots n_k.
    \end{equation*}
    Let $p_1, \ldots, p_k$ be distinct primes such that $p_i \equiv 1 \pmod{n_i}$ for all $i = 1, \ldots, k.$ (Existence is given by \Cref{thm:infprimesmodone}.)

    Note that each $U(p_i)$ is cyclic with order $p_i - 1,$ a multiple of $n_i.$ Thus, there exists a subgroup $H_i \le U(p_i)$ with
    \begin{equation*} 
        \frac{U(p_i)}{H_i} \cong C_{n_i},
    \end{equation*}
    for each $i = 1, \ldots, k.$

    Thus, we have 
    \begin{equation*} 
        \frac{U(p_1) \times \cdots \times U(p_k)}{H_1 \times \cdots \times H_k} \cong C_{n_1} \times \cdots \times C_{n_k} \cong G.
    \end{equation*}

    By the Chinese Remainder Theorem, we have 
    \begin{equation*} 
        U(p_1) \times \cdots \times U(p_k) \cong U(m) \cong \Gal(\mathbb{Q}(\zeta_m)/\mathbb{Q}),
    \end{equation*} 
    where $m = p_1 \cdots p_k.$ Let $H$ be the subgroup of $\Gal(\mathbb{Q}(\zeta_m)/\mathbb{Q})$ corresponding to $H_1 \times \cdots \times H_k,$ under this isomorphism.

    Thus, we have
    \begin{equation*} 
        \frac{\Gal(\mathbb{Q}(\zeta_m)/\mathbb{Q})}{H} \cong G.
    \end{equation*}
    By the Galois correspondence, we see that $G \cong \Gal(\mathbb{Q}(\zeta_m)^H/\mathbb{Q}).$
\end{proof}

\dedekindcharacters*\label{thm:dedekindcharacters2}
\begin{flushright}\hyperref[thm:dedekindcharacters]{\upsym}\end{flushright}
\begin{proof}
    If $n = 1,$ then the statement is clearly true since $\chi_1$ does not take the value $0.$

    Suppose that $n \ge 2.$ Suppose that $\chi_1, \ldots, \chi_n$ are linearly dependent. Among all relations of linear dependence, choose $m \ge 2$ to be the one with the least number of non-zero coefficients. (We have $m \ge 2$ by the first line.) By renumbering, we may assume that we have
    \begin{equation*} 
        a_1 \chi_1 + \cdots + a_m \chi_m = 0
    \end{equation*}
    with $a_1, \ldots, a_m \in \mathbb{K} \setminus \{0\}.$ Thus, for any $g \in G,$ we have
    \begin{equation} \label{eq:005}
        a_1 \chi_1(g) + \cdots + a_m \chi_m(g) = 0.
    \end{equation} 
    Now, fix $g_0 \in G$ such that $\chi_1(g_0) \neq \chi_m(g_0).$ (Exists since $m \ge 1$ and $\chi_1 \neq \chi_m.$) Then, \Cref{eq:005} gives
    \begin{equation*} 
        a_1 \chi_1(g_0g) + \cdots + a_m \chi_m(g_0g) = 0
    \end{equation*}
    for all $g \in G.$ Since each $\chi_i$ is a homomorphism, we have
    \begin{equation} \label{eq:006}
        a_1 \chi_1(g_0)\chi_1(g) + \cdots + a_m \chi_m(g_0)\chi_m(g) = 0.
    \end{equation}
    Multiplying \Cref{eq:005} with $\chi_m(g_0)$ and subtracting from \Cref{eq:006} gives
    \begin{equation*} 
        a_1(\chi_1(g_0) - \chi_m(g_0))\chi_1(g) + \cdots + a_{m - 1}(\chi_{m - 1}(g_0) - \chi_{m}(g_0))\chi_{m - 1}(g) = 0.
    \end{equation*}
    The above holds for all $g \in G.$ But the first coefficient is non-zero. This is an equation of linear dependence with $\le m - 1$ non-zero coefficients. This is a contradiction.
\end{proof}

\primeigenvalue*\label{lem:primeigenvalue2}
\begin{flushright}\hyperref[lem:primeigenvalue]{\upsym}\end{flushright}
\begin{proof}
    The order of $\sigma$ is $n$ and hence, it satisfies $T^n - 1 = 0.$ (As an operator.)

    We contend that $T^n - 1 \in \mathbb{F}[T]$ is the minimal polynomial of $\sigma.$ Indeed, if $\sigma$ satisfies a polynomial of degree $m < n,$ then the distinct characters $\sigma, \ldots, \sigma^m$ are linearly dependent. This contradicts \Cref{thm:dedekindcharacters}, since we can view $\sigma, \ldots, \sigma^m$ as distinct characters of $\mathbb{E}^\times$ in $\mathbb{E}.$

    Hence, $T^n - 1$ is the minimal polynomial of $\sigma.$ Since $\zeta \in \mathbb{F}$ is a root of $T^n - 1,$ it is an eigenvalue of $\sigma.$
\end{proof}

(In case you're not aware of minimal polynomials: We have shown that $T^n - 1$ is the least degree polynomial that is satisfied by $\sigma.$ Use this to conclude that $T^n - 1$ divides every polynomial $p(T) \in \mathbb{F}[T]$ such that $p(\sigma) = 0.$ In particular, it must divide the characteristic polynomial (here we use Cayley Hamilton) and thus, $\zeta$ is an eigenvalue.)

\cyclicextprimroot*\label{thm:cyclicextprimroot2}
\begin{flushright}\hyperref[thm:cyclicextprimroot]{\upsym}\end{flushright}
\begin{proof}
    By \Cref{lem:primeigenvalue}, we see that $\zeta$ is an eigenvalue of $\sigma.$ Thus, there exists an eigenvector $a \in \mathbb{E}^\times$ such that $\sigma(a) = \zeta a$ and hence, $\sigma^i(a) = \zeta^i a.$

    Since $\zeta$ is a primitive $n$-th root, we see that $a, \zeta a, \ldots, \zeta^{n - 1}a$ are all distinct and hence, $a$ has at least $n$ Galois conjugates and so, 
    \begin{equation*} 
        [\mathbb{F}(a) : \mathbb{F}] \ge [\mathbb{F}(a) : \mathbb{F}]_s \ge n.
    \end{equation*}
    Since $[\mathbb{E} : \mathbb{F}] = n,$ we see that $\mathbb{F}(a) = \mathbb{E}.$ 

    Now, note that $\sigma(a^n) = (\sigma(a))^n = \zeta^na^n = a^n$ and thus, $a^n \in \mathbb{E}^G = \mathbb{F}.$
\end{proof}

\subfieldsofprimcyclic*\label{prop:subfieldsofprimcyclic2}
\begin{flushright}\hyperref[prop:subfieldsofprimcyclic]{\upsym}\end{flushright}
\begin{proof}
    Clearly, each $\mathbb{F}(a^d)$ is indeed an intermediate subfield of $\mathbb{E}/\mathbb{F}.$ We show that these are the only ones.

    Note that since $G$ is cyclic of order $n,$ it has exactly one subgroup of order $d,$ for every divisor $d$ of $n.$ In turn, $\mathbb{E}/\mathbb{F}$ has exactly one intermediate subfield of degree $n/d$ over $\mathbb{F}.$ We show that $\mathbb{F}(a^d)$ has this property and thus, we have covered all intermediate subfields.

    To this end, first note that $(a^d)^{n/d} \in \mathbb{F}$ and thus,
    \begin{equation*} 
        [\mathbb{F}(a^d) : \mathbb{F}] \le n/d.
    \end{equation*}
    On the other hand, $a$ satisfies $x^d - a^d \in \mathbb{F}(a^d)[x]$ and so,
    \begin{equation*} 
        [\mathbb{E} : \mathbb{F}(a^d)] \le d.
    \end{equation*}

    Since $[\mathbb{E} : \mathbb{F}] = n,$ the \nameref{thm:towerlaw} forces both of the above inequalities to be equalities.
\end{proof}

\artinschreier*\label{thm:artinschreier2}
\begin{flushright}\hyperref[thm:artinschreier]{\upsym}\end{flushright}
\begin{proof}
    \phantom{hi}
    \begin{enumerate}[leftmargin=*]
        \item Let $G \vcentcolon= \Gal(\mathbb{E}/\mathbb{F}) = \langle \sigma\rangle.$ Define the $\mathbb{F}$-linear map $T : \mathbb{E} \to \mathbb{E}$ as 
        \begin{equation*} 
            T \vcentcolon= \sigma - \id_{\mathbb{E}}.
        \end{equation*}
        Note that
        \begin{equation*} 
            \ker(T) = \{a \in \mathbb{E} : \sigma(a) = a\} = \mathbb{E}^G = \mathbb{F}.
        \end{equation*}
        Also, we have
        \begin{equation*} 
            T^p = (\sigma - \id_{\mathbb{E}})^p = \sigma^p - \id_{\mathbb{E}} = 0
        \end{equation*}
        and so, $\im(T^{p - 1}) \subset \ker(T) = \mathbb{F}.$ However, note that $T^{p - 1} \neq 0$ since that would give a non-trivial relation between the distinct $\mathbb{E}^\times$ characters $1, \sigma, \ldots, \sigma^{p - 1},$ contradicting \nameref{thm:dedekindcharacters}.

        Thus, $\im(T^{p - 1})$ is at least one dimensional over $\mathbb{F}.$ Since it is contained in $\mathbb{F},$ we have $\im(T^{p - 1}) = \mathbb{F}.$

        Let $b \in \mathbb{E}$ be such that $T^{p - 1}(b) = 1$ and put $a = T^{p - 2}(b) \in \mathbb{E}.$ Note that
        \begin{equation*} 
            \sigma(a) = T(a) + a = 1 + a.
        \end{equation*}
        Thus, $\sigma^i(a) = i + a$ for $i = 0, \ldots, p - 1.$ All of these are distinct. Thus, $\mathbb{E} = \mathbb{F}(a).$ (Compare the separability degree.)

        Now, note that
        \begin{equation*} 
            \sigma(a^p - a) = (1 + a)^p - (1 + a) = a^p - a
        \end{equation*}
        and thus, $a^p - a \in \mathbb{E}^G = \mathbb{F}.$
        %
        \item Suppose $b \in \mathbb{F}$ is such that $f(x) \vcentcolon= x^p - x - b$ has no root in $\mathbb{F}.$ Let $\mathbb{E}$ be a splitting field of $f(x)$ over $\mathbb{F}$ and let $\alpha \in \mathbb{E}$ be a root. Then, $\alpha + 1, \ldots, \alpha + (p - 1)$ are also roots. Thus,
        \begin{equation*} 
            \mathbb{E} = \mathbb{F}(\alpha, \ldots, \alpha + p - 1) = \mathbb{F}(\alpha).
        \end{equation*}

        Now, write $f(x) = g_1(x) \cdots g_r(x)$ for irreducible $g_i(x) \in \mathbb{F}[x].$ Now, if $\beta \in \mathbb{E}$ is a root of some $g_i(x),$ then $\mathbb{E} = \mathbb{F}(\beta),$ by the same argument as above and hence, each $g_i$ has degree $d \vcentcolon= [\mathbb{F}(\beta) : \mathbb{F}] > 1.$\footnote{Strictly greater since $\beta \notin \mathbb{F}.$} Thus, we have
        \begin{equation*} 
            p = \deg(f(x)) = rd.
        \end{equation*}
        Since $p$ is prime and $d > 1,$ we have $d = p$ and $r = 1.$

        Thus, $[\mathbb{E} : \mathbb{F}] = d = p$ and $G$ is generated by the automorphism $\sigma$ determined by $\sigma(\alpha) = \alpha + 1.$ \qedhere
    \end{enumerate}
\end{proof}

\section{Some Group Theory}

\pgroupssolvable*\label{prop:pgroupssolvable2}
\begin{flushright}\hyperref[prop:pgroupssolvable]{\upsym}\end{flushright}
\begin{proof}
    We prove this by induction on $n.$ If $n = 0, 1,$ then $G$ is abelian and hence, solvable. Suppose $n > 1$ and groups of order $p^{k}$ for $0 \le k < n$ are solvable.

    Let $Z(G) \unlhd G$ denote the center of $G.$ We have $\md{Z(G)} > 1$ and thus, $\overline{G} = G/Z(G)$ is a group of order $p^k$ for some $k < n.$ By induction hypothesis, $\overline{G}$ has a series
    \begin{equation*} 
        \overline{G} = \overline{G_0} \supset \overline{G_1} \supset \cdots \supset \overline{G_{s}} = 1.
    \end{equation*}
    By the correspondence theorem, the above lifts to a series
    \begin{equation*} 
        G = G_0 \supset G_1 \supset \cdots \supset G_{s} = Z(G) \supset G_{s + 1} \vcentcolon= 1.
    \end{equation*}
    Since the quotients $G_i/G_{i - 1}$ are isomorphic to $\overline{G_i}/\overline{G_{i - 1}}$ for $i = 1, \ldots, s,$ we see that the above is an abelian series except possibly at the right-most stage. However, $Z(G)$ is abelian and so, the right-most stage is verified as well.
\end{proof}

\commutatorresults*\label{prop:commutatorresults2}
\begin{flushright}\hyperref[prop:commutatorresults]{\upsym}\end{flushright}
\begin{proof}
    \phantom{hi}
    \begin{enumerate}[leftmargin=*]
        \item Let $g, h \in G.$ Then, 
        \begin{equation*} 
            f([g, h]) = f(g^{-1}h^{-1}gh) = f(g)^{-1}f(h)^{-1}f(g)f(h) = [f(g), f(h)].
        \end{equation*}
        Thus, $f(G') \subset H'$ and we may consider the homomorphism $f'|_{G'} : G' \to H'.$ Applying the result again gives
        \begin{equation*} 
            f(G^{(2)}) = f((G')') \subset (H')' = H^{(2)}.
        \end{equation*}  
        Inductively, we get the result for all $s \ge 1.$

        If $f$ is onto, then every commutator is in the image $f(G')$ and thus, $H' = f(G').$

        Thus, we may consider $f$ as an onto homomorphism $f : G' \to H'.$ As before, induction gives the result for all $s \ge 1.$
        %
        \item Let $a \in G.$ The inner automorphism $i_a : G \to G$ restricts to one of $K$ since $K \unlhd G.$ By the previous part, $i_a(K') = K'$ and thus, $K$ is normal. $G' \unlhd G$ follows since $G \unlhd G.$
        %
        \item $G/K$ is abelian $\iff$ $ghK = hgK$ for all $h, g \in G$ $\iff$ $g^{-1}h^{-1}gh \in K$ for all $g, h \in K$ $\iff$ $G' \le K.$ \qedhere
    \end{enumerate}
\end{proof}

\solvableifftrivialderiv*\label{prop:solvableifftrivialderiv2}
\begin{flushright}\hyperref[prop:solvableifftrivialderiv]{\upsym}\end{flushright}
\begin{proof}
    \forward Suppose $G$ is solvable. Then, there is an abelian series
    \begin{equation} \label{eq:007}
        1 = G_0 \unlhd G_1 \unlhd \cdots \unlhd G_s = G
    \end{equation}
    for $G.$ We show by induction on $s$ that $G^{(s)} = 1.$ 

    If $s = 1,$ then $G$ is abelian and $G^{(1)} = 1.$ Now, let $s > 1$ and assume that $G^{(s - 1)} = 1$ whenever $G$ has an abelian series of length $s - 1.$ Let $G$ be a group with an abelian series of length $s$ as in \Cref{eq:007}. Then,
    \begin{equation*} 
        1 = G_0 \unlhd G_1 \unlhd \cdots \unlhd G_{s - 1}
    \end{equation*}
    is an abelian series for $G_{s - 1}.$ By induction hypothesis, we have $G_{s - 1}^{(s - 1)} = 1.$ Since $G/G_{s - 1}$ is abelian, we have $G' \subset G_{s - 1},$ by \Cref{prop:commutatorresults}. Thus,
    \begin{equation*} 
        G^{(s)} = (G')^{(s - 1)} \subset (G_{s - 1})^{(s - 1)} = 1.
    \end{equation*}

    \backward Suppose that $G^{(s)} = 1$ for some $s.$ Then,
    \begin{equation*} 
        1 = G^{(s)} \unlhd G^{(s - 1)} \unlhd \cdots \unlhd G^{(1)} \unlhd G
    \end{equation*}
    is an abelian series.
\end{proof}

\deriveofquotient*\label{prop:deriveofquotient2}
\begin{flushright}\hyperref[prop:deriveofquotient]{\upsym}\end{flushright}
\begin{proof}
    Let $\pi : G \to G/K$ be the natural onto map. Then, $\pi(G^{(s)}) = (G/K)^{(s)},$ by \Cref{prop:commutatorresults}. By the correspondence theorem, we see that $\langle G^{(s)}, K\rangle/K = (G/K)^{s}.$ 
\end{proof}

\twoofthreesolvable*\label{prop:twoofthreesolvable2}
\begin{flushright}\hyperref[prop:twoofthreesolvable]{\upsym}\end{flushright}
\begin{proof}
    For the first two parts, let $s$ be such that $G^{(s)} = 1.$ (Exists by \Cref{prop:solvableifftrivialderiv}.) Using the same result, it suffices to show that $H^{(s)} = 1$ for the first two parts.
    \begin{enumerate}[leftmargin=*]
        \item $H^{(s)} \cong i(H^{(s)}) \subset G^{(s)} = 1.$
        \item Since $f$ is onto, we have $H^{(s)} = f(G^{(s)}) = 1.$
        \item There exist $s$ and $t$ such that $K^{(s)} = 1$ and $(G/K)^{(t)} = 1.$\\
        By \Cref{prop:deriveofquotient}, we have $(G/K)^{(t)} = \langle G^{(t)}, K\rangle/K.$ Since this is trivial, we have $G^{(t)} \subset K$ and so, $G^{(s + t)} \subset K^{(s)} = 1.$ \qedhere
    \end{enumerate}
\end{proof}

\refiningabelianseries*\label{prop:refiningabelianseries2}
\begin{flushright}\hyperref[prop:refiningabelianseries]{\upsym}\end{flushright}
\begin{proof}
    Since $G$ is solvable, there exists an abelian series
    \begin{equation*} 
        1 = G_0 \unlhd G_1 \unlhd \cdots \unlhd G_s = G.
    \end{equation*}
    We show that between $G_i$ and $G_{i + 1},$ we can insert groups $H_1^{(i)}, \ldots, H_{r_i}^{(i)}$ such that 
    \begin{equation*} 
        G_i \unlhd H_1^{(i)} \unlhd \cdots \unlhd H_{r_i}^{(i)} \unlhd G_{i + 1}
    \end{equation*}
    and each quotient above is cyclic of prime order.

    Note that by the correspondence theorem of subgroups of the original group and a quotient group, it suffices to prove that for $s = 1.$

    That is, assume that $G$ is an abelian group. We show that there exists a chain
    \begin{equation*} 
        1 = G_0 \unlhd \cdots \unlhd G_s = G
    \end{equation*}
    such that the quotients are cyclic of prime order.

    Let $\md{G} = p_1 \cdots p_n,$ where $p_i$ are (not necessarily distinct) primes. We prove the statement by induction on $n.$ If $n = 0$ or $1,$ the result is obvious. Assume $n \ge 2$ and that the result is true for $n - 1.$ Then, since $p_n \mid G,$ there exists an element $g \in G$ order $p_n.$ Let $G_1 \vcentcolon= \langle g\rangle.$ Then, $G_1 \unlhd G$ since $G$ is abelian. By induction, $G/G_1$ has a normal series where the quotients are cyclic of prime order. Lift that chain to complete the proof.
\end{proof}

\Angenerator*\label{lem:Angenerator2}
\begin{flushright}\hyperref[lem:Angenerator]{\upsym}\end{flushright}
\begin{proof}
    Clearly, every three cycle $(a b c) = (a c)(a b)$ is indeed in $A_n.$ Let $H \le A_4$ be the subgroup generated by the $3$-cycles.

    Let $\tau_1 = (i j)$ and $\tau_2 = (r s)$ be distinct transpositions. Then, we have
    \begin{equation*} 
        \tau_1\tau_2 = \begin{cases}
            (i j r)(r s j) & \tau_1 \text{ and } \tau_2 \text{ are disjoint},\\
            (irs) & \text{otherwise}.
        \end{cases}
    \end{equation*}
    Thus, $H$ contains all products of distinct pairs of transpositions. Since these generate $A_n$ (by definition), we have $H = A_n.$

    Now, assume that $n \ge 3.$ Recall that if $\sigma \in S_n$ is any permutation and $(j_1, \ldots, j_k)$ is a $k$-cycle, then
    \begin{equation*} 
        \sigma(j_1, \ldots, j_k)\sigma^{-1} = (\sigma(j_1), \ldots, \sigma(j_k)).
    \end{equation*}

    Now, let $(i j k)$ and $(r s t)$ be any two $3$-cycles. Define $\gamma \in S_n$ by
    \begin{equation*} 
        \gamma(u) \vcentcolon= \begin{cases}
            r & u = i, \\
            s & u = j, \\
            t & u = k, \\
            u & \text{otherwise}.
        \end{cases}
    \end{equation*}
    Clearly, the above is indeed a bijection from $[n]$ to itself. Then, we have
    \begin{equation*} 
        \gamma\cdot(i j k)\cdot\gamma^{-1} = (r s t).
    \end{equation*}
    Thus, if $\gamma$ is even, then the above shows that the $3$-cycles are conjugate in $A_n.$ Otherwise, pick distinct $u, v \in [n] \setminus \{r, s, t\}$ (exist since $n \ge 5$) and define $\sigma \vcentcolon= (i j) \cdot \gamma.$ Then,
    \begin{equation*} 
        \sigma\cdot(i j k)\cdot\sigma^{-1} = (u v)(r s t)(u v)^{-1} = (r s t)
    \end{equation*}
    and $\sigma \in A_n.$
\end{proof}

\SnAnnotsolvable*\label{thm:SnAnnotsolvable2}
\begin{flushright}\hyperref[thm:SnAnnotsolvable]{\upsym}\end{flushright}
\begin{proof}
    In view of \Cref{prop:twoofthreesolvable}, it suffices to show that $A_n$ is not solvable. We now show that $A_n' = A_n$ and hence, $A_n^{(s)} = A_n \neq 1$ for all $s \ge 1.$

    We actually show that every $3$-cycle $(i j k) \in A_n$ is a commutator. Then, by \Cref{lem:Angenerator}, it follows that $A_n' = A_n.$ Since $n \ge 5,$ we can distinct $r, v \in [n] \setminus \{i, j, k\}.$ Then, we have
    \begin{equation*} 
        [(jkv), (ikr)] = (vkj)(rki)(jkv)(ikr) = (vkj)(ivj) = (ikj). \qedhere
    \end{equation*}
\end{proof}

\Ansimple*\label{thm:Ansimple2}
\begin{flushright}\hyperref[thm:Ansimple]{\upsym}\end{flushright}
\begin{proof}
    Suppose $1 \neq N \unlhd A_n.$ We show that $N = A_n.$ If $N$ contains a $3$-cycle, then $N$ contains all $3$-cycles since $N$ is normal in $A_4$ and all $3$-cycles in $A_n$ are conjugates, by \Cref{lem:Angenerator}. But that lemma also tells us that $A_n$ is generated by $3$-cycles. Thus, we get $N = A_4.$ So, it suffices to show that $N$ contains a $3$-cycle.

    For $\sigma \in S_n$ and $j \in [n],$ we say that $j$ is a fixed point of $\sigma$ if $\sigma(j) = j.$ Pick $\sigma \in N \setminus \{1\}$ with maximum number of fixed points in $N \setminus \{1\}.$ We will show that $\sigma$ is a $3$-cycle.

    Write $\sigma = \tau_1 \cdots \tau_g$ where $\tau_1, \ldots, \tau_g$ are disjoint cycles of length at least $2$ and $g \ge 1.$ This is possible since $\sigma \neq 1.$

    \textbf{Case 1.} Each $\tau_i$ has length exactly $2.$ Then, since $\sigma$ is even, we have $g \ge 2.$\\
    Let $\tau_1 = (i j)$ and $\tau_2 = (r s).$ Since $n \ge 5,$ we can fix $k \in [n] \setminus \{i, j, r, s\}$ and set $\tau = (r s k) \in A_n.$ Consider the commutator
    \begin{equation*} 
        \sigma' = [\sigma, \tau] = \sigma^{-1}\underbrace{(\tau^{-1}\sigma\tau)}_{\in N} \in N
    \end{equation*}
    Let $\gamma = \tau_3 \cdots \tau_g$ so that
    \begin{equation*} 
        \sigma = (i j)(r s)\gamma
    \end{equation*}
    with $\gamma$ fixing $i, j, r, s.$ (Since the $\tau$s were disjoint.)

    Note that $\tau\sigma(k) = \tau\gamma(k) = \gamma(k),$ since $\gamma$ restricts to a permutation on $[n] \setminus \{i, j, r, s\}.$ On the other hand, we have $\sigma\tau(k) = \sigma(r) = s \neq k.$ Thus, $\tau\sigma \neq \sigma\tau$ and hence, $\sigma' \neq 1.$

    But note that $\sigma'$ fixes all fixed points of $\sigma,$ with possible exception of $k.$\footnote{By this, we mean that it was possible that $\sigma$ fixed $k.$} However, $\sigma'$ also fixes $i$ and $j.$ Thus, $\sigma' \in N \setminus \{1\}$ has more fixed points than $\sigma.$ A contradiction.

    \textbf{Case 2.} There is some $\tau_i$ with length at least $3.$ Since all the $\tau$s commute, we may assume $\tau_1 = (i j k \ldots)$ has length at least $3.$ If $\sigma = (i j k),$ then we are done.

    Otherwise, there are at least two other elements $r , s$ apart from $i, j, k$ that $\sigma$ does not fix.\footnote{If $g = 1,$ then $\tau_1$ is a cycle with odd number of elements since $\sigma \in A_n.$ If $g \ge 2,$ then $\tau_2$ has at least two elements which it moves.} Let $\tau = (r s k) \in A_n$ and consider $\sigma' = [\sigma, \tau].$ Note that $\sigma'(j) \neq j$ and thus, $\sigma' \neq 1.$ Thus, $\sigma' \in N \setminus \{1\}.$

    However, note that $\sigma'(i) = i$ and $\sigma'$ fixes every fixed element of $\sigma.$ (Since $\tau$ only moves those elements already moved by $\sigma.$) Thus, $\sigma'$ fixes more elements than $\sigma,$ a contradiction.

    Thus, $\sigma$ is a $3$-cycle and we are done.
\end{proof}

\genconsectranpose*\label{thm:genconsectranpose2}
\begin{flushright}\hyperref[thm:genconsectranpose]{\upsym}\end{flushright}
\begin{proof}
    For $n = 2,$ the theorem is clear. Assume $n \ge 3.$ Then, by \Cref{thm:gentranspose}, it suffices to show that every transposition is generated by the above list. Let $(ij) \in S_n$ be a transposition. If $i = 1$ or $j = 1,$ then it is in the above list. Assume $i \neq 1 \neq j.$ Then, we have
    \begin{equation*} 
        (ij) = (1i)(1j)(1i). \qedhere
    \end{equation*}
\end{proof}

\genconsectranposespecial*\label{thm:genconsectranposespecial2}
\begin{flushright}\hyperref[thm:genconsectranposespecial]{\upsym}\end{flushright}
\begin{proof}
    Again, by \Cref{thm:gentranspose}, it suffices to show that every transposition is generated by the above list.

    Let $(a \ b) \in S_n$ be a transposition. Without loss of generality, we assume that $a < b.$\footnote{Note that $(a \ b) = (b \ a).$} We show that $(a \ b)$ is a product of elements of the given list by induction on $b - a.$

    If $b - a = 1,$ then $(a \ b)$ is in the list itself. Assume $b - a = k > 1$ and the theorem is true for $k - 1.$ Note that we have
    \begin{equation*} 
        (a \ b) = (a \ a + 1)(a + 1 \ b)(a \ a + 1).
    \end{equation*}
    Since $(a + 1) - a = 1$ and $b - (a + 1) = k - 1,$ we are done.
\end{proof}

\gentransposecycle*\label{thm:gentransposecycle2}
\begin{flushright}\hyperref[thm:gentransposecycle]{\upsym}\end{flushright}
\begin{proof}
    The theorem is clearly true for $n = 2.$ Assume $n \ge 3.$

    By \Cref{thm:genconsectranposespecial}, it suffices to show the two elements above generate all transpositions of the form $(i \ i + 1)$ for $1 \le i < n.$ 

    Let $\sigma \vcentcolon= (12 \ldots n).$ Then, for $k = 1, \ldots, n - 2,$ we have
    \begin{equation*} 
        \sigma^k(1 \ 2)\sigma^{-k} = (\sigma^k(1) \ \sigma^k(2)) = (k + 1 \ k + 2). \qedhere
    \end{equation*}
\end{proof}

\genprimetranscycle*\label{cor:genprimetranscycle2}
\begin{flushright}\hyperref[cor:genprimetranscycle]{\upsym}\end{flushright}
\begin{proof}
    Let renumbering, we may assume that the transposition is $(12).$ The $p$-cycle is of the form $(1 a_1 \ldots a_{p - 2}) =\vcentcolon \sigma.$ Since $p$ is a prime, there exists $k$ such that $\sigma^k$ is of the form $(1 2 b_3 \ldots b_{p - 3}).$ By renumbering again, we may assume that $b_i = i$ for $i = 3, \ldots, n.$ By \Cref{thm:gentransposecycle}, we are done.
\end{proof}

\section{Galois Groups of Composite Extensions}

\galoisEFK*\label{prop:galoisEFK2}
\begin{flushright}\hyperref[prop:galoisEFK]{\upsym}\end{flushright}
\begin{proof}
    As $\mathbb{E}/\mathbb{F}$ is Galois, $\mathbb{E}$ is a splitting field of a family of separable polynomials $\{f_i(x)\}_{i \in I} \subset \mathbb{F}[x]$ over $\mathbb{F}.$ Then, $\mathbb{E}\mathbb{K}$ is splitting of the same family over $\mathbb{K}$ and thus, is Galois over $\mathbb{K}.$

    Now, assume that $\mathbb{K}/\mathbb{E}$ is also Galois. Then, $\mathbb{K}$ is a splitting field of a family of separable polynomials $\{g_j(x)\}_{j \in J} \subset \mathbb{F}[x]$ over $\mathbb{F}.$ Then, $\mathbb{E}\mathbb{F}$ is a splitting field the the family $\{f_i(x)\}_{i \in I} \cup \{g_j(x)\}_{j \in J} \subset \mathbb{F}[x]$ over $\mathbb{F}$ and thus, Galois.

    Now we show the same for the intersection. Let $\sigma : (\mathbb{E} \cap \mathbb{K}) \to \overline{F}$ be an $\mathbb{F}$-embedding. Extend it to an $\mathbb{F}$-embedding $\tau : \mathbb{E}\mathbb{K} \to \overline{F}.$\\
    Since $\mathbb{E}/\mathbb{F}$ and $\mathbb{K}/\mathbb{F}$ are normal, we get $\tau(\mathbb{E}) = \mathbb{E}$ and $\tau(\mathbb{K}) = \mathbb{K}.$ Therefore, $\tau(\mathbb{E} \cap \mathbb{K}) \subset \mathbb{E} \cap \mathbb{K}.$ But since $(\mathbb{E} \cap \mathbb{K})/\mathbb{F}$ is algebraic, we have $\tau(\mathbb{E} \cap \mathbb{K}) = \mathbb{E} \cap \mathbb{K},$ by \Cref{lem:algebraicautomorphism}. Thus, $\sigma(\mathbb{E} \cap \mathbb{K}) = \mathbb{E} \cap \mathbb{K},$ as desired and so, $\mathbb{E} \cap \mathbb{K}$ is Galois over $\mathbb{F}.$ (We have used \Cref{thm:normalequivalent}.)
\end{proof}

\secondiso*\label{prop:secondiso2}
\begin{flushright}\hyperref[prop:secondiso]{\upsym}\end{flushright}
\begin{proof}
    First note that $\sigma$ is actually well-defined. Indeed, if $\sigma \in \Gal(\mathbb{E}\mathbb{K}/\mathbb{K}),$ then $\sigma$ fixes $\mathbb{K}$ and in particular, $\mathbb{F}.$ Thus, so does $\sigma|_{\mathbb{E}}.$ That is is a homomorphism is clear.

    Now, suppose that $\sigma \in \Gal(\mathbb{E}\mathbb{K}/\mathbb{K})$ is such that $\sigma|_{\mathbb{E}} = \id_{\mathbb{E}}.$ By definition of the Galois group, we have $\sigma|_{\mathbb{K}} = \id_{\mathbb{K}}.$ Thus, $\sigma$ fixes both $\mathbb{E}$ and $\mathbb{K}$ and in turn, $\mathbb{E}\mathbb{K}.$ Hence, $\psi$ is injective.

    Let $H \vcentcolon= \im(\psi) \le G \vcentcolon= \Gal(\mathbb{E}/\mathbb{F}).$ Note that $\mathbb{E} \cap \mathbb{K} \subset \mathbb{E}^H.$ Indeed, if $a \in \mathbb{E} \cap \mathbb{K}$ and $\tau = \psi(\sigma) \in H$ for some $\sigma \in \Gal(\mathbb{E}\mathbb{K}/\mathbb{K}),$ then $\tau(a) = \sigma(a) = a,$ since $\sigma$ fixes $\mathbb{K}.$

    On the other hand, if $a \in \mathbb{E} \setminus (\mathbb{E} \cap \mathbb{K}),$ then $a \in \mathbb{E}\mathbb{K} \setminus \mathbb{K}$ and hence, there exists $\sigma \in \Gal(\mathbb{E}\mathbb{K}/\mathbb{K})$ such that $\sigma(a) \neq a.$ (See \Cref{thm:fixfieldinjectiveIG} and \Cref{rem:nonbasemoved}.) Thus, $a \notin \mathbb{E}^H.$ Hence, $\mathbb{E}^H = \mathbb{E} \cap \mathbb{K}.$ 

    Now, note $H$ is finite since $G$ is so. By \nameref{thm:artin}, we have 
    \begin{equation*} 
        \Gal(\mathbb{E}\mathbb{K}/\mathbb{K}) \cong H = \Gal(\mathbb{E}/\mathbb{E}^H) = \Gal(\mathbb{E}/(\mathbb{E} \cap \mathbb{K})). \qedhere
    \end{equation*}
\end{proof}

\secondisoindex*\label{cor:secondisoindex2}
\begin{flushright}\hyperref[cor:secondisoindex]{\upsym}\end{flushright}
\begin{proof}
    The equation about the degrees follows from \Cref{prop:orderofgalgroup}.

    Thus,
    \begin{equation*} 
        [\mathbb{E}\mathbb{K} : \mathbb{F}] = [\mathbb{E}\mathbb{K} : \mathbb{K}][\mathbb{K} : \mathbb{F}] = [\mathbb{E} : \mathbb{E} \cap \mathbb{K}][\mathbb{K} : \mathbb{F}] = \frac{[\mathbb{E} : \mathbb{F}]}{[\mathbb{E} \cap \mathbb{K} : \mathbb{F}]}[\mathbb{K} : \mathbb{F}].
    \end{equation*}
    The last statement now follows.
\end{proof}

\galoiscompositeproduct*\label{thm:galoiscompositeproduct2}
\begin{flushright}\hyperref[thm:galoiscompositeproduct]{\upsym}\end{flushright}
\begin{proof}
    That $\psi$ is a well-defined homomorphism is clear. (Same proof as \Cref{prop:secondiso}.) Suppose $\sigma \in \ker(\psi).$ Then, $\sigma(a) = a$ for all $a \in \mathbb{E}$ and for all $a \in \mathbb{K}.$ Thus, $\sigma = \id_{\mathbb{E}\mathbb{K}}$ and hence, $\psi$ is injective.

    Suppose that $\mathbb{E} \cap \mathbb{K} = \mathbb{F},$ then by \Cref{cor:secondisoindex}, we have
    \begin{equation*} 
        \md{\Gal(\mathbb{E}\mathbb{K}/\mathbb{F})} = [\mathbb{E}\mathbb{K} : \mathbb{F}] = [\mathbb{E} : \mathbb{F}][\mathbb{K} : \mathbb{F}] = \md{\Gal(\mathbb{E}/\mathbb{F}) \times \Gal(\mathbb{K}/\mathbb{F})}
    \end{equation*}
    and thus, comparing cardinalities gives that $\psi$ is onto as well.
\end{proof}

\section{Normal Closure of an Algebraic Extension}
\normalclosureproperties*\label{prop:normalclosureproperties2}
\begin{flushright}\hyperref[prop:normalclosureproperties]{\upsym}\end{flushright}
\begin{proof}
    \phantom{hi}
    \begin{enumerate}[leftmargin=*]
        \item $\mathbb{K}$ is normal by \Cref{thm:normalequivalent}. That it contains $\mathbb{E}$ is trivial.
        \item Since $\mathbb{K}' \supset \mathbb{E},$ given any $a \in \mathbb{E},$ the polynomial $\irr(a, \mathbb{F})$ must factor completely in $\mathbb{K}',$ by definition of normality. Thus, it contains the splitting field of $\irr(a, \mathbb{F})$ over $\mathbb{F}.$ Since this is true for all $a \in \mathbb{E},$ $\mathbb{K}' \supset \mathbb{K}.$
        \item Write $\mathbb{E} = \mathbb{F}(a_1, \ldots, a_n).$ Then, consider the splitting field $\mathbb{K}$ of $\{\irr(a_i, \mathbb{F}) \mid 1 \le i \le n\}$ over $\mathbb{F}.$ Then, $\mathbb{K}$ is normal over $\mathbb{F}$ and any normal extension of $\mathbb{F}$ must contain $\mathbb{K}.$ Thus, $\mathbb{K}$ is the normal closure. $\mathbb{K}/\mathbb{F}$ is clearly a finite extension.
        \item Since $\irr(a, \mathbb{F})$ is separable over $\mathbb{F}$ for each $a \in \mathbb{E},$ we see that $\mathbb{K}/\mathbb{F}$ is normal, in view of \Cref{prop:seppolysplittingfields}.
        \item Let $K \vcentcolon= \Gal(\mathbb{K}/\mathbb{E}).$ Note that $K$ is not normal in $G$ since $\mathbb{E}/\mathbb{F}$ is not normal. (Recall \Cref{thm:galoisiffnormal}, which was for infinite extensions as well.)

        Thus, we see that $\mathbb{K}^H \supsetneq \mathbb{K}^K = \mathbb{E}.$ By \Cref{thm:galoisiffnormal} again, we see that $\mathbb{K}^H/\mathbb{F}$ is normal. Thus, $\mathbb{K}^H$ is a normal extension of $\mathbb{F}$ containing $\mathbb{E}$ which is contained in $\mathbb{K}.$ By minimality of $\mathbb{K},$ we have $\mathbb{K}^H = \mathbb{K}$ and thus, $H = 1.$ \qedhere
    \end{enumerate}
\end{proof}

\section{Solvability by Radicals}

\radextproperties*\label{prop:radextproperties2}
\begin{flushright}\hyperref[prop:radextproperties]{\upsym}\end{flushright}
\begin{proof}
    \phantom{hi}
    \begin{enumerate}[leftmargin=*]
        \item Let
        \begin{equation*} 
            \mathbb{F} = \mathbb{F}_0 \subset \mathbb{F}_1 \subset \cdots \subset \mathbb{F}_n \subset \mathbb{E}
        \end{equation*}
        and
        \begin{equation*} 
            \mathbb{E} = \mathbb{E}_0 \subset \mathbb{E}_1 \subset \cdots \subset \mathbb{E}_m \subset \mathbb{K}
        \end{equation*}
        be towers of simple radical extensions. Append the two together to see that $\mathbb{K}/\mathbb{F}$ is a radical extension.
        %
        \item Let
        \begin{equation*} 
            \mathbb{F} = \mathbb{F}_0 \subset \mathbb{F}_1 \subset \cdots \subset \mathbb{F}_n \subset \mathbb{E}
        \end{equation*}
        be a tower of simple radical extensions. Then, there exist $a_i \in \mathbb{F}_i$ such that
        \begin{equation*} 
            \mathbb{F}_i = \mathbb{F}_{i - 1}(a_i)
        \end{equation*}
        for $i = 1, \ldots, n,$ such that a power of $a_i$ is in $\mathbb{F}_{i - 1}.$

        Consider the tower
        \begin{equation*} 
            \mathbb{K} \subset \mathbb{K}(a_1) \subset \cdots \subset \mathbb{K}(a_1, \ldots, a_m) = \mathbb{E}\mathbb{K}.
        \end{equation*}

        Clearly, each extension above is a simple radical extension. Thus, $\mathbb{E}\mathbb{K}/\mathbb{K}$ is a radical extension. If $\mathbb{K}/\mathbb{F}$ is also radical, then the previous part gives us that $\mathbb{E}\mathbb{K}/\mathbb{F}$ is also radical. \qedhere
    \end{enumerate}
\end{proof}

\sepgaloisradical*\label{prop:sepgaloisradical2}
\begin{flushright}\hyperref[prop:sepgaloisradical]{\upsym}\end{flushright}
\begin{proof}
    Let $n \vcentcolon= [\mathbb{E} : \mathbb{F}].$ (Note that $n < \infty$ since $\mathbb{E}/\mathbb{F}$ is a radical extension.) Since $\mathbb{E}/\mathbb{F}$ is separable, there are $n$ distinct $\mathbb{F}$-embeddings
    \begin{equation*} 
        \sigma_1, \ldots, \sigma_n : \mathbb{E} \to \overline{\mathbb{F}}.
    \end{equation*}
    We show that $\mathbb{K} = \sigma_1(\mathbb{E}) \cdots \sigma_n(\mathbb{E})$ is the smallest Galois extension of $\mathbb{F}$ containing $\mathbb{E}.$

    By the \nameref{thm:pet}, we know that $\mathbb{E} = \mathbb{F}(a)$ for some $a \in \mathbb{E}.$ Then, the roots of $p(x) \vcentcolon= \irr(a, \mathbb{F})$ in $\overline{\mathbb{F}}$ are precisely $\sigma_1(a), \ldots, \sigma_n(a).$ Let $\mathbb{K} \vcentcolon= \mathbb{F}(\sigma_1(a), \ldots, \sigma_n(a)).$ Then, $\mathbb{K}$ is a splitting field of a separable polynomial and hence, Galois over $\mathbb{K}.$ Moreover, it contains $\mathbb{E}.$ It is clear any such another field must contain $\mathbb{K}.$ Thus, $\mathbb{K}$ satisfies the hypothesis of the theorem.

    Note that we have $\mathbb{K} = \sigma_1(\mathbb{E}) \cdots \sigma_n(\mathbb{E}).$ Since $\sigma(\mathbb{E}_i) \cong \mathbb{E}_i,$ we see that each $\sigma(\mathbb{E}_i)/\mathbb{F}$ is a radical extension and thus, so is $\mathbb{K}/\mathbb{F},$ by \Cref{prop:radextproperties}.
\end{proof}

\solvradicalimpliesgroup*\label{thm:solvradicalimpliesgroup2}
\begin{flushright}\hyperref[thm:solvradicalimpliesgroup]{\upsym}\end{flushright}
\begin{proof}
    Let
    \begin{equation*} 
        \mathbb{F} = \mathbb{F}_0 \subset \mathbb{F}_1 \subset \cdots \subset \mathbb{F}_r = \mathbb{K}
    \end{equation*}
    be a sequence of simple radical extensions with $\mathbb{F}_i = \mathbb{F}_{i - 1}(a_i)$ such that $a_i^{n_i} \in \mathbb{F}_{i - 1}$ for $i = 1, \ldots, r$ and $\mathbb{K}$ contains a splitting field $\mathbb{E}$ of $f(x)$ over $\mathbb{F}.$ 

    Since $\chr(\mathbb{F}) = 0,$ we know that $\mathbb{K}/\mathbb{F}$ is separable. Thus, by \Cref{prop:sepgaloisradical}, we may assume that $\mathbb{K}/\mathbb{F}$ is Galois. Let $n \vcentcolon= n_1 \cdots n_r$ and $\mathbb{L}$ be the splitting field of $x^n - 1$ over $\mathbb{K}.$

    Then, $\mathbb{L} = \mathbb{K}(\omega)$ where $\omega$ is any primitive $n$-th root of unity. Consider the fields $\mathbb{L}_0, \ldots, \mathbb{L}_r = \mathbb{L}$ defined as $\mathbb{L}_i \vcentcolon= \mathbb{F}_i(\omega).$ 

    Since $\mathbb{K}/\mathbb{F}$ is Galois, $\mathbb{K}$ is the splitting of some $g(x) \in \mathbb{F}[x]$ over $\mathbb{F}.$ Then, $\mathbb{L}$ is a splitting field of $(x^n - 1)g(x) \in \mathbb{F}[x]$ over $\mathbb{F}.$ Thus, $\mathbb{L}$ is Galois over $\mathbb{F}$ and in turn, over all $\mathbb{L}_i.$

    Let $H_i \vcentcolon= \Gal(\mathbb{L}/\mathbb{L}_i)$ for $i = 0, \ldots, r.$ See the diagram (at the end of this proof) for a picture. By \hyperref[thm:FTGT]{FTGT}, we have 
    \begin{equation*} 
        \G_f \cong \Gal(\mathbb{E}/\mathbb{F}) \cong \frac{\Gal(\mathbb{L}/\mathbb{F})}{\Gal(\mathbb{L}/\mathbb{E})}.
    \end{equation*}
    (Note that $\mathbb{L}/\mathbb{E}$ is normal since $\mathbb{L}$ is a splitting field over $\mathbb{E}.$)

    Thus, to prove that $\G_f$ is solvable, it is enough to prove that $\Gal(\mathbb{L}/\mathbb{F})$ is solvable, by \Cref{prop:twoofthreesolvable}. 

    Note that $\mathbb{L}_i = \mathbb{L}_{i - 1}(a_i)$ and that $\mathbb{L}_{i - 1} \ni \omega$ and so, $\mathbb{L}_{i - 1}$ contains a primitive $n_i$-th root of unity. Thus, $\mathbb{L}_i$ is a splitting field of $x^{n_i} - a_i^{n_i} \in \mathbb{L}_{i - 1}$ over $\mathbb{L}_{i - 1}.$ Hence, $\mathbb{L}_i/\mathbb{L}_{i - 1}$ is Galois. Thus, $H_{i - 1} \unlhd H_i$ for all $i = 1, \ldots, r.$

    Moreover, by \Cref{prop:nthrootsnonunity}, we see that $\Gal(\mathbb{L}_i/\mathbb{L}_{i - 1})$ is cyclic. Since $H_{i}/H_{i - 1} \cong \Gal(\mathbb{L}_i/\mathbb{L}_{i - 1}),$ we see that
    \begin{equation*} 
        1 = H_r \unlhd H_{r - 1} \unlhd \cdots \unlhd H_0 = \Gal(\mathbb{L}/\mathbb{L}_0)
    \end{equation*}
    is an abelian series for $\Gal(\mathbb{L}/\mathbb{L}_0)$ and hence, it is solvable.

    On the other hand, we know that $\Gal(\mathbb{L}_0/\mathbb{F})$ is abelian, by \Cref{prop:Gfabeliansubgroup}. Again, by \Cref{prop:twoofthreesolvable}, we see that $\Gal(\mathbb{L}/\mathbb{F})$ is solvable, as desired.

    \begin{center}
        \begin{tikzcd}
                         &  &                                                                         &  & \mathbb{K}(\omega) = \mathbb{L}_r = \mathbb{L} \arrow[rr, no head]         &  & \Gal(\mathbb{L}/\mathbb{L}) = 1                            \\
                         &  & \mathbb{K} = \mathbb{F}_r \arrow[rru, no head]                                   &  &                                                                   &  &                                                            \\
                         &  &                                                                         &  & \mathbb{F}_{r-1}(\omega) = \mathbb{L}_{r-1} \arrow[rr, no head] \arrow[uu, no head] &  & \Gal(\mathbb{L}/\mathbb{L}_{r - 1}) = H_{r - 1} \arrow[uu, no head] \\
                         &  & \mathbb{F}_{r-1} \arrow[uu, no head] \arrow[rru, no head]                                 &  &                                                                   &  &                                                            \\
        \mathbb{E} \arrow[rruuu, no head] &  &                                                                         &  & \mathbb{F}_1(\omega) = \mathbb{L}_1 \arrow[rr, no head] \arrow[uu, dashed, no head] &  & \Gal(\mathbb{L}/\mathbb{L}_1) = H_1 \arrow[uu, dashed, no head]     \\
                                 &  & \mathbb{F}_1 \arrow[rru, no head] \arrow[uu, dashed, no head]                             &  &                                                                   &  &                                                            \\
                                 &  &                                                                         &  & \mathbb{F}_0(\omega) = \mathbb{L}_0 \arrow[rr, no head] \arrow[uu, no head]         &  & \Gal(\mathbb{L}/\mathbb{L}_0) = H_0 \arrow[uu, no head]             \\
                                 &  & \mathbb{F} = \mathbb{F}_0 \arrow[rru, no head] \arrow[uu, no head] \arrow[lluuu, no head] &  &                                                                   &  &                                                           
        \end{tikzcd}
    \end{center}    
\end{proof}

\solvgroupimpliesradical*\label{thm:solvgroupimpliesradical2}
\begin{flushright}\hyperref[thm:solvgroupimpliesradical]{\upsym}\end{flushright}
\begin{proof}
    Let $\mathbb{K}$ be a splitting field of $f(x)$ over $\mathbb{F}$ and $[\mathbb{K} : \mathbb{F}] = n.$ Let $\mathbb{L}$ be a splitting field of $x^n - 1$ over $\mathbb{K}$ and $\omega \in \mathbb{L}$ be a primitive $n$-th root of unity. We have $\mathbb{L} = \mathbb{K}(\omega).$ Put $\mathbb{E} = \mathbb{F}(\omega).$ Then, $\mathbb{L}$ is a splitting of $f(x)$ over $\mathbb{E}.$\footnote{The embedding is given as $\sigma \mapsto \sigma_{\mathbb{K}}.$ It is injective because $\sigma$ fixes $\omega$ to begin with.} Since $H = \Gal(\mathbb{L}/\mathbb{E})$ embeds into $\Gal(\mathbb{K}/\mathbb{F}) \cong \G_f,$ $H$ is also a solvable group, by \Cref{prop:twoofthreesolvable2}. Note that $\mathbb{E}/\mathbb{F}$ is a simple radical extension. Thus, if we show that $\mathbb{L}/\mathbb{E}$ is a radical extension, then we are done. (\Cref{prop:radextproperties}.)

    Since $H$ is finite, by \Cref{prop:refiningabelianseries}, we have an abelian series
    \begin{equation*} 
        1 = H_k \unlhd H_{k - 1} \unlhd \cdots \unlhd H_0 = H
    \end{equation*}
    such that $H_i/H_{i + 1}$ is cyclic of prime order $p_{i + 1}$ for $i = 0, \ldots, k - 1.$ Note that $n = p_1 \cdots p_k.$

    Let $\mathbb{E}_i \vcentcolon= \mathbb{L}^{H_i}$ for $i = 1, \ldots, k.$ Then, $[\mathbb{E}_i : \mathbb{E}_{i - 1}] = \md{H_{i - 1}/H_i} = p_i.$ Since $\mathbb{E}_{i - 1}$ contains $\omega,$ it has a primitive $p_i$-th root of unity. Thus, $\mathbb{E}_i/\mathbb{E}_{i - 1}$ is a simple radical extension, by \Cref{thm:cyclicextprimroot}. Thus, $\mathbb{L}/\mathbb{E}$ is a radical extension.
\end{proof}