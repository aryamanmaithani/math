\documentclass[12pt,oneside]{book}
\usepackage{amsmath, amssymb, amsfonts, amsthm, mathtools}
\usepackage{thmtools}
\usepackage{witharrows}
\usepackage[utf8]{inputenc}
\usepackage[inline]{enumitem}
\usepackage[colorlinks=true]{hyperref}
\usepackage{parskip}
\usepackage{tikz-cd}
\usepackage{tikz}
\usetikzlibrary{decorations.markings}
\usetikzlibrary{arrows.meta}

\numberwithin{equation}{chapter}
\theoremstyle{definition}

\newtheorem{fakethm}{Theorem}[chapter]
\newtheorem{fakeprop}[fakethm]{Proposition}
\newtheorem{fakelem}[fakethm]{Lemma}
\newtheorem{fakecor}[fakethm]{Corollary}
\newtheorem{fakerem}[fakethm]{Remark}

\newtheorem{fakedefn}[fakethm]{Definition}
\newtheorem{fakeex}[fakethm]{Example}
\newtheorem{fakeexe}[fakethm]{Exercise}
% \newtheorem*{conject}{Conjecture}

% -------------------
% Coloured theorem and other environments
% -------------------
\usepackage{framed}
\newenvironment{thm}
{\colorlet{shadecolor}{orange!15}\begin{shaded}\begin{fakethm}}
{\end{fakethm}\end{shaded}}

\newenvironment{prop}
{\colorlet{shadecolor}{orange!15}\begin{shaded}\begin{fakeprop}}
{\end{fakeprop}\end{shaded}}

\newenvironment{lem}
{\colorlet{shadecolor}{orange!15}\begin{shaded}\begin{fakelem}}
{\end{fakelem}\end{shaded}}

\newenvironment{cor}
{\colorlet{shadecolor}{orange!15}\begin{shaded}\begin{fakecor}}
{\end{fakecor}\end{shaded}}

\newenvironment{rem}
{\colorlet{shadecolor}{red!25}\begin{shaded}\begin{fakerem}}
{\end{fakerem}\end{shaded}}

\newenvironment{defn}
{\colorlet{shadecolor}{green!15}\begin{shaded}\begin{fakedefn}}
{\end{fakedefn}\end{shaded}}

\newenvironment{exe}
{\colorlet{shadecolor}{purple!15}\begin{shaded}\begin{fakeexe}}
{\end{fakeexe}\end{shaded}}

\newenvironment{ex}
{\colorlet{shadecolor}{purple!15}\begin{shaded}\begin{fakeex}}
{\end{fakeex}\end{shaded}}

\newenvironment{disclaimer}
{\begin{tcolorbox}[parbox=false, colback = red!30]\bfseries}
{\end{tcolorbox}}

\newenvironment{aside}
{\begin{tcolorbox}[parbox=false]}
{\end{tcolorbox}}

\setlength\parindent{0pt}

\pagestyle{plain}

\usepackage{titlesec}
\titleformat{\section}[block]{\sffamily\Large\filcenter\bfseries}{\S\thesection.}{0.25cm}{\Large}
\titleformat{\subsection}[block]{\large\bfseries\sffamily}{\S\S\thesubsection.}{0.2cm}{\large}

\usepackage[a4paper]{geometry}
\usepackage{lipsum}

\usepackage{cleveref}
\crefname{fakethm}{theorem}{theorems}
\crefname{fakelem}{lemma}{lemmas}
\crefname{fakedefn}{definition}{definitions}
\crefname{fakeprop}{proposition}{propositions}
\crefname{fakecor}{corollary}{corollaries}
\crefname{fakeexe}{exercise}{exercises}
\crefname{fakeex}{example}{examples}
\crefname{fakerem}{remark}{remarks}
\crefname{equation}{}{}

\newcommand{\downsym}{[$\downarrow$]}
\newcommand{\upsym}{[$\uparrow$]}

\usepackage{mdframed}
\newenvironment{blockquote}
{\begin{mdframed}[skipabove=0pt, skipbelow=0pt, innertopmargin=4pt, innerbottommargin=4pt, bottomline=false,topline=false,rightline=false, linewidth=2pt]}
{\end{mdframed}}


% \renewcommand{\familydefault}{\sfdefault}
\usepackage{mathpazo}
\usepackage{euler}

\usepackage{fancyhdr}
\setlength{\headheight}{15.2pt}
\pagestyle{fancy}
\fancyhf{}
\fancyhead[L]{\sffamily{\S\textbf{\nouppercase{\rightmark}}}}
\fancyhead[r]{\sffamily{\thepage}}

\newcommand{\deff}[1]{{\color{blue}#1}}
\newcommand{\md}[1]{{\left\lvert #1 \right\lvert}}
\newcommand{\andd}{\quad\text{and}\quad}
\newcommand{\forward}{($\Rightarrow$)\ }
\newcommand{\backward}{($\Leftarrow$)\ }

\DeclareMathOperator{\chr}{char}
\DeclareMathOperator{\Frac}{Frac}
\DeclareMathOperator{\irr}{irr}
\DeclareMathOperator{\id}{id}
\DeclareMathOperator{\disc}{disc}
\DeclareMathOperator{\D}{D}
\DeclareMathOperator{\G}{G}
\DeclareMathOperator{\Gal}{Gal}
\DeclareMathOperator{\Aut}{Aut}
\DeclareMathOperator{\im}{im}
\DeclareMathOperator{\Stab}{Stab}
\DeclareMathOperator{\N}{N}
\DeclareMathOperator{\Tr}{Tr}
\DeclareMathOperator{\Hom}{Hom}
\let\subset\subseteq
\let\supset\supseteq
\let\le\leqslant
\let\ge\geqslant
\let\emptyset\varnothing

\title{MA-414\\Galois Theory}
\author{Aryaman Maithani\\\url{https://aryamanmaithani.github.io/}}
\date{Last updated: \today}

\begin{document}
\maketitle
\tableofcontents

\chapter*{Preface}
\addcontentsline{toc}{chapter}{Preface}
These are my notes on Galois theory, based on \href{https://nptel.ac.in/courses/111101001}{this NPTEL course}. It is assumed that the reader has taken a first course on group and ring theory. Some basic results are stated with proof in the next chapter, along with notations. 

I had wanted to make these notes (for my reference) as a way to collect the results in one place. Due to this, the proofs are all at the end. You can jump from a result to a proof (and back) using the hyperlinked arrows.
\pagestyle{fancy}

\setcounter{chapter}{-1}
\chapter{Preliminaries} \label{chap:00}

\section{Notations and Conventions}
\begin{enumerate}
    \item $\mathbb{N}$ will denote the set of \textbf{positive} integers. That is, $\mathbb{N} = \{1, 2, \ldots\}.$
    \item $\mathbb{Z}$ will denote the set of integers.
    \item $\mathbb{N}_0$ will denote the set of all \textbf{non-negative} integers. \\
    That is, $\mathbb{N}_0 = \{0, 1, 2, \ldots\} = \mathbb{N} \cup \{0\}.$
    \item $\mathbb{Q}$ will denote the set of rationals.
    \item $\mathbb{R}$ will denote the set of real numbers.
    \item $\mathbb{C}$ will denote the set of complex numbers.
    \item Blackboard letters like $\mathbb{F}, \mathbb{E}, \mathbb{K}, \mathbb{L}$ will denote an arbitrary field.
    \item Given any field $\mathbb{F},$ $\mathbb{F}^\times$ denotes the group of units of $\mathbb{F}.$ That is, $\mathbb{F}^\times = \mathbb{F}\setminus\{0\}.$
    \item Given a ring $R,$ $R^\times$ denotes the group of units of $R.$ 
    \item Whenever we write ``$F \subset E$ are fields,'' we mean that $\mathbb{E}$ is a field and $\mathbb{F}$ is a subfield of $\mathbb{E}.$
    \item $\zeta_n \vcentcolon= \exp\left(\dfrac{2\pi\iota}{n}\right).$
    \item The degree of the zero polynomial is $-\infty.$
    \item Given a group $G$ and $g \in G,$ we denote the order of $g$ (in $G$) as $o(g).$
    % \item Given a group $G$ and subgroups $H_1, H_2 \unlhd G,$ we denote by $\langle H_1, H_2\rangle$ the smallest subgroup of $G$ containing $H_1$ and $H_2.$
    \item For $n \ge 1,$ we denote $\{1, \ldots, n\}$ as $[n].$
\end{enumerate}

\section{Field Theory}

We shall assume that the reader is familiar with the definitions and basic properties of groups and rings. All rings in this document will be assumed to be commutative with identity. 

We list some basic definition and properties. The proofs might be a bit terse and you should not have much problem filling in the details. (This won't be the case in the later chapters!)

\begin{defn}%[]
    An \deff{integral domain} is a ring with $0 \neq 1$ such $ab = 0 \implies a = 0$ or $b = 0.$
\end{defn}

\begin{defn}%[Field]
    A \deff{field} $(\mathbb{F}, +, \cdot)$ is a ring with $0 \neq 1$ such that every non-zero element has a multiplicative inverse.
\end{defn}

\begin{ex}
    $\mathbb{Q}, \mathbb{R}, \mathbb{C}$ are all fields.
\end{ex}

\begin{defn}%[]
    Given an integral domain $R,$ the field of fractions of $R$ is denoted by $\Frac(R).$
\end{defn}

\begin{defn}%[]
    A \deff{ring homomorphism} is a map $\varphi : R \to S$ between rings such that
    \begin{enumerate}
        \item $\varphi(ab) = \varphi(a)\varphi(b)$ for all $a, b \in R,$
        \item $\varphi(a + b) = \varphi(a) + \varphi(b)$ for all $a, b \in R,$
        \item $\varphi(1_R) = 1_S.$
    \end{enumerate}
    A \deff{field homomorphism} is a ring homomorphism between fields.
\end{defn}

\begin{defn}%[]
    Given a prime $p \in \mathbb{N},$ $\mathbb{Z}/p\mathbb{Z}$ is a field, which we denote as $\mathbb{F}_p.$
\end{defn}

\begin{defn}%[]
    Let $\mathbb{F}$ be a field. The \deff{characteristic} of $\mathbb{F}$ is defined to be the smallest positive integer $n$ such that
    \begin{equation*} 
        \underbrace{1_{\mathbb{F}} + \cdots + 1_{\mathbb{F}}}_{n} = 0_{\mathbb{F}}.
    \end{equation*}
    If no such $n$ exists, then the characteristic is defined to be $0.$ 

    This is denoted by $\chr \mathbb{F}.$
\end{defn}

From now on, we shall omit the subscript $\mathbb{F}$ when it is clear what the $0$ and $1$ are.

\begin{prop}
    If $\chr \mathbb{F} > 0,$ then $\chr \mathbb{F}$ is prime.
\end{prop}
\begin{proof} 
    Let $n \vcentcolon= \chr \mathbb{F}$ and let $n = ab$ for some $a, b \in \mathbb{F}.$ By distributivity and definition of $n,$ we have
    \begin{equation*} 
        \underbrace{(1 + \cdots + 1)}_{a}\underbrace{(1 + \cdots + 1)}_{b} = 0.
    \end{equation*}
    Since $\mathbb{F}$ is a field, one of the above two terms is $0.$ Without loss of generality, the first term is $0.$ By definition, $n = \chr \mathbb{F} \le a.$ But $a \mid n \implies a \le n.$

    Thus, $a = n.$
\end{proof}

\begin{prop}
    Every field contains an isomorphic copy of either $\mathbb{Q}$ or $\mathbb{F}_p$ for some prime $p.$ In fact, this copy is precisely $\Frac(\mathbb{Z}/\langle \chr\mathbb{F}\rangle).$
\end{prop}
\begin{proof} 
    Given a field $\mathbb{F},$ consider the ring homomorphism $\varphi : \mathbb{Z} \to \mathbb{F}$ given by $1 \mapsto 1.$ \\
    Then, $\mathbb{F}$ contains an isomorphic copy of $\mathbb{Z}/\ker \varphi.$ Note that $\varphi = \langle n\rangle,$ where $n = \chr\mathbb{F}.$ If $n > 0,$ then $n$ is prime and we are done.

    If $n = 0,$ then $\mathbb{F}$ contains an isomorphic copy of $\mathbb{Z}.$ Thus, it must contain $\mathbb{Q}.$\footnote{Either argue by explicitly constructing an isomorphism or use the universal property of fraction fields.}
\end{proof}

\begin{defn}%[]
    Given a field $\mathbb{F},$ the \deff{prime subfield} of $\mathbb{F}$ is defined as the smallest subfield of $\mathbb{F}.$ It is the intersection of all subfields of $\mathbb{F}.$ 
\end{defn}

\begin{prop}
    \phantom{hi}
    \begin{enumerate}
        \item The prime subfield of $\mathbb{F}$ is isomorphic to $\Frac(\mathbb{Z}/\langle \chr\mathbb{F}\rangle).$
        \item Let $\varphi : \mathbb{F} \to \mathbb{E}$ be a field homomorphism. Then, $\chr \mathbb{F} = \chr \mathbb{E}$ and $\varphi$ is injective. 
        \item Let $\mathbb{F} \subset \mathbb{E}$ be fields. $\mathbb{F}$ and $\mathbb{E}$ have the same prime subfield. Any field homomorphism $\varphi : \mathbb{F} \to \mathbb{E}$ fixes this prime subfield.
    \end{enumerate}
\end{prop}

\begin{defn}%[]
    Since any field homomorphism is injective, we also call them \deff{embeddings}.
\end{defn}

\begin{defn}
    Given fields $\mathbb{F} \subset \mathbb{E}_1, \mathbb{E}_2,$ and \deff{$\mathbb{F}$-isomorphism} from $\mathbb{E}_1$ to $\mathbb{E}_2$ is a field homomorphism $\varphi : \mathbb{E}_1 \to \mathbb{E}_2$ fixing $\mathbb{F}.$
\end{defn}

\begin{defn}%[]
    Given rings $R \subset S,$ and $\alpha \in S,$ we define $R[\alpha]$ to be the smallest subring of $S$ containing $\alpha$ and $R.$ 

    Given fields $\mathbb{F} \subset \mathbb{K},$ and $\alpha \in \mathbb{K},$ we define $\mathbb{F}(\alpha)$ to be the smallest subfield of $\mathbb{K}$ containing $\alpha$ and $\mathbb{F}.$ 

    Similarly, given a set $A \subset R$ (or $A \subset \mathbb{F}$), we can talk about $R[A]$ (or $\mathbb{F}(A)$) to be the smallest ring (or subfield) \deff{generated by $A$ over $R$ (or $\mathbb{F}$)}.
\end{defn}

\begin{prop} \label{prop:FAdesc}
    Let $\mathbb{F} \subset \mathbb{E}$ be fields and $A \subset \mathbb{E}$ a set. 

    If $A = \emptyset,$ then $\mathbb{F}(A) = \mathbb{F}.$ Assume $A \neq \emptyset.$

    Let 
    \begin{equation*} 
        M \vcentcolon= \{a_1a_2 \cdots a_n \mid n \in \mathbb{N},\; a_1, \ldots, a_n \in A\}
    \end{equation*}
    be the set of all finite products (monomials) of elements of $A.$

    Let
    \begin{equation*} 
        S \vcentcolon= \{b_0 + b_1m_1 + \cdots + b_nm_n \mid n \in \mathbb{N}_{0},\; m_1, \ldots, m_n \in M,\;b_0, b_1, \ldots, b_n \in \mathbb{F}\}
    \end{equation*}
    be the set of all finite sums of elements of $M.$ (These are polynomials in $A$ with coefficients in $\mathbb{F}.$)

    Then,
    \begin{equation} \label{eq:FAdesc}
        \mathbb{F}(A) = \left\{\frac{s_1}{s_2} \mid s_1, s_2 \in S \text{ and } s_2 \neq 0\right\}.
    \end{equation}
\end{prop}

\begin{proof} 
    The case $A = \emptyset$ is trivial. Assume $A \neq \emptyset.$

    Let the set on the right in \Cref{eq:FAdesc} be called $Q.$ 

    Note that $M$ is closed under products and $S$ is closed under sums and products both. Moreover, $S$ contains $\mathbb{F}$ as the constant polynomials. Using this, it is clear that $Q$ is a subfield of $\mathbb{E}.$ By taking denominator $1,$ we also see that $S \subset Q.$ Since $\mathbb{F} \subset S$ and $A \subset M \subset S,$ we see that $Q$ is a subfield of $\mathbb{E}$ containing $A$ and $\mathbb{F}.$ Thus, $\mathbb{F}(A) \subset Q.$

    On the other hand, note that $M \subset \mathbb{F}(A)$ since $A \subset \mathbb{F}(A).$ Since $\mathbb{F} \subset \mathbb{F}(A)$ as well, we get $S \subset \mathbb{F}(A).$ Thus, $Q \subset \mathbb{F}(A).$ (In all the assertions, we have used that $\mathbb{F}(A)$ is a subfield of $\mathbb{E}$ and thus, has the required closure properties.)
\end{proof}

\begin{cor} \label{cor:FAdescfinite}
    Let $\mathbb{F} \subset \mathbb{E}$ be fields and $A \subset \mathbb{E}$ a set. If $a \in \mathbb{F}(A),$ then there exists a finite set $B \subset A$ such that $a \in \mathbb{F}(B).$
\end{cor}

\begin{proof} 
    Let $a \in F(A).$ Let $M, S$ be as in \Cref{prop:FAdesc}. Then, $a = s_1/s_2$ for some $s_1, s_2 \in S.$ Then, each $s_i$ is a polynomial in some finitely many $a_i \in A$ with coefficients in $\mathbb{F}.$ Let $B$ be the set of finitely many $a_i.$ Then, $a \in \mathbb{F}(B).$
\end{proof}

\begin{prop}
    If $\mathbb{F}$ is a finite field, then $\chr(\mathbb{F}) =\vcentcolon p > 0$ and $\md{\mathbb{F}} = p^n$ for some $n \in \mathbb{N}.$
\end{prop}
\begin{proof} 
    $\chr(\mathbb{F}) = 0$ is not possible since $\mathbb{Z}$ is infinite and so, the homomorphism $\varphi : \mathbb{Z} \to \mathbb{F}$ given by $1 \mapsto 1$ cannot be injective.

    Now, $\mathbb{F}$ contains $\mathbb{F}_p$ as a subfield and hence, is a vector space over $\mathbb{F}.$ Since $\md{\mathbb{F}} < \infty,$ we have $\dim_{\mathbb{F}_p}(\mathbb{F}) =\vcentcolon n < \infty.$

    It is clear now that $\md{\mathbb{F}} = \md{\mathbb{F}_p}^n = p^n.$
\end{proof}

\begin{thm}
    Let $f(x) \in \mathbb{F}[x]$ have a degree $n \ge 1.$ Then, $f(x)$ has at most $n$ roots in $\mathbb{F}.$
\end{thm}
\begin{proof} 
    Induct on $n$ and use the fact that if $ab = 0 \implies a = 0$ or $b = 0,$ in a field.
\end{proof}

\begin{thm} \label{thm:finsubgroupcyclic}
    Let $\mathbb{F}$ be a field. Let $U$ be a finite subgroup of $\mathbb{F}^\times.$ Then, $U$ is cyclic. 
\end{thm}
We give two proofs.
\begin{proof} 
    This proof uses the following fact: Let $G$ be an abelian group and $a, b \in G$ have orders $m$ and $n.$ Then, there exist $c \in G$ with order $\operatorname{lcm}(m, n).$ (This needs a little argument. $c = ab$ works if $\gcd(m, n) = 1.$ The general case has to be reduced to that.)

    Let $n \vcentcolon= \md{U}.$ Let $a \in U$ be an element with maximal order, say $d.$ Then, we have
    \begin{equation*} 
        d = \operatorname{lcm} \{\operatorname{order}(u) \mid u \in U\}.
    \end{equation*}
    Thus, all $n$ elements of $U \subset \mathbb{F}$ satisfy the polynomial $x^d - 1 \in \mathbb{F}[x].$ Since $\mathbb{F}$ is a field, we have $n \le d.$ Thus, $d = n$ and $U = \langle a\rangle.$
\end{proof}

\begin{proof} 
    This prove uses the structure theorem of abelian groups. Let $n \vcentcolon= \md{U}.$

    Write $U \cong \mathbb{Z}/d_1\mathbb{Z} \times \cdots \times \mathbb{Z}/d_r\mathbb{Z}$ where $1 < d_1 \mid d_2 \mid \cdots \mid d_r$ and $n = d_1 \cdots d_r.$ Now, every element of $U$ satisfies $x^{d_r} - 1.$ Thus, as earlier, we have $d_r = n$ and hence, $n = 1.$ This means $U \cong \mathbb{Z}/n\mathbb{Z}$ is cyclic. 
\end{proof}

\begin{prop} \label{prop:divisibilityofpoly}
    Let $\mathbb{F} \subset \mathbb{K}$ be fields and $f(x), g(x) \in \mathbb{F}[x].$ \\
    Then, $f(x) \mid g(x)$ in $\mathbb{F}[x]$ iff every root of $f(x) \mid g(x)$ in $\mathbb{K}[x].$

    In particular, if $f(x)$ factorises linearly into distinct factors in $\mathbb{K}[x],$ then it suffices to show that every root of $f(x)$ is also one of $g(x).$
\end{prop}

\begin{proof}
    \forward This is obvious because a factorisation $g(x) = f(x)h(x)$ in $\mathbb{F}[x]$ also holds in $\mathbb{K}[x].$

    \backward If $f(x) = 0,$ then the result is true. Assume $f(x) \neq 0.$ \\
    By the division algorithm, we may write
    \begin{equation*} 
        g(x) = f(x)q(x) + r(x)
    \end{equation*}
    for unique $q(x), r(x) \in \mathbb{F}[x]$ with $\deg(r(x)) < \deg(q(x)).$

    The above is also a division in $\mathbb{K}[x].$ But $f(x) \mid g(x)$ in $\mathbb{K}[x]$ and so, uniqueness forces $r(x) = 0.$
\end{proof} 
\chapter{Algebraic extensions}
\section{Extensions and Degrees}

\begin{defn}%[]
    Let $\mathbb{F}$ be a subfield of $\mathbb{K}.$ We say that $\mathbb{K}$ is an \deff{extension field} of $\mathbb{F}$ and $\mathbb{F}$ is called the base field. We also denote this by $\mathbb{K}/\mathbb{F}.$
\end{defn}

\begin{rem}
    The above is not to be confused with any sort of quotient. In fact, since the only ideals of a field $\mathbb{K}$ are $0$ and $\mathbb{K},$ there is no discussion about quotienting.
\end{rem}

\begin{defn}%[]
    Let $\mathbb{K}/\mathbb{F}$ be a field extension. Then, we may regard $\mathbb{K}$ as a vector space over $\mathbb{F}.$ We denote $\dim_{\mathbb{F}}\mathbb{K}$ by $[\mathbb{K} : \mathbb{F}]$ and call it the \deff{degree} of the field extension $\mathbb{K}/\mathbb{F}.$
\end{defn}

\begin{defn}%[]
    The field extension $\mathbb{K}/\mathbb{F}$ is said to be a \deff{finite extension} if $[\mathbb{K} : \mathbb{F}]$ is finite. 
\end{defn}

\begin{defn}%[]
    The field extension $\mathbb{K}/\mathbb{F}$ is said to be a \deff{simple extension} if there exists $\alpha \in \mathbb{K}$ such that $\mathbb{K} = \mathbb{F}(\alpha).$
\end{defn}

\begin{defn}
    Let $\mathbb{K}/\mathbb{F}$ be a field extension and let $\alpha \in \mathbb{K}.$ $\alpha$ is said to be \deff{algebraic over $\mathbb{F}$} if there exists a non-zero polynomial $f(x) \in \mathbb{F}[x]$ such that $f(\alpha) = 0.$

    $\alpha$ is said to be \deff{transcendental over $\mathbb{F}$} if it is not algebraic over $\mathbb{F}.$

    If every element of $\mathbb{K}$ is algebraic over $\mathbb{F},$ then $\mathbb{K}/\mathbb{F}$ is called an \deff{algebraic extension}.
\end{defn}

\begin{ex}
    Note that every element of $\mathbb{F}$ is algebraic over $\mathbb{F}.$
\end{ex}

Here's a simple proposition that we leave as an easy exercise.

\begin{prop} \label{prop:decompalgisalg}
    Let $\mathbb{F} \subset \mathbb{E} \subset \mathbb{K}$ be fields and $\alpha \in \mathbb{K}.$ \\
    If $\alpha$ is algebraic over $\mathbb{F},$ then $\alpha$ is algebraic over $\mathbb{E}.$ \\
    If $\mathbb{K}/\mathbb{F}$ is algebraic, then so are $\mathbb{K}/\mathbb{E}$ and $\mathbb{E}/\mathbb{F}.$
\end{prop}

\begin{restatable}[]{prop}{finextisalg}
\label{prop:finextisalg}
    Every finite extension is an algebraic extension. \hfill\hyperref[prop:finextisalg2]{\downsym}
\end{restatable}


\begin{ex}
    Consider the extensions $\mathbb{Q} \subset \mathbb{R} \subset \mathbb{C}$ and $\pi\iota \in \mathbb{C}.$

    It is known that $\pi \in \mathbb{R}$ is transcendental over $\mathbb{Q}.$ An easy consequence of this is that $\pi\iota \in \mathbb{C}$ is also transcendental over $\mathbb{Q}.$ However, $\pi\iota$ is algebraic over $\mathbb{R}$ since it satisfies $x^2 + \pi^2 \in \mathbb{R}[x] \setminus \{0\}.$

    Thus, the property of being algebraic/transcendental depends on the base field. In particular, $\mathbb{C}/\mathbb{Q}$ is not an algebraic extension. However, in view of the earlier proposition, $\mathbb{C}/\mathbb{R}$ is.
\end{ex}

\begin{ex}
    Let $\mathbb{K}$ be a finite field and $\mathbb{F}$ be its prime subfield. Then, $\mathbb{K}$ is a finite dimensional $\mathbb{F}$-vector space and thus, $\mathbb{K}/\mathbb{F}$ is an algebraic extension.
\end{ex}

\begin{rem}
    The converse of the proposition is not true. We shall see later that
    \begin{equation*} 
        \mathbb{A} \vcentcolon= \{\alpha \in \mathbb{C} : \alpha \text{ is algebraic over }\mathbb{Q}\}
    \end{equation*}
    is a subfield of $\mathbb{C}$ such that $\dim_{\mathbb{Q}}(\mathbb{A}) = \infty.$ However, $\mathbb{A}/\mathbb{Q}$ is clearly algebraic, by construction.
\end{rem}

\begin{restatable}[]{prop}{uniquemonicirred}
\label{prop:uniquemonicirred}
    Let $\mathbb{K}/\mathbb{F}$ be a field extension and $\alpha \in \mathbb{K}$ be algebraic over $\mathbb{F}.$ Then, the following are true. 
    \begin{enumerate}
        \item There exists a unique monic irreducible polynomial $f(x) \in \mathbb{F}[x]$ such that $f(\alpha) = 0.$ 
        \item $f(x)$ generates the kernel of the map $\mathbb{F}[x] \to \mathbb{F}[\alpha] \subset \mathbb{K}$ given by $p(x) \mapsto p(\alpha).$
        \item If $g(x) \in \mathbb{F}[x]$ is such that $g(\alpha) = 0,$ then $f(x) \mid g(x).$ 
        \item In particular, $f(x)$ has the least positive degree among all polynomials in $\mathbb{F}[x]$ satisfied by $\alpha.$ \hfill\hyperref[prop:uniquemonicirred2]{\downsym}
    \end{enumerate}

\end{restatable}

Of course, ``irreducible'' above means ``irreducible in $\mathbb{F}[x].$''

\begin{defn}%[]
    Given a field extension $\mathbb{K}/\mathbb{F}$ and $\alpha \in \mathbb{K}$ with is algebraic over $\mathbb{F},$ the irreducible monic polynomial $f(x) \in \mathbb{F}[x]$ having $\alpha$ as a root is called the \deff{irreducible monic polynomial of $\alpha$ over $\mathbb{F}.$} It is denoted by $\irr(\alpha, \mathbb{F}).$

    The degree of $\irr(\alpha, \mathbb{F})$ is called the \deff{degree of $\alpha$ over $\mathbb{F}$} and is denoted by $\deg_{\mathbb{F}}\alpha.$
\end{defn}

\begin{ex}
    \phantom{hi}
    \begin{enumerate}
        \item Let $\alpha \in \mathbb{C}$ be a square root of $\iota.$ Then, $\alpha$ satisfies $f(x) \vcentcolon= x^4 + 1.$ Show that $f(x) = \irr(\alpha, \mathbb{Q}).$

        However, $\irr(\alpha, \mathbb{Q}(\iota)) = x^2 - \iota.$ Thus, degree also depends on the base field.
        \item Let $p$ be a prime and $\zeta_p \vcentcolon= \exp\left(\dfrac{2\pi\iota}{p}\right) \in \mathbb{C}.$ Then, $\zeta_p^p = 1.$ Note that $x^p - 1 = (x - 1)\Phi_p(x)$ where
        \begin{equation*} 
            \Phi_p(x) \vcentcolon= x^{p - 1} + \cdots + 1.
        \end{equation*}
        Then, $\Phi_p(\zeta_p) = 0.$ Use Eisenstein's criterion on $\Phi_p(x + 1)$ to conclude that $\Phi_p(x)$ is irreducible in $\mathbb{Q}[x]$ and hence, $\Phi_p(x) = \irr(\zeta_p, \mathbb{Q}).$
    \end{enumerate}
\end{ex}

\begin{restatable}[]{prop}{adjoiningalg}
\label{prop:adjoiningalg}
    Let $\mathbb{K}/\mathbb{F}$ be a field extension and $\alpha \in \mathbb{K}$ be algebraic over $\mathbb{F}.$ Let $f(x) \vcentcolon= \irr(\alpha, \mathbb{F})$ and $n \vcentcolon= \deg f(x).$ Then,
    \begin{enumerate}
         \item $\mathbb{F}[\alpha] = \mathbb{F}(\alpha) \cong \mathbb{F}[x]/\langle f(x)\rangle.$
         \item $\dim_{\mathbb{F}}(\mathbb{F}(\alpha)) = n$ and $\{1, \alpha, \ldots, \alpha^{n - 1}\}$ is an $\mathbb{F}$-basis of $\mathbb{F}(\alpha).$ \hfill\hyperref[prop:adjoiningalg2]{\downsym}
     \end{enumerate} 
\end{restatable}

\begin{cor} \label{cor:adjoinalgisfin}
    Let $\mathbb{K}/\mathbb{F}$ be a field extension and $\alpha \in \mathbb{K}$ be algebraic over $\mathbb{F}.$ Then, $\mathbb{F}(\alpha)/\mathbb{F}$ is a finite and hence, algebraic extension, by \Cref{prop:finextisalg}.
\end{cor}

\begin{restatable}[]{prop}{isocarryingalphtobet}
\label{prop:isocarryingalphtobet}
    Let $\alpha, \beta \in \mathbb{K} \supset \mathbb{F}$ be algebraic over $\mathbb{F}.$ Then, there exists an $\mathbb{F}$-isomorphism $\psi : \mathbb{F}(\alpha) \to \mathbb{F}(\beta)$ such that $\psi(\alpha) = \beta$ iff $\irr(\alpha, \mathbb{F}) = \irr(\beta, \mathbb{F}).$ \hfill\hyperref[prop:isocarryingalphtobet2]{\downsym}
\end{restatable}

\begin{defn}%[]
    The extension $\mathbb{K}/\mathbb{F}$ is said to be \deff{a quadratic extension} if $[\mathbb{K} : \mathbb{F}] = 2.$
\end{defn}

\begin{rem}
    Note that if $\mathbb{K}/\mathbb{F}$ is a quadratic extension and $\alpha \in \mathbb{K}\setminus\mathbb{F},$ then $[\mathbb{F}(\alpha) : \mathbb{F}] > 1$ and hence, $[\mathbb{F}(\alpha) : \mathbb{F}] = 2.$ Thus, $\mathbb{F}(\alpha) = \mathbb{K}.$

    That is, all quadratic extensions are simple.
\end{rem}

\begin{restatable}[Tower law]{thm}{towerlaw}
\label{thm:towerlaw}
    Let $\mathbb{F} \subset \mathbb{E} \subset \mathbb{K}$ be a tower of fields. Then,
    \begin{equation*} 
        [\mathbb{K} : \mathbb{F}] = [\mathbb{K} : \mathbb{E}][\mathbb{E} : \mathbb{F}].
    \end{equation*}
    In particular, the left side is $\infty$ iff the right side is. \hfill\hyperref[thm:towerlaw2]{\downsym}
\end{restatable}

\begin{cor}
    Let $\mathbb{K}/\mathbb{F}$ be a finite extension and $\alpha \in \mathbb{K}.$ Then, $\deg_{\mathbb{F}} \alpha \mid [\mathbb{K} : \mathbb{F}].$
\end{cor}
\begin{proof}
    Consider the tower $\mathbb{F} \subset \mathbb{F}(\alpha) \subset \mathbb{K}.$
\end{proof}

\begin{restatable}[]{prop}{adjoinalgsfinext}
\label{prop:adjoinalgsfinext}
    Let $\mathbb{K}/\mathbb{F}$ be a field extension and let $\alpha_1, \ldots, \alpha_n \in \mathbb{K}$ be algebraic over $\mathbb{F}.$ Then, $\mathbb{F}(\alpha_1, \ldots, \alpha_n)$ is a finite (and hence, algebraic) extension of $\mathbb{F}.$ \hfill\hyperref[prop:adjoinalgsfinext2]{\downsym}
\end{restatable}

\begin{restatable}[]{cor}{compalgisalg}
\label{cor:compalgisalg}
   Let $\mathbb{F} \subset \mathbb{E}$ and $\mathbb{E} \subset \mathbb{K}$ be algebraic extensions. Then, $\mathbb{F} \subset \mathbb{K}$ is an algebraic extension.  \hfill\hyperref[cor:compalgisalg2]{\downsym}
\end{restatable}

\begin{restatable}[]{cor}{algclosureisfield}
\label{cor:algclosureisfield}
    Let $\mathbb{K}/\mathbb{F}$ be a field extension. Then,
    \begin{equation*} 
        \mathbb{A} \vcentcolon= \{\alpha \in \mathbb{K} : \alpha \text{ is algebraic over }\mathbb{F}\}
    \end{equation*} is a subfield of $\mathbb{K}$ containing $\mathbb{F}.$ \\
    Moreover, $\mathbb{A}/\mathbb{F}$ is an algebraic extension. \hfill\hyperref[cor:algclosureisfield2]{\downsym}
\end{restatable}

\section{Compositum of fields}

\begin{defn}%[]
    Let $\mathbb{E}_1, \mathbb{E}_2 \subset \mathbb{K}$ be fields. The \deff{compositum} of $\mathbb{E}_1$ and $\mathbb{E}_2$ is the smallest subfield of $\mathbb{K}$ containing $\mathbb{E}_1$ and $\mathbb{E}_2.$ It is denoted by $\mathbb{E}_1\mathbb{E}_2.$
\end{defn}

\begin{ex}
    Suppose $\mathbb{F} \subset \mathbb{E}_1, \mathbb{E}_2 \subset \mathbb{K}$ and $\mathbb{E}_1 = \mathbb{F}(\alpha_1, \ldots, \alpha_n).$ Then,
    \begin{equation*} 
        \mathbb{E}_1\mathbb{E}_2 = \mathbb{E}_2(\alpha_1, \ldots, \alpha_n).
    \end{equation*}
\end{ex}

\begin{ex} \label{ex:compositecyclo}
    Let $m$ and $n$ be coprime positive integers. Consider the subfields $\mathbb{F} \vcentcolon= \mathbb{Q}(\zeta_m)$ and $\mathbb{E} \vcentcolon= \mathbb{Q}(\zeta_n)$ of $\mathbb{C}.$ Then,
    \begin{equation*} 
        \mathbb{E}\mathbb{F} = \mathbb{Q}(\zeta_{mn}).
    \end{equation*}
    $\subset$ is clear since $\zeta_n = \zeta_{mn}^m$ and similarly, $\zeta_m = \zeta_{mn}^n.$

    On the other hand, since $\gcd(m, n) = 1,$ there exist integers $a, b \in \mathbb{Z}$ such that $am + bn = 1.$ Thus,
    \begin{equation*} 
        \frac{a}{n} + \frac{b}{m} = \frac{1}{mn}
    \end{equation*}
    and hence
    \begin{equation*} 
        \zeta_{mn} = \zeta_n^a\zeta_m^b.
    \end{equation*}
\end{ex}

\begin{restatable}[]{prop}{intdomfinextfield}
\label{prop:intdomfinextfield}
    Let $\mathbb{F}$ be a field which is a subring of an integral domain $R.$ Suppose $R$ is finite dimensional as an $\mathbb{F}$ vector space. Then, $R$ is a field. \hfill\hyperref[prop:intdomfinextfield2]{\downsym}
\end{restatable}

\begin{restatable}[]{prop}{descofcompositum}
\label{prop:descofcompositum}
    Let $\mathbb{F} \subset \mathbb{E}_1, \mathbb{E}_2 \subset \mathbb{K}$ be fields. Consider
    \begin{equation*} 
        \mathbb{L} = \left\{\sum_{i = 1}^{n} \alpha_i\beta_i : n \in \mathbb{N}, \alpha_i \in \mathbb{E}_1, \beta_i \in \mathbb{E}_2\right\}.
    \end{equation*}
    That is, let $\mathbb{L}$ be the set of all finite sums of products of elements of $\mathbb{E}_1$ and $\mathbb{E}_2.$

    Suppose $d \vcentcolon= [\mathbb{E}_1 : \mathbb{F}][\mathbb{E}_2 : \mathbb{F}] < \infty.$ \\
    Then $\mathbb{L} = \mathbb{E}_1\mathbb{E}_2$ and $[\mathbb{L} : \mathbb{F}] \le d.$ 

     If $[\mathbb{E}_1 : \mathbb{F}]$ and $[\mathbb{E}_2 : \mathbb{F}]$ are coprime, then equality holds. \hfill\hyperref[prop:descofcompositum2]{\downsym}
\end{restatable}

Diagrammatically, this can be depicted as
\begin{center}
    \begin{tikzcd}
                                       & \mathbb{K} \arrow[d, no head]                                                       &              \\
                                       & \mathbb{E}_1\mathbb{E}_2 \arrow[ld, "\le m"', no head] \arrow[rd, "\le n", no head] &              \\
\mathbb{E}_1 \arrow[rd, "n"', no head] &                                                                                     & \mathbb{E}_2 \\
                                       & \mathbb{F} \arrow[ru, "m"', no head]                                                &             
    \end{tikzcd}
\end{center}

\section{Splitting Fields}

\begin{defn}%[]
    Let $\mathbb{F}$ be a field and $f(x) \in \mathbb{F}[x]$ be a non-constant monic polynomial of degree $n$ with leading coefficient $a \in \mathbb{F}^\times.$ A field $\mathbb{K} \supset \mathbb{F}$ is called a \deff{splitting field of $f(x)$ over $\mathbb{F}$} if there exist $r_1, \ldots, r_n \in \mathbb{K}$ so that $f(x) = a(x - r_1)\cdots(x - r_n)$ and $\mathbb{K} = \mathbb{F}(r_1, \ldots, r_n).$
\end{defn}

Note that $r_1, \ldots, r_n$ above need not be distinct.

\begin{ex}
    Consider $\mathbb{F} = \mathbb{Q},$ $f(x) = x^2 + 1 \in \mathbb{Q}[x]$ and $\mathbb{K} = \mathbb{C}.$ While $f(x)$ does factor linearly over $\mathbb{C},$ $\mathbb{C}$ is \textbf{not} a splitting field of $f(x)$ over $\mathbb{Q}$ since $\mathbb{C} \neq \mathbb{Q}(\iota, -\iota).$

    On the other hand, $\mathbb{C}$ \emph{is} a splitting field of $f(x) \in \mathbb{R}[x]$ over $\mathbb{R}.$
\end{ex}

\begin{cor}
    Let $f(x) \in \mathbb{F}[x]$ be non-constant and $\mathbb{K}$ be a splitting field of $f(x)$ over $\mathbb{F}.$ Then, $\mathbb{K}/\mathbb{F}$ is an algebraic extension.
\end{cor}
\begin{proof} 
    Follows from \Cref{prop:adjoinalgsfinext}.
\end{proof}

\begin{restatable}[]{thm}{rootcanbeadjoined}
\label{thm:rootcanbeadjoined}
    Let $\mathbb{F}$ be a field and $f(x) \in \mathbb{F}[x]$ be non-constant. Then, there exists a field $\mathbb{K} \supset \mathbb{F}$ such that $f(x)$ has a root in $\mathbb{K}.$ \hfill\hyperref[thm:rootcanbeadjoined2]{\downsym}
\end{restatable}

\begin{restatable}[Existence of Splitting Field]{thm}{splitfieldexists}
\label{thm:splitfieldexists}
    Let $\mathbb{F}$ be a field. Any polynomial $f(x) \in \mathbb{F}[x]$ of positive degree has a splitting field. \hfill\hyperref[thm:splitfieldexists2]{\downsym}
\end{restatable}
\chapter{Symmetric Polynomials}

\begin{defn}%[]
    Given a ring $R,$ consider the polynomial ring $S = R[u_1, \ldots, u_n].$ Let $S_n$ denote the symmetric group. Then, any $\tau \in S_n$ induces an automorphism $g_{\tau} : S \to S$ by
    \begin{equation*} 
        g_{\tau}(f(u_1, \ldots, u_n)) = f(u_{\tau(1)}, \ldots, u_{\tau(n)}).
    \end{equation*}
\end{defn}

\begin{ex}
    Consider $R = \mathbb{Z}$ and $n = 3.$ Suppose $\tau = (12).$ Consider the polynomial $f = u_1 + u_2^2 + u_3^3.$ Then, $g_{\tau}(f) = u_2 + u_1^2 + u_3^3.$
\end{ex}

\begin{defn}%[]
    A polynomial $f \in R[u_1, \ldots, u_n]$ is said to be a \deff{symmetric polynomial (in $n$ variables)} if 
    \begin{equation*} 
        f(u_1, \ldots, u_n) = f(u_{\tau(1)}, \ldots, u_{\tau(n)})
    \end{equation*} 
    for all $\tau \in S_n.$
\end{defn}

\begin{defn}%[]
    Let $S = R[u_1, \ldots, u_n].$ Consider $f(T) \in S[T]$ given by
    \begin{equation*} 
        f(T) = (T - u_1) \cdots (T - u_n).
    \end{equation*}
    Write $f(T)$ as 
    \begin{equation*} 
        f(T) = T^n - \sigma_1 T^{n - 1} + \cdots + (-1)^n \sigma_n,
    \end{equation*}
    for $\sigma_1, \ldots, \sigma_n \in S.$

    Then, $\sigma_1, \ldots, \sigma_n$ are symmetric polynomials, which are called the \deff{elementary symmetric polynomials (in $n$ variables)}.
\end{defn}

\begin{rem}
    Note that one can explicitly write down the elementary symmetric polynomials. We have
    \begin{align*} 
        \sigma_1 &= \sum_{i = 1}^{n} u_i,\\
        \sigma_2 &= \sum_{1 \le i_1 < i_1 \le n} u_{i_1}u_{i_2},\\
        & \vdots \\
        \sigma_n &= u_1 \cdots u_n.
    \end{align*}
    It is now easy to verify that these are all indeed symmetric polynomials.
\end{rem}

\section{Fundamental theorem of Symmetric Polynomials}

\begin{defn}%[]
    Given an elementary symmetric polynomial $\sigma_i \in R[u_1, \ldots, u_n]$ in $n$ variables (for $n \ge 1$), we define the elementary symmetric polynomial $\sigma_i^0$ in $(n - 1)$ variables as
    \begin{equation*} 
        \sigma_i^0 \vcentcolon= \sigma_1(u_1, \ldots, u_{n - 1}, 0).
    \end{equation*}
\end{defn}

\begin{ex}
    Consider $n = 3.$ Then, $\sigma_2 = u_1u_2 + u_1u_3 + u_2u_3.$ Then, $\sigma_2^0 = u_1u_2.$ This is the second symmetric polynomial in two variables. 

    In fact, any elementary symmetric polynomial in $n - 1$ variables is of the form $\sigma_i^0$ for the corresponding elementary symmetric polynomial $\sigma_i$ in $n$ variables.
\end{ex}

\begin{restatable}[Fundamental Theorem of Symmetric Polynomials]{thm}{FTSP}
\label{thm:FTSP}
    Let $R$ be a commutative ring. Then, every symmetric polynomial in $S \vcentcolon= R[u_1, \ldots, u_n]$ is a polynomial in the elementary symmetric polynomials in a unique way.

    More precisely, if $f(u_1, \ldots, u_n)$ is symmetric, then there exists a unique $g \in R[x_1, \ldots, x_n]$ such that
    \begin{equation*} 
        g(\sigma_1, \ldots, \sigma_n) = f(u_1, \ldots, u_n).
    \end{equation*}
    (The above is equality in $S.$) \hfill\hyperref[thm:FTSP2]{\downsym}
\end{restatable}

\section{Newton's identities for power sum symmetric polynomials}

\begin{defn}%[]
    Let $S = R[u_1, \ldots, u_n].$ For $k \ge 1,$ define
    \begin{equation*} 
        w_k = u_1^k + \cdots + u_n^k.
    \end{equation*}
\end{defn}

\begin{restatable}[Newton's Identities]{thm}{powersumformulae}
\label{thm:powersumformulae}
    We have
    \begin{equation} \label{eq:newident}
        w_k = \begin{cases}
            \sigma_1 w_{k - 1} - \sigma_2w_{k - 2} + \cdots + (-1)^k \sigma_{k - 1}w_1 + (-1)^{k + 1}\sigma_k k & k \le n,\\
            \sigma_1 w_{k - 1} - \sigma_2w_{k - 2} + \cdots + (-1)^{n + 1} \sigma_{n}w_{k - n} & k > n.\\
                
        \end{cases}
    \end{equation} 
    \hfill\hyperref[thm:powersumformulae2]{\downsym}
\end{restatable}

Note that the last term is $(-1)^{k + 1} \sigma_k {\color{red}k}.$ One might have expected that it would be an `$n$' instead but that is not the case.

\section{Discriminant of a polynomial}

\begin{defn}%[]
    Let $f(x) \in \mathbb{F}[x]$ be a non-constant monic polynomial and $\mathbb{K}$ be a splitting field of $f(x)$ over $\mathbb{F}.$ Write
    \begin{equation*} 
        f(x) = (x - r_1) \cdots (x - r_n)
    \end{equation*}
    for $r_1, \ldots, r_n \in \mathbb{K}.$ Then, the \deff{discriminant of $f(x)$} is defined as
    \begin{equation*} 
        \disc_{\mathbb{K}}(f(x)) \vcentcolon= \prod_{1 \le i < j \le n} (r_i - r_j)^2.
    \end{equation*}
\end{defn}

\begin{rem} \label{rem:discrepeatedroots}
    Note that $\disc_{\mathbb{K}}(f(x)) = 0 \iff f(x)$ has repeated roots in $\mathbb{K}.$

    Moreover, by construction, $\disc_{\mathbb{K}}(f(x))$ has a square root in $\mathbb{K},$ namely
    \begin{equation*} 
        \prod_{1 \le i < j \le n} (r_i - r_j) \in \mathbb{K}.
    \end{equation*}
\end{rem}

\begin{restatable}[]{prop}{independencediscriminant}
\label{prop:independencediscriminant}
    Let $f(x) \in \mathbb{F}[x]$ be non-constant and monic. Suppose $\mathbb{K}$ and $\mathbb{K}'$ are two splitting fields of $f(x)$ over $\mathbb{F}.$ Then,
    \begin{equation*} 
        \disc_{\mathbb{K}}(f(x)) = \disc_{\mathbb{K}'}(f(x)) \in \mathbb{F}.
    \end{equation*} 

    In other words, the discriminant takes values in $\mathbb{F}$ and is independent of the splitting field chosen. \hfill\hyperref[prop:independencediscriminant2]{\downsym}
\end{restatable}

In view of the (proof of the) above proposition, we have the following alternate definition of discriminant.

\begin{defn}%[]
    Let $f(x) = x^n - \sigma_1x^{n - 1} + \cdots + (-1)^n\sigma_n \in \mathbb{F}[x]$ be a monic polynomial. Define $w_k$ for $k = 1, \ldots, 2n - 2$ as in \Cref{eq:newident}. Then, 
    \begin{equation*} 
        \disc(f(x)) \vcentcolon= \det \begin{bmatrix}
            n & w_1 & \cdots & w_{n - 1}\\
            w_1 & w_2 & \cdots & w_n\\
            w_2 & w_3 & \cdots & w_{n + 1}\\
            \vdots & \vdots & \ddots & \vdots \\
            w_{n - 1} & w_n & \cdots & w_{2n - 2}\\
        \end{bmatrix}.
    \end{equation*}
\end{defn}

\begin{restatable}[Discriminant in terms of derivative]{prop}{discderivative}
\label{prop:discderivative}
    Suppose $f(x) = \prod_{i = 1}^{n}(x - r_i).$ Then, $\disc(f(x)) = (-1)^{\binom{n}{2}}\prod_{i = 1}^{n}f'(r_i).$ \hfill\hyperref[prop:discderivative2]{\downsym}
\end{restatable}

\begin{ex}[Discriminant of a quadratic]
    Let $x^2 + bx + c \in \mathbb{F}[x]$ be a quadratic. We have $\sigma_1 = -b,$ $\sigma_2 = c.$ Thus, we have
    \begin{align*} 
        w_1 &= -b,\\
        w_2 &= b^2 - 2c.
    \end{align*}
    Thus,
    \begin{equation*} 
        \disc(f(x)) = \det\begin{bmatrix}
            2 & -b\\
            -b & b^2 - 2c
        \end{bmatrix} = b^2 - 4c.
    \end{equation*}
    This is the usual discriminant of a quadratic.
\end{ex}

\begin{ex}[Discriminant of a special cubic] \label{ex:disccubic}
    Let $x^3 + px + q \in \mathbb{F}[x]$ be a cubic. Here, $\sigma_1 = 0,$ $\sigma_2 = p,$ and $\sigma_3 = -q.$ Then, Newton's identities become
    \begin{align*} 
        w_1 &= 0,\\
        w_2 &= -2p,\\
        w_3 &= -3q,\\
        w_4 &= 2p^2.
    \end{align*}
    Thus, $\disc(f(x)) = -4p^3 - 27q^2.$
\end{ex}

\section{The Fundamental Theorem of Algebra}

Recall the following facts.

\begin{restatable}[]{lem}{FTAprelim}
\label{lem:FTAprelim}
    \phantom{hi}
    \begin{enumerate}
        \item Every real polynomial of odd degree has a real root.
        \item Every complex number has a square root. Thus, every complex quadratic polynomial has a root in $\mathbb{C}.$ \hfill\hyperref[lem:FTAprelim2]{\downsym}
    \end{enumerate} 
\end{restatable}

\begin{restatable}[Fundamental Theorem of Algebra]{thm}{FTA}
\label{thm:FTA}
    Every non-constant complex polynomial has a root in $\mathbb{C}.$ \hfill\hyperref[thm:FTA2]{\downsym}
\end{restatable}
\chapter{Algebraic Closure of a Field}

\section{Existence}

\begin{defn}%[]
    A field $\mathbb{K}$ is called an \deff{algebraically closed field} if every non-constant polynomial $f(x) \in \mathbb{K}[x]$ has a root in $\mathbb{K}.$
\end{defn}

\begin{defn}%[]
    Let $\mathbb{K}/\mathbb{F}$ be a field extension. We say that $\mathbb{K}$ is an \deff{algebraic closure of $\mathbb{F}$} if $\mathbb{K}$ is algebraically closed and $\mathbb{K}/\mathbb{F}$ is an algebraic extension.
\end{defn}

We have the following simple proposition.
\begin{prop}
    \phantom{hi}
    \begin{enumerate}
        \item $\mathbb{K}$ is algebraically closed iff every non-constant polynomials factors as a product of linear factors.
        \item $\mathbb{C}$ is algebraically closed.
        \item If $\mathbb{K}$ is algebraically closed and $\mathbb{L}/\mathbb{K}$ is an algebraic extension, then $\mathbb{L} = \mathbb{K}.$
    \end{enumerate}
\end{prop}

\begin{restatable}[]{prop}{alglcosureinalgclosedisclosed}
\label{prop:alglcosureinalgclosedisclosed}
    Let $\mathbb{F} \subset \mathbb{K}$ be an extension where $\mathbb{K}$ is algebraically closed. Define,
    \begin{equation*} 
        \mathbb{A} \vcentcolon= \{\alpha \in \mathbb{K} : \alpha \text{ is algebraic over }\mathbb{F}\}.
    \end{equation*}
    Then, $\mathbb{A}$ is an algebraic closure of $\mathbb{F}.$ \hfill\hyperref[prop:alglcosureinalgclosedisclosed2]{\downsym}
\end{restatable}

\begin{restatable}[]{lem}{unionoffields}
\label{lem:unionoffields}
    Let $\{\mathbb{F}_i\}_{i \ge 1}$ be a sequence of fields as
    \begin{equation*} 
        \mathbb{F}_1 \subset \mathbb{F}_2 \subset \cdots.
    \end{equation*}
    Then, $\mathbb{F} \vcentcolon= \bigcup_{i \ge 1}\mathbb{F}_i$ is a field with the following operations:
    Given $a, b \in \mathbb{F},$ there exist smallest $i, j \in \mathbb{N}$ with $a \in \mathbb{F}_i$ and $b \in \mathbb{F}_j.$ Then, $a, b \in \mathbb{F}_{i + j}.$ Define $a + b$ and $ab$ to be the corresponding elements from $\mathbb{F}_{i + j}.$

    Moreover, each $\mathbb{F}_i$ is a subfield of $\mathbb{F}.$ \hfill\hyperref[lem:unionoffields2]{\downsym}
\end{restatable}

Note that the ``smallest'' above is just to ensure that the operations are well-defined. Since $\mathbb{F}_i \subset \mathbb{F}_j$ (note that we always use this to mean ``is a subfield of'') for $i \le j,$ we can actually pick any $i$ and $j.$

\begin{restatable}[Existence of Algebraic Closed Extension]{thm}{algclosedext}
\label{thm:algclosedext}
    Let $\mathbb{F}$ be a field. Then, there exists an algebraically closed field containing $\mathbb{F}.$ \hfill\hyperref[thm:algclosedext2]{\downsym}
\end{restatable}
The proof we have given is due to Artin.

\begin{restatable}[Existence of Algebraic Closure]{cor}{algclosure}
\label{cor:algclosure}
    Every field $\mathbb{F}$ has an algebraic closure. \hfill\hyperref[cor:algclosure2]{\downsym}
\end{restatable}

\section{Uniqueness}

\begin{restatable}[]{prop}{rootsandextensions}
\label{prop:rootsandextensions}
    Let $\sigma : \mathbb{F} \to \mathbb{L}$ be an embedding of fields where $\mathbb{L}$ is \underline{algebraically closed}. Let $\alpha \in \mathbb{K} \supset \mathbb{F}$ be algebraic over $\mathbb{F}$ and $p(x) = \irr(\alpha, \mathbb{F}).$ \\
    Write $p(x) = \sum a_i x^i$ and define $p^{\sigma}(x) \vcentcolon= \sum \sigma(a_i) x^i.$ Then, $\tau \mapsto \tau(\alpha)$ is a bijection between the sets
    \begin{equation*} 
        \{\tau : \mathbb{F}(\alpha) \to \mathbb{L} \mid \tau \text{ is an embedding and }\tau|_{\mathbb{F}} = \sigma\} \leftrightarrow \{\beta \in \mathbb{L} \mid p^{\sigma}(\beta) = 0\}.
    \end{equation*} \hfill\hyperref[prop:rootsandextensions2]{\downsym}
\end{restatable}

\begin{rem}
    The above proposition says that the number of ways to extend from $\mathbb{F}$ to $\mathbb{F}(\alpha)$ is precisely the number of roots of that $p(x)$ has in $\mathbb{L}.$ (Not exactly, we need to apply $\sigma$ to the coefficients. This is essentially saying that we consider $\mathbb{F}$ as a subfield under $\mathbb{L}.$) In particular, this set is non-empty since $\mathbb{L}$ is algebraically closed. \\
    Note that this number need not be $\deg(f(x)).$ We shall see in the next chapter that a polynomial may be irreducible but still have repeated roots in its splitting field.
\end{rem}

\begin{restatable}[]{thm}{extendtoalgextension}
\label{thm:extendtoalgextension}
    Let $\sigma : \mathbb{F} \to \mathbb{L}$ be an embedding where $\mathbb{L}$ is algebraically closed. Let $\mathbb{K}/\mathbb{F}$ be an algebraic extension. Then, there exists an embedding $\tau : \mathbb{K} \to \mathbb{L}$ extending $\sigma.$ \\
    Moreover, if $\mathbb{K}$ is an algebraic closure of $\mathbb{F}$ and $\mathbb{L}$ of $\sigma(\mathbb{K}),$ then $\tau$ is an isomorphism extending $\sigma.$ \hfill\hyperref[thm:extendtoalgextension2]{\downsym}
\end{restatable}

\begin{cor}[Isomorphism of algebraic closures]
    If $\mathbb{K}_1$ and $\mathbb{K}_2$ are two algebraic closures of $\mathbb{F},$ then they are $\mathbb{F}$-isomorphic.
\end{cor}
\begin{proof} 
    Apply previous proposition to the inclusion $i : \mathbb{F} \hookrightarrow \mathbb{E}_2$ to extend it to an $\mathbb{F}$-isomorphism $\tau : \mathbb{E}_1 \to \mathbb{E}_2.$
\end{proof}

\begin{defn}%[]
    Given a field $\mathbb{F},$ we use $\overline{\mathbb{F}}$ to denote an algebraic closure of $\mathbb{F}.$ 
\end{defn}

\begin{restatable}[Isomorphism of splitting fields]{thm}{isosplitting}
\label{thm:isosplitting}
    Let $\mathbb{E}$ and $\mathbb{E}'$ be two splitting fields of a non-constant polynomial $f(x) \in \mathbb{F}[x]$ over $\mathbb{F}.$ Then, they are $\mathbb{F}$-isomorphic. \hfill\hyperref[thm:isosplitting2]{\downsym}
\end{restatable}
\chapter{Separable extensions}

\section{Derivatives}

\begin{defn}%[]
    \label{defn:formalderivative}
    Let $\mathbb{F}$ be a field. Define the $\mathbb{F}$-linear map $\D_{\mathbb{F}} : \mathbb{F}[x] \to \mathbb{F}[x]$ by
    \begin{equation*} 
        \D_{\mathbb{F}}\left(\sum_{i = 0}^{n} a_ix^i\right) = \sum_{i = 1}^{n}ia_i x^{i - 1}.
    \end{equation*}
    Given $f(x) \in \mathbb{F}[x],$ we call $\D_{\mathbb{F}}(f(x))$ the \deff{(formal) derivative} of $f(x)$ and also denote it by $f'(x).$
\end{defn}

\begin{rem}
    Note that the above definition requires no notion of limits. For the case of $\mathbb{F} = \mathbb{R}$ or $\mathbb{C},$ it coincides with the usual definition if we identify a polynomial with the function it represents. We shall not require this, however.
\end{rem}

We have the follow easy-to-check proposition.
\begin{prop}
    Let $f(x), g(x) \in \mathbb{F}[x]$ and $a \in \mathbb{F}$ be arbitrary. Then,
    \begin{enumerate}
        \item $(f \pm ag)'(x) = f'(x) \pm ag'(x),$
        \item $(fg)'(x) = f'(x)g(x) + f(x)g'(x).$
    \end{enumerate}
\end{prop}
The first point is just verifying that $\D_{\mathbb{F}}$ is indeed $\mathbb{F}$-linear.

\begin{prop}
    Let $\mathbb{F} \subset \mathbb{E}$ be a field extension. Then, $\D_{\mathbb{E}}|_{\mathbb{F}} = \D_{\mathbb{F}}.$ Thus, the notation $f'(x)$ is unambiguous.
\end{prop}

\begin{defn}%[]
    Let $f(x) \in \mathbb{F}[x]$ be a non-constant monic polynomial. Let $\mathbb{E}$ be a splitting field of $f(x)$ over $\mathbb{F}.$ In $\mathbb{E}[x],$ factorise $f(x)$ uniquely as
    \begin{equation*} 
        f(x) = (x - r_1)^{e_1} \cdots (x - r_g)^{e_g},
    \end{equation*}
    where $r_1, \ldots, r_g \in \mathbb{E}$ are distinct and each $e_i \in \mathbb{N}.$ 

    The numbers $e_1, \ldots, e_g$ are called the \deff{multiplicities} of the roots $r_1, \ldots, r_g.$ \\
    If $e_i = 1$ for some $i,$ then $r_i$ is called a \deff{simple root} and a \deff{repeated root} otherwise.

    If each $e_i = 1,$ then $f(x)$ is said to be a \deff{separable polynomial}.

    If $f$ is not monic, we have the same definitions upon division by the leading coefficient.
\end{defn}

\begin{rem}
    Note that the definition of ``separable polynomial'' is ad hoc since the separability presumably depends on the splitting field. However, in view of \Cref{rem:discrepeatedroots}, we see that separability depends only on $\disc(f(x)),$ which we had seen to be independent of the splitting field. (\Cref{prop:independencediscriminant}.) \\
    The next proposition shows something even stronger.

    Also, note that one might think that an irreducible polynomial is always separable. We will see an example of how that is not true, in general. (\Cref{ex:FXnotperfect}.) Over fields of characteristic $0,$ however, it is true. We shall prove that as well. (\Cref{prop:irredsepderiv}.)
\end{rem}

\begin{restatable}[]{prop}{multindepsplitting}
\label{prop:multindepsplitting}
    The number of roots and their multiplicities are independent of the splitting field chosen for $f(x)$ over $\mathbb{F}.$ \hfill\hyperref[prop:multindepsplitting2]{\downsym}
\end{restatable}

\begin{restatable}[]{prop}{derivcritreproot}
\label{prop:derivcritreproot}
    Let $f(x) \in \mathbb{F}[x]$ be a monic and let $r \in \mathbb{E} \supset \mathbb{F}$ be a root of $f(x).$ \\
    Then, $r$ is a repeated root iff $f'(r) = 0.$ \hfill\hyperref[prop:derivcritreproot2]{\downsym}
\end{restatable}

\begin{restatable}[The Derivative Criterion for Separability]{thm}{derivcritsep}
\label{thm:derivcritsep}
    Let $f(x) \in \mathbb{F}[x]$ be a monic polynomial.
    \begin{enumerate}
         \item If $f'(x) = 0,$ then every root of $f(x)$ is a multiple root.
         \item If $f'(x) \neq 0,$ then $f(x)$ has all roots simple iff $\gcd(f(x), f'(x)) = 1.$ \hfill\hyperref[thm:derivcritsep2]{\downsym}
     \end{enumerate} 
\end{restatable}

\begin{restatable}[]{prop}{irredsepderiv}
\label{prop:irredsepderiv}
    Let $f(x) \in \mathbb{F}[x]$ be irreducible and non-constant.
    \begin{enumerate}
         \item $f(x)$ is separable iff $f'(x) \neq 0.$
         \item If $\chr(\mathbb{F}) = 0,$ then $f(x)$ is separable. 
     \end{enumerate} 
     In other words, irreducible polynomials over fields of characteristic $0$ are separable. \hfill\hyperref[prop:irredsepderiv2]{\downsym}
\end{restatable}

\begin{ex} \label{ex:FXnotperfect}
    Let $p \in \mathbb{N}$ be a prime. Consider the field $\mathbb{F}_p(X)$ and the polynomial $f(T) = T^p - X \in \mathbb{F}_p(X)[T].$ \\
    Then, $f(T)$ is irreducible, by applying Eisenstein at the prime $X.$ However, $f'(T) = 0$ and hence, not separable.

    The above can essentially be attributed to the fact that $X$ has no $p$-th root in $\mathbb{F}_p(X).$ In fact, as we shall see, the existence of $p$-th roots will play an important role.

    It should also be clear that we can replace $\mathbb{F}_p$ with any field of characteristic $p$ in the above.
\end{ex}

\begin{defn}%[]
    Let $\mathbb{F}$ be a field of prime characteristic $p.$ Define
    \begin{equation*} 
        \mathbb{F}^p \vcentcolon= \{\alpha^p \in \mathbb{F} : \alpha \in \mathbb{F}\}.
    \end{equation*}
    That is, $\mathbb{F}^p$ is the set of all $p$-th powers of elements of $\mathbb{F}.$
\end{defn}

\begin{prop}
    $\mathbb{F}^p$ is a subfield of $\mathbb{F}.$
\end{prop}
\begin{proof} 
    Only closure under addition is not so obvious. For this, note that $(x + y)^p = x^p + y^p$ for all $x, y \in \mathbb{F}.$
\end{proof}

\begin{restatable}[]{prop}{xppolyirredorroot}
\label{prop:xppolyirredorroot}
    Let $\mathbb{F}$ be a field with $\chr(\mathbb{F}) = p > 0.$ Then, $x^p - a \in \mathbb{F}[x]$ is either irreducible in $\mathbb{F}[x]$ or $a \in \mathbb{F}^p.$ \hfill\hyperref[prop:xppolyirredorroot2]{\downsym}
\end{restatable}

In other words, either the above polynomial either has a root or is irreducible.

\begin{restatable}[]{prop}{nonseppowerp}
\label{prop:nonseppowerp}
    Let $f(x) \in \mathbb{F}[x]$ be an irreducible polynomial and let $p \vcentcolon= \chr(\mathbb{F}) > 0.$ If $f(x)$ is not separable, then there exists $g(x) \in \mathbb{F}[x]$ such that $f(x) = g(x^p).$ \hfill\hyperref[prop:nonseppowerp2]{\downsym}
\end{restatable}

\section{Perfect fields}
\begin{defn}%[]
    Let $\mathbb{F} \subset \mathbb{K}$ be a field extension. 

    An algebraic element $\alpha \in \mathbb{K}$ over $\mathbb{F}$ is called a \deff{separable element over $\mathbb{F}$} if $\irr(\alpha, \mathbb{F})$ is separable over $\mathbb{F}.$

    We say that $\mathbb{K}/\mathbb{F}$ is a \deff{separable field extension} if every $\alpha \in \mathbb{K}$ is separable (and in particular, algebraic). 

    We say that $\mathbb{F}$ is a \deff{perfect field} if every algebraic extension of $\mathbb{F}$ is separable. Equivalently, every irreducible polynomial in $\mathbb{F}[x]$ is separable.
\end{defn}

\begin{ex}
    \phantom{hi}
    \begin{enumerate}
        \item We had seen (in \Cref{ex:FXnotperfect}) that $\mathbb{F}_p(X)$ is not perfect for any prime $p.$ (Or more generally, $\mathbb{F}(X)$ is not perfect if $\chr(\mathbb{F}) \neq 0.$) 
        \item By \Cref{prop:irredsepderiv}, we have that every field of characteristic $0$ is perfect.
    \end{enumerate}
\end{ex}

\begin{restatable}[]{thm}{perfectiffppower}
\label{thm:perfectiffppower}
    Let $\mathbb{F}$ be a field with characteristic $p > 0.$ Then, $\mathbb{F}$ is perfect iff $\mathbb{F} = \mathbb{F}^p.$ \hfill\hyperref[thm:perfectiffppower2]{\downsym}
\end{restatable}

\begin{restatable}[]{cor}{finitefieldperfect}
\label{cor:finitefieldperfect}
    Every finite field is perfect. \hfill\hyperref[cor:finitefieldperfect2]{\downsym}
\end{restatable}

\section{Extensions of embeddings}
\begin{restatable}[]{prop}{samemultirredpoly}
\label{prop:samemultirredpoly}
    Let $f(x) \in \mathbb{F}[x]$ be an irreducible monic polynomial. Then, all roots of $f(x)$ have equal multiplicity (in any splitting field). \\
    If $\chr(\mathbb{F}) = 0,$ then all roots are simple. \\
    If $\chr(\mathbb{F}) =\vcentcolon p > 0,$ then all roots have multiplicity $p^n$ for some $n \in \mathbb{N}_0.$ \hfill\hyperref[prop:samemultirredpoly2]{\downsym}
\end{restatable}
Note that by \Cref{prop:multindepsplitting}, the $n$ also does not depend on choice of splitting field.

\begin{restatable}[]{thm}{separabledegreedef}
\label{thm:separabledegreedef}
    Let $\sigma : \mathbb{F} \to \mathbb{L}$ be an embedding of fields where $\mathbb{L}$ is an algebraic closure of $\sigma(\mathbb{F}).$ Similarly, let $\tau : \mathbb{F} \to \mathbb{L}'$ be an embedding of fields where $\mathbb{L}'$ is an algebraic closure of $\tau(\mathbb{F}).$ Let $\mathbb{E}$ be an algebraic extension of $\mathbb{F}.$

    Let $S_\sigma$ (resp. $S_\tau$) denote the set of extensions of $\sigma$ (resp. $\tau$) to embeddings of $\mathbb{E}$ into $\mathbb{L}$ (resp. $\mathbb{L}'$). Let $\lambda : \mathbb{L} \to \mathbb{L}'$ be an isomorphism extending $\tau \circ \sigma^{-1} : \sigma(\mathbb{F}) \to \tau(\mathbb{F}).$  

    The map $\psi : S_\sigma \to S_\tau$ given by $\psi(\widetilde{\sigma}) = \lambda \circ \widetilde{\sigma}$ is a bijection. \hfill\hyperref[thm:separabledegreedef2]{\downsym}

    \begin{center}
        \begin{tikzcd}[ampersand replacement=\&]
        \mathbb{L}' \arrow[dd, no head]                  \&  \&                                                                                                     \&  \& \mathbb{L} \arrow[llll, "\lambda"'] \arrow[dd, no head] \\
                                                         \&  \&                                                                                                     \&  \&                                                         \\
        \widetilde{\tau}(\mathbb{E}) \arrow[dd, no head] \&  \& \mathbb{E} \arrow[ll, "\widetilde{\tau} \in S_\tau"'] \arrow[rr, "\widetilde{\sigma} \in S_\sigma"] \&  \& \widetilde{\sigma}(\mathbb{E}) \arrow[dd, no head]      \\
                                                         \&  \&                                                                                                     \&  \&                                                         \\
        \tau(\mathbb{F})                                 \&  \& \mathbb{F} \arrow[rr, "\sigma"] \arrow[ll, "\tau"']                                                 \&  \& \sigma(\mathbb{F})                                     
        \end{tikzcd}
    \end{center}  
\end{restatable}

\begin{rem}
    What the above proposition is really saying is that the ``number'' (cardinality) of extensions does not depend on $\mathbb{L}$ \textbf{\emph{or}} on the embedding $\sigma.$ Note that since $\mathbb{E}$ is an arbitrary algebraic extension, the set $S_\sigma$ need not be finite. 

    Thus, we may assume $\mathbb{L} \supset \mathbb{F}$ to be an algebraic closure of $\mathbb{F}$ and $\sigma$ to be the natural inclusion.
\end{rem}

\begin{defn}%[]
    If $\mathbb{E}/\mathbb{F}$ is an algebraic extension, then the cardinality of $S_\sigma$ (as in \Cref{thm:separabledegreedef}) is called the \deff{separable degree} of $\mathbb{E}/\mathbb{F}$ and is denoted $[\mathbb{E} : \mathbb{F}]_s.$
\end{defn}

\begin{rem}
    Note that if $\sigma : \mathbb{F} \to \mathbb{L}$ is an embedding into an algebraically closed field $\mathbb{L},$ and $\widetilde{\sigma} : \mathbb{E} \to \mathbb{L}$ is an extension of $\sigma,$ where $\mathbb{E}/\mathbb{F}$ is algebraic, then $\widetilde{\sigma}(\mathbb{E})$ is actually contained in the algebraic closure of $\sigma(\mathbb{F})$ within $\mathbb{L}.$ Thus, it is fine even if $\mathbb{L}$ is not an algebraic closure of $\sigma(\mathbb{F}).$
\end{rem}

\begin{prop} \label{prop:sepdeglessthannordeg}
    Let $\alpha \in \mathbb{E} \supset \mathbb{F}$ be algebraic over $\mathbb{F}$ and $n \vcentcolon= \deg(\irr(\alpha, \mathbb{F})).$ Then, $[\mathbb{F}(\alpha) : \mathbb{F}]_s \le n = [\mathbb{F}(a) : \mathbb{F}]$ with equality iff $\alpha$ is separable over $\mathbb{F}.$
\end{prop}
\begin{proof} 
    By \Cref{prop:rootsandextensions}, we know that $[\mathbb{F}(\alpha) : \mathbb{F}]_s$ is exactly the number of roots of $p(x) \vcentcolon= \irr(\alpha, \mathbb{F})$ in $\overline{\mathbb{F}}.$ This is at most $n = \deg(p(x)).$ Moreover, equality implies that all roots are distinct and hence, $\alpha$ is separable.
\end{proof}

\begin{restatable}[Tower Law for separable degree]{thm}{towerlawsep}
\label{thm:towerlawsep}
    Let $\mathbb{F} \subset \mathbb{E} \subset \mathbb{K}$ be a tower of finite algebraic extensions. Then, $[\mathbb{E} : \mathbb{F}]_s \le [\mathbb{E} : \mathbb{F}]$ and
    \begin{equation*} 
        [\mathbb{K} : \mathbb{F}]_s = [\mathbb{K} : \mathbb{E}]_s [\mathbb{E} : \mathbb{F}]_s.
    \end{equation*} \hfill\hyperref[thm:towerlawsep2]{\downsym}
\end{restatable}

\begin{cor}
    Let $\mathbb{F} \subset \mathbb{E} \subset \mathbb{K}$ be a tower of finite algebraic extensions. Then, $[\mathbb{K} : \mathbb{F}] = [\mathbb{K} : \mathbb{F}]_s$ iff equality holds at each stage.
\end{cor}

\begin{restatable}[]{thm}{sepiffdegequal}
\label{thm:sepiffdegequal}
    Let $\mathbb{E}/\mathbb{F}$ be a finite extension. Then, $\mathbb{E}/\mathbb{F}$ is separable iff $[\mathbb{E} : \mathbb{F}]_s = [\mathbb{E} : \mathbb{F}].$ \hfill\hyperref[thm:sepiffdegequal2]{\downsym}
\end{restatable}

\begin{cor} \label{cor:adjoiningsepissep}
    Let $\alpha \in \mathbb{E} \supset \mathbb{F}$ be separable over $\mathbb{F}.$ Then, $\mathbb{F}(\alpha)/\mathbb{F}$ is a separable extension.
\end{cor}
\begin{proof}
    By \Cref{prop:sepdeglessthannordeg}, we have $[\mathbb{F}(\alpha) : \mathbb{F}]_s = [\mathbb{F}(\alpha) : \mathbb{F}].$ By \Cref{thm:sepiffdegequal}, this means that $\mathbb{F}(\alpha)/\mathbb{F}$ is separable.
\end{proof}

\begin{restatable}[]{prop}{compdecompsep}
\label{prop:compdecompsep}
    Let $\mathbb{F} \subset \mathbb{E} \subset \mathbb{K}$ be a tower of fields. Then,\\
    $\mathbb{K}/\mathbb{F}$ is separable iff $\mathbb{K}/\mathbb{E}$ and $\mathbb{E}/\mathbb{F}$ are separable. \hfill\hyperref[prop:compdecompsep2]{\downsym}
\end{restatable}

\begin{cor}
    Let $f(x) \in \mathbb{F}[x]$ be a separable polynomial and $\mathbb{E} \supset \mathbb{F}$ be a splitting field of $f(x)$ over $\mathbb{F}.$ Then, $\mathbb{E}/\mathbb{F}$ is separable.
\end{cor}
\begin{proof}
    Write $\mathbb{E} = \mathbb{F}(r_1, \ldots, r_n)$ where $f(x) = a(x - r_1) \cdots (x - r_n)$ and use the previous corollary and proposition repeatedly.
\end{proof}

\begin{restatable}[]{prop}{sepdegdividesdeg}
\label{prop:sepdegdividesdeg}
    Let $\mathbb{E}/\mathbb{F}$ be a finite extension. Then, $[\mathbb{E} : \mathbb{F}]_s$ divides $[\mathbb{E} : \mathbb{F}].$ If $\chr(\mathbb{F}) =\vcentcolon p > 0,$ then quotient $\dfrac{[\mathbb{E} : \mathbb{F}]}{[\mathbb{E} : \mathbb{F}]_s}$ is a power of $p.$ \hfill\hyperref[prop:sepdegdividesdeg2]{\downsym}
\end{restatable}

\chapter{Finite fields}

\section{Existence and Uniqueness}

In this section, $p$ will denote an arbitrary prime number.

\begin{restatable}[Uniqueness of finite fields]{thm}{uniquefinfields}
\label{thm:uniquefinfields}
    Let $\mathbb{K}$ and $\mathbb{L}$ be finite fields with same cardinality. Then, $\mathbb{K}$ and $\mathbb{L}$ are isomorphic. \hfill\hyperref[thm:uniquefinfields2]{\downsym}
\end{restatable}

\begin{defn}%[]
    We shall denote \emph{the} finite field with $p^n$ elements by $\mathbb{F}_{p^n}.$
\end{defn}

\begin{rem}
    We have not yet shown that $\mathbb{F}_{p^n}$ exists for every prime $p$ and $n \in \mathbb{N}.$ Have only shown uniqueness up to isomorphism.
\end{rem}

\begin{restatable}[Existence of finite fields]{thm}{existencefinfields}
\label{thm:existencefinfields}
    Fix a prime $p$ and an algebraic closure $\overline{\mathbb{F}}_p.$ For every $n \in \mathbb{N},$ there exists a unique subfield of $\overline{\mathbb{F}}_p$ of size $p^n,$ denoted $\mathbb{F}_{p^n}.$ Moreover
    \begin{equation*} 
        \overline{\mathbb{F}}_p = \bigcup_{n \in \mathbb{N}} \mathbb{F}_{p^n}.
    \end{equation*}
    \hfill\hyperref[thm:existencefinfields2]{\downsym}
\end{restatable}

Here's an interesting application to finite fields.

\begin{restatable}[]{prop}{xfourplusone}
\label{prop:xfourplusone}
    The polynomial $f(x) \vcentcolon= x^4 + 1$ is irreducible in $\mathbb{Z}[x]$ but it is reducible in $\mathbb{F}_p$ for every prime $p.$ \hfill\hyperref[prop:xfourplusone2]{\downsym}
\end{restatable}

\section{Gauss' Necklace Formula}

Recall the M\"obius inversion formula.

\begin{defn}%[]
    The \deff{M\"obius function} $\mu : \mathbb{N} \to \mathbb{N}$ is defined as
    \begin{equation*} 
        \mu(n) \vcentcolon= \begin{cases}
            1 & n = 1,\\
            (-1)^r & n \text{ is a product of }r \text{ distinct primes},\\
            0 & p^2 \mid n \text{ for some prime }p.
        \end{cases}
    \end{equation*}
\end{defn}

\begin{thm}[M\"obius inversion formula] \label{thm:mobiusinv}
    Let $f, g : \mathbb{N} \to \mathbb{N}$ be functions satisfying
    \begin{equation*} 
        f(n) = \sum_{d \mid n} g(d).
    \end{equation*}
    Then, they also satisfy
    \begin{equation*} 
        g(n) = \sum_{d \mid n} f\left(\frac{n}{d}\right)\mu(d).
    \end{equation*}
\end{thm}

\textbf{Notation:} For the remaining of this section, $p$ is an odd prime and $q$ is a positive integral power of $p.$

\begin{restatable}[]{lem}{xdxxnxdiv}
\label{lem:xdxxnxdiv}
    If $m \mid n,$ then $x^{q^m} - x \mid x^{q^n} - x$ in $\mathbb{F}_q[x].$ \hfill\hyperref[lem:xdxxnxdiv2]{\downsym}
\end{restatable}

\begin{restatable}[]{lem}{irreddivsplitpoly}
\label{lem:irreddivsplitpoly}
    Let $f(x) \in \mathbb{F}_q[x]$ be a monic irreducible polynomial. \\
    Then, $f(x) \mid x^{q^n} - x$ iff $\deg(f(x)) \mid n.$ \hfill\hyperref[lem:irreddivsplitpoly2]{\downsym}
\end{restatable}

\begin{rem}
    This shows that the monic factorisation of $x^{q^n} - x$ in $\mathbb{F}_q[x]$ consists of every (monic) irreducible polynomial of degree $d$ as a factor, where $d$ runs over all divisors of $n.$ (No factor can be repeated twice since the polynomial is separable.)
\end{rem}

\begin{restatable}[Gauss]{thm}{gaussnecklace}
\label{thm:gaussnecklace}
    The number of irreducible polynomials of degree $n$ over $\mathbb{F}_{q}$ is given by
    \begin{equation*} 
        N_q(n) = \frac{1}{n}\sum_{d \mid n} \mu(d)q^{n/d}.
    \end{equation*} \hfill\hyperref[thm:gaussnecklace2]{\downsym}
\end{restatable}

\section{Primitive Element Theorem}

\begin{defn}%[]
    Let $\mathbb{E}/\mathbb{F}$ be a field extension. An element $\alpha \in \mathbb{E}$ is called a \deff{primitive element for $\mathbb{E}$ over $\mathbb{F}$} if $\mathbb{E} = \mathbb{F}(\alpha).$

    We say that \deff{$\mathbb{E}$ is primitive over $\mathbb{F}$} if there exists a primitive element for $\mathbb{E}$ over $\mathbb{F}.$
\end{defn}

\begin{restatable}[Primitive Element Theorem]{thm}{pet}
\label{thm:pet}
    Let $\mathbb{K}/\mathbb{F}$ be a finite extension. 
    \begin{enumerate}
        \item There is a primitive element for $\mathbb{K}/\mathbb{F}$ iff the number of intermediate subfields $\mathbb{E}$ such that $\mathbb{F} \subset \mathbb{E} \subset \mathbb{K}$ is finite.
        \item If $\mathbb{K}/\mathbb{F}$ is a separable extension, then it has a primitive element. \hfill\hyperref[thm:pet2]{\downsym}
    \end{enumerate}
\end{restatable}

\chapter{Normal extensions}

\begin{defn}%[]
    An algebraic extension $\mathbb{E}/\mathbb{F}$ is called a \deff{normal extension} if whenever $f(x) \in \mathbb{F}[x]$ is irreducible and has a root in $\mathbb{E},$ then $f(x)$ splits into linear factors in $\mathbb{E}[x].$ 
\end{defn}

\begin{defn}%[]
    Let $\mathbb{E}/\mathbb{F}$ be an extension and $\mathcal{F} = \{f_i(x)\}_{i \in I}$ be a (possibly infinite) family of non-constant polynomials in $\mathbb{F}[x].$ Then, $\mathbb{E}$ is said to be a \deff{splitting field for the family $\mathcal{F}$ over $\mathbb{F}$} if each $f_i(x)$ splits as a product of linear factors in $\mathbb{E}[x]$ and is generated by the roots of the polynomials.
\end{defn}

\begin{rem} \label{rem:splitfamilyexists}
    Note that a splitting field of any family always exists, since an algebraic closure always exists. So, we consider $A \subset \overline{\mathbb{F}}$ to be the set of roots of all the polynomials of the family $\mathcal{F}$ and then put $\mathbb{E} \vcentcolon= \mathbb{F}(A) \subset \overline{\mathbb{F}}.$
\end{rem}

\begin{restatable}[]{prop}{seppolysplittingfields}
\label{prop:seppolysplittingfields}
    Let $\mathbb{F}$ be a field $\mathcal{F} \subset \mathbb{F}[x]$ be a family of separable polynomials. Then, the splitting field of $\mathbb{E} \subset \overline{\mathbb{F}}$ of $\mathcal{F}$ over $\mathbb{F}$ is a separable extension. \hfill\hyperref[prop:seppolysplittingfields2]{\downsym}
\end{restatable}

\begin{restatable}[]{lem}{algebraicautomorphism}
\label{lem:algebraicautomorphism}
    Let $\mathbb{E}/\mathbb{F}$ be an algebraic extension. Let $\sigma : \mathbb{E} \to \mathbb{E}$ be an $\mathbb{F}$-embedding. Then, $\sigma$ is an automorphism of $\mathbb{E}.$ \hfill\hyperref[lem:algebraicautomorphism2]{\downsym}
\end{restatable}

\begin{restatable}[]{thm}{normalequivalent}
\label{thm:normalequivalent}
    Let $\mathbb{F}$ be a field and fix an algebraic closure $\overline{\mathbb{F}}$ of $\mathbb{F}.$ Let $\mathbb{F} \subset \mathbb{E} \subset \overline{\mathbb{F}}$ be fields. Then, the following are equivalent:
    \begin{enumerate}
         \item \label{item:001} Every $\mathbb{F}$-embedding $\sigma : \mathbb{E} \to \overline{\mathbb{F}}$ is an automorphism of $\mathbb{E}.$
         \item \label{item:002} $\mathbb{E}$ is a splitting field of a family of polynomials in $\mathbb{F}[x].$
         \item \label{item:003} $\mathbb{E}/\mathbb{F}$ is a normal extension. \hfill\hyperref[thm:normalequivalent2]{\downsym}
     \end{enumerate} 
\end{restatable}

\begin{restatable}[]{prop}{operationsonnormalexts}
\label{prop:operationsonnormalexts}
    Let $\mathbb{F} \subset \mathbb{E}_1, \mathbb{E}_2 \subset \mathbb{K}$ be fields. Suppose that $\mathbb{E}_i/\mathbb{F}$ are normal. Then, so are $\mathbb{E}_1\mathbb{E}_2/\mathbb{F}$ and $(\mathbb{E}_1 \cap \mathbb{E}_2)/\mathbb{F}.$ \hfill\hyperref[prop:operationsonnormalexts2]{\downsym}
\end{restatable}

\begin{ex}
    Quadratic extensions are always normal. Indeed, pick $\alpha \in \mathbb{E} \setminus \mathbb{F}.$ Then, $\mathbb{E} = \mathbb{F}(\alpha)$ is a splitting field of $\irr(\alpha, \mathbb{F})$ over $\mathbb{F}.$
\end{ex}

\begin{rem}
    Unlike the ``tower laws'' for algebraic and separable extensions, the ``composition'' of normal extensions need not be normal. For example, consider the chain
    \begin{equation*} 
        \mathbb{Q} \subset \mathbb{Q}(\sqrt{2}) \subset \mathbb{Q}(\sqrt[4]{2}).
    \end{equation*}
    Each successive extension is quadratic and hence, normal. However, $\mathbb{Q}(\sqrt[4]{2})/\mathbb{Q}$ is not normal since the irreducible (via Eisenstein) polynomial $x^4 - 2 \in \mathbb{Q}[x]$ has a root in $\mathbb{Q}(\sqrt[4]{2})$ but does not factor completely.

    On the other hand, consider
    \begin{equation*} 
        \mathbb{Q} \subset \mathbb{Q}(\sqrt[4]{2}) \subset \mathbb{Q}(\sqrt[4]{2}, \iota).
    \end{equation*}
    Then, $\mathbb{Q}(\sqrt[4]{2}, \iota)/\mathbb{Q}$ is normal since $(\sqrt[4]{2}, \iota)$ is the splitting field for $x^4 - 2$ over $\mathbb{Q}$ but $(\sqrt[4]{2})/\mathbb{Q}$ is not.
\end{rem}

However, one part of the ``tower property'' \emph{does} hold, as can be easily verified, either directly from the definition or using one of the equivalences proven above.

\begin{prop} \label{prop:decompnormal}
    Let $\mathbb{F} \subset \mathbb{E} \subset \mathbb{K}$ be fields such that $\mathbb{K}/\mathbb{F}$ is normal. Then, $\mathbb{K}/\mathbb{E}$ is normal.
\end{prop}

\begin{rem}
    The above phenomenon is related (at least in the case of finite extensions) to the phenomenon that ``is a normal subgroup'' is not transitive either. Given groups $H \le K \le G,$ it is possible that $H$ is normal in $K$ and $K$ in $G$ but $H$ is not normal in $G.$ 

    Similarly, if we know that $H$ is normal in $G,$ then we can conclude that $H$ is normal in $K$ but $K$ need not be normal in $G.$
\end{rem}
\chapter{Galois Extensions}
\section{Definitions}

\begin{defn}%[]
    A field extension $\mathbb{E}/\mathbb{F}$ is called a \deff{Galois extension} if it is normal and separable. The \deff{Galois group of a Galois extension} $\mathbb{E}/\mathbb{F}$ is the group of all $\mathbb{F}$-automorphisms of $\mathbb{E}$ under the operation of composition of maps. It is denoted $\Gal(\mathbb{E}/\mathbb{F}).$

    If $f(x) \in \mathbb{F}[x]$ is a separable polynomial and $\mathbb{E}$ is a splitting field of $f(x)$ over $\mathbb{F},$ then $\mathbb{E}/\mathbb{F}$ is a Galois extension and the \deff{Galois group of $f(x)$ over $\mathbb{F}$} is defined to be $\Gal(\mathbb{E}/\mathbb{F})$ and denoted as $\Gal(f(x), \mathbb{F})$ or simply $\G_f$ if $\mathbb{F}$ is clear.
\end{defn}

\begin{rem}
    Note that the definition of the $\Gal(f(x), \mathbb{F})$ does not depend on the splitting field chosen, up to isomorphism. Indeed, let $\mathbb{E}$ and $\mathbb{E}'$ be two splitting fields of $f(x)$ over $\mathbb{F}.$ By \Cref{thm:isosplitting}, there is an $\mathbb{F}$-isomorphism $\tau : \mathbb{E} \to \mathbb{E}'.$ Then, $\sigma \mapsto \tau \circ \sigma \circ \tau^{-1}$ is an isomorphism from $\Gal(\mathbb{E}/\mathbb{F})$ to $\Gal(\mathbb{E}'/\mathbb{F}).$
\end{rem}

\begin{ex}
    Here are some examples and non-examples.
    \begin{enumerate}
        \item Let $\mathbb{E}/\mathbb{F}$ be an extension of finite fields. Then, $\md{\mathbb{F}} = q$ and $\md{\mathbb{E}} = q^n$ for some prime power $q$ and $n \in \mathbb{N}.$ Then, $\mathbb{E}$ is a splitting field for $x^{q^n} - x \in \mathbb{F}[x]$ over $\mathbb{F}.$ Thus, the extension is normal. \\
        Since the fields are finite, it is also separable.
        %
        \item The extension $\mathbb{Q}(\sqrt[3]{2})/\mathbb{Q}$ is \textbf{not} Galois. Since $\chr(\mathbb{Q}) = 0,$ it is separable. However, it is not normal. Indeed, the irreducible (by Eisenstein) polynomial $x^3 - 2 \in \mathbb{Q}[x]$ has a root in $\mathbb{Q}(\sqrt[3]{2})$ but it does not split as a product of linear factors.
        %
        \item The extension $\mathbb{F}_p(X)(X^{1/p})/\mathbb{F}_p(X)$ is not separable and hence, \textbf{not} Galois. It \emph{is} normal since the bigger field is the splitting field of $T^p - X \in \mathbb{F}_p(X)[T].$
    \end{enumerate}
\end{ex}

\begin{restatable}[]{prop}{orderofgalgroup}
\label{prop:orderofgalgroup}
    Let $\mathbb{E}/\mathbb{F}$ be a finite Galois extension. Then, $\md{\Gal(\mathbb{E}/\mathbb{F})} = [\mathbb{E} : \mathbb{F}]_s = [\mathbb{E} : \mathbb{F}].$ \hfill\hyperref[prop:orderofgalgroup2]{\downsym}
\end{restatable}
Note that the last equality is simply by definition of a Galois extension (and \Cref{thm:sepiffdegequal}).

\begin{rem}
    The above proposition shows why normality and separability are both needed. If the extension is normal but not separable, then the order of the group would be the separable degree.

    On the other hand, if the extension is separable but not normal, then there would be an extension $\sigma : \mathbb{E} \to \overline{\mathbb{F}}$ would map $\mathbb{E}$ outside $\mathbb{E}$ and so, not all extensions will belong to the Galois group.

    As an example, consider $\mathbb{Q}(\sqrt[3]{2})/\mathbb{Q}.$ Since there is only one root of $x^3 - 2$ in $\mathbb{Q}(\sqrt[3]{2}),$ there is only one $\mathbb{Q}$-automorphism of $\mathbb{Q}(\sqrt[3]{2}).$
\end{rem}

\begin{restatable}[]{prop}{frobgenerates}
\label{prop:frobgenerates}
    Let $q$ be a prime power.

    The Galois group of the Galois extension $\mathbb{F}_{q^n}/\mathbb{F}_q$ is a cyclic group of order $n$ generated by the Frobenius automorphism $\varphi : \mathbb{F}_{q^n} \to \mathbb{F}_{q^n}$ defined as $a \mapsto a^q.$ \hfill\hyperref[prop:frobgenerates2]{\downsym}
\end{restatable}

\begin{ex}
    A field extension $\mathbb{K}/\mathbb{F}$ is called \deff{biquadratic} if $[\mathbb{K} : \mathbb{F}] = 4$ and $\mathbb{K}$ is generated over $\mathbb{F}$ by roots of two irreducible quadratic separable polynomials. 

    In particular, $\mathbb{K}/\mathbb{F}$ is a Galois extension. Write $\mathbb{K} = \mathbb{F}(\alpha, \beta)$ and let $p(x) \vcentcolon= \irr(\alpha, \mathbb{F})$ and $q(x) \vcentcolon= \irr(\beta, \mathbb{F}).$ Let $\overline{\alpha}, \overline{\beta} \in \mathbb{K}$ denote the other root of $p(x)$ and $q(x).$ By assumption of separability, $\overline{\alpha} \neq/ \alpha$ and $\overline{\beta} \neq \beta.$

    Since $[\mathbb{F}(\alpha, \beta) : \mathbb{F}] = 4,$ the quadratic $p(x)$ is irreducible over $\mathbb{F}(\beta)$ and similarly for $q(x)$ over $\mathbb{F}(\alpha).$ Thus, the four automorphisms are determined by sending $\alpha$ to $\alpha$ or $\overline{\alpha}$ and $\beta$ to $\beta$ or $\overline{\beta}.$

    Define the automorphisms $\tau, \sigma : \mathbb{K} \to \mathbb{K}$ by
    \begin{align*} 
        \tau(\alpha) = \overline{\alpha},\;\tau(\beta) = \beta,\\
        \sigma(\alpha) = \alpha,\;\sigma(\beta) = \overline{\beta}.
    \end{align*}
    Then, $\tau^2 = \sigma^2 = \id_{\mathbb{K}}.$ Thus, $\Gal(\mathbb{K}/\mathbb{F}) \cong \mathbb{Z}/2\mathbb{Z} \times \mathbb{Z}/2\mathbb{Z},$ the Klein-$4$ group.
\end{ex}

\begin{ex}[Galois group of a separable cubic] \label{ex:galsepcubic}
    We show the role of the discriminant in determining the Galois group of a cubic.

    Let $\mathbb{F}$ be a field with $\chr(\mathbb{F}) \neq 2, 3.$ Let $f(x) = x^3 + px + q \in \mathbb{F}[x]$ be an irreducible cubic. In particular, $f(x)$ has no roots in $\mathbb{F}.$ We wish to show that $f(x)$ is separable. Note that
    \begin{equation*} 
        f'(x) = 3x^2 + p \neq 0,
    \end{equation*}
    since $\chr(\mathbb{F}) \neq 3.$ Thus, $f(x)$ is separable, by \Cref{prop:irredsepderiv}.

    Thus, a splitting field $\mathbb{E}$ of $f(x)$ over $\mathbb{F}$ has degree either $3$ or $6.$ By \Cref{prop:orderofgalgroup}, we know that $\md{\Gal(\mathbb{E}/\mathbb{F})} = 3$ or $6.$ We see now how the discriminant determines this. 

    Let $\mathbb{E} = \mathbb{F}(\alpha_1, \alpha_2, \alpha_3),$ where $f(x) = \prod_{i = 1}^3 (x - \alpha_i).$ Any $\sigma \in \Gal(\mathbb{E}/\mathbb{F})$ permutes these roots. Let $p_\sigma \in S_3$ denote the corresponding permutation. It is easy to see that $\sigma \mapsto p_\sigma$ is injective. (Action of $\sigma$ on $\sigma_i$ completely determines the automorphism.) Under this, we identify $\Gal(\mathbb{E}/\mathbb{F})$ with a subgroup of $S_3.$

    Thus, $\Gal(\mathbb{E}/\mathbb{F}) = A_3$ or $S_3.$ Let
    \begin{equation*} 
        \delta = (\alpha_1 - \alpha_2)(\alpha_2 - \alpha_3)(\alpha_3 - \alpha_1).
    \end{equation*}
    Then, $\delta^2 = \disc(f(x)) = -(4p^3 + 27q^2) \in \mathbb{F}.$ (Recall we had calculated this discriminant in \Cref{ex:disccubic}.)

    Thus, $[\mathbb{F}(\delta) : \mathbb{F}] \le 2.$ Now, if $\delta \in \mathbb{F},$ then $\Gal(\mathbb{E}/\mathbb{F})$ cannot have any odd permutations since they do not fix $\delta$ and hence, $\Gal(\mathbb{E}/\mathbb{F}) = A_3.$

    On the other hand, if $\delta \notin \mathbb{F},$ then $2 = [\mathbb{F}(\delta) : \mathbb{F}] \mid [\mathbb{E} : \mathbb{F}]$ and so, $\Gal(\mathbb{E}/\mathbb{F}) = S_3.$

    Note that $\delta \in \mathbb{F} \iff \disc(f(x))$ is a perfect square in $\mathbb{F}.$ Thus, the above is characterised entirely by $\disc(f(x))$ being a perfect square.

    For example, if $f(x) = x^3 + x + 1 \in \mathbb{Q}[x],$ then $\disc(f(x)) = -31$ and so, $\Gal(\mathbb{E}/\mathbb{Q}) \cong S_3.$ On the other hand, if $f(x) = x^3 - 3x + 1,$ then $\disc(f(x)) = 81 = 9^2$ and thus, $\Gal(\mathbb{E}/\mathbb{Q}) \cong A_3.$
\end{ex}

\section{The Fundamental Theorem of Galois Theory}
\begin{defn}%[]
    Let $\mathbb{E}$ be a field and $G$ be \underline{a} group of automorphisms of $\mathbb{E}.$ Then,
    \begin{equation*} 
        \mathbb{E}^G \vcentcolon= \{a \in \mathbb{E} : \sigma(a) = a \text{ for all } \sigma \in G\}
    \end{equation*}
    is called the \deff{fixed field of $G$ acting on $E$}.
\end{defn}
\begin{rem}
    As one can easily show, the above is indeed a field. 

    Note that $G$ is not necessarily the group of \emph{all} automorphisms of $\mathbb{E}.$
\end{rem}

\begin{restatable}[Fundamental Theorem of Galois Theory (FTGT)]{thm}{FTGT}
\label{thm:FTGT}
    Let $\mathbb{K}/\mathbb{F}$ be a \underline{finite} Galois extension. Consider the sets
    \begin{equation*} 
        \mathcal{I} = \{\mathbb{E} \mid \mathbb{E} \text{ is an intermediate field of }\mathbb{K}/\mathbb{F}\} \andd \mathcal{G} = \{H \mid H \le \Gal(\mathbb{K}/\mathbb{F})\}.
    \end{equation*}
    \begin{enumerate}
         \item \label{item:G1} The maps 
         \begin{equation*} 
             E \mapsto \Gal(\mathbb{K}/\mathbb{E}) \andd H \mapsto \mathbb{K}^H
         \end{equation*}
         give a one-to-one correspondence between $\mathcal{I}$ and $\mathcal{G},$ called the \deff{Galois correspondence}. Moreover, these are inclusion reversing.
         %
         \item \label{item:G2} $\mathbb{E}/\mathbb{F}$ is Galois iff $\Gal(\mathbb{K}/\mathbb{E}) \unlhd \Gal(\mathbb{K}/\mathbb{F})$ and in this case,
         \begin{equation*} 
             \Gal(\mathbb{E}/\mathbb{F}) \cong \frac{\Gal(\mathbb{K}/\mathbb{F})}{\Gal(\mathbb{K}/\mathbb{E})}.
         \end{equation*}
         %
         \item \label{item:G3} $\mathbb{K}/\mathbb{E}$ is always Galois and $\md{\Gal(\mathbb{K}/\mathbb{E})} = [\mathbb{K} : \mathbb{E}] = \dfrac{[\mathbb{K} : \mathbb{F}]}{[\mathbb{E} : \mathbb{F}]}.$
         \item \label{item:G5} If $\mathbb{E}_1, \mathbb{E}_2 \in \mathcal{I}$ correspond to $H_1$ and $H_2,$ then $\mathbb{E}_1 \cap \mathbb{E}_2$ corresponds to $\langle H_1, H_2\rangle$ and $\mathbb{E}_1\mathbb{E}_2$ to $H_1 \cap H_2.$
     \end{enumerate} \hfill\hyperref[thm:FTGT2]{\downsym}
\end{restatable}

The proof of the above will be given in many steps. Parts of it will be proven for infinite Galois extensions as well. Note that \ref{item:G3} follows from \Cref{prop:orderofgalgroup}.

For the rest of the section, $\mathbb{K}/\mathbb{F}$ will denote a {\color{purple}(possibly infinite)} Galois extension and $\mathcal{I}$ and $\mathcal{G}$ will be as in \Cref{thm:FTGT}.

\begin{restatable}[]{thm}{fixfieldinjectiveIG}
\label{thm:fixfieldinjectiveIG}
    Let $\mathbb{K}/\mathbb{F}$ be a {\color{purple}(possibly infinite)} Galois extension and put $G = \Gal(\mathbb{K}/\mathbb{F}).$ Then,
    \begin{enumerate}
         \item $\mathbb{F} = \mathbb{K}^G.$
         \item Let $\mathbb{E} \in \mathcal{I}.$ Then, $\mathbb{K}/\mathbb{E}$ is Galois and the map $E \mapsto \Gal(\mathbb{K}/\mathbb{E})$ is an injective map from $\mathcal{I}$ to $\mathcal{G}.$ \hfill\hyperref[thm:fixfieldinjectiveIG2]{\downsym}
     \end{enumerate} 
\end{restatable}

\begin{rem} \label{rem:nonbasemoved}
    The above again shows the need for Galois extension. For example, consider the non-Galois extension $\mathbb{Q}(\sqrt[3]{2})/\mathbb{Q}.$ If we consider $G$ to be the ``Galois group,'' that is, $G$ to be the group of automorphisms of $\mathbb{Q}(\sqrt[3]{2})$ which fix $\mathbb{Q},$ we see that $G$ is trivial. Thus, $\mathbb{Q}(\sqrt[3]{2})^G = \mathbb{Q}(\sqrt[3]{2}).$

    However, for Galois extensions, the above says that the only field which is fixed by all the Galois automorphisms is precisely the base field.
\end{rem}

\begin{restatable}[]{lem}{degboundedbyn}
\label{lem:degboundedbyn}
    Let $\mathbb{E}/\mathbb{F}$ be a separable extension and $n \in \mathbb{N}.$ Suppose that for all $\alpha \in \mathbb{E},$ $[\mathbb{F}(\alpha) : \mathbb{F}] \le n.$ Then, $[\mathbb{E} : \mathbb{F}] \le n.$ \hfill\hyperref[lem:degboundedbyn2]{\downsym}
\end{restatable}

\begin{rem}
    Note that the above did not assume a priori that $\mathbb{E}/\mathbb{F}$ is finite. If that were the case, then the \nameref{thm:pet} would yield the answer.

    The above is not true without the assumption of separability. For example, consider $\mathbb{F} = \mathbb{F}_p(X, Y)$ where $p$ is a prime. Consider $\mathbb{E} = \mathbb{F}(X^{1/p}, Y^{1/p}).$

    Then, $\alpha^p \in \mathbb{F}$ for all $\alpha \in \mathbb{E}$ (exercise) and thus, $[\mathbb{E}(\alpha) : \mathbb{F}] \le p$ for all $\alpha \in \mathbb{E}.$ However, $[\mathbb{E} : \mathbb{F}] = p^2 > p.$
\end{rem}

\begin{restatable}[Artin's Theorem]{thm}{artin}
\label{thm:artin}
    Let $\mathbb{E}$ be a field and $G$ a \underline{finite} group of automorphisms of $\mathbb{E}.$ Then,
    \begin{enumerate}
         \item $\mathbb{E}/\mathbb{E}^G$ is a \emph{finite} Galois extension.
         \item $\Gal(\mathbb{E}/\mathbb{E}^G) = G.$
         \item $[\mathbb{E} : \mathbb{E}^G] = \md{G}.$ \hfill\hyperref[thm:artin2]{\downsym}
     \end{enumerate} 
\end{restatable}

\begin{restatable}[]{thm}{galoissubgroupscompositum}
\label{thm:galoissubgroupscompositum}
    Let $\mathbb{K}/\mathbb{F}$ be a {\color{purple}(possibly infinite)} Galois extension with Galois group $G.$ Let $\mathbb{E}_1$ and $\mathbb{E}_2$ be intermediate subfields of $\mathbb{K}/\mathbb{F}.$ Let $H_i \vcentcolon= \Gal(\mathbb{K}/\mathbb{E}_i)$ for $i = 1, 2.$Then
    \begin{equation*} 
        \mathbb{E}_1\mathbb{E}_2 = \mathbb{K}^{H_1 \cap H_2},\; \mathbb{E}_1 \cap \mathbb{E}_2 = \mathbb{K}^{\langle H_1, H_2\rangle}, \text{ and } \mathbb{E}_1 \subset \mathbb{E}_2 \iff H_1 \supset H_2.
    \end{equation*} \hfill\hyperref[thm:galoissubgroupscompositum2]{\downsym}
\end{restatable}

\begin{rem}
    Essentially the thing to keep in mind is that smaller subfields corresponding to larger subgroups. Now, given two subfields/subgroups, we have the corresponding smallest (or largest) subfield/subgroup containing them (or being contained in them). The above shows that the Galois correspondence (in one direction) preserves them.

    (The smallest field containing the subfields is the fixed field of the action of the largest subgroup contained in the Galois groups.\\
    The largest field containing the subfields is the fixed field of the action of the smallest subgroup containing the Galois groups.)
\end{rem}

\begin{restatable}[]{prop}{isomorphismgalois}
\label{prop:isomorphismgalois}
    Let $\mathbb{K}/\mathbb{F}$ be a {\color{purple}(possibly infinite)} Galois extension. Let $\lambda : \mathbb{K} \to \lambda(\mathbb{K})$ be an isomorphism of fields. Then,
    \begin{enumerate}
         \item $\lambda(\mathbb{K})/\lambda(\mathbb{F})$ is a Galois extension.
         \item $\Gal(\lambda(\mathbb{K})/\lambda(\mathbb{F})) = \lambda \Gal(\mathbb{K}/\mathbb{F}) \lambda^{-1} \cong \Gal(\mathbb{K}/\mathbb{F}).$ \hfill\hyperref[prop:isomorphismgalois2]{\downsym}
    \end{enumerate} 
\end{restatable}

\begin{restatable}[]{thm}{galoisiffnormal}
\label{thm:galoisiffnormal}
    Let $\mathbb{K}/\mathbb{F}$ be a {\color{purple}(possibly infinite)} Galois extension. Let $\mathbb{E}$ be an intermediate subfield of $\mathbb{K}/\mathbb{F}.$ Then,
    \begin{enumerate}
        \item $\mathbb{E}/\mathbb{F}$ is Galois iff $\Gal(\mathbb{K}/\mathbb{E}) \unlhd \Gal(\mathbb{K}/\mathbb{F}).$
        \item If $\mathbb{E}/\mathbb{F}$ is Galois, then
        \begin{equation*} 
             \Gal(\mathbb{E}/\mathbb{F}) \cong \frac{\Gal(\mathbb{K}/\mathbb{F})}{\Gal(\mathbb{K}/\mathbb{E})}.
        \end{equation*}
        \hfill\hyperref[thm:galoisiffnormal2]{\downsym}
     \end{enumerate} 
\end{restatable}

With this, we can now prove the \nameref{thm:FTGT}. \hfill\hyperref[thm:FTGT2]{\downsym}


\section{Applications of FTGT}
We give another proof of the Fundamental Theorem of Algebra.

\begin{restatable}[Fundamental Theorem of Algebra]{thm}{ftagalois}
\label{thm:ftagalois}
    The field of complex numbers is algebraically closed. \hfill\hyperref[thm:ftagalois2]{\downsym}
\end{restatable}

\begin{ex}[Symmetric rational functions]
    Let $\mathbb{E} = \mathbb{F}(x_1, \ldots, x_n)$ be the fraction field of $R = \mathbb{F}[x_1, \ldots, x_n],$ where $x_i$ are indeterminates over the field $\mathbb{F}.$

    We had seen that the symmetric polynomials in $R$ are the polynomials in the symmetric polynomials. We now prove an analogous result for symmetric rational functions. 

    Note that $S_n$ acts on $\mathbb{E}$ in the natural way. More precisely, if $\sigma \in S_n,$ then we have the $\mathbb{F}$-automorphism $\varphi_\sigma : \mathbb{E} \to \mathbb{E}$ determined by $\varphi_\sigma(x_i) = x_{\sigma(i)}.$ Note that $\varphi_{\sigma_1\sigma_2} = \varphi_{\sigma_1} \circ \varphi_{\sigma_2}$ and thus, $G = \{\varphi_\sigma : \sigma \in S_n\}$ is a group of automorphisms of $\mathbb{E}$ and is isomorphic to $S_n.$

    Let $\sigma_1, \ldots, \sigma_n \in \mathbb{E}$ be the elementary symmetric polynomials in $x_1, \ldots, x_n.$ Let $X$ be an indeterminate over $\mathbb{E}$ and consider the polynomial ring $\mathbb{E}[X].$ \\
    Each the automorphisms $\varphi_\sigma$ to automorphisms of $\mathbb{E}[X]$ by fixing $X.$ We denote the extension again by $\varphi_\sigma.$

    Consider
    \begin{align*} 
        g(X) &\vcentcolon= (X - x_1) \cdots (X - x_n)\\
        &= X^n - \sigma_1 X^{n - 1} + \cdots + (-1)^n\sigma_n.
    \end{align*}

    Let $\sigma \in S_n$ be arbitrary. Applying $\varphi_\sigma$ to the first line above yields
    \begin{equation*} 
        \varphi_\sigma(g(X)) = (X - x_{\sigma(1)}) \cdots (X - x_{\sigma(n)}) = g(X).
    \end{equation*}
    Thus, each $\varphi_\sigma$ fixes $g(X)$ and in turn, it fixes the coefficients $\sigma_1, \ldots, \sigma_n.$ Thus,
    \begin{equation*} 
        \mathbb{F}(\sigma_1, \ldots, \sigma_n) \subset \mathbb{E}^G.
    \end{equation*}
    Note that
    \begin{equation*} 
        \mathbb{E} = \mathbb{F}(\sigma_1, \ldots, \sigma_n, x_1, \ldots, x_n)
    \end{equation*}
    and so, $\mathbb{E}$ is a splitting field of $g(X)$ over $\mathbb{F}(\sigma_1, \ldots, \sigma_n).$ Since $g(X)$ is separable, we see that $\mathbb{E}/\mathbb{F}(\sigma_1, \ldots, \sigma_n)$ is a Galois extension. 

    Now, if $\pi \in \Gal(\mathbb{E}/\mathbb{F}(\sigma_1, \ldots, \sigma_n)),$ then $\pi$ permutes the roots of $g(X)$ and fixes $\mathbb{F}.$ Thus, $\pi = \varphi_\sigma$ for some $\sigma \in S_n.$ Thus, $G = \Gal(\mathbb{E}/\mathbb{F}(\sigma_1, \ldots, \sigma_n)).$

    Thus, we see that
    \begin{equation*} 
        \mathbb{F}(\sigma_1, \ldots, \sigma_n) = \mathbb{E}^G.
    \end{equation*}
    The left is the field of all rational functions in the symmetric polynomials. The right is the field of all rational functions fixed by $S_n,$ that is, the symmetric rational functions.
\end{ex}

\chapter{Cyclotomic Extensions}

\section{Roots of unity}

\begin{defn}%[]
    Let $\mathbb{F}$ be a field. A root $\zeta \in \mathbb{F}$ of $x^n - 1 \in \mathbb{F}[x]$ is called an \deff{$n$-th root of unity in $\mathbb{F}$}.
\end{defn}

\begin{rem}
    Suppose that $\chr(\mathbb{F}) = p > 0$ and $n = p^em$ with $p \nmid m.$ Then, $x^n = (x^m - 1)^{p^e}.$ By the derivative criterion, $x^m - 1$ is separable. Thus, the splitting field of $x^n - 1$ is the same as that of $x^m - 1$ and the roots are the same too (ignoring multiplicity). Thus, we either consider fields of characteristic $0$ or assume that $(\chr(\mathbb{F}), n) = 1.$
\end{rem}

\begin{defn}%[]
    Let $\mathbb{F}$ be a field and $n \in \mathbb{K}.$ \\
    Suppose that $\chr(\mathbb{F}) = 0$ or $\gcd(\chr(\mathbb{F}), n) = 1.$ Let $Z = \{z_1, \ldots, z_n\} \subset \overline{\mathbb{F}}^\times$ is a cyclic subgroup (\Cref{thm:finsubgroupcyclic}). Any of the $\varphi(n)$ generators of $Z$ is called a \deff{primitive $n$-th root of unity}.

    A primitive root of unity over $\mathbb{Q}$ is denoted by $\zeta_n$ and we define $\Phi_n(x) \vcentcolon= \irr(\zeta_n, \mathbb{Q}).$ 
\end{defn}

\begin{rem}
    We shall soon show that $\irr(\zeta_n, \mathbb{Q})$ is independent of the primitive root chosen. This is \textbf{not} the case in general (see \Cref{ex:irrunityFtwo}).
\end{rem}

\begin{defn}%[]
    A splitting field of $x^n - 1$ over $\mathbb{F}$ is called a \deff{cyclotomic extension of order $n$ over $\mathbb{F}$}.
\end{defn}

\begin{restatable}[]{prop}{Gfabeliansubgroup}
\label{prop:Gfabeliansubgroup}
    Let $\chr(\mathbb{F}) = 0$ or $\gcd(\chr(\mathbb{F}), n) = 1$ and $f(x) = x^n - 1 \in \mathbb{F}[x].$ Then, $\G_f$ is isomorphic to a subgroup of $(\mathbb{Z}/n\mathbb{Z})^\times.$ In particular, $\G_f$ is an abelian group and $\md{\G_f} \mid \varphi(n).$ \hfill\hyperref[prop:Gfabeliansubgroup2]{\downsym}
\end{restatable}

\begin{ex} \label{ex:irrunityFtwo}
    Let us consider $\mathbb{F} = \mathbb{F}_2.$ We shall consider the $n$-th roots of unity for odd $n$ so that $\gcd(n, 2) = 1.$ In this example, we will consider $n = 3$ and $7.$ Since these are prime, we know that there are $2$ and $6$ primitive roots in each case.

    First, consider $x^3 - 1 = (x - 1)(x^2 + x + 1).$ The quadratic factor is irreducible since it has no root. Any root $z$ of the quadratic is a primitive cube root of unity.

    Now, consider $n = 7.$ Then, we have
    \begin{equation*} 
        x^7 - 1 = (x - 1)(x^3 + x^2 + 1)(x^3 + x + 1).
    \end{equation*}
    Note that both the cubics are irreducible since they have no roots in $\mathbb{F}.$ Since any root apart from $1$ is a primitive root, we see that any of the roots of the two cubics is a primitive root. 

    In particular, note that are $6$ primitive $7$-th roots of unity over $\mathbb{F}$ with two minimal polynomials. However, we will see that this does not happen over $\mathbb{Q}.$
\end{ex}

\begin{restatable}[]{prop}{nthrootsnonunity}
\label{prop:nthrootsnonunity}
    Let $x^n - a = f(x) \in \mathbb{F}[x]$ and suppose $\mathbb{F}$ has $n$ distinct roots of $x^n - 1.$ Then, $\G_f$ is a cyclic group and $\md{\G_f}$ divides $n.$ \hfill\hyperref[prop:nthrootsnonunity2]{\downsym}
\end{restatable}

\begin{restatable}[]{thm}{cyclotomicQ}
\label{thm:cyclotomicQ}
    Let $n \in \mathbb{N}$ fix a primitive root $n$-th root of unity $\zeta_n \in \overline{\mathbb{Q}}$ and let $\Phi_n(x) \vcentcolon= \irr(\zeta_n, \mathbb{Q}).$ Then,
    \begin{enumerate}
         \item $\Phi_n(x) \in \mathbb{Z}[x],$
         \item every primitive $n$-th root of unity is a root of $\Phi_n(x),$
         \item $[\mathbb{Q}(\zeta_n) : \mathbb{Q}] = \varphi(n),$ and
         \item $\Gal(\mathbb{Q}(\zeta_n)/\mathbb{Q}) \cong (\mathbb{Z}/n\mathbb{Z})^\times.$ \hfill\hyperref[thm:cyclotomicQ2]{\downsym}
     \end{enumerate} 
\end{restatable}

\section{Computation of Cyclotomic Polynomials}
As earlier, $\Phi_n(x)$ defines the irreducible polynomial of any primitive $n$-th root of unity.

\begin{restatable}[]{thm}{cycloreccurence}
\label{thm:cycloreccurence}
    We have $\Phi_1(x) = x - 1$ and
    \begin{equation*} 
        \Phi_n(x) = \frac{x^n - 1}{\displaystyle\prod_{\substack{d \mid n\\ d < n}} \Phi_d(x)}
    \end{equation*}
    for $n > 1.$ \hfill\hyperref[thm:cycloreccurence2]{\downsym}
\end{restatable}

\begin{ex}[First few cyclotomic polynomials]
    \begin{align*} 
        \Phi_1(x) &= x - 1, \\
        \Phi_2(x) &= \frac{x^2 - 1}{x - 1} = x + 1, \\
        \Phi_3(x) &= \frac{x^3 - 1}{x - 1} = x^2 + x + 1, \\
        \Phi_4(x) &= \frac{x^4 - 1}{(x - 1)(x + 1)} = x^2 + 1, \\
        \Phi_5(x) &= \frac{x^5 - 1}{x - 1} = x^4 + x^3 + x^2 + x + 1, \\
        \Phi_6(x) &= \frac{x^6 - 1}{(x - 1)(x^2 - 1)(x^3 - 1)} = x^2 - x + 1, \\
        \Phi_7(x) &= \frac{x^7 - 1}{x - 1} = x^6 + x^5 + \cdots + x + 1.
    \end{align*}
    Note that the above may indicate that the coefficients are always $0, \pm 1.$ However, that is \textbf{not} the case.

    However, the first example of that is $\Phi_{105}(x).$ The coefficient of $x^{41}$ is $-2.$ (Every other coefficient is $0, \pm 1.$)
\end{ex}

\section{Subfields of \texorpdfstring{$\mathbb{Q}(\zeta_n)$}{Q(zn)}}

\begin{restatable}[]{prop}{cyclocyclic}
\label{prop:cyclocyclic}
    Let $p$ be a prime. Then, $\Gal(\mathbb{Q}(\zeta_p)/\mathbb{Q})$ is cyclic of order $p - 1.$ Consequently, given any divisor $d \mid p - 1,$ there is a unique intermediate subfield $\mathbb{E}$ of $\mathbb{Q}(\zeta_p)/\mathbb{Q}$ such that $[\mathbb{E} : \mathbb{Q}] = d.$ Equivalently, there is a unique intermediate $\mathbb{E}$ such that $[\mathbb{Q}(\zeta_p) : \mathbb{E}] = \frac{p - 1}{d}.$ \hfill\hyperref[prop:cyclocyclic2]{\downsym}
\end{restatable}

\begin{restatable}[]{lem}{cyclodisc}
\label{lem:cyclodisc}
    Let $p$ be an odd prime. Then $\disc(\Phi_p(x)) = (-1)^{\binom{p}{2}}p^{p - 2}.$ \hfill\hyperref[lem:cyclodisc2]{\downsym}
\end{restatable}

\begin{restatable}[]{prop}{uniquequadraticcyclosubfield}
\label{prop:uniquequadraticcyclosubfield}
    Let $p$ be an odd prime. The field $\mathbb{Q}(\zeta_p)$ contains a unique quadratic extension of $\mathbb{Q},$ namely
    \begin{equation*} 
        \mathbb{Q}\left(\sqrt{\disc(\Phi_p(x))}\right) = \mathbb{Q}\left(\sqrt{(-1)^{\binom{p}{2}}}p\right),
    \end{equation*}
    which is real if $p \equiv 1 \pmod{4}$ and (non-real) complex if $p \equiv 3 \pmod{4}.$ \hfill\hyperref[prop:uniquequadraticcyclosubfield2]{\downsym}
\end{restatable}

\begin{restatable}[]{cor}{quadincyclo}
\label{cor:quadincyclo}
    Every quadratic extension of $\mathbb{Q}$ is contained in a cyclotomic extension. \hfill\hyperref[cor:quadincyclo2]{\downsym}
\end{restatable}

\begin{restatable}[]{prop}{quadgeneratorcyclo}
\label{prop:quadgeneratorcyclo}
    Let $p$ be an odd prime and $\mathbb{F} \subset \mathbb{Q}(\zeta_p)$ be a subfield such that $[\mathbb{Q}(\zeta_p) : \mathbb{F}] = 2.$ Then,
    \begin{equation*} 
        \mathbb{F} = \mathbb{Q}(\zeta_p + \zeta_p^{-1}).
    \end{equation*} \hfill\hyperref[prop:quadgeneratorcyclo2]{\downsym}
\end{restatable}

\begin{restatable}[]{prop}{fixedfieldcyclosubgroup}
\label{prop:fixedfieldcyclosubgroup}
    Let $p > 2$ be a prime number. Let $H$ be a subgroup of $G \vcentcolon= \Gal(\mathbb{Q}(\zeta_p)/\mathbb{Q}).$ Define
    \begin{equation*} 
        \beta \vcentcolon= \sum_{\sigma \in H} \sigma(\zeta_p).
    \end{equation*}
    Then,
    \begin{equation*} 
        \mathbb{Q}(\zeta_p)^H = \mathbb{Q}(\beta_H).
    \end{equation*} 
    \hfill\hyperref[prop:fixedfieldcyclosubgroup2]{\downsym}
\end{restatable}

\begin{ex}
    Let $p = 7$ and $\omega = \zeta_7.$ Then, $[\mathbb{Q}(\omega + \omega^{-1}) : \mathbb{Q}] = 3.$ Let us find the irreducible polynomial of $\omega + \omega^{-1}.$

    Note that the degree of this is $3.$ Since this is also the separable degree, we see that $\omega + \omega^{-1}$ has an orbit of size $3$ under $G \vcentcolon= \Gal(\mathbb{Q}(\omega)/\mathbb{Q}).$

    If $\{\beta_1, \beta_2, \beta_3\}$ is the orbit of $\omega$ under $G,$ then note that the polynomial
    \begin{equation*} 
        f(x) = (x - \beta_1) (x - \beta_2) (x - \beta_3)
    \end{equation*}
    is fixed by $G$ and hence, must be in $\mathbb{Q}[x].$ Since it is of the correct degree, it is the irreducible polynomial of $\omega + \omega^{-1}.$

    Thus, we now find the orbit. Note that $G \cong (\mathbb{Z}/7\mathbb{Z})^\times.$ The latter is generated by $\bar{3}.$ Thus, consider the automorphism $\sigma \in G$ determined by $\sigma(\omega) = \omega^3.$ Then, $G = \langle \sigma\rangle.$

    Now, we have
    \begin{align*} 
        \sigma(\omega + \omega^{-1}) &= \omega^3 + \omega^{-3} = \omega^3 + \omega^4 =\vcentcolon \beta_2\\
        \sigma^2(\omega + \omega^{-1}) &= \omega^9 + \omega^{-9} = \omega^2 + \omega^5 =\vcentcolon \beta_3.
    \end{align*}
    Since the above elements are distinct from $\omega + \omega^{-1} =\vcentcolon \beta_1,$ we have the orbit as
    \begin{equation*} 
        \{\beta_1, \beta_2, \beta_3\}.
    \end{equation*}
    Thus, we have
    \begin{equation*} 
        \irr(\alpha, \mathbb{Q}) = \prod_{i = 1}^{3}(x - \beta_i) = x^3 + x^2 - 2x - 1.
    \end{equation*}
\end{ex}
\chapter{Abelian and Cyclic extensions}

\section{Inverse Galois Problem}
The inverse Galois problem asks whether every finite group appears as the Galois group of some Galois extension of $\mathbb{Q}.$ This is currently unsolved. We prove this for finite abelian groups.

\begin{defn}%[]
	A Galois extension $\mathbb{E}/\mathbb{F}$ is called \deff{abelian} (resp., \deff{cyclic}) if $\Gal(\mathbb{E}/\mathbb{F})$ is abelian (resp., cyclic).
\end{defn}

\begin{restatable}[]{lem}{pequivonemodn}
\label{lem:pequivonemodn}
	Let $p$ be a prime number and $n$ be relatively prime to $p.$ Suppose $\bar{\Phi}_n(x)$ has a root in $\mathbb{F}_p.$ Then, $p \equiv 1 \pmod{n}.$ \hfill\hyperref[lem:pequivonemodn2]{\downsym}
\end{restatable}

\begin{restatable}[]{thm}{infprimesmodone}
\label{thm:infprimesmodone}
	Let $n \in \mathbb{N}.$ Then, there are infinitely many primes $p$ such that $p \equiv 1 \pmod{n}.$ \hfill\hyperref[thm:infprimesmodone2]{\downsym}
\end{restatable}

\begin{restatable}[]{thm}{fingroupQextension}
\label{thm:fingroupQextension}
	Let $G$ be a finite abelian group. Then, there exists an extension $\mathbb{K}/\mathbb{Q}$ such that $G \cong \Gal(\mathbb{K}/\mathbb{Q}).$ \hfill\hyperref[thm:fingroupQextension2]{\downsym}
\end{restatable}

In fact, there is a stronger version of the above theorem, which we do not prove.

\begin{thm}[Kronecker–Weber]
	Let $G$ be a finite abelian group. Then, there exists $n \in \mathbb{N}$ and a tower of fields
	\begin{equation*} 
		\mathbb{Q} \subset \mathbb{K} \subset \mathbb{Q}(\zeta_n)
	\end{equation*}
	such that $\Gal(\mathbb{K}/\mathbb{Q}) = n.$

	In other words, every finite abelian Galois extension of $\mathbb{Q}$ is contained in a cyclotomic extension.
\end{thm}

\section{Cyclic Galois Extensions}
\begin{defn}%[]
	Let $G$ be a group and $\mathbb{K}$ a field. A \deff{character of $G$ in $\mathbb{K}$} is a homomorphism $\chi : G \to \mathbb{K}^\times.$
\end{defn}

\begin{rem}
	Note that the set of all functions from $G$ to $\mathbb{K}$ is a vector space over $\mathbb{K}$ with point-wise operations. Thus, we can talk about linear independence of characters.
\end{rem}

\begin{restatable}[Dedekind]{thm}{dedekindcharacters}
\label{thm:dedekindcharacters}
	Let $\chi_1, \ldots, \chi_n : G \to \mathbb{K}^\times$ be distinct characters. Then, $\chi_1, \ldots, \chi_n$ are linearly independent. \hfill\hyperref[thm:dedekindcharacters2]{\downsym}
\end{restatable}

\begin{restatable}[]{lem}{primeigenvalue}
\label{lem:primeigenvalue}
	Let $n \in \mathbb{N}$ and $\mathbb{F}$ be a field containing a primitive $n$-th root of unity $\zeta.$ Suppose that $\mathbb{E}/\mathbb{F}$ is a cyclic Galois extension of degree $n$ with $G \vcentcolon= \Gal(\mathbb{E}/\mathbb{F}) = \langle \sigma\rangle.$ Then, $\zeta$ is an eigenvalue of the $\mathbb{F}$-linear map $\sigma.$ \hfill\hyperref[lem:primeigenvalue2]{\downsym}
\end{restatable}

\begin{restatable}[]{thm}{cyclicextprimroot}
\label{thm:cyclicextprimroot}
	Let $\mathbb{E}/\mathbb{F}$ be a cyclic Galois extension of degree $n.$ Let $G \vcentcolon= \Gal(\mathbb{E}/\mathbb{F}) = \langle \sigma\rangle$ and $\zeta \in \mathbb{F}$ be a primitive $n$-th root of unity. Then, there exists $a \in \mathbb{E}$ such that $\mathbb{E} = \mathbb{F}(a)$ and $a^n \in \mathbb{F}.$ \hfill\hyperref[thm:cyclicextprimroot2]{\downsym}
\end{restatable}

\begin{restatable}[]{prop}{subfieldsofprimcyclic}
\label{prop:subfieldsofprimcyclic}
	Let $\mathbb{E}/\mathbb{F}$ be a cyclic Galois extension of degree $n$ where $\mathbb{F}$ has a primitive $n$-th root of unity. Let $\mathbb{E} = \mathbb{F}(a),$ where $a \in \mathbb{E}$ is such that $a^n \in \mathbb{F},$ in view of \Cref{thm:cyclicextprimroot}.

	Then, the intermediate subfields of $\mathbb{E}/\mathbb{F}$ are $\mathbb{F}(a^d)$ where $d$ is a divisor of $n.$ \hfill\hyperref[prop:subfieldsofprimcyclic2]{\downsym}
\end{restatable}
\chapter{Some Group Theory}

Although already mentioned in \Cref{chap:00}, we repeat: $[n] \vcentcolon= \{1, \ldots, n\}$ for $n \in \mathbb{N}.$

\section{Solvable groups}

\begin{defn}%[]
	Let $G$ be a group. A sequence of subgroups
	\begin{equation*} 
		1 = G_0 \subset G_1 \subset \cdots \subset G_{s} = G
	\end{equation*}
	is called a \deff{normal series for $G$} if $G_i$ is a normal subgroup of $G_{i - 1}$ for $i = 1, \ldots, s.$ The \deff{length} of this series is $s.$ The normal series is called \deff{abelian} (resp., \deff{cyclic}) if the quotients $G_i/G_{i - 1}$ are abelian (resp., cyclic) for $i = 1, \ldots, s.$

	A group having an abelian series is called a \deff{solvable group}.
\end{defn}

\begin{rem}
	Note that the length is the number of inclusions, whereas there are $s + 1$ subgroups in the above series (including $1$ and $G$).
\end{rem}

\begin{ex}[Solvable groups]
	\phantom{hi}
	\begin{enumerate}
		\item Any abelian group $G$ is solvable with 
		\begin{equation*} 
			1 \unlhd G
		\end{equation*} 
		being an abelian series. In particular, so are $S_1$ and $S_2.$
		\item $S_3$ is solvable since 
		\begin{equation*} 
			1 \unlhd A_3 \unlhd S_3
		\end{equation*}
		is an abelian series. Indeed, $A_3$ is normal in $S_3$ since it has index $2$ and the quotient has order $2$ and hence, is abelian. Since $A_3$ has order $3,$ it is abelian; thus, $1 \unlhd A_3$ and $A_3/1$ is abelian.
		\item $S_4$ is solvable as well with
		\begin{equation*} 
			1 \unlhd V_4 \unlhd A_4 \unlhd S_4
		\end{equation*}
		being an abelian series. Here, $V_4 = \{1, (12)(34), (13)(24), (14)(23)\}.$

		We only need to verify that $V_4 \unlhd A_4.$ (The quotient will be abelian since it has order $3.$) That $V_4 \le A_4$ is clear since all the permutations are indeed even. Now, from the cycle type, we see that $V_4$ is actually normal in $S_4$ itself.
		%
		\item As we shall see later, $S_n$ is not solvable for $n \ge 5.$
	\end{enumerate}
\end{ex}

\begin{restatable}[]{prop}{pgroupssolvable}
\label{prop:pgroupssolvable}
	Any group with order $p^n$ is solvable, where $p$ is a prime and $n \in \mathbb{N}_0.$ \hfill\hyperref[prop:pgroupssolvable2]{\downsym}
\end{restatable}

\begin{defn}%[]
	Let $G$ be a group. The \deff{commutator} of $g, h \in G$ is defined as
	\begin{equation*} 
		[g, h] \vcentcolon= g^{-1}h^{-1}gh.
	\end{equation*}
	The \deff{derived subgroup} of $G$ denoted by $G'$ or $G^{(1)}$ or $[G, G]$ is the subgroup generated by all the commutators in $G.$ The \deff{$k$-th derived subgroup} of $G$ is defined inductively as $G^{(k)} = \left(G^{(k - 1)}\right)'$ for $k \ge 2.$
\end{defn}

\begin{rem}
	\phantom{hi}
	\begin{enumerate}
		\item $[g, h] = 1$ iff $g$ and $h$ commute.
		\item As a result, $G' = 1$ iff $G$ is abelian.
		\item If $H \le G,$ then $H' \le G'.$
		\item In general, the derived subgroup is \emph{generated} by commutators and is not equal to the set of commutators itself. (The smallest example is a certain group of order $96.$)
	\end{enumerate}
\end{rem}

\begin{defn}%[]
	Let $G$ be a group and $a \in G.$ Then, the \deff{inner automorphism $i_a$} is the automorphism $i_a \in \Aut(G)$ defined as
	\begin{equation*} 
		i_a(g) \vcentcolon= a^{-1}ga.
	\end{equation*}
\end{defn}

Clearly, $i_a$ is a homomorphism. To see that it an isomorphism, note that $i_{a^{-1}}$ is an inverse.

\begin{restatable}[]{prop}{commutatorresults}
\label{prop:commutatorresults}
	Let $f : G \to H$ be a homomorphism of groups and $s \in \mathbb{N}.$
	\begin{enumerate}
	 	\item $f(G^{(s)}) \le H^{(s)}.$ If $f$ is onto, then $f(G^{(s)}) = H^{(s)}.$
	 	\item If $K \unlhd G,$ then $K' \unlhd G.$ In particular, $G' \unlhd G.$
	 	\item If $K \unlhd G,$ then $G/K$ is abelian iff $G' \le K.$ \hfill\hyperref[prop:commutatorresults2]{\downsym}
	 \end{enumerate} 
\end{restatable}

\begin{rem}
	The last point essentially says that the derived subgroup is the smallest subgroup one must quotient by, to get an abelian group.
\end{rem}

\begin{restatable}[]{prop}{solvableifftrivialderiv}
\label{prop:solvableifftrivialderiv}
	A group $G$ is solvable iff $G^{(s)} = 1$ for some $s \in \mathbb{N}.$ \hfill\hyperref[prop:solvableifftrivialderiv2]{\downsym}
\end{restatable}

\begin{restatable}[]{prop}{deriveofquotient}
\label{prop:deriveofquotient}
	Let $K \unlhd G$ be groups. Then,
	\begin{equation*} 
		\left(\frac{G}{K}\right)^{(s)} = \frac{\langle G^{(s)}, K\rangle}{K}.
	\end{equation*}
	\hfill\hyperref[prop:deriveofquotient2]{\downsym}
\end{restatable}

\begin{restatable}[]{prop}{twoofthreesolvable}
\label{prop:twoofthreesolvable}
	Let $G$ and $H$ be groups.
	\begin{enumerate}
	 	\item If $G$ is solvable and there is an injection $i : H \to G,$ then $H$ is solvable. In particular, subgroups of solvable groups are solvable.
	 	\item If $G$ is solvable and there is a surjection $f : G \to H,$ then $H$ is solvable. In particular, quotients of solvable groups are solvable.
	 	\item If $K \unlhd G$ is such that $K$ and $G/K$ are solvable, then $G$ is solvable. \hfill\hyperref[prop:twoofthreesolvable2]{\downsym}
	\end{enumerate} 
\end{restatable}

\begin{rem}
	For those familiar with the lingo, the above proposition says:

	Let $0 \to H \to G \to K \to 0$ be an exact sequence of groups. Then, $G$ is solvable iff $H$ and $K$ are solvable.
\end{rem}

\begin{restatable}[]{prop}{refiningabelianseries}
\label{prop:refiningabelianseries}
	Let $G$ be a finite solvable group. Then, there exists a normal series
	\begin{equation*} 
		1 = G_0 \unlhd G_1 \unlhd \cdots \unlhd G_s = G
	\end{equation*}
	such that $G_i/G_{i - 1}$ is cyclic of prime order for all $i = 1, \ldots, s.$ \hfill\hyperref[prop:refiningabelianseries2]{\downsym}
\end{restatable}

\section{Some results about Symmetric Groups}

\begin{restatable}[]{lem}{Angenerator}
\label{lem:Angenerator}
	For $n \ge 3,$ $A_n$ is generated by $3$-cycles. If $n \ge 5,$ then all the $3$-cycles are conjugates in $A_n.$ \hfill\hyperref[lem:Angenerator2]{\downsym}
\end{restatable}

\begin{restatable}[]{thm}{SnAnnotsolvable}
\label{thm:SnAnnotsolvable}
	The groups $S_n$ and $A_n$ are not solvable for $n \ge 5.$ \hfill\hyperref[thm:SnAnnotsolvable2]{\downsym}
\end{restatable}

\begin{restatable}[]{thm}{Ansimple}
\label{thm:Ansimple}
	The alternating group $A_n$ is simple for $n \ge 5.$ \hfill\hyperref[thm:Ansimple2]{\downsym}
\end{restatable}

\subsection{Generators of Symmetric Groups}

Of course, everyone knows the first one.

\begin{thm} \label{thm:gentranspose}
	For $n \ge 2,$ $S_n$ is generated by its transpositions.
\end{thm}

\begin{restatable}[]{thm}{genconsectranpose}
\label{thm:genconsectranpose}
	For $n \ge 2,$ $S_n$ is generated by the $n - 1$ transpositions
	\begin{equation*} 
		(12), (13), \ldots, (1n).
	\end{equation*} \hfill\hyperref[thm:genconsectranpose2]{\downsym}
\end{restatable}

\begin{restatable}[]{thm}{genconsectranposespecial}
\label{thm:genconsectranposespecial}
	For $n \ge 2,$ $S_n$ is generated by the $n - 1$ transpositions
	\begin{equation*} 
		(1 \ 2), (2 \ 3), \ldots, (n - 1 \ n).
	\end{equation*}  \hfill\hyperref[thm:genconsectranposespecial2]{\downsym}
\end{restatable}

\begin{restatable}[]{thm}{gentransposecycle}
\label{thm:gentransposecycle}
	For $n \ge 2,$ $S_n$ is generated by the transposition $(12)$ and the $n$-cycle $(12 \ldots n).$ \hfill\hyperref[thm:gentransposecycle2]{\downsym}
\end{restatable}

\begin{restatable}[]{cor}{genprimetranscycle}
\label{cor:genprimetranscycle}
	Let $p \ge 3$ be a prime. Then, $S_p$ is generated by any pair of transposition and $p$-cycle. \hfill\hyperref[cor:genprimetranscycle2]{\downsym}
\end{restatable}

\begin{rem}
	In general, it is not true that any transposition and $n$-cycle generates $S_n.$ For example, $(12)$ and $(1234)$ do not generate $S_4.$ To see this, consider the dihedral group $D_8$ of order $8$ as a subgroup of $S_4$ by numbering the vertices of a square as $1, 2, 3, 4.$ Then, $(12), (1234) \in D_8 \subsetneq S_4$ and thus, $\langle (12), (1234)\rangle \subset D_8 \subsetneq S_4.$
\end{rem}
\chapter{Galois Groups of Composite Extensions}

In this section, $\mathbb{F}$ be a field and $\overline{\mathbb{F}}$ some fixed algebraic closure of $\mathbb{F}.$ Whenever we talk about extensions $\mathbb{E}/\mathbb{F}$ and $\mathbb{K}/\mathbb{F},$ it will be understood that $\mathbb{E}, \mathbb{K} \subset \overline{\mathbb{F}}.$ In particular, it makes sense to talk about $\mathbb{E}\mathbb{K}$ and $\mathbb{E} \cap \mathbb{K}.$

\begin{restatable}[]{prop}{galoisEFK}
\label{prop:galoisEFK}
	If $\mathbb{E}/\mathbb{F}$ is a Galois extension and $\mathbb{K}/\mathbb{F}$ is a field extension, then $\mathbb{E}\mathbb{K}/\mathbb{K}$ is Galois. Moreover, if $\mathbb{F}/\mathbb{K}$ is also Galois, then $\mathbb{E}\mathbb{K}/\mathbb{F}$ and $(\mathbb{E} \cap \mathbb{K})/\mathbb{F}$ are Galois. \hfill\hyperref[prop:galoisEFK2]{\downsym}

	\begin{center}
		\begin{tikzcd}[ampersand replacement=\&]
		\& \overline{\mathbb{F}} \arrow[dd, no head] \& \\
		\& \& \\
                                                                       \& \mathbb{E}\mathbb{K} \arrow[rd, no head] \arrow[ld, no head]                          \&            \\
\mathbb{E} \arrow[rddd, "\text{Galois}"', no head, dashed, bend right] \&                                                                                        \& \mathbb{K} \\
                                                                       \& \mathbb{E} \cap \mathbb{K} \arrow[dd, no head] \arrow[ru, no head] \arrow[lu, no head] \&            \\
                                                                       \&                                                                                        \&            \\
                                                                       \& \mathbb{F}                                                                             \&           
	\end{tikzcd}
	\end{center}
\end{restatable}

\begin{restatable}[]{prop}{secondiso}
\label{prop:secondiso}
	Let $\mathbb{E}/\mathbb{F}$ be a finite Galois extension and $\mathbb{K}/\mathbb{F}$ be a field extension (with $\mathbb{E}, \mathbb{K} \subset \overline{\mathbb{F}}$). Then, the map
	\begin{equation*} 
		\psi : \Gal(\mathbb{E}\mathbb{K}/\mathbb{K}) \to \Gal(\mathbb{E}/\mathbb{F})
	\end{equation*}
	defined by $\psi(\sigma) = \sigma|_{\mathbb{E}}$ is injective and induces an isomorphism
	\begin{equation*} 
		\Gal(\mathbb{E}\mathbb{K}/\mathbb{K}) \cong \Gal(\mathbb{E}/\mathbb{E} \cap \mathbb{K}).
	\end{equation*} \hfill\hyperref[prop:secondiso2]{\downsym}

	\begin{center}
		\begin{tikzcd}[ampersand replacement=\&]
		                               \& \mathbb{E}\mathbb{K} \arrow[rd, red, no head]       \&                                \\
		\mathbb{E} \arrow[ru, no head] \&                                                \& \mathbb{K} \arrow[ld, no head] \\
		                               \& \mathbb{E} \cap \mathbb{K} \arrow[lu, red, no head] \&                               
		\end{tikzcd}
	\end{center}
\end{restatable}

\begin{restatable}[]{cor}{secondisoindex}
\label{cor:secondisoindex}
	Let $\mathbb{E}/\mathbb{F}$ be a finite Galois extension and $\mathbb{K}/\mathbb{F}$ any field extension. Then,
	\begin{equation*} 
		[\mathbb{E}\mathbb{K} : \mathbb{K}] = [\mathbb{E} : \mathbb{E} \cap \mathbb{K}].
	\end{equation*}
	In particular, $[\mathbb{E}\mathbb{K} : \mathbb{F}] = [\mathbb{E} : \mathbb{F}][\mathbb{K} : \mathbb{F}]$ iff $\mathbb{E} \cap \mathbb{K} = \mathbb{F}.$ \hfill\hyperref[cor:secondisoindex2]{\downsym}
\end{restatable}

\begin{restatable}[]{thm}{galoiscompositeproduct}
\label{thm:galoiscompositeproduct}
	Let $\mathbb{E}/\mathbb{F}$ and $\mathbb{K}/\mathbb{F}$ be finite Galois extensions with $\mathbb{E}, \mathbb{K} \subset \overline{\mathbb{F}}.$ Then, the homomorphism
	\begin{equation*} 
		\psi : \Gal(\mathbb{E}\mathbb{K}/\mathbb{F}) \to \Gal(\mathbb{E}/\mathbb{F}) \times \Gal(\mathbb{K}/\mathbb{F}), \quad \psi(\sigma) = (\sigma|_{\mathbb{E}}, \sigma|_{\mathbb{K}})
	\end{equation*}
	is injective. If $\mathbb{E} \cap \mathbb{K} = \mathbb{F},$ then $\psi$ is an isomorphism. \hfill\hyperref[thm:galoiscompositeproduct2]{\downsym}
\end{restatable}
\chapter{Normal Closure of an Algebraic Extension}

\begin{defn}%[]
	\label{defn:normalclosure}
	Let $\mathbb{E}/\mathbb{F}$ be an algebraic extension and $\mathbb{E} \subset \overline{\mathbb{F}}.$ The \deff{normal closure of $\mathbb{E}/\mathbb{F}$ in $\overline{\mathbb{F}}$} is the splitting field $\mathbb{K}$ over $\mathbb{F}$ of the polynomials $\{\irr(\alpha, \mathbb{F}) \mid \alpha \in \mathbb{E}\}.$
\end{defn}

\begin{restatable}[]{prop}{normalclosureproperties}
\label{prop:normalclosureproperties}
	Let the notations be as in \Cref{defn:normalclosure}. The following are true.
	\begin{enumerate}
	 	\item $\mathbb{K}$ is a normal extension of $\mathbb{F}$ containing $\mathbb{E}.$
	 	\item Any such normal extension $\mathbb{K}' \subset \overline{\mathbb{F}}$ as above contains $\mathbb{K}.$
	 	\item If $\mathbb{E}/\mathbb{F}$ is a finite extension, then so is $\mathbb{K}/\mathbb{F}.$
	 	\item If $\mathbb{E}/\mathbb{F}$ is separable, then $\mathbb{K}/\mathbb{F}$ is Galois.
	 	\item Suppose $\mathbb{E}/\mathbb{F}$ is separable and not normal. Suppose $H \le \Gal(\mathbb{K}/\mathbb{E}) \le \Gal(\mathbb{K}/\mathbb{F}) =\vcentcolon G$ is normal in $G.$ Then, $H = 1.$ \hfill\hyperref[prop:normalclosureproperties2]{\downsym}
	 \end{enumerate} 
\end{restatable}
\chapter{Solvability by Radicals}

\section{Radical extensions}

\begin{defn}%[]
	A field extension $\mathbb{K}/\mathbb{F}$ is called a \deff{simple radical extension} if $\mathbb{K} = \mathbb{F}(a)$ and $a^n \in \mathbb{F}$ for some $a \in \mathbb{K}$ and some $n \in \mathbb{N}.$

	We say that $\mathbb{K}/\mathbb{F}$ is a \deff{radical extension} if there is a sequence of field extensions
	\begin{equation*} 
		\mathbb{F} = \mathbb{F}_0 \subset \mathbb{F}_1 \subset \cdots \subset \mathbb{F}_n = \mathbb{K}
	\end{equation*}
	such that $\mathbb{F}_i/\mathbb{F}_{i - 1}$ is a simple radical extension for $i = 1, \ldots, n.$

	A polynomial $f(x) \in \mathbb{F}[x]$ is called \deff{solvable by radicals over $\mathbb{F}$} if a splitting field of $f(x)$ over $\mathbb{F}$ is contained in a radical extension of $\mathbb{F}.$
\end{defn}

\begin{rem}
	Note that radical extensions are finite extensions.
\end{rem}

\begin{restatable}[]{prop}{radextproperties}
\label{prop:radextproperties}
	Let $\mathbb{F}, \mathbb{E}, \mathbb{K} \subset \overline{\mathbb{F}}$ be fields. 
	\begin{enumerate}
		\item Suppose $\mathbb{F} \subset \mathbb{E} \subset \mathbb{K}.$ If $\mathbb{E}/\mathbb{F}$ and $\mathbb{K}/\mathbb{E}$ are radical extensions, then so is $\mathbb{K}/\mathbb{F}.$ 
		\item Suppose $\mathbb{F} \subset \mathbb{E}, \mathbb{K}$ are such that $\mathbb{E}/\mathbb{F}$ is a radical extension. Then, $\mathbb{E}\mathbb{K}/\mathbb{K}$ is a radical extension. If $\mathbb{K}/\mathbb{F}$ is also a radical extension, then so is $\mathbb{E}\mathbb{K}/\mathbb{F}.$ \hfill\hyperref[prop:radextproperties2]{\downsym}
	\end{enumerate}
\end{restatable}

\begin{restatable}[]{prop}{sepgaloisradical}
\label{prop:sepgaloisradical}
	Let $\mathbb{E}/\mathbb{F}$ be a separable radical extension. Let $\mathbb{K} \subset \overline{\mathbb{F}}$ be the smallest Galois extension of $\mathbb{E}$ containing $\mathbb{E}.$ Then, $\mathbb{K}$ is a radical extension of $\mathbb{F}.$ \hfill\hyperref[prop:sepgaloisradical2]{\downsym}
\end{restatable}

Note that the $\mathbb{K}$ above is simply the normal closure.

\section{Solvability Criterion}

\begin{restatable}[]{thm}{solvradicalimpliesgroup}
\label{thm:solvradicalimpliesgroup}
	Let $\mathbb{F}$ be a field with $\chr(\mathbb{F}) = 0.$ If $f(x) \in \mathbb{F}[x]$ is solvable by radicals, then $\G_f$ is a solvable group.  \hfill\hyperref[thm:solvradicalimpliesgroup2]{\downsym}
\end{restatable}

\begin{ex}[Quintic not solvable by radicals]
	Suppose $f(x) \in \mathbb{Q}[x]$ is an irreducible quintic (degree five) polynomial which has exactly $3$ roots. Let $\mathbb{E} = \mathbb{Q}(a) \subset \mathbb{C}$ be a splitting field of $f(x)$ over $\mathbb{Q}.$ Any $\sigma \in \G_f$ will permute the roots of $f(x)$ and thus, we can identify $\G_f$ with a subgroup of $S_5.$

	Then, $\G_f \cong \Gal(\mathbb{E}/\mathbb{Q})$ has order divisible by $5.$ Thus, $\G_f$ contains an element of order $5$ and thus, a $5$-cycle.

	On the other hand, the automorphism is a non-trivial automorphism of order $2.$ Thus, $\G_f$ contains a $5$-cycle and a transposition. By \Cref{cor:genprimetranscycle}, we have $\G_f = S_5.$

	By \Cref{thm:SnAnnotsolvable}, we see that $\G_f$ is not solvable and thus, $f(x)$ is not solvable by radicals over $\mathbb{Q}.$

	Such an $f(x)$ does indeed exist. For example, consider
	\begin{equation*} 
		f(x) \vcentcolon= x^5 - 16x + 2.
	\end{equation*}
	$f(x)$ is irreducible by Eisenstein at $2.$ Elementary calculus techniques show that $f(x)$ has exactly $3$ real roots.
\end{ex}

\begin{restatable}[]{thm}{solvgroupimpliesradical}
\label{thm:solvgroupimpliesradical}
	Let $\mathbb{F}$ be a field with $\chr(\mathbb{F}) = 0$ and $f(x) \in \mathbb{F}[x].$ If $\G_f$ is a solvable group, then $f(x)$ is solvable by radicals. \hfill\hyperref[thm:solvgroupimpliesradical2]{\downsym}
\end{restatable}

Putting \Cref{thm:solvradicalimpliesgroup} and \Cref{thm:solvgroupimpliesradical} together, we get the following.

\begin{thm}[Solvability via radicals]
	Let $\mathbb{F}$ be a field with $\chr(\mathbb{F}) = 0$ and $f(x) \in \mathbb{F}[x].$ $f(x)$ is solvable by radicals if and only if $\G_f$ is a solvable group. 
\end{thm}

\begin{ex}
	Note that ``solvable by radicals'' does not necessarily mean that the splitting field is a radical extension.

	Consider the polynomial $f(x) = x^3 - 3x + 1 \in \mathbb{Z}[x].$ Reducing modulo $2,$ we see that polynomial is irreducible since it has no root in $\mathbb{F}_2.$ Thus, $f(x)$ is irreducible in $\mathbb{Z}[x]$ and in turn, over $\mathbb{Q}[x].$

	Let $\mathbb{E}$ be a splitting field of $f(x)$ over $\mathbb{Q}.$ We show that $\mathbb{E}$ is not a radical extension of $\mathbb{Q}.$
	Note that $\disc(f(x)) = 81$ and thus, $\G_f \cong A_3,$ by \Cref{ex:galsepcubic}. Thus, $[\mathbb{E} : \mathbb{Q}] = 3.$ Let $r$ be a real root of $f(x).$ Then, we may assume that $\mathbb{E} = \mathbb{Q}(r),$ by consideration of degree. In particular, $\mathbb{E} \subset \mathbb{R}.$ 

	Now, for the sake of contradiction, suppose that $\mathbb{E}/\mathbb{Q}$ is a radical extension. Since $3$ is prime, there is no proper intermediate subfield of $\mathbb{E}/\mathbb{Q}.$ This means that $\mathbb{E}$ itself is a simple radical extension over $\mathbb{Q}.$

	Let $\mathbb{E} = \mathbb{Q}(a)$ where $a^n \in \mathbb{Q}$ for some $n \in \mathbb{N}.$ Let $g(x) \vcentcolon= \irr(a, \mathbb{Q}).$ Then, $\mathbb{E}$ is a splitting field of $g(x)$ over $\mathbb{Q}.$ Moreover, $g(x) \mid (x^n - a^n) \in \mathbb{Q}[x].$ Thus, every root $b \in \mathbb{E}$ of $g(x)$ satisfies $b^n = a^n$ or $(b/a)^n = 1.$ Note that $b, a \in \mathbb{E} \subset \mathbb{R}.$ But there are at most $2$ roots of unity in $\mathbb{R}$ and hence, $g(x)$ has at most $2$ roots in $\mathbb{E}.$ This is a contradiction since $g(x)$ is a separable cubic and $\mathbb{E}$ is its splitting field.
\end{ex}
\chapter{Solutions of Cubic and Quartic equations} \label{chap:solutionscubicandquartics}

In this chapter, we assume that $\mathbb{F}$ is a field of characteristic different from $2$ or $3.$ We shall describe algorithms for solving an arbitrary cubic and quartic polynomials over $\mathbb{F}$ in terms of radicals.

\section{Cubics}

Consider a cubic of the form $f(x) \vcentcolon= x^3 + px + q \in \mathbb{F}[x].$ (Note that we can assume any cubic to be of this form since we can always kill the square term by ``completing the cube'' and then scale to make the leading coefficient unity.)

Now, we introduce two new variables $u$ and $v.$ We will get our roots to be of the form $u + v.$ \\
We expand the equation $f(u + v) = 0$ to get
\begin{equation*} 
	u^3 + v^3 + q + (3uv + p)(u + v) = 0.
\end{equation*}

We now set
\begin{equation} \label{eq:008}
	u^3 + v^3 + q = 0
\end{equation}
and
\begin{equation} \label{eq:009}
	3uv + p = 0.
\end{equation}

From \Cref{eq:009}, we have $uv = -p/3.$ Multiplying \Cref{eq:008} with $u^3$ and using $uv = -p/3$ gives
\begin{equation*} 
	u^6 + qu^3 - p^3/27 = 0.
\end{equation*}

The above is a quadratic in $u^3.$ Put $D = -(4p^3 + 27q^2).$ (Recall that this is the discriminant! \Cref{ex:disccubic}.) Bu the quadratic formula, we get
\begin{equation*} 
	u^3 = \frac{-q \pm \sqrt{q^2 + (4p^3/27)}}{2} = -\frac{q}{2} \pm \sqrt{-\frac{D}{108}}.
\end{equation*}

By symmetry, in $u$ and $v,$ we set
\begin{equation*} 
	A \vcentcolon= -\frac{q}{2} + \sqrt{-\frac{D}{108}} = u^3 \andd B \vcentcolon= -\frac{q}{2} - \sqrt{-\frac{D}{108}} = v^3.
\end{equation*}

Let $\omega$ be a primitive cube root of unity. Thus, we see that the possible values of $u$ and $v$ are given as
\begin{equation*} 
	u = \sqrt[3]{A},\, \omega\sqrt[3]{A},\, \omega^2\sqrt[3]{A}, \andd v = \sqrt[3]{B},\, \omega\sqrt[3]{B},\, \omega^2\sqrt[3]{B}.
\end{equation*}

However, we cannot choose $u$ and $v$ independently. We need to ensure that $uv = -p/3.$ 

First, choose cube roots $\sqrt[3]{A}$ and $\sqrt[3]{B}$ such that $\sqrt[3]{A}\sqrt[3]{B} = -p/3.$ (The reason we can do this is because $AB = -p^3/27.$)

Then, the three roots of $f(x)$ are seen to be 
\begin{equation*} 
	\sqrt[3]{A} + \sqrt[3]{B},\,\omega\sqrt[3]{A} + \omega^2\sqrt[3]{B},\,\omega^2\sqrt[3]{A} + \omega\sqrt[3]{B}.
\end{equation*}

\begin{ex}[Negative discriminant]
	Suppose $f(x) = x^3 + px + q \in \mathbb{R}[x]$ with $\disc(f(x)) < 0.$ In this case, $A$ and $B$ are real. Moreover, we can choose the cube roots of $A$ and $B$ to be real. We get the roots as
	\begin{align*} 
		r_1 &= \sqrt[3]{A} + \sqrt[3]{B} \in \mathbb{R}, \\
		r_2 &= -\frac{\sqrt[3]{A} + \sqrt[3]{B}}{2} + \iota\sqrt{3}\left(\frac{\sqrt[3]{A} - \sqrt[3]{B}}{2}\right), \\
		r_3 &= \overline{r_2}.
	\end{align*}
	Note that the roots are distinct. This can be seen by either observing that $A \neq B$ or that $\disc(f(x)) \neq 0.$ 
\end{ex}

\begin{ex}[Positive discriminant]
	Suppose $f(x) = x^3 + px + q \in \mathbb{R}[x]$ with $\disc(f(x)) > 0.$ Then, we have
	\begin{equation*} 
		A = -\frac{q}{2} + \iota\sqrt{\frac{D}{108}} \andd B = \overline{A}.
	\end{equation*}
	Let $a + \iota b$ be a cube root of $\sqrt[3]{A}.$ Then, since $B = \overline{A},$ we know the cube roots of $B.$ Since we wish the product to be $-p/3 \in \mathbb{R},$ we pick $\sqrt[3]{B} = a - \iota b.$ Thus, the roots are
	\begin{align*} 
		r_1 &= 2a, \\
		r_2 &= -a - b\sqrt{3}, \\
		r_3 &= -a + b\sqrt{3}.
	\end{align*}
	In particular, all the roots are real and distinct.
\end{ex}

\section{Quartics}

As before, it suffices to consider a polynomial of the form
\begin{equation*} 
	g(y) = y^4 + py^2 + qy + r \in \mathbb{F}[y].
\end{equation*}
Let $r_1, \ldots, r_4$ be the roots of $g(y).$ Consider the following quantities
\begin{equation*} 
	\theta_1 \vcentcolon= (r_1 + r_2)(r_3 + r_4),\, \theta_2 \vcentcolon= (r_1 + r_3)(r_2 + r_4),\, \theta_3 \vcentcolon= (r_1 + r_4)(r_2 + r_3).
\end{equation*}

Now, note that we compute the elementary symmetric polynomials in $\theta_i$ since these will be elementary symmetric polynomials in $r_j$ and we already know those in terms of $p, q, r.$ In particular, we may compute the monic cubic polynomial having $\theta_1, \theta_2, \theta_3$ as roots. This is called the \deff{resolvent cubic} of $g(y).$ This turns out to be
\begin{equation*} 
	h(x) \vcentcolon= x^3 - 2px^2 + (p^2 - 4r)x + q^2.
\end{equation*}

Using the relation $r_1 + r_2 + r_3 + r_4 = 0,$ we get 
\begin{equation*} 
	(r_1 + r_2)^2 = (r_3 + r_4)^2 = -\theta_1
\end{equation*} 
and so on. Fixing a square root for each $-\theta_i,$ we get.
\begin{align*} 
	r_1 + r_2 &= \sqrt{-\theta_1}, \quad r_3 + r_4 = -\sqrt{-\theta_1}, \\
	r_1 + r_3 &= \sqrt{-\theta_2}, \quad r_2 + r_4 = -\sqrt{-\theta_2}, \\
	r_1 + r_4 &= \sqrt{-\theta_3}, \quad r_2 + r_3 = -\sqrt{-\theta_3}.
\end{align*}
One can show that the product of the elements on the left is $-q,$ i.e., the choice of square roots must satisfy
\begin{equation*} 
	\sqrt{-\theta_1}\sqrt{-\theta_2}\sqrt{-\theta_3} = -q.
\end{equation*}
Thus, two of the square roots determine the third. Now, using the relation $r_2 + r_3 + r_4 = -r_1,$ adding the four equations on the left lead to the following solutions.
\begin{align*} 
	2r_1 =  \sqrt{-\theta_1} + \sqrt{-\theta_2} + \sqrt{-\theta_3}, \\
	2r_2 =  \sqrt{-\theta_1} - \sqrt{-\theta_2} - \sqrt{-\theta_3}, \\
	2r_3 = -\sqrt{-\theta_1} + \sqrt{-\theta_2} - \sqrt{-\theta_3}, \\
	2r_4 = -\sqrt{-\theta_1} - \sqrt{-\theta_2} + \sqrt{-\theta_3}.
\end{align*}

Thus, the roots of the resolvent cubic determine the roots of the quartic. 

\begin{prop}
	The discriminants of the quartic $g(y)$ and its resolvent $h(x)$ are equal.
\end{prop}
\begin{proof} 
	The differences of roots are 
	\begin{equation*} 
		\theta_1 - \theta_2 = (r_2 - r_3)(r_4 - r_1),\, \theta_1 - \theta_3 = (r_2 - r_4)(r_3 - r_1),\, \theta_2 - \theta_3 = (r_3 - r_4)(r_2 - r_1).
	\end{equation*}	
	It is now clear that the discriminants are equal.
\end{proof}
\chapter{Galois Groups of Quartic Polynomials}

\section{Galois group as a group of permutations}

In this chapter, we shall frequently consider the Galois group of a separable polynomial of degree $n$ as a subgroup of $S_n.$ To recall how this is done: Let $f(x) \in \mathbb{F}[x]$ be a monic separable polynomial with (distinct) roots $r_1, \ldots, r_n \in \overline{\mathbb{F}}$ in a splitting field $\mathbb{E} = \mathbb{F}(r_1, \ldots, r_n).$ Let $G \vcentcolon= \Gal(\mathbb{E}/\mathbb{F})$ be its Galois group. Note that any $\sigma \in G$ is a permutation of $R = \{r_1, \ldots, r_n\}.$ Identifying $R$ with $[n],$ we see that $\sigma|_R \in S_n.$ \\
Define $\psi : G \to S_n$ by $\sigma \mapsto \sigma|_R.$ This is an injective homomorphism since $\sigma$ is completely determined by its action on $R$ since $\mathbb{E} = \mathbb{F}(R).$ We denote the image of $\psi$ by $\G_f,$ the Galois group of $f(x).$

By \hyperref[thm:FTGT]{FTGT}, there is an intermediate subfield of $\mathbb{E}/\mathbb{F}$ corresponding to $\G_f \cap A_n.$

\begin{restatable}[]{thm}{alternatingsubgroupdiscriminantroot}
\label{thm:alternatingsubgroupdiscriminantroot}
	Let $\mathbb{F}$ be a field with $\chr(\mathbb{F}) \neq 2$ and $f(x) \in \mathbb{F}[x],$ a monic separable polynomial with (distinct) roots $r_1, \ldots, r_n \in \overline{\mathbb{F}}.$ Put $\mathbb{E} = \mathbb{F}(r_1, \ldots, r_n)$ and 
	\begin{equation*} 
		\delta = \prod_{1 \le i < j \le n} (r_i - r_j).
	\end{equation*}
	Then, $E^{\G_f \cap A_n} = \mathbb{F}(\delta).$ \hfill\hyperref[thm:alternatingsubgroupdiscriminantroot2]{\downsym}
\end{restatable}

\begin{defn}
	A subgroup $H \le S_n$ is called a \deff{transitive subgroup} if $H$ acts transitively on $\{1, \ldots, n\}.$ \\
	In other words, given any $i, j \in \{1, \ldots, n\},$ there exists $\sigma \in H$ with $\sigma(i) = j.$
\end{defn}

\begin{restatable}[]{thm}{transitivegaloisgroupiffirreducible}
\label{thm:transitivegaloisgroupiffirreducible}
	Let $f(x) \in \mathbb{F}[x]$ be a separable polynomial of degree $n.$ Then, $f(x)$ is irreducible if and only if $\G_f$ is a transitive subgroup of $S_n.$ \hfill\hyperref[thm:transitivegaloisgroupiffirreducible2]{\downsym}
\end{restatable}

\section{Transitive subgroups of \texorpdfstring{$S_4$}{S4}}

Let $H \le S_n$ be a transitive subgroup. Then, there is only one orbit of $H$ on $[n].$ In particular, this orbit has order $n.$ By the orbit-stabiliser theorem, it follows that $n \mid \md{H}.$

By \Cref{thm:transitivegaloisgroupiffirreducible}, the orders of possible Galois groups of irreducible separable quartics are $4,$ $8,$ $12,$ and $24.$ These groups are listed below.

\begin{enumerate}
	\item Isomorphic to $C_4.$ \\
	These are the groups generated by an element of order $4.$ Since we are in $S_4,$ these are the groups generated by a $4$-cycle. There are six $4$-cycles in $S_4$ and in turn, there are three subgroups of $S_4$ isomorphic to $C_4.$
	%
	\item Isomorphic to $V,$ the Klein-$4$ group. \\
	This must contain three elements of order $2.$ Thus, it is forced to be 
	\begin{equation*} 
		V = \{(1), (12)(34), (13)(24), (14)(23)\}.
	\end{equation*}
	Looking at the cycle types, we see that $V \unlhd S_4.$
	%
	\item Order $8.$ This is a Sylow $2$-subgroup of $S_4$ and thus, all of these are isomorphic. The isomorphism type turns out to be that of $D_8.$ \\
	These are $H_1 = \langle V, (12)\rangle,$ $H_2 = \langle V, (13)\rangle,$ and $H_3 = \langle V, (14)\rangle.$
	%
	\item $A_4$ is the only subgroup of order $12$ in $S_4$ and $A_4 \unlhd S_4.$
	%
	\item $S_4$ is the only subgroup of order $24$ in $S_4.$
\end{enumerate}

\section{Calculation of Galois group of quartic polynomials}

Let $\mathbb{F}$ be a field of characteristic not $2.$ Let $f(x) = x^4 + b_1x^3 + b_2x^2 + b_3x + b_4 \in \mathbb{F}[x]$ be separable. By the change $x' = x + \frac{b_1}{4},$ we may assume that there is no $x^3$ term. This change only changes the roots of $f(x)$ by addition of a constant. Thus, the discriminant is unchanged. Moreover, the constant is in $\mathbb{F}$ and thus, the splitting field is unchanged and hence, so is the Galois group. \\
So, let $f(x) = x^4 + bx^2 + cx + d \in \mathbb{F}[x]$ be a separable polynomial with roots $r_1, \ldots, r_4$ in a splitting field $\mathbb{E}$ of $f(x)$ over $\mathbb{F}.$ As before, we consider $\G_f \le S_4.$ Set
\begin{equation*} 
	\underline{t} \vcentcolon= \{t_1 = r_1r_2 + r_3r_4,\, t_2 = r_1r_3 + r_2r_4,\, t_3 = r_1r_4 + r_2r_3\}.
\end{equation*}

\begin{defn}
	The monic cubic having $t_1, t_2, t_3$ as roots is called the \deff{resolvent} of $f(x).$
\end{defn}
\begin{rem}
	We had defined resolvent in \Cref{chap:solutionscubicandquartics} in a different manner. For this chapter, we shall use the above definition. 

	As earlier, it can be shown that the resolvent is actually an element of $\mathbb{F}[x]$ and is explicitly given as
	\begin{equation*} 
		x^3 - bx^2 + 4dx + 2bd - c^2.
	\end{equation*}

	By computing the differences $t_i - t_j,$ it is also clear that the $f(x)$ has the same discriminant as its resolvent.
\end{rem}

Also, recall that there is a unique subgroup of $S_4$ isomorphic to the Klein-$4$ group. We denote it by $V.$ Moreover, $V \unlhd S_4.$ It is also visible that $V$ fixes each element of $\underline{t}.$

Lastly, define as before $H_1 = \langle V, (12)\rangle,$ $H_2 = \langle V, (13)\rangle,$ and $H_3 = \langle V, (14)\rangle.$

\begin{restatable}[]{prop}{stabiliserofti}
\label{prop:stabiliserofti}
	$\Stab t_i = H_i.$ \hfill\hyperref[prop:stabiliserofti2]{\downsym}
\end{restatable}

\begin{restatable}[]{prop}{galoisintersectklein}
\label{prop:galoisintersectklein}
	$\mathbb{E}^{\G_f \cap V} = \mathbb{F}(\underline{t})$ and $\Gal(\mathbb{F}(\underline{t})/\mathbb{F}) = \G_f/\G_f \cap V.$ \hfill\hyperref[prop:galoisintersectklein2]{\downsym}
\end{restatable}

\begin{restatable}[]{prop}{resolventquarticrootinF}
\label{prop:resolventquarticrootinF}
	The resolvent cubic of a separable quartic has a root in $\mathbb{F}$ if and only if $\G_f \subset H_i$ for some $i.$ \hfill\hyperref[prop:resolventquarticrootinF2]{\downsym}
\end{restatable}

\begin{restatable}[]{thm}{classifyingirreduciblequarticgalois}
\label{thm:classifyingirreduciblequarticgalois}
	Let $f(x) \in \mathbb{F}[x]$ an irreducible separable quartic with $\chr(\mathbb{F}) \neq 2.$ Let $r(x)$ denote the resolvent cubic of $f(x).$
	\begin{enumerate}
		\item If $r(x)$ is irreducible in $\mathbb{F}[x]$ and $\disc(f(x)) \notin \mathbb{F}^2,$ then $\G_f \cong S_4.$
		\item If $r(x)$ is irreducible in $\mathbb{F}[x]$ and $\disc(f(x)) \in \mathbb{F}^2,$ then $\G_f \cong A_4.$
		\item If $r(x)$ splits completely in $\mathbb{F}[x],$ then $\G_f \cong V.$
		\item Suppose $r(x)$ has exactly one root in $\mathbb{F}.$ 
		\begin{enumerate}
			\item If $f(x)$ is irreducible in $\mathbb{F}(\underline{t})[x],$ then $\G_f \cong D_8.$
			\item If $f(x)$ is reducible in $\mathbb{F}(\underline{t})[x],$ then $\G_f \cong C_4.$ \hfill\hyperref[thm:classifyingirreduciblequarticgalois2]{\downsym}
		\end{enumerate}
	\end{enumerate}
\end{restatable}

\begin{ex}
	Let us now show that all the above possibilities do happen over $\mathbb{F} = \mathbb{Q}.$
	\begin{enumerate}
		\item ($\G_f = C_4$) Let $f(x) = x^4 + 5x^2 + 5.$ Then,
		\begin{equation*} 
			r(x) = x^3 - 5x^2 - 20x + 100 = (x - 5)(x - 2\sqrt{5})(x + 2\sqrt{5}).
		\end{equation*}
		Thus, $\mathbb{F}(\underline{t}) = \mathbb{Q}(\sqrt{5}).$ $f(x)$ is irreducible over $\mathbb{Q},$ by Eisenstein but not over $\mathbb{F}(\underline{t})$ as seen by
		\begin{equation*} 
			f(x) = \left(x^2 + \frac{5 + \sqrt{5}}{2}\right)\left(x^2 - \frac{5 - \sqrt{5}}{2}\right).
		\end{equation*}
		Thus, $\G_f \cong C_4.$
		%
		\item ($\G_f = V$) Let $f(x) = x^4 + 1 \in \mathbb{Q}[x].$ Then, the resolvent is $r(x) = x(x - 2)(x + 2).$ Thus, $\G_f = V.$
		%
		\item ($\G_f = D_8$) Let $f(x) = x^4 - 3.$ Then,
		\begin{equation*} 
			r(x) = x(x + 2\iota\sqrt{3})(x - 2\iota\sqrt{3}).
		\end{equation*}
		Thus, $\mathbb{F}(\underline{t}) = \mathbb{Q}(\iota\sqrt{3}).$ Note that $f(x)$ factors in $\overline{\mathbb{Q}}$ as
		\begin{equation*} 
			f(x) = (x - \iota\sqrt[4]{3})(x + \iota\sqrt[4]{3})(x - \sqrt[4]{3})(x + \sqrt[4]{3}).
		\end{equation*}
		Thus, $f(x)$ has no root in $\mathbb{F}(\underline{t})$ but is irreducible over $\mathbb{Q}$ and thus, $\G_f \cong D_8.$
		%
		\item ($\G_f = A_4$) Let $f(x) = x^4 - 8x + 12.$ Then, $r(x) = x^3 - 48x - 64.$ By the rational root test, we see that $r(x)$ has no roots in $\mathbb{Q}$ and hence, is irreducible. Moreover, so is $f(x),$ by Eisenstein. Now, $\disc(f(x)) = \disc(r(x)) = 2^{12}3^4$ is a square in $\mathbb{Q}$ and thus, $\G_f = A_4.$
		%
		\item ($\G_f = S_4$) Let $f(x) = x^4 - x + 1.$ Then, $r(x) = x^3 - 4x - 1.$ Both are irreducible over $\mathbb{Q}.$ (For $f(x),$ go modulo $2$ and for $r(x),$ use the rational root test.) Now, $\disc(f(x)) = \disc(r(x)) = 229 \notin \mathbb{Q}^2$ and thus, $\G_f \cong S_4.$
	\end{enumerate}
\end{ex}
\chapter{Norm, Trace, and Hilbert's Theorem 90}

\section{Norm and Trace}
\begin{defn} \label{defn:normtrace}
	Let $\mathbb{E}/\mathbb{F}$ be a finite separable extension of degree $n.$ Let $\sigma_1, \ldots, \sigma_n : \mathbb{E} \to \overline{\mathbb{F}}$ be the distinct $\mathbb{F}$-embeddings. For $a \in \mathbb{E},$ define the \deff{norm} and \deff{trace} of $a$ by
	\begin{align*} 
		\N_{\mathbb{E}/\mathbb{F}}(a) &\vcentcolon= \sigma_1(a) \cdots \sigma_n(a), \\
		\Tr_{\mathbb{E}/\mathbb{K}}(a) &\vcentcolon= \sigma_1(a) + \cdots + \sigma_n(a)
	\end{align*}
\end{defn}

We shall omit the subscript when the extension is clear.

\begin{ex}
	Let $m \in \mathbb{Z}$ be square free. Consider the quadratic extension $\mathbb{Q}(\sqrt{m})/\mathbb{Q}.$ Its Galois group consists of the identity and the ``conjugation'' map determined by $\sigma(\sqrt{m}) = -\sqrt{m}.$ 

	Thus, given $a + b\sqrt{m} \in \mathbb{Q}(\sqrt{m})$ with $a, b \in \mathbb{Q},$ we have
	\begin{equation*} 
		\Tr(a + b\sqrt{m}) = 2a \andd \N(a + b\sqrt{m}) = a^2 - mb^2.
	\end{equation*}
	For $m = -1,$ we recover the familiar norm $\N(a + \iota b) = a^2 + b^2.$
\end{ex}

\begin{restatable}[]{prop}{propertiesnormtrace}
\label{prop:propertiesnormtrace}
	Let $\mathbb{E}/\mathbb{F}$ be a finite separable extension.
	\begin{enumerate}
		\item $\N_{\mathbb{E}/\mathbb{F}} : \mathbb{E}^\times \to \mathbb{F}^\times$ is a group homomorphism. \\
		(In particular, $N_{\mathbb{E}/\mathbb{F}}$ takes values in $\mathbb{F}.$)
		%
		\item If $\mathbb{E} = \mathbb{F}(a)$ and $\irr(a, \mathbb{F}) = x^n + a_{n - 1}x^{n - 1} + \cdots + a_0,$ then
		\begin{equation*} 
			\N_{\mathbb{E}/\mathbb{F}}(a) = (-1)^na_0, \andd \Tr_{\mathbb{E}/\mathbb{F}}(a) = -a_{n - 1}.
		\end{equation*}
		%
		\item $\Tr_{\mathbb{E}/\mathbb{F}} : \mathbb{E} \to \mathbb{F}$ is a surjective $\mathbb{F}$-linear map. \\
		(In particular, $\Tr_{\mathbb{E}/\mathbb{F}}$ takes values in $\mathbb{F}.$) 
		\item Let $\mathbb{K}$ be an intermediate subfield of $\mathbb{E}/\mathbb{F}.$ Then,
		\begin{equation*} 
			\N_{\mathbb{E}/\mathbb{F}} = \N_{\mathbb{K}/\mathbb{F}} \circ \N_{\mathbb{E}/\mathbb{K}}, \andd \Tr_{\mathbb{E}/\mathbb{F}} = \Tr_{\mathbb{K}/\mathbb{F}} \circ \Tr_{\mathbb{E}/\mathbb{K}}.
		\end{equation*}
		(The above compositions make sense, by the earlier parts.) \hfill\hyperref[prop:propertiesnormtrace2]{\downsym}
		%
	\end{enumerate} 
\end{restatable}

\begin{restatable}[]{prop}{normtracelinearmap}
\label{prop:normtracelinearmap}
	Let $\mathbb{E}/\mathbb{F}$ be a finite separable extension of degree $n,$ and let $a \in \mathbb{E}.$ Let $m_a : \mathbb{E} \to \mathbb{E}$ be the $\mathbb{F}$-linear map defined as $x \mapsto ax.$ Then,
	\begin{equation*} 
		\N_{\mathbb{E}/\mathbb{F}}(a) = \det(m_a) \andd \Tr_{\mathbb{E}/\mathbb{F}}(a) = \Tr(m_a).
	\end{equation*} \hfill\hyperref[prop:normtracelinearmap2]{\downsym}
\end{restatable}

\begin{restatable}[]{prop}{tracebilinearmaps}
\label{prop:tracebilinearmaps}
	Let $\mathbb{E}/\mathbb{F}$ be a finite separable extension. 
	\begin{enumerate}
		\item The map $\varphi : \mathbb{E} \times \mathbb{E} \to \mathbb{F}$ given by $(x, y) \mapsto \Tr(xy)$ is $\mathbb{F}$-bilinear.
		\item The map $\Tr_x : \mathbb{E} \to \mathbb{F}$ given by $y \mapsto \Tr(xy)$ is $\mathbb{F}$-linear for all $x \in \mathbb{E}.$
		\item The map $\psi : \mathbb{E} \to \Hom_{\mathbb{F}}(\mathbb{E}, \mathbb{F})$ given by $x \mapsto \Tr_x$ is an isomorphism of $\mathbb{F}$-vector spaces.
	\end{enumerate}\hfill\hyperref[prop:tracebilinearmaps2]{\downsym}
\end{restatable}

\begin{restatable}[Hilbert's Theorem 90 (multiplicative form)]{thm}{hilbertmultiplicative}
\label{thm:hilbertmultiplicative}
	Let $\mathbb{E}/\mathbb{F}$ be a cyclic Galois extension with $\Gal(\mathbb{E}/\mathbb{F}) = \langle \sigma\rangle,$ and $\beta \in \mathbb{E}.$ Then,
	\begin{equation*} 
		\N_{\mathbb{E}/\mathbb{F}}(\beta) = 1 \iff \beta = \frac{\alpha}{\sigma(\alpha)} \text{ for some } \alpha \in \mathbb{E}^\times.
	\end{equation*} 
	\hfill\hyperref[thm:hilbertmultiplicative2]{\downsym}
\end{restatable}

\begin{restatable}[]{cor}{splittingfieldnthroots}
\label{cor:splittingfieldnthroots}
	Let $\mathbb{F}$ be a field, and $n \in \mathbb{N}$ be such that $\gcd(n, \chr(\mathbb{F})) = 1.$ Assume that $\mathbb{F}$ has a primitive $n$-th root of $1.$ Let $\mathbb{E}/\mathbb{F}$ be a cyclic Galois extension. Then, $\mathbb{E}$ is the splitting field of $x^n - a \in \mathbb{F}[x]$ for some $a \in \mathbb{F}.$ \hfill\hyperref[cor:splittingfieldnthroots2]{\downsym}
\end{restatable}

\begin{restatable}[Hilbert's Theorem 90 (additive form)]{thm}{hilbertadditive}
\label{thm:hilbertadditive}
	Let $\mathbb{E}/\mathbb{F}$ be a cyclic Galois extension with $\Gal(\mathbb{E}/\mathbb{F}) = \langle \sigma\rangle,$ and $\beta \in \mathbb{E}.$ Then,
	\begin{equation*} 
		\Tr_{\mathbb{E}/\mathbb{F}}(\beta) = 0 \iff \beta = \alpha - \sigma(\alpha) \text{ for some } \alpha \in \mathbb{E}.
	\end{equation*}  \hfill\hyperref[thm:hilbertadditive2]{\downsym}
\end{restatable}

\begin{restatable}[Artin-Schreier]{cor}{artinschreier}
\label{cor:artinschreier}
	Let $\mathbb{F}$ be a field with $\chr(\mathbb{F}) =\vcentcolon p > 0.$ Let $\mathbb{E}/\mathbb{F}$ be a cyclic degree extension of degree $p.$ Then, $\mathbb{E}$ is a splitting field of $f(x) \vcentcolon= x^p - x - a \in \mathbb{F}[x]$ for some $a \in \mathbb{F}$ and $\mathbb{E} = \mathbb{F}(\alpha),$ where $\alpha \in \mathbb{E}$ is a root of $f(x).$ \hfill\hyperref[cor:artinschreier2]{\downsym}
\end{restatable}

\begin{ex}[Rational points on the unit circle]
	We wish to find all rational points $(a, b) \in \mathbb{Q}^2$ satisfying $a^2 + b^2 = 1.$ 

	We claim that these are precisely the points of the form
	\begin{equation*} 
		(a, b) = \left(\frac{c^2 - d^2}{c^2 + d^2}, \frac{2cd}{c^2 + d^2}\right)
	\end{equation*}
	for $c, d \in \mathbb{Z}$ not both zero. (It is clear that every point of the above form is indeed a rational point on the unit circle.)

	The above is an immediate consequence of \nameref{thm:hilbertmultiplicative}. Indeed, considering the degree $2$ extension $\mathbb{Q}(\iota)/\mathbb{Q}$ shows that $\N(a + \iota b) = 1$ and thus, there exists $c + \iota d \in \mathbb{Q}(i)^\times$ such that
	\begin{equation*} 
		a + \iota b = \frac{c + \iota d}{c - \iota d} = \frac{c^2 - d^2}{c^2 + d^2} + \iota \frac{2cd}{c^2 + d^2}.
	\end{equation*}
	Comparing the real and imaginary parts gives the result, after clearing the denominators.
\end{ex} % only last NPTEL chapter left

\chapter{Proofs}
\section{Algebraic extensions}

\finextisalg*\label{prop:finextisalg2}
\begin{flushright}\hyperref[prop:finextisalg]{\upsym}\end{flushright}
\begin{proof}
    Let $\mathbb{K}/\mathbb{F}$ be a finite extension with $n \vcentcolon= \dim_{\mathbb{F}}(\mathbb{K}).$ Let $b \in \mathbb{K}$ be arbitrary. Consider the multiset $\{1, b, \ldots, b^{n}\}.$ It has $n + 1$ elements and thus, is linearly dependent. Thus, there exist $a_0, \ldots, a_{n} \in \mathbb{F}$ not all $0$ such that
    \begin{equation*} 
        a_0 + a_1b + \cdots + a_nb^n = 0.
    \end{equation*}
    Then, $f(x) \vcentcolon= a_0 + a_1b + \cdots + a_nx^n \in \mathbb{F}[x]$ is a non-zero polynomial such that $f(b) = 0.$
\end{proof}

\uniquemonicirred*\label{prop:uniquemonicirred2}
\begin{flushright}\hyperref[prop:uniquemonicirred]{\upsym}\end{flushright}
\begin{proof}
    Define $\psi : \mathbb{F}[x] \to \mathbb{K}$ by $p(x) \mapsto p(\alpha).$ Since $\alpha$ is algebraic, $I \vcentcolon= \ker(\psi)$ is non-zero.

    Since $\mathbb{F}[x]$ is a PID, we have $I = \langle f(x)\rangle$ for some $0 \neq f(x) \in \mathbb{F}[x].$ Since $\mathbb{F}[x]/I$ is isomorphic to a subring of $\mathbb{K},$ it is an integral domain and hence, $f(x)$ is irreducible. By scaling, we may assume that $f(x)$ is monic. Clearly, any other $g(x)$ as in the proposition is in the kernel and hence, $f(x) \mid g(x).$

    In particular, if $g(x)$ is irreducible and monic, then $f(x) \mid g(x) \implies g(x) = af(x)$ for some $a \in \mathbb{F}^\times.$ Since $g(x)$ is also monic, we have $a = 1.$
\end{proof}

\adjoiningalg*\label{prop:adjoiningalg2}
\begin{flushright}\hyperref[prop:adjoiningalg]{\upsym}\end{flushright}
\begin{proof}
    Consider the substitution homomorphism $\psi : \mathbb{F}[x] \to \mathbb{F}[\alpha]$ given by $p(x) \mapsto p(\alpha).$

    By \Cref{prop:uniquemonicirred}, we know that $\ker(\psi) = \langle f(x)\rangle.$ Since $f(x) \neq 0,$ the ideal $\langle f(x)\rangle$ is maximal. 

    Since $\psi$ is onto and $\ker(\psi)$ maximal, we see that $\mathbb{F}[\alpha]$ is in fact a field and hence, $\mathbb{F}[\alpha] = \mathbb{F}(\alpha).$

    Consider $B = \{1, \alpha, \ldots, \alpha^{n - 1}\}.$ \\
    Using $f(x),$ we may recursively write all higher powers of $\alpha$ as an $\mathbb{F}$-linear combination of elements of $B.$ Thus, $B$ spans $\mathbb{F}[\alpha].$ \\
    For linear independence, suppose that $a_0, \ldots, a_{n - 1} \in \mathbb{F}$satisfy
    \begin{equation*} 
        a_0 + a_1\alpha + \cdots + a_{n - 1}\alpha^{n - 1} = 0.
    \end{equation*}
    Then, we get a polynomial $g(x) = a_0 + a_1x + \cdots a_{n - 1}x^{n - 1} \in \mathbb{F}[x]$ satisfied by $\alpha.$ Since $\deg(g(x)) < \deg(f(x)),$ we see that $g(x) = 0,$ again by \Cref{prop:uniquemonicirred}.
\end{proof}

\isocarryingalphtobet*\label{prop:isocarryingalphtobet2}
\begin{flushright}\hyperref[prop:isocarryingalphtobet]{\upsym}\end{flushright}
\begin{proof}
    $(\implies)$ Let $\psi : \mathbb{F}(\alpha) \to \mathbb{F}(\beta)$ be as mentioned.\\
    Put $f(x) \vcentcolon= \irr(\alpha, \mathbb{F})$ and $g(x) \vcentcolon= \irr(\beta, \mathbb{F}).$ Then, 
    \[\begin{WithArrows}[displaystyle]
        0 &= \psi(0) \\
        &= \psi(f(\alpha)) \Arrow{$\psi$ is an $\mathbb{F}$-isomorphism} \\
        &= f(\psi(\alpha)) \\
        &= f(\beta).
    \end{WithArrows}\]
     Thus, $g(x) \mid f(x).$ Since both are irreducible and monic, $g(x) = f(x).$

     $(\impliedby)$ Let $f(x) \vcentcolon= \irr(\alpha, \mathbb{F}) = \irr(\beta, \mathbb{F}).$ \\
     The isomorphisms $\mathbb{F}(\alpha) \cong \mathbb{F}[x]/\langle f(x)\rangle \cong \mathbb{F}(\beta)$ are $\mathbb{F}$-isomorphisms and so is their composition.
\end{proof}

\towerlaw*\label{thm:towerlaw2}
\begin{flushright}\hyperref[thm:towerlaw]{\upsym}\end{flushright}
\begin{proof}
    If $\mathbb{K}/\mathbb{F}$ is a finite extension, then so are $\mathbb{K}/\mathbb{E}$ (pick a finite basis of $\mathbb{K}/\mathbb{F},$ it is a spanning set for $\mathbb{K}/\mathbb{E}$) and $\mathbb{E}/\mathbb{F}$ ($\mathbb{E}$ is an $\mathbb{F}$-subspace of $\mathbb{K}.$)

    Thus, if either of $\mathbb{K}/\mathbb{E}$ or $\mathbb{E}/\mathbb{F}$ is not a finite extension, then neither is $\mathbb{K}/\mathbb{F}.$

    Now, assume that both $n \vcentcolon= [\mathbb{K} : \mathbb{E}]$ and $m \vcentcolon= [\mathbb{E} : \mathbb{F}]$ are finite. Let $\{\alpha_i\}_{i = 1}^n \subset \mathbb{K}$ be an $\mathbb{E}$-basis and $\{\beta_j\}_{j = 1}^m \subset \mathbb{E}$ be an $\mathbb{F}$-basis.

    Put $B \vcentcolon= \{\alpha_i\beta_j : 1 \le i \le n,\; 1 \le j \le m\} \subset \mathbb{K}.$ We show that $B$ is an $\mathbb{F}$-basis of $\mathbb{K}.$

    \textbf{Spanning.} Let $a \in \mathbb{K}$ be arbitrary. Write 
    \begin{equation*} 
        a = \sum_{i = 1}^{n} a_i \alpha_i
    \end{equation*}
    for $a_i \in \mathbb{E}.$ For each $i = 1, \ldots, n,$ write
    \begin{equation*} 
        a_i = \sum_{j = 1}^{m} b_{ij} \beta_j
    \end{equation*}
    for $b_{ij} \in \mathbb{F}.$ Then,
    \begin{equation*} 
        a = \sum_{i = 1}^{n}\sum_{j = 1}^{m}b_{ij} (\alpha_i\beta_j)
    \end{equation*}
    is an $\mathbb{F}$-linear combination of elements of $B.$

    \textbf{Linear independence.} Let $\{b_{ij} : 1 \le i \le n,\; 1 \le j \le m\} \subset \mathbb{F}$ be such that
    \begin{equation*} 
        \sum_{\substack{1 \le i \le n \\ 1 \le j \le m}} b_{ij}\alpha_i\beta_j = 0.
    \end{equation*} 
    Group the above to get
    \begin{equation*} 
        \sum_{i = 1}^{n}\left[\sum_{j = 1}^{m}b_{ij} \alpha_i\right]\beta_j = 0.
    \end{equation*}
    Linear independence of $\{\beta_j\}$ forces $\sum_{j = 1}^{m}b_{ij} \alpha_i = 0$ for all $i.$ In turn, linear independence of $\{\alpha_i\}$ that forces each $b_{ij}$ to be $0.$

    Note that $B$ actually has cardinality $mn.$ (Why?) This finishes the proof.
\end{proof}

\adjoinalgsfinext*\label{prop:adjoinalgsfinext2}
\begin{flushright}\hyperref[prop:adjoinalgsfinext]{\upsym}\end{flushright}
\begin{proof}
    Consider the tower
    \begin{equation*} 
        \mathbb{F} \subset \mathbb{F}(\alpha_1) \subset \mathbb{F}(\alpha_1, \alpha_2) \subset \cdots \subset \mathbb{F}(\alpha_1, \ldots, \alpha_n).
    \end{equation*}
    At each stage, an element being adjoined is algebraic over the previous field. (\Cref{prop:decompalgisalg}.)

    Thus, each consecutive degree above is finite. (\Cref{cor:adjoinalgisfin}.)

    By the \nameref{thm:towerlaw}, so is the overall degree.
\end{proof}

\compalgisalg*\label{cor:compalgisalg2}
\begin{flushright}\hyperref[cor:compalgisalg]{\upsym}\end{flushright}
\begin{proof}
    Let $\alpha \in \mathbb{K}.$ Let $\irr(\alpha, \mathbb{E}) =\vcentcolon f(x) = a_0 + \cdots + a_{n - 1}x^{n - 1} + x^n.$

    Let $\mathbb{L} \vcentcolon= \mathbb{F}(a_0, \ldots, a_{n - 1}).$

    Then, $\mathbb{L}$ is finite over $\mathbb{F}$ since each $a_i \in \mathbb{E}$ is algebraic over $\mathbb{F}.$ Moreover, $0 \neq f(x) \in \mathbb{L}[x].$ Thus, $\alpha$ is algebraic over $\mathbb{L}$ and hence, $\mathbb{L}(\alpha)$ is finite over $\mathbb{L}.$

    By the \nameref{thm:towerlaw}, $\mathbb{L}/\mathbb{F}$ is finite and thus, $\alpha$ is algebraic over $\mathbb{F}.$ (\Cref{prop:finextisalg}.)
\end{proof}

\algclosureisfield*\label{cor:algclosureisfield2}
\begin{flushright}\hyperref[cor:algclosureisfield]{\upsym}\end{flushright}
\begin{proof}
    $\mathbb{F} \subset \mathbb{A}$ is clear. We show that $\mathbb{A}$ is a subfield. Let $\alpha, \beta \in \mathbb{A}$ with $\beta \neq 0.$ Then, $\mathbb{L} \vcentcolon= \mathbb{F}(\alpha, \beta)$ is a finite extension over $\mathbb{F}.$ \\
    Thus, all elements of $\mathbb{L}$ are algebraic over $\mathbb{F}.$ In particular, so are $\alpha \pm \beta,$ $\alpha\beta$ and $\alpha\beta^{-1}.$
\end{proof}

\intdomfinextfield*\label{prop:intdomfinextfield2}
\begin{flushright}\hyperref[prop:intdomfinextfield]{\upsym}\end{flushright}
\begin{proof}
    We only need to show that every non-zero element of $R$ has a multiplicative inverse (in $R$). Let $0 \neq a \in R$ be arbitrary. Since $\dim_{\mathbb{F}}(R) < \infty,$ there is a smallest $n \ge 1$ such that the set $\{1, a, \ldots, a^n\}$ is linearly dependent over $\mathbb{F}.$ Then, let $b_0, \ldots, b_{n} \in \mathbb{F}$ be not all zero such that
    \begin{equation*} 
        b_0 + b_1a + \cdots b_na^n = 0.
    \end{equation*} 
    If $b_n = 0,$ then the minimality of $n$ is contradicted. If $b_0 = 0,$ then we may cancel $a$ ($R$ is an integral domain and $a \neq 0$) and again contradict the minimality of $n.$ Thus, we get
    \begin{equation*} 
        a(b_1 + \cdots + b_na^{n - 1}) = -b_0.
    \end{equation*}
    This shows that the element
    \begin{equation*} 
        -\frac{1}{b_0}(b_1 + \cdots + b_na^{n - 1}) \in R
    \end{equation*}
    is a multiplicative inverse of $a.$
\end{proof}

\descofcompositum*\label{prop:descofcompositum2}
\begin{flushright}\hyperref[prop:descofcompositum]{\upsym}\end{flushright}
\begin{proof}
    Simple computations show that $\mathbb{L}$ is indeed a subring of $\mathbb{K}.$ If $\{\alpha_1, \ldots, \alpha_n\}$ and $\{\beta_1, \ldots, \beta_m\}$ are $\mathbb{F}$-bases for $\mathbb{E}_1$ and $\mathbb{E}_2,$ then clearly $\{\alpha_i\beta_j : 1 \le i \le n,\; 1 \le j \le m\}$ spans $\mathbb{L}$ over $\mathbb{F}.$ Thus, $\dim_{\mathbb{F}}(\mathbb{L}) \le mn = d.$ 

    Note that $\mathbb{L}$ is clearly the smallest subring of $\mathbb{K}$ containing $\mathbb{E}_1$ and $E_2.$ Since $\mathbb{L}$ is a subring of $\mathbb{K},$ it is an integral domain and hence, $\mathbb{L}$ is a field, by \Cref{prop:intdomfinextfield}. Thus, $\mathbb{L} = \mathbb{E}_1\mathbb{E}_2.$

    Lastly, note that $[\mathbb{E}_i : \mathbb{F}]$ divides $[\mathbb{L} : \mathbb{F}],$ in view of the \nameref{thm:towerlaw}. In particular, if $\gcd(m, n) = 1,$ then $mn \mid [\mathbb{L} : \mathbb{F}].$ Since $[\mathbb{L} : \mathbb{F}] \le mn,$ we are done.
\end{proof}

\rootcanbeadjoined*\label{thm:rootcanbeadjoined2}
\begin{flushright}\hyperref[thm:rootcanbeadjoined]{\upsym}\end{flushright}
\begin{proof}
    Let $g(x)$ be an irreducible factor of $f(x).$

    Put $\mathbb{K} = \mathbb{F}[x]/\langle g(x)\rangle.$ Since $g(x)$ is irreducible and non-zero, the quotient is indeed a field. Clearly, $\mathbb{F}$ is a subfield under the identification $a \mapsto \bar{a}.$ Moreover, $\bar{x}$ is a root of $g(x).$
\end{proof}

\splitfieldexists*\label{thm:splitfieldexists2}
\begin{flushright}\hyperref[thm:splitfieldexists]{\upsym}\end{flushright}
\begin{proof}
    Let $n \vcentcolon= \deg(f).$ By \Cref{thm:rootcanbeadjoined}, there exists a field $\mathbb{F}_1 \supset \mathbb{F}$ such that $f(x)$ has a root in $\mathbb{F}_1.$ Calling this root $a_1,$ we see that
    \begin{equation*} 
        f(x) = (x - a_1)f_1(x)
    \end{equation*}
    with $\deg(f_1) = n - 1.$ Continuing inductively, we get fields
    \begin{equation*} 
        \mathbb{F}_n \supset \cdots \supset \mathbb{F}_1 \supset \mathbb{F}
    \end{equation*}
    with $a_i \in \mathbb{F}_i,$ such that
    \begin{equation*} 
        f(x) = a(x - a_1) \cdots (x - a_n).
    \end{equation*}
    Then, $\mathbb{K} = \mathbb{F}(a_1, \ldots, a_n) \subset \mathbb{F}_n$ is a splitting field.
\end{proof}

\section{Symmetric Polynomials}
\FTSP*\label{thm:FTSP2}
\begin{flushright}\hyperref[thm:FTSP]{\upsym}\end{flushright}
\begin{proof}
    \textbf{Existence.} We apply induction on $n.$ The case $n = 1$ is clear since every polynomial is symmetric and $\sigma_1 = u_1.$ So, $g = f$ itself works\footnote{Being slightly sloppy since the indeterminates are different. We mean that you must take the same coefficients}.

    Suppose the theorem is true for $n - 1.$ Now, to prove the theorem for $n,$ apply induction on $\deg(f).$ If $f$ is constant, then again $g = f$ works. Suppose $\deg(f) \ge 1.$ Define
    \begin{equation*} 
        f^0 \vcentcolon= f(u_1, \ldots, u_{n - 1}, 0) \in R[u_1, \ldots, u_{n - 1}].
    \end{equation*}
    Then, $f^0$ is a symmetric polynomial in $n - 1$ variables. By induction hypothesis (on variables), there exists $g \in R[x_1, \ldots, x_{n - 1}]$ such that
    \begin{equation*} 
        f^0(u_1, \ldots, u_{n - 1}) = g(\sigma_1^0, \ldots, \sigma_{n - 1}^0).
    \end{equation*}
    Define $f_1 \in R[u_1, \ldots, u_n]$ by
    \begin{equation*} 
        f_1(u_1, \ldots, u_n) = f(u_1, \ldots, u_n) - g(\sigma_1, \ldots, \sigma_{n - 1}).
    \end{equation*}
    Then, $f_1(u_1, \ldots, u_{n - 1}, 0) = 0.$ Thus, $u_n \mid f_1.$ However, note that $f_1$ is symmetric and thus, $\sigma_n \mid f_1.$ Thus, we can write
    \begin{equation*} 
        f_1(u_1, \ldots, u_n) = \sigma_n h(u_1, \ldots, u_n)
    \end{equation*}
    for some $h \in R[u_1, \ldots, u_n].$ Since $\sigma_n$ is not a zero-divisor in $R[u_1, \ldots, u_n],$ we see that $h$ is also symmetric with $\deg(h) < \deg(f).$ Thus, by inductive hypothesis, $h$ is a polynomial in $\sigma_1, \ldots, \sigma_n$ and hence, $f$ is so.

    \textbf{Uniqueness.} It suffices to show that the elementary symmetric polynomials are algebraically independent. That is, to show that the map
    \begin{equation*} 
        \varphi : R[z_1, \ldots, z_n] \to R[u_1, \ldots, u_n]
    \end{equation*}
    defined by 
    \begin{equation*} 
        z_i \mapsto \sigma_i \andd \varphi|_R = \id_R
    \end{equation*}
    is an injection.

    We prove this by induction on $n.$ For $n = 1,$ it is clear since $\sigma_1 = u_1,$ an indeterminate. Assume that $n > 1$ and that the result is true for $n - 1.$ If $\varphi$ is not an injection, then we pick a nonzero polynomial $f(z_1, \ldots, z_n) \in \ker(\varphi)$ of least degree. Write $f$ as a polynomial in $z_n$ as
    \begin{equation*} 
        f(z_1, \ldots, z_n) = f_0(z_1, \ldots, z_{n - 1}) + \cdots + f_d(z_1, \ldots, z_{n - 1})z_n^d
    \end{equation*}
    with $f_d \neq 0.$ Minimality of $d$ (and the fact that $\sigma_n$ is not a zero-divisor) forces that $f_0 \neq 0.$ Since $f \in \ker(\varphi),$ we have
    \begin{equation*} 
        f_0(\sigma_1, \ldots, \sigma_{n - 1}) + \cdots + f_d(\sigma_1, \ldots, \sigma_{n - 1})\sigma_n^d = 0.
    \end{equation*}
    The above is an equality in $R[u_1, \ldots, u_n].$ Put $u_n = 0$ to get
    \begin{equation*} 
        f_0(\sigma_1^0, \ldots, \sigma_{n - 1}^0) = 0.
    \end{equation*}
    But the above shows that the corresponding $\varphi$ for $n - 1$ variables is not injective. A contradiction.
\end{proof}

\powersumformulae*\label{thm:powersumformulae2}
\begin{flushright}\hyperref[thm:powersumformulae]{\upsym}\end{flushright}
\begin{proof}
    Let $z$ be an indeterminate over $S \vcentcolon= R[u_1, \ldots, u_n].$ Note that 
    \begin{equation} \label{eq:001}
        (1 - u_1z) \cdots (1 - u_nz) = 1 - \sigma_1z + \cdots + (-1)^n \sigma_n z^n =\vcentcolon \sigma(z).
    \end{equation}
    Define $w(z) \in S[\![z]\!]$ as
    \begin{align*} 
        w(z) &= \sum_{k = 1}^{\infty} w_kz^k\\
        &= \sum_{k = 1}^{\infty}\left(\sum_{i = 1}^{n}u_i^k\right)z^k\\
        &= \sum_{i = 1}^{n} \left(\sum_{k = 1}^{\infty}(u_iz)^k\right)\\
        &= \sum_{i = 1}^{n} \frac{u_iz}{1 - u_iz}.
    \end{align*}
    Now, since $\sigma(z) = (1 - u_1z) \cdots (1 - u_nz),$ we get
    \begin{equation*} 
        \sigma'(z) = - \sum_{i = 1}^{n} \frac{u_i \sigma(z)}{1 - u_i z},
    \end{equation*}
    where we have taken the formal derivative in $S[\![z]\!].$ Rearranging the above gives
    \begin{equation*} 
        -\frac{z\sigma'(z)}{\sigma(z)} = \sum_{i = 1}^{n}\frac{u_i z}{1 - u_i z} = w(z)
    \end{equation*}
    and hence,
    \begin{equation*} 
        w(z)\sigma(z) = -z\sigma'(z).
    \end{equation*}
    Computing $\sigma'(z)$ from \Cref{eq:001} gives
    \begin{equation*} 
        w(z)\sigma(z) = \sigma_1z - 2\sigma_2z^2 + \cdots + (-1)^{n + 1}n\sigma_nz^n.
    \end{equation*}
    Comparing the coefficients of $z^k$ on both sides gives the result.
\end{proof}

\independencediscriminant*\label{prop:independencediscriminant2}
\begin{flushright}\hyperref[prop:independencediscriminant]{\upsym}\end{flushright}
\begin{proof}
    Let $r_1, \ldots, r_n \in \mathbb{K}$ be such that $f(x) = (x - r_1) \cdots (x - r_n).$

    Consider the Vandermonde matrix
    \begin{equation*} 
        M = \begin{bmatrix}
            1 & 1 & \cdots & 1\\
            r_1 & r_2 & \cdots & r_n\\
            r_1^2 & r_2^2 & \cdots & r_n^2\\
            \vdots & \vdots & \ddots & \vdots \\
            r_1^{n - 1} & r_2^{n - 1} & \cdots & r_n^{n - 1}\\
        \end{bmatrix}.
    \end{equation*}
    Then, $\disc_{\mathbb{K}}(f(x)) = (\det(M))^2 = \det(MM^{\mathsf{T}}).$ As before, let $\sigma_1, \ldots, \sigma_n \in \mathbb{F}[u_1, \ldots, u_n]$ be the elementary symmetric polynomials. Put
    \begin{equation*} 
        s_i \vcentcolon= \sigma_i(r_1, \ldots, r_n).
    \end{equation*}
    Then, note that
    \begin{equation*} 
        f(x) = x^n - s_1x^{n - 1} + \cdots + (-1)^ns_n
    \end{equation*}
    and hence, $s_i \in \mathbb{F}$ for all $i = 1, \ldots, n.$ Also, define
    \begin{equation*} 
        v_k \vcentcolon= r_1^k + \cdots + r_n^k
    \end{equation*}
    for all $k \ge 1.$ In view of \nameref{thm:powersumformulae}, we see that each $v_k \in \mathbb{F}$ as well. Moreover, note that
    \begin{equation*} 
        MM^{\mathsf{T}} = \begin{bmatrix}
            n & v_1 & \cdots & v_{n - 1}\\
            v_1 & v_2 & \cdots & v_n\\
            v_2 & v_3 & \cdots & v_{n + 1}\\
            \vdots & \vdots & \ddots & \vdots \\
            v_{n - 1} & v_n & \cdots & v_{2n - 2}\\
        \end{bmatrix}.
    \end{equation*} 
    Thus, $\disc_{\mathbb{K}}(f(x)) = \det(MM^{\mathsf{T}}) \in \mathbb{F}.$

    Note that $v_k$ can be calculated directly in terms of $s_i,$ the coefficients of $f(x).$ Thus, the discriminant does not depend on the choice of the splitting field.
\end{proof}

\discderivative*\label{prop:discderivative2}
\begin{flushright}\hyperref[prop:discderivative]{\upsym}\end{flushright}
\begin{proof}
    Note that
    \begin{equation*}
        f'(x) = \sum_{i = 1}^{n}\frac{f(x)}{x - r_i} = \sum_{i = 1}^{n}\prod_{\substack{j = 1 \\ j \neq i}}^{n}(x- r_j)
    \end{equation*}
    and thus,
    \begin{equation*} 
        f'(r_i) = \prod_{\substack{j = 1 \\ j \neq i}}^{n}(r_i - r_j).
    \end{equation*}
    The result now follows.
\end{proof}

\FTAprelim*\label{lem:FTAprelim2}
\begin{flushright}\hyperref[lem:FTAprelim]{\upsym}\end{flushright}
\begin{proof} 
    The first follows from intermediate value property. For the second, given $a + b \iota \in \mathbb{C}$ with $a, b \in \mathbb{R},$ define $c, d \in \mathbb{R}$ by
    \begin{equation*} 
        c \vcentcolon= \sqrt{\frac{1}{2}[a + \sqrt{a^2 + b^2}]} \andd d \vcentcolon= \sqrt{\frac{1}{2}[-a + \sqrt{a^2 + b^2}]}.
    \end{equation*}
    Then, $(c + d \iota)^2 = a + b\iota.$
\end{proof}
\FTA*\label{thm:FTA2}
\begin{flushright}\hyperref[thm:FTA]{\upsym}\end{flushright}
\begin{proof}
    Let $g(x) \in \mathbb{C}[x]$ be a non-constant polynomial. Then, $f(x) = g(x)\bar{g}(x)$ is a non-constant polynomial with real coefficients. Here, $\bar{g}(x)$ denotes the polynomial whose coefficients are complex conjugates of those of $g(x).$ Note that if $f(z) = 0$ for some $z \in \mathbb{C},$ then $g(z) = 0$ or $\bar{g}(z) = 0.$ If $\bar{g}(z) = 0,$ then $g(\bar{z}) = 0.$ In either case, $g$ has a complex root.

    Thus, it suffices to show that all non-constant real polynomials have a root in $\mathbb{C}.$ Given any $f(x) \in \mathbb{R}[x],$ we can write $\deg(f) = 2^nq$ for unique $n \ge 0$ and odd $q \in \mathbb{N}.$

    We prove the statement by induction on $n.$ If $n = 0,$ then $f$ has odd degree and hence, has a real root. \\
    Suppose $n \ge 1$ and the statement is true for $n - 1.$ Let $d \vcentcolon= \deg(f)$ and $\mathbb{K} = \mathbb{C}(\alpha_1, \ldots, \alpha_d)$ be a splitting field of $f(x)$ over $\mathbb{C},$ where the $\alpha_i$ are the roots of $f(x).$ For $r \in \mathbb{R},$ define
    \begin{equation*} 
        y_{ij}(r) = \alpha_i + \alpha_j + r\alpha_i\alpha_j
    \end{equation*}
    for $1 \le i \le j \le d.$ There are $\binom{d + 1}{2}$ such pairs $(i, j).$ Hence, the polynomial
    \begin{equation*} 
        h_r(x) \vcentcolon= \prod_{1 \le i \le j \le d} (x - y_{ij}(r))
    \end{equation*}
    has degree
    \begin{equation*} 
        \deg(h_r(x)) = \binom{d + 1}{2} = \frac{d}{2}(d + 1) = 2^{n - 1}\underbrace{q(d + 1)}_{\text{odd}}.
    \end{equation*}
    Note that the coefficients of $h_r(x)$ are elementary symmetric polynomials in $y_{ij}$s. Thus, they are symmetric polynomials in $\alpha_i, \ldots, \alpha_d.$ Hence, they are polynomials in the coefficients of $f(x).$ Thus, $h(x) \in \mathbb{R}[x].$ By inductive hypothesis (on $n$), we see that $h_r(x)$ has a root $z_r \in \mathbb{C} \subset \mathbb{K}.$ Thus, $z_r = y_{i(r)j(r)}(r)$ for some pair $(i(r), j(r))$ with $1 \le i(r) \le j(r) \le d.$

    Let $P = \{(i, j) : 1 \le i \le j \le d\}$ and define $\varphi : \mathbb{R} \to P$ by $r \mapsto (i(r), j(r)).$ Since $P$ is finite and $\mathbb{R}$ is not, $\varphi$ is not one-one and thus, there exist $c \neq d \in \mathbb{R}$ with
    \begin{equation*} 
        (i(c), j(c)) = (i(d), j(d)) =\vcentcolon (a, b) \in P.
    \end{equation*}
    Thus,
    \begin{equation*} 
        z_c = \alpha_a + \alpha_b + c\alpha_a\alpha_b \andd z_d = \alpha_a + \alpha_b + d\alpha_a\alpha_b.
    \end{equation*}
    Note that a priori, we only know that $\alpha_a, \alpha_b \in \mathbb{K}.$ But note that
    \begin{equation*} 
        \alpha_a\alpha_b = \frac{z_c - z_d}{d - c} \in \mathbb{C}
    \end{equation*}
    and consequently,
    \begin{equation*} 
        \alpha_a + \alpha_b = z_c - c\alpha_a\alpha_b \in \mathbb{C}.
    \end{equation*}
    Thus, $\alpha_a\alpha_b$ and $\alpha_a + \alpha_b \in \mathbb{C}.$ However, these are roots of the quadratic
    \begin{equation*} 
        x^2 - (\alpha_a + \alpha_b)x + \alpha_a\alpha_b \in \mathbb{C}[x].
    \end{equation*}
    Thus, $\alpha_a \in \mathbb{C}.$ But $\alpha_a$ was a root of $f(x),$ as desired.
\end{proof}

\section{Algebraic Closure of a Field}
\alglcosureinalgclosedisclosed*\label{prop:alglcosureinalgclosedisclosed2}
\begin{flushright}\hyperref[prop:alglcosureinalgclosedisclosed]{\upsym}\end{flushright}
\begin{proof}
    By \Cref{cor:algclosureisfield}, we already know that $\mathbb{A}/\mathbb{F}$ is actually an algebraic extension. We just need to show that $\mathbb{A}$ is algebraically closed. To this end, let $f(x) \in \mathbb{A}[x]$ be non-constant. Then, $f(x)$ has a root $\alpha \in \mathbb{K}.$ But then, $\alpha$ is algebraic over $\mathbb{A}$ and hence, over $\mathbb{F}.$ (\Cref{cor:compalgisalg}.) Thus, $\alpha \in \mathbb{A}.$
\end{proof}

\unionoffields*\label{lem:unionoffields2}
\begin{flushright}\hyperref[lem:unionoffields]{\upsym}\end{flushright}
\begin{proof}
    The operations are clearly well-defined. It is easy to see that the desired commutative and associative laws hold since they hold in each $\mathbb{F}_i.$ The $0$ and $1$ are those of each $\mathbb{F}_i.$ The appropriate inverses of any $a \in \mathbb{F}$ also exist in any $\mathbb{F}_i$ containing $a.$ The last sentence is also easy to check.
\end{proof}
\algclosedext*\label{thm:algclosedext2}
\begin{flushright}\hyperref[thm:algclosedext]{\upsym}\end{flushright}
\begin{proof}
    We first show that given any field $\mathbb{F},$ we can create a field $\mathbb{F}_1 \supset \mathbb{F}$ containing roots of any non-constant polynomial in $\mathbb{F}[x].$ Let $S$ be a set of indeterminates which are in one-to-one correspondence with set of all polynomials in $\mathbb{F}[x]$ with degree $\ge 1.$ Let $x_f \in S$ denote the indeterminate corresponding to $f.$

    Consider the (very large) polynomial ring $\mathbb{F}[S].$ Let 
    \begin{equation*} 
        I = \langle f(x_f)  : f \in \mathbb{F}[x],\;\deg(f) \ge 1\rangle
    \end{equation*}
    be the ideal generated by the polynomials $f(x_f) \in \mathbb{F}[S].$ We contend that $1 \notin I.$ Suppose the contrary. Then,
    \begin{equation*} 
        1 = g_1 f_1(x_{f_1}) + \cdots + g_n f_n(x_{f_n})
    \end{equation*}
    for some $g_1, \ldots, g_n \in \mathbb{F}[S].$ Note that these polynomials $g_j$ only involve finitely many variables. Let $x_i \vcentcolon= x_{f_i}$ for $i = 1, \ldots, n$ and let $x_{n + 1}, \ldots, x_m$ be the remaining variables in $g_1, \ldots, g_n.$ Then, we have
    \begin{equation*} 
        \sum_{i = 1}^{n} g_i(x_1, \ldots, x_n, x_{n + 1}, \ldots, x_m)f_i(x_i) = 1.
    \end{equation*}
    Now, let $\mathbb{E} \supset \mathbb{F}$ be an extension containing roots $\alpha_i$ of $f_i.$ (Note that $\deg(f_i) \ge 1$ and thus, we may use \Cref{thm:rootcanbeadjoined}.) Then, putting $x_i = \alpha_i$ for $i = 1, \ldots, n$ and putting $x_{n + 1} = \cdots = x_m = 0$ in the above equation gives a contradiction.

    Thus, $1 \notin I$ and hence, $I$ is a proper ideal of $\mathbb{F}[S].$ Thus, it is contained in some maximal ideal $\mathfrak{m} \subset \mathbb{F}[S].$ Put $\mathbb{F}_1 \vcentcolon= \mathbb{F}[S]/\mathfrak{m}.$ Then, $\mathbb{F}_1$ is a field extension of $\mathbb{F}.$ \\
    Note that $\overline{x_f} = x_f + \mathfrak{m} \in \mathbb{F}_1$ is a root of $f(x) \in \mathbb{F}[x].$ Thus, we have constructed a field $\mathbb{F}_1$ in which every non-constant polynomial of $\mathbb{F}[x]$ has a root.

    Repeating the procedure, we get fields 
    \begin{equation*} 
        \mathbb{F} = \mathbb{F}_0 \subset \mathbb{F}_1 \subset \mathbb{F}_2 \subset \mathbb{F}_3 \subset \cdots
    \end{equation*} 
    such that every non-constant polynomial in $\mathbb{F}_i$ has a root in $\mathbb{F}_{i + 1}.$

    Now, put $\mathbb{K} = \bigcup_{i \ge 0}\mathbb{F}_i.$ This is a field as per \Cref{lem:unionoffields}, having each $\mathbb{F}_i$ as a subfield. 

    Now, if $f(x) \in \mathbb{K}[x],$ then $f(x) \in \mathbb{F}_n[x]$ for some $n.$ This has a root in $\mathbb{F}_{n + 1} \subset \mathbb{K},$ as desired.
\end{proof}

\algclosure*\label{cor:algclosure2}
\begin{flushright}\hyperref[cor:algclosure]{\upsym}\end{flushright}
\begin{proof}
    Let $\mathbb{L} \supset \mathbb{F}$ be algebraically closed. (Existence given by \Cref{thm:algclosedext}.) Define
    \begin{equation*} 
        \mathbb{K} \vcentcolon= \{\alpha \in \mathbb{L} : \alpha \text{ is algebraic over } \mathbb{F}\}.
    \end{equation*}
    By \Cref{prop:alglcosureinalgclosedisclosed}, $\mathbb{K}$ is an algebraic closure of $\mathbb{F}.$
\end{proof}

\rootsandextensions*\label{prop:rootsandextensions2}
\begin{flushright}\hyperref[prop:rootsandextensions]{\upsym}\end{flushright}
\begin{proof}
    First, we note that the map is indeed well-defined. Let $\tau$ be an embedding extending $\sigma.$ Then,
    \begin{equation*} 
        \tau(p(\alpha)) = p^{\sigma}(\tau(\alpha)) = 0
    \end{equation*}
    and thus, $\tau(\alpha)$ is indeed a root of $p^{\sigma}.$ 

    Now, let $\beta \in L$ be such that $p^{\sigma}(\beta) = 0.$ Define $\tau_{\beta} : \mathbb{F}(\alpha) \to \mathbb{L}$ by $\tau_{\beta}(f(\alpha)) = f^{\sigma}(\beta)$ for $f(x) \in \mathbb{F}[x].$\footnote{Note that elements of $\mathbb{F}(\alpha)$ are precisely polynomials in $\alpha.$} We now show that $\tau_{\beta}$ is well-defined. 

    Suppose $f(\alpha) = g(\alpha).$ Then, $p(x) \mid f(x) - g(x)$ and hence, $p^{\sigma}(x) \mid f^{\sigma}(x) - g^{\sigma}(x).$ Thus, $f^{\sigma}(\beta) = g^{\sigma}(\beta).$ Thus, $\tau_{\beta}$ is well-defined. It is clearly a homomorphism (and hence, an embedding). Moreover, it extends $\sigma.$

    It is now easily seen that $\beta \mapsto \tau_{\beta}$ is a two-sided inverse of the map $\tau \mapsto \tau(\alpha).$
\end{proof}

\extendtoalgextension*\label{thm:extendtoalgextension2}
\begin{flushright}\hyperref[thm:extendtoalgextension]{\upsym}\end{flushright}
\begin{proof}
    Consider the set
    \begin{equation*} 
        \Sigma \vcentcolon= \{(\mathbb{E}, \tau) \mid \mathbb{F} \subset \mathbb{E} \subset \mathbb{K} \text{ are fields and } \tau : \mathbb{E} \to \mathbb{L} \text{ such that }\tau|_{\mathbb{F}} = \sigma\}.
    \end{equation*}
    Note that $\Sigma \neq \emptyset$ since $(\mathbb{F}, \sigma) \in \Sigma.$ Define the relation $\le$ on $\Sigma$ by
    \begin{equation*} 
        (\mathbb{E}, \tau) \le (\mathbb{E}', \tau') \iff \mathbb{E} \subset \mathbb{E}' \text{ and } \tau'|_{\mathbb{E}} = \tau.
    \end{equation*}
    Then, $(\Sigma, \le)$ is a partially ordered set. Moreover, if $\Lambda = \{(\mathbb{E}_\alpha, \tau_\alpha)\}_{\alpha \in I}$ is a chain in $\Sigma,$ then $\mathbb{E} \vcentcolon= \bigcup_{\alpha \in I}\mathbb{E}_\alpha$ is a subfield of $\mathbb{K}$ and $\tau : \mathbb{E} \to \mathbb{L}$ defined as $\tau(x) \vcentcolon= \tau_\alpha(x)$ for $x \in \mathbb{F}_\alpha$ is well-defined. (The proof is similar to that of \Cref{lem:unionoffields}.) Moreover, $(\mathbb{E}, \tau)$ is an upper bound of $\Lambda.$   

    Thus, by Zorn's lemma, there exists a maximal element $(\mathbb{E}, \tau) \in \Sigma.$ We contend that $\mathbb{E} = \mathbb{K}.$ If not, then pick $\alpha \in \mathbb{K} \setminus \mathbb{E}.$ By \Cref{prop:rootsandextensions}, we can extend $\tau$ to an embedding $\tau' : \mathbb{E}(\alpha) \to \mathbb{L}.$ But this contradicts maximality of $(\mathbb{E}, \tau).$

    Now, suppose that $\mathbb{K}$ is an algebraic closure of $\mathbb{F}$ and $\mathbb{L}$ of $\sigma(\mathbb{F}).$ We have
    \begin{equation*} 
        \sigma(\mathbb{F}) \subset \tau(\mathbb{K}) \subset \mathbb{L}
    \end{equation*}
    and thus, $L/\tau(\mathbb{K})$ is also algebraic. But $\tau(\mathbb{K})$ is also algebraically closed and thus, $\mathbb{L} = \tau(\mathbb{K}).$
\end{proof}

\isosplitting*\label{thm:isosplitting2}
\begin{flushright}\hyperref[thm:isosplitting]{\upsym}\end{flushright}
\begin{proof}
    Let $\overline{\mathbb{E}}$ be an algebraic closure of $\mathbb{E}.$ Then, it is also one of $\mathbb{F}.$ Thus, there exists an embedding $\tau : \mathbb{E}' \to \overline{\mathbb{E}}$ extending the inclusion $i : \mathbb{F} \hookrightarrow \overline{\mathbb{E}}.$

    Let $f(x) = a(x - \alpha_1) \cdots (x - \alpha_n)$ be a factorisation of $f(x)$ in $\mathbb{E}'[x].$ Then,
    \begin{equation*} 
        f^{\tau}(x) = a(x - \tau(\alpha_1)) \cdots (x - \tau(\alpha_n)) \in \overline{\mathbb{E}}[x].
    \end{equation*}
    (Note that $a \in \mathbb{F}^\times.$)
    Note that we have $\mathbb{E}' = \mathbb{F}(\alpha_1, \ldots, \alpha_n)$ and so, $\tau(\mathbb{E}') = \mathbb{F}(\tau(\alpha_1), \ldots, \tau(\alpha_n)).$ Thus, $\tau(\mathbb{E}')$ is a splitting field of $f^{\tau}.$ But $f^{\tau} = f$ since $f(x) \in \mathbb{F}[x]$ and $\tau$ extends the inclusion map. Thus, $\tau(\mathbb{E}') = \mathbb{E},$ since any algebraic closure contains a unique splitting field.
\end{proof}

\section{Separable extensions}
\multindepsplitting*\label{prop:multindepsplitting2}
\begin{flushright}\hyperref[prop:multindepsplitting]{\upsym}\end{flushright}
\begin{proof}
    Let $\mathbb{E}$ and $\mathbb{K}$ be splitting fields for $f(x)$ over $\mathbb{F}.$ By \Cref{thm:isosplitting}, there exists an $\mathbb{F}$-isomorphism $\tau : \mathbb{E} \to \mathbb{K}.$ In turn, we get an isomorphism
    \begin{align*} 
        \varphi_\tau : \mathbb{E}[x] &\to \mathbb{K}[x]\\
        \sum a_i x^i &\mapsto \sum \tau(a_i) x^i.
    \end{align*}
    Now, let $f(x) = \prod_{i = 1}^{g}(x - r_i)^{e_i}$ be the unique factorisation of $f(x)$ in $\mathbb{E}[x].$ The above isomorphism shows that 
    \begin{equation*} 
        f(x)= \prod_{i = 1}^{g}(x - \tau(r_i))^{e_i}
    \end{equation*}
    is the unique factorisation of $f(x)$ in $\mathbb{K}[x].$ The result follows.
\end{proof}

\derivcritreproot*\label{prop:derivcritreproot2}
\begin{flushright}\hyperref[prop:derivcritreproot]{\upsym}\end{flushright}
\begin{proof}
    \forward If $r$ is a repeated root, then write $f(x) = (x - r)^2g(x)$ for $g \in \mathbb{E}[x].$ Then, taking the derivative gives
    \begin{equation*} 
        f'(x) = 2(x - r)g(x) + (x - r)^2g'(x).
    \end{equation*}
    Thus, $f'(r) = 0.$

    \backward Write $f(x) = (x - r)g(x).$ Then,
    \begin{equation*} 
        0 = f'(r) = (r - r)g'(r) + g(r) = g(r).
    \end{equation*}
    Thus, $(x - r) \mid g(x)$ and hence, $(x - r)^2 \mid f(x).$
\end{proof}

\derivcritsep*\label{thm:derivcritsep2}
\begin{flushright}\hyperref[thm:derivcritsep]{\upsym}\end{flushright}
\begin{proof}
    Let $\mathbb{E}$ be a splitting field of $f(x).$
    \begin{enumerate}[leftmargin=*]
        \item Let $r \in \mathbb{E}$ be a root of $f(x).$ Then, $f'(r) = 0,$ by hypothesis and thus, $r$ is a repeated root, by \Cref{prop:derivcritreproot}.
        %
        \item Suppose $f'(x) \neq 0.$\\
        \forward Suppose $f(x)$ has simple roots. We need to show that $f(x)$ and $f'(x)$ have no common root. Let $r$ be a root of $f(x).$ Then $f'(r) \neq 0,$ by \Cref{prop:derivcritreproot}.

        \backward Suppose $\gcd(f(x), f'(x)) = 1$ and $r \in \mathbb{E}$ is an arbitrary root of $f(x).$ Then, $f'(r) \neq 0.$ Thus, $r$ is a simple root. \qedhere
    \end{enumerate}
\end{proof}

\irredsepderiv*\label{prop:irredsepderiv2}
\begin{flushright}\hyperref[prop:irredsepderiv]{\upsym}\end{flushright}
\begin{proof}
    Let $\mathbb{E}$ be a splitting field of $f(x)$ over $\mathbb{F}.$
    \begin{enumerate}[leftmargin=*]
        \item \forward $f(x)$ has no repeated roots and thus, $f'(x) \neq 0,$ by \Cref{thm:derivcritsep}.

        \backward Suppose $f'(x) \neq 0$ and $f(x)$ has a repeated root $r \in \mathbb{E}.$ Then, by \Cref{prop:derivcritreproot}, $f'(r) = 0.$ Thus, $g(x) \vcentcolon= \gcd(f(x), f'(x)) \neq 1.$ Irreducibility of $f(x)$ forces $f(x) = g(x).$ But then, $f(x) \mid f'(x),$ which is a contradiction since $\deg(f'(x)) < \deg(f(x)).$
        %
        \item If $f(x)$ is non-constant, then $f'(x) \neq 0.$ The previous part applies. \qedhere
    \end{enumerate} 
\end{proof}

\xppolyirredorroot*\label{prop:xppolyirredorroot2}
\begin{flushright}\hyperref[prop:xppolyirredorroot]{\upsym}\end{flushright}
\begin{proof}
    Suppose $f(x)$ is not irreducible. Write $f(x) = g(x)h(x)$ with $1 \le \deg(g(x)) =\vcentcolon m < p.$ Let $b \in \mathbb{E}$ be a root in a splitting field $\mathbb{E}$ of $f(x)$ over $\mathbb{F}.$ Then, $b^p = a.$ Thus, $f(x)$ factorises in $\mathbb{E}[x]$ as
    \begin{equation*} 
        f(x) = x^p - b^p = (x - b)^p.
    \end{equation*}
    Since $\mathbb{E}[x]$ is a UFD, we see that $g(x) = (x - b)^m.$ (We may assume that $g(x)$ is monic.) However, note that the coefficient of $x^{m - 1}$ is $mb.$ By assumption, $mb \in \mathbb{F}.$ Since $1 \le m < p,$ we see that $b \in \mathbb{F}.$ Thus, $a = b^p \in \mathbb{F}^p.$     
\end{proof}

\nonseppowerp*\label{prop:nonseppowerp2}
\begin{flushright}\hyperref[prop:nonseppowerp]{\upsym}\end{flushright}
\begin{proof}
    Since $f(x)$ is irreducible and not separable, we must have $f'(x) = 0.$ Write
    \begin{equation*} 
        f(x) = a_0 + a_1x + \cdots + a_nx^n
    \end{equation*}
    and note that
    \begin{equation*} 
        0 = f'(x) = a_1 + 2a_2x + \cdots + n a_n x^{n - 1}.
    \end{equation*}
    Thus, $ka_k = 0$ for all $k = 1, \ldots, n.$ If $\gcd(k, p) = 1,$ then we may cancel $k$ to see that $a_k = 0$ whenever $p \nmid k.$ Thus, $f(x)$ is of the form
    \begin{equation*} 
        f(x) = a_0 + a_px^p + \cdots + a_{mp} x^{mp}
    \end{equation*}
    for some $m \in \mathbb{N}.$ Thus, $g(x) = a_0 + a_p x + \cdots + a_{mp} x^m$ works.
\end{proof}

\perfectiffppower*\label{thm:perfectiffppower2}
\begin{flushright}\hyperref[thm:perfectiffppower]{\upsym}\end{flushright}
\begin{proof}
    \forward Suppose $\mathbb{F} \neq \mathbb{F}^p.$ Pick $\alpha \in \mathbb{F} \setminus \mathbb{F}^p.$ Then, $x^p - \alpha$ is irreducible (by \Cref{prop:xppolyirredorroot}) but not separable, by \Cref{prop:irredsepderiv}.

    \backward Suppose $\mathbb{F} = \mathbb{F}^p$ and $f(x) \in \mathbb{F}[x]$ is irreducible and not separable. By \Cref{prop:nonseppowerp}, we can write 
    \begin{equation*} 
        f(x) = \sum_{i = 0}^{m} a_i x^{ip}.
    \end{equation*} 
    Let $b_i \in \mathbb{F}$ be such that $a_i = b_i^p.$ Then,
    \begin{equation*} 
        f(x) = \sum_{i = 0}^{m} a_i x^{ip} = \sum_{i = 0}^{m} b_i^p x^{ip} = \left(\underbrace{\sum_{i = 0}^{m}b_i x^i}_{\in \mathbb{F}[x]}\right)^p,
    \end{equation*}
    contradicting the irreducibility of $f(x)$ in $\mathbb{F}[x].$
\end{proof}

\finitefieldperfect*\label{cor:finitefieldperfect2}
\begin{flushright}\hyperref[cor:finitefieldperfect]{\upsym}\end{flushright}
\begin{proof}
    Let $\mathbb{F}$ be a finite field of characteristic $p > 0.$ We show that $\mathbb{F} = \mathbb{F}^p.$ 

    Note that $\md{\mathbb{F}} = p^n$ for some $n \in \mathbb{N}.$ Thus, by Lagrange's theorem from group theory, we see that $\alpha^{p^n - 1} = 1$ for all $\alpha \in \mathbb{F}^\times.$ Thus, $\alpha^{p^n} = \alpha$ for all $\alpha \in \mathbb{F}.$ (This holds for $\alpha = 0$ as well.)

    Thus, given any arbitrary $\alpha \in \mathbb{F},$ put $\beta = \alpha^{p^{n - 1}}$ to get $\alpha = \beta^p \in \mathbb{F}^p.$
\end{proof}

\samemultirredpoly*\label{prop:samemultirredpoly2}
\begin{flushright}\hyperref[prop:samemultirredpoly]{\upsym}\end{flushright}
\begin{proof}
    Let $\overline{\mathbb{F}} \supset \mathbb{F}$ be an algebraic closure of $\mathbb{F}.$ Let $\alpha, \beta \in \overline{\mathbb{F}}$ be roots of $f.$ We have an $\mathbb{F}$-isomorphism $\sigma : \mathbb{F}(\alpha) \to \mathbb{F}(\beta)$ determined by $\alpha \mapsto \beta.$ 

    Thus, $\sigma$ can be extended to an automorphism $\tau$ of $\overline{\mathbb{F}}.$ Then, write $f(x) = (x - \alpha)^mh(x)$ where $m$ is the multiplicity of $\alpha$ and $h(x) \in \overline{\mathbb{F}}[x].$ Applying $\tau,$ we get
    \begin{equation*} 
        f(x) = f^\tau(x) = (x - \beta)^m h^\tau(x).
    \end{equation*}
    Thus, the multiplicity of $\beta$ is at least $m.$ By symmetry, we have equality.

    If $\chr(\mathbb{F}) = 0,$ then $f(x)$ is separable (\Cref{prop:irredsepderiv}) and thus, all roots are simple.

    Now, assume that $\chr(\mathbb{F}) =\vcentcolon p > 0.$ Let $n \in \mathbb{N}_0$ be the largest such that there exists a polynomial $g(x) \in \mathbb{F}[x]$ with $f(x) = g(x^{p^n}).$ (Note that we can take $g = f$ and $n = 0$ if no positive $n$ exists.)

    Then, $g$ is irreducible since $f$ is so. Moreover, $g$ must be separable. Indeed, if not, then we can write $g(x) = h(x^p)$ for some $h(x) \in \mathbb{F}[x],$ by \Cref{prop:nonseppowerp}. Then, $f(x) = h(x^{p^{n + 1}})$ contradicting maximality of $n.$

    Thus, $g(x)$ factors in $\overline{\mathbb{F}}$ as $g(x) = (x - r_1) \cdots (x - r_g)$ for distinct $r_g.$ Since $\overline{\mathbb{F}}$ is algebraically closed, we can find $s_1, \ldots, s_g$ necessarily distinct such that $s_i^{p^n} = r_i.$ Then, we have
    \begin{equation*} 
        f(x) = g(x^{p^n}) = (x - s_1)^{p^n} \cdots (x - s_g)^{p^n},
    \end{equation*}
    as desired.
\end{proof}

\separabledegreedef*\label{thm:separabledegreedef2}
\begin{flushright}\hyperref[thm:separabledegreedef]{\upsym}\end{flushright}
\begin{proof}
    If $\widetilde{\sigma} \in S_\sigma,$ then for any $x \in \mathbb{F},$ we have
    \begin{equation*} 
        (\lambda \circ \widetilde{\sigma})(x) = \lambda(\sigma(x)) = (\tau \circ \sigma^{-1})(\sigma(x)) = \tau(x).
    \end{equation*}
    Thus, $\psi$ actually maps into $S_\tau.$ Since $\lambda$ is an isomorphism, $\psi$ is easily seen to be a bijection. Explicitly, the inverse of $\psi$ can be seen to be $\widetilde{\tau} \mapsto \lambda^{-1} \circ \tau.$
\end{proof}

\towerlawsep*\label{thm:towerlawsep2}
\begin{flushright}\hyperref[thm:towerlawsep]{\upsym}\end{flushright}
\begin{proof}
    First, we show that the separable degree is multiplicative. Let $n \vcentcolon= [\mathbb{K} : \mathbb{E}]_s$ and $m \vcentcolon= [\mathbb{E} : \mathbb{F}]_s$ and $\sigma : \mathbb{F} \to \mathbb{L}$ be an embedding into an algebraically closed field $\mathbb{L}.$ 

    Let $\sigma_1, \ldots, \sigma_m : \mathbb{E} \to \mathbb{L}$ be extensions of $\sigma.$ Then, each $\sigma_i$ has extensions $\sigma_i^{(1)}, \ldots, \sigma_i^{(n)} : \mathbb{K} \to \mathbb{L}.$ Note that $\{\sigma_i^{(j)} : 1 \le i \le m,\; 1 \le j \le n\}$ has cardinality $mn.$ (All the extensions obtained are distinct.)

    Clearly, any embedding $\tau : \mathbb{K} \to \mathbb{L}$ extending $\sigma$ is obtained this way. ($\tau|_{\mathbb{E}}$ is $\sigma_i$ for some $i$ and thus, $\tau = \sigma_i^{(j)}$ for some $j.$) 

    Thus, $[\mathbb{K} : \mathbb{F}]_s = mn,$ as desired. 

    Now, since $\mathbb{E}/\mathbb{F}$ is finite, we can construct $\alpha_1, \ldots, \alpha_g$ such that $\mathbb{E} = \mathbb{F}(\alpha_1, \ldots, \alpha_g).$ We have the chain
    \begin{equation*} 
        \mathbb{F} \subset \mathbb{F}(\alpha_1) \subset \mathbb{F}(\alpha_1, \alpha_2) \subset \cdots \subset \mathbb{F}(\alpha_1, \ldots, \alpha_g).
    \end{equation*}
    Note that by \Cref{prop:sepdeglessthannordeg}, we know that 
    \begin{equation*} 
        [\mathbb{F}(\alpha_1, \ldots, \alpha_{i + 1}) : \mathbb{F}(\alpha_1, \ldots, \alpha_i)]_s \le [\mathbb{F}(\alpha_1, \ldots, \alpha_{i + 1}) : \mathbb{F}(\alpha_1, \ldots, \alpha_i)]
    \end{equation*}
    for all $i = 0, \ldots, g - 1.$ Since both degrees are multiplicative, we are done.
\end{proof}

\sepiffdegequal*\label{thm:sepiffdegequal2}
\begin{flushright}\hyperref[thm:sepiffdegequal]{\upsym}\end{flushright}
\begin{proof}
     Write $\mathbb{E} = \mathbb{F}(\alpha_1, \ldots, \alpha_n)$ for $\alpha_i \in \mathbb{E}.$ (Note that $\mathbb{E}/\mathbb{F}$ is a finite extension.)

    Put 
    \begin{equation*} 
        \mathbb{F}_0 \vcentcolon= \mathbb{F} \andd \mathbb{F}_i \vcentcolon= \mathbb{F}(\alpha_1, \ldots, \alpha_i),
    \end{equation*} 
    for $i = 1, \ldots, n.$

    \forward Assume $\mathbb{E}/\mathbb{F}$ is separable. Then, since each $\alpha_i$ is separable over $\mathbb{F},$ it follows that $\alpha_i$ is separable over $\mathbb{F}_i$ for $i = 1, \ldots, n.$ (Note that $\irr(\alpha_i, \mathbb{F}_i) \mid \irr(\alpha_i, \mathbb{F}).$) Thus, we see that 
    \begin{equation*} 
        [\mathbb{F}_{i} : \mathbb{F}_{i - 1}]_s = [\mathbb{F}_{i} : \mathbb{F}_{i - 1}]
    \end{equation*}
    for all $i = 1, \ldots, n.$ Multiplying gives $[\mathbb{E} : \mathbb{F}]_s = [\mathbb{E}:\mathbb{F}].$

    \backward Let $\alpha \in \mathbb{E}$ be arbitrary. Consider the tower
    \begin{equation*} 
        \mathbb{F} \subset \mathbb{F}(\alpha) \subset \mathbb{E}.
    \end{equation*}
    Since, we have the equality $[\mathbb{E} : \mathbb{F}]_s = [\mathbb{E} : \mathbb{F}],$ we also have the equality $[\mathbb{F}(\alpha) : \mathbb{F}]_s = [\mathbb{F}(\alpha) : \mathbb{F}],$ by the previous corollary. Thus, $\alpha$ is separable over $\mathbb{F},$ by \Cref{prop:sepdeglessthannordeg}.
\end{proof}

\compdecompsep*\label{prop:compdecompsep2}
\begin{flushright}\hyperref[prop:compdecompsep]{\upsym}\end{flushright}
\begin{proof}
    For both parts, we first note that if $\alpha \in \mathbb{K}$ is algebraic over $\mathbb{F},$ then it is also algebraic over $\mathbb{E}.$ Moreover, $\irr(\alpha, \mathbb{E}) \mid \irr(\alpha, \mathbb{F}).$ (The divisibility is in $\mathbb{E}[x].$)

    \forward Let $\alpha \in \mathbb{K}$ be arbitrary. Then, $\alpha$ is algebraic over $\mathbb{F}$ and hence, over $\mathbb{E}.$ Since $\irr(\alpha, \mathbb{F})$ has no repeated roots, neither does its factor $\irr(\alpha, \mathbb{E}).$ Thus, $\mathbb{K}/\mathbb{E}$ is separable. \\
    Now, let $\beta \in \mathbb{E}$ be arbitrary. Then, $\beta \in \mathbb{K}$ and thus, $\irr(\alpha, \mathbb{F})$ is separable. Thus, $\mathbb{E}/\mathbb{F}$ is separable.

    \backward Let $\alpha \in \mathbb{K}$ be arbitrary. Note that $\alpha$ is algebraic over $\mathbb{E},$ since it is separable over $\mathbb{E}.$ Let $\irr(\alpha, \mathbb{E}) = a_1 + \cdots + a_{n }x^{n - 1} + x^n \in \mathbb{E}[x].$ 

    Put 
    \begin{equation*} 
        \mathbb{F}_0 \vcentcolon= \mathbb{F} \andd \mathbb{F}_i \vcentcolon= \mathbb{F}(a_1, \ldots, a_i),
    \end{equation*} 
    for $i = 1, \ldots, n.$ By \forward, we see that $a_i$ is separable over $\mathbb{F}_{i - 1}$ and hence, 
    \begin{equation} \label{eq:002} \tag{$*$}
        [\mathbb{F}_i : \mathbb{F}_{i - 1}]_s = [\mathbb{F}_i : \mathbb{F}_{i - 1}]
    \end{equation} 
    for all $i = 1, \ldots, n.$

    Finally, put $\mathbb{F}_{n + 1} \vcentcolon= \mathbb{F}_n(\alpha).$ Then, \Cref{eq:002} holds for $i = n + 1$ as well, since $\alpha$ is separable over $\mathbb{F}_n.$ (Note that $\irr(\alpha, \mathbb{F}_n) = \irr(\alpha, \mathbb{E}),$ by our construction and the latter is separable by assumption.)

    Thus, upon multiplying, we get $[\mathbb{F}_{n + 1} : \mathbb{F}]_s = [\mathbb{F}_{n + 1} : \mathbb{F}]$ and hence, $\mathbb{F}_{n + 1}/\mathbb{F}$ is separable. Since $\alpha \in \mathbb{F}_{n + 1},$ we see that $\alpha$ is separable over $\mathbb{F}$ and hence, $\mathbb{K}/\mathbb{F}$ is separable.
\end{proof}

\sepdegdividesdeg*\label{prop:sepdegdividesdeg2}
\begin{flushright}\hyperref[prop:sepdegdividesdeg]{\upsym}\end{flushright}
\begin{proof}
    Clearly the statement is true if $\chr(\mathbb{F}) = 0$ since we have equality of degrees. Suppose $\chr(\mathbb{F}) =\vcentcolon p > 0.$

    First, suppose that $\mathbb{E} = \mathbb{F}(\alpha)$ for some $\alpha \in \mathbb{E}.$ Let $p(x) \vcentcolon= \irr(\alpha, \mathbb{F})$ and $d \vcentcolon= \deg(p(x)).$ By \Cref{prop:samemultirredpoly}, $p(x)$ factors in $\overline{\mathbb{F}}[x]$ as
    \begin{equation*} 
        p(x) = (x - \alpha)^{p^n} (x - \alpha_2)^{p^n} \cdots (x - \alpha_g)^{p^n},
    \end{equation*}
    where $\alpha_2, \ldots, \alpha_g \in \overline{\mathbb{F}}\setminus\{\alpha\}$ are distinct. Note that we have $gp^n = d.$ By \Cref{prop:rootsandextensions}, we know that $[\mathbb{F}(\alpha) : \mathbb{F}]_s = g.$ Thus, the statement is true.

    For a general finite extension $\mathbb{E}/\mathbb{F},$ write $\mathbb{E} = \mathbb{F}(\beta_1, \ldots, \beta_k)$ and use the fact that degrees are multiplicative.
\end{proof}

\section{Finite fields}
\uniquefinfields*\label{thm:uniquefinfields2}
\begin{flushright}\hyperref[thm:uniquefinfields]{\upsym}\end{flushright}
\begin{proof}
    Let $q \vcentcolon= \md{\mathbb{K}}$ and $p \vcentcolon= \chr(\mathbb{K}).$ Then, $q = p^n$ for some $n \in \mathbb{N}.$ Note that $\mathbb{K}^\times$ is a group of order $q - 1.$ By Lagrange's theorem, we have $a^{q - 1} = 1$ for all $a \in \mathbb{K}^\times.$ In turn, we get $a^q - a = 0$ for \emph{all} $a \in \mathbb{K}.$

    Hence, $\mathbb{K}$ is a splitting field of $x^q - x$ over $\mathbb{F}_p$ and so is $\mathbb{L}.$ By \Cref{thm:isosplitting}, $\mathbb{K}$ and $\mathbb{L}$ are isomorphic.
\end{proof}

\existencefinfields*\label{thm:existencefinfields2}
\begin{flushright}\hyperref[thm:existencefinfields]{\upsym}\end{flushright}
\begin{proof}
    Fix $n \in \mathbb{N}$ and let $q = p^n.$ $\overline{\mathbb{F}}_p$ contains a unique splitting field of $x^q - x =\vcentcolon f(x)$ over $\mathbb{F}_p.$ We show that this splitting field has $q$ elements. Consider
    \begin{equation*} 
        \mathbb{K} = \{\alpha \in \overline{\mathbb{F}}_p \mid f(\alpha) = 0\}.
    \end{equation*}
    Then, $\md{\mathbb{K}} = q$ since $f(x)$ is separable, by \Cref{thm:derivcritsep}. 

    Thus, $\mathbb{K}$ is the desired splitting field. Conversely any other field with $q$ elements would be the set of roots of $x^q - x$ and hence, we have uniqueness.

    We now show that $\overline{\mathbb{F}}_p = \bigcup_{k \ge 1}\mathbb{F}_{p^k}.$ Let $\alpha \in \overline{\mathbb{F}}_p$ and let $d \vcentcolon= \deg_{\mathbb{F}}(\alpha).$ Then, $[\mathbb{F}(\alpha) : \mathbb{F}_p] = d$ and hence, $\alpha \in \mathbb{F}(\alpha) = \mathbb{F}_{p^d}.$
\end{proof}


\xfourplusone*\label{prop:xfourplusone2}
\begin{flushright}\hyperref[prop:xfourplusone]{\upsym}\end{flushright}
\begin{proof}
    For irreducibility over $\mathbb{Z}[x],$ note that
    \begin{equation*} 
        f(x + 1) = x^4 + 4x^3 + 6x^2 + 4x + 2
    \end{equation*}
    is Eisenstein at the prime $2.$

    Now, let $p$ be a prime. If $p = 2,$ the we have $x^4 + 1 = (x + 1)^4.$ Let $p > 2$ be an odd prime. Then, $p^2 \equiv 1 \pmod{8}.$ Hence, we have
    \begin{equation*} 
        x^4 + 1 \mid x^8 - 1 \mid x^{p^2 - 1} - 1 \mid x^{p^2} - x.
    \end{equation*}
    For the sake of contradiction, assume that $x^4 + 1$ is irreducible and let $\alpha \in \overline{\mathbb{F}}_p$ be a root. Then, $[\mathbb{F}_p(\alpha) : \mathbb{F}_p] = \deg(x^4 + 1) = 4.$

    But $\alpha$ is clearly contained in the splitting of $x^{p^2} - x$ over $\mathbb{F}_p,$ which is $\mathbb{F}_{p^2} \subset \overline{\mathbb{F}}_p$ and so, $\alpha$ is contained in a degree $2$ extension. This is a contradiction.
\end{proof}

\xdxxnxdiv*\label{lem:xdxxnxdiv2}
\begin{flushright}\hyperref[lem:xdxxnxdiv]{\upsym}\end{flushright}
\begin{proof}
    Fix an algebraic closure $\overline{\mathbb{F}}_q.$ Since $f(x) \vcentcolon= x^{q^m} - x$ is separable, it suffices to show that every root of $f(x)$ is also a root of $x^{q^n} - x =\vcentcolon g(x).$ (Recall \Cref{prop:divisibilityofpoly}.)

    To this end, let $\alpha$ be a root of $f(x).$ We have 
    \begin{equation*} 
        \alpha^{q^m} = \alpha.
    \end{equation*}
    Now raise both sides to the power $q^m$ to obtain 
    \begin{equation*} 
        \alpha^{q^{2m}} = \alpha^{q^m} = \alpha.
    \end{equation*}
    Continue repeatedly to get 
    \begin{equation*} 
        \alpha^{q^{km}} = \alpha
    \end{equation*}
    for all $k \in \mathbb{N}.$ In particular, for $k = n/m,$ the above is true. This gives us that $g(\alpha) = 0,$ as desired.
\end{proof}

\irreddivsplitpoly*\label{lem:irreddivsplitpoly2}
\begin{flushright}\hyperref[lem:irreddivsplitpoly]{\upsym}\end{flushright}
\begin{proof}
    \forward Suppose $f(x) \mid x^{q^n} - x.$ Then, $\mathbb{F}_{q^n}$ contains all the roots of $f(x).$ Let $\alpha \in \overline{\mathbb{F}}_{q}$ be a root of $f(x).$ Thus, $\alpha \in \mathbb{F}_{q^n}.$ Considering the tower $\mathbb{F}_q \subset \mathbb{F}_q(\alpha) \subset \mathbb{F}_{q^n}$ shows that $\deg(f(x)) = [\mathbb{F}_q(\alpha) : \mathbb{F}_q]$ divides $[\mathbb{F}_{q^n} : \mathbb{F}_q] = n.$

    \backward Let $d \vcentcolon= \deg(f(x)) \mid n.$ Fix an algebraic closure $\overline{\mathbb{F}}_q$ of $\mathbb{F}_q.$ We show that every root of $f(x)$ in $\overline{\mathbb{F}}_q$ satisfies $x^{q^d} - x.$ Since this divides $x^{q^n} - x,$ we would be done.

    Let $\alpha \in \overline{\mathbb{F}}_q$ be a root of $f(x).$ Then, $[\mathbb{F}(\alpha) : \mathbb{F}] = d$ and thus, by \Cref{thm:existencefinfields}, we have that
    \begin{equation*} 
        \mathbb{F}(\alpha) = \mathbb{F}_{q^d} = \{\beta^{q^d} - \beta = 0 \mid \beta \in \overline{\mathbb{F}}_q\}.
    \end{equation*}
    (Note that any algebraic closure $\overline{\mathbb{F}}_q$ is also an algebraic closure of $\mathbb{F}_p \subset \mathbb{F}_q.$)

    Thus, $\alpha$ satisfies $x^{q^d} - x,$ as desired.
\end{proof}

\gaussnecklace*\label{thm:gaussnecklace2}
\begin{flushright}\hyperref[thm:gaussnecklace]{\upsym}\end{flushright}
\begin{proof}
    Note that $x^{q^n} - x$ is a separable polynomial. By \Cref{lem:irreddivsplitpoly}, we see that
    \begin{equation*} 
        x^{q^n} - x = \prod_{d \mid n} f_1^{(d)}(x) \cdots f_{N_q(d)}^{(d)}(x),
    \end{equation*}
    where $f_1^{(d)}(x), \ldots, f_{N_q(d)}^{(d)}(x)$ are all the irreducible monic polynomials of degree $d.$ 

    Equating the degrees of both sides gives
    \begin{equation*} 
        q^n = \sum_{d \mid n} d{N_q(d)}.
    \end{equation*}
    Thus, defining $f(n) \vcentcolon= q^n$ and $g(n) \vcentcolon= nN_q(n),$ we use \nameref{thm:mobiusinv} to conclude that
    \begin{equation*} 
        nN_q(n) = \sum_{d \mid n} \mu(d)q^{n/d}. \qedhere
    \end{equation*}
\end{proof}

\pet*\label{thm:pet2}
\begin{flushright}\hyperref[thm:pet]{\upsym}\end{flushright}
\begin{proof}
    If $\mathbb{F}$ is a finite, then $\mathbb{K}$ is also finite and hence, $\mathbb{K}^\times$ is cyclic by \Cref{thm:finsubgroupcyclic}. A generator of $\mathbb{K}^\times$ is clearly a primitive element of $\mathbb{K}$ over $\mathbb{F}.$ Clearly, there are only finitely many intermediate subfields as well. 

    Thus, we may assume that $\mathbb{F}$ is infinite.
    \begin{enumerate}[leftmargin=*]
        \item \forward Let $\mathbb{K} = \mathbb{F}(\alpha)$ for some $\alpha \in \mathbb{K}$ and let $f(x) \vcentcolon= \irr(\alpha, \mathbb{F}).$ Let $\mathbb{E}$ be an intermediate subfield. 

        Let $h_{\mathbb{E}}(x) \vcentcolon= \irr(\alpha, \mathbb{E}).$ Then, $h_{\mathbb{E}}(x) \mid f(x)$ for all intermediate subfields $\mathbb{E}.$

        Now, let $\mathbb{E}_0 \subset \mathbb{E}$ be the field obtained by adjoining the coefficients of $h(x)$ to $\mathbb{F}.$ Then, $\irr(\alpha, \mathbb{E}) = \irr(\alpha, \mathbb{E}_{0}).$ Note that we also have $\mathbb{K} = \mathbb{E}(\alpha) = \mathbb{E}_0(\alpha).$ Thus, we get that
        \begin{equation*} 
            [\mathbb{K} : \mathbb{E}] = \deg(\irr(\alpha, \mathbb{E})) = \deg(\irr(\alpha, \mathbb{E}_0)) = [\mathbb{K} : \mathbb{E}_0]
        \end{equation*}
        and hence, $\mathbb{E} = \mathbb{E}_0.$

        This shows that if $\mathbb{E}$ and $\mathbb{E}'$ are intermediate fields with $h_{\mathbb{E}} = h_{\mathbb{E}'},$ then $\mathbb{E} = \mathbb{E}'.$ Since $f(x)$ only has finitely many monic divisors, there are only finitely many intermediate subfields.

        \backward Suppose $\mathbb{K}/\mathbb{F}$ has finitely many intermediate subfields. Write $\mathbb{K} = \mathbb{F}(\alpha_1, \ldots, \alpha_n).$

        Assume that $n = 2.$ We show that $\mathbb{K}/\mathbb{F}$ has a primitive element. The general case then follows inductively. \\
        Thus, we have $\mathbb{K} = \mathbb{F}(\alpha_1, \alpha_2).$ 

        For each $c \in \mathbb{F},$ we have the subfield $\mathbb{F}(\alpha_1 + c\alpha_2).$ Since $\mathbb{F}$ is finite and there are only finitely many intermediate subfields, there exist $c \neq d \in \mathbb{F}$ such that 
        \begin{equation*} 
            \mathbb{F}(\alpha_1 + c\alpha_2) = \mathbb{F}(\alpha_1 + d\alpha_2) =\vcentcolon \mathbb{L}.
        \end{equation*} 
        We show that $\mathbb{L} = \mathbb{K}.$ (Note that $\mathbb{L}$ is primitive over $\mathbb{F}.$)

        By the above, we see that $(c - d)\alpha_2 \in \mathbb{L}$ and hence, $\alpha_2 \in \mathbb{L}.$ In turn, $\alpha_1 \in \mathbb{L}.$ Thus,
        \begin{equation*} 
            \mathbb{L} \subset \mathbb{K} = \mathbb{F}(\alpha_1, \alpha_2) \subset \mathbb{L}
        \end{equation*}
        and hence, we have equality.
        %
        %
        %
        \item Now, assume that $\mathbb{K}/\mathbb{F}$ is a finite separable extension. By the same inductive argument as earlier, it is sufficient to prove the existence of a primitive element when $\mathbb{K} = \mathbb{F}(\alpha, \beta)$ for some $\alpha, \beta \in \mathbb{K}.$ Fix an algebraic closure $\overline{\mathbb{F}}$ of $\mathbb{F}.$

        As earlier, we show that there exists $c \in \mathbb{F}$ such that
        \begin{equation} \label{eq:003} \tag{$*$}
            \mathbb{K} = \mathbb{F}(\alpha + c\beta).
        \end{equation}

        We now seek a condition on $c$ that implies \Cref{eq:003}. Let $n \vcentcolon= [\mathbb{K} : \mathbb{F}] = [\mathbb{K} : \mathbb{F}]_s.$ (Equality by \Cref{thm:sepiffdegequal}.) \\
        Then, by definition of separable degree, there exist $n$ embeddings $\sigma_1, \ldots, \sigma_n : \mathbb{K} \to \overline{\mathbb{F}}$ extending the natural inclusion. 

        Now, if $c \in \mathbb{F}$ is such that the \deff{conjugates} $\sigma_i(\alpha + c\beta)$ are distinct for $i = 1, \ldots, n,$ then this means that 
        \begin{equation*} 
            n = [\mathbb{K} : \mathbb{F}]_s \ge [\mathbb{F}(\alpha + c\beta) : \mathbb{F}]_s \ge n = [\mathbb{K} : \mathbb{F}]
        \end{equation*}
        and thus, \Cref{eq:003} holds. Our job now is to find such a $c \in \mathbb{F}$ for which the conjugates are distinct. 

        Let $c \in \mathbb{F}$ be arbitrary. Then, $\sigma_i(\alpha + c\beta) = \sigma_i(\alpha) + c\sigma_i(\beta).$ Consider the polynomial
        \begin{equation*} 
            f(x) \vcentcolon= \prod_{1 \le i < j \le n}\left[(\sigma_i(\alpha) - \sigma_j(\alpha)) + x(\sigma_i(\beta) - \sigma_j(\beta))\right] \in \mathbb{K}[x].
        \end{equation*}

        Thus, the conjugates of $c$ are distinct iff $f(c) \neq 0.$ Note that if $\sigma_i$ and $\sigma_j$ agree on $\alpha$ and $\beta,$ then $\sigma_i = \sigma_j$ since $\mathbb{K} = \mathbb{F}(\alpha, \beta).$ Thus, $f(x)$ above is not the zero polynomial. But since $\mathbb{F}$ is infinite, there exists $c \in \mathbb{F}$ such that $f(c) \neq 0$ and thus, we are done. \qedhere
    \end{enumerate}
\end{proof}

\section{Normal extensions}

\seppolysplittingfields*\label{prop:seppolysplittingfields2}
\begin{flushright}\hyperref[prop:seppolysplittingfields]{\upsym}\end{flushright}
\begin{proof}
    Let $a \in \mathbb{E} = \mathbb{F}(A)$ where $A$ is as in \Cref{rem:splitfamilyexists}. By \Cref{cor:FAdescfinite}, there is a finite set $\{a_1, \ldots, a_n\} \subset A$ such that $a \in \mathbb{F}(a_1, \ldots, a_n).$ Since each $a_i$ is a root of a separable, it is separable. By applying \Cref{cor:adjoiningsepissep} (repeatedly), we see that $\mathbb{F}(a_1, \ldots, a_n)/\mathbb{F}$ is a separable extension and thus, $a$ is separable over $\mathbb{F}.$
\end{proof}

\algebraicautomorphism*\label{lem:algebraicautomorphism2}
\begin{flushright}\hyperref[lem:algebraicautomorphism]{\upsym}\end{flushright}
\begin{proof}
    We only need to prove that $\sigma$ is onto. Let $\alpha \in \mathbb{E}$ be arbitrary. Put $p(x) \vcentcolon= \irr(\alpha, \mathbb{F}).$ Let $\mathbb{K} \subset \mathbb{E}$ be the subfield generated by the roots of $p(x)$ in $\mathbb{E}.$ Then, $\mathbb{K}$ is a finite dimensional vector space over $\mathbb{F}$ and $\alpha \in \mathbb{K}.$ Since $\sigma$ is an $\mathbb{F}$-embedding, it maps roots of $p(x)$ to roots of $p(x).$ Thus, $\sigma(\mathbb{K}) \subset \mathbb{K}.$

    But $\sigma$ is an $\mathbb{F}$-linear map and $\mathbb{K}$ is a finite dimensional $\mathbb{F}$-vector space. Thus, $\sigma|_{\mathbb{K}}$ is onto and contains $\alpha$ in its image.
\end{proof}

\normalequivalent*\label{thm:normalequivalent2}
\begin{flushright}\hyperref[thm:normalequivalent]{\upsym}\end{flushright}
\begin{proof}
    \ref{item:001} $\Rightarrow$ \ref{item:002}: Let $a \in \mathbb{E}$ and $p_a(x) = \irr(a, \mathbb{F}).$ If $b \in \overline{\mathbb{F}}$ is a root of $p_a(x),$ then there exists an $\mathbb{F}$-isomorphism $\mathbb{F}(a) \to \overline{\mathbb{F}}$ with $a \mapsto b.$ Extend this to a map $\sigma : \mathbb{E} \to \overline{\mathbb{F}}.$ By hypothesis, we have $\mathbb{E} = \sigma(\mathbb{E}) \ni b.$ Thus, $\mathbb{E}$ is a splitting field of the family $\{p_a(x)\}_{a \in \mathbb{E}}.$

    \ref{item:002} $\Rightarrow$ \ref{item:003}: Let $\mathbb{E}$ be a spitting field of $\{p_i(x)\}_{i \in I} \subset \mathbb{F}[x]$ over $\mathbb{F}.$ Let $f(x) \in \mathbb{F}[x]$ be an irreducible polynomial having a root $a \in \mathbb{E}.$ Let $b \in \overline{\mathbb{F}}$ be any root of $f(x).$ There exists an $\mathbb{F}$-embedding $\mathbb{F}(a) \to \overline{\mathbb{F}}$ with $a \mapsto b.$ Extend this to an $\mathbb{F}$-embedding $\sigma : \mathbb{E} \to \overline{\mathbb{F}}.$ Since $\sigma$ fixes $\mathbb{F},$ it maps roots of $p_i(x)$ to its roots for all $i \in I.$ Since $\mathbb{E}$ is generated by these roots, we see that $\sigma(\mathbb{E}) \subset \mathbb{E}$ and hence, $b \in \mathbb{E}.$

    \ref{item:003} $\Rightarrow$ \ref{item:001}: Let $\sigma : \mathbb{E} \to \overline{\mathbb{F}}$ be an $\mathbb{F}$-embedding. Let $a \in \mathbb{E}.$ Then, $p(x) \vcentcolon= \irr(\alpha, \mathbb{F})$ splits into linear factors in $\mathbb{E}.$ Since $\sigma(a)$ is a root of $p(x),$ we have $\sigma(a) \in \mathbb{E}.$ Thus, $\sigma(\mathbb{E}) \subset \mathbb{E}.$ By \Cref{lem:algebraicautomorphism}, we have that $\sigma$ is an automorphism. (Note that $\mathbb{E}/\mathbb{F}$ is indeed algebraic since $\mathbb{E} \subset \overline{\mathbb{F}}.$)
\end{proof}

\operationsonnormalexts*\label{prop:operationsonnormalexts2}
\begin{flushright}\hyperref[prop:operationsonnormalexts]{\upsym}\end{flushright}
\begin{proof}
    Fix an algebraic closure $\overline{\mathbb{F}} \supset \mathbb{K}.$

    Let $\sigma : \mathbb{E}_1\mathbb{E}_2 \to \overline{\mathbb{F}}$ be an $\mathbb{F}$-embedding. Then, $\sigma(\mathbb{E}_1\mathbb{E}_2) = \sigma(\mathbb{E}_1)\sigma(\mathbb{E}_2) = \mathbb{E}_1\mathbb{E}_2.$ Since this is true for all $\mathbb{F}$-embeddings, $\mathbb{E}_1\mathbb{E}_2/\mathbb{F}$ is normal, by \Cref{thm:normalequivalent}.

    Similar calculation shows the same for intersection as well.
\end{proof}

\section{Galois Extensions}
\orderofgalgroup*\label{prop:orderofgalgroup2}
\begin{flushright}\hyperref[prop:orderofgalgroup]{\upsym}\end{flushright}
\begin{proof}
    Fix an algebraic closure $\overline{\mathbb{F}} \supset \mathbb{E}.$ 

    Let $n \vcentcolon= [\mathbb{E} : \mathbb{F}]_s.$ Let $\sigma_1, \ldots, \sigma_n : \mathbb{E} \to \overline{\mathbb{F}}$ be $\mathbb{F}$-embeddings. Then, normality of $\mathbb{E}/\mathbb{F}$ implies that $\sigma_i \in \Gal(\mathbb{E}/\mathbb{F}).$ Thus, $\md{\Gal(\mathbb{E}/\mathbb{F})} \ge n.$

    On the other hand, if $\sigma \in \Gal(\mathbb{E}/\mathbb{F}),$ then $\sigma$ is an $\mathbb{F}$-embedding of $\mathbb{E}$ into $\overline{\mathbb{F}}$ upon composition by the inclusion. Thus, $\Gal(\mathbb{E}/\mathbb{F}) = \{\sigma_1, \ldots, \sigma_n\}.$
\end{proof}

\frobgenerates*\label{prop:frobgenerates2}
\begin{flushright}\hyperref[prop:frobgenerates]{\upsym}\end{flushright}
\begin{proof}
    Note that $\varphi$ does indeed fix $\mathbb{F}_q$ since any $a \in \mathbb{F}_q$ satisfies $x^q - x$ and thus, $\varphi \in \Gal(\mathbb{F}_{q^n}/\mathbb{F}_q).$

    By \Cref{prop:orderofgalgroup}, we know that $\md{\Gal(\mathbb{F}_{q^n}/\mathbb{F}_q)} = n.$ Thus, it suffices to show that $\varphi$ has order no less than $n.$ Let order of $\varphi$ be $d.$ It suffices to show that $d \ge n.$ Note that
    \begin{equation*} 
        \varphi^d(a) = a^{q^d}.
    \end{equation*}
    Thus, if $\varphi^d = \id_{\mathbb{F}_{q^n}},$ then every element of $\mathbb{F}_{q^n}$ satisfies $x^{q^d} - x.$ Thus, the degree is at least $q^n.$ Thus, $q^d \ge q^n$ or $d \ge n.$
\end{proof}

\fixfieldinjectiveIG*\label{thm:fixfieldinjectiveIG2}
\begin{flushright}\hyperref[thm:fixfieldinjectiveIG]{\upsym}\end{flushright}
\begin{proof}
    \phantom{hi}
    \begin{enumerate}[leftmargin=*]
        \item Clearly, $\mathbb{F} \subset \mathbb{K}^G,$ by definition of the Galois group. Only the reverse inclusion needs to be shown.

        Let $a \in \mathbb{K}^G.$ Then, $a$ is separable over $\mathbb{F}$ and hence, $[\mathbb{F}(a) : \mathbb{F}]_s = [\mathbb{F}(a) : \mathbb{F}],$ by \Cref{cor:adjoiningsepissep} and \Cref{thm:sepiffdegequal}.

        Thus, if $a \notin \mathbb{F},$ then $[\mathbb{F}(a) : \mathbb{F}] > 1$ and so, there is one non-identity embedding $\mathbb{F}(a) \to \mathbb{K},$ which would necessarily move $a.$ Thus, we must have $a \in \mathbb{F}.$
        %
        \item The fact that $\mathbb{K}/\mathbb{E}$ is separable follows from \Cref{prop:compdecompsep} and that it is normal follows from \Cref{prop:decompnormal}. Thus, $\mathbb{K}/\mathbb{E}$ is Galois.

        Now, if $\mathbb{E}, \mathbb{E}' \in \mathcal{I}$ are such that 
        \begin{equation*} 
            H \vcentcolon= \Gal(\mathbb{K}/\mathbb{E}) = \Gal(\mathbb{K}/\mathbb{E}') =\vcentcolon H',
        \end{equation*}
        then the first part gives
        \begin{equation*} 
            \mathbb{E} = \mathbb{K}^H = \mathbb{K}^{H'} = \mathbb{E}'
        \end{equation*}
        and thus, the map is an injection. \qedhere
    \end{enumerate}
\end{proof}

\degboundedbyn*\label{lem:degboundedbyn2}
\begin{flushright}\hyperref[lem:degboundedbyn]{\upsym}\end{flushright}
\begin{proof}
    Let $\beta \in \mathbb{E}$ be such that $[\mathbb{F}(\beta) : \mathbb{F}]$ is maximal. Note that $[\mathbb{F}(\beta) : \mathbb{F}] \le n,$ by hypothesis. It suffices to show that $\mathbb{E} = \mathbb{F}(\beta).$

    Suppose that $\mathbb{E} \neq \mathbb{F}(\beta).$ Then, pick $\alpha \in \mathbb{E} \setminus \mathbb{F}(\beta).$ Then, $\mathbb{F}(\alpha, \beta)$ is a separable extension and thus, there exists $\eta \in \mathbb{F}(\alpha, \beta) \subset \mathbb{E}$ such that $\mathbb{F}(\alpha, \beta) = \mathbb{F}(\eta),$ by the \nameref{thm:pet}.

    But this is a contradiction since $\mathbb{F}(\beta) \subsetneq \mathbb{F}(\alpha, \beta) = \mathbb{F}(\eta)$ implies that $[\mathbb{F}(\eta) : \mathbb{F}] > [\mathbb{F}(\beta) : \mathbb{F}],$ contradicting the maximality of $\beta.$
\end{proof}

\artin*\label{thm:artin2}
\begin{flushright}\hyperref[thm:artin]{\upsym}\end{flushright}
\begin{proof}
    Let $G = \{\sigma_1, \ldots, \sigma_n\}$ and $\md{G} = n.$
    \begin{enumerate}[leftmargin=*]
        \item Let $\alpha \in \mathbb{E}.$ Consider $S = \{\sigma_1(\alpha), \ldots, \sigma_n(\alpha)\}.$ Note that the elements written need not all be distinct. Let $r \vcentcolon= \md{S}.$ Without loss of generality, assume that $S = \{\sigma_1(\alpha), \ldots, \sigma_r(\alpha)\}.$

        Let $\tau \in G.$ Then, $\tau(S) = S.$\footnote{Each $\tau\sigma_i$ is an element of $G$ and $\tau\sigma_i(\alpha)$ are distinct for $i = 1, \ldots, r.$} Thus, $\tau|_{S}$ is a permutation of $S.$ Consider the polynomial
        \begin{equation*} 
            f(x) \vcentcolon= (x - \sigma_1(\alpha)) \cdots (x - \sigma_r(\alpha)).
        \end{equation*}
        The coefficients of $f(x)$ are symmetric functions of $\sigma_1(\alpha), \ldots, \sigma_r(\alpha)$ and thus, are fixed by every $\tau \in G,$ by the previous observation. Thus, $f(x) \in \mathbb{E}^G[x].$

        Note that $f(\alpha) = 0$ since one of the $\sigma_i$ is the identity map. Thus, $\irr(\alpha, \mathbb{E}^G) \mid f(x).$ Note that $f(x)$ has distinct roots, by construction. In particular, $\alpha$ is separable over $\mathbb{E}^G.$ Since $\alpha \in \mathbb{E}$ was arbitrary, this tells us that $\mathbb{E}/\mathbb{E}^G$ is separable.

        Moreover, $f(x)$ splits completely in $\mathbb{E}[x]$ and thus, so does $\irr(\alpha, \mathbb{E}^G).$ Thus, $\mathbb{E}/\mathbb{E}^G$ is normal as well and hence, Galois.

        To see that it is finite, note that $[\mathbb{E}^G(\alpha) : \mathbb{E}^G] = r \le n$ and thus, $[\mathbb{E} : \mathbb{E}^G],$ by \Cref{thm:artin}.
        %
        \item Note that $G \subset \Gal(\mathbb{E}/\mathbb{E}^G).$ As we noted earlier, $[\mathbb{E} : \mathbb{E}^G] \le n = \md{G}.$ 

        By \Cref{prop:orderofgalgroup}, we have $\Gal(\mathbb{E}/\mathbb{E}^G) = [\mathbb{E} : \mathbb{E}^G].$ Thus, comparing cardinalities gives $G = \Gal(\mathbb{E}/\mathbb{E}^G).$
        %
        \item Follows from the second part. \qedhere
    \end{enumerate}
\end{proof}

\galoissubgroupscompositum*\label{thm:galoissubgroupscompositum2}
\begin{flushright}\hyperref[thm:galoissubgroupscompositum]{\upsym}\end{flushright}
\begin{proof}
    The third assertion about the inclusion is obvious since $H_1 \supset H_2$ implies that every element fixed by $H_2$ is also fixed by $H_1.$ Since the extensions are Galois, the fields fields are precisely the $\mathbb{E}_i,$ by \Cref{thm:fixfieldinjectiveIG}.

    Note that $\mathbb{K}/\mathbb{E}_i$ is Galois and thus, $\mathbb{E}_i = \mathbb{K}^{H_i} \subset \mathbb{K}^{H_1 \cap H_2}$ for $i = 1, 2.$ Thus, $\mathbb{E}_1\mathbb{E}_2 \subset \mathbb{K}^{H_1 \cap H_2}.$

    On the other hand, if $\sigma \in G$ fixes $\mathbb{E}_1\mathbb{E}_2,$ then it fixes both $\mathbb{E}_1$ and $\mathbb{E}_2.$ Thus, $\Gal(\mathbb{K}/\mathbb{E}_1\mathbb{E}_2) \subset H_1 \cap H_2$ and so, $\mathbb{E}_1\mathbb{E}_2 \supset \mathbb{K}^{H_1 \cap H_2}.$

    Let $H \vcentcolon= \Gal(\mathbb{K}/(\mathbb{E}_1 \cap \mathbb{E}_2)).$ Note that $H_1, H_2 \subset H$ since every $\sigma \in H_i$ fixes $\mathbb{E}_i$ and thus, fixes the intersection. Thus, $\langle H_1, H_2\rangle \subset H$ or $\mathbb{E}_1 \cap \mathbb{E}_2 \subset \mathbb{K}^{\langle H_1, H_2\rangle}.$

    On the other hand, 
    \begin{equation*} 
        \mathbb{K}^{\langle H_1, H_2\rangle} \subset \mathbb{K}^{H_i} = \mathbb{E}_i
    \end{equation*}
    and thus,
    \begin{equation*} 
        \mathbb{K}^{\langle H_1, H_2\rangle} \subset \mathbb{E}_1 \cap \mathbb{E}_2.
    \end{equation*}
\end{proof}

\isomorphismgalois*\label{prop:isomorphismgalois2}
\begin{flushright}\hyperref[prop:isomorphismgalois]{\upsym}\end{flushright}
\begin{proof}
    \phantom{hi}
    \begin{enumerate}[leftmargin=*]
        \item We use \Cref{thm:normalequivalent}. Since $\mathbb{K}/\mathbb{F}$ is Galois, $\mathbb{K}$ is the splitting field of a family of separable polynomials $\{f_i(x) : i \in I\}$ over $\mathbb{F}.$ Then, $\lambda(\mathbb{K})$ is the splitting field of the separable polynomials $\{f^{\lambda}_i(x) : i \in I\}$ over $\lambda(\mathbb{F}).$
        %
        \item Define $\psi : \Gal(\mathbb{K}/\mathbb{F}) \to \Gal(\lambda(\mathbb{K})/\lambda(\mathbb{F}))$ be $\sigma \mapsto \lambda\sigma\lambda^{-1}.$ Clearly, $\psi$ is a well-defined homomorphism. It is easy to see that $\tau \mapsto \lambda^{-1}\tau\lambda$ acts as an inverse. \qedhere
    \end{enumerate}
\end{proof}

\galoisiffnormal*\label{thm:galoisiffnormal2}
\begin{flushright}\hyperref[thm:galoisiffnormal]{\upsym}\end{flushright}
\begin{proof}
    Let $\mathbb{E}/\mathbb{F}$ be Galois. Define 
    \begin{align*} 
        \psi : \Gal(\mathbb{K}/\mathbb{F}) &\to \Gal(\mathbb{E}/\mathbb{F})\\
        \psi(\sigma) &= \sigma|_{\mathbb{E}}.
    \end{align*}
    Note that the above is well-defined since $\mathbb{E}$ is normal and so, $\sigma|_{\mathbb{E}}$ is indeed an automorphism of $\mathbb{F}.$ (That it fixes $\mathbb{F}$ is obvious since $\sigma$ did so.) Clearly, $\psi$ is a homomorphism. However, now note that
    \begin{equation*} 
        \ker(\psi) = \{\sigma \in \Gal(\mathbb{K}/\mathbb{F}) \mid \sigma|_{\mathbb{E}} = \id_{\mathbb{E}}\} = \Gal(\mathbb{K}/\mathbb{E}).
    \end{equation*}
    Thus, $\Gal(\mathbb{K}/\mathbb{E})$ is a normal subgroup of $\Gal(\mathbb{K}/\mathbb{F}).$

    Moreover, since $\mathbb{K}/\mathbb{E}$ is an algebraic and normal extension, every automorphism of $\mathbb{E}$ can indeed be extended to an automorphism of $\mathbb{K}.$\footnote{First extend it to a map $\mathbb{K} \to \overline{\mathbb{E}} \supset \mathbb{K}.$ Normality then forces the map to be an automorphism of $\mathbb{K}.$} Thus, $\psi$ is a surjective map and thus,
    \begin{equation*} 
        \Gal(\mathbb{E}/\mathbb{F}) \cong \frac{\Gal(\mathbb{K}/\mathbb{F})}{\Gal(\mathbb{K}/\mathbb{E})}.
    \end{equation*}
    This proves one direction of the first part as well as the second part.

    Conversely, suppose that $\Gal(\mathbb{K}/\mathbb{E}) \unlhd \Gal(\mathbb{K}/\mathbb{F}).$ Let $\lambda : \mathbb{K} \to \mathbb{K}$ be any $\mathbb{F}$-isomorphism. We first show that $\lambda(\mathbb{E}) = \mathbb{E}.$ By \Cref{prop:isomorphismgalois}, we have
    \begin{equation*} 
        \Gal(\mathbb{K}/\mathbb{E}) = \lambda\Gal(\mathbb{K}/\mathbb{E})\lambda^{-1} = \Gal(\lambda(\mathbb{K})/\lambda(\mathbb{E})) = \Gal(\mathbb{K}/\lambda(\mathbb{E})).
    \end{equation*}
    Thus, $\Gal(\mathbb{K}/\mathbb{E}) = \Gal(\mathbb{K}/\lambda(\mathbb{E})).$ By \Cref{thm:fixfieldinjectiveIG}, we get $\mathbb{E} = \lambda(\mathbb{E}).$

    Now, to show that $\mathbb{E}/\mathbb{F}$ is normal, let $\sigma : \mathbb{E} \to \overline{\mathbb{F}} \supset \mathbb{E}$ be an $\mathbb{F}$-embedding. Then, $\sigma$ can be extended to an $\mathbb{F}$-embedding $\lambda : \mathbb{K} \to \overline{\mathbb{F}}.$ Since $\mathbb{K}/\mathbb{F}$ is normal, we have $\lambda(\mathbb{K}) = \mathbb{K}.$ By the above, we have $\sigma(\mathbb{E}) = \lambda(\mathbb{E}) = \mathbb{E}.$
\end{proof}

\FTGT*\label{thm:FTGT2}
\begin{flushright}\hyperref[thm:FTGT]{\upsym}\end{flushright}
\begin{proof}
    Note that only the first part needs to be proven. We have proven the others (\Cref{thm:galoisiffnormal}, \Cref{prop:orderofgalgroup}, \Cref{thm:galoissubgroupscompositum}).

    Let $\Psi : \mathcal{I} \to \mathcal{G}$ be the map $\mathbb{E} \mapsto \Gal(\mathbb{K}/\mathbb{E}).$ Let $\Phi : \mathcal{G} \to \mathcal{I}$ denote the map $H \mapsto \mathbb{K}^H.$ The fact that these maps reverse inclusion is obvious.

    By \Cref{thm:fixfieldinjectiveIG}, we know that $\Psi$ is an injection.

    Let $H \in \mathcal{G}.$ Then, $H$ is finite and is the Galois group of $\mathbb{K}/\mathbb{K}^H,$ by \Cref{thm:artin}. Thus, $\Psi$ is onto.

    Hence, $\Psi$ is bijective. Therefore, to show that $\Phi = \Psi^{-1},$ it suffices to show only that $\Phi \circ \Psi = \id_{\mathcal{I}}.$

    To this end, let $\mathbb{E} \in \mathcal{I}$ be arbitrary. Then, $H \vcentcolon= \Psi(\mathbb{K}/\mathbb{E})$ is the Galois group of $\mathbb{K}/\mathbb{E}.$ Thus, $\mathbb{E} = \mathbb{K}^H,$ by \Cref{thm:fixfieldinjectiveIG}. In other words
    \begin{equation*} 
        \mathbb{E} = \Phi(\Psi(\mathbb{E})). \qedhere
    \end{equation*}
\end{proof}

\ftagalois*\label{thm:ftagalois2}
\begin{flushright}\hyperref[thm:ftagalois]{\upsym}\end{flushright}
\begin{proof}
    Let $g(x) \in \mathbb{C}[x]$ be a non-constant polynomial. Then, $f(x) = g(x)\bar{g}(x)$ is a non-constant polynomial with real coefficients. Here, $\bar{g}(x)$ denotes the polynomial whose coefficients are complex conjugates of those of $g(x).$ Note that if $f(z) = 0$ for some $z \in \mathbb{C},$ then $g(z) = 0$ or $\bar{g}(z) = 0.$ If $\bar{g}(z) = 0,$ then $g(\bar{z}) = 0.$ In either case, $g$ has a complex root. Thus, it suffices to show that $f(x)$ has a root in $\mathbb{C}.$

    Let $\mathbb{E}$ denote a splitting field of $f(x)$ over $\mathbb{C}.$ Then, it is a splitting of $(x^2 + 1)f(x)$ over $\mathbb{R}.$ It suffices to show that $\mathbb{E} = \mathbb{C}.$ 

    Since $\mathbb{R}$ has no proper odd degree extensions,\footnote{Every odd degree real polynomial has a root in $\mathbb{R}.$} we see that $2 \mid [\mathbb{E} : \mathbb{R}].$ Thus, $G = \Gal(\mathbb{E}/\mathbb{R})$ has a Sylow-$2$ subgroup, say $S.$

    Now, if $S \neq G,$ then $\mathbb{E} \supset \mathbb{E}^S \supsetneq \mathbb{R}.$ However, note that 
    \begin{equation*} 
        [\mathbb{E}^S : \mathbb{R}] = \frac{[\mathbb{E} : \mathbb{R}]}{[\mathbb{E} : \mathbb{E}^S]} = \frac{\md{G}}{\md{S}}
    \end{equation*} 
    is odd. But $\mathbb{R}$ has no proper odd degree extension and thus, $S = G.$

    Thus, $G$ is a $2$-group. (That is, $\md{G} = 2^n$ for some $n \in \mathbb{N}.$) If $\md{G} = 2,$ then $\mathbb{C} = \mathbb{E}$ are we are done.

    Thus, $\md{G} \ge 4.$ Then, $\md{\Gal(\mathbb{E}/\mathbb{C})} \ge 2.$ Let $H \le \Gal(\mathbb{E}/\mathbb{C})$ be a subgroup of index $2.$ Then, $[\mathbb{E}^H : \mathbb{C}] = 2,$ which is a contradiction, since $\mathbb{C}$ has no quadratic extensions. Thus, $\mathbb{C} = \mathbb{E}.$ 
\end{proof}

\section{Cyclotomic Extensions}
\Gfabeliansubgroup*\label{prop:Gfabeliansubgroup2}
\begin{flushright}\hyperref[prop:Gfabeliansubgroup]{\upsym}\end{flushright}
\begin{proof}
    As $f(x)$ is separable, it has $n$ distinct roots in $\overline{\mathbb{F}}.$ Let $Z = \{z_1, \ldots, z_n\}$ be the set of roots and $\mathbb{E} = \mathbb{F}(z_1, \ldots, z_n).$ By \Cref{thm:finsubgroupcyclic}, we know that $Z$ is cyclic. The map $\psi : \Gal(\mathbb{E}/\mathbb{F}) \to \Aut(Z)$ given as $\sigma \mapsto \sigma|_Z$ is an injective group homomorphism. Note that $\Aut(Z) \cong (\mathbb{Z}/n\mathbb{Z})^\times,$ which proves the result.
\end{proof}

\nthrootsnonunity*\label{prop:nthrootsnonunity2}
\begin{flushright}\hyperref[prop:nthrootsnonunity]{\upsym}\end{flushright}
\begin{proof}
    Let $Z = \{z_1, \ldots, z_n\} \subset \mathbb{F}^\times$ be the set of roots of $x^n - 1.$ Let $r$ be a root of $f(x)$ in a splitting field $\mathbb{E}$ of $f(x).$ Then, $rz_1, \ldots, rz_n$ are $n$ distinct roots of $f(x)$ and hence, all the roots. Thus, $\mathbb{E} = \mathbb{F}(r).$

    Let $\sigma, \tau \in \Gal(\mathbb{E}/\mathbb{F}).$ Then, $\sigma(r) = z_{\sigma}r$ and $\tau(r) = z_{\tau}r$ for some $z_\sigma, z_\tau \in Z.$ In turn, we see $\sigma\tau(r) = z_{\sigma}z_{\tau}r.$ Thus, the map
    \begin{equation*} 
        \psi : \Gal(\mathbb{E}/\mathbb{F}) \to Z
    \end{equation*}
    defined by $\psi(\sigma) = z_{\sigma}$ is a group homomorphism. Moreover it is injective since every $\mathbb{F}$-automorphism of $\mathbb{E} = \mathbb{F}(r)$ is uniquely determined by its action on $r.$ Thus, $\G_f$ is isomorphic to a subgroup of $Z$ and we are done.
\end{proof}

\cyclotomicQ*\label{thm:cyclotomicQ2}
\begin{flushright}\hyperref[thm:cyclotomicQ]{\upsym}\end{flushright}
\begin{proof}
    We have $x^n - 1 = \Phi_n(x) h(x),$ where $h(x) \in \mathbb{Q}[x]$ is monic. Thus, by Gauss' Lemma, we have $\Phi_n(x) \in \mathbb{Z}[x].$

    Now, suppose that $p$ is prime not dividing $n.$ We contend that $\Phi(\zeta_n^p) = 0.$ Indeed, suppose not. Then, $h(\zeta_n^p) = 0.$ Alternately, $\zeta_n$ is a root of $h(x^p) \in \mathbb{Q}[x].$ But note that $\Phi_n(x)$ is the minimal polynomial of $\zeta_n$ over $\mathbb{Q}.$ Thus, we can write
    \begin{equation*} 
        h(x^p) = \Phi_n(x) g(x)
    \end{equation*}
    for monic $g(x) \in \mathbb{Z}[x].$ (Again, by Gauss' Lemma.) Reduce the above equation $\mod p$ to get
    \begin{equation*} 
        (\bar{h}(x))^p = \bar{\Phi}_n(x) \bar{g}(x).
    \end{equation*}
    (Note that every element $a \in \mathbb{Z}/p\mathbb{Z}$ satisfies $a^p = a$ and so, $\bar{h}(x^p) = \bar{h}(x))^p.$)

    From the above, we see that $\bar{\Phi}_n(x)$ and $\bar{h}(x)$ have a common factor of $\mathbb{F}_p[x].$ ($\mathbb{F}_p[x]$ is a UFD. Factorise both sides of the above equation into primes.)

    But this, in turn, implies that
    \begin{equation*} 
        x^n - 1 = \bar{\Phi}_n(x) \bar{h}(x)
    \end{equation*}
    in $\mathbb{F}_p[x].$ In particular, $x^n - 1 \in \mathbb{F}_p[x]$ has repeated roots in $\overline{\mathbb{F}}_p.$ This is a contradiction since $x^n - 1$ is separable because $\gcd(n, p) = 1.$

    Thus, $\Phi_n(\zeta_n^p) = 0.$ Now, if $a \in \mathbb{N}$ is any integer such that $\gcd(a, n) = 1,$ we factorise $a = p_1 \cdots p_r$ where $p_1, \ldots, p_r$ are (not necessarily distinct) primes not dividing $n.$ Now, note that $\zeta_n^{p_1}$ is again a primitive root of unity satisfying $\Phi_n(x).$ Thus, the above argument applies and we get $\Phi_n\left((\zeta_n^{p_1})^{p_2}\right) = 0.$ Again, since $\gcd(n, p_1p_2) = 1,$ we see that $\zeta_n^{p_1p_2}$ is a primitive root and so on. Thus, 
    \begin{equation*} 
        \Phi_n(\zeta_n^a) = 0
    \end{equation*}
    for every $a \in \mathbb{N}$ with $\gcd(a, n) = 1.$ As $a$ varies over all such integers, we see that every primitive root of unity is a root of $\Phi_n(x).$

    In particular, $\Phi_n(x)$ has $\varphi(n)$ many distinct roots, each with multiplicity $1.$ Thus, $[\mathbb{Q}(\zeta_n) : \mathbb{Q}] = \varphi(n).$

    By \Cref{prop:Gfabeliansubgroup}, we already know that $\Gal(\mathbb{Q}(\zeta_n)/\mathbb{Q})$ is isomorphic to a subgroup of $(\mathbb{Z}/n\mathbb{Z})^\times.$ By comparing cardinalities, we see that the groups are isomorphic.
\end{proof}

\cycloreccurence*\label{thm:cycloreccurence2}
\begin{flushright}\hyperref[thm:cycloreccurence]{\upsym}\end{flushright}
\begin{proof}
    Clearly, $\Phi_1(x) = x - 1.$ Let $\zeta_n$ be a primitive $n$-th root of unity. By \Cref{thm:cyclotomicQ}, we know that the other roots of $\Phi_n(x)$ are $\zeta_n^i$ for $i \in \{1, \ldots, n\}$ with $\gcd(i, n) = 1.$ Thus,
    \begin{equation*} 
        \Phi_n(x) = \prod_{\substack{1 \le i \le n \\ \gcd(n, i) = 1}}(x - \zeta_n^i).
    \end{equation*}
    In turn, we have
    \begin{equation*} 
        x^n - 1 = \prod_{d \mid n} \Phi_d(x).
    \end{equation*}
    (Factor the above in $\overline{\mathbb{Q}}$ and note that every root of the left side is a primitive $d$-th root of unity for some unique $d.$ Since the $n$-th roots form a group of order $n,$ we must have $d \mid n.$ Conversely, every such $d$-th root is indeed a root of $x^n - 1$ and no two different cyclotomic polynomials have a common root.)

    Thus,
    \begin{equation*} 
        \Phi_n(x) = \frac{x^n - 1}{\displaystyle\prod_{\substack{d \mid n\\ d < n}} \Phi_d(x)}. \qedhere
    \end{equation*}
\end{proof}

\cyclocyclic*\label{prop:cyclocyclic2}
\begin{flushright}\hyperref[prop:cyclocyclic]{\upsym}\end{flushright}
\begin{proof}
    Note that $\Gal(\mathbb{Q}(\zeta_p)/\mathbb{Q}) \cong (\mathbb{Z}/p\mathbb{Z})^\times,$  by \Cref{thm:cyclotomicQ}. Since $\mathbb{Z}/p\mathbb{Z} = \mathbb{F}_p$ is a finite field, \Cref{thm:finsubgroupcyclic} tells us that $\mathbb{F}_p^\times$ is cyclic.

    Recall the general fact about finite cyclic groups: given a cyclic group $G$ of order $n,$ there is a unique subgroup of \emph{index} $d$ for every $d \mid n.$ 

    Using this with the Galois correspondence gives the last statement.
\end{proof}

\cyclodisc*\label{lem:cyclodisc2}
\begin{flushright}\hyperref[lem:cyclodisc]{\upsym}\end{flushright}
\begin{proof}
    We shall use \nameref{prop:discderivative}. First, we note that we have
    \begin{equation*} 
        x^p - 1 = \Phi_p(x)(x - 1)
    \end{equation*}
    and thus,
    \begin{equation*} 
        px^{p - 1} = \Phi_p'(x)(x - 1) + \Phi_p(x).
    \end{equation*}
    Substituting $\zeta_n^i$ above for $i = 1, \ldots, p - 1$ gives
    \begin{equation*} 
        \frac{p}{\zeta_p^i} = \Phi_p'(\zeta_p^i)(\zeta_p^i - 1).
    \end{equation*}
    (We have used $\zeta_p^{p - 1} = \zeta_p^{-1}$ to simplify the left hand side.)

    Thus, we have
    \begin{equation} \label{eq:004} \tag{$\prod$}
        \prod_{i = 1}^{p - 1} \Phi_p'(\zeta_p^i) = \prod_{i = 1}^{p - 1} \frac{p}{\zeta_p^i(\zeta_p^i - 1)}.
    \end{equation}
    Note that we have the following expressions for $\Phi_p(x).$
    \begin{align*} 
        \Phi_p(x) &= (x - \zeta_p)(x - \zeta_p^2) \cdots (x - \zeta_p^{p - 1})\\
        &= x^{p - 1} + \cdots + x + 1.
    \end{align*}
    Thus,
    \begin{equation*} 
        \prod_{i = 1}^{p - 1} \zeta_p^i = (-1)^{p - 1}\andd \prod_{i = 1}^{p - 1}(\zeta_p^i - 1) = (-1)^{p - 1}\Phi_p(1).
    \end{equation*}
    Since $p$ is odd, we have $(-1)^{p - 1} = 1$ and putting it back in \Cref{eq:004} gives
    \begin{equation*} 
        \prod_{i = 1}^{p - 1} \Phi_p'(\zeta_p^i) = \frac{p^{p - 1}}{1 \cdot \Phi_p(1)} = p^{p - 2}.
    \end{equation*}

    Now using the formula of discriminant in terms of derivatives, we get
    \begin{equation*} 
        \disc(\Phi_p(x)) = (-1)^{\binom{p - 1}{2}}p^{p - 2} = (-1)^{\binom{p}{2}}p^{p - 2}. \qedhere
    \end{equation*}
\end{proof}

\uniquequadraticcyclosubfield*\label{prop:uniquequadraticcyclosubfield2}
\begin{flushright}\hyperref[prop:uniquequadraticcyclosubfield]{\upsym}\end{flushright}
\begin{proof}
    The existence and uniqueness of quadratic subfield is given by \Cref{prop:cyclocyclic}, since $2 \mid p - 1.$ 

    Note that $\disc(\Phi_p(x))$ is not a perfect square in $\mathbb{Q}.$ On the other hand, by definition of $\disc(\Phi_p(x)),$ it is clear that $\disc(\Phi_p(x))$ has a square root in any splitting field of $\Phi_p(x).$ (Recall \Cref{rem:discrepeatedroots}.) Thus, $\sqrt{\disc(\Phi_p(x))} \in \mathbb{Q}(\zeta_p) \setminus \mathbb{Q}.$

    Hence, this generates the unique quadratic extension. Moreover note that
    \begin{equation*} 
        (-1)^{\binom{p}{2}} = (-1)^{\frac{p - 1}{2}}.
    \end{equation*}
    Thus, the square root is real iff $p \equiv 1 \pmod{4}.$
\end{proof}

\quadincyclo*\label{cor:quadincyclo2}
\begin{flushright}\hyperref[cor:quadincyclo]{\upsym}\end{flushright}
\begin{proof}
    Any quadratic extension of $\mathbb{Q}$ is of the form $\mathbb{Q}(\sqrt{d})$ for some square free integer $d.$ (Negative or positive.)

    Let $\zeta_n \vcentcolon= \exp\left(\dfrac{2 \pi \iota}{n}\right).$ Note that $\zeta_n$ is indeed a primitive $n$-th root of unity.

    Let $p$ be an odd prime and note that $\mathbb{Q}(\sqrt{-p}) \subset \mathbb{Q}(\zeta_p)$ if $p \equiv 3 \pmod{4}$ and $\mathbb{Q}(\sqrt{p}) \subset \mathbb{Q}(\zeta_p)$ if $p \equiv 1 \pmod{4}.$ Also, $\sqrt{2} \in \mathbb{Q}(\zeta_8)$.\footnote{Note that $(\zeta_8 + \zeta_8^{-1})^2 = 2.$} Lastly, $\iota \in \mathbb{Q}(\zeta_4)$ and $\mathbb{Q}(\zeta_4) \subset \mathbb{Q}(\zeta_8).$

    Armed with these facts, we note that if $d = \pm p_1 \cdots p_r$ where $p_i$ are distinct odd primes, then,
    \begin{equation*} 
        \mathbb{Q}(\sqrt{d}) \subset \mathbb{Q}(\zeta_{p_1}, \ldots, \zeta_{p_r}, \zeta_4) = \mathbb{Q}(\zeta_{4p_1 \cdots p_r}).
    \end{equation*}
    On the other hand, if $d = \pm 2 p_1 \cdots p_r$ where $p_i$ are distinct odd primes, then,
    \begin{equation*} 
        \mathbb{Q}(\sqrt{d}) \subset \mathbb{Q}(\zeta_{p_1}, \ldots, \zeta_{p_r}, \zeta_8) = \mathbb{Q}(\zeta_{8p_1 \cdots p_r}).
    \end{equation*}
    In both the above equations, the last equality follows from \Cref{ex:compositecyclo}.
\end{proof}

\quadgeneratorcyclo*\label{prop:quadgeneratorcyclo2}
\begin{flushright}\hyperref[prop:quadgeneratorcyclo]{\upsym}\end{flushright}
\begin{proof}
    Note that $\zeta_p$ is a root of the quadratic 
    \begin{equation*} 
        x^2 - (\zeta_p + \zeta_p^{-1})x + 1 \in \mathbb{Q}(\zeta_p + \zeta_p^{-1}).
    \end{equation*}
    Thus, $[\mathbb{Q}(\zeta_p) : \mathbb{Q}(\zeta_p + \zeta_p^{-1})] \le 2.$ The degree will be $1$ iff $\mathbb{Q}(\zeta_p) = \mathbb{Q}(\zeta_p + \zeta_p^{-1}).$ However, note that the latter is contained in $\mathbb{R}$ whereas the former is not. Thus, $[\mathbb{Q}(\zeta_p) : \mathbb{Q}(\zeta_p + \zeta_p^{-1})] = 2.$

    Now, by \Cref{prop:cyclocyclic}, there is a unique intermediate subfield $\mathbb{E}$ of $\mathbb{Q}(\zeta_p)/\mathbb{Q}$ satisfying $[\mathbb{Q}(\zeta_p) : \mathbb{E}] = 2.$ Thus, $\mathbb{E} = \mathbb{Q}(\zeta_p + \zeta_p^{-1}).$
\end{proof}

\fixedfieldcyclosubgroup*\label{prop:fixedfieldcyclosubgroup2}
\begin{flushright}\hyperref[prop:fixedfieldcyclosubgroup]{\upsym}\end{flushright}
\begin{proof}
    Fix $p$ and let $\zeta \vcentcolon= \zeta_p.$

    Clearly, $\beta_H \in \mathbb{Q}(\zeta)^H$ since given any $\tau \in H,$ we have
    \begin{equation*} 
        \tau(\beta_H) = \tau\left(\sum_{\sigma \in H} \sigma(\zeta)\right) = \sum_{\sigma \in H} \tau\sigma(\zeta) = \beta_H,
    \end{equation*}
    since the map $\sigma \mapsto \tau\sigma$ is a bijection from $H$ to itself.

    Thus, $\mathbb{Q}(\beta_H) \subset \mathbb{Q}(\zeta)^H.$ By the Galois correspondence, we know that there exists a subgroup $K$ with $H \le K \le G$ such that
    \begin{equation*} 
        \mathbb{Q}(\beta_H) = \mathbb{Q}(\zeta)^K.
    \end{equation*}
    (In fact, we know exactly what this subgroup is, namely $\Gal(\mathbb{Q}(\zeta)/\mathbb{Q}(\beta_h)).$)

    It suffices to prove that $H = K.$ Suppose not. Then, $H \subsetneq K$ and $\beta$ is fixed by every element of $K.$ Pick $\tau \in K \setminus H.$ We show that $\tau(\beta_H) \neq \beta_H$ and reach a contradiction. 

    Note that the set
    \begin{equation*} 
        B = \{\zeta, \zeta^2, \ldots, \zeta^{p - 1}\}
    \end{equation*}
    is a $\mathbb{Q}$-basis for $\mathbb{Q}(\zeta).$ Moreover, the above is the set of all roots of $\irr(\zeta, \mathbb{Q}).$ Thus, any $\sigma \in G$ permutes $B.$ Since any $\sigma \in G$ is determined by its action on $\zeta,$ we see that the elements $\sigma(\zeta)$ are distinct for distinct $\sigma \in G$ and hence, linearly independent.

    Thus, if $\tau(\beta_H) = \beta_H,$ then there is some $\sigma \in H$ such that $\tau\sigma = \id_{\mathbb{Q}(\zeta)}$ but then $\tau = \sigma^{-1} \in H,$ a contradiction. Thus, $\tau(\beta_H) \neq \beta_H$ but that contradicts the fact that $K$ fixes $\mathbb{Q}(\beta_H).$ Thus, $\mathbb{Q}(\beta_H) = \mathbb{Q}(\zeta)^H.$
\end{proof}

\section{Abelian and Cyclic extensions}

\pequivonemodn*\label{lem:pequivonemodn2}
\begin{flushright}\hyperref[lem:pequivonemodn]{\upsym}\end{flushright}
\begin{proof}
    Let $k \in \mathbb{Z}$ be such that $\bar{k} \in \mathbb{F}_p$ is a root of $\bar{\Phi}_n(x).$ Then, $p \mid \Phi_n(k)$ in $\mathbb{Z}.$ In turn, $p \mid k^n - 1$ or $k^n \equiv 1 \pmod{p}.$

    We contend that $o(\bar{k}) = n$ in $(\mathbb{F}_p)^\times.$ Suppose not. Then, $m \vcentcolon= o(\bar{k}) < n.$ Then, $m \mid n$ and so, we have
    \begin{align*} 
        x^n - 1 &= \prod_{d \mid n} \Phi_d(x)\\
        &= \Phi_n(x) \prod_{\substack{d \mid n \\ d \neq n}} \Phi_d(x) \\
        &= \Phi_n(x) \cdot \prod_{d \mid m} \Phi_d(x) \cdot \prod_{\substack{d \nmid m \\ d \neq n}} \Phi_d(x)\\
        &= \Phi_n(x) (x^m - 1) h(x)
    \end{align*}
    for some $h(x) \in \mathbb{Z}[x].$ We have used \Cref{thm:cycloreccurence} in the above.

    Going $\mod p$ gives
    \begin{equation*} 
        x^n - 1 = \bar{\Phi}_n(x) (x^m - 1) \bar{h}(x).
    \end{equation*}
    However, note that $\bar{k}$ is a root of both $\bar{\Phi}_n(x)$ and $x^m - 1$ and so, $x^n - 1$ has repeated roots in $\mathbb{F}_p.$ This is a contradiction since $p \nmid n.$

    Thus, $o(\bar{k}) = n$ and in particular, $n \mid (p - 1),$ as desired.
\end{proof}

\infprimesmodone*\label{thm:infprimesmodone2}
\begin{flushright}\hyperref[thm:infprimesmodone]{\upsym}\end{flushright}
\begin{proof}
    Suppose to the contrary that $p_1, \ldots, p_r$ are all such primes. Let $m = np_1 \cdots p_r.$ Consider the cyclotomic polynomial $\Phi_m(x).$ Since it is monic (and non-constant), we have
    \begin{equation*} 
        \lim_{x\to \infty} \Phi_m(mx) = \infty.
    \end{equation*}
    In particular, there exists $k \in \mathbb{N}$ such that $\Phi_m(mk) \ge 2.$ Thus, it has a prime factor $p.$ Then,
    \begin{equation*} 
        p \mid (mk)^m - 1
    \end{equation*}
    and thus, $p \nmid (mk).$ Hence, $\gcd(p, n) = 1.$ Consequently, $p \neq p_1, \ldots, p_r.$ But $\bar{\Phi}_m(\overline{mk}) = 0$ and so, $p \equiv 1 \pmod{mk}.$ In turn, we have
    \begin{equation*} 
        p \equiv 1 \pmod{n},
    \end{equation*}
    a contradiction.
\end{proof}

\fingroupQextension*\label{thm:fingroupQextension2}
\begin{flushright}\hyperref[thm:fingroupQextension]{\upsym}\end{flushright}
\begin{proof}
    We may assume that $\md{G} =\vcentcolon n \ge 2.$ For $m \in \mathbb{N},$ define $C_m \vcentcolon= \mathbb{Z}/m\mathbb{Z}$ and $U(m) \vcentcolon= (\mathbb{Z}/m\mathbb{Z})^\times.$ We have
    \begin{equation*} 
        G \cong C_{n_1} \times \cdots \times C_{n_k}
    \end{equation*}
    for some integers $n_1, \ldots, n_k \ge 2$ with
    \begin{equation*} 
        n = n_1 \cdots n_k.
    \end{equation*}
    Let $p_1, \ldots, p_k$ be distinct primes such that $p_i \equiv 1 \pmod{n_i}$ for all $i = 1, \ldots, k.$ (Existence is given by \Cref{thm:infprimesmodone}.)

    Note that each $U(p_i)$ is cyclic with order $p_i - 1,$ a multiple of $n_i.$ Thus, there exists a subgroup $H_i \le U(p_i)$ with
    \begin{equation*} 
        \frac{U(p_i)}{H_i} \cong C_{n_i},
    \end{equation*}
    for each $i = 1, \ldots, k.$

    Thus, we have 
    \begin{equation*} 
        \frac{U(p_1) \times \cdots \times U(p_k)}{H_1 \times \cdots \times H_k} \cong C_{n_1} \times \cdots \times C_{n_k} \cong G.
    \end{equation*}

    By the Chinese Remainder Theorem, we have 
    \begin{equation*} 
        U(p_1) \times \cdots \times U(p_k) \cong U(m) \cong \Gal(\mathbb{Q}(\zeta_m)/\mathbb{Q}),
    \end{equation*} 
    where $m = p_1 \cdots p_k.$ Let $H$ be the subgroup of $\Gal(\mathbb{Q}(\zeta_m)/\mathbb{Q})$ corresponding to $H_1 \times \cdots \times H_k,$ under this isomorphism.

    Thus, we have
    \begin{equation*} 
        \frac{\Gal(\mathbb{Q}(\zeta_m)/\mathbb{Q})}{H} \cong G.
    \end{equation*}
    By the Galois correspondence, we see that $G \cong \Gal(\mathbb{Q}(\zeta_m)^H/\mathbb{Q}).$
\end{proof}

\dedekindcharacters*\label{thm:dedekindcharacters2}
\begin{flushright}\hyperref[thm:dedekindcharacters]{\upsym}\end{flushright}
\begin{proof}
    If $n = 1,$ then the statement is clearly true since $\chi_1$ does not take the value $0.$

    Suppose that $n \ge 2.$ Suppose that $\chi_1, \ldots, \chi_n$ are linearly dependent. Among all relations of linear dependence, choose $m \ge 2$ to be the one with the least number of non-zero coefficients. (We have $m \ge 2$ by the first line.) By renumbering, we may assume that we have
    \begin{equation*} 
        a_1 \chi_1 + \cdots + a_m \chi_m = 0
    \end{equation*}
    with $a_1, \ldots, a_m \in \mathbb{K} \setminus \{0\}.$ Thus, for any $g \in G,$ we have
    \begin{equation} \label{eq:005}
        a_1 \chi_1(g) + \cdots + a_m \chi_m(g) = 0.
    \end{equation} 
    Now, fix $g_0 \in G$ such that $\chi_1(g_0) \neq \chi_m(g_0).$ (Exists since $m \ge 1$ and $\chi_1 \neq \chi_m.$) Then, \Cref{eq:005} gives
    \begin{equation*} 
        a_1 \chi_1(g_0g) + \cdots + a_m \chi_m(g_0g) = 0
    \end{equation*}
    for all $g \in G.$ Since each $\chi_i$ is a homomorphism, we have
    \begin{equation} \label{eq:006}
        a_1 \chi_1(g_0)\chi_1(g) + \cdots + a_m \chi_m(g_0)\chi_m(g) = 0.
    \end{equation}
    Multiplying \Cref{eq:005} with $\chi_m(g_0)$ and subtracting from \Cref{eq:006} gives
    \begin{equation*} 
        a_1(\chi_1(g_0) - \chi_m(g_0))\chi_1(g) + \cdots + a_{m - 1}(\chi_{m - 1}(g_0) - \chi_{m}(g_0))\chi_{m - 1}(g) = 0.
    \end{equation*}
    The above holds for all $g \in G.$ But the first coefficient is non-zero. This is an equation of linear dependence with $\le m - 1$ non-zero coefficients. This is a contradiction.
\end{proof}

\primeigenvalue*\label{lem:primeigenvalue2}
\begin{flushright}\hyperref[lem:primeigenvalue]{\upsym}\end{flushright}
\begin{proof}
    The order of $\sigma$ is $n$ and hence, it satisfies $T^n - 1 = 0.$ (As an operator.)

    We contend that $T^n - 1 \in \mathbb{F}[T]$ is the minimal polynomial of $\sigma.$ Indeed, if $\sigma$ satisfies a polynomial of degree $m < n,$ then the distinct operators $\sigma, \ldots, \sigma^m$ are linearly dependent. This contradicts \Cref{thm:dedekindcharacters}, since we can view $\sigma, \ldots, \sigma^m$ as distinct characters of $\mathbb{E}^\times$ in $\mathbb{E}.$

    Hence, $T^n - 1$ is the minimal polynomial of $\sigma.$ Since $\zeta \in \mathbb{F}$ is a root of $T^n - 1,$ it is an eigenvalue of $\sigma.$
\end{proof}

(In case you're not aware of minimal polynomials: We have shown that $T^n - 1$ is the least degree polynomial that is satisfied by $\sigma.$ Use this to conclude that $T^n - 1$ divides every polynomial $p(T) \in \mathbb{F}[T]$ such that $p(\sigma) = 0.$ In particular, it must divide the characteristic polynomial (here we use Cayley Hamilton) and thus, $\zeta$ is an eigenvalue.)

\cyclicextprimroot*\label{thm:cyclicextprimroot2}
\begin{flushright}\hyperref[thm:cyclicextprimroot]{\upsym}\end{flushright}
\begin{proof}
    Let $G \vcentcolon= \Gal(\mathbb{E}/\mathbb{F}) = \langle \sigma\rangle$ and $\zeta \in \mathbb{F}$ be a primitive $n$-th root of unity. By \Cref{lem:primeigenvalue}, we see that $\zeta$ is an eigenvalue of $\sigma.$ Thus, there exists an eigenvector $a \in \mathbb{E}^\times$ such that $\sigma(a) = \zeta a$ and hence, $\sigma^i(a) = \zeta^i a.$

    Since $\zeta$ is a primitive $n$-th root, we see that $a, \zeta a, \ldots, \zeta^{n - 1}a$ are all distinct and hence, $a$ has at least $n$ Galois conjugates and so, 
    \begin{equation*} 
        [\mathbb{F}(a) : \mathbb{F}] \ge [\mathbb{F}(a) : \mathbb{F}]_s \ge n.
    \end{equation*}
    Since $[\mathbb{E} : \mathbb{F}] = n,$ we see that $\mathbb{F}(a) = \mathbb{E}.$ 

    Now, note that $\sigma(a^n) = (\sigma(a))^n = \zeta^na^n = a^n$ and thus, $a^n \in \mathbb{E}^G = \mathbb{F}.$
\end{proof}

\subfieldsofprimcyclic*\label{prop:subfieldsofprimcyclic2}
\begin{flushright}\hyperref[prop:subfieldsofprimcyclic]{\upsym}\end{flushright}
\begin{proof}
    Clearly, each $\mathbb{F}(a^d)$ is indeed an intermediate subfield of $\mathbb{E}/\mathbb{F}.$ We show that these are the only ones.

    Note that since $G$ is cyclic of order $n,$ it has exactly one subgroup of order $d,$ for every divisor $d$ of $n.$ In turn, $\mathbb{E}/\mathbb{F}$ has exactly one intermediate subfield of degree $n/d$ over $\mathbb{F}.$ We show that $\mathbb{F}(a^d)$ has this property and thus, we have covered all intermediate subfields.

    To this end, first note that $(a^d)^{n/d} \in \mathbb{F}$ and thus,
    \begin{equation*} 
        [\mathbb{F}(a^d) : \mathbb{F}] \le n/d.
    \end{equation*}
    On the other hand, $a$ satisfies $x^d - a^d \in \mathbb{F}(a^d)[x]$ and so,
    \begin{equation*} 
        [\mathbb{E} : \mathbb{F}(a^d)] \le d.
    \end{equation*}

    Since $[\mathbb{E} : \mathbb{F}] = n,$ the \nameref{thm:towerlaw} forces both of the above inequalities to be equalities.
\end{proof}

\artinschreier*\label{thm:artinschreier2}
\begin{flushright}\hyperref[thm:artinschreier]{\upsym}\end{flushright}
\begin{proof}
    \phantom{hi}
    \begin{enumerate}[leftmargin=*]
        \item Let $G \vcentcolon= \Gal(\mathbb{E}/\mathbb{F}) = \langle \sigma\rangle.$ Define the $\mathbb{F}$-linear map $T : \mathbb{E} \to \mathbb{E}$ as 
        \begin{equation*} 
            T \vcentcolon= \sigma - \id_{\mathbb{E}}.
        \end{equation*}
        Note that
        \begin{equation*} 
            \ker(T) = \{a \in \mathbb{E} : \sigma(a) = a\} = \mathbb{E}^G = \mathbb{F}.
        \end{equation*}
        Also, we have
        \begin{equation*} 
            T^p = (\sigma - \id_{\mathbb{E}})^p = \sigma^p - \id_{\mathbb{E}} = 0
        \end{equation*}
        and so, $\im(T^{p - 1}) \subset \ker(T) = \mathbb{F}.$ However, note that $T^{p - 1} \neq 0$ since that would give a non-trivial relation between the distinct $\mathbb{E}^\times$ characters $1, \sigma, \ldots, \sigma^{p - 1},$ contradicting \nameref{thm:dedekindcharacters}.

        Thus, $\im(T^{p - 1})$ is at least one dimensional over $\mathbb{F}.$ Since it is contained in $\mathbb{F},$ we have $\im(T^{p - 1}) = \mathbb{F}.$

        Let $b \in \mathbb{E}$ be such that $T^{p - 1}(b) = 1$ and put $a = T^{p - 2}(b) \in \mathbb{E}.$ Note that
        \begin{equation*} 
            \sigma(a) = T(a) + a = 1 + a.
        \end{equation*}
        Thus, $\sigma^i(a) = i + a$ for $i = 0, \ldots, p - 1.$ All of these are distinct. Thus, $\mathbb{E} = \mathbb{F}(a).$ (Compare the separability degree.)

        Now, note that
        \begin{equation*} 
            \sigma(a^p - a) = (1 + a)^p - (1 + a) = a^p - a
        \end{equation*}
        and thus, $a^p - a \in \mathbb{E}^G = \mathbb{F}.$
        %
        \item Suppose $b \in \mathbb{F}$ is such that $f(x) \vcentcolon= x^p - x - b$ has no root in $\mathbb{F}.$ Let $\mathbb{E}$ be a splitting field of $f(x)$ over $\mathbb{F}$ and let $\alpha \in \mathbb{E}$ be a root. Then, $\alpha + 1, \ldots, \alpha + (p - 1)$ are also roots. Thus,
        \begin{equation*} 
            \mathbb{E} = \mathbb{F}(\alpha, \ldots, \alpha + p - 1) = \mathbb{F}(\alpha).
        \end{equation*}

        Now, write $f(x) = g_1(x) \cdots g_r(x)$ for irreducible $g_i(x) \in \mathbb{F}[x].$ Now, if $\beta \in \mathbb{E}$ is a root of some $g_i(x),$ then $\mathbb{E} = \mathbb{F}(\beta),$ by the same argument as above and hence, each $g_i$ has degree $d \vcentcolon= [\mathbb{F}(\beta) : \mathbb{F}] > 1.$\footnote{Strictly greater since $\beta \notin \mathbb{F}.$} Thus, we have
        \begin{equation*} 
            p = \deg(f(x)) = rd.
        \end{equation*}
        Since $p$ is prime and $d > 1,$ we have $d = p$ and $r = 1.$

        Thus, $[\mathbb{E} : \mathbb{F}] = d = p$ and $G$ is generated by the automorphism $\sigma$ determined by $\sigma(\alpha) = \alpha + 1.$ \qedhere
    \end{enumerate}
\end{proof}

\section{Some Group Theory}

\pgroupssolvable*\label{prop:pgroupssolvable2}
\begin{flushright}\hyperref[prop:pgroupssolvable]{\upsym}\end{flushright}
\begin{proof}
    We prove this by induction on $n.$ If $n = 0, 1,$ then $G$ is abelian and hence, solvable. Suppose $n > 1$ and groups of order $p^{k}$ for $0 \le k < n$ are solvable.

    Let $Z(G) \unlhd G$ denote the center of $G.$ We have $\md{Z(G)} > 1$ and thus, $\overline{G} = G/Z(G)$ is a group of order $p^k$ for some $k < n.$ By induction hypothesis, $\overline{G}$ has a series
    \begin{equation*} 
        \overline{G} = \overline{G_0} \supset \overline{G_1} \supset \cdots \supset \overline{G_{s}} = 1.
    \end{equation*}
    By the correspondence theorem, the above lifts to a series
    \begin{equation*} 
        G = G_0 \supset G_1 \supset \cdots \supset G_{s} = Z(G) \supset G_{s + 1} \vcentcolon= 1.
    \end{equation*}
    Since the quotients $G_i/G_{i - 1}$ are isomorphic to $\overline{G_i}/\overline{G_{i - 1}}$ for $i = 1, \ldots, s,$ we see that the above is an abelian series except possibly at the right-most stage. However, $Z(G)$ is abelian and so, the right-most stage is verified as well.
\end{proof}

\commutatorresults*\label{prop:commutatorresults2}
\begin{flushright}\hyperref[prop:commutatorresults]{\upsym}\end{flushright}
\begin{proof}
    \phantom{hi}
    \begin{enumerate}[leftmargin=*]
        \item Let $g, h \in G.$ Then, 
        \begin{equation*} 
            f([g, h]) = f(g^{-1}h^{-1}gh) = f(g)^{-1}f(h)^{-1}f(g)f(h) = [f(g), f(h)].
        \end{equation*}
        Thus, $f(G') \subset H'$ and we may consider the homomorphism $f'|_{G'} : G' \to H'.$ Applying the result again gives
        \begin{equation*} 
            f(G^{(2)}) = f((G')') \subset (H')' = H^{(2)}.
        \end{equation*}  
        Inductively, we get the result for all $s \ge 1.$

        If $f$ is onto, then every commutator is in the image $f(G')$ and thus, $H' = f(G').$

        Thus, we may consider $f$ as an onto homomorphism $f : G' \to H'.$ As before, induction gives the result for all $s \ge 1.$
        %
        \item Let $a \in G.$ The inner automorphism $i_a : G \to G$ restricts to one of $K$ since $K \unlhd G.$ By the previous part, $i_a(K') = K'$ and thus, $K$ is normal. $G' \unlhd G$ follows since $G \unlhd G.$
        %
        \item $G/K$ is abelian $\iff$ $ghK = hgK$ for all $h, g \in G$ $\iff$ $g^{-1}h^{-1}gh \in K$ for all $g, h \in K$ $\iff$ $G' \le K.$ \qedhere
    \end{enumerate}
\end{proof}

\solvableifftrivialderiv*\label{prop:solvableifftrivialderiv2}
\begin{flushright}\hyperref[prop:solvableifftrivialderiv]{\upsym}\end{flushright}
\begin{proof}
    \forward Suppose $G$ is solvable. Then, there is an abelian series
    \begin{equation} \label{eq:007}
        1 = G_0 \unlhd G_1 \unlhd \cdots \unlhd G_s = G
    \end{equation}
    for $G.$ We show by induction on $s$ that $G^{(s)} = 1.$ 

    If $s = 1,$ then $G$ is abelian and $G^{(1)} = 1.$ Now, let $s > 1$ and assume that $G^{(s - 1)} = 1$ whenever $G$ has an abelian series of length $s - 1.$ Let $G$ be a group with an abelian series of length $s$ as in \Cref{eq:007}. Then,
    \begin{equation*} 
        1 = G_0 \unlhd G_1 \unlhd \cdots \unlhd G_{s - 1}
    \end{equation*}
    is an abelian series for $G_{s - 1}.$ By induction hypothesis, we have $G_{s - 1}^{(s - 1)} = 1.$ Since $G/G_{s - 1}$ is abelian, we have $G' \subset G_{s - 1},$ by \Cref{prop:commutatorresults}. Thus,
    \begin{equation*} 
        G^{(s)} = (G')^{(s - 1)} \subset (G_{s - 1})^{(s - 1)} = 1.
    \end{equation*}

    \backward Suppose that $G^{(s)} = 1$ for some $s.$ Then,
    \begin{equation*} 
        1 = G^{(s)} \unlhd G^{(s - 1)} \unlhd \cdots \unlhd G^{(1)} \unlhd G
    \end{equation*}
    is an abelian series.
\end{proof}

\deriveofquotient*\label{prop:deriveofquotient2}
\begin{flushright}\hyperref[prop:deriveofquotient]{\upsym}\end{flushright}
\begin{proof}
    Let $\pi : G \to G/K$ be the natural onto map. Then, $\pi(G^{(s)}) = (G/K)^{(s)},$ by \Cref{prop:commutatorresults}. By the correspondence theorem, we see that $\langle G^{(s)}, K\rangle/K = (G/K)^{s}.$ 
\end{proof}

\twoofthreesolvable*\label{prop:twoofthreesolvable2}
\begin{flushright}\hyperref[prop:twoofthreesolvable]{\upsym}\end{flushright}
\begin{proof}
    For the first two parts, let $s$ be such that $G^{(s)} = 1.$ (Exists by \Cref{prop:solvableifftrivialderiv}.) Using the same result, it suffices to show that $H^{(s)} = 1$ for the first two parts.
    \begin{enumerate}[leftmargin=*]
        \item $H^{(s)} \cong i(H^{(s)}) \subset G^{(s)} = 1.$
        \item Since $f$ is onto, we have $H^{(s)} = f(G^{(s)}) = 1.$
        \item There exist $s$ and $t$ such that $K^{(s)} = 1$ and $(G/K)^{(t)} = 1.$\\
        By \Cref{prop:deriveofquotient}, we have $(G/K)^{(t)} = \langle G^{(t)}, K\rangle/K.$ Since this is trivial, we have $G^{(t)} \subset K$ and so, $G^{(s + t)} \subset K^{(s)} = 1.$ \qedhere
    \end{enumerate}
\end{proof}

\refiningabelianseries*\label{prop:refiningabelianseries2}
\begin{flushright}\hyperref[prop:refiningabelianseries]{\upsym}\end{flushright}
\begin{proof}
    Since $G$ is solvable, there exists an abelian series
    \begin{equation*} 
        1 = G_0 \unlhd G_1 \unlhd \cdots \unlhd G_s = G.
    \end{equation*}
    We show that between $G_i$ and $G_{i + 1},$ we can insert groups $H_1^{(i)}, \ldots, H_{r_i}^{(i)}$ such that 
    \begin{equation*} 
        G_i \unlhd H_1^{(i)} \unlhd \cdots \unlhd H_{r_i}^{(i)} \unlhd G_{i + 1}
    \end{equation*}
    and each quotient above is cyclic of prime order.

    Note that by the correspondence theorem of subgroups of the original group and a quotient group, it suffices to prove that for $s = 1.$

    That is, assume that $G$ is an abelian group. We show that there exists a chain
    \begin{equation*} 
        1 = G_0 \unlhd \cdots \unlhd G_s = G
    \end{equation*}
    such that the quotients are cyclic of prime order.

    Let $\md{G} = p_1 \cdots p_n,$ where $p_i$ are (not necessarily distinct) primes. We prove the statement by induction on $n.$ If $n = 0$ or $1,$ the result is obvious. Assume $n \ge 2$ and that the result is true for $n - 1.$ Then, since $p_n \mid G,$ there exists an element $g \in G$ order $p_n.$ Let $G_1 \vcentcolon= \langle g\rangle.$ Then, $G_1 \unlhd G$ since $G$ is abelian. By induction, $G/G_1$ has a normal series where the quotients are cyclic of prime order. Lift that chain to complete the proof.
\end{proof}

\Angenerator*\label{lem:Angenerator2}
\begin{flushright}\hyperref[lem:Angenerator]{\upsym}\end{flushright}
\begin{proof}
    Clearly, every three cycle $(a b c) = (a c)(a b)$ is indeed in $A_n.$ Let $H \le A_4$ be the subgroup generated by the $3$-cycles.

    Let $\tau_1 = (i j)$ and $\tau_2 = (r s)$ be distinct transpositions. Then, we have
    \begin{equation*} 
        \tau_1\tau_2 = \begin{cases}
            (i j r)(r s j) & \tau_1 \text{ and } \tau_2 \text{ are disjoint},\\
            (irs) & \text{otherwise}.
        \end{cases}
    \end{equation*}
    Thus, $H$ contains all products of distinct pairs of transpositions. Since these generate $A_n$ (by definition), we have $H = A_n.$

    Now, assume that $n \ge 3.$ Recall that if $\sigma \in S_n$ is any permutation and $(j_1, \ldots, j_k)$ is a $k$-cycle, then
    \begin{equation*} 
        \sigma(j_1, \ldots, j_k)\sigma^{-1} = (\sigma(j_1), \ldots, \sigma(j_k)).
    \end{equation*}

    Now, let $(i j k)$ and $(r s t)$ be any two $3$-cycles. Define $\gamma \in S_n$ by
    \begin{equation*} 
        \gamma(u) \vcentcolon= \begin{cases}
            r & u = i, \\
            s & u = j, \\
            t & u = k, \\
            u & \text{otherwise}.
        \end{cases}
    \end{equation*}
    Clearly, the above is indeed a bijection from $[n]$ to itself. Then, we have
    \begin{equation*} 
        \gamma\cdot(i j k)\cdot\gamma^{-1} = (r s t).
    \end{equation*}
    Thus, if $\gamma$ is even, then the above shows that the $3$-cycles are conjugate in $A_n.$ Otherwise, pick distinct $u, v \in [n] \setminus \{r, s, t\}$ (exist since $n \ge 5$) and define $\sigma \vcentcolon= (i j) \cdot \gamma.$ Then,
    \begin{equation*} 
        \sigma\cdot(i j k)\cdot\sigma^{-1} = (u v)(r s t)(u v)^{-1} = (r s t)
    \end{equation*}
    and $\sigma \in A_n.$
\end{proof}

\SnAnnotsolvable*\label{thm:SnAnnotsolvable2}
\begin{flushright}\hyperref[thm:SnAnnotsolvable]{\upsym}\end{flushright}
\begin{proof}
    In view of \Cref{prop:twoofthreesolvable}, it suffices to show that $A_n$ is not solvable. We now show that $A_n' = A_n$ and hence, $A_n^{(s)} = A_n \neq 1$ for all $s \ge 1.$

    We actually show that every $3$-cycle $(i j k) \in A_n$ is a commutator. Then, by \Cref{lem:Angenerator}, it follows that $A_n' = A_n.$ Since $n \ge 5,$ we can distinct $r, v \in [n] \setminus \{i, j, k\}.$ Then, we have
    \begin{equation*} 
        [(jkv), (ikr)] = (vkj)(rki)(jkv)(ikr) = (vkj)(ivj) = (ikj). \qedhere
    \end{equation*}
\end{proof}

\Ansimple*\label{thm:Ansimple2}
\begin{flushright}\hyperref[thm:Ansimple]{\upsym}\end{flushright}
\begin{proof}
    Suppose $1 \neq N \unlhd A_n.$ We show that $N = A_n.$ If $N$ contains a $3$-cycle, then $N$ contains all $3$-cycles since $N$ is normal in $A_4$ and all $3$-cycles in $A_n$ are conjugates, by \Cref{lem:Angenerator}. But that lemma also tells us that $A_n$ is generated by $3$-cycles. Thus, we get $N = A_4.$ So, it suffices to show that $N$ contains a $3$-cycle.

    For $\sigma \in S_n$ and $j \in [n],$ we say that $j$ is a fixed point of $\sigma$ if $\sigma(j) = j.$ Pick $\sigma \in N \setminus \{1\}$ with maximum number of fixed points in $N \setminus \{1\}.$ We will show that $\sigma$ is a $3$-cycle.

    Write $\sigma = \tau_1 \cdots \tau_g$ where $\tau_1, \ldots, \tau_g$ are disjoint cycles of length at least $2$ and $g \ge 1.$ This is possible since $\sigma \neq 1.$

    \textbf{Case 1.} Each $\tau_i$ has length exactly $2.$ Then, since $\sigma$ is even, we have $g \ge 2.$\\
    Let $\tau_1 = (i j)$ and $\tau_2 = (r s).$ Since $n \ge 5,$ we can fix $k \in [n] \setminus \{i, j, r, s\}$ and set $\tau = (r s k) \in A_n.$ Consider the commutator
    \begin{equation*} 
        \sigma' = [\sigma, \tau] = \sigma^{-1}\underbrace{(\tau^{-1}\sigma\tau)}_{\in N} \in N
    \end{equation*}
    Let $\gamma = \tau_3 \cdots \tau_g$ so that
    \begin{equation*} 
        \sigma = (i j)(r s)\gamma
    \end{equation*}
    with $\gamma$ fixing $i, j, r, s.$ (Since the $\tau$s were disjoint.)

    Note that $\tau\sigma(k) = \tau\gamma(k) = \gamma(k),$ since $\gamma$ restricts to a permutation on $[n] \setminus \{i, j, r, s\}.$ On the other hand, we have $\sigma\tau(k) = \sigma(r) = s \neq k.$ Thus, $\tau\sigma \neq \sigma\tau$ and hence, $\sigma' \neq 1.$

    But note that $\sigma'$ fixes all fixed points of $\sigma,$ with possible exception of $k.$\footnote{By this, we mean that it was possible that $\sigma$ fixed $k.$} However, $\sigma'$ also fixes $i$ and $j.$ Thus, $\sigma' \in N \setminus \{1\}$ has more fixed points than $\sigma.$ A contradiction.

    \textbf{Case 2.} There is some $\tau_i$ with length at least $3.$ Since all the $\tau$s commute, we may assume $\tau_1 = (i j k \ldots)$ has length at least $3.$ If $\sigma = (i j k),$ then we are done.

    Otherwise, there are at least two other elements $r , s$ apart from $i, j, k$ that $\sigma$ does not fix.\footnote{If $g = 1,$ then $\tau_1$ is a cycle with odd number of elements since $\sigma \in A_n.$ If $g \ge 2,$ then $\tau_2$ has at least two elements which it moves.} Let $\tau = (r s k) \in A_n$ and consider $\sigma' = [\sigma, \tau].$ Note that $\sigma'(j) \neq j$ and thus, $\sigma' \neq 1.$ Thus, $\sigma' \in N \setminus \{1\}.$

    However, note that $\sigma'(i) = i$ and $\sigma'$ fixes every fixed element of $\sigma.$ (Since $\tau$ only moves those elements already moved by $\sigma.$) Thus, $\sigma'$ fixes more elements than $\sigma,$ a contradiction.

    Thus, $\sigma$ is a $3$-cycle and we are done.
\end{proof}

\genconsectranpose*\label{thm:genconsectranpose2}
\begin{flushright}\hyperref[thm:genconsectranpose]{\upsym}\end{flushright}
\begin{proof}
    For $n = 2,$ the theorem is clear. Assume $n \ge 3.$ Then, by \Cref{thm:gentranspose}, it suffices to show that every transposition is generated by the above list. Let $(ij) \in S_n$ be a transposition. If $i = 1$ or $j = 1,$ then it is in the above list. Assume $i \neq 1 \neq j.$ Then, we have
    \begin{equation*} 
        (ij) = (1i)(1j)(1i). \qedhere
    \end{equation*}
\end{proof}

\genconsectranposespecial*\label{thm:genconsectranposespecial2}
\begin{flushright}\hyperref[thm:genconsectranposespecial]{\upsym}\end{flushright}
\begin{proof}
    Again, by \Cref{thm:gentranspose}, it suffices to show that every transposition is generated by the above list.

    Let $(a \ b) \in S_n$ be a transposition. Without loss of generality, we assume that $a < b.$\footnote{Note that $(a \ b) = (b \ a).$} We show that $(a \ b)$ is a product of elements of the given list by induction on $b - a.$

    If $b - a = 1,$ then $(a \ b)$ is in the list itself. Assume $b - a = k > 1$ and the theorem is true for $k - 1.$ Note that we have
    \begin{equation*} 
        (a \ b) = (a \ a + 1)(a + 1 \ b)(a \ a + 1).
    \end{equation*}
    Since $(a + 1) - a = 1$ and $b - (a + 1) = k - 1,$ we are done.
\end{proof}

\gentransposecycle*\label{thm:gentransposecycle2}
\begin{flushright}\hyperref[thm:gentransposecycle]{\upsym}\end{flushright}
\begin{proof}
    The theorem is clearly true for $n = 2.$ Assume $n \ge 3.$

    By \Cref{thm:genconsectranposespecial}, it suffices to show the two elements above generate all transpositions of the form $(i \ i + 1)$ for $1 \le i < n.$ 

    Let $\sigma \vcentcolon= (12 \ldots n).$ Then, for $k = 1, \ldots, n - 2,$ we have
    \begin{equation*} 
        \sigma^k(1 \ 2)\sigma^{-k} = (\sigma^k(1) \ \sigma^k(2)) = (k + 1 \ k + 2). \qedhere
    \end{equation*}
\end{proof}

\genprimetranscycle*\label{cor:genprimetranscycle2}
\begin{flushright}\hyperref[cor:genprimetranscycle]{\upsym}\end{flushright}
\begin{proof}
    Let renumbering, we may assume that the transposition is $(12).$ The $p$-cycle is of the form $(1 a_1 \ldots a_{p - 2}) =\vcentcolon \sigma.$ Since $p$ is a prime, there exists $k$ such that $\sigma^k$ is of the form $(1 2 b_3 \ldots b_{p - 3}).$ By renumbering again, we may assume that $b_i = i$ for $i = 3, \ldots, n.$ By \Cref{thm:gentransposecycle}, we are done.
\end{proof}

\section{Galois Groups of Composite Extensions}

\galoisEFK*\label{prop:galoisEFK2}
\begin{flushright}\hyperref[prop:galoisEFK]{\upsym}\end{flushright}
\begin{proof}
    As $\mathbb{E}/\mathbb{F}$ is Galois, $\mathbb{E}$ is a splitting field of a family of separable polynomials $\{f_i(x)\}_{i \in I} \subset \mathbb{F}[x]$ over $\mathbb{F}.$ Then, $\mathbb{E}\mathbb{K}$ is splitting of the same family over $\mathbb{K}$ and thus, is Galois over $\mathbb{K}.$

    Now, assume that $\mathbb{K}/\mathbb{F}$ is also Galois. Then, $\mathbb{K}$ is a splitting field of a family of separable polynomials $\{g_j(x)\}_{j \in J} \subset \mathbb{F}[x]$ over $\mathbb{F}.$ Then, $\mathbb{E}\mathbb{F}$ is a splitting field the the family $\{f_i(x)\}_{i \in I} \cup \{g_j(x)\}_{j \in J} \subset \mathbb{F}[x]$ over $\mathbb{F}$ and thus, Galois.

    Now we show the same for the intersection. Let $\sigma : (\mathbb{E} \cap \mathbb{K}) \to \overline{F}$ be an $\mathbb{F}$-embedding. Extend it to an $\mathbb{F}$-embedding $\tau : \mathbb{E}\mathbb{K} \to \overline{F}.$\\
    Since $\mathbb{E}/\mathbb{F}$ and $\mathbb{K}/\mathbb{F}$ are normal, we get $\tau(\mathbb{E}) = \mathbb{E}$ and $\tau(\mathbb{K}) = \mathbb{K}.$ Therefore, $\tau(\mathbb{E} \cap \mathbb{K}) \subset \mathbb{E} \cap \mathbb{K}.$ But since $(\mathbb{E} \cap \mathbb{K})/\mathbb{F}$ is algebraic, we have $\tau(\mathbb{E} \cap \mathbb{K}) = \mathbb{E} \cap \mathbb{K},$ by \Cref{lem:algebraicautomorphism}. Thus, $\sigma(\mathbb{E} \cap \mathbb{K}) = \mathbb{E} \cap \mathbb{K},$ as desired and so, $\mathbb{E} \cap \mathbb{K}$ is Galois over $\mathbb{F}.$ (We have used \Cref{thm:normalequivalent}.)
\end{proof}

\secondiso*\label{prop:secondiso2}
\begin{flushright}\hyperref[prop:secondiso]{\upsym}\end{flushright}
\begin{proof}
    First note that $\sigma$ is actually well-defined. Indeed, if $\sigma \in \Gal(\mathbb{E}\mathbb{K}/\mathbb{K}),$ then $\sigma$ fixes $\mathbb{K}$ and in particular, $\mathbb{F}.$ Thus, so does $\sigma|_{\mathbb{E}}.$ That it is a homomorphism is clear.

    Now, suppose that $\sigma \in \Gal(\mathbb{E}\mathbb{K}/\mathbb{K})$ is such that $\sigma|_{\mathbb{E}} = \id_{\mathbb{E}}.$ By definition of the Galois group, we have $\sigma|_{\mathbb{K}} = \id_{\mathbb{K}}.$ Thus, $\sigma$ fixes both $\mathbb{E}$ and $\mathbb{K}$ and in turn, $\mathbb{E}\mathbb{K}.$ Hence, $\psi$ is injective.

    Let $H \vcentcolon= \im(\psi) \le G \vcentcolon= \Gal(\mathbb{E}/\mathbb{F}).$ Note that $\mathbb{E} \cap \mathbb{K} \subset \mathbb{E}^H.$ Indeed, if $a \in \mathbb{E} \cap \mathbb{K}$ and $\tau = \psi(\sigma) \in H$ for some $\sigma \in \Gal(\mathbb{E}\mathbb{K}/\mathbb{K}),$ then $\tau(a) = \sigma(a) = a,$ since $\sigma$ fixes $\mathbb{K}.$

    On the other hand, if $a \in \mathbb{E} \setminus (\mathbb{E} \cap \mathbb{K}),$ then $a \in \mathbb{E}\mathbb{K} \setminus \mathbb{K}$ and hence, there exists $\sigma \in \Gal(\mathbb{E}\mathbb{K}/\mathbb{K})$ such that $\sigma(a) \neq a.$ (See \Cref{thm:fixfieldinjectiveIG} and \Cref{rem:nonbasemoved}.) Thus, $a \notin \mathbb{E}^H.$ Hence, $\mathbb{E}^H = \mathbb{E} \cap \mathbb{K}.$ 

    Now, note $H$ is finite since $G$ is so. By \nameref{thm:artin}, we have 
    \begin{equation*} 
        \Gal(\mathbb{E}\mathbb{K}/\mathbb{K}) \cong H = \Gal(\mathbb{E}/\mathbb{E}^H) = \Gal(\mathbb{E}/(\mathbb{E} \cap \mathbb{K})). \qedhere
    \end{equation*}
\end{proof}

\secondisoindex*\label{cor:secondisoindex2}
\begin{flushright}\hyperref[cor:secondisoindex]{\upsym}\end{flushright}
\begin{proof}
    The first equation about the degrees follows from \Cref{prop:orderofgalgroup}.

    Thus,
    \begin{equation*} 
        [\mathbb{E}\mathbb{K} : \mathbb{F}] = [\mathbb{E}\mathbb{K} : \mathbb{K}][\mathbb{K} : \mathbb{F}] = [\mathbb{E} : \mathbb{E} \cap \mathbb{K}][\mathbb{K} : \mathbb{F}] = \frac{[\mathbb{E} : \mathbb{F}]}{[\mathbb{E} \cap \mathbb{K} : \mathbb{F}]}[\mathbb{K} : \mathbb{F}].
    \end{equation*}
    The last statement now follows.
\end{proof}

\galoiscompositeproduct*\label{thm:galoiscompositeproduct2}
\begin{flushright}\hyperref[thm:galoiscompositeproduct]{\upsym}\end{flushright}
\begin{proof}
    That $\psi$ is a well-defined homomorphism is clear. (Same proof as \Cref{prop:secondiso}.) Suppose $\sigma \in \ker(\psi).$ Then, $\sigma(a) = a$ for all $a \in \mathbb{E}$ and for all $a \in \mathbb{K}.$ Thus, $\sigma = \id_{\mathbb{E}\mathbb{K}}$ and hence, $\psi$ is injective.

    Suppose that $\mathbb{E} \cap \mathbb{K} = \mathbb{F},$ then by \Cref{cor:secondisoindex}, we have
    \begin{equation*} 
        \md{\Gal(\mathbb{E}\mathbb{K}/\mathbb{F})} = [\mathbb{E}\mathbb{K} : \mathbb{F}] = [\mathbb{E} : \mathbb{F}][\mathbb{K} : \mathbb{F}] = \md{\Gal(\mathbb{E}/\mathbb{F}) \times \Gal(\mathbb{K}/\mathbb{F})}
    \end{equation*}
    and thus, comparing cardinalities gives that $\psi$ is onto as well.
\end{proof}

\section{Normal Closure of an Algebraic Extension}
\normalclosureproperties*\label{prop:normalclosureproperties2}
\begin{flushright}\hyperref[prop:normalclosureproperties]{\upsym}\end{flushright}
\begin{proof}
    \phantom{hi}
    \begin{enumerate}[leftmargin=*]
        \item $\mathbb{K}$ is normal by \Cref{thm:normalequivalent}. That it contains $\mathbb{E}$ is trivial.
        \item Since $\mathbb{K}' \supset \mathbb{E},$ given any $a \in \mathbb{E},$ the polynomial $\irr(a, \mathbb{F})$ must factor completely in $\mathbb{K}',$ by definition of normality. Thus, it contains the splitting field of $\irr(a, \mathbb{F})$ over $\mathbb{F}.$ Since this is true for all $a \in \mathbb{E},$ $\mathbb{K}' \supset \mathbb{K}.$
        \item Write $\mathbb{E} = \mathbb{F}(a_1, \ldots, a_n).$ Then, consider the splitting field $\mathbb{K}$ of $\{\irr(a_i, \mathbb{F}) \mid 1 \le i \le n\}$ over $\mathbb{F}.$ Then, $\mathbb{K}$ is normal over $\mathbb{F}$ and any normal extension of $\mathbb{F}$ must contain $\mathbb{K}.$ Thus, $\mathbb{K}$ is the normal closure. $\mathbb{K}/\mathbb{F}$ is clearly a finite extension.
        \item Since $\irr(a, \mathbb{F})$ is separable over $\mathbb{F}$ for each $a \in \mathbb{E},$ we see that $\mathbb{K}/\mathbb{F}$ is normal, in view of \Cref{prop:seppolysplittingfields}.
        \item Let $K \vcentcolon= \Gal(\mathbb{K}/\mathbb{E}).$ Note that $K$ is not normal in $G$ since $\mathbb{E}/\mathbb{F}$ is not normal. (Recall \Cref{thm:galoisiffnormal}, which was for infinite extensions as well.)

        Thus, we see that $\mathbb{K}^H \supsetneq \mathbb{K}^K = \mathbb{E}.$ By \Cref{thm:galoisiffnormal} again, we see that $\mathbb{K}^H/\mathbb{F}$ is normal. Thus, $\mathbb{K}^H$ is a normal extension of $\mathbb{F}$ containing $\mathbb{E}$ which is contained in $\mathbb{K}.$ By minimality of $\mathbb{K},$ we have $\mathbb{K}^H = \mathbb{K}$ and thus, $H = 1.$ \qedhere
    \end{enumerate}
\end{proof}

\section{Solvability by Radicals}

\radextproperties*\label{prop:radextproperties2}
\begin{flushright}\hyperref[prop:radextproperties]{\upsym}\end{flushright}
\begin{proof}
    \phantom{hi}
    \begin{enumerate}[leftmargin=*]
        \item Let
        \begin{equation*} 
            \mathbb{F} = \mathbb{F}_0 \subset \mathbb{F}_1 \subset \cdots \subset \mathbb{F}_n = \mathbb{E}
        \end{equation*}
        and
        \begin{equation*} 
            \mathbb{E} = \mathbb{E}_0 \subset \mathbb{E}_1 \subset \cdots \subset \mathbb{E}_m = \mathbb{K}
        \end{equation*}
        be towers of simple radical extensions. Append the two together to see that $\mathbb{K}/\mathbb{F}$ is a radical extension.
        %
        \item Let
        \begin{equation*} 
            \mathbb{F} = \mathbb{F}_0 \subset \mathbb{F}_1 \subset \cdots \subset \mathbb{F}_n = \mathbb{E}
        \end{equation*}
        be a tower of simple radical extensions. Then, there exist $a_i \in \mathbb{F}_i$ such that
        \begin{equation*} 
            \mathbb{F}_i = \mathbb{F}_{i - 1}(a_i)
        \end{equation*}
        for $i = 1, \ldots, n,$ such that a power of $a_i$ is in $\mathbb{F}_{i - 1}.$

        Consider the tower
        \begin{equation*} 
            \mathbb{K} \subset \mathbb{K}(a_1) \subset \cdots \subset \mathbb{K}(a_1, \ldots, a_m) = \mathbb{E}\mathbb{K}.
        \end{equation*}

        Clearly, each extension above is a simple radical extension. Thus, $\mathbb{E}\mathbb{K}/\mathbb{K}$ is a radical extension. If $\mathbb{K}/\mathbb{F}$ is also radical, then the previous part gives us that $\mathbb{E}\mathbb{K}/\mathbb{F}$ is also radical. \qedhere
    \end{enumerate}
\end{proof}

\sepgaloisradical*\label{prop:sepgaloisradical2}
\begin{flushright}\hyperref[prop:sepgaloisradical]{\upsym}\end{flushright}
\begin{proof}
    Let $n \vcentcolon= [\mathbb{E} : \mathbb{F}].$ (Note that $n < \infty$ since $\mathbb{E}/\mathbb{F}$ is a radical extension.) Since $\mathbb{E}/\mathbb{F}$ is separable, there are $n$ distinct $\mathbb{F}$-embeddings
    \begin{equation*} 
        \sigma_1, \ldots, \sigma_n : \mathbb{E} \to \overline{\mathbb{F}}.
    \end{equation*}
    We show that compositum $\mathbb{K} = \sigma_1(\mathbb{E}) \cdots \sigma_n(\mathbb{E})$ is the smallest Galois extension of $\mathbb{F}$ containing $\mathbb{E}.$

    By the \nameref{thm:pet}, we know that $\mathbb{E} = \mathbb{F}(a)$ for some $a \in \mathbb{E}.$ Then, the roots of $p(x) \vcentcolon= \irr(a, \mathbb{F})$ in $\overline{\mathbb{F}}$ are precisely $\sigma_1(a), \ldots, \sigma_n(a).$ Let $\mathbb{K} \vcentcolon= \mathbb{F}(\sigma_1(a), \ldots, \sigma_n(a)).$ Then, $\mathbb{K}$ is a splitting field of a separable polynomial and hence, Galois over $\mathbb{K}.$ Moreover, it contains $\mathbb{E}.$ It is clear any such another field must contain $\mathbb{K}.$ Thus, $\mathbb{K}$ satisfies the hypothesis of the theorem.

    Note that we have $\mathbb{K} = \sigma_1(\mathbb{E}) \cdots \sigma_n(\mathbb{E}).$ Since $\sigma(\mathbb{E}_i) \cong \mathbb{E}_i,$ we see that each $\sigma(\mathbb{E}_i)/\mathbb{F}$ is a radical extension and thus, so is $\mathbb{K}/\mathbb{F},$ by \Cref{prop:radextproperties}.
\end{proof}

\solvradicalimpliesgroup*\label{thm:solvradicalimpliesgroup2}
\begin{flushright}\hyperref[thm:solvradicalimpliesgroup]{\upsym}\end{flushright}
\begin{proof}
    Let
    \begin{equation*} 
        \mathbb{F} = \mathbb{F}_0 \subset \mathbb{F}_1 \subset \cdots \subset \mathbb{F}_r = \mathbb{K}
    \end{equation*}
    be a sequence of simple radical extensions with $\mathbb{F}_i = \mathbb{F}_{i - 1}(a_i)$ such that $a_i^{n_i} \in \mathbb{F}_{i - 1}$ for $i = 1, \ldots, r$ and $\mathbb{K}$ contains a splitting field $\mathbb{E}$ of $f(x)$ over $\mathbb{F}.$ 

    Since $\chr(\mathbb{F}) = 0,$ we know that $\mathbb{K}/\mathbb{F}$ is separable. Thus, by \Cref{prop:sepgaloisradical}, we may assume that $\mathbb{K}/\mathbb{F}$ is Galois. Let $n \vcentcolon= n_1 \cdots n_r$ and $\mathbb{L}$ be the splitting field of $x^n - 1$ over $\mathbb{K}.$

    Then, $\mathbb{L} = \mathbb{K}(\omega)$ where $\omega$ is any primitive $n$-th root of unity. Consider the fields $\mathbb{L}_0, \ldots, \mathbb{L}_r = \mathbb{L}$ defined as $\mathbb{L}_i \vcentcolon= \mathbb{F}_i(\omega).$ 

    Since $\mathbb{K}/\mathbb{F}$ is Galois, $\mathbb{K}$ is the splitting of some $g(x) \in \mathbb{F}[x]$ over $\mathbb{F}.$ Then, $\mathbb{L}$ is a splitting field of $(x^n - 1)g(x) \in \mathbb{F}[x]$ over $\mathbb{F}.$ Thus, $\mathbb{L}$ is Galois over $\mathbb{F}$ and in turn, over all $\mathbb{L}_i.$

    Let $H_i \vcentcolon= \Gal(\mathbb{L}/\mathbb{L}_i)$ for $i = 0, \ldots, r.$ See the diagram (at the end of this proof) for a picture. By \hyperref[thm:FTGT]{FTGT}, we have 
    \begin{equation*} 
        \G_f \cong \Gal(\mathbb{E}/\mathbb{F}) \cong \frac{\Gal(\mathbb{L}/\mathbb{F})}{\Gal(\mathbb{L}/\mathbb{E})}.
    \end{equation*}
    (Note that $\mathbb{L}/\mathbb{E}$ is normal since $\mathbb{L}$ is a splitting field over $\mathbb{E}.$)

    Thus, to prove that $\G_f$ is solvable, it is enough to prove that $\Gal(\mathbb{L}/\mathbb{F})$ is solvable, by \Cref{prop:twoofthreesolvable}. 

    Note that $\mathbb{L}_i = \mathbb{L}_{i - 1}(a_i)$ and that $\mathbb{L}_{i - 1} \ni \omega$ and so, $\mathbb{L}_{i - 1}$ contains a primitive $n_i$-th root of unity. Thus, $\mathbb{L}_i$ is a splitting field of $x^{n_i} - a_i^{n_i} \in \mathbb{L}_{i - 1}$ over $\mathbb{L}_{i - 1}.$ Hence, $\mathbb{L}_i/\mathbb{L}_{i - 1}$ is Galois. Thus, $H_{i - 1} \unlhd H_i$ for all $i = 1, \ldots, r.$

    Moreover, by \Cref{prop:nthrootsnonunity}, we see that $\Gal(\mathbb{L}_i/\mathbb{L}_{i - 1})$ is cyclic. Since $H_{i}/H_{i - 1} \cong \Gal(\mathbb{L}_i/\mathbb{L}_{i - 1}),$ we see that
    \begin{equation*} 
        1 = H_r \unlhd H_{r - 1} \unlhd \cdots \unlhd H_0 = \Gal(\mathbb{L}/\mathbb{L}_0)
    \end{equation*}
    is an abelian series for $\Gal(\mathbb{L}/\mathbb{L}_0)$ and hence, it is solvable.

    On the other hand, we know that $\Gal(\mathbb{L}_0/\mathbb{F})$ is abelian, by \Cref{prop:Gfabeliansubgroup}. Again, by \Cref{prop:twoofthreesolvable}, we see that $\Gal(\mathbb{L}/\mathbb{F})$ is solvable, as desired.

    \begin{center}
        \begin{tikzcd}
                         &  &                                                                         &  & \mathbb{K}(\omega) = \mathbb{L}_r = \mathbb{L} \arrow[rr, no head]         &  & \Gal(\mathbb{L}/\mathbb{L}) = 1                            \\
                         &  & \mathbb{K} = \mathbb{F}_r \arrow[rru, no head]                                   &  &                                                                   &  &                                                            \\
                         &  &                                                                         &  & \mathbb{F}_{r-1}(\omega) = \mathbb{L}_{r-1} \arrow[rr, no head] \arrow[uu, no head] &  & \Gal(\mathbb{L}/\mathbb{L}_{r - 1}) = H_{r - 1} \arrow[uu, no head] \\
                         &  & \mathbb{F}_{r-1} \arrow[uu, no head] \arrow[rru, no head]                                 &  &                                                                   &  &                                                            \\
        \mathbb{E} \arrow[rruuu, no head] &  &                                                                         &  & \mathbb{F}_1(\omega) = \mathbb{L}_1 \arrow[rr, no head] \arrow[uu, dashed, no head] &  & \Gal(\mathbb{L}/\mathbb{L}_1) = H_1 \arrow[uu, dashed, no head]     \\
                                 &  & \mathbb{F}_1 \arrow[rru, no head] \arrow[uu, dashed, no head]                             &  &                                                                   &  &                                                            \\
                                 &  &                                                                         &  & \mathbb{F}_0(\omega) = \mathbb{L}_0 \arrow[rr, no head] \arrow[uu, no head]         &  & \Gal(\mathbb{L}/\mathbb{L}_0) = H_0 \arrow[uu, no head]             \\
                                 &  & \mathbb{F} = \mathbb{F}_0 \arrow[rru, no head] \arrow[uu, no head] \arrow[lluuu, no head] &  &                                                                   &  &                                                           
        \end{tikzcd}
    \end{center}    
\end{proof}

\solvgroupimpliesradical*\label{thm:solvgroupimpliesradical2}
\begin{flushright}\hyperref[thm:solvgroupimpliesradical]{\upsym}\end{flushright}
\begin{proof}
    Let $\mathbb{K}$ be a splitting field of $f(x)$ over $\mathbb{F}$ and $[\mathbb{K} : \mathbb{F}] = n.$ Let $\mathbb{L}$ be a splitting field of $x^n - 1$ over $\mathbb{K}$ and $\omega \in \mathbb{L}$ be a primitive $n$-th root of unity. We have $\mathbb{L} = \mathbb{K}(\omega).$ Put $\mathbb{E} = \mathbb{F}(\omega).$ Then, $\mathbb{L}$ is a splitting of $f(x)$ over $\mathbb{E}.$\footnote{The embedding is given as $\sigma \mapsto \sigma_{\mathbb{K}}.$ It is injective because $\sigma$ fixes $\omega$ to begin with.} Since $H = \Gal(\mathbb{L}/\mathbb{E})$ embeds into $\Gal(\mathbb{K}/\mathbb{F}) \cong \G_f,$ $H$ is also a solvable group, by \Cref{prop:twoofthreesolvable2}. Note that $\mathbb{E}/\mathbb{F}$ is a simple radical extension. Thus, if we show that $\mathbb{L}/\mathbb{E}$ is a radical extension, then we are done. (\Cref{prop:radextproperties}.)

    Since $H$ is finite, by \Cref{prop:refiningabelianseries}, we have an abelian series
    \begin{equation*} 
        1 = H_k \unlhd H_{k - 1} \unlhd \cdots \unlhd H_0 = H
    \end{equation*}
    such that $H_i/H_{i + 1}$ is cyclic of prime order $p_{i + 1}$ for $i = 0, \ldots, k - 1.$ Note that $n = p_1 \cdots p_k.$

    Let $\mathbb{E}_i \vcentcolon= \mathbb{L}^{H_i}$ for $i = 1, \ldots, k.$ Then, $[\mathbb{E}_i : \mathbb{E}_{i - 1}] = \md{H_{i - 1}/H_i} = p_i.$ Since $\mathbb{E}_{i - 1}$ contains $\omega,$ it has a primitive $p_i$-th root of unity. Thus, $\mathbb{E}_i/\mathbb{E}_{i - 1}$ is a simple radical extension, by \Cref{thm:cyclicextprimroot}. Thus, $\mathbb{L}/\mathbb{E}$ is a radical extension.
\end{proof}

\section{Solutions of Cubic and Quartic equations}

\section{Galois Groups of Quartic Polynomials}

\alternatingsubgroupdiscriminantroot*\label{thm:alternatingsubgroupdiscriminantroot2}
\begin{flushright}\hyperref[thm:alternatingsubgroupdiscriminantroot]{\upsym}\end{flushright}
\begin{proof}
    Note that any transposition maps $\delta$ to $-\delta.$ Thus, all permutations in $\G_f \cap A_n$ fix $\delta$ and in turn, $\mathbb{F}(\delta) \subset \mathbb{E}^{\G_f \cap A_n}.$

    Let $d = [\G_f : \G_f \cap A_n].$ Then, $d \le 2.$ If $d = 1,$ then $\G_f = \G_f \cap A_n$ which means that
    \begin{equation*} 
        \mathbb{F}(\delta) \subset \mathbb{E}^{\G_f \cap A_n} = \mathbb{E}^{\G_f} = \mathbb{F} \subset \mathbb{F}(\delta),
    \end{equation*}
    and we are done.

    Now, assume $d = 2.$ Then, $\G_f \cap A_n \neq A_n$ and hence, $\G_f$ has an odd permutation $\sigma.$ Since $\delta \neq 0,$ and $\chr(\mathbb{F}) \neq 2,$ we see that $\delta \neq -\delta$ and thus, $\delta$ is not fixed by $\sigma.$ Thus, $\delta \notin \mathbb{F}$ and $\mathbb{F}(\delta)$ is a degree $2$ extension of $\mathbb{F}.$\footnote{Recall that $\delta^2$ is the discriminant of $f(x)$ and thus, is an element of $\mathbb{F}.$ } But $\mathbb{E}^{\G_f \cap A_n}$ is also a degree $2$ extension of $\mathbb{E}^{\G_f} = \mathbb{F},$ since $d = 2.$ Since we already have the inclusion $\mathbb{F}(\delta) \subset \mathbb{E}^{\G_f \cap A_n},$ we are done.
\end{proof}

\transitivegaloisgroupiffirreducible*\label{thm:transitivegaloisgroupiffirreducible2}
\begin{flushright}\hyperref[thm:transitivegaloisgroupiffirreducible]{\upsym}\end{flushright}
\begin{proof}
    Let $r_1, \ldots, r_n \in \overline{\mathbb{F}}$ be the roots of $f(x),$ and let $\mathbb{E} = \mathbb{F}(r_1, \ldots, r_n)$ be the splitting field. 

    \forward Suppose $f(x)$ is irreducible. Let $i, j \in \{1, \ldots, n\}$ be distinct. Since $f(x)$ is irreducible, we see that 
    \begin{equation*} 
        \irr(r_i, \mathbb{F}) = f(x) = \irr(r_j, \mathbb{F}).
    \end{equation*}
    By \Cref{prop:isocarryingalphtobet}, there exists an $\mathbb{F}$-isomorphism $\tau : \mathbb{F}(r_i) \to \mathbb{F}(r_j).$ Extending this to an isomorphism $\sigma : \mathbb{E} \to \mathbb{E}$ gives $\sigma \in \G_f$ with $\sigma(r_i) = r_j.$

    \backward Suppose $\G_f$ is a transitive subgroup of $S_n.$ For the sake of contradiction, assume that
    \begin{equation*} 
        f(x) = g(x)h(x)
    \end{equation*}
    for polynomials $g(x), h(x) \in \mathbb{F}[x]$ of positive degree. Let $r$ be a root of $g(x)$ and $s$ of $h(x).$ By transitivity, there exists an $\mathbb{F}$-automorphism $\sigma : \mathbb{E} \to \mathbb{E}$ such that $\sigma(r) = s.$ But $g(x)$ and $h(x)$ are fixed by $\sigma$ and so, we see that $s$ is a root of $g(x)$ and $h(x)$ both. But this is a contradiction since $f(x)$ has distinct roots.
\end{proof}

\stabiliserofti*\label{prop:stabiliserofti2}
\begin{flushright}\hyperref[prop:stabiliserofti]{\upsym}\end{flushright}
\begin{proof}
    We prove this for $i = 1,$ for ease of notation. Clearly, $H_1$ fixes $t_i$ and thus, $H_1 \subset \Stab t_1.$ Moreover, note that
    \begin{equation*} 
        S_4 = H_1 \sqcup (13)H_1 \sqcup (14)H_1
    \end{equation*}
    and $(13),$ $(14)$ do not fix $t_1.$ Thus, $H_1 = \Stab t_1.$ 
\end{proof}

\galoisintersectklein*\label{prop:galoisintersectklein2}
\begin{flushright}\hyperref[prop:galoisintersectklein]{\upsym}\end{flushright}
\begin{proof}
    Since $V$ fixes each $t_i,$ we have $\mathbb{F}(\underline{t}) \subset \mathbb{E}^{\G_f \cap V}.$ Note that 
    \begin{equation*} 
        V = H_1 \cap H_2 \cap H_3.
    \end{equation*}
    Thus, if $\sigma \in \G_f$ fixes $t_1, t_2, t_3,$ then $\sigma \in V,$ by \Cref{prop:stabiliserofti}. Thus, $\Gal(\mathbb{E}/\mathbb{F}(\underline{t})) \subset \G_f \cap V$ and thus, we get the reverse inclusion $\mathbb{E}^{\G_f \cap V} = \mathbb{F}(\underline{t})$ as well.

    The second equality now follows since $V$ is normal in $S_4$ and thus, $\G_f \cap V$ is normal in $\G_f.$
\end{proof}

\resolventquarticrootinF*\label{prop:resolventquarticrootinF2}
\begin{flushright}\hyperref[prop:resolventquarticrootinF]{\upsym}\end{flushright}
\begin{proof}
    Recall that the roots of the resolvent are precisely the $t_i.$ 

    \forward Suppose $t_i \in \mathbb{F}$ for some $i.$ Thus, $\G_f$ fixes $t_i$ and hence, $\G_f \subset H_i.$

    \backward If $\G_f \subset H_i$ for some $i,$ then every $\sigma \in \G_f$ fixes $t_i$ and thus, $t_i \in \mathbb{F}.$
\end{proof}

\classifyingirreduciblequarticgalois*\label{thm:classifyingirreduciblequarticgalois2}
\begin{flushright}\hyperref[thm:classifyingirreduciblequarticgalois]{\upsym}\end{flushright}
\begin{proof}
    Let $r_1, \ldots, r_4 \in \overline{\mathbb{F}}$ be the roots of $f(x)$ and $\mathbb{E} = \mathbb{F}(r_1, \ldots, r_4)$ be the splitting field. 

    Since $f(x)$ is irreducible in $\mathbb{F}[x],$ we see that $\G_f$ is a transitive subgroup. Thus, $\md{\G_f} \in \{4, 8, 12, 24\}.$ Also, $\md{\G_f \cap V} \in \{1, 2, 4\}.$ Note that $\mathbb{F}(\underline{t})$ is the splitting field of $r(x).$ Thus, $\md{\G_f/\G_f \cap V} = \md{\G_r} \in \{1, 2, 3, 6\},$ where the first equality follows from \Cref{prop:galoisintersectklein}. \\
    Since the first and third sets have no element in common, it follows that $\md{\G_f \cap V} > 1.$ Moreover, since we must have
    \begin{equation*} 
        \md{V \cap \G_f} \cdot \md{\frac{\G_f}{V \cap \G_f}} = \md{\G_f},
    \end{equation*}
    the possibilities are reduced to the following sets
    \begin{equation*} 
        \{2, 4\} \cdot \{1, 2, 3, 6\} = \{4, 8, 12, 24\}.
    \end{equation*}

    \begin{enumerate}[leftmargin=*]
        \item Assume that $r(x)$ is irreducible in $\mathbb{F}[x]$ and $\disc(r(x)) \in \mathbb{F}^2.$ By \Cref{ex:galsepcubic}, it follows that $\G_r \cong A_3$ and hence, $\md{\G_f/\G_f \cap V} = 3.$ The only way this is possible is if $\md{\G_f} = 12$ or $\G_f \cong A_4.$
        %
        \item Assume that $r(x)$ is irreducible in $\mathbb{F}[x]$ and $\disc(r(x)) \notin \mathbb{F}^2.$ By \Cref{ex:galsepcubic}, it follows that $\G_r \cong S_3$ and hence, $\md{\G_f/\G_f \cap V} = 6.$ Thus, $\md{\G_f}$ is either $12$ or $24.$ If it is the latter, then $\G_f \cong S_4,$ as desired. We show that the former is not possible. \\
        Indeed, if $\md{\G_f} = 12,$ then $\G_f = A_4$ and thus, $\md{\G_f \cap V} = 4,$ which gives $\md{\G_f/\G_f \cap V} = 3,$ a contradiction.\footnote{The point to note here is that we explicitly know how $A_4$ and $V$ intersect, within $S_4.$}
        %
        \item Assume that $r(x)$ has all its roots in $\mathbb{F}.$ Then, 
        \begin{equation*} 
            \mathbb{E}^{\G_f \cap V} = \mathbb{F}(\underline{t}) = \mathbb{F} = \mathbb{E}^{\G_f}.
        \end{equation*}
        (The first equality follows from \Cref{prop:galoisintersectklein}.) 

        Thus, $\G_f \cap V = \G_f$ or $\G_f \subset V.$ Since $\md{V} = 4 \le \md{\G_f},$ it follows that $\G_f = V,$ as desired.
        %
        \item Assume that $r(x)$ has exactly one root in $\mathbb{F}.$ Then, $[\mathbb{F}(\underline{t}) : \mathbb{F}] = 2 = \md{\G_f/\G_f \cap V}.$ Thus, $\md{\G_f}$ is either $4$ or $8.$
        \begin{enumerate}
            \item Assume that $f(x)$ is irreducible over $\mathbb{F}(\underline{t}).$ Then,
            \begin{equation*} 
                \md{\G_f \cap V} = [\mathbb{E} : \mathbb{F}(\underline{t})] \ge 4.
            \end{equation*}
            Thus, $\md{\G_f \cap V} = 4$ and hence $\md{\G_f} = 4 \cdot 2 = 8$ which implies $G \cong D_8.$
            %
            \item Assume that $f(x)$ is reducible over $\mathbb{F}(\underline{t}).$ We already know that $\md{\G_f} = 4$ or $8.$ Thus, it is (isomorphic to) one of $C_4,$ $V,$ or $D_8.$ We show that the last two are not possible.

            Suppose $\G_f \cong D_8.$ Then,
            \begin{equation*} 
                [\mathbb{E} : \mathbb{F}(\underline{t})] = \frac{[\mathbb{E} : \mathbb{F}]}{[\mathbb{F}(\underline{t}) : \mathbb{F}]} = \frac{8}{2} = 4.
            \end{equation*}
            Thus, $\Gal(\mathbb{E}/\mathbb{F}(\underline{t})) \cong V$ (by \Cref{prop:galoisintersectklein}). But this is transitive, which contradicts the reduciblity of $f(x)$ over $\mathbb{F}(\underline{t}).$

            Now, suppose $\G_f = V.$ Then, $\G_r = \G_f/(\G_f \cap V) = \{1\}.$ But $\md{\G_r} = 2,$ a contradiction. \qedhere
        \end{enumerate}
    \end{enumerate}
\end{proof}

\section{Norm, Trace, and Hilbert's Theorem 90}

\propertiesnormtrace*\label{prop:propertiesnormtrace2}
\begin{flushright}\hyperref[prop:propertiesnormtrace]{\upsym}\end{flushright}
\begin{proof}
    \begin{enumerate}[leftmargin=*]
        \item It is clear that $\N_{\mathbb{E}/\mathbb{F}}(ab) = \N_{\mathbb{E}/\mathbb{F}}(a)\N_{\mathbb{E}/\mathbb{F}}(b)$ for all $a, b \in \mathbb{E}.$ Moreover, if $a \neq 0,$ then $\N_{\mathbb{E}/\mathbb{F}}(a) \neq 0.$

        Now, suppose $a \in \mathbb{E}^\times$ and let $\mathbb{L}$ be the normal closure of $\mathbb{E}/\mathbb{F}.$ Then, $\mathbb{L}/\mathbb{F}$ is a Galois extension and $\sigma_1(a), \ldots, \sigma_n(a) \in \mathbb{L}.$ Then, $ \N_{\mathbb{E}/\mathbb{F}}(a)$ is fixed by every $\sigma \in \Gal(\mathbb{L}/\mathbb{F})$ and hence, $a \in \mathbb{F}^\times.$
        %
        \item Suppose $\mathbb{E} = \mathbb{F}(a).$ Then, it is clear that the irreducible polynomial $f(x) = \irr(a, \mathbb{F})$ is simply
        \begin{equation*} 
            f(x) = (x - \sigma_1(a)) \cdots (x - \sigma_n(a))
        \end{equation*}
        and thus, $\N_{\mathbb{E}/\mathbb{F}}(a) = (-1)^n$ and $\Tr_{\mathbb{E}/\mathbb{F}}(a) = -a_{n - 1}.$
        %
        \item Let $a \in \mathbb{E}.$ Consider the extensions $\mathbb{F} \subset \mathbb{F}(a) \subset \mathbb{E}.$ Let $d \vcentcolon= [\mathbb{E} : \mathbb{F}(a)].$ \\
        Then, we see that
        \begin{equation*} 
            \Tr_{\mathbb{E}/\mathbb{F}}(a) = d \cdot \Tr_{\mathbb{F}(a)/\mathbb{F}}(a).
        \end{equation*}
        By the previous part, we see that $\Tr_{\mathbb{F}(a)/\mathbb{F}}(a) \in \mathbb{F}$ and thus, $\Tr_{\mathbb{E}/\mathbb{F}}(a) \in \mathbb{F}.$

        Now, let $\sigma_1, \ldots, \sigma_n$ be the $\mathbb{F}$-embeddings of $\mathbb{E}$ into $\overline{\mathbb{F}}.$ We have shown that
        \begin{equation*} 
            \Tr_{\mathbb{E}/\mathbb{F}} = \sigma_1 + \cdots + \sigma_n
        \end{equation*}
        is a $\mathbb{F}$-linear map from $\mathbb{E}$ to $\mathbb{F}.$ By \nameref{thm:dedekindcharacters}, it follows that $\sigma_1 + \cdots + \sigma_n$ is not the zero map and thus, $\Tr_{\mathbb{E}/\mathbb{F}}$ is surjective.
        %
        \item Let $\{\tau_j\}$ be the $\mathbb{F}$-embeddings $\mathbb{K} \to \overline{\mathbb{F}}$ and let $\{\sigma_i\}$ be the family of $\mathbb{K}$-embeddings $\mathbb{E} \to \overline{\mathbb{F}}.$ Each $\tau_j$ can be extended to an automorphism $\widetilde{\tau_j}$ of $\overline{\mathbb{F}}.$ Then, $\{\widetilde{\tau_j}\sigma_i\}$ is the set of all $\mathbb{F}$-embeddings $\mathbb{E} \to \overline{\mathbb{F}}.$ For an arbitrary $x \in \mathbb{E},$ we have
        \begin{equation*} 
            (\N_{\mathbb{K}/\mathbb{F}} \circ \N_{\mathbb{E}/\mathbb{K}})(x) = \N_{\mathbb{K}/\mathbb{F}}\left(\prod_{i = 1}^{n} \sigma_i(x)\right) = \prod_{j = 1}^{m}\prod_{i = 1}^{n} \tau_j\sigma_i(x) = \N_{\mathbb{E}/\mathbb{F}}(x). \qedhere
        \end{equation*}
    \end{enumerate} 
\end{proof}

\normtracelinearmap*\label{prop:normtracelinearmap2}
\begin{flushright}\hyperref[prop:normtracelinearmap]{\upsym}\end{flushright}
\begin{proof}
    Fix $a \in \mathbb{E}$ and let $\mathbb{K} = \mathbb{F}(a).$ Set
    \begin{equation*} 
        f(x) \vcentcolon= \irr(a, \mathbb{F}) = x^d + a_{d - 1}x^d + \cdots + a_1x + a_0.
    \end{equation*}
    Let $\{v_1, \ldots, v_e\}$ be a $\mathbb{K}$-basis for $\mathbb{E}.$ Then,
    \begin{equation*} 
        \{v_ia^j : i = 1, \ldots, e,\, j = 0, \ldots, d - 1\}
    \end{equation*}
    is an $\mathbb{F}$-basis of $\mathbb{E}.$ Order this basis as
    \begin{equation*} 
        B = (v_1, a_1v_1, \ldots, a_1^{d - 1}v_1, \ldots, v_e, av_e, \ldots, a^{d - 1}v_e).
    \end{equation*}
    Consider the matrix
    \begin{equation*} 
        A = \begin{bmatrix}
            0 & 0 & 0 & \cdots & 0 & -a_0 \\
            1 & 0 & 0 & \cdots & 0 & -a_1 \\
            0 & 1 & 0 & \cdots & 0 & -a_2 \\
            \vdots & \vdots & \vdots & \ddots & \vdots & \vdots \\
            0 & 0 & 0 & \cdots & 1 & -a_{d - 1}
        \end{bmatrix}.
    \end{equation*}
    Then, note that the matrix of $m_a$ with respect to $B$ is the $n \times n$ block matrix given as
    \begin{equation*} 
        \begin{bmatrix}
            A & O & O & \cdots & O \\
            O & A & O & \cdots & O \\
            O & O & A & \cdots & O \\
            \vdots & \vdots & \vdots & \ddots & \vdots \\
            O & O & O & \cdots & A \\
        \end{bmatrix}.
    \end{equation*}
    Moreover, the characteristic polynomial of $A$ is $f(x).$ In particular,
    \begin{equation*} 
        \det(A) = \det(A - 0I) = (-1)^df(0) = (-1)^d a_0 = \N_{\mathbb{K}/\mathbb{F}}(a).
    \end{equation*}
    The last equality above uses \Cref{prop:propertiesnormtrace}. Similarly, we also get
    \begin{equation*} 
        \Tr(A) = -a_{d - 1} = \Tr_{\mathbb{K}/\mathbb{F}}(a).
    \end{equation*}
    Therefore, using \Cref{prop:propertiesnormtrace} again, we get
    \begin{align*} 
        \N_{\mathbb{E}/\mathbb{F}}(a) &= (\N_{\mathbb{K}/\mathbb{F}} \circ \N_{\mathbb{E}/\mathbb{F}})(a) = \N_{\mathbb{K}/\mathbb{F}}(a^e) = (\det A)^e = \det m_a, \\
        \Tr_{\mathbb{E}/\mathbb{F}}(a) &= (\Tr_{\mathbb{K}/\mathbb{F}} \circ \Tr_{\mathbb{E}/\mathbb{F}})(a) = \Tr_{\mathbb{K}/\mathbb{F}}(ea) = e \Tr A = \Tr m_a. \qedhere
    \end{align*}
\end{proof}

\tracebilinearmaps*\label{prop:tracebilinearmaps2}
\begin{flushright}\hyperref[prop:tracebilinearmaps]{\upsym}\end{flushright}
\begin{proof}
    The first statement is easy to see and it clearly implies the next. In turn, this shows that $\psi$ is a map from $\mathbb{E}$ to $\Hom(\mathbb{E}, \mathbb{F}).$ It is again easy to see that $\psi$ is $\mathbb{F}$-linear. Since $\mathbb{E}$ and $\Hom_{\mathbb{F}}(\mathbb{E}, \mathbb{F})$ have the same dimension as $\mathbb{F}$-vector spaces, it suffices to show that $\psi$ is injective.

    Suppose that $x \in \mathbb{E}^\times$ is such that $\psi(x) = 0.$ Let $y \in \mathbb{E}$ be arbitrary. Then, we have
    \begin{equation*} 
        \Tr(y) = \Tr(xx^{-1}y) = \Tr_x(x^{-1}y) = \psi(x)(x^{-1}y) = 0.
    \end{equation*}
    But the above is a contradiction since $\Tr$ is not the zero map (\Cref{prop:propertiesnormtrace}).
\end{proof}

\hilbertmultiplicative*\label{thm:hilbertmultiplicative2}
\begin{flushright}\hyperref[thm:hilbertmultiplicative]{\upsym}\end{flushright}
\begin{proof}
    Let $[\mathbb{E} : \mathbb{F}] = n.$

    \backward If $\beta = \alpha/\sigma(\alpha),$ then
    \begin{equation*} 
        \N_{\mathbb{E}/\mathbb{F}}(\beta) = \beta \sigma(\beta) \cdots \sigma^{n - 1}(\beta) = \frac{\alpha}{\sigma(\alpha)} \frac{\sigma(\alpha)}{\sigma^2(\alpha)} \cdots \frac{\sigma^{n - 1}(\alpha)}{\sigma^n(\alpha)} = 1,
    \end{equation*}
    since $\sigma^n(\alpha) = \alpha.$

    \forward Suppose $\N_{\mathbb{E}/\mathbb{F}}(\beta) = 1.$ Then, the map
    \begin{equation*} 
        \id_{\mathbb{E}} + \beta \sigma + \beta \sigma(\beta) \sigma^2 + \beta \sigma(\beta) \sigma^2(\beta) \sigma^3 + \cdots + \beta \sigma(\beta) \cdots \sigma^{n - 2}(\beta) \sigma^{n - 1}
    \end{equation*}
    is a nonzero map $\mathbb{E} \to \mathbb{E},$ due to \nameref{thm:dedekindcharacters}. Let $\alpha \in \mathbb{E}^\times$ be in its image and $\theta$ be any preimage of $\alpha.$ That is,
    \begin{equation*} 
        \alpha = \theta + \beta \sigma(\theta) + \beta \sigma(\beta) \sigma^2(\theta) + \cdots + \beta \sigma(\beta) \cdots \sigma^{n - 2}(\beta) \sigma^{n - 1}(\theta)
    \end{equation*}
    and thus,
    \begin{equation*} 
        \beta\sigma(\alpha) = \beta \sigma(\theta) + \beta \sigma(\beta) \sigma^2(\theta) + \cdots + \underbrace{\beta \sigma(\beta) \sigma^2(\beta) \cdots \sigma^{n - 1}(\beta)}_{= \N_{\mathbb{E}/\mathbb{F}}(\beta) = 1} \underbrace{\sigma^{n}(\theta)}_{= \theta} = \alpha.
    \end{equation*}
    Therefore, $\beta = \frac{\sigma(\alpha)}{\alpha}.$
\end{proof}

\splittingfieldnthroots*\label{cor:splittingfieldnthroots2}
\begin{flushright}\hyperref[cor:splittingfieldnthroots]{\upsym}\end{flushright}
\begin{proof}
    Let $\Gal(\mathbb{E}/\mathbb{F}) = \langle \sigma\rangle,$ and $\zeta \in \mathbb{F}$ be a primitive $n$-th root of $1.$ Then, $\N_{\mathbb{E}/\mathbb{F}}(\zeta^{-1}) = \zeta^{-n} = 1.$ By \nameref{thm:hilbertmultiplicative}, there exists $\alpha \in \mathbb{E}^\times$ such that $\sigma(\alpha) = \zeta \alpha.$ In turn, $\sigma^i(\alpha) = \zeta^i\alpha$ for $i = 1, \ldots, n$ and thus, $\alpha$ has $n$ distinct conjugates in $\mathbb{E}.$ Since $[\mathbb{E} : \mathbb{F}] = n,$ it follows that $\mathbb{E} = \mathbb{F}(\alpha).$ \\
    Moreover, $\irr(\alpha, \mathbb{F}) = x^n - \alpha^n.$ Thus, it suffices to show that $\alpha^n \in \mathbb{F}.$ To this end, note that 
    \begin{equation*} 
        \sigma(\alpha^n) = (\sigma(\alpha))^n = \zeta^n\alpha^n = \alpha^n
    \end{equation*}
    and thus, $\alpha^n \in \mathbb{E}^{\langle \sigma\rangle} = \mathbb{F}.$
\end{proof}

\hilbertadditive*\label{thm:hilbertadditive2}
\begin{flushright}\hyperref[thm:hilbertadditive]{\upsym}\end{flushright}
\begin{proof}
    The proof is essentially the same as that of the \hyperref[thm:hilbertmultiplicative]{multiplicative form}. The direction \backward is simple and for \forward, let $\theta \in \mathbb{E}$ be such that $\Tr(\theta) \neq 0$ and consider the element
    \begin{equation*} 
        \alpha \vcentcolon= \frac{1}{\Tr(\theta)}[\beta\theta + (\beta + \sigma(\beta))\sigma(\theta) + \cdots + (\beta + \sigma(\beta) + \cdots + \sigma^{n - 2}(\beta))\sigma^{n - 2}(\theta)].
    \end{equation*}
    Use $\Tr(\beta) = 0$ to deduce $\alpha - \sigma(\alpha) = \beta.$
\end{proof}

\artinschreiercor*\label{cor:artinschreiercor2}
\begin{flushright}\hyperref[cor:artinschreiercor]{\upsym}\end{flushright}
\begin{proof}
    Let $\Gal(\mathbb{E}/\mathbb{F}) = \langle \sigma\rangle.$ Note that $\Tr(-1) = p \cdot (-1) = 0$ and thus, there exists $\alpha \in \mathbb{E}$ such that $-1 = \alpha - \sigma(\alpha)$ or $\sigma(\alpha) = \alpha + 1.$ Thus, $\sigma^i(\alpha) = \alpha + i$ for $i = 0, \ldots, p - 1.$ Since $\chr(\mathbb{F}) = p,$ all these elements are distinct and thus, $\alpha$ has $p$ distinct conjugates in $\mathbb{E}.$ Since $[\mathbb{E} : \mathbb{F}] = p,$ it follows that $\mathbb{E} = \mathbb{F}(\alpha).$ 

    Lastly, note that
    \begin{equation*} 
        \sigma(\alpha^p - \alpha) = (\alpha + 1)^p - (\alpha + 1) = \alpha^p - \alpha
    \end{equation*}
    and thus, $\alpha^p - \alpha =\vcentcolon a \in \mathbb{F}.$ It can be checked that all the conjugates of $\alpha$ are roots of $x^p - x - \alpha$ and we are done.
\end{proof} 
\end{document}