\documentclass[12pt,oneside]{book}
\usepackage{amsmath, amssymb, amsfonts, amsthm, mathtools}
\usepackage{thmtools}
\usepackage{witharrows}
\usepackage[utf8]{inputenc}
\usepackage[inline]{enumitem}
\usepackage[colorlinks=true]{hyperref}
\usepackage{parskip}
\usepackage{tikz-cd}
\usepackage{tikz}
\usetikzlibrary{decorations.markings}
\usetikzlibrary{arrows.meta}

\theoremstyle{definition}
\newtheorem{thm}{Theorem}
\numberwithin{thm}{chapter}
\newtheorem{lem}[thm]{Lemma}
\newtheorem{defn}[thm]{Definition}
\newtheorem{prop}[thm]{Proposition}
\newtheorem{cor}[thm]{Corollary}
\newtheorem{ex}[thm]{Example}
\newtheorem{rem}[thm]{Remark}


\setlength\parindent{0pt}

\let\emptyset\varnothing

\pagestyle{plain}

\usepackage{titlesec}
\titleformat{\section}[block]{\sffamily\Large\filcenter\bfseries}{\S\thesection.}{0.25cm}{\Large}
\titleformat{\subsection}[block]{\large\bfseries\sffamily}{\S\S\thesubsection.}{0.2cm}{\large}

\usepackage[a4paper]{geometry}
\usepackage{lipsum}

\usepackage{cleveref}
\crefname{thm}{theorem}{theorems}
\crefname{lem}{lemma}{lemmas}
\crefname{defn}{definition}{definitions}
\crefname{prop}{proposition}{propositions}
\crefname{cor}{corollary}{corollaries}
\crefname{ex}{example}{examples}
\crefname{rem}{remark}{remarks}
\crefname{equation}{}{}

\newcommand{\downsym}{[$\downarrow$]}
\newcommand{\upsym}{[$\uparrow$]}

\usepackage{mdframed}
\newenvironment{blockquote}
{\begin{mdframed}[skipabove=0pt, skipbelow=0pt, innertopmargin=4pt, innerbottommargin=4pt, bottomline=false,topline=false,rightline=false, linewidth=2pt]}
{\end{mdframed}}

\renewcommand{\familydefault}{\sfdefault}

\usepackage{fancyhdr}
\setlength{\headheight}{15.2pt}
\pagestyle{fancy}
\fancyhf{}
\fancyhead[L]{\sffamily{\S\textbf{\nouppercase{\rightmark}}}}
\fancyhead[R]{\sffamily{\thepage}}

\newcommand{\deff}[1]{{\color{blue}#1}}
\newcommand{\md}[1]{{\left\lvert #1 \right\lvert}}
\newcommand{\andd}{\quad\text{and}\quad}
\newcommand{\forward}{($\Rightarrow$)\ }
\newcommand{\backward}{($\Leftarrow$)\ }

\DeclareMathOperator{\chr}{char}
\DeclareMathOperator{\Frac}{Frac}
\DeclareMathOperator{\irr}{irr}
\DeclareMathOperator{\id}{id}
\DeclareMathOperator{\disc}{disc}
\DeclareMathOperator{\D}{D}

\title{MA-414\\Galois Theory}
\author{Aryaman Maithani\\\url{https://aryamanmaithani.github.io/}}
\date{Last updated: \today}

\begin{document}
\maketitle
\tableofcontents
\setcounter{chapter}{-1}
\chapter{Preliminaries} \label{chap:00}

\section{Notations and Conventions}
\begin{enumerate}
    \item $\mathbb{N}$ will denote the set of \textbf{positive} integers. That is, $\mathbb{N} = \{1, 2, \ldots\}.$
    \item $\mathbb{Z}$ will denote the set of integers.
    \item $\mathbb{N}_0$ will denote the set of all \textbf{non-negative} integers. \\
    That is, $\mathbb{N}_0 = \{0, 1, 2, \ldots\} = \mathbb{N} \cup \{0\}.$
    \item $\mathbb{Q}$ will denote the set of rationals.
    \item $\mathbb{R}$ will denote the set of real numbers.
    \item $\mathbb{C}$ will denote the set of complex numbers.
    \item Blackboard letters like $\mathbb{F}, \mathbb{E}, \mathbb{K}, \mathbb{L}$ will denote an arbitrary field.
    \item Given any field $\mathbb{F},$ $\mathbb{F}^\times$ denotes the group of units of $\mathbb{F}.$ That is, $\mathbb{F}^\times = \mathbb{F}\setminus\{0\}.$
    \item Given a ring $R,$ $R^\times$ denotes the group of units of $R.$ 
    \item Whenever we write ``$\mathbb{F} \subset \mathbb{E}$ are fields,'' we mean that $\mathbb{E}$ is a field and $\mathbb{F}$ is a subfield of $\mathbb{E}.$
    \item $\zeta_n \vcentcolon= \exp\left(\dfrac{2\pi\iota}{n}\right).$
    \item The degree of the zero polynomial is $-\infty.$
    \item Given a group $G$ and $g \in G,$ we denote the order of $g$ (in $G$) as $o(g).$
    % \item Given a group $G$ and subgroups $H_1, H_2 \unlhd G,$ we denote by $\langle H_1, H_2\rangle$ the smallest subgroup of $G$ containing $H_1$ and $H_2.$
    \item For $n \ge 1,$ we denote $\{1, \ldots, n\}$ as $[n].$
\end{enumerate}

\section{Field Theory}

We shall assume that the reader is familiar with the definitions and basic properties of groups and rings. All rings in this document will be assumed to be commutative with identity. 

We list some basic definition and properties. The proofs might be a bit terse and you should not have much problem filling in the details. (This won't be the case in the later chapters!)

\begin{defn}%[]
    An \deff{integral domain} is a ring with $0 \neq 1$ such $ab = 0 \implies a = 0$ or $b = 0.$
\end{defn}

\begin{defn}%[Field]
    A \deff{field} $(\mathbb{F}, +, \cdot)$ is a ring with $0 \neq 1$ such that every non-zero element has a multiplicative inverse.
\end{defn}

\begin{ex}
    $\mathbb{Q}, \mathbb{R}, \mathbb{C}$ are all fields.
\end{ex}

\begin{defn}%[]
    Given an integral domain $R,$ the field of fractions of $R$ is denoted by $\Frac(R).$
\end{defn}

\begin{defn}%[]
    A \deff{ring homomorphism} is a map $\varphi : R \to S$ between rings such that
    \begin{enumerate}
        \item $\varphi(ab) = \varphi(a)\varphi(b)$ for all $a, b \in R,$
        \item $\varphi(a + b) = \varphi(a) + \varphi(b)$ for all $a, b \in R,$
        \item $\varphi(1_R) = 1_S.$
    \end{enumerate}
    A \deff{field homomorphism} is a ring homomorphism between fields.
\end{defn}

\begin{defn}%[]
    Given a prime $p \in \mathbb{N},$ $\mathbb{Z}/p\mathbb{Z}$ is a field, which we denote as $\mathbb{F}_p.$
\end{defn}

\begin{defn}%[]
    Let $\mathbb{F}$ be a field. The \deff{characteristic} of $\mathbb{F}$ is defined to be the smallest positive integer $n$ such that
    \begin{equation*} 
        \underbrace{1_{\mathbb{F}} + \cdots + 1_{\mathbb{F}}}_{n} = 0_{\mathbb{F}}.
    \end{equation*}
    If no such $n$ exists, then the characteristic is defined to be $0.$ 

    This is denoted by $\chr \mathbb{F}.$
\end{defn}

From now on, we shall omit the subscript $\mathbb{F}$ when it is clear what the $0$ and $1$ are.

\begin{prop}
    If $\chr \mathbb{F} > 0,$ then $\chr \mathbb{F}$ is prime.
\end{prop}
\begin{proof} 
    Let $n \vcentcolon= \chr \mathbb{F}$ and let $n = ab$ for some $a, b \in \mathbb{F}.$ By distributivity and definition of $n,$ we have
    \begin{equation*} 
        \underbrace{(1 + \cdots + 1)}_{a}\underbrace{(1 + \cdots + 1)}_{b} = 0.
    \end{equation*}
    Since $\mathbb{F}$ is a field, one of the above two terms is $0.$ Without loss of generality, the first term is $0.$ By definition, $n = \chr \mathbb{F} \le a.$ But $a \mid n \implies a \le n.$

    Thus, $a = n.$
\end{proof}

\begin{prop}
    Every field contains an isomorphic copy of either $\mathbb{Q}$ or $\mathbb{F}_p$ for some prime $p.$ In fact, this copy is precisely $\Frac(\mathbb{Z}/\langle \chr\mathbb{F}\rangle).$
\end{prop}
\begin{proof} 
    Given a field $\mathbb{F},$ consider the ring homomorphism $\varphi : \mathbb{Z} \to \mathbb{F}$ given by $1 \mapsto 1.$ \\
    Then, $\mathbb{F}$ contains an isomorphic copy of $\mathbb{Z}/\ker \varphi.$ Note that $\varphi = \langle n\rangle,$ where $n = \chr\mathbb{F}.$ If $n > 0,$ then $n$ is prime and we are done.

    If $n = 0,$ then $\mathbb{F}$ contains an isomorphic copy of $\mathbb{Z}.$ Thus, it must contain $\mathbb{Q}.$\footnote{Either argue by explicitly constructing an isomorphism or use the universal property of fraction fields.}
\end{proof}

\begin{defn}%[]
    Given a field $\mathbb{F},$ the \deff{prime subfield} of $\mathbb{F}$ is defined as the smallest subfield of $\mathbb{F}.$ It is the intersection of all subfields of $\mathbb{F}.$ 
\end{defn}

\begin{prop}
    \phantom{hi}
    \begin{enumerate}
        \item The prime subfield of $\mathbb{F}$ is isomorphic to $\Frac(\mathbb{Z}/\langle \chr\mathbb{F}\rangle).$
        \item Let $\varphi : \mathbb{F} \to \mathbb{E}$ be a field homomorphism. Then, $\chr \mathbb{F} = \chr \mathbb{E}$ and $\varphi$ is injective. 
        \item Let $\mathbb{F} \subset \mathbb{E}$ be fields. $\mathbb{F}$ and $\mathbb{E}$ have the same prime subfield. Any field homomorphism $\varphi : \mathbb{F} \to \mathbb{E}$ fixes this prime subfield.
    \end{enumerate}
\end{prop}

\begin{defn}%[]
    Since any field homomorphism is injective, we also call them \deff{embeddings}.
\end{defn}

\begin{defn}
    Given fields $\mathbb{F} \subset \mathbb{E}_1, \mathbb{E}_2,$ an \deff{$\mathbb{F}$-homomorphism} from $\mathbb{E}_1$ to $\mathbb{E}_2$ is a field homomorphism $\varphi : \mathbb{E}_1 \to \mathbb{E}_2$ fixing $\mathbb{F}.$ If $\varphi$ is also an isomorphism, then it is called an \deff{$\mathbb{F}$-isomorphism}.
\end{defn}

\begin{defn}%[]
    Given rings $R \subset S,$ and $\alpha \in S,$ we define $R[\alpha]$ to be the smallest subring of $S$ containing $\alpha$ and $R.$ 

    Given fields $\mathbb{F} \subset \mathbb{K},$ and $\alpha \in \mathbb{K},$ we define $\mathbb{F}(\alpha)$ to be the smallest subfield of $\mathbb{K}$ containing $\alpha$ and $\mathbb{F}.$ 

    Similarly, given a set $A \subset R$ (or $A \subset \mathbb{F}$), we can talk about $R[A]$ (or $\mathbb{F}(A)$) to be the smallest subring (or subfield) \deff{generated by $A$ over $R$ (or $\mathbb{F}$)}.
\end{defn}

\begin{prop} \label{prop:FAdesc}
    Let $\mathbb{F} \subset \mathbb{E}$ be fields and $A \subset \mathbb{E}$ a set. 

    If $A = \emptyset,$ then $\mathbb{F}(A) = \mathbb{F}.$ Assume $A \neq \emptyset.$

    Let 
    \begin{equation*} 
        M \vcentcolon= \{a_1a_2 \cdots a_n \mid n \in \mathbb{N},\; a_1, \ldots, a_n \in A\}
    \end{equation*}
    be the set of all finite products (monomials) of elements of $A.$

    Let
    \begin{equation*} 
        S \vcentcolon= \{b_0 + b_1m_1 + \cdots + b_nm_n \mid n \in \mathbb{N}_{0},\; m_1, \ldots, m_n \in M,\;b_0, b_1, \ldots, b_n \in \mathbb{F}\}
    \end{equation*}
    be the set of all finite sums of elements of $M.$ (These are polynomials in $A$ with coefficients in $\mathbb{F}.$)

    Then,
    \begin{equation} \label{eq:FAdesc}
        \mathbb{F}(A) = \left\{\frac{s_1}{s_2} \mid s_1, s_2 \in S \text{ and } s_2 \neq 0\right\}.
    \end{equation}
\end{prop}

\begin{proof} 
    The case $A = \emptyset$ is trivial. Assume $A \neq \emptyset.$

    Let the set on the right in \Cref{eq:FAdesc} be called $Q.$ 

    Note that $M$ is closed under products and $S$ is closed under sums and products both. Moreover, $S$ contains $\mathbb{F}$ as the constant polynomials. Using this, it is clear that $Q$ is a subfield of $\mathbb{E}.$ By taking denominator $1,$ we also see that $S \subset Q.$ Since $\mathbb{F} \subset S$ and $A \subset M \subset S,$ we see that $Q$ is a subfield of $\mathbb{E}$ containing $A$ and $\mathbb{F}.$ Thus, $\mathbb{F}(A) \subset Q.$

    On the other hand, note that $M \subset \mathbb{F}(A)$ since $A \subset \mathbb{F}(A).$ Since $\mathbb{F} \subset \mathbb{F}(A)$ as well, we get $S \subset \mathbb{F}(A).$ Thus, $Q \subset \mathbb{F}(A).$ (In all the assertions, we have used that $\mathbb{F}(A)$ is a subfield of $\mathbb{E}$ and thus, has the required closure properties.)
\end{proof}

\begin{cor} \label{cor:FAdescfinite}
    Let $\mathbb{F} \subset \mathbb{E}$ be fields and $A \subset \mathbb{E}$ a set. If $a \in \mathbb{F}(A),$ then there exists a finite set $B \subset A$ such that $a \in \mathbb{F}(B).$
\end{cor}

\begin{proof} 
    Let $a \in F(A).$ Let $M, S$ be as in \Cref{prop:FAdesc}. Then, $a = s_1/s_2$ for some $s_1, s_2 \in S.$ Then, each $s_i$ is a polynomial in some finitely many $a_i \in A$ with coefficients in $\mathbb{F}.$ Let $B$ be the set of those finitely many $a_i.$ Then, $a \in \mathbb{F}(B).$
\end{proof}

\begin{prop}
    If $\mathbb{F}$ is a finite field, then $\chr(\mathbb{F}) =\vcentcolon p > 0$ and $\md{\mathbb{F}} = p^n$ for some $n \in \mathbb{N}.$
\end{prop}
\begin{proof} 
    $\chr(\mathbb{F}) = 0$ is not possible since $\mathbb{Z}$ is infinite and so, the homomorphism $\varphi : \mathbb{Z} \to \mathbb{F}$ given by $1 \mapsto 1$ cannot be injective.

    Now, $\mathbb{F}$ contains $\mathbb{F}_p$ as a subfield and hence, is a vector space over $\mathbb{F}.$ Since $\md{\mathbb{F}} < \infty,$ we have $\dim_{\mathbb{F}_p}(\mathbb{F}) =\vcentcolon n < \infty.$

    It is clear now that $\md{\mathbb{F}} = \md{\mathbb{F}_p}^n = p^n.$
\end{proof}

\begin{thm} \label{thm:numberofrootsinfield}
    Let $f(x) \in \mathbb{F}[x]$ have a degree $n \ge 1.$ Then, $f(x)$ has at most $n$ roots in $\mathbb{F}.$
\end{thm}
\begin{proof} 
    Induct on $n$ and use the fact that $ab = 0 \implies a = 0$ or $b = 0,$ in a field.
\end{proof}

\begin{thm} \label{thm:finsubgroupcyclic}
    Let $\mathbb{F}$ be a field. Let $U$ be a finite subgroup of $\mathbb{F}^\times.$ Then, $U$ is cyclic. 
\end{thm}
We give three proofs. The third is the slickest one, which I got from \href{https://mathoverflow.net/questions/54735}{this Mathoverflow post}.
\begin{proof} 
    This proof uses the following fact: Let $G$ be an abelian group and $a, b \in G$ have orders $m$ and $n.$ Then, there exist $c \in G$ with order $\operatorname{lcm}(m, n).$ (This needs a little argument. $c = ab$ works if $\gcd(m, n) = 1.$ The general case has to be reduced to that.)

    Let $n \vcentcolon= \md{U}.$ Let $a \in U$ be an element with maximal order, say $d.$ Then, we have
    \begin{equation*} 
        d = \operatorname{lcm} \{\operatorname{order}(u) \mid u \in U\}.
    \end{equation*}
    Thus, all $n$ elements of $U \subset \mathbb{F}$ satisfy the polynomial $x^d - 1 \in \mathbb{F}[x].$ Since $\mathbb{F}$ is a field, we have $n \le d.$ Thus, $d = n$ and $U = \langle a\rangle.$
\end{proof}

\begin{proof} 
    This prove uses the structure theorem of abelian groups. Let $n \vcentcolon= \md{U}.$

    Write $U \cong \mathbb{Z}/d_1\mathbb{Z} \times \cdots \times \mathbb{Z}/d_r\mathbb{Z}$ where $1 < d_1 \mid d_2 \mid \cdots \mid d_r$ and $n = d_1 \cdots d_r.$ Now, every element of $U$ satisfies $x^{d_r} - 1.$ Thus, as earlier, we have $d_r = n$ and hence, $n = 1.$ This means $U \cong \mathbb{Z}/n\mathbb{Z}$ is cyclic. 
\end{proof}
\begin{proof} 
    This proof uses just the following simple fact: If $x, y$ are elements of (finite) coprime order in an abelian group, then the order of $xy$ is the product of the orders.

    We now prove the result by induction on $\md{U}$. Clearly, it is true for $\md{U} = 1$. Assume $\md{U} \ge 2$.

    \textbf{Case 1.} $\md{U} = p^{k}$ with $p$ prime and $k \ge 1$. \\
    In this case, if $U$ is not cyclic, then all $p^{k}$ elements of $U$ satisfy $x^{p^{k - 1}} - 1 = 0$, a contradiction to \Cref{thm:numberofrootsinfield}.

    \textbf{Case 2.} $\md{U} = ab$ for some coprime integers $a, b > 1$. \\
    Consider the homomorphism $U \to U$ given by $u \mapsto u^{a}$. Let $A$ be the kernel and $B$ be its image. Note that every $u \in A$ satisfies $u^{a} = 1$ and every $u \in B$ satisfies $u^{b} = 1$. Thus, by \Cref{thm:numberofrootsinfield}, we have $\md{A} \le a < \md{U}$ and $\md{B} \le b < \md{U}$. \\
    Since $A$ and $B$ are subgroups of $U$, the induction hypothesis applies. Let $x$ and $y$ be cyclic generators for $A$ and $B$. Then, the order of $xy$ is $\md{U}$ and we are done.
\end{proof}

\begin{prop} \label{prop:divisibilityofpoly}
    Let $\mathbb{F} \subset \mathbb{K}$ be fields and $f(x), g(x) \in \mathbb{F}[x].$ \\
    Then, $f(x) \mid g(x)$ in $\mathbb{F}[x]$ iff $f(x) \mid g(x)$ in $\mathbb{K}[x].$

    In particular, if $f(x)$ factorises linearly into distinct factors in $\mathbb{K}[x],$ then it suffices to show that every root of $f(x)$ is also one of $g(x).$
\end{prop}

\begin{proof}
    \forward This is obvious because a factorisation $g(x) = f(x)h(x)$ in $\mathbb{F}[x]$ also holds in $\mathbb{K}[x].$

    \backward If $f(x) = 0,$ then the result is true. Assume $f(x) \neq 0.$ \\
    By the division algorithm, we may write
    \begin{equation*} 
        g(x) = f(x)q(x) + r(x)
    \end{equation*}
    for unique $q(x), r(x) \in \mathbb{F}[x]$ with $\deg(r(x)) < \deg(q(x)).$

    The above is also a division in $\mathbb{K}[x].$ But $f(x) \mid g(x)$ in $\mathbb{K}[x]$ and so, uniqueness forces $r(x) = 0.$
\end{proof} 
\chapter{Algebraic extensions}

\begin{defn}%[]
    Let $\mathbb{F}$ be a subfield of $\mathbb{K}.$ We say that $\mathbb{K}$ is an \deff{extension field} of $\mathbb{F}$ and $\mathbb{F}$ is called the base field. We also denote this by $\mathbb{K}/\mathbb{F}.$
\end{defn}

\begin{rem}
    The above is not to be confused with any sort of quotient. In fact, since the only ideals of a field $\mathbb{K}$ are $0$ and $\mathbb{K},$ there is no discussion about quotienting.
\end{rem}

\begin{defn}%[]
    Let $\mathbb{K}/\mathbb{F}$ be a field extension. Then, we may regard $\mathbb{K}$ as a vector space over $\mathbb{F}.$ We denote $\dim_{\mathbb{F}}\mathbb{K}$ by $[\mathbb{K} : \mathbb{F}]$ and call it the \deff{degree} of the field extension $\mathbb{K}/\mathbb{F}.$
\end{defn}

\begin{defn}%[]
    The field extension $\mathbb{K}/\mathbb{F}$ is said to be a \deff{finite extension} if $[\mathbb{K} : \mathbb{F}]$ is finite. 
\end{defn}

\begin{defn}%[]
    The field extension $\mathbb{K}/\mathbb{F}$ is said to be a \deff{simple extension} if there exists $\alpha \in \mathbb{K}$ such that $\mathbb{K} = \mathbb{F}(\alpha).$
\end{defn}

\begin{defn}
    Let $\mathbb{K}/\mathbb{F}$ be a field extension and let $\alpha \in \mathbb{K}.$ $\alpha$ is said to be \deff{algebraic over $\mathbb{F}$} if there exists a non-zero polynomial $f(x) \in \mathbb{F}[x]$ such that $f(\alpha) = 0.$

    $\alpha$ is said to be \deff{transcendental over $\mathbb{F}$} if it is not algebraic over $\mathbb{F}.$

    If every element of $\mathbb{K}$ is algebraic over $\mathbb{F},$ then $\mathbb{K}/\mathbb{F}$ is called an \deff{algebraic extension}.
\end{defn}

\begin{ex}
    Note that every element of $\mathbb{F}$ is algebraic over $\mathbb{F}.$
\end{ex}

Here's a simple proposition that we leave as an easy exercise.

\begin{prop} \label{prop:decompalgisalg}
    Let $\mathbb{F} \subset \mathbb{E} \subset \mathbb{K}$ be fields and $\alpha \in \mathbb{K}.$ \\
    If $\alpha$ is algebraic over $\mathbb{F},$ then $\alpha$ is algebraic over $\mathbb{E}.$ \\
    If $\mathbb{K}/\mathbb{F}$ is algebraic, then so are $\mathbb{K}/\mathbb{E}$ and $\mathbb{E}/\mathbb{F}.$
\end{prop}

\begin{restatable}[]{prop}{finextisalg}
\label{prop:finextisalg}
    Every finite extension is an algebraic extension. \hfill\hyperref[prop:finextisalg2]{\downsym}
\end{restatable}


\begin{ex}
    Consider the extensions $\mathbb{Q} \subset \mathbb{R} \subset \mathbb{C}$ and $\pi\iota \in \mathbb{C}.$

    It is known that $\pi \in \mathbb{R}$ is transcendental over $\mathbb{Q}.$ An easy consequence of this is that $\pi\iota \in \mathbb{C}$ is also transcendental over $\mathbb{Q}.$ However, $\pi\iota$ is algebraic over $\mathbb{R}$ since it satisfies $x^2 + \pi^2 \in \mathbb{R}[x] \setminus \{0\}.$

    Thus, the property of being algebraic/transcendental depends on the base field. In particular, $\mathbb{C}/\mathbb{Q}$ is not an algebraic extension. However, in view of the earlier proposition, $\mathbb{C}/\mathbb{R}$ is.
\end{ex}

\begin{ex}
    Let $\mathbb{K}$ be a finite field and $\mathbb{F}$ be its prime subfield. Then, $\mathbb{K}$ is a finite dimensional $\mathbb{F}$-vector space and thus, $\mathbb{K}/\mathbb{F}$ is an algebraic extension.
\end{ex}

\begin{rem}
    The converse of the proposition is not true. We shall see later that
    \begin{equation*} 
        \mathbb{A} \vcentcolon= \{\alpha \in \mathbb{C} : \alpha \text{ is algebraic over }\mathbb{Q}\}
    \end{equation*}
    is a subfield of $\mathbb{C}$ such that $\dim_{\mathbb{Q}}(\mathbb{A}) = \infty.$ However, $\mathbb{A}/\mathbb{Q}$ is clearly algebraic, by construction.
\end{rem}

\begin{restatable}[]{prop}{uniquemonicirred}
\label{prop:uniquemonicirred}
    Let $\mathbb{K}/\mathbb{F}$ be a field extension and $\alpha \in \mathbb{K}$ be algebraic over $\mathbb{F}.$ Then, the following are true. 
    \begin{enumerate}
        \item There exists a unique monic irreducible polynomial $f(x) \in \mathbb{F}[x]$ such that $f(\alpha) = 0.$ 
        \item $f(x)$ generates the kernel of the map $\mathbb{F}[x] \to \mathbb{F}[\alpha] \subset \mathbb{K}$ given by $p(x) \mapsto p(\alpha).$
        \item If $g(x) \in \mathbb{F}[x]$ is such that $g(\alpha) = 0,$ then $f(x) \mid g(x).$ 
        \item In particular, $f(x)$ has the least positive degree among all polynomials in $\mathbb{F}[x]$ satisfied by $\alpha.$ \hfill\hyperref[prop:uniquemonicirred2]{\downsym}
    \end{enumerate}

\end{restatable}

Of course, ``irreducible'' above means ``irreducible in $\mathbb{F}[x].$''

\begin{defn}%[]
    Given a field extension $\mathbb{K}/\mathbb{F}$ and $\alpha \in \mathbb{K}$ with is algebraic over $\mathbb{F},$ the irreducible monic polynomial $f(x) \in \mathbb{F}[x]$ having $\alpha$ as a root is called the \deff{irreducible monic polynomial of $\alpha$ over $\mathbb{F}.$} It is denoted by $\irr(\alpha, \mathbb{F}).$

    The degree of $\irr(\alpha, \mathbb{F})$ is called the \deff{degree of $\alpha$} and is denoted by $\deg_{\mathbb{F}}\alpha.$
\end{defn}

\begin{ex}
    \phantom{hi}
    \begin{enumerate}
        \item Let $\alpha \in \mathbb{C}$ be a square root of $\iota.$ Then, $\alpha$ satisfies $f(x) \vcentcolon= x^4 + 1.$ Show that $f(x) = \irr(\alpha, \mathbb{Q}).$

        However, $\irr(\alpha, \mathbb{Q}(i)) = x^2 - \iota.$ Thus, degree also depends on the base field.
        \item Let $p$ be a prime and $\zeta_p \vcentcolon= \exp\left(\dfrac{2\pi\iota}{p}\right) \in \mathbb{C}.$ Then, $\zeta_p^p = 1.$ Note that $x^p - 1 = (x - 1)\Phi_p(x)$ where
        \begin{equation*} 
            \Phi_p(x) \vcentcolon= x^{p - 1} + \cdots + 1.
        \end{equation*}
        Then, $\Phi_p(\zeta_p) = 0.$ Use Eisenstein's criterion on $\Phi_p(x + 1)$ to conclude that $\Phi_p(x)$ is irreducible in $\mathbb{Q}[x]$ and hence, $\Phi_p(x) = \irr(\zeta_p, \mathbb{Q}).$
    \end{enumerate}
\end{ex}

\begin{restatable}[]{prop}{adjoiningalg}
\label{prop:adjoiningalg}
    Let $\mathbb{K}/\mathbb{F}$ be a field extension and $\alpha \in \mathbb{K}$ be algebraic over $\mathbb{F}.$ Let $f(x) \vcentcolon= \irr(\alpha, \mathbb{F})$ and $n \vcentcolon= \deg f(x).$ Then,
    \begin{enumerate}
         \item $\mathbb{F}[\alpha] = \mathbb{F}(\alpha) \cong \mathbb{F}[x]/\langle f(x)\rangle.$
         \item $\dim_{\mathbb{F}}(\mathbb{F}(\alpha)) = n$ and $\{1, \alpha, \ldots, \alpha^{n - 1}\}$ is an $\mathbb{F}$-basis of $\mathbb{F}(\alpha).$ \hfill\hyperref[prop:adjoiningalg2]{\downsym}
     \end{enumerate} 
\end{restatable}

\begin{cor} \label{cor:adjoinalgisfin}
    Let $\mathbb{K}/\mathbb{F}$ be a field extension and $\alpha \in \mathbb{K}$ be algebraic over $\mathbb{F}.$ Then, $\mathbb{F}(\alpha)/\mathbb{F}$ is a finite and hence, algebraic extension, by \Cref{prop:finextisalg}.
\end{cor}

\begin{restatable}[]{prop}{isocarryingalphtobet}
\label{prop:isocarryingalphtobet}
    Let $\alpha, \beta \in \mathbb{K} \supset \mathbb{F}$ be algebraic over $\mathbb{F}.$ Then, there exists and $\mathbb{F}$-isomorphism $\psi : \mathbb{F}(\alpha) \to \mathbb{F}(\beta)$ such that $\psi(\alpha) = \beta$ iff $\irr(\alpha, \mathbb{F}) = \irr(\beta, \mathbb{F}).$ \hfill\hyperref[prop:isocarryingalphtobet2]{\downsym}
\end{restatable}

\begin{defn}%[]
    The extension $\mathbb{K}/\mathbb{F}$ is said to be \deff{a quadratic extension} if $[\mathbb{K} : \mathbb{F}] = 2.$
\end{defn}

\begin{rem}
    Note that if $\mathbb{K}/\mathbb{F}$ is a quadratic extension and $\alpha \in \mathbb{K}\setminus\mathbb{F},$ then $[\mathbb{F}(\alpha) : \mathbb{F}] > 1$ and hence, $\mathbb{F}(\alpha) = 2.$ Thus, $\mathbb{F}(\alpha) = \mathbb{K}.$

    That is, all quadratic extensions are simple.
\end{rem}

\begin{restatable}[Tower law]{thm}{towerlaw}
\label{thm:towerlaw}
    Let $\mathbb{F} \subset \mathbb{E} \subset \mathbb{K}$ be a tower of fields. Then,
    \begin{equation*} 
        [\mathbb{K} : \mathbb{F}] = [\mathbb{K} : \mathbb{E}][\mathbb{E} : \mathbb{F}].
    \end{equation*}
    In particular, the left side is $\infty$ iff the right side is. \hfill\hyperref[thm:towerlaw2]{\downsym}
\end{restatable}

\begin{cor}
    Let $\mathbb{K}/\mathbb{F}$ be a finite extension and $\alpha \in \mathbb{K}.$ Then, $\deg_{\mathbb{F}} \alpha \mid [\mathbb{K} : \mathbb{F}].$
\end{cor}
\begin{proof}
    Consider the tower $\mathbb{F} \subset \mathbb{F}(\alpha) \subset \mathbb{K}.$
\end{proof}

\begin{restatable}[]{prop}{adjoinalgsfinext}
\label{prop:adjoinalgsfinext}
    Let $\mathbb{K}/\mathbb{F}$ be a field extension and let $\alpha_1, \ldots, \alpha_n \in \mathbb{K}$ be algebraic over $\mathbb{F}.$ Then, $\mathbb{F}(\alpha_1, \ldots, \alpha_n)$ is a finite (and hence, algebraic) extension of $\mathbb{F}.$ \hfill\hyperref[prop:adjoinalgsfinext2]{\downsym}
\end{restatable}

\begin{restatable}[]{cor}{compalgisalg}
\label{cor:compalgisalg}
   Let $\mathbb{F} \subset \mathbb{E}$ and $\mathbb{E} \subset \mathbb{K}$ be algebraic extensions. Then, $\mathbb{F} \subset \mathbb{K}$ is an algebraic extension.  \hfill\hyperref[cor:compalgisalg2]{\downsym}
\end{restatable}

\begin{restatable}[]{cor}{algclosureisfield}
\label{cor:algclosureisfield}
    Let $\mathbb{K}/\mathbb{F}$ be a field extension. Then,
    \begin{equation*} 
        \mathbb{A} \vcentcolon= \{\alpha \in \mathbb{K} : \alpha \text{ is algebraic over }\mathbb{F}\}
    \end{equation*} is a subfield of $\mathbb{K}$ containing $\mathbb{F}.$ \\
    Moreover, $\mathbb{A}/\mathbb{F}$ is an algebraic extension. \hfill\hyperref[cor:algclosureisfield2]{\downsym}
\end{restatable}

\section{Compositum of fields}

\begin{defn}%[]
    Let $\mathbb{E}_1, \mathbb{E}_2 \subset \mathbb{K}$ be fields. The \deff{compositum} of $\mathbb{E}_1$ and $\mathbb{E}_2$ is the smallest subfield of $\mathbb{K}$ containing $\mathbb{E}_1$ and $\mathbb{E}_2.$ It is denoted by $\mathbb{E}_1\mathbb{E}_2.$
\end{defn}

\begin{ex}
    Suppose $\mathbb{F} \subset \mathbb{E}_1, \mathbb{E}_2 \subset \mathbb{K}$ and $\mathbb{E}_1 = \mathbb{F}(\alpha_1, \ldots, \alpha_n).$ Then,
    \begin{equation*} 
        \mathbb{E}_1\mathbb{E}_2 = \mathbb{E}_2(\alpha_1, \ldots, \alpha_n).
    \end{equation*}
\end{ex}

\begin{ex} \label{ex:compositecyclo}
    Let $m$ and $n$ be coprime positive integers. Consider the subfields $\mathbb{F} \vcentcolon= \mathbb{Q}(\zeta_m)$ and $\mathbb{E} \vcentcolon= \mathbb{Q}(\zeta_n)$ of $\mathbb{C}.$ Then,
    \begin{equation*} 
        \mathbb{E}\mathbb{F} = \mathbb{Q}(\zeta_{mn}).
    \end{equation*}
    $\subset$ is clear since $\zeta_n = \zeta_{mn}^m$ and similarly, $\zeta_m = \zeta_{mn}^n.$

    On the other hand, since $\gcd(m, n) = 1,$ there exist integers $a, b \in \mathbb{Z}$ such that $am + bn = 1.$ Thus,
    \begin{equation*} 
        \frac{a}{n} + \frac{b}{m} = \frac{1}{mn}
    \end{equation*}
    and hence
    \begin{equation*} 
        \zeta_{mn} = \zeta_n^a\zeta_m^b.
    \end{equation*}
\end{ex}

\begin{restatable}[]{prop}{intdomfinextfield}
\label{prop:intdomfinextfield}
    Let $\mathbb{F}$ be a field which is a subring of an integral domain $R.$ Suppose $R$ is finite dimensional as an $\mathbb{F}$ vector space. Then, $R$ is a field. \hfill\hyperref[prop:intdomfinextfield2]{\downsym}
\end{restatable}

\begin{restatable}[]{prop}{descofcompositum}
\label{prop:descofcompositum}
    Let $\mathbb{F} \subset \mathbb{E}_1, \mathbb{E}_2 \subset \mathbb{K}$ be fields. Consider
    \begin{equation*} 
        \mathbb{L} = \left\{\sum_{i = 1}^{n} \alpha_i\beta_i : n \in \mathbb{N}, \alpha_i \in \mathbb{E}_1, \beta_i \in \mathbb{E}_2\right\}.
    \end{equation*}
    That is, let $\mathbb{L}$ be the set of all finite sums of products of elements of $\mathbb{E}_1$ and $\mathbb{E}_2.$

    Suppose $d \vcentcolon= [\mathbb{E}_1 : \mathbb{K}][\mathbb{E}_2 : \mathbb{K}] < \infty.$ \\
    Then $\mathbb{L} = \mathbb{E}_1\mathbb{E}_2$ and $[\mathbb{L} : \mathbb{F}] \le d.$ 

     If $[\mathbb{E}_1 : \mathbb{F}]$ and $[\mathbb{E}_2 : \mathbb{F}]$ are coprime, then equality holds. \hfill\hyperref[prop:descofcompositum2]{\downsym}
\end{restatable}

Diagrammatically, this can be depicted as
\begin{center}
    \begin{tikzcd}
                                       & \mathbb{K} \arrow[d, no head]                                                       &              \\
                                       & \mathbb{E}_1\mathbb{E}_2 \arrow[ld, "\le m"', no head] \arrow[rd, "\le n", no head] &              \\
\mathbb{E}_1 \arrow[rd, "n"', no head] &                                                                                     & \mathbb{E}_2 \\
                                       & \mathbb{F} \arrow[ru, "m"', no head]                                                &             
    \end{tikzcd}
\end{center}

\section{Splitting Fields}

\begin{defn}%[]
    Let $\mathbb{F}$ be a field and $f(x) \in \mathbb{F}[x]$ be a non-constant monic polynomial of degree $n$ with leading coefficient $a \in \mathbb{F}^\times.$ A field $\mathbb{K} \supset \mathbb{F}$ is called a \deff{splitting field of $f(x)$ over $\mathbb{F}$} if there exist $r_1, \ldots, r_n \in \mathbb{K}$ so that $f(x) = a(x - r_1)\cdots(x - r_n)$ and $\mathbb{K} = \mathbb{F}(r_1, \ldots, r_n).$
\end{defn}

Note that $r_1, \ldots, r_n$ above need not be distinct.

\begin{ex}
    Consider $\mathbb{F} = \mathbb{Q},$ $f(x) = x^2 + 1 \in \mathbb{Q}[x]$ and $\mathbb{K} = \mathbb{C}.$ While $f(x)$ does factor linearly over $\mathbb{C},$ $\mathbb{C}$ is \textbf{not} a splitting field of $f(x)$ over $\mathbb{Q}$ since $\mathbb{C} \neq \mathbb{Q}(\iota, -\iota).$

    On the other hand, $\mathbb{C}$ \emph{is} a splitting field of $f(x) \in \mathbb{R}[x]$ over $\mathbb{R}.$
\end{ex}

\begin{cor}
    Let $f(x) \in \mathbb{F}[x]$ be non-constant and $\mathbb{K}$ be a splitting field of $f(x)$ over $\mathbb{F}.$ Then, $\mathbb{K}/\mathbb{F}$ is an algebraic extension.
\end{cor}
\begin{proof} 
    Follows from \Cref{prop:adjoinalgsfinext}.
\end{proof}

\begin{restatable}[]{thm}{rootcanbeadjoined}
\label{thm:rootcanbeadjoined}
    Let $\mathbb{F}$ be a field and $f(x) \in \mathbb{F}[x]$ be non-constant. Then, there exists a field $\mathbb{K} \supset \mathbb{F}$ such that $f(x)$ has a root in $\mathbb{K}.$ \hfill\hyperref[thm:rootcanbeadjoined2]{\downsym}
\end{restatable}

\begin{restatable}[Existence of Splitting Field]{thm}{splitfieldexists}
\label{thm:splitfieldexists}
    Let $\mathbb{F}$ be a field. Any polynomial $f(x) \in \mathbb{F}[x]$ of positive degree has a splitting field. \hfill\hyperref[thm:splitfieldexists2]{\downsym}
\end{restatable}
\chapter{Symmetric Polynomials}
\section{Basic Definitions}

\begin{defn}%[]
    Given a ring $R,$ consider the polynomial ring $S = R[u_1, \ldots, u_n].$ Let $S_n$ denote the symmetric group. Then, any $\tau \in S_n$ induces an automorphism $g_{\tau} : S \to S$ by
    \begin{equation*} 
        g_{\tau}(f(u_1, \ldots, u_n)) = f(u_{\tau(1)}, \ldots, u_{\tau(n)}).
    \end{equation*}
\end{defn}

\begin{ex}
    Consider $R = \mathbb{Z}$ and $n = 3.$ Suppose $\tau = (12).$ Consider the polynomial $f = u_1 + u_2^2 + u_3^3.$ Then, $g_{\tau}(f) = u_2 + u_1^2 + u_3^3.$
\end{ex}

\begin{defn}%[]
    A polynomial $f \in R[u_1, \ldots, u_n]$ is said to be a \deff{symmetric polynomial (in $n$ variables)} if 
    \begin{equation*} 
        f(u_1, \ldots, u_n) = f(u_{\tau(1)}, \ldots, u_{\tau(n)})
    \end{equation*} 
    for all $\tau \in S_n.$ In other words, $g_{\tau}(f) = f$ for all $\tau \in S_n.$
\end{defn}

\begin{defn}%[]
    Let $S = R[u_1, \ldots, u_n].$ Consider $f(T) \in S[T]$ given by
    \begin{equation*} 
        f(T) = (T - u_1) \cdots (T - u_n).
    \end{equation*}
    Write $f(T)$ as 
    \begin{equation*} 
        f(T) = T^n - \sigma_1 T^{n - 1} + \cdots + (-1)^n \sigma_n,
    \end{equation*}
    for $\sigma_1, \ldots, \sigma_n \in S.$

    Then, $\sigma_1, \ldots, \sigma_n$ are symmetric polynomials, which are called the \deff{elementary symmetric polynomials (in $n$ variables)}.
\end{defn}

\begin{rem}
    Note that one can explicitly write down the elementary symmetric polynomials. We have
    \begin{align*} 
        \sigma_1 &= \sum_{i_1 = 1}^{n} u_{i_1},\\
        \sigma_2 &= \sum_{1 \le i_1 < i_2 \le n} u_{i_1}u_{i_2},\\
        & \vdots \\
        \sigma_n &= u_1 \cdots u_n.
    \end{align*}
    It is now easy to verify that these are all indeed symmetric polynomials.
\end{rem}

\section{Fundamental theorem of Symmetric Polynomials}

\begin{defn}%[]
    Given an elementary symmetric polynomial $\sigma_i \in R[u_1, \ldots, u_n]$ in $n$ variables (for $n \ge 2$), we define the elementary symmetric polynomial $\sigma_i^0$ in $(n - 1)$ variables as
    \begin{equation*} 
        \sigma_i^0 \vcentcolon= \sigma_1(u_1, \ldots, u_{n - 1}, 0).
    \end{equation*}
\end{defn}

\begin{ex}
    Consider $n = 3.$ Then, $\sigma_2 = u_1u_2 + u_1u_3 + u_2u_3.$ Then, $\sigma_2^0 = u_1u_2.$ This is the second symmetric polynomial in two variables. 

    In fact, any elementary symmetric polynomial in $n - 1$ variables is of the form $\sigma_i^0$ for the corresponding elementary symmetric polynomial $\sigma_i$ in $n$ variables.
\end{ex}

\begin{restatable}[Fundamental Theorem of Symmetric Polynomials]{thm}{FTSP}
\label{thm:FTSP}
    Let $R$ be a commutative ring. Then, every symmetric polynomial in $S \vcentcolon= R[u_1, \ldots, u_n]$ is a polynomial in the elementary symmetric polynomials in a unique way.

    More precisely, if $f(u_1, \ldots, u_n)$ is symmetric, then there exists a unique $g \in R[x_1, \ldots, x_n]$ such that
    \begin{equation*} 
        g(\sigma_1, \ldots, \sigma_n) = f(u_1, \ldots, u_n).
    \end{equation*}
    (The above is equality in $S.$) \hfill\hyperref[thm:FTSP2]{\downsym}
\end{restatable}

\section{Newton's identities for power sum symmetric polynomials}

\begin{defn}%[]
    Let $S = R[u_1, \ldots, u_n].$ For $k \ge 1,$ define
    \begin{equation*} 
        w_k = u_1^k + \cdots + u_n^k.
    \end{equation*}
\end{defn}

\begin{restatable}[Newton's Identities]{thm}{powersumformulae}
\label{thm:powersumformulae}
    We have
    \begin{equation} \label{eq:newident}
        w_k = \begin{cases}
            \sigma_1 w_{k - 1} - \sigma_2w_{k - 2} + \cdots + (-1)^k \sigma_{k - 1}w_1 + (-1)^{k + 1}\sigma_k k & k \le n,\\
            \sigma_1 w_{k - 1} - \sigma_2w_{k - 2} + \cdots + (-1)^{n + 1} \sigma_{n}w_{k - n} & k > n.\\
                
        \end{cases}
    \end{equation} 
    \hfill\hyperref[thm:powersumformulae2]{\downsym}
\end{restatable}

Note that the last term is $(-1)^{k + 1} \sigma_k {\color{red}k}.$ One might have expected that it would be an `$n$' instead but that is not the case.

\section{Discriminant of a polynomial}

\begin{defn}%[]
    Let $f(x) \in \mathbb{F}[x]$ be a non-constant monic polynomial and $\mathbb{K}$ be a splitting field of $f(x)$ over $\mathbb{F}.$ Write
    \begin{equation*} 
        f(x) = (x - r_1) \cdots (x - r_n)
    \end{equation*}
    for $r_1, \ldots, r_n \in \mathbb{K}.$ Then, the \deff{discriminant of $f(x)$} is defined as
    \begin{equation*} 
        \disc_{\mathbb{K}}(f(x)) \vcentcolon= \prod_{1 \le i < j \le n} (r_i - r_j)^2.
    \end{equation*}
\end{defn}

\begin{rem} \label{rem:discrepeatedroots}
    Note that $\disc_{\mathbb{K}}(f(x)) = 0 \iff f(x)$ has repeated roots in $\mathbb{K}.$

    Moreover, by construction, $\disc_{\mathbb{K}}(f(x))$ has a square root in $\mathbb{K},$ namely
    \begin{equation*} 
        \prod_{1 \le i < j \le n} (r_i - r_j) \in \mathbb{K}.
    \end{equation*}
\end{rem}

\begin{restatable}[]{prop}{independencediscriminant}
\label{prop:independencediscriminant}
    Let $f(x) \in \mathbb{F}[x]$ be non-constant and monic. Suppose $\mathbb{K}$ and $\mathbb{K}'$ are two splitting fields of $f(x)$ over $\mathbb{F}.$ Then,
    \begin{equation*} 
        \disc_{\mathbb{K}}(f(x)) = \disc_{\mathbb{K}'}(f(x)) \in \mathbb{F}.
    \end{equation*} 

    In other words, the discriminant takes values in $\mathbb{F}$ and is independent of the splitting field chosen. \hfill\hyperref[prop:independencediscriminant2]{\downsym}
\end{restatable}

In view of the (proof of the) above proposition, we have the following alternate definition of discriminant. (See the remark right after the definition, if you are confused.)

\begin{defn}%[]
    Let $f(x) = x^n - \sigma_1x^{n - 1} + \cdots + (-1)^n\sigma_n \in \mathbb{F}[x]$ be a monic polynomial. Define $w_k$ for $k = 1, \ldots, 2n - 2$ as in \Cref{eq:newident}. Then, 
    \begin{equation*} 
        \disc(f(x)) \vcentcolon= \det \begin{bmatrix}
            n & w_1 & \cdots & w_{n - 1}\\
            w_1 & w_2 & \cdots & w_n\\
            w_2 & w_3 & \cdots & w_{n + 1}\\
            \vdots & \vdots & \ddots & \vdots \\
            w_{n - 1} & w_n & \cdots & w_{2n - 2}\\
        \end{bmatrix}.
    \end{equation*}
\end{defn}

\begin{rem}
    In the above, $\sigma_i$ are not the elementary symmetric polynomials, they are simply elements of $\mathbb{F}.$ We are \emph{defining} $w_k$ recursively in terms of $\sigma_i$ using the relations given in \Cref{eq:newident}.

    An alternate (but longer) definition could have been to start with $f(x)  = x^n - a_1x^{n - 1} + \cdots a_n \in \mathbb{F}[x]$ and define 
    \begin{equation*} 
        w_k \vcentcolon= \begin{cases}
            a_1 w_{k - 1} - a_2w_{k - 2} + \cdots + (-1)^k a_{k - 1}w_1 + (-1)^{k + 1}a_k k & k \le n,\\
            a_1 w_{k - 1} - a_2w_{k - 2} + \cdots + (-1)^{n + 1} a_{n}w_{k - n} & k > n                
        \end{cases}
    \end{equation*}
    and then write the determinant.
\end{rem}

\begin{restatable}[Discriminant in terms of derivative]{prop}{discderivative}
\label{prop:discderivative}
    Suppose $f(x) = \prod_{i = 1}^{n}(x - r_i).$ Then, $\disc(f(x)) = (-1)^{\binom{n}{2}}\prod_{i = 1}^{n}f'(r_i).$ \hfill\hyperref[prop:discderivative2]{\downsym}
\end{restatable}

The derivative is formally defined later, it is \Cref{defn:formalderivative}.

\begin{ex}[Discriminant of a quadratic]
    Let $x^2 + bx + c \in \mathbb{F}[x]$ be a quadratic. We have $\sigma_1 = -b,$ $\sigma_2 = c.$ Thus, we have
    \begin{align*} 
        w_1 &= -b,\\
        w_2 &= b^2 - 2c.
    \end{align*}
    Thus,
    \begin{equation*} 
        \disc(f(x)) = \det\begin{bmatrix}
            2 & -b\\
            -b & b^2 - 2c
        \end{bmatrix} = b^2 - 4c.
    \end{equation*}
    This is the usual discriminant of a quadratic.
\end{ex}

\begin{ex}[Discriminant of a special cubic] \label{ex:disccubic}
    Let $x^3 + px + q \in \mathbb{F}[x]$ be a cubic. Here, $\sigma_1 = 0,$ $\sigma_2 = p,$ and $\sigma_3 = -q.$ Then, Newton's identities become
    \begin{align*} 
        w_1 &= 0,\\
        w_2 &= -2p,\\
        w_3 &= -3q,\\
        w_4 &= 2p^2.
    \end{align*}
    Thus, $\disc(f(x)) = -4p^3 - 27q^2.$
\end{ex}

\section{The Fundamental Theorem of Algebra}

Recall the following facts.

\begin{restatable}[]{lem}{FTAprelim}
\label{lem:FTAprelim}
    \phantom{hi}
    \begin{enumerate}
        \item Every real polynomial of odd degree has a real root.
        \item Every complex number has a square root. Thus, every complex quadratic polynomial has a root in $\mathbb{C}.$ \hfill\hyperref[lem:FTAprelim2]{\downsym}
    \end{enumerate} 
\end{restatable}

\begin{restatable}[Fundamental Theorem of Algebra]{thm}{FTA}
\label{thm:FTA}
    Every non-constant complex polynomial has a root in $\mathbb{C}.$ \hfill\hyperref[thm:FTA2]{\downsym}
\end{restatable}
\chapter{Algebraic Closure of a Field}

\section{Existence}

\begin{defn}%[]
    A field $\mathbb{K}$ is called an \deff{algebraically closed field} if every non-constant polynomial $f(x) \in \mathbb{K}[x]$ has a root in $\mathbb{K}.$
\end{defn}

\begin{defn}%[]
    Let $\mathbb{K}/\mathbb{F}$ be a field extension. We say that $\mathbb{K}$ is an \deff{algebraic closure of $\mathbb{F}$} if $\mathbb{K}$ is algebraically closed and $\mathbb{K}/\mathbb{F}$ is an algebraic extension.
\end{defn}

We have the following simple proposition.
\begin{prop}
    \phantom{hi}
    \begin{enumerate}
        \item $\mathbb{K}$ is algebraically closed iff every non-constant polynomials factors as a product of linear factors.
        \item $\mathbb{C}$ is algebraically closed.
        \item If $\mathbb{K}$ is algebraically closed and $\mathbb{L}/\mathbb{K}$ is an algebraic extension, then $\mathbb{L} = \mathbb{K}.$
    \end{enumerate}
\end{prop}

\begin{restatable}[]{prop}{alglcosureinalgclosedisclosed}
\label{prop:alglcosureinalgclosedisclosed}
    Let $\mathbb{F} \subset \mathbb{K}$ be an extension where $\mathbb{K}$ is algebraically closed. Define,
    \begin{equation*} 
        \mathbb{A} \vcentcolon= \{\alpha \in \mathbb{K} : \alpha \text{ is algebraic over }\mathbb{F}\}.
    \end{equation*}
    Then, $\mathbb{A}$ is an algebraic closure of $\mathbb{F}.$ \hfill\hyperref[prop:alglcosureinalgclosedisclosed2]{\downsym}
\end{restatable}

\begin{restatable}[]{lem}{unionoffields}
\label{lem:unionoffields}
    Let $\{\mathbb{F}_i\}_{i \ge 1}$ be a sequence of fields as
    \begin{equation*} 
        \mathbb{F}_1 \subset \mathbb{F}_2 \subset \cdots.
    \end{equation*}
    Then, $\mathbb{F} \vcentcolon= \bigcup_{i \ge 1}\mathbb{F}_i$ is a field with the following operations:
    Given $a, b \in \mathbb{F},$ there exist smallest $i, j \in \mathbb{N}$ with $a \in \mathbb{F}_i$ and $b \in \mathbb{F}_j.$ Then, $a, b \in \mathbb{F}_{i + j}.$ Define $a + b$ and $ab$ to be the corresponding elements from $\mathbb{F}_{i + j}.$

    Moreover, each $\mathbb{F}_i$ is a subfield of $\mathbb{F}.$ \hfill\hyperref[lem:unionoffields2]{\downsym}
\end{restatable}

Note that the ``smallest'' above is just to ensure that the operations are well-defined. Since $\mathbb{F}_i \subset \mathbb{F}_j$ (note that we always use this to mean ``is a subfield of'') for $i \le j,$ we can actually pick any $i$ and $j.$

\begin{restatable}[Existence of Algebraic Closed Extension]{thm}{algclosedext}
\label{thm:algclosedext}
    Let $\mathbb{F}$ be a field. Then, there exists an algebraically closed field containing $\mathbb{F}.$ \hfill\hyperref[thm:algclosedext2]{\downsym}
\end{restatable}
The proof we have given is due to Artin.

\begin{restatable}[Existence of Algebraic Closure]{cor}{algclosure}
\label{cor:algclosure}
    Every field $\mathbb{F}$ has an algebraic closure. \hfill\hyperref[cor:algclosure2]{\downsym}
\end{restatable}

\section{Uniqueness}

\begin{restatable}[]{prop}{rootsandextensions}
\label{prop:rootsandextensions}
    Let $\sigma : \mathbb{F} \to \mathbb{L}$ be an embedding of fields where $\mathbb{L}$ is \underline{algebraically closed}. Let $\alpha \in \mathbb{K} \supset \mathbb{F}$ be algebraic over $\mathbb{F}$ and $p(x) = \irr(\alpha, \mathbb{F}).$ \\
    Write $p(x) = \sum a_i x^i$ and define $p^{\sigma}(x) \vcentcolon= \sum \sigma(a_i) x^i.$ Then, $\tau \mapsto \tau(\alpha)$ is a bijection between the sets
    \begin{equation*} 
        \{\tau : \mathbb{F}(\alpha) \to \mathbb{L} \mid \tau \text{ is an embedding and }\tau|_{\mathbb{F}} = \sigma\} \leftrightarrow \{\beta \in \mathbb{L} \mid p^{\sigma}(\beta) = 0\}.
    \end{equation*} \hfill\hyperref[prop:rootsandextensions2]{\downsym}
\end{restatable}

\begin{rem}
    The above proposition says that the number of ways to extend from $\mathbb{F}$ to $\mathbb{F}(\alpha)$ is precisely the number of roots of that $p(x)$ has in $\mathbb{L}.$ (Not exactly, we need to apply $\sigma$ to the coefficients. This is essentially saying that we consider $\mathbb{F}$ as a subfield under $\mathbb{L}.$) In particular, this set is non-empty since $\mathbb{L}$ is algebraically closed. \\
    Note that this number need not be $\deg(f(x)).$ We shall see in the next chapter that a polynomial may be irreducible but still have repeated roots in its splitting field.
\end{rem}

\begin{restatable}[]{thm}{extendtoalgextension}
\label{thm:extendtoalgextension}
    Let $\sigma : \mathbb{F} \to \mathbb{L}$ be an embedding where $\mathbb{L}$ is algebraically closed. Let $\mathbb{K}/\mathbb{F}$ be an algebraic extension. Then, there exists an embedding $\tau : \mathbb{K} \to \mathbb{L}$ extending $\sigma.$ \\
    Moreover, if $\mathbb{K}$ is an algebraic closure of $\mathbb{F}$ and $\mathbb{L}$ of $\sigma(\mathbb{K}),$ then $\tau$ is an isomorphism extending $\sigma.$ \hfill\hyperref[thm:extendtoalgextension2]{\downsym}
\end{restatable}

\begin{cor}[Isomorphism of algebraic closures]
    If $\mathbb{K}_1$ and $\mathbb{K}_2$ are two algebraic closures of $\mathbb{F},$ then they are $\mathbb{F}$-isomorphic.
\end{cor}
\begin{proof} 
    Apply previous proposition to the inclusion $i : \mathbb{F} \hookrightarrow \mathbb{E}_2$ to extend it to an $\mathbb{F}$-isomorphism $\tau : \mathbb{E}_1 \to \mathbb{E}_2.$
\end{proof}

\begin{defn}%[]
    Given a field $\mathbb{F},$ we use $\overline{\mathbb{F}}$ to denote an algebraic closure of $\mathbb{F}.$ 
\end{defn}

\begin{restatable}[Isomorphism of splitting fields]{thm}{isosplitting}
\label{thm:isosplitting}
    Let $\mathbb{E}$ and $\mathbb{E}'$ be two splitting fields of a non-constant polynomial $f(x) \in \mathbb{F}[x]$ over $\mathbb{F}.$ Then, they are $\mathbb{F}$-isomorphic. \hfill\hyperref[thm:isosplitting2]{\downsym}
\end{restatable}
\chapter{Separable extensions}

\section{Derivatives}

\begin{defn}%[]
    Let $\mathbb{F}$ be a field. Define the $\mathbb{F}$-linear map $\D_{\mathbb{F}} : \mathbb{F}[x] \to \mathbb{F}[x]$ by
    \begin{equation*} 
        \D_{\mathbb{F}}\left(\sum_{i = 0}^{n} a_ix^i\right) = \sum_{i = 1}^{n}ia_i x^{i - 1}.
    \end{equation*}
    Given $f(x) \in \mathbb{F}[x],$ we call $\D_{\mathbb{F}}(f(x))$ the \deff{derivative} of $f(x)$ and also denote it by $f'(x).$
\end{defn}

\begin{rem}
    Note that the above definition requires no notion of limits. For the case of $\mathbb{F} = \mathbb{R}$ or $\mathbb{C},$ then it coincides with the usual definition if we identify a polynomial with the function it represents. We shall not require this, however.
\end{rem}

We have the follow easy-to-check proposition.
\begin{prop}
    Let $f(x), g(x) \in \mathbb{F}[x]$ and $a \in \mathbb{F}$ be arbitrary. Then,
    \begin{enumerate}
        \item $(f \pm ag)'(x) = f'(x) \pm ag'(x),$
        \item $(fg)'(x) = f'(x)g(x) + f(x)g'(x).$
    \end{enumerate}
\end{prop}
The first point is just verifying that $\D_{\mathbb{F}}$ is indeed $\mathbb{F}$-linear.

\begin{prop}
    Let $\mathbb{F} \subset \mathbb{E}$ be a field extension. Then, $\D_{\mathbb{E}}|_{\mathbb{F}} = \D_{\mathbb{F}}.$ Thus, the notation $f'(x)$ is unambiguous.
\end{prop}

\begin{defn}%[]
    Let $f(x) \in \mathbb{F}[x]$ be a non-constant monic polynomial. Let $\mathbb{E}$ be a splitting field of $f(x)$ over $\mathbb{F}.$ In $\mathbb{E}[x],$ factorise $f(x)$ uniquely as
    \begin{equation*} 
        f(x) = (x - r_1)^{e_1} \cdots (x - r_g)^{e_g},
    \end{equation*}
    where $r_1, \ldots, r_g \in \mathbb{E}$ are distinct and each $e_i \in \mathbb{N}.$ 

    The numbers $e_1, \ldots, e_g$ are called the \deff{multiplicities} of the roots $r_1, \ldots, r_n.$ \\
    If $e_i = 1$ for some $i,$ then $r_i$ is called a \deff{simple root} and a \deff{repeated root} otherwise.

    If each $e_i = 1,$ then $f(x)$ is said to be a \deff{separable polynomial}.

    If $f$ is not monic, we have the same definitions upon division by the leading coefficient.
\end{defn}

\begin{rem}
    Note that the definition of ``separable polynomial'' is ad hoc since the separability presumably depends on the splitting field. However, in view of \Cref{rem:discrepeatedroots}, we see that separability depends only on $\disc(f(x)),$ which we have seen to be independent of the splitting field.\\
    The next proposition shows something even stronger.

    Also, note that one might think that an irreducible polynomial is always separable. We will see an example of how that is not true, in general. Over fields of characteristic $0,$ however, it is true. We shall prove that as well.
\end{rem}

\begin{restatable}[]{prop}{multindepsplitting}
\label{prop:multindepsplitting}
    The number of roots and their multiplicities are independent of the splitting field chosen for $f(x)$ over $\mathbb{F}.$ \hfill\hyperref[prop:multindepsplitting2]{\downsym}
\end{restatable}

\begin{restatable}[]{prop}{derivcritreproot}
\label{prop:derivcritreproot}
    Let $f(x) \in \mathbb{F}[x]$ be a monic and let $r \in \mathbb{E} \supset \mathbb{F}$ be a root of $f(x).$ \\
    Then, $r$ is a repeated root iff $f'(r) = 0.$ \hfill\hyperref[prop:derivcritreproot2]{\downsym}
\end{restatable}

\begin{restatable}[The Derivative Criterion for Separability]{thm}{derivcritsep}
\label{thm:derivcritsep}
    Let $f(x) \in \mathbb{F}[x]$ be a monic polynomial.
    \begin{enumerate}
         \item If $f'(x) = 0,$ then every root of $f(x)$ is a multiple root.
         \item If $f'(x) \neq 0,$ then $f(x)$ has simple roots iff $\gcd(f(x), f'(x)) = 1.$ \hfill\hyperref[thm:derivcritsep2]{\downsym}
     \end{enumerate} 
\end{restatable}

\begin{restatable}[]{prop}{irredsepderiv}
\label{prop:irredsepderiv}
    Let $f(x) \in \mathbb{F}[x]$ be irreducible and non-constant.
    \begin{enumerate}
         \item $f(x)$ is separable iff $f'(x) \neq 0.$
         \item If $\chr(\mathbb{F}) = 0,$ then $f(x)$ is separable. 
     \end{enumerate} 
     In other words, irreducible polynomials over fields of characteristic $0$ are separable. \hfill\hyperref[prop:irredsepderiv2]{\downsym}
\end{restatable}

\begin{ex}
    Let $p \in \mathbb{N}$ be a prime. Consider the field $\mathbb{F}_p(X)$ and the polynomial $f(T) = T^p - X \in \mathbb{F}_p(X)[T].$ \\
    Then, $f(T)$ is irreducible, by applying Eisenstein at the prime $X.$ However, $f'(T) = 0$ and hence, not separable.

    In fact, as we shall see, the existence of $p$-th roots will play an important role.
\end{ex}

\begin{defn}%[]
    Let $\mathbb{F}$ be a field of prime characteristic $p.$ Define
    \begin{equation*} 
        \mathbb{F}^p \vcentcolon= \{\alpha^p \in \mathbb{F} : \alpha \in \mathbb{F}\}.
    \end{equation*}
    That is, $\mathbb{F}^p$ is the set of all $p$-th powers of elements of $\mathbb{F}.$
\end{defn}

\begin{prop}
    $\mathbb{F}^p$ is a subfield of $\mathbb{F}.$
\end{prop}
\begin{proof} 
    Only closure under addition is not so obvious. Note that $(x + y)^p = x^p + y^p$ for all $x, y \in \mathbb{F}.$
\end{proof}

\begin{restatable}[]{prop}{xppolyirredorroot}
\label{prop:xppolyirredorroot}
    Let $\mathbb{F}$ be a field with $\chr(\mathbb{F}) = p > 0.$ Then, $x^p - a \in \mathbb{F}[x]$ is either irreducible in $\mathbb{F}[x]$ or $a \in \mathbb{F}^p.$ \hfill\hyperref[prop:xppolyirredorroot2]{\downsym}
\end{restatable}

\begin{restatable}[]{prop}{nonseppowerp}
\label{prop:nonseppowerp}
    Let $f(x) \in \mathbb{F}[x]$ be an irreducible polynomial and let $p \vcentcolon= \chr(\mathbb{F}) > 0.$ If $f(x)$ is not separable, then there exists $g(x) \in \mathbb{F}[x]$ such that $f(x) = g(x^p).$ \hfill\hyperref[prop:nonseppowerp2]{\downsym}
\end{restatable}

\section{Perfect fields}
\begin{defn}%[]
    Let $\mathbb{F} \subset \mathbb{K}$ be a field extension. 

    An algebraic element $\alpha \in \mathbb{K}$ over $\mathbb{F}$ is called a \deff{separable element over $\mathbb{F}$} if $\irr(\alpha, \mathbb{F})$ is separable over $\mathbb{F}.$

    We say that $\mathbb{K}/\mathbb{F}$ is a \deff{separable field extension} if every $\alpha \in \mathbb{K}$ is separable (and in particular, algebraic). 

    We say that $\mathbb{F}$ is a \deff{perfect field} if every algebraic extension of $\mathbb{F}$ is separable. Equivalently, every irreducible polynomial in $\mathbb{F}[x]$ is separable.
\end{defn}

\begin{ex}
    \phantom{hi}
    \begin{enumerate}
        \item We had seen that $\mathbb{F}_p(X)$ is not perfect for any prime $p.$
        \item By \Cref{prop:irredsepderiv}, we have that every field of characteristic $0$ is perfect.
    \end{enumerate}
\end{ex}

\begin{restatable}[]{thm}{perfectiffppower}
\label{thm:perfectiffppower}
    Let $\mathbb{F}$ be a field with characteristic $p > 0.$ Then, $\mathbb{F}$ is perfect iff $\mathbb{F} = \mathbb{F}^p.$ \hfill\hyperref[thm:perfectiffppower2]{\downsym}
\end{restatable}

\begin{restatable}[]{cor}{finitefieldperfect}
\label{cor:finitefieldperfect}
    Every finite field is perfect. \hfill\hyperref[cor:finitefieldperfect2]{\downsym}
\end{restatable}

\section{Extensions of embeddings}
\begin{restatable}[]{prop}{samemultirredpoly}
\label{prop:samemultirredpoly}
    Let $f(x) \in \mathbb{F}[x]$ be an irreducible monic polynomial. Then, all roots of $f(x)$ have equal multiplicity (in any splitting field). \\
    If $\chr(\mathbb{F}) = 0,$ then all roots are simple. \\
    If $\chr(\mathbb{F}) =\vcentcolon p > 0,$ then all roots have multiplicity $p^n$ for some $n \in \mathbb{N}_0.$ \hfill\hyperref[prop:samemultirredpoly2]{\downsym}
\end{restatable}
Note that by \Cref{prop:multindepsplitting}, the $n$ also does not depend on choice of splitting field.

\begin{restatable}[]{thm}{separabledegreedef}
\label{thm:separabledegreedef}
    Let $\sigma : \mathbb{F} \to \mathbb{L}$ be an embedding of fields where $\mathbb{L}$ is an algebraic closure of $\sigma(\mathbb{F}).$ Similarly, let $\tau : \mathbb{F} \to \mathbb{L}'$ be an embedding of fields where $\mathbb{L}'$ is an algebraic closure of $\tau(\mathbb{F}).$ Let $\mathbb{E}$ be an algebraic extension of $\mathbb{F}.$

    Let $S_\sigma$ (resp. $S_\tau$) denote the set of extensions of $\sigma$ (resp. $\tau$) to embeddings of $\mathbb{E}$ into $\mathbb{L}$ (resp. $\mathbb{L}'$). Let $\lambda : \mathbb{L} \to \mathbb{L}'$ be an isomorphism extending $\tau \circ \sigma^{-1} : \sigma(\mathbb{F}) \to \tau(\mathbb{F}).$  

    The map $\psi : S_\sigma \to S_\tau$ given by $\psi(\widetilde{\sigma}) = \lambda \circ \widetilde{\sigma}$ is a bijection. \hfill\hyperref[thm:separabledegreedef2]{\downsym}

    \begin{center}
        \begin{tikzcd}[ampersand replacement=\&]
        \mathbb{L}' \arrow[dd, no head]                  \&  \&                                                                                                     \&  \& \mathbb{L} \arrow[llll, "\lambda"'] \arrow[dd, no head] \\
                                                         \&  \&                                                                                                     \&  \&                                                         \\
        \widetilde{\tau}(\mathbb{E}) \arrow[dd, no head] \&  \& \mathbb{E} \arrow[ll, "\widetilde{\tau} \in S_\tau"'] \arrow[rr, "\widetilde{\sigma} \in S_\sigma"] \&  \& \widetilde{\sigma}(\mathbb{E}) \arrow[dd, no head]      \\
                                                         \&  \&                                                                                                     \&  \&                                                         \\
        \tau(\mathbb{F})                                 \&  \& \mathbb{F} \arrow[rr, "\sigma"] \arrow[ll, "\tau"']                                                 \&  \& \sigma(\mathbb{F})                                     
        \end{tikzcd}
    \end{center}  
\end{restatable}

\begin{rem}
    What the above proposition is really saying is that the ``number'' (cardinality) of extensions does not depend on $\mathbb{L}$ \textbf{\emph{or}} on the embedding $\sigma.$ Note that since $\mathbb{E}$ is an arbitrary algebraic extension, the set $S_\sigma$ need not be finite. 

    Thus, we may assume $\mathbb{L} \supset \mathbb{F}$ to be an algebraic closure and $\sigma$ to be the natural inclusion.
\end{rem}

\begin{defn}%[]
    If $\mathbb{E}/\mathbb{F}$ is an algebraic extension, then the cardinality of $S_\sigma$ (as in \Cref{thm:separabledegreedef}) is called the \deff{separable degree} of $\mathbb{E}/\mathbb{F}$ and is denoted $[\mathbb{E} : \mathbb{F}]_s.$
\end{defn}

\begin{rem}
    Note that if $\sigma : \mathbb{F} \to \mathbb{L}$ is an embedding into an algebraically closed field $\mathbb{L},$ and $\widetilde{\sigma} : \mathbb{E} \to \mathbb{L}$ is an extension of $\sigma,$ where $\mathbb{E}/\mathbb{F}$ is algebraic, then $\widetilde{\sigma}(\mathbb{E})$ is actually contained in the algebraic closure of $\sigma(\mathbb{F})$ within $\mathbb{L}.$ Thus, it is fine even if $\mathbb{L}$ is not an algebraic closure of $\sigma(\mathbb{F}).$
\end{rem}

\begin{prop} \label{prop:sepdeglessthannordeg}
    Let $\alpha \in \mathbb{E} \supset \mathbb{F}$ be algebraic over $\mathbb{F}$ and $n \vcentcolon= \deg(\irr(\alpha, \mathbb{F})).$ Then, $[\mathbb{F}(\alpha) : \mathbb{F}]_s \le n = [\mathbb{F}(a) : \mathbb{F}]$ with equality iff $\alpha$ is separable over $\mathbb{F}.$
\end{prop}
\begin{proof} 
    By \Cref{prop:rootsandextensions}, we know that $[\mathbb{F}(\alpha) : \mathbb{F}]_s$ is exactly the number of roots of $p(x) = \irr(\alpha, \mathbb{F})$ in $\overline{\mathbb{F}}.$ This is at most $n = \deg(p(x)).$ Moreover, equality implies that all roots are distinct and hence, $\alpha$ is separable.
\end{proof}

\begin{restatable}[Tower Law for separable degree]{thm}{towerlawsep}
\label{thm:towerlawsep}
    Let $\mathbb{F} \subset \mathbb{E} \subset \mathbb{K}$ be a tower of finite algebraic extensions. Then, $[\mathbb{E} : \mathbb{F}]_s \le [\mathbb{E} : \mathbb{F}]$ and
    \begin{equation*} 
        [\mathbb{K} : \mathbb{F}]_s = [\mathbb{K} : \mathbb{E}]_s [\mathbb{E} : \mathbb{F}]_s.
    \end{equation*} \hfill\hyperref[thm:towerlawsep2]{\downsym}
\end{restatable}

\begin{cor}
    Let $\mathbb{F} \subset \mathbb{E} \subset \mathbb{K}$ be a tower of finite algebraic extensions. Then, $[\mathbb{K} : \mathbb{F}] = [\mathbb{K} : \mathbb{F}]_s$ iff equality holds at each stage.
\end{cor}

\begin{restatable}[]{thm}{sepiffdegequal}
\label{thm:sepiffdegequal}
    Let $\mathbb{E}/\mathbb{F}$ be a finite extension. Then, $\mathbb{E}/\mathbb{F}$ is separable iff $[\mathbb{E} : \mathbb{F}]_s = [\mathbb{E} : \mathbb{F}].$ \hfill\hyperref[thm:sepiffdegequal2]{\downsym}
\end{restatable}

\begin{cor} \label{cor:adjoiningsepissep}
    Let $\alpha \in \mathbb{E} \supset \mathbb{F}$ be separable over $\mathbb{F}.$ Then, $\mathbb{F}(\alpha)/\mathbb{F}$ is a separable extension.
\end{cor}
\begin{proof}
    By \Cref{prop:sepdeglessthannordeg}, we have $[\mathbb{F}(\alpha) : \mathbb{F}]_s = [\mathbb{F}(\alpha) : \mathbb{F}].$ By \Cref{thm:sepiffdegequal}, this means that $\mathbb{F}(\alpha)/\mathbb{F}$ is separable.
\end{proof}

\begin{restatable}[]{prop}{compdecompsep}
\label{prop:compdecompsep}
    Let $\mathbb{F} \subset \mathbb{E} \subset \mathbb{K}$ be a tower of fields. Then,\\
    $\mathbb{K}/\mathbb{F}$ is separable iff $\mathbb{K}/\mathbb{E}$ and $\mathbb{E}/\mathbb{F}$ are separable. \hfill\hyperref[prop:compdecompsep2]{\downsym}
\end{restatable}

\begin{cor}
    Let $f(x) \in \mathbb{F}[x]$ be a separable polynomial and $\mathbb{E} \supset \mathbb{F}$ be a splitting field of $f(x)$ over $\mathbb{F}.$ Then, $\mathbb{E}/\mathbb{F}$ is separable.
\end{cor}
\begin{proof}
    Write $\mathbb{E} = \mathbb{F}(r_1, \ldots, r_n)$ where $f(x) = a(x - r_1) \cdots (x - r_n)$ and use the previous corollary and proposition repeatedly.
\end{proof}

\begin{restatable}[]{prop}{sepdegdividesdeg}
\label{prop:sepdegdividesdeg}
    Let $\mathbb{E}/\mathbb{F}$ be a finite extension. Then, $[\mathbb{E} : \mathbb{F}]_s$ divides $[\mathbb{E} : \mathbb{F}].$ If $\chr(\mathbb{F}) =\vcentcolon p > 0,$ then quotient $\dfrac{[\mathbb{E} : \mathbb{F}]}{[\mathbb{E} : \mathbb{F}]_s}$ is a power of $p.$ \hfill\hyperref[prop:sepdegdividesdeg2]{\downsym}
\end{restatable}


\section{Finite fields}

\section{Existence and Uniqueness}

In this section, $p$ will denote an arbitrary prime number.

\begin{restatable}[Uniqueness of finite fields]{thm}{uniquefinfields}
\label{thm:uniquefinfields}
    Let $\mathbb{K}$ and $\mathbb{L}$ be finite fields with same cardinality. Then, $\mathbb{K}$ and $\mathbb{L}$ are isomorphic. \hfill\hyperref[thm:uniquefinfields2]{\downsym}
\end{restatable}

\begin{defn}%[]
    We shall denote the finite with $p^n$ elements by $\mathbb{F}_{p^n}.$
\end{defn}

\begin{rem}
    We have not yet shown that $\mathbb{F}_{p^n}$ for every prime $p$ and $n \in \mathbb{N}.$ Have only shown uniqueness up to isomorphism.
\end{rem}

\begin{restatable}[Existence of finite fields]{thm}{existencefinfields}
\label{thm:existencefinfields}
    Fix a prime $p$ and an algebraic closure $\overline{\mathbb{F}}_p.$ For every $n \in \mathbb{N},$ there exists a unique subfield of $\overline{\mathbb{F}}_p$ of size $p^n,$ denoted $\mathbb{F}_{p^n}.$ Moreover
    \begin{equation*} 
        \overline{\mathbb{F}}_p = \bigcup_{n \in \mathbb{N}} \mathbb{F}_{p^n}.
    \end{equation*}
    \hfill\hyperref[thm:existencefinfields2]{\downsym}
\end{restatable}

\section{Gauss' Necklace Formula}

Recall the M\"obius inversion formula.

\begin{defn}%[]
    The \deff{M\"obius function} $\mu : \mathbb{N} \to \mathbb{N}$ is defined as
    \begin{equation*} 
        \mu(n) \vcentcolon= \begin{cases}
            1 & n = 1,\\
            (-1)^r & n \text{ is a product of }r \text{ distinct primes},\\
            0 & p^2 \mid n \text{ for some prime }p.
        \end{cases}
    \end{equation*}
\end{defn}

\begin{thm}[M\"obius inversion formula]
    Let $f, g : \mathbb{N} \to \mathbb{N}$ be functions satisfying
    \begin{equation*} 
        f(n) = \sum_{d \mid n} g(d).
    \end{equation*}
    Then, they also satisfy
    \begin{equation*} 
        g(n) = \sum_{d \mid n} f\left(\frac{n}{d}\right)\mu(d).
    \end{equation*}
\end{thm}

\begin{restatable}[Gauss]{thm}{gaussnecklace}
\label{thm:gaussnecklace}
    The number of irreducible polynomials of degree $n$ over $\mathbb{F}_{q}$ is given by
    \begin{equation*} 
        N_q(n) = \frac{1}{n}\sum_{d \mid n} \mu(d)q^{n/d}.
    \end{equation*} \hfill\hyperref[thm:gaussnecklace2]{\downsym}
\end{restatable}

\gaussnecklace*\label{thm:gaussnecklace2}
\begin{flushright}\hyperref[thm:gaussnecklace]{\upsym}\end{flushright}
\begin{proof}
    
\end{proof}

\chapter{Proofs}

\finextisalg*\label{prop:finextisalg2}
\begin{flushright}\hyperref[prop:finextisalg]{\upsym}\end{flushright}
\begin{proof}
    Let $\mathbb{K}/\mathbb{F}$ be a finite extension with $n \vcentcolon= \dim_{\mathbb{F}}(\mathbb{K}).$ Let $b \in \mathbb{K}$ be arbitrary. Consider the multiset $\{1, b, \ldots, b^{n}\}.$ It has $n + 1$ elements and thus, is linearly dependent. Thus, there exist $a_0, \ldots, a_{n} \in \mathbb{F}$ not all $0$ such that
    \begin{equation*} 
        a_0 + a_1b + \cdots + a_nb^n = 0.
    \end{equation*}
    Then, $f(x) \vcentcolon= a_0 + a_1b + \cdots + a_nx^n \in \mathbb{F}[x]$ is a non-zero polynomial such that $f(b) = 0.$
\end{proof}

\uniquemonicirred*\label{prop:uniquemonicirred2}
\begin{flushright}\hyperref[prop:uniquemonicirred]{\upsym}\end{flushright}
\begin{proof}
    Define $\psi : \mathbb{F}[x] \to \mathbb{K}$ by $p(x) \mapsto p(\alpha).$ Since $\alpha$ is algebraic, $I \vcentcolon= \ker(\psi)$ is non-zero.

    Since $\mathbb{F}[x]$ is a PID, we have $I = \langle f(x)\rangle$ for some $0 \neq f(x) \in \mathbb{F}[x].$ Since $\mathbb{F}[x]/I$ is isomorphic to a subring of $\mathbb{K},$ it is an integral domain and hence, $f(x)$ is irreducible. By scaling, we may assume that $f(x)$ is monic. Clearly, any other $g(x)$ as in the proposition is in the kernel and hence, $f(x) \mid g(x).$

    In particular, if $g(x)$ is irreducible and monic, then $f(x) \mid g(x) \implies g(x) = af(x)$ for some $a \in \mathbb{F}^\times.$ Since $g(x)$ is also monic, we have $a = 1.$
\end{proof}

\adjoiningalg*\label{prop:adjoiningalg2}
\begin{flushright}\hyperref[prop:adjoiningalg]{\upsym}\end{flushright}
\begin{proof}
    Consider the substitution homomorphism $\psi : \mathbb{F}[x] \to \mathbb{F}[\alpha]$ given by $p(x) \mapsto p(\alpha).$

    By \Cref{prop:uniquemonicirred}, we know that $\ker(\psi) = \langle f(x)\rangle.$ Since $f(x) \neq 0,$ the ideal $\langle f(x)\rangle$ is maximal. 

    Since $\psi$ is onto and $\ker(\psi)$ maximal, we see that $\mathbb{F}[\alpha]$ is in fact a field and hence, $\mathbb{F}[\alpha] = \mathbb{F}(\alpha).$

    Consider $B = \{1, \alpha, \ldots, \alpha^{n - 1}\}.$ \\
    Using $f(x),$ we may recursively write all higher powers of $\alpha$ as an $\mathbb{F}$-linear combination of elements of $B.$ Thus, $B$ spans $\mathbb{F}[\alpha].$ \\
    For linear independence, suppose that $a_0, \ldots, a_{n - 1} \in \mathbb{F}$satisfy
    \begin{equation*} 
        a_0 + a_1\alpha + \cdots + a_{n - 1}\alpha^{n - 1} = 0.
    \end{equation*}
    Then, we get a polynomial $g(x) = a_0 + a_1x + \cdots a_{n - 1}x^{n - 1} \in \mathbb{F}[x]$ satisfied by $\alpha.$ Since $\deg(g(x)) < \deg(f(x)),$ we see that $g(x) = 0,$ again by \Cref{prop:uniquemonicirred}.
\end{proof}

\isocarryingalphtobet*\label{prop:isocarryingalphtobet2}
\begin{flushright}\hyperref[prop:isocarryingalphtobet]{\upsym}\end{flushright}
\begin{proof}
    $(\implies)$ Let $\psi : \mathbb{F}(\alpha) \to \mathbb{F}(\beta)$ be as mentioned.\\
    Put $f(x) \vcentcolon= \irr(\alpha, \mathbb{F})$ and $g(x) \vcentcolon= \irr(\beta, \mathbb{F}).$ Then, 
    \[\begin{WithArrows}[displaystyle]
        0 &= \psi(0) \\
        &= \psi(f(\alpha)) \Arrow{$\psi$ is an $\mathbb{F}$-isomorphism} \\
        &= f(\psi(\alpha)) \\
        &= f(\beta).
    \end{WithArrows}\]
     Thus, $g(x) \mid f(x).$ Since both are irreducible and monic, $g(x) = f(x).$

     $(\impliedby)$ Let $f(x) \vcentcolon= \irr(\alpha, \mathbb{F}) = \irr(\beta, \mathbb{F}).$ \\
     The isomorphisms $\mathbb{F}(\alpha) \cong \mathbb{F}[x]/\langle f(x)\rangle \cong \mathbb{F}(\beta)$ are $\mathbb{F}$-isomorphisms and so is their composition.
\end{proof}

\towerlaw*\label{thm:towerlaw2}
\begin{flushright}\hyperref[thm:towerlaw]{\upsym}\end{flushright}
\begin{proof}
    If $\mathbb{K}/\mathbb{F}$ is a finite extension, then so are $\mathbb{K}/\mathbb{E}$ (pick a finite basis of $\mathbb{K}/\mathbb{F},$ it is a spanning set for $\mathbb{K}/\mathbb{E}$) and $\mathbb{E}/\mathbb{F}$ ($\mathbb{E}$ is an $\mathbb{F}$-subspace of $\mathbb{K}.$)

    Thus, if either of $\mathbb{K}/\mathbb{E}$ or $\mathbb{E}/\mathbb{F}$ is not a finite extension, then neither is $\mathbb{K}/\mathbb{F}.$

    Now, assume that both $n \vcentcolon= [\mathbb{K} : \mathbb{E}]$ and $m \vcentcolon= [\mathbb{E} : \mathbb{F}]$ are finite. Let $\{\alpha_i\}_{i = 1}^n \subset \mathbb{K}$ be an $\mathbb{E}$-basis and $\{\beta_j\}_{j = 1}^m \subset \mathbb{E}$ be an $\mathbb{F}$-basis.

    Put $B \vcentcolon= \{\alpha_i\beta_j : 1 \le i \le n,\; 1 \le j \le m\} \subset \mathbb{K}.$ We show that $B$ is an $\mathbb{F}$-basis of $\mathbb{K}.$

    \textbf{Spanning.} Let $a \in \mathbb{K}$ be arbitrary. Write 
    \begin{equation*} 
        a = \sum_{i = 1}^{n} a_i \alpha_i
    \end{equation*}
    for $a_i \in \mathbb{E}.$ For each $i = 1, \ldots, n,$ write
    \begin{equation*} 
        a_i = \sum_{j = 1}^{m} b_{ij} \beta_j
    \end{equation*}
    for $j \in \mathbb{F}.$ Then,
    \begin{equation*} 
        a = \sum_{i = 1}^{n}\sum_{j = 1}^{m}b_{ij} (\alpha_i\beta_j)
    \end{equation*}
    is an $\mathbb{F}$-linear combination of elements of $B.$

    \textbf{Linear independence.} Let $\{b_{ij} : 1 \le i \le n,\; 1 \le j \le m\} \subset \mathbb{F}$ be such that
    \begin{equation*} 
        \sum_{\substack{1 \le i \le n \\ 1 \le j \le m}} b_{ij}\alpha_i\beta_j = 0.
    \end{equation*} 
    Group the above to get
    \begin{equation*} 
        \sum_{i = 1}^{n}\left[\sum_{j = 1}^{m}b_{ij} \alpha_i\right]\beta_j = 0.
    \end{equation*}
    Linear independence of $\{\beta_j\}$ forces $\sum_{j = 1}^{m}b_{ij} \alpha_i = 0$ for all $i.$ In turn, linear independence of $\{\alpha_i\}$ that forces each $b_{ij}$ to be $0.$

    Note that $B$ actually has cardinality $mn.$ (Why?) This finishes the proof.
\end{proof}

\adjoinalgsfinext*\label{prop:adjoinalgsfinext2}
\begin{flushright}\hyperref[prop:adjoinalgsfinext]{\upsym}\end{flushright}
\begin{proof}
    Consider the tower
    \begin{equation*} 
        \mathbb{F} \subset \mathbb{F}(\alpha_1) \subset \mathbb{F}(\alpha_1, \alpha_2) \subset \cdots \subset \mathbb{F}(\alpha_1, \ldots, \alpha_n).
    \end{equation*}
    At each stage, an element being adjoined is algebraic over the previous field. (\Cref{prop:decompalgisalg}.)

    Thus, each consecutive degree above is finite. (\Cref{cor:adjoinalgisfin}.)

    By the \nameref{thm:towerlaw}, so is the overall degree.
\end{proof}

\compalgisalg*\label{cor:compalgisalg2}
\begin{flushright}\hyperref[cor:compalgisalg]{\upsym}\end{flushright}
\begin{proof}
    Let $\alpha \in \mathbb{K}.$ Let $\irr(\alpha, \mathbb{E}) =\vcentcolon f(x) = a_0 + \cdots + a_{n - 1}x^{n - 1} + x^n.$

    Let $\mathbb{L} \vcentcolon= \mathbb{F}(a_0, \ldots, a_{n - 1}).$

    Then, $\mathbb{L}$ is finite over $\mathbb{F}$ since each $a_i \in \mathbb{R}$ is algebraic over $\mathbb{F}.$ Moreover, $0 \neq f(x) \in \mathbb{L}[x].$ Thus, $\alpha$ is algebraic over $\mathbb{L}$ and hence, $\mathbb{L}(\alpha)$ is finite over $\mathbb{L}.$

    By the \nameref{thm:towerlaw}, $\mathbb{L}/\mathbb{F}$ is finite and thus, $\alpha$ is algebraic over $\mathbb{F}.$ (\Cref{prop:finextisalg}.)
\end{proof}

\algclosureisfield*\label{cor:algclosureisfield2}
\begin{flushright}\hyperref[cor:algclosureisfield]{\upsym}\end{flushright}
\begin{proof}
    $\mathbb{F} \subset \mathbb{A}$ is clear. We show that $\mathbb{A}$ is a subfield. Let $\alpha, \beta \in \mathbb{A}$ with $\beta \neq 0.$ Then, $\mathbb{L} \vcentcolon= \mathbb{F}(\alpha, \beta)$ is a finite extension over $\mathbb{F}.$ \\
    Thus, all elements of $\mathbb{L}$ are algebraic over $\mathbb{F}.$ In particular, so are $\alpha \pm \beta,$ $\alpha\beta$ and $\alpha\beta^{-1}.$
\end{proof}

\intdomfinextfield*\label{prop:intdomfinextfield2}
\begin{flushright}\hyperref[prop:intdomfinextfield]{\upsym}\end{flushright}
\begin{proof}
    We only need to show that every non-zero element of $R$ has a multiplicative inverse (in $R$). Let $0 \neq a \in R$ be arbitrary. Since $\dim_{\mathbb{F}}(R) < \infty,$ there is a smallest $n \ge 1$ such that the set $\{1, a, \ldots, a^n\}$ is linearly dependent. Then, let $b_0, \ldots, b_{n} \in \mathbb{F}$ be not all zero such that
    \begin{equation*} 
        b_0 + b_1a + \cdots b_na^n = 0.
    \end{equation*} 
    If $b_n = 0,$ then the minimality of $n$ is contradicted. If $b_0 = 0,$ then we may cancel $a$ ($R$ is an integral domain and $a \neq 0$) and again contradict the minimality of $n.$ Thus, we get
    \begin{equation*} 
        a(b_1 + \cdots + b_na^{n - 1}) = -b_0.
    \end{equation*}
    This shows that
    \begin{equation*} 
        -\frac{1}{b_0}(b_1 + \cdots + b_na^{n - 1}) \in R
    \end{equation*}
    is a multiplicative inverse of $a.$
\end{proof}

\descofcompositum*\label{prop:descofcompositum2}
\begin{flushright}\hyperref[prop:descofcompositum]{\upsym}\end{flushright}
\begin{proof}
    Simple computations show that $\mathbb{L}$ is indeed a subring of $\mathbb{K}.$ If $\{\alpha_1, \ldots, \alpha_n\}$ and $\{\beta_1, \ldots, \beta_m\}$ are $\mathbb{F}$-bases for $\mathbb{E}_1$ and $\mathbb{E}_2,$ then clearly $\{\alpha_i\beta_j : 1 \le i \le n,\; 1 \le j \le m\}$ spans $\mathbb{L}$ over $\mathbb{F}.$ Thus, $\dim_{\mathbb{F}}(\mathbb{L}) \le mn = d.$ 
    
    Since $\mathbb{L}$ is a subring of $\mathbb{K},$ it is an integral domain and hence, $\mathbb{L}$ is a field, by \Cref{prop:intdomfinextfield}.

    Lastly, note that $[\mathbb{E}_i : \mathbb{F}]$ divides $[\mathbb{L} : \mathbb{F}],$ in view of the \nameref{thm:towerlaw}. In particular, if $\gcd(m, n) = 1,$ then $mn \mid [\mathbb{L} : \mathbb{F}].$ Since $[\mathbb{L} : \mathbb{F}] \le mn,$ we are done.
\end{proof}

\rootcanbeadjoined*\label{thm:rootcanbeadjoined2}
\begin{flushright}\hyperref[thm:rootcanbeadjoined]{\upsym}\end{flushright}
\begin{proof}
    Let $g(x)$ be an irreducible factor of $f(x).$

    Put $\mathbb{K} = \mathbb{F}[x]/\langle g(x)\rangle.$ Since $g(x)$ is irreducible and non-zero, the quotient is indeed a field. Clearly, $\mathbb{F}$ is a subfield under the identification $a \mapsto \bar{a}.$ Moreover, $\bar{x}$ is a root of $g(x).$
\end{proof}

\splitfieldexists*\label{thm:splitfieldexists2}
\begin{flushright}\hyperref[thm:splitfieldexists]{\upsym}\end{flushright}
\begin{proof}
    Let $n \vcentcolon= \deg(f).$ By \Cref{thm:rootcanbeadjoined2}, there exists a field $\mathbb{F}_1 \supset \mathbb{F}$ such that $f(x)$ has a root in $\mathbb{F}_1.$ Calling this root $a_1,$ we see that
    \begin{equation*} 
        f(x) = (x - a_1)f_1(x)
    \end{equation*}
    with $\deg(f_1) = n - 1.$ Continuing inductively, we get fields
    \begin{equation*} 
        \mathbb{F}_n \supset \cdots \supset \mathbb{F}_1 \supset \mathbb{F}
    \end{equation*}
    with $a_i \in \mathbb{F}_i,$ such that
    \begin{equation*} 
        f(x) = a(x - a_1) \cdots (x - a_n).
    \end{equation*}
    Then, $\mathbb{K} = \mathbb{F}(a_1, \ldots, a_n) \subset \mathbb{F}_n$ is a splitting field.
\end{proof}

\FTSP*\label{thm:FTSP2}
\begin{flushright}\hyperref[thm:FTSP]{\upsym}\end{flushright}
\begin{proof}
    \textbf{Existence.} We apply induction on $n.$ The case $n = 1$ is clear since every polynomial is symmetric and $\sigma_1 = u_1.$ So, $g = f$ itself works\footnote{Being slightly sloppy since the indeterminates are different. We mean that you must take the same coefficients}.

    Suppose the theorem is true for $n - 1.$ Now, to prove the theorem for $n,$ apply induction on $\deg(f).$ If $f$ is constant, then again $g = f$ works. Suppose $\deg(f) \ge 1.$ Define
    \begin{equation*} 
        f^0 \vcentcolon= f(u_1, \ldots, u_{n - 1}, 0) \in R[u_1, \ldots, u_{n - 1}].
    \end{equation*}
    Then, $f^0$ is a symmetric polynomial in $n - 1$ variables. By induction hypothesis (on variables), there exists $g \in R[x_1, \ldots, x_{n - 1}]$ such that
    \begin{equation*} 
        f^0(u_1, \ldots, u_{n - 1}) = g(\sigma_1^0, \ldots, \sigma_{n - 1}^0).
    \end{equation*}
    Define $f_1 \in R[u_1, \ldots, u_n]$ by
    \begin{equation*} 
        f_1(u_1, \ldots, u_n) = f(u_1, \ldots, u_n) - g(\sigma_1, \ldots, \sigma_{n - 1}).
    \end{equation*}
    Then, $f_1(u_1, \ldots, u_{n - 1}, 0) = 0.$ Thus, $u_n \mid f_1.$ However, note that $f_1$ is symmetric and thus, $\sigma_n \mid f_1.$ Thus, we can write
    \begin{equation*} 
        f_1(u_1, \ldots, u_n) = \sigma_n h(u_1, \ldots, u_n)
    \end{equation*}
    for some $h \in R[u_1, \ldots, u_n].$ Since $\sigma_n$ is not a zero-divisor in $R[u_1, \ldots, u_n],$ we see that $h$ is also symmetric with $\deg(h) < \deg(f).$ Thus, by inductive hypothesis, $h$ is a polynomial in $\sigma_1, \ldots, \sigma_n$ and hence, $f$ is so.

    \textbf{Uniqueness.} It suffices to show that the elementary symmetric polynomials are algebraically independent. That is, to show that the map
    \begin{equation*} 
        \varphi : R[z_1, \ldots, z_n] \to R[u_1, \ldots, u_n]
    \end{equation*}
    defined by 
    \begin{equation*} 
        z_i \mapsto \sigma_i \andd \varphi|_R = \id_R
    \end{equation*}
    is an injection.

    We prove this by induction on $n.$ For $n = 1,$ it is clear since $\sigma_1 = u_1,$ an indeterminate. Assume that $n \ge 1$ and that the result is true for $n - 1.$ If $\varphi$ is not an injection, then we pick a nonzero polynomial $f(z_1, \ldots, z_n) \in \ker(\varphi)$ of least degree. Write $f$ as a polynomial in $z_n$ as
    \begin{equation*} 
        f(z_1, \ldots, z_n) = f_0(z_1, \ldots, z_{n - 1}) + \cdots + f_d(z_1, \ldots, z_{n - 1})z_n^d
    \end{equation*}
    with $f_d \neq 0.$ Minimality of $d$ (and the fact that $\sigma_n$ is not a zero-divisor) forces that $f_0 \neq 0.$ Since $f \in \ker(\varphi),$ we have
    \begin{equation*} 
        f_0(\sigma_1, \ldots, \sigma_{n - 1}) + \cdots + f_d(\sigma_1, \ldots, \sigma_{n - 1})\sigma_n^d = 0.
    \end{equation*}
    The above is an equality in $R[u_1, \ldots, u_n].$ Put $u_n = 0$ to get
    \begin{equation*} 
        f_0(\sigma_1^0, \ldots, \sigma_{n - 1}^0) = 0.
    \end{equation*}
    But the above shows that the corresponding $\varphi$ for $n - 1$ variables is not injective. A contradiction.
\end{proof}

\powersumformulae*\label{thm:powersumformulae2}
\begin{flushright}\hyperref[thm:powersumformulae]{\upsym}\end{flushright}
\begin{proof}
    Let $z$ be an indeterminate over $S \vcentcolon= R[u_1, \ldots, u_n].$ Note that 
    \begin{equation} \label{eq:001}
        (1 - u_1z) \cdots (1 - u_nz) = 1 - \sigma_1z + \cdots + (-1)^n \sigma_n z^n =\vcentcolon \sigma(z).
    \end{equation}
    Define $w(z) \in S[\![z]\!]$ as
    \begin{align*} 
        w(z) &= \sum_{k = 1}^{\infty} w_kz^k\\
        &= \sum_{k = 1}^{\infty}\left(\sum_{i = 1}^{n}u_i^k\right)z^k\\
        &= \sum_{i = 1}^{n} \left(\sum_{k = 1}^{\infty}(u_iz)^k\right)\\
        &= \sum_{i = 1}^{n} \frac{u_iz}{1 - u_iz}.
    \end{align*}
    Now, since $\sigma(z) = (1 - u_1z) \cdots (1 - u_nz),$ we get
    \begin{equation*} 
        \sigma'(z) = - \sum_{i = 1}^{n} \frac{u_i \sigma(z)}{1 - u_i z},
    \end{equation*}
    where we have taken the formal derivative in $S[\![z]\!].$ Rearranging the above gives
    \begin{equation*} 
        -\frac{z\sigma'(z)}{\sigma(z)} = \sum_{i = 1}^{n}\frac{u_i z}{1 - u_i z} = w(z)
    \end{equation*}
    and hence,
    \begin{equation*} 
        w(z)\sigma(z) = -z\sigma'(z).
    \end{equation*}
    Computing $\sigma'(z)$ from \Cref{eq:001} gives
    \begin{equation*} 
        w(z)\sigma(z) = \sigma_1z - 2\sigma_2z^2 + \cdots + (-1)^{n + 1}n\sigma_nz^n.
    \end{equation*}
    Comparing the coefficients of $x^k$ on both sides gives the result.
\end{proof}

\independencediscriminant*\label{prop:independencediscriminant2}
\begin{flushright}\hyperref[prop:independencediscriminant]{\upsym}\end{flushright}
\begin{proof}
    Let $r_1, \ldots, r_n \in \mathbb{K}$ be such that $f(x) = (x - r_1) \cdot (x - r_n).$

    Consider the Vandermonde matrix
    \begin{equation*} 
        M = \begin{bmatrix}
            1 & 1 & \cdots & 1\\
            r_1 & r_2 & \cdots & r_n\\
            r_1^2 & r_2^2 & \cdots & r_n^2\\
            \vdots & \vdots & \ddots & \vdots \\
            r_1^{n - 1} & r_2^{n - 1} & \cdots & r_n^{n - 1}\\
        \end{bmatrix}.
    \end{equation*}
    Then, $\disc_{\mathbb{K}}(f(x)) = (\det(M))^2 = \det(MM^{\mathsf{T}}).$ As before, let $\sigma_1, \ldots, \sigma_n \in R[u_1, \ldots, u_n]$ be the elementary symmetric polynomials. Put
    \begin{equation*} 
        s_i \vcentcolon= \sigma_i(r_1, \ldots, r_n).
    \end{equation*}
    Then, note that
    \begin{equation*} 
        f(x) = x^n - s_1x^{n - 1} + \cdots + (-1)^ns_n
    \end{equation*}
    and hence, $s_i \in \mathbb{F}$ for all $i = 1, \ldots, n.$ Also, define
    \begin{equation*} 
        v_k \vcentcolon= r_1^k + \cdots + r_n^k
    \end{equation*}
    for all $k \ge 1.$ In view of \nameref{thm:powersumformulae}, we see that each $v_k \in \mathbb{F}$ as well. Moreover, note that
    \begin{equation*} 
        MM^{\mathsf{T}} = \begin{bmatrix}
            n & v_1 & \cdots & v_{n - 1}\\
            v_1 & v_2 & \cdots & v_n\\
            v_2 & v_3 & \cdots & v_{n + 1}\\
            \vdots & \vdots & \ddots & \vdots \\
            v_{n - 1} & v_n & \cdots & v_{2n - 2}\\
        \end{bmatrix}.
    \end{equation*} 
    Thus, $\disc_{\mathbb{K}}(f(x)) = \det(MM^{\mathsf{T}}) \in \mathbb{F}.$

    Note that since $v_k$ can be calculated directly in terms of $s_i,$ which are coefficients of $\mathbb{F}.$ Thus, the discriminant does not depend on the choice of the splitting field.
\end{proof}

\discderivative*\label{prop:discderivative2}
\begin{flushright}\hyperref[prop:discderivative]{\upsym}\end{flushright}
\begin{proof}
    Note that
    \begin{equation*}
        f'(x) = \sum_{i = 1}^{n}\frac{f(x)}{x - r_i} = \sum_{i = 1}^{n}\prod_{\substack{j = 1 \\ j \neq i}}^{n}(x- r_j)
    \end{equation*}
    and thus,
    \begin{equation*} 
        f'(r_i) = \prod_{\substack{j = 1 \\ j \neq i}}^{n}(r_i - r_j).
    \end{equation*}
    The result now follows.
\end{proof}

\FTAprelim*\label{lem:FTAprelim2}
\begin{flushright}\hyperref[lem:FTAprelim]{\upsym}\end{flushright}
\begin{proof} 
    The first follows from intermediate value property. For the second, given $a + b \iota \in \mathbb{C}$ with $a, b \in \mathbb{R},$ define $c, d \in \mathbb{R}$ by
    \begin{equation*} 
        c \vcentcolon= \sqrt{\frac{1}{2}[a + \sqrt{a^2 + b^2}]} \andd d \vcentcolon= \sqrt{\frac{1}{2}[-a + \sqrt{a^2 + b^2}]}.
    \end{equation*}
    Then, $(c + d \iota)^2 = z.$
\end{proof}
\FTA*\label{thm:FTA2}
\begin{flushright}\hyperref[thm:FTA]{\upsym}\end{flushright}
\begin{proof}
    Let $g(x) \in \mathbb{C}[x]$ be a non-constant polynomial. Then, $f(x) = g(x)\bar{g}(x)$ is a non-constant polynomial with real coefficients. Here, $\bar{g}(x)$ denotes the polynomial whose coefficients are complex conjugates of those of $g(x).$ Note that if $f(z) = 0$ for some $z \in \mathbb{C},$ then $g(z) = 0$ or $\bar{g}(z) = 0.$ If $\bar{g}(z) = 0,$ then $g(\bar{z}) = 0.$ In either case, $g$ has a complex root.

    Thus, it suffices to show that all non-constant real polynomials have a root in $\mathbb{C}.$ Given any $f(x) \in \mathbb{R}[x],$ we can write $\deg(f) = 2^nq$ for unique $n \ge 0$ and odd $q \in \mathbb{N}.$

    We prove the statement by induction on $n.$ If $n = 0,$ then $f$ has odd degree and hence, has a real root. \\
    Suppose $n \ge 1$ and the statement is true for $n - 1.$ Let $d \vcentcolon= \deg(f)$ and $\mathbb{K} = \mathbb{C}(\alpha_1, \ldots, \alpha_d)$ be a splitting field of $f(x)$ over $\mathbb{C},$ where the $\alpha_i$ are the roots of $f(x).$ For $r \in \mathbb{R},$ define
    \begin{equation*} 
        y_{ij}(r) = \alpha_i + \alpha_j + r\alpha_i\alpha_j
    \end{equation*}
    for $1 \le i \le j \le d.$ There are $\binom{d + 1}{2}$ such pairs $(i, j).$ Hence, the polynomial
    \begin{equation*} 
        h_r(x) \vcentcolon= \prod_{1 \le i \le j \le d} (x - y_{ij}(r))
    \end{equation*}
    has degree
    \begin{equation*} 
        \deg(h(x)) = \binom{d + 1}{2} = \frac{d}{2}(d + 1) = 2^{n - 1}\underbrace{q(d + 1)}_{\text{odd}}.
    \end{equation*}
    Note that the coefficients of $h_r(x)$ are elementary symmetric polynomials in $y_{ij}$s. Thus, they are symmetric polynomials in $\alpha_i, \ldots, \alpha_d.$ Hence, they are polynomials in the coefficients of $f(x).$ Thus, $h(x) \in \mathbb{R}[x].$ By inductive hypothesis (on $n$), we see that $h_r(x)$ has a root $z_r \in \mathbb{C} \subset \mathbb{K}.$ Thus, $z_r = y_{i(r)j(r)}(r)$ for some pair $(i(r), j(r))$ with $1 \le i(r) \le j(r) \le d.$

    Let $P = \{(i, j) : 1 \le i \le j \le d\}$ and define $\varphi : \mathbb{R} \to P$ by $r \mapsto (i(r), j(r)).$ Since $P$ is finite and $\mathbb{R}$ is not, $\varphi$ is not one-one and thus, there exist $c \neq d \in \mathbb{R}$ with
    \begin{equation*} 
        (i(c), j(c)) = (i(d), j(d)) =\vcentcolon (a, b) \in P.
    \end{equation*}
    Thus,
    \begin{equation*} 
        z_c = \alpha_a + \alpha_b + c\alpha_a\alpha_b = z_d = \alpha_a + \alpha_b + d\alpha_a\alpha_b.
    \end{equation*}
    Note that a priori, we only know that $\alpha_a, \alpha_b \in \mathbb{K}.$ But note that
    \begin{equation*} 
        \alpha_a\alpha_b = \frac{z_c - z_d}{d - c} \in \mathbb{C}
    \end{equation*}
    and consequently,
    \begin{equation*} 
        \alpha_a + \alpha_b = z_c - c\alpha_a\alpha_b \in \mathbb{C}.
    \end{equation*}
    Thus, $\alpha_a\alpha_b$ and $\alpha_a + \alpha_b \in \mathbb{C}.$ However, these are roots of the quadratic
    \begin{equation*} 
        x^2 - (\alpha_a + \alpha_b)x + \alpha_a\alpha_b \in \mathbb{C}[x].
    \end{equation*}
    Thus, $\alpha_a \in \mathbb{C}.$ But $\alpha_a$ was a root of $f(x),$ as desired.
\end{proof}


\alglcosureinalgclosedisclosed*\label{prop:alglcosureinalgclosedisclosed2}
\begin{flushright}\hyperref[prop:alglcosureinalgclosedisclosed]{\upsym}\end{flushright}
\begin{proof}
    By \Cref{cor:algclosureisfield}, we already know that $\mathbb{A}/\mathbb{F}$ is actually an algebraic extension. We just need to show that $\mathbb{A}$ is algebraically closed. To this end, let $f(x) \in \mathbb{A}[x]$ be non-constant. Then, $f(x)$ has a root $\alpha \in \mathbb{K}.$ But then, $\alpha$ is algebraic over $\mathbb{A}$ and hence, over $\mathbb{F}.$ (\Cref{cor:compalgisalg}.) Thus, $\alpha \in \mathbb{A}.$
\end{proof}

\unionoffields*\label{lem:unionoffields2}
\begin{flushright}\hyperref[lem:unionoffields]{\upsym}\end{flushright}
\begin{proof}
    The operations are clearly well-defined. It is easy to see that the desired commutative and associative laws hold since they hold in each $\mathbb{F}_i.$ The $0$ and $1$ are those of each $\mathbb{F}_i.$ The appropriate inverses of any $a \in \mathbb{F}$ also exist in any $\mathbb{F}_i$ containing $a.$ The last sentence is also easy to check.
\end{proof}
\algclosedext*\label{thm:algclosedext2}
\begin{flushright}\hyperref[thm:algclosedext]{\upsym}\end{flushright}
\begin{proof}
    We first show that given any field $\mathbb{F},$ we can create a field $\mathbb{F}_1 \supset \mathbb{F}$ containing roots of any non-constant polynomial in $\mathbb{F}[x].$ Let $S$ be a set of indeterminates which are in one-to-one correspondence with set of all polynomials in $\mathbb{F}[x]$ with degree $\ge 1.$ Let $x_f \in S$ denote the indeterminate corresponding to $f.$

    Consider the (very large) polynomial ring $\mathbb{F}[S].$ Let 
    \begin{equation*} 
        I = \langle f(x_f)  : f \in \mathbb{F}[x],\;\deg(f) \ge 1\rangle
    \end{equation*}
    be the ideal generated by the polynomials $f(x_f) \in \mathbb{F}[S].$ We contend that $1 \notin I.$ Suppose the contrary. Then,
    \begin{equation*} 
        1 = g_1 f_1(x_{f_1}) + \cdots + g_n f_n(x_{f_n})
    \end{equation*}
    for some $g_1, \ldots, g_n \in \mathbb{F}[S].$ Note that these polynomials $g_j$ only involve finitely many variables. Let $x_i \vcentcolon= x_{f_i}$ for $i = 1, \ldots, n$ and let $x_{n + 1}, \ldots, x_m$ be the remaining variables in $g_1, \ldots, g_n.$ Then, we have
    \begin{equation*} 
        \sum_{i = 1}^{n} g_i(x_1, \ldots, x_n, x_{n + 1}, \ldots, x_m)f_i(x_i) = 1.
    \end{equation*}
    Now, let $\mathbb{E} \supset \mathbb{F}$ be an extension containing roots $\alpha_i$ of $f_i.$ (Note that $\deg(f_i) \ge 1$ and thus, we may use \Cref{thm:rootcanbeadjoined}.) Then, putting $x_i = \alpha_i$ for $i = 1, \ldots, n$ and putting $x_{n + 1} = \cdots = x_m = 0$ in the above equation gives a contradiction.

    Thus, $1 \notin I$ and hence, $I$ is a proper ideal of $\mathbb{F}[S].$ Thus, it is contained in some maximal ideal $\mathfrak{m} \subset \mathbb{F}[S].$ Put $\mathbb{F}_1 \vcentcolon= \mathbb{F}[S]/\mathfrak{m}.$ Then, $\mathbb{F}_1$ is a field extension of $\mathbb{F}.$ \\
    Note that $\overline{x_f} = x_f + \mathfrak{m} \in \mathbb{F}_1$ is a root of $f(x) \in \mathbb{F}[x].$ Thus, we have constructed a field $\mathbb{F}_1$ in which every non-constant polynomial of $\mathbb{F}[x]$ has a root.

    Repeating the procedure, we get fields 
    \begin{equation*} 
        \mathbb{F} = \mathbb{F}_0 \subset \mathbb{F}_1 \subset \mathbb{F}_2 \subset \mathbb{F}_3 \subset \cdots
    \end{equation*} 
    such that every non-constant polynomial in $\mathbb{F}_i$ has a root in $\mathbb{F}_{i + 1}.$

    Now, put $\mathbb{K} = \bigcup_{i \ge 0}\mathbb{F}_i.$ This is a field as per \Cref{lem:unionoffields}, having each $\mathbb{F}_i$ as a subfield. 

    Now, if $f(x) \in \mathbb{K}[x],$ then $f(x) \in \mathbb{F}_n[x]$ for some $n.$ This has a root in $\mathbb{F}_{n + 1} \subset \mathbb{K},$ as desired.
\end{proof}

\algclosure*\label{cor:algclosure2}
\begin{flushright}\hyperref[cor:algclosure]{\upsym}\end{flushright}
\begin{proof}
    Let $\mathbb{L} \supset \mathbb{F}$ be algebraically closed. (Existence given by \Cref{thm:algclosedext}.) Define
    \begin{equation*} 
        \mathbb{K} \vcentcolon= \{\alpha \in \mathbb{L} : \alpha \text{ is algebraic over }\mathbb{K}\}.
    \end{equation*}
    By \Cref{prop:alglcosureinalgclosedisclosed}, $\mathbb{K}$ is an algebraic closure of $\mathbb{F}.$
\end{proof}

\rootsandextensions*\label{prop:rootsandextensions2}
\begin{flushright}\hyperref[prop:rootsandextensions]{\upsym}\end{flushright}
\begin{proof}
    First, we note that the map is indeed well-defined. Let $\tau$ be an embedding extending $\sigma.$ Then,
    \begin{equation*} 
        \tau(p(\alpha)) = p^{\sigma}(\tau(\alpha)) = 0
    \end{equation*}
    and thus, $\tau(\alpha)$ is indeed a root of $p^{\sigma}.$ 

    Now, let $\beta \in L$ be such that $p^{\sigma}(\beta) = 0.$ Define $\tau_{\beta} : \mathbb{F}(\alpha) \to \mathbb{L}$ by $\tau_{\beta}(f(\alpha)) = f^{\sigma}(\beta)$ for $f(x) \in \mathbb{F}[x].$\footnote{Note that elements of $\mathbb{F}(\alpha)$ are precisely polynomials in $\alpha.$} We now show that $\tau_{\beta}$ is well-defined. 

    Suppose $f(\alpha) = g(\alpha).$ Then, $p(x) \mid f(x) - g(x)$ and hence, $p^{\sigma}(x) \mid f^{\sigma}(x) - g^{\sigma}(x).$ Thus, $f^{\sigma}(\beta) = g^{\sigma}(\beta).$ Thus, $\tau_{\beta}$ is well-defined. It is clearly a homomorphism (and hence, an embedding). Moreover, it extends $\sigma.$

    It is now easily seen that $\beta \mapsto \tau_{\beta}$ is a two-sided inverse of the map $\tau \mapsto \tau(\alpha).$
\end{proof}

\extendtoalgextension*\label{thm:extendtoalgextension2}
\begin{flushright}\hyperref[thm:extendtoalgextension]{\upsym}\end{flushright}
\begin{proof}
    Consider the set
    \begin{equation*} 
        \Sigma \vcentcolon= \{(\mathbb{E}, \tau) \mid \mathbb{F} \subset \mathbb{E} \subset \mathbb{K} \text{ are fields and } \tau : \mathbb{E} \to \mathbb{L} \text{ such that }\tau|_{\mathbb{F}} = \sigma\}.
    \end{equation*}
    Note that $\Sigma \neq \emptyset$ since $(\mathbb{F}, \sigma) \in \Sigma.$ Define the relation $\le$ on $\Sigma$ by
    \begin{equation*} 
        (\mathbb{E}, \tau) \le (\mathbb{E}', \tau') \iff \mathbb{E} \subset \mathbb{E}' \text{ and } \tau'|_{\mathbb{E}} = \tau.
    \end{equation*}
    Then, $(\Sigma, \le)$ is a partially ordered set. Moreover, if $\Lambda = \{(\mathbb{E}_\alpha, \tau_\alpha)\}_{\alpha \in I}$ is a chain in $\Sigma,$ then $\mathbb{E} \vcentcolon= \bigcup_{\alpha \in I}\mathbb{F}_\alpha$ is a subfield of $\mathbb{E}$ and $\tau : \mathbb{E} \to \mathbb{L}$ defined as $\tau(x) \vcentcolon= \tau_\alpha(x)$ for $x \in \mathbb{F}_\alpha$ is well-defined. (The proof is similar to that of \Cref{lem:unionoffields}.) Moreover, $(\mathbb{E}, \tau)$ is an upper bound of $\Lambda.$   

    Thus, by Zorn's lemma, there exists a maximal element $(\mathbb{E}, \tau) \in \Sigma.$ We contend that $\mathbb{E} = \mathbb{K}.$ If not, then pick $\alpha \in \mathbb{K} \setminus \mathbb{E}.$ By \Cref{prop:rootsandextensions}, we can extend $\tau$ to an embedding $\tau' : \mathbb{E}(\alpha) \to \mathbb{L}.$ But this contradicts maximality of $(\mathbb{E}, \tau).$

    Now, suppose that $\mathbb{K}$ is an algebraic closure of $\mathbb{F}$ and $\mathbb{L}$ of $\sigma(\mathbb{F}).$ We have
    \begin{equation*} 
        \sigma(\mathbb{F}) \subset \tau(\mathbb{K}) \subset \mathbb{L}
    \end{equation*}
    and thus, $L/\tau(\mathbb{K})$ is also algebraic. But $\tau(\mathbb{K})$ is also algebraically closed and thus, $\mathbb{L} = \tau(\mathbb{K}).$
\end{proof}

\isosplitting*\label{thm:isosplitting2}
\begin{flushright}\hyperref[thm:isosplitting]{\upsym}\end{flushright}
\begin{proof}
    Let $\overline{\mathbb{E}}$ be an algebraic closure of $\mathbb{E}.$ Then, it is also one of $\mathbb{F}.$ Thus, there exists an embedding $\tau : \mathbb{E}' \to \overline{\mathbb{E}}$ extending the inclusion $i : \mathbb{F} \hookrightarrow \overline{\mathbb{E}}.$

    Let $f(x) = a(x - \alpha_1) \cdots (x - \alpha_n)$ be a factorisation of $f(x)$ in $\mathbb{E}'[x].$ Then,
    \begin{equation*} 
        f^{\tau}(x) = (x - \tau(\alpha_1)) \cdots (x - \tau(\alpha_n)) \in \overline{\mathbb{E}}[x].
    \end{equation*}
    Note that we have $\mathbb{E}' = \mathbb{F}(\alpha_1, \ldots, \alpha_n)$ and so, $\tau(\mathbb{E}') = \mathbb{F}(\tau(\alpha_1), \ldots, \tau(\alpha_n)).$ Thus, $\tau(\mathbb{E}')$ is a splitting field of $f^{\tau}.$ But $f^{\tau} = f$ since $f(x) \in \mathbb{F}[x]$ and $\tau$ extends the inclusion map. Thus, $\tau(\mathbb{E}') = \mathbb{E},$ since any algebraic closure contains a unique splitting field.
\end{proof}


\multindepsplitting*\label{prop:multindepsplitting2}
\begin{flushright}\hyperref[prop:multindepsplitting]{\upsym}\end{flushright}
\begin{proof}
    Let $\mathbb{E}$ and $\mathbb{K}$ be splitting fields for $f(x)$ over $\mathbb{F}.$ By \Cref{thm:isosplitting}, there exists an $\mathbb{F}$-isomorphism $\tau : \mathbb{E} \to \mathbb{K}.$ In turn, we get an isomorphism
    \begin{align*} 
        \varphi_\tau : \mathbb{E}[x] &\to \mathbb{K}[x]\\
        \sum a_i x^i &\mapsto \sum \tau(a_i) x^i.
    \end{align*}
    Now, let $f(x) = \prod_{i = 1}^{g}(x - r_i)^{e_i}$ be the unique factorisation of $f(x)$ in $\mathbb{E}[x].$ The above isomorphism shows that 
    \begin{equation*} 
        f(x)= \prod_{i = 1}^{g}(x - \tau(r_i))^{e_i}
    \end{equation*}
    is the unique factorisation of $f(x)$ in $\mathbb{K}[x].$ The result follows.
\end{proof}

\derivcritreproot*\label{prop:derivcritreproot2}
\begin{flushright}\hyperref[prop:derivcritreproot]{\upsym}\end{flushright}
\begin{proof}
    \forward If $r$ is a repeated root, then write $f(x) = (x - r)^2g(x)$ for $g \in \mathbb{E}[x].$ Then, taking the derivative gives
    \begin{equation*} 
        f'(x) = 2(x - r)g(x) + (x - r)^2g'(x).
    \end{equation*}
    Thus, $f'(r) = 0.$

    \backward Write $f(x) = (x - r)g(x).$ Then,
    \begin{equation*} 
        0 = f'(r) = (r - r)g'(r) + g(r) = g(r).
    \end{equation*}
    Thus, $(x - r) \mid g(x)$ and hence, $(x - r)^2 \mid f(x).$
\end{proof}

\derivcritsep*\label{thm:derivcritsep2}
\begin{flushright}\hyperref[thm:derivcritsep]{\upsym}\end{flushright}
\begin{proof}
    Let $\mathbb{E}$ be a splitting field of $f(x).$
    \begin{enumerate}
        \item Let $r \in \mathbb{E}$ be a root of $f(x).$ Then, $f'(r) = 0,$ by hypothesis and thus, $r$ is a repeated root, by \Cref{prop:derivcritreproot}.
        %
        \item Suppose $f'(x) \neq 0.$\\
        \forward Suppose $f(x)$ has simple roots. We need to show that $f(x)$ and $f'(x)$ have no common root. Let $r$ be a root of $f(x).$ Then $f'(r) \neq 0,$ by \Cref{prop:derivcritreproot}.

        \backward Suppose $\gcd(f(x), f'(x)) = 1$ and $r \in \mathbb{E}$ is an arbitrary root of $f(x).$ Then, $f'(r) \neq 0.$ Thus, $r$ is a simple root. \qedhere
    \end{enumerate}
\end{proof}

\irredsepderiv*\label{prop:irredsepderiv2}
\begin{flushright}\hyperref[prop:irredsepderiv]{\upsym}\end{flushright}
\begin{proof}
    Let $\mathbb{E}$ be a splitting field of $f(x)$ over $\mathbb{F}.$
    \begin{enumerate}
        \item \forward $f(x)$ has no repeated roots and thus, $f'(x) \neq 0,$ by \Cref{prop:irredsepderiv}.

        \backward Suppose $f'(x) \neq 0$ and $f(x)$ has a repeated root $r \in \mathbb{E}.$ Then, by \Cref{prop:derivcritreproot}, $f'(r) = 0.$ Thus, $g(x) \vcentcolon= \gcd(f(x), f'(x)) \neq 1.$ Irreducibility of $g(x)$ forces $f(x) = g(x).$ But then, $f(x) \mid f'(x),$ which is a contradiction since $\deg(f'(x)) < \deg(f(x)).$
        %
        \item If $f(x)$ is non-constant, then $f'(x) \neq 0.$ The previous part applies.
    \end{enumerate} 
\end{proof}

\xppolyirredorroot*\label{prop:xppolyirredorroot2}
\begin{flushright}\hyperref[prop:xppolyirredorroot]{\upsym}\end{flushright}
\begin{proof}
    Suppose $f(x)$ is not irreducible. Write $f(x) = g(x)h(x)$ with $1 \le \deg(g(x)) =\vcentcolon m < p.$ Let $b \in \mathbb{E}$ be a root in a splitting field $\mathbb{E}$ of $f(x)$ over $\mathbb{F}.$ Then, $b^p = a.$ Thus, $f(x)$ factorises in $\mathbb{E}[x]$ as
    \begin{equation*} 
        f(x) = x^p - b^p = (x - b)^p.
    \end{equation*}
    Since $\mathbb{E}[x]$ is a UFD, we see that $g(x) = (x - b)^m.$ (We may assume that $g(x)$ is monic.) However, note that the coefficient of $x^{m - 1}$ is $mb.$ By assumption, $mb \in \mathbb{F}.$ Since $1 \le m < p,$ we see that $b \in \mathbb{F}.$ Thus, $a = b^p \in \mathbb{F}^p.$     
\end{proof}

\nonseppowerp*\label{prop:nonseppowerp2}
\begin{flushright}\hyperref[prop:nonseppowerp]{\upsym}\end{flushright}
\begin{proof}
    Since $f(x)$ is irreducible and not separable, we must have $f'(x) = 0.$ Write
    \begin{equation*} 
        f(x) = a_0 + a_1x + \cdots + a_nx^n
    \end{equation*}
    and note that
    \begin{equation*} 
        0 = f'(x) = a_1 + 2a_2x + \cdots + n a_n x^{n - 1}.
    \end{equation*}
    Thus, $ka_k = 0$ for all $k = 1, \ldots, n.$ If $\gcd(k, p) = 1,$ then we may cancel $k$ to see that $a_k = 0$ whenever $p \nmid k.$ Thus, $f(x)$ is of the form
    \begin{equation*} 
        f(x) = a_0 + a_px^p + \cdots + a_{mp} x^{mp}
    \end{equation*}
    for some $m \in \mathbb{N}.$ Thus, $g(x) = a_0 + a_p x + \cdots + a_{mp} x^m$ works.
\end{proof}

\perfectiffppower*\label{thm:perfectiffppower2}
\begin{flushright}\hyperref[thm:perfectiffppower]{\upsym}\end{flushright}
\begin{proof}
    \forward Suppose $\mathbb{F} \neq \mathbb{F}^p.$ Pick $\alpha \in \mathbb{F} \setminus \mathbb{F}^p.$ Then, $x^p - \alpha$ is irreducible (by \Cref{prop:xppolyirredorroot}) but not separable, by \Cref{prop:irredsepderiv}.

    \backward Suppose $\mathbb{F} = \mathbb{F}^p$ and $f(x) \in \mathbb{F}[x]$ is irreducible and not separable. By \Cref{prop:nonseppowerp}, we can write 
    \begin{equation*} 
        f(x) = \sum_{i = 0}^{m} a_i x^{ip}.
    \end{equation*} 
    Let $b_i \in \mathbb{F}$ be such that $a_i = b_i^p.$ Then,
    \begin{equation*} 
        f(x) = \sum_{i = 0}^{m} a_i x^{ip} = \sum_{i = 0}^{m} b_i^p x^{ip} = \left(\underbrace{\sum_{i = 0}^{m}b_i x^i}_{\in \mathbb{F}[x]}\right)^p,
    \end{equation*}
    contradicting the irreducibility of $f(x)$ in $\mathbb{F}[x].$
\end{proof}

\finitefieldperfect*\label{cor:finitefieldperfect2}
\begin{flushright}\hyperref[cor:finitefieldperfect]{\upsym}\end{flushright}
\begin{proof}
    Let $\mathbb{F}$ be a finite field of characteristic $p > 0.$ We show that $\mathbb{F} = \mathbb{F}^p.$ 

    Note that $\md{\mathbb{F}} = p^n$ for some $n \in \mathbb{N}.$ Thus, by Lagrange's theorem from group theory, we see that $\alpha^{p^n - 1} = 1$ for all $\alpha \in \mathbb{F}^\times.$ Thus, $\alpha^{p^n} = \alpha$ for all $\alpha \in \mathbb{F}.$ (This holds for $\alpha = 0$ as well.)

    Thus, given any arbitrary $\alpha \in \mathbb{F},$ put $\beta = \alpha^{p^{n - 1}}$ to get $\alpha = \beta^p \in \mathbb{F}^p.$
\end{proof}

\samemultirredpoly*\label{prop:samemultirredpoly2}
\begin{flushright}\hyperref[prop:samemultirredpoly]{\upsym}\end{flushright}
\begin{proof}
    Let $\overline{\mathbb{F}} \supset \mathbb{F}$ be an algebraic closure of $\mathbb{F}.$ Let $\alpha, \beta \in \overline{\mathbb{F}}$ be roots of $f.$ We have an $\mathbb{F}$-isomorphism $\sigma : \mathbb{F}(\alpha) \to \mathbb{F}(\beta)$ determined by $\alpha \mapsto \beta.$ 

    Thus, $\sigma$ can be extended to an automorphism $\tau$ of $\overline{\mathbb{F}}.$ Then, write $f(x) = (x - \alpha)^mh(x)$ where $m$ is the multiplicity of $\alpha$ and $h(x) \in \overline{\mathbb{F}}[x].$ Applying $\tau,$ we get
    \begin{equation*} 
        f(x) = f^\tau(x) = (x - \beta)^m h^\tau(x).
    \end{equation*}
    Thus, the multiplicity of $\beta$ is at least $m.$ By symmetry, we have equality.

    If $\chr(\mathbb{F}) = 0,$ then $f(x)$ is separable (\Cref{thm:derivcritsep}) and thus, all roots are simple.

    Now, assume that $\chr(\mathbb{F}) =\vcentcolon p > 0.$ Let $n \in \mathbb{N}_0$ be the largest such that there exists a polynomial $g(x) \in \mathbb{F}[x]$ with $f(x) = g(x^{p^n}).$ (Note that we can take $g = f$ and $n = 0$ if no positive $n$ exists.)

    Then, $g$ is irreducible since $f$ is so. Moreover, $g$ must be separable. Indeed, if not, then we can write $g(x) = h(x^p)$ for some $h(x) \in \mathbb{F}[x],$ by \Cref{prop:irredsepderiv}. Then, $f(x) = g(x^{p^{n + 1}})$ contradicting maximality of $n.$

    Thus, $g(x)$ factors in $\overline{\mathbb{F}}$ as $g(x) = (x - r_1) \cdots (x - r_g)$ for distinct $r_g.$ Since $\overline{\mathbb{F}}$ is algebraically closed, we can find $s_1, \ldots, s_g$ necessarily distinct such that $s_i^{p^n} = r_i.$ Then, we have
    \begin{equation*} 
        f(x) = g(x^{p^n}) = (x - s_1)^{p^n} \cdots (x - s_g)^{p^n},
    \end{equation*}
    as desired.
\end{proof}

\separabledegreedef*\label{thm:separabledegreedef2}
\begin{flushright}\hyperref[thm:separabledegreedef]{\upsym}\end{flushright}
\begin{proof}
    If $\widetilde{\sigma} \in S_\sigma,$ then for any $x \in \mathbb{F},$ we have
    \begin{equation*} 
        (\lambda \circ \widetilde{\sigma})(x) = \lambda(\sigma(x)) = (\tau \circ \sigma^{-1})(\sigma(x)) = \tau(x).
    \end{equation*}
    Thus, $\psi$ actually maps into $S_\tau.$ Since $\lambda$ is an isomorphism, $\psi$ is easily seen to be a bijection. Explicitly, the inverse of $\psi$ can be seen to be $\widetilde{\tau} \mapsto \lambda^{-1} \circ \tau.$
\end{proof}

\towerlawsep*\label{thm:towerlawsep2}
\begin{flushright}\hyperref[thm:towerlawsep]{\upsym}\end{flushright}
\begin{proof}
    First, we show that the separable degree is multiplicative. Let $n \vcentcolon= [\mathbb{K} : \mathbb{E}]_s$ and $m \vcentcolon= [\mathbb{E} : \mathbb{F}]_s$ and $\sigma : \mathbb{F} \to \mathbb{L}$ be an embedding into an algebraically closed field $\mathbb{L}.$ 

    Let $\sigma_1, \ldots, \sigma_m : \mathbb{E} \to \mathbb{L}$ be extensions of $\mathbb{F}.$ Then, each $\sigma_i$ has extensions $\sigma_i^{(1)}, \ldots, \sigma_i^{(n)} : \mathbb{K} \to \mathbb{L}.$ Note that $\{\sigma_i^{(j)} : 1 \le i \le m,\; 1 \le j \le n\}$ has cardinality $mn.$ (All the extensions obtained are distinct.)

    Clearly, any embedding $\tau : \mathbb{K} \to \mathbb{L}$ extending $\tau$ is obtained this way. ($\tau|_{\mathbb{E}}$ is $\sigma_i$ for some $i$ and thus, $\tau = \sigma_i^{(j)}$ for some $j.$) 

    Thus, $[\mathbb{K} : \mathbb{F}]_s = mn,$ as desired. 

    Now, since $\mathbb{E}/\mathbb{F}$ is finite, we can construct $\alpha_1, \ldots, \alpha_g$ such that $\mathbb{E} = \mathbb{F}(\alpha_1, \ldots, \alpha_g).$ We have the chain
    \begin{equation*} 
        \mathbb{F} \subset \mathbb{F}(\alpha_1) \subset \mathbb{F}(\alpha_1, \alpha_2) \subset \cdots \subset \mathbb{F}(\alpha_1, \ldots, \alpha_n).
    \end{equation*}
    Note that by \Cref{prop:sepdeglessthannordeg}, we know that 
    \begin{equation*} 
        [\mathbb{F}(\alpha_1, \ldots, \alpha_{i + 1}) : \mathbb{F}(\alpha_1, \ldots, \alpha_i)]_s \le [\mathbb{F}(\alpha_1, \ldots, \alpha_{i + 1}) : \mathbb{F}(\alpha_1, \ldots, \alpha_i)]
    \end{equation*}
    for all $i = 0, \ldots, n - 1.$ Since both degrees are multiplicative, we are done.
\end{proof}

\sepiffdegequal*\label{thm:sepiffdegequal2}
\begin{flushright}\hyperref[thm:sepiffdegequal]{\upsym}\end{flushright}
\begin{proof}
     Write $\mathbb{E} = \mathbb{F}(\alpha_1, \ldots, \alpha_n)$ for $\alpha_i \in \mathbb{E}.$ (Note that $\mathbb{E}/\mathbb{F}$ is a finite extension.)

    Put 
    \begin{equation*} 
        \mathbb{F}_0 \vcentcolon= \mathbb{F} \andd \mathbb{F}_i \vcentcolon= \mathbb{F}(\alpha_1, \ldots, \alpha_i),
    \end{equation*} 
    for $i = 1, \ldots, n.$

    \forward Assume $\mathbb{E}/\mathbb{F}$ is separable. Then, since each $\alpha_i$ is separable over $\mathbb{F},$ it follows that $\alpha_i$ is separable over $\mathbb{F}$ for $i = 1, \ldots, n.$ (Note that $\irr(\alpha, \mathbb{F}_i) \mid \irr(\alpha, \mathbb{F}).$) Thus, we see that 
    \begin{equation*} 
        [\mathbb{F}_{i} : \mathbb{F}_{i - 1}]_s = [\mathbb{F}_{i} : \mathbb{F}_{i - 1}]
    \end{equation*}
    for all $i = 1, \ldots, n.$ Multiplying gives $[\mathbb{E} : \mathbb{F}]_s = [\mathbb{E}:\mathbb{F}].$

    \backward Let $\alpha \in \mathbb{E}$ be arbitrary. Consider the tower
    \begin{equation*} 
        \mathbb{F} \subset \mathbb{F}(\alpha) \subset \mathbb{E}.
    \end{equation*}
    Since, we have the equality $[\mathbb{E} : \mathbb{F}]_s = [\mathbb{E} : \mathbb{F}],$ we also have the equality $[\mathbb{F}(\alpha) : \mathbb{F}]_s = [\mathbb{F}(\alpha) : \mathbb{F}],$ by the previous corollary. Thus, $\alpha$ is separable over $\mathbb{F},$ by \Cref{prop:sepdeglessthannordeg}.
\end{proof}

\compdecompsep*\label{prop:compdecompsep2}
\begin{flushright}\hyperref[prop:compdecompsep]{\upsym}\end{flushright}
\begin{proof}
    For both parts, we first note that if $\alpha \in \mathbb{K}$ is algebraic over $\mathbb{F},$ then it is also algebraic over $\mathbb{E}.$ Moreover, $\irr(\alpha, \mathbb{E}) \mid \irr(\alpha, \mathbb{F}).$ (The divisibility is in $\mathbb{E}[x].$)

    \forward Let $\alpha \in \mathbb{K}$ be arbitrary. Then, $\alpha$ is algebraic over $\mathbb{F}$ and hence, over $\mathbb{E}.$ Since $\irr(\alpha, \mathbb{F})$ has no repeated roots, neither does its factor $\irr(\alpha, \mathbb{E}).$ Thus, $\mathbb{K}/\mathbb{E}$ is separable. \\
    Now, let $\beta \in \mathbb{E}$ be arbitrary. Then, $\beta \in \mathbb{K}$ and thus, $\irr(\alpha, \mathbb{F})$ is separable. Thus, $\mathbb{E}/\mathbb{F}$ is separable.

    \backward Let $\alpha \in \mathbb{K}$ be arbitrary. Note that $\alpha$ is algebraic over $\mathbb{E},$ since it is separable over $\mathbb{E}.$ Let $\irr(\alpha, \mathbb{E}) = a_1 + \cdots + a_{n }x^{n - 1} + x^n \in \mathbb{E}[x].$ 

    Put 
    \begin{equation*} 
        \mathbb{F}_0 \vcentcolon= \mathbb{F} \andd \mathbb{F}_i \vcentcolon= \mathbb{F}(a_1, \ldots, a_i),
    \end{equation*} 
    for $i = 1, \ldots, n.$ By \forward, we see that $a_i$ is separable over $\mathbb{F}_{i - 1}$ and hence, 
    \begin{equation} \label{eq:002} \tag{$*$}
        [\mathbb{F}_i : \mathbb{F}_{i - 1}]_s = [\mathbb{F}_i : \mathbb{F}_{i - 1}]
    \end{equation} 
    for all $i = 1, \ldots, n.$

    Finally, put $\mathbb{F}_{n + 1} \vcentcolon= \mathbb{F}_n(\alpha).$ Then, \Cref{eq:002} holds for $i = n + 1$ as well, since $\alpha$ is separable over $\mathbb{F}_n.$ (Note that $\irr(\alpha, \mathbb{F}_n) = \irr(\alpha, \mathbb{E}),$ by our construction and the latter is separable by assumption.)

    Thus, upon multiplying, we get $[\mathbb{F}_{n + 1} : \mathbb{F}]_s = [\mathbb{F}_{n + 1} : \mathbb{F}]$ and hence, $\mathbb{F}_{n + 1}/\mathbb{F}$ is separable. Since $\alpha \in \mathbb{F}_{n + 1},$ we see that $\alpha$ is separable over $\mathbb{F}$ and hence, $\mathbb{K}/\mathbb{F}$ is separable.
\end{proof}

\sepdegdividesdeg*\label{prop:sepdegdividesdeg2}
\begin{flushright}\hyperref[prop:sepdegdividesdeg]{\upsym}\end{flushright}
\begin{proof}
    Clearly the statement is true if $\chr(\mathbb{F}) = 0$ since we have equality of degrees. Suppose $\chr(\mathbb{F}) =\vcentcolon p > 0.$

    First, suppose that $\mathbb{E} = \mathbb{F}(\alpha)$ for some $\alpha \in \mathbb{E}.$ Let $p(x) \vcentcolon= \irr(\alpha, \mathbb{F})$ and $d \vcentcolon= \deg(p(x)).$ By \Cref{prop:samemultirredpoly}, $p(x)$ factors in $\overline{\mathbb{F}}[x]$ as
    \begin{equation*} 
        p(x) = (x - \alpha)^{p^n} (x - \alpha_2)^{p^n} \cdots (x - \alpha_g)^{p^n},
    \end{equation*}
    where $\alpha_2, \ldots, \alpha_g \in \overline{\mathbb{F}}\setminus\{\alpha\}$ are distinct. Note that we have $gp^n = d.$ By \Cref{prop:rootsandextensions}, we know that $[\mathbb{F}(\alpha) : \mathbb{F}]_s = g.$ Thus, the statement is true.

    For a general finite extension $\mathbb{E}/\mathbb{F},$ write $\mathbb{E} = \mathbb{F}(\beta_1, \ldots, \beta_k)$ and use the fact that degrees are multiplicative.
\end{proof}

\uniquefinfields*\label{thm:uniquefinfields2}
\begin{flushright}\hyperref[thm:uniquefinfields]{\upsym}\end{flushright}
\begin{proof}
    Let $q \vcentcolon= \md{\mathbb{K}}$ and $p \vcentcolon= \chr(\mathbb{K}).$ Then, $q = p^n$ for some $n \in \mathbb{N}.$ Note that $\mathbb{K}^\times$ is a group of order $q - 1.$ By Lagrange's theorem, we have $a^{q - 1} = 1$ for all $a \in \mathbb{K}^\times.$ In turn, we get $a^q - a = 0$ for \emph{all} $a \in \mathbb{K}.$

    Hence, $\mathbb{K}$ is a splitting field of $x^q - x$ over $\mathbb{F}_p$ and so is $\mathbb{L}.$ By \Cref{thm:isosplitting}, $\mathbb{K}$ and $\mathbb{L}$ are isomorphic.
\end{proof}

\existencefinfields*\label{thm:existencefinfields2}
\begin{flushright}\hyperref[thm:existencefinfields]{\upsym}\end{flushright}
\begin{proof}
    Fix $n \in \mathbb{N}$ and let $q = p^n.$ $\overline{\mathbb{F}}_p$ contains a unique splitting field of $x^q - x =\vcentcolon f(x)$ over $\mathbb{F}_p.$ We show that this splitting field has $q$ elements. Consider
    \begin{equation*} 
        \mathbb{K} = \{\alpha \in \overline{\mathbb{F}}_p \mid f(\alpha) = 0\}.
    \end{equation*}
    Then, $\md{\mathbb{K}} = q$ since $f(x)$ is separable, by \Cref{thm:derivcritsep}. 

    Thus, $\mathbb{K}$ is the desired splitting field. Conversely any other field with $q$ elements would be the set of roots of $x^q - x$ and hence, we have uniqueness.

    We now show that $\overline{\mathbb{F}}_p = \bigcup_{k \ge 1}\mathbb{F}_{p^k}.$ Let $\alpha \in \overline{\mathbb{F}}_p$ and let $d \vcentcolon= \deg_{\mathbb{F}}(\alpha).$ Then, $[\mathbb{F}(\alpha) : \mathbb{F}] = d$ and hence, $\alpha \in \mathbb{F}(\alpha) = \mathbb{F}_{p^d}.$
\end{proof}
\end{document}