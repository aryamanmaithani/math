\chapter{Galois Extensions}
\section{Definitions}

\begin{defn}%[]
    A field extension $\mathbb{E}/\mathbb{F}$ is called a \deff{Galois extension} if it is normal and separable. The \deff{Galois group of a Galois extension} $\mathbb{E}/\mathbb{F}$ is the group of all $\mathbb{F}$-automorphisms of $\mathbb{E}$ under the operation of composition of maps. It is denoted $\Gal(\mathbb{E}/\mathbb{F}).$

    If $f(x) \in \mathbb{F}[x]$ is a separable polynomial and $\mathbb{E}$ is a splitting field of $f(x)$ over $\mathbb{F},$ then $\mathbb{E}/\mathbb{F}$ is a Galois extension and the \deff{Galois group of $f(x)$ over $\mathbb{F}$} is defined to be $\Gal(\mathbb{E}/\mathbb{F})$ and denoted as $\Gal(f(x), \mathbb{F})$ or simply $\G_f$ if $\mathbb{F}$ is clear.
\end{defn}

\begin{rem}
    Note that the definition of the $\Gal(f(x), \mathbb{F})$ does not depend on the splitting field chosen, up to isomorphism. Indeed, let $\mathbb{E}$ and $\mathbb{E}'$ be two splitting fields of $f(x)$ over $\mathbb{F}.$ By \Cref{thm:isosplitting}, there is an $\mathbb{F}$-isomorphism $\tau : \mathbb{E} \to \mathbb{E}'.$ Then, $\sigma \mapsto \tau \circ \sigma \circ \tau^{-1}$ is an isomorphism from $\Gal(\mathbb{E}/\mathbb{F})$ to $\Gal(\mathbb{E}'/\mathbb{F}).$
\end{rem}

\begin{ex}
    Here are some examples and non-examples.
    \begin{enumerate}
        \item Let $\mathbb{E}/\mathbb{F}$ be an extension of finite fields. Then, $\md{\mathbb{F}} = q$ and $\md{\mathbb{E}} = q^n$ for some prime power $q$ and $n \in \mathbb{N}.$ Then, $\mathbb{E}$ is a splitting field for $x^{q^n} - x \in \mathbb{F}[x]$ over $\mathbb{F}.$ Thus, the extension is normal. \\
        Since the fields are finite, it is also separable.
        %
        \item The extension $\mathbb{Q}(\sqrt[3]{2})/\mathbb{Q}$ is \textbf{not} Galois. Since $\chr(\mathbb{Q}) = 0,$ it is separable. However, it is not normal. Indeed, the irreducible (by Eisenstein) polynomial $x^3 - 2 \in \mathbb{Q}[x]$ has a root in $\mathbb{Q}(\sqrt[3]{2})$ but it does not split as a product of linear factors.
        %
        \item The extension $\mathbb{F}_p(X)(X^{1/p})/\mathbb{F}_p(X)$ is not separable and hence, \textbf{not} Galois. It \emph{is} normal since the bigger field is the splitting field of $T^p - X \in \mathbb{F}_p(X)[T].$
    \end{enumerate}
\end{ex}

\begin{restatable}[]{prop}{orderofgalgroup}
\label{prop:orderofgalgroup}
    Let $\mathbb{E}/\mathbb{F}$ be a finite Galois extension. Then, $\md{\Gal(\mathbb{E}/\mathbb{F})} = [\mathbb{E} : \mathbb{F}]_s = [\mathbb{E} : \mathbb{F}].$ \hfill\hyperref[prop:orderofgalgroup2]{\downsym}
\end{restatable}
Note that the last equality is simply by definition of a Galois extension (and \Cref{thm:sepiffdegequal}).

\begin{rem}
    The above proposition shows why normality and separability are both needed. If the extension is normal but not separable, then the order of the group would be the separable degree.

    On the other hand, if the extension is separable but not normal, then there would be an extension $\sigma : \mathbb{E} \to \overline{\mathbb{F}}$ would map $\mathbb{E}$ outside $\mathbb{E}$ and so, not all extensions will belong to the Galois group.

    As an example, consider $\mathbb{Q}(\sqrt[3]{2})/\mathbb{Q}.$ Since there is only one root of $x^3 - 2$ in $\mathbb{Q}(\sqrt[3]{2}),$ there is only one $\mathbb{Q}$-automorphism of $\mathbb{Q}(\sqrt[3]{2}).$
\end{rem}

\begin{restatable}[]{prop}{frobgenerates}
\label{prop:frobgenerates}
    Let $q$ be a prime power.

    The Galois group of the Galois extension $\mathbb{F}_{q^n}/\mathbb{F}_q$ is a cyclic group of order $n$ generated by the Frobenius automorphism $\varphi : \mathbb{F}_{q^n} \to \mathbb{F}_{q^n}$ defined as $a \mapsto a^q.$ \hfill\hyperref[prop:frobgenerates2]{\downsym}
\end{restatable}

\begin{ex}
    A field extension $\mathbb{K}/\mathbb{F}$ is called \deff{biquadratic} if $[\mathbb{K} : \mathbb{F}] = 4$ and $\mathbb{K}$ is generated over $\mathbb{F}$ by roots of two irreducible quadratic separable polynomials. 

    In particular, $\mathbb{K}/\mathbb{F}$ is a Galois extension. Write $\mathbb{K} = \mathbb{F}(\alpha, \beta)$ and let $p(x) \vcentcolon= \irr(\alpha, \mathbb{F})$ and $q(x) \vcentcolon= \irr(\beta, \mathbb{F}).$ Let $\overline{\alpha}, \overline{\beta} \in \mathbb{K}$ denote the other root of $p(x)$ and $q(x).$ By assumption of separability, $\overline{\alpha} \neq/ \alpha$ and $\overline{\beta} \neq \beta.$

    Since $[\mathbb{F}(\alpha, \beta) : \mathbb{F}] = 4,$ the quadratic $p(x)$ is irreducible over $\mathbb{F}(\beta)$ and similarly for $q(x)$ over $\mathbb{F}(\alpha).$ Thus, the four automorphisms are determined by sending $\alpha$ to $\alpha$ or $\overline{\alpha}$ and $\beta$ to $\beta$ or $\overline{\beta}.$

    Define the automorphisms $\tau, \sigma : \mathbb{K} \to \mathbb{K}$ by
    \begin{align*} 
        \tau(\alpha) = \overline{\alpha},\;\tau(\beta) = \beta,\\
        \sigma(\alpha) = \alpha,\;\sigma(\beta) = \overline{\beta}.
    \end{align*}
    Then, $\tau^2 = \sigma^2 = \id_{\mathbb{K}}.$ Thus, $\Gal(\mathbb{K}/\mathbb{F}) \cong \mathbb{Z}/2\mathbb{Z} \times \mathbb{Z}/2\mathbb{Z},$ the Klein-$4$ group.
\end{ex}

\begin{ex}[Galois group of a separable cubic] \label{ex:galsepcubic}
    We show the role of the discriminant in determining the Galois group of a cubic.

    Let $\mathbb{F}$ be a field with $\chr(\mathbb{F}) \neq 2, 3.$ Let $f(x) = x^3 + px + q \in \mathbb{F}[x]$ be an irreducible cubic. In particular, $f(x)$ has no roots in $\mathbb{F}.$ We wish to show that $f(x)$ is separable. Note that
    \begin{equation*} 
        f'(x) = 3x^2 + p \neq 0,
    \end{equation*}
    since $\chr(\mathbb{F}) \neq 3.$ Thus, $f(x)$ is separable, by \Cref{prop:irredsepderiv}.

    Thus, a splitting field $\mathbb{E}$ of $f(x)$ over $\mathbb{F}$ has degree either $3$ or $6.$ By \Cref{prop:orderofgalgroup}, we know that $\md{\Gal(\mathbb{E}/\mathbb{F})} = 3$ or $6.$ We see now how the discriminant determines this. 

    Let $\mathbb{E} = \mathbb{F}(\alpha_1, \alpha_2, \alpha_3),$ where $f(x) = \prod_{i = 1}^3 (x - \alpha_i).$ Any $\sigma \in \Gal(\mathbb{E}/\mathbb{F})$ permutes these roots. Let $p_\sigma \in S_3$ denote the corresponding permutation. It is easy to see that $\sigma \mapsto p_\sigma$ is injective. (Action of $\sigma$ on $\sigma_i$ completely determines the automorphism.) Under this, we identify $\Gal(\mathbb{E}/\mathbb{F})$ with a subgroup of $S_3.$

    Thus, $\Gal(\mathbb{E}/\mathbb{F}) = A_3$ or $S_3.$ Let
    \begin{equation*} 
        \delta = (\alpha_1 - \alpha_2)(\alpha_2 - \alpha_3)(\alpha_3 - \alpha_1).
    \end{equation*}
    Then, $\delta^2 = \disc(f(x)) = -(4p^3 + 27q^2) \in \mathbb{F}.$ (Recall we had calculated this discriminant in \Cref{ex:disccubic}.)

    Thus, $[\mathbb{F}(\delta) : \mathbb{F}] \le 2.$ Now, if $\delta \in \mathbb{F},$ then $\Gal(\mathbb{E}/\mathbb{F})$ cannot have any odd permutations since they do not fix $\delta$ and hence, $\Gal(\mathbb{E}/\mathbb{F}) = A_3.$

    On the other hand, if $\delta \notin \mathbb{F},$ then $2 = [\mathbb{F}(\delta) : \mathbb{F}] \mid [\mathbb{E} : \mathbb{F}]$ and so, $\Gal(\mathbb{E}/\mathbb{F}) = S_3.$

    Note that $\delta \in \mathbb{F} \iff \disc(f(x))$ is a perfect square in $\mathbb{F}.$ Thus, the above is characterised entirely by $\disc(f(x))$ being a perfect square.

    For example, if $f(x) = x^3 + x + 1 \in \mathbb{Q}[x],$ then $\disc(f(x)) = -31$ and so, $\Gal(\mathbb{E}/\mathbb{Q}) \cong S_3.$ On the other hand, if $f(x) = x^3 - 3x + 1,$ then $\disc(f(x)) = 81 = 9^2$ and thus, $\Gal(\mathbb{E}/\mathbb{Q}) \cong A_3.$
\end{ex}

\section{The Fundamental Theorem of Galois Theory}
\begin{defn}%[]
    Let $\mathbb{E}$ be a field and $G$ be \underline{a} group of automorphisms of $\mathbb{E}.$ Then,
    \begin{equation*} 
        \mathbb{E}^G \vcentcolon= \{a \in \mathbb{E} : \sigma(a) = a \text{ for all } \sigma \in G\}
    \end{equation*}
    is called the \deff{fixed field of $G$ acting on $E$}.
\end{defn}
\begin{rem}
    As one can easily show, the above is indeed a field. 

    Note that $G$ is not necessarily the group of \emph{all} automorphisms of $\mathbb{E}.$
\end{rem}

\begin{restatable}[Fundamental Theorem of Galois Theory (FTGT)]{thm}{FTGT}
\label{thm:FTGT}
    Let $\mathbb{K}/\mathbb{F}$ be a \underline{finite} Galois extension. Consider the sets
    \begin{equation*} 
        \mathcal{I} = \{\mathbb{E} \mid \mathbb{E} \text{ is an intermediate field of }\mathbb{K}/\mathbb{F}\} \andd \mathcal{G} = \{H \mid H \le \Gal(\mathbb{K}/\mathbb{F})\}.
    \end{equation*}
    \begin{enumerate}
         \item \label{item:G1} The maps 
         \begin{equation*} 
             E \mapsto \Gal(\mathbb{K}/\mathbb{E}) \andd H \mapsto \mathbb{K}^H
         \end{equation*}
         give a one-to-one correspondence between $\mathcal{I}$ and $\mathcal{G},$ called the \deff{Galois correspondence}. Moreover, these are inclusion reversing.
         %
         \item \label{item:G2} $\mathbb{E}/\mathbb{F}$ is Galois iff $\Gal(\mathbb{K}/\mathbb{E}) \unlhd \Gal(\mathbb{K}/\mathbb{F})$ and in this case,
         \begin{equation*} 
             \Gal(\mathbb{E}/\mathbb{F}) \cong \frac{\Gal(\mathbb{K}/\mathbb{F})}{\Gal(\mathbb{K}/\mathbb{E})}.
         \end{equation*}
         %
         \item \label{item:G3} $\mathbb{K}/\mathbb{E}$ is always Galois and $\md{\Gal(\mathbb{K}/\mathbb{E})} = [\mathbb{K} : \mathbb{E}] = \dfrac{[\mathbb{K} : \mathbb{F}]}{[\mathbb{E} : \mathbb{F}]}.$
         \item \label{item:G5} If $\mathbb{E}_1, \mathbb{E}_2 \in \mathcal{I}$ correspond to $H_1$ and $H_2,$ then $\mathbb{E}_1 \cap \mathbb{E}_2$ corresponds to $\langle H_1, H_2\rangle$ and $\mathbb{E}_1\mathbb{E}_2$ to $H_1 \cap H_2.$
     \end{enumerate} \hfill\hyperref[thm:FTGT2]{\downsym}
\end{restatable}

The proof of the above will be given in many steps. Parts of it will be proven for infinite Galois extensions as well. Note that \ref{item:G3} follows from \Cref{prop:orderofgalgroup}.

For the rest of the section, $\mathbb{K}/\mathbb{F}$ will denote a {\color{purple}(possibly infinite)} Galois extension and $\mathcal{I}$ and $\mathcal{G}$ will be as in \Cref{thm:FTGT}.

\begin{restatable}[]{thm}{fixfieldinjectiveIG}
\label{thm:fixfieldinjectiveIG}
    Let $\mathbb{K}/\mathbb{F}$ be a {\color{purple}(possibly infinite)} Galois extension and put $G = \Gal(\mathbb{K}/\mathbb{F}).$ Then,
    \begin{enumerate}
         \item $\mathbb{F} = \mathbb{K}^G.$
         \item Let $\mathbb{E} \in \mathcal{I}.$ Then, $\mathbb{K}/\mathbb{E}$ is Galois and the map $E \mapsto \Gal(\mathbb{K}/\mathbb{E})$ is an injective map from $\mathcal{I}$ to $\mathcal{G}.$ \hfill\hyperref[thm:fixfieldinjectiveIG2]{\downsym}
     \end{enumerate} 
\end{restatable}

\begin{rem} \label{rem:nonbasemoved}
    The above again shows the need for Galois extension. For example, consider the non-Galois extension $\mathbb{Q}(\sqrt[3]{2})/\mathbb{Q}.$ If we consider $G$ to be the ``Galois group,'' that is, $G$ to be the group of automorphisms of $\mathbb{Q}(\sqrt[3]{2})$ which fix $\mathbb{Q},$ we see that $G$ is trivial. Thus, $\mathbb{Q}(\sqrt[3]{2})^G = \mathbb{Q}(\sqrt[3]{2}).$

    However, for Galois extensions, the above says that the only field which is fixed by all the Galois automorphisms is precisely the base field.
\end{rem}

\begin{restatable}[]{lem}{degboundedbyn}
\label{lem:degboundedbyn}
    Let $\mathbb{E}/\mathbb{F}$ be a separable extension and $n \in \mathbb{N}.$ Suppose that for all $\alpha \in \mathbb{E},$ $[\mathbb{F}(\alpha) : \mathbb{F}] \le n.$ Then, $[\mathbb{E} : \mathbb{F}] \le n.$ \hfill\hyperref[lem:degboundedbyn2]{\downsym}
\end{restatable}

\begin{rem}
    Note that the above did not assume a priori that $\mathbb{E}/\mathbb{F}$ is finite. If that were the case, then the \nameref{thm:pet} would yield the answer.

    The above is not true without the assumption of separability. For example, consider $\mathbb{F} = \mathbb{F}_p(X, Y)$ where $p$ is a prime. Consider $\mathbb{E} = \mathbb{F}(X^{1/p}, Y^{1/p}).$

    Then, $\alpha^p \in \mathbb{F}$ for all $\alpha \in \mathbb{E}$ (exercise) and thus, $[\mathbb{E}(\alpha) : \mathbb{F}] \le p$ for all $\alpha \in \mathbb{E}.$ However, $[\mathbb{E} : \mathbb{F}] = p^2 > p.$
\end{rem}

\begin{restatable}[Artin's Theorem]{thm}{artin}
\label{thm:artin}
    Let $\mathbb{E}$ be a field and $G$ a \underline{finite} group of automorphisms of $\mathbb{E}.$ Then,
    \begin{enumerate}
         \item $\mathbb{E}/\mathbb{E}^G$ is a \emph{finite} Galois extension.
         \item $\Gal(\mathbb{E}/\mathbb{E}^G) = G.$
         \item $[\mathbb{E} : \mathbb{E}^G] = \md{G}.$ \hfill\hyperref[thm:artin2]{\downsym}
     \end{enumerate} 
\end{restatable}

\begin{restatable}[]{thm}{galoissubgroupscompositum}
\label{thm:galoissubgroupscompositum}
    Let $\mathbb{K}/\mathbb{F}$ be a {\color{purple}(possibly infinite)} Galois extension with Galois group $G.$ Let $\mathbb{E}_1$ and $\mathbb{E}_2$ be intermediate subfields of $\mathbb{K}/\mathbb{F}.$ Let $H_i \vcentcolon= \Gal(\mathbb{K}/\mathbb{E}_i)$ for $i = 1, 2.$Then
    \begin{equation*} 
        \mathbb{E}_1\mathbb{E}_2 = \mathbb{K}^{H_1 \cap H_2},\; \mathbb{E}_1 \cap \mathbb{E}_2 = \mathbb{K}^{\langle H_1, H_2\rangle}, \text{ and } \mathbb{E}_1 \subset \mathbb{E}_2 \iff H_1 \supset H_2.
    \end{equation*} \hfill\hyperref[thm:galoissubgroupscompositum2]{\downsym}
\end{restatable}

\begin{rem}
    Essentially the thing to keep in mind is that smaller subfields corresponding to larger subgroups. Now, given two subfields/subgroups, we have the corresponding smallest (or largest) subfield/subgroup containing them (or being contained in them). The above shows that the Galois correspondence (in one direction) preserves them.

    (The smallest field containing the subfields is the fixed field of the action of the largest subgroup contained in the Galois groups.\\
    The largest field containing the subfields is the fixed field of the action of the smallest subgroup containing the Galois groups.)
\end{rem}

\begin{restatable}[]{prop}{isomorphismgalois}
\label{prop:isomorphismgalois}
    Let $\mathbb{K}/\mathbb{F}$ be a {\color{purple}(possibly infinite)} Galois extension. Let $\lambda : \mathbb{K} \to \lambda(\mathbb{K})$ be an isomorphism of fields. Then,
    \begin{enumerate}
         \item $\lambda(\mathbb{K})/\lambda(\mathbb{F})$ is a Galois extension.
         \item $\Gal(\lambda(\mathbb{K})/\lambda(\mathbb{F})) = \lambda \Gal(\mathbb{K}/\mathbb{F}) \lambda^{-1} \cong \Gal(\mathbb{K}/\mathbb{F}).$ \hfill\hyperref[prop:isomorphismgalois2]{\downsym}
    \end{enumerate} 
\end{restatable}

\begin{restatable}[]{thm}{galoisiffnormal}
\label{thm:galoisiffnormal}
    Let $\mathbb{K}/\mathbb{F}$ be a {\color{purple}(possibly infinite)} Galois extension. Let $\mathbb{E}$ be an intermediate subfield of $\mathbb{K}/\mathbb{F}.$ Then,
    \begin{enumerate}
        \item $\mathbb{E}/\mathbb{F}$ is Galois iff $\Gal(\mathbb{K}/\mathbb{E}) \unlhd \Gal(\mathbb{K}/\mathbb{F}).$
        \item If $\mathbb{E}/\mathbb{F}$ is Galois, then
        \begin{equation*} 
             \Gal(\mathbb{E}/\mathbb{F}) \cong \frac{\Gal(\mathbb{K}/\mathbb{F})}{\Gal(\mathbb{K}/\mathbb{E})}.
        \end{equation*}
        \hfill\hyperref[thm:galoisiffnormal2]{\downsym}
     \end{enumerate} 
\end{restatable}

With this, we can now prove the \nameref{thm:FTGT}. \hfill\hyperref[thm:FTGT2]{\downsym}


\section{Applications of FTGT}
We give another proof of the Fundamental Theorem of Algebra.

\begin{restatable}[Fundamental Theorem of Algebra]{thm}{ftagalois}
\label{thm:ftagalois}
    The field of complex numbers is algebraically closed. \hfill\hyperref[thm:ftagalois2]{\downsym}
\end{restatable}

\begin{ex}[Symmetric rational functions]
    Let $\mathbb{E} = \mathbb{F}(x_1, \ldots, x_n)$ be the fraction field of $R = \mathbb{F}[x_1, \ldots, x_n],$ where $x_i$ are indeterminates over the field $\mathbb{F}.$

    We had seen that the symmetric polynomials in $R$ are the polynomials in the symmetric polynomials. We now prove an analogous result for symmetric rational functions. 

    Note that $S_n$ acts on $\mathbb{E}$ in the natural way. More precisely, if $\sigma \in S_n,$ then we have the $\mathbb{F}$-automorphism $\varphi_\sigma : \mathbb{E} \to \mathbb{E}$ determined by $\varphi_\sigma(x_i) = x_{\sigma(i)}.$ Note that $\varphi_{\sigma_1\sigma_2} = \varphi_{\sigma_1} \circ \varphi_{\sigma_2}$ and thus, $G = \{\varphi_\sigma : \sigma \in S_n\}$ is a group of automorphisms of $\mathbb{E}$ and is isomorphic to $S_n.$

    Let $\sigma_1, \ldots, \sigma_n \in \mathbb{E}$ be the elementary symmetric polynomials in $x_1, \ldots, x_n.$ Let $X$ be an indeterminate over $\mathbb{E}$ and consider the polynomial ring $\mathbb{E}[X].$ \\
    Each the automorphisms $\varphi_\sigma$ to automorphisms of $\mathbb{E}[X]$ by fixing $X.$ We denote the extension again by $\varphi_\sigma.$

    Consider
    \begin{align*} 
        g(X) &\vcentcolon= (X - x_1) \cdots (X - x_n)\\
        &= X^n - \sigma_1 X^{n - 1} + \cdots + (-1)^n\sigma_n.
    \end{align*}

    Let $\sigma \in S_n$ be arbitrary. Applying $\varphi_\sigma$ to the first line above yields
    \begin{equation*} 
        \varphi_\sigma(g(X)) = (X - x_{\sigma(1)}) \cdots (X - x_{\sigma(n)}) = g(X).
    \end{equation*}
    Thus, each $\varphi_\sigma$ fixes $g(X)$ and in turn, it fixes the coefficients $\sigma_1, \ldots, \sigma_n.$ Thus,
    \begin{equation*} 
        \mathbb{F}(\sigma_1, \ldots, \sigma_n) \subset \mathbb{E}^G.
    \end{equation*}
    Note that
    \begin{equation*} 
        \mathbb{E} = \mathbb{F}(\sigma_1, \ldots, \sigma_n, x_1, \ldots, x_n)
    \end{equation*}
    and so, $\mathbb{E}$ is a splitting field of $g(X)$ over $\mathbb{F}(\sigma_1, \ldots, \sigma_n).$ Since $g(X)$ is separable, we see that $\mathbb{E}/\mathbb{F}(\sigma_1, \ldots, \sigma_n)$ is a Galois extension. 

    Now, if $\pi \in \Gal(\mathbb{E}/\mathbb{F}(\sigma_1, \ldots, \sigma_n)),$ then $\pi$ permutes the roots of $g(X)$ and fixes $\mathbb{F}.$ Thus, $\pi = \varphi_\sigma$ for some $\sigma \in S_n.$ Thus, $G = \Gal(\mathbb{E}/\mathbb{F}(\sigma_1, \ldots, \sigma_n)).$

    Thus, we see that
    \begin{equation*} 
        \mathbb{F}(\sigma_1, \ldots, \sigma_n) = \mathbb{E}^G.
    \end{equation*}
    The left is the field of all rational functions in the symmetric polynomials. The right is the field of all rational functions fixed by $S_n,$ that is, the symmetric rational functions.
\end{ex}
