\chapter{Some Group Theory}

Although already mentioned in \Cref{chap:00}, we repeat: $[n] \vcentcolon= \{1, \ldots, n\}$ for $n \in \mathbb{N}.$

\section{Solvable groups}

\begin{defn}%[]
	Let $G$ be a group. A sequence of subgroups
	\begin{equation*} 
		1 = G_0 \subset G_1 \subset \cdots \subset G_{s} = G
	\end{equation*}
	is called a \deff{normal series for $G$} if $G_i$ is a normal subgroup of $G_{i - 1}$ for $i = 1, \ldots, s.$ The \deff{length} of this series is $s.$ The normal series is called \deff{abelian} (resp., \deff{cyclic}) if the quotients $G_i/G_{i - 1}$ are abelian (resp., cyclic) for $i = 1, \ldots, s.$

	A group having an abelian series is called a \deff{solvable group}.
\end{defn}

\begin{rem}
	Note that the length is the number of inclusions, whereas there are $s + 1$ subgroups in the above series (including $1$ and $G$).
\end{rem}

\begin{ex}[Solvable groups]
	\phantom{hi}
	\begin{enumerate}
		\item Any abelian group $G$ is solvable with 
		\begin{equation*} 
			1 \unlhd G
		\end{equation*} 
		being an abelian series. In particular, so are $S_1$ and $S_2.$
		\item $S_3$ is solvable since 
		\begin{equation*} 
			1 \unlhd A_3 \unlhd S_3
		\end{equation*}
		is an abelian series. Indeed, $A_3$ is normal in $S_3$ since it has index $2$ and the quotient has order $2$ and hence, is abelian. Since $A_3$ has order $3,$ it is abelian; thus, $1 \unlhd A_3$ and $A_3/1$ is abelian.
		\item $S_4$ is solvable as well with
		\begin{equation*} 
			1 \unlhd V_4 \unlhd A_4 \unlhd S_4
		\end{equation*}
		being an abelian series. Here, $V_4 = \{1, (12)(34), (13)(24), (14)(23)\}.$

		We only need to verify that $V_4 \unlhd A_4.$ (The quotient will be abelian since it has order $3.$) That $V_4 \le A_4$ is clear since all the permutations are indeed even. Now, from the cycle type, we see that $V_4$ is actually normal in $S_4$ itself.
		%
		\item As we shall see later, $S_n$ is not solvable for $n \ge 5.$
	\end{enumerate}
\end{ex}

\begin{restatable}[]{prop}{pgroupssolvable}
\label{prop:pgroupssolvable}
	Any group with order $p^n$ is solvable, where $p$ is a prime and $n \in \mathbb{N}_0.$ \hfill\hyperref[prop:pgroupssolvable2]{\downsym}
\end{restatable}

\begin{defn}%[]
	Let $G$ be a group. The \deff{commutator} of $g, h \in G$ is defined as
	\begin{equation*} 
		[g, h] \vcentcolon= g^{-1}h^{-1}gh.
	\end{equation*}
	The \deff{derived subgroup} of $G$ denoted by $G'$ or $G^{(1)}$ or $[G, G]$ is the subgroup generated by all the commutators in $G.$ The \deff{$k$-th derived subgroup} of $G$ is defined inductively as $G^{(k)} = \left(G^{(k - 1)}\right)'$ for $k \ge 2.$
\end{defn}

\begin{rem}
	\phantom{hi}
	\begin{enumerate}
		\item $[g, h] = 1$ iff $g$ and $h$ commute.
		\item As a result, $G' = 1$ iff $G$ is abelian.
		\item If $H \le G,$ then $H' \le G'.$
		\item In general, the derived subgroup is \emph{generated} by commutators and is not equal to the set of commutators itself. (The smallest example is a certain group of order $96.$)
	\end{enumerate}
\end{rem}

\begin{defn}%[]
	Let $G$ be a group and $a \in G.$ Then, the \deff{inner automorphism $i_a$} is the automorphism $i_a \in \Aut(G)$ defined as
	\begin{equation*} 
		i_a(g) \vcentcolon= a^{-1}ga.
	\end{equation*}
\end{defn}

Clearly, $i_a$ is a homomorphism. To see that it an isomorphism, note that $i_{a^{-1}}$ is an inverse.

\begin{restatable}[]{prop}{commutatorresults}
\label{prop:commutatorresults}
	Let $f : G \to H$ be a homomorphism of groups and $s \in \mathbb{N}.$
	\begin{enumerate}
	 	\item $f(G^{(s)}) \le H^{(s)}.$ If $f$ is onto, then $f(G^{(s)}) = H^{(s)}.$
	 	\item If $K \unlhd G,$ then $K' \unlhd G.$ In particular, $G' \unlhd G.$
	 	\item If $K \unlhd G,$ then $G/K$ is abelian iff $G' \le K.$ \hfill\hyperref[prop:commutatorresults2]{\downsym}
	 \end{enumerate} 
\end{restatable}

\begin{rem}
	The last point essentially says that the derived subgroup is the smallest subgroup one must quotient by, to get an abelian group.
\end{rem}

\begin{restatable}[]{prop}{solvableifftrivialderiv}
\label{prop:solvableifftrivialderiv}
	A group $G$ is solvable iff $G^{(s)} = 1$ for some $s \in \mathbb{N}.$ \hfill\hyperref[prop:solvableifftrivialderiv2]{\downsym}
\end{restatable}

\begin{restatable}[]{prop}{deriveofquotient}
\label{prop:deriveofquotient}
	Let $K \unlhd G$ be groups. Then,
	\begin{equation*} 
		\left(\frac{G}{K}\right)^{(s)} = \frac{\langle G^{(s)}, K\rangle}{K}.
	\end{equation*}
	\hfill\hyperref[prop:deriveofquotient2]{\downsym}
\end{restatable}

\begin{restatable}[]{prop}{twoofthreesolvable}
\label{prop:twoofthreesolvable}
	Let $G$ and $H$ be groups.
	\begin{enumerate}
	 	\item If $G$ is solvable and there is an injection $i : H \to G,$ then $H$ is solvable. In particular, subgroups of solvable groups are solvable.
	 	\item If $G$ is solvable and there is a surjection $f : G \to H,$ then $H$ is solvable. In particular, quotients of solvable groups are solvable.
	 	\item If $K \unlhd G$ is such that $K$ and $G/K$ are solvable, then $G$ is solvable. \hfill\hyperref[prop:twoofthreesolvable2]{\downsym}
	\end{enumerate} 
\end{restatable}

\begin{rem}
	For those familiar with the lingo, the above proposition says:

	Let $0 \to H \to G \to K \to 0$ be an exact sequence of groups. Then, $G$ is solvable iff $H$ and $K$ are solvable.
\end{rem}

\begin{restatable}[]{prop}{refiningabelianseries}
\label{prop:refiningabelianseries}
	Let $G$ be a finite solvable group. Then, there exists a normal series
	\begin{equation*} 
		1 = G_0 \unlhd G_1 \unlhd \cdots \unlhd G_s = G
	\end{equation*}
	such that $G_i/G_{i - 1}$ is cyclic of prime order for all $i = 1, \ldots, s.$ \hfill\hyperref[prop:refiningabelianseries2]{\downsym}
\end{restatable}

\section{Some results about Symmetric Groups}

\begin{restatable}[]{lem}{Angenerator}
\label{lem:Angenerator}
	For $n \ge 3,$ $A_n$ is generated by $3$-cycles. If $n \ge 5,$ then all the $3$-cycles are conjugates in $A_n.$ \hfill\hyperref[lem:Angenerator2]{\downsym}
\end{restatable}

\begin{restatable}[]{thm}{SnAnnotsolvable}
\label{thm:SnAnnotsolvable}
	The groups $S_n$ and $A_n$ are not solvable for $n \ge 5.$ \hfill\hyperref[thm:SnAnnotsolvable2]{\downsym}
\end{restatable}

\begin{restatable}[]{thm}{Ansimple}
\label{thm:Ansimple}
	The alternating group $A_n$ is simple for $n \ge 5.$ \hfill\hyperref[thm:Ansimple2]{\downsym}
\end{restatable}

\subsection{Generators of Symmetric Groups}

Of course, everyone knows the first one.

\begin{thm} \label{thm:gentranspose}
	For $n \ge 2,$ $S_n$ is generated by its transpositions.
\end{thm}

\begin{restatable}[]{thm}{genconsectranpose}
\label{thm:genconsectranpose}
	For $n \ge 2,$ $S_n$ is generated by the $n - 1$ transpositions
	\begin{equation*} 
		(12), (13), \ldots, (1n).
	\end{equation*} \hfill\hyperref[thm:genconsectranpose2]{\downsym}
\end{restatable}

\begin{restatable}[]{thm}{genconsectranposespecial}
\label{thm:genconsectranposespecial}
	For $n \ge 2,$ $S_n$ is generated by the $n - 1$ transpositions
	\begin{equation*} 
		(1 \ 2), (2 \ 3), \ldots, (n - 1 \ n).
	\end{equation*}  \hfill\hyperref[thm:genconsectranposespecial2]{\downsym}
\end{restatable}

\begin{restatable}[]{thm}{gentransposecycle}
\label{thm:gentransposecycle}
	For $n \ge 2,$ $S_n$ is generated by the transposition $(12)$ and the $n$-cycle $(12 \ldots n).$ \hfill\hyperref[thm:gentransposecycle2]{\downsym}
\end{restatable}

\begin{restatable}[]{cor}{genprimetranscycle}
\label{cor:genprimetranscycle}
	Let $p \ge 3$ be a prime. Then, $S_p$ is generated by any pair of transposition and $p$-cycle. \hfill\hyperref[cor:genprimetranscycle2]{\downsym}
\end{restatable}

\begin{rem}
	In general, it is not true that any transposition and $n$-cycle generates $S_n.$ For example, $(12)$ and $(1234)$ do not generate $S_4.$ To see this, consider the dihedral group $D_8$ of order $8$ as a subgroup of $S_4$ by numbering the vertices of a square as $1, 2, 3, 4.$ Then, $(12), (1234) \in D_8 \subsetneq S_4.$
\end{rem}