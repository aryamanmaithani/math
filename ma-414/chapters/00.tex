\chapter{Preliminaries} \label{chap:00}

\section{Notations and Conventions}
\begin{enumerate}
    \item $\mathbb{N}$ will denote the set of \textbf{positive} integers. That is, $\mathbb{N} = \{1, 2, \ldots\}.$
    \item $\mathbb{Z}$ will denote the set of integers.
    \item $\mathbb{N}_0$ will denote the set of all \textbf{non-negative} integers. \\
    That is, $\mathbb{N}_0 = \{0, 1, 2, \ldots\} = \mathbb{N} \cup \{0\}.$
    \item $\mathbb{Q}$ will denote the set of rationals.
    \item $\mathbb{R}$ will denote the set of real numbers.
    \item $\mathbb{C}$ will denote the set of complex numbers.
    \item Blackboard letters like $\mathbb{F}, \mathbb{E}, \mathbb{K}, \mathbb{L}$ will denote an arbitrary field.
    \item Given any field $\mathbb{F},$ $\mathbb{F}^\times$ denotes the group of units of $\mathbb{F}.$ That is, $\mathbb{F}^\times = \mathbb{F}\setminus\{0\}.$
    \item Given a ring $R,$ $R^\times$ denotes the group of units of $R.$ 
    \item Whenever we write ``$F \subset E$ are fields,'' we mean that $\mathbb{E}$ is a field and $\mathbb{F}$ is a subfield of $\mathbb{E}.$
    \item $\zeta_n \vcentcolon= \exp\left(\dfrac{2\pi\iota}{n}\right).$
    \item The degree of the zero polynomial is $-\infty.$
    \item Given a group $G$ and $g \in G,$ we denote the order of $g$ (in $G$) as $o(g).$
    % \item Given a group $G$ and subgroups $H_1, H_2 \unlhd G,$ we denote by $\langle H_1, H_2\rangle$ the smallest subgroup of $G$ containing $H_1$ and $H_2.$
    \item For $n \ge 1,$ we denote $\{1, \ldots, n\}$ as $[n].$
\end{enumerate}

\section{Field Theory}

We shall assume that the reader is familiar with the definitions and basic properties of groups and rings. All rings in this document will be assumed to be commutative with identity. 

We list some basic definition and properties. The proofs might be a bit terse and you should not have much problem filling in the details. (This won't be the case in the later chapters!)

\begin{defn}%[]
    An \deff{integral domain} is a ring with $0 \neq 1$ such $ab = 0 \implies a = 0$ or $b = 0.$
\end{defn}

\begin{defn}%[Field]
    A \deff{field} $(\mathbb{F}, +, \cdot)$ is a ring with $0 \neq 1$ such that every non-zero element has a multiplicative inverse.
\end{defn}

\begin{ex}
    $\mathbb{Q}, \mathbb{R}, \mathbb{C}$ are all fields.
\end{ex}

\begin{defn}%[]
    Given an integral domain $R,$ the field of fractions of $R$ is denoted by $\Frac(R).$
\end{defn}

\begin{defn}%[]
    A \deff{ring homomorphism} is a map $\varphi : R \to S$ between rings such that
    \begin{enumerate}
        \item $\varphi(ab) = \varphi(a)\varphi(b)$ for all $a, b \in R,$
        \item $\varphi(a + b) = \varphi(a) + \varphi(b)$ for all $a, b \in R,$
        \item $\varphi(1_R) = 1_S.$
    \end{enumerate}
    A \deff{field homomorphism} is a ring homomorphism between fields.
\end{defn}

\begin{defn}%[]
    Given a prime $p \in \mathbb{N},$ $\mathbb{Z}/p\mathbb{Z}$ is a field, which we denote as $\mathbb{F}_p.$
\end{defn}

\begin{defn}%[]
    Let $\mathbb{F}$ be a field. The \deff{characteristic} of $\mathbb{F}$ is defined to be the smallest positive integer $n$ such that
    \begin{equation*} 
        \underbrace{1_{\mathbb{F}} + \cdots + 1_{\mathbb{F}}}_{n} = 0_{\mathbb{F}}.
    \end{equation*}
    If no such $n$ exists, then the characteristic is defined to be $0.$ 

    This is denoted by $\chr \mathbb{F}.$
\end{defn}

From now on, we shall omit the subscript $\mathbb{F}$ when it is clear what the $0$ and $1$ are.

\begin{prop}
    If $\chr \mathbb{F} > 0,$ then $\chr \mathbb{F}$ is prime.
\end{prop}
\begin{proof} 
    Let $n \vcentcolon= \chr \mathbb{F}$ and let $n = ab$ for some $a, b \in \mathbb{F}.$ By distributivity and definition of $n,$ we have
    \begin{equation*} 
        \underbrace{(1 + \cdots + 1)}_{a}\underbrace{(1 + \cdots + 1)}_{b} = 0.
    \end{equation*}
    Since $\mathbb{F}$ is a field, one of the above two terms is $0.$ Without loss of generality, the first term is $0.$ By definition, $n = \chr \mathbb{F} \le a.$ But $a \mid n \implies a \le n.$

    Thus, $a = n.$
\end{proof}

\begin{prop}
    Every field contains an isomorphic copy of either $\mathbb{Q}$ or $\mathbb{F}_p$ for some prime $p.$ In fact, this copy is precisely $\Frac(\mathbb{Z}/\langle \chr\mathbb{F}\rangle).$
\end{prop}
\begin{proof} 
    Given a field $\mathbb{F},$ consider the ring homomorphism $\varphi : \mathbb{Z} \to \mathbb{F}$ given by $1 \mapsto 1.$ \\
    Then, $\mathbb{F}$ contains an isomorphic copy of $\mathbb{Z}/\ker \varphi.$ Note that $\varphi = \langle n\rangle,$ where $n = \chr\mathbb{F}.$ If $n > 0,$ then $n$ is prime and we are done.

    If $n = 0,$ then $\mathbb{F}$ contains an isomorphic copy of $\mathbb{Z}.$ Thus, it must contain $\mathbb{Q}.$\footnote{Either argue by explicitly constructing an isomorphism or use the universal property of fraction fields.}
\end{proof}

\begin{defn}%[]
    Given a field $\mathbb{F},$ the \deff{prime subfield} of $\mathbb{F}$ is defined as the smallest subfield of $\mathbb{F}.$ It is the intersection of all subfields of $\mathbb{F}.$ 
\end{defn}

\begin{prop}
    \phantom{hi}
    \begin{enumerate}
        \item The prime subfield of $\mathbb{F}$ is isomorphic to $\Frac(\mathbb{Z}/\langle \chr\mathbb{F}\rangle).$
        \item Let $\varphi : \mathbb{F} \to \mathbb{E}$ be a field homomorphism. Then, $\chr \mathbb{F} = \chr \mathbb{E}$ and $\varphi$ is injective. 
        \item Let $\mathbb{F} \subset \mathbb{E}$ be fields. $\mathbb{F}$ and $\mathbb{E}$ have the same prime subfield. Any field homomorphism $\varphi : \mathbb{F} \to \mathbb{E}$ fixes this prime subfield.
    \end{enumerate}
\end{prop}

\begin{defn}%[]
    Since any field homomorphism is injective, we also call them \deff{embeddings}.
\end{defn}

\begin{defn}
    Given fields $\mathbb{F} \subset \mathbb{E}_1, \mathbb{E}_2,$ and \deff{$\mathbb{F}$-isomorphism} from $\mathbb{E}_1$ to $\mathbb{E}_2$ is a field homomorphism $\varphi : \mathbb{E}_1 \to \mathbb{E}_2$ fixing $\mathbb{F}.$
\end{defn}

\begin{defn}%[]
    Given rings $R \subset S,$ and $\alpha \in S,$ we define $R[\alpha]$ to be the smallest subring of $S$ containing $\alpha$ and $R.$ 

    Given fields $\mathbb{F} \subset \mathbb{K},$ and $\alpha \in \mathbb{K},$ we define $\mathbb{F}(\alpha)$ to be the smallest subfield of $\mathbb{K}$ containing $\alpha$ and $\mathbb{F}.$ 

    Similarly, given a set $A \subset R$ (or $A \subset \mathbb{F}$), we can talk about $R[A]$ (or $\mathbb{F}(A)$) to be the smallest ring (or subfield) \deff{generated by $A$ over $R$ (or $\mathbb{F}$)}.
\end{defn}

\begin{prop} \label{prop:FAdesc}
    Let $\mathbb{F} \subset \mathbb{E}$ be fields and $A \subset \mathbb{E}$ a set. 

    If $A = \emptyset,$ then $\mathbb{F}(A) = \mathbb{F}.$ Assume $A \neq \emptyset.$

    Let 
    \begin{equation*} 
        M \vcentcolon= \{a_1a_2 \cdots a_n \mid n \in \mathbb{N},\; a_1, \ldots, a_n \in A\}
    \end{equation*}
    be the set of all finite products (monomials) of elements of $A.$

    Let
    \begin{equation*} 
        S \vcentcolon= \{b_0 + b_1m_1 + \cdots + b_nm_n \mid n \in \mathbb{N}_{0},\; m_1, \ldots, m_n \in M,\;b_0, b_1, \ldots, b_n \in \mathbb{F}\}
    \end{equation*}
    be the set of all finite sums of elements of $M.$ (These are polynomials in $A$ with coefficients in $\mathbb{F}.$)

    Then,
    \begin{equation} \label{eq:FAdesc}
        \mathbb{F}(A) = \left\{\frac{s_1}{s_2} \mid s_1, s_2 \in S \text{ and } s_2 \neq 0\right\}.
    \end{equation}
\end{prop}

\begin{proof} 
    The case $A = \emptyset$ is trivial. Assume $A \neq \emptyset.$

    Let the set on the right in \Cref{eq:FAdesc} be called $Q.$ 

    Note that $M$ is closed under products and $S$ is closed under sums and products both. Moreover, $S$ contains $\mathbb{F}$ as the constant polynomials. Using this, it is clear that $Q$ is a subfield of $\mathbb{E}.$ By taking denominator $1,$ we also see that $S \subset Q.$ Since $\mathbb{F} \subset S$ and $A \subset M \subset S,$ we see that $Q$ is a subfield of $\mathbb{E}$ containing $A$ and $\mathbb{F}.$ Thus, $\mathbb{F}(A) \subset Q.$

    On the other hand, note that $M \subset \mathbb{F}(A)$ since $A \subset \mathbb{F}(A).$ Since $\mathbb{F} \subset \mathbb{F}(A)$ as well, we get $S \subset \mathbb{F}(A).$ Thus, $Q \subset \mathbb{F}(A).$ (In all the assertions, we have used that $\mathbb{F}(A)$ is a subfield of $\mathbb{E}$ and thus, has the required closure properties.)
\end{proof}

\begin{cor} \label{cor:FAdescfinite}
    Let $\mathbb{F} \subset \mathbb{E}$ be fields and $A \subset \mathbb{E}$ a set. If $a \in \mathbb{F}(A),$ then there exists a finite set $B \subset A$ such that $a \in \mathbb{F}(B).$
\end{cor}

\begin{proof} 
    Let $a \in F(A).$ Let $M, S$ be as in \Cref{prop:FAdesc}. Then, $a = s_1/s_2$ for some $s_1, s_2 \in S.$ Then, each $s_i$ is a polynomial in some finitely many $a_i \in A$ with coefficients in $\mathbb{F}.$ Let $B$ be the set of finitely many $a_i.$ Then, $a \in \mathbb{F}(B).$
\end{proof}

\begin{prop}
    If $\mathbb{F}$ is a finite field, then $\chr(\mathbb{F}) =\vcentcolon p > 0$ and $\md{\mathbb{F}} = p^n$ for some $n \in \mathbb{N}.$
\end{prop}
\begin{proof} 
    $\chr(\mathbb{F}) = 0$ is not possible since $\mathbb{Z}$ is infinite and so, the homomorphism $\varphi : \mathbb{Z} \to \mathbb{F}$ given by $1 \mapsto 1$ cannot be injective.

    Now, $\mathbb{F}$ contains $\mathbb{F}_p$ as a subfield and hence, is a vector space over $\mathbb{F}.$ Since $\md{\mathbb{F}} < \infty,$ we have $\dim_{\mathbb{F}_p}(\mathbb{F}) =\vcentcolon n < \infty.$

    It is clear now that $\md{\mathbb{F}} = \md{\mathbb{F}_p}^n = p^n.$
\end{proof}

\begin{thm}
    Let $f(x) \in \mathbb{F}[x]$ have a degree $n \ge 1.$ Then, $f(x)$ has at most $n$ roots in $\mathbb{F}.$
\end{thm}
\begin{proof} 
    Induct on $n$ and use the fact that if $ab = 0 \implies a = 0$ or $b = 0,$ in a field.
\end{proof}

\begin{thm} \label{thm:finsubgroupcyclic}
    Let $\mathbb{F}$ be a field. Let $U$ be a finite subgroup of $\mathbb{F}^\times.$ Then, $U$ is cyclic. 
\end{thm}
We give two proofs.
\begin{proof} 
    This proof uses the following fact: Let $G$ be an abelian group and $a, b \in G$ have orders $m$ and $n.$ Then, there exist $c \in G$ with order $\operatorname{lcm}(m, n).$ (This needs a little argument. $c = ab$ works if $\gcd(m, n) = 1.$ The general case has to be reduced to that.)

    Let $n \vcentcolon= \md{U}.$ Let $a \in U$ be an element with maximal order, say $d.$ Then, we have
    \begin{equation*} 
        d = \operatorname{lcm} \{\operatorname{order}(u) \mid u \in U\}.
    \end{equation*}
    Thus, all $n$ elements of $U \subset \mathbb{F}$ satisfy the polynomial $x^d - 1 \in \mathbb{F}[x].$ Since $\mathbb{F}$ is a field, we have $n \le d.$ Thus, $d = n$ and $U = \langle a\rangle.$
\end{proof}

\begin{proof} 
    This prove uses the structure theorem of abelian groups. Let $n \vcentcolon= \md{U}.$

    Write $U \cong \mathbb{Z}/d_1\mathbb{Z} \times \cdots \times \mathbb{Z}/d_r\mathbb{Z}$ where $1 < d_1 \mid d_2 \mid \cdots \mid d_r$ and $n = d_1 \cdots d_r.$ Now, every element of $U$ satisfies $x^{d_r} - 1.$ Thus, as earlier, we have $d_r = n$ and hence, $n = 1.$ This means $U \cong \mathbb{Z}/n\mathbb{Z}$ is cyclic. 
\end{proof}

\begin{prop} \label{prop:divisibilityofpoly}
    Let $\mathbb{F} \subset \mathbb{K}$ be fields and $f(x), g(x) \in \mathbb{F}[x].$ \\
    Then, $f(x) \mid g(x)$ in $\mathbb{F}[x]$ iff every root of $f(x) \mid g(x)$ in $\mathbb{K}[x].$

    In particular, if $f(x)$ factorises linearly into distinct factors in $\mathbb{K}[x],$ then it suffices to show that every root of $f(x)$ is also one of $g(x).$
\end{prop}

\begin{proof}
    \forward This is obvious because a factorisation $g(x) = f(x)h(x)$ in $\mathbb{F}[x]$ also holds in $\mathbb{K}[x].$

    \backward If $f(x) = 0,$ then the result is true. Assume $f(x) \neq 0.$ \\
    By the division algorithm, we may write
    \begin{equation*} 
        g(x) = f(x)q(x) + r(x)
    \end{equation*}
    for unique $q(x), r(x) \in \mathbb{F}[x]$ with $\deg(r(x)) < \deg(q(x)).$

    The above is also a division in $\mathbb{K}[x].$ But $f(x) \mid g(x)$ in $\mathbb{K}[x]$ and so, uniqueness forces $r(x) = 0.$
\end{proof} 