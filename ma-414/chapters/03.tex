\chapter{Algebraic Closure of a Field}

\section{Existence}

\begin{defn}%[]
    A field $\mathbb{K}$ is called an \deff{algebraically closed field} if every non-constant polynomial $f(x) \in \mathbb{K}[x]$ has a root in $\mathbb{K}.$
\end{defn}

\begin{defn}%[]
    Let $\mathbb{K}/\mathbb{F}$ be a field extension. We say that $\mathbb{K}$ is an \deff{algebraic closure of $\mathbb{F}$} if $\mathbb{K}$ is algebraically closed and $\mathbb{K}/\mathbb{F}$ is an algebraic extension.
\end{defn}

We have the following simple proposition.
\begin{prop}
    \phantom{hi}
    \begin{enumerate}
        \item $\mathbb{K}$ is algebraically closed iff every non-constant polynomials factors as a product of linear factors.
        \item $\mathbb{C}$ is algebraically closed.
        \item If $\mathbb{K}$ is algebraically closed and $\mathbb{L}/\mathbb{K}$ is an algebraic extension, then $\mathbb{L} = \mathbb{K}.$
    \end{enumerate}
\end{prop}

\begin{restatable}[]{prop}{alglcosureinalgclosedisclosed}
\label{prop:alglcosureinalgclosedisclosed}
    Let $\mathbb{F} \subset \mathbb{K}$ be an extension where $\mathbb{K}$ is algebraically closed. Define,
    \begin{equation*} 
        \mathbb{A} \vcentcolon= \{\alpha \in \mathbb{K} : \alpha \text{ is algebraic over }\mathbb{F}\}.
    \end{equation*}
    Then, $\mathbb{A}$ is an algebraic closure of $\mathbb{F}.$ \hfill\hyperref[prop:alglcosureinalgclosedisclosed2]{\downsym}
\end{restatable}

\begin{restatable}[]{lem}{unionoffields}
\label{lem:unionoffields}
    Let $\{\mathbb{F}_i\}_{i \ge 1}$ be a sequence of fields as
    \begin{equation*} 
        \mathbb{F}_1 \subset \mathbb{F}_2 \subset \cdots.
    \end{equation*}
    Then, $\mathbb{F} \vcentcolon= \bigcup_{i \ge 1}\mathbb{F}_i$ is a field with the following operations:
    Given $a, b \in \mathbb{F},$ there exist smallest $i, j \in \mathbb{N}$ with $a \in \mathbb{F}_i$ and $b \in \mathbb{F}_j.$ Then, $a, b \in \mathbb{F}_{i + j}.$ Define $a + b$ and $ab$ to be the corresponding elements from $\mathbb{F}_{i + j}.$

    Moreover, each $\mathbb{F}_i$ is a subfield of $\mathbb{F}.$ \hfill\hyperref[lem:unionoffields2]{\downsym}
\end{restatable}

Note that the ``smallest'' above is just to ensure that the operations are well-defined. Since $\mathbb{F}_i \subset \mathbb{F}_j$ (note that we always use this to mean ``is a subfield of'') for $i \le j,$ we can actually pick any $i$ and $j.$

\begin{restatable}[Existence of Algebraic Closed Extension]{thm}{algclosedext}
\label{thm:algclosedext}
    Let $\mathbb{F}$ be a field. Then, there exists an algebraically closed field containing $\mathbb{F}.$ \hfill\hyperref[thm:algclosedext2]{\downsym}
\end{restatable}
The proof we have given is due to Artin.

\begin{restatable}[Existence of Algebraic Closure]{cor}{algclosure}
\label{cor:algclosure}
    Every field $\mathbb{F}$ has an algebraic closure. \hfill\hyperref[cor:algclosure2]{\downsym}
\end{restatable}

\section{Uniqueness}

\begin{restatable}[]{prop}{rootsandextensions}
\label{prop:rootsandextensions}
    Let $\sigma : \mathbb{F} \to \mathbb{L}$ be an embedding of fields where $\mathbb{L}$ is \underline{algebraically closed}. Let $\alpha \in \mathbb{K} \supset \mathbb{F}$ be algebraic over $\mathbb{F}$ and $p(x) = \irr(\alpha, \mathbb{F}).$ \\
    Write $p(x) = \sum a_i x^i$ and define $p^{\sigma}(x) \vcentcolon= \sum \sigma(a_i) x^i.$ Then, $\tau \mapsto \tau(\alpha)$ is a bijection between the sets
    \begin{equation*} 
        \{\tau : \mathbb{F}(\alpha) \to \mathbb{L} \mid \tau \text{ is an embedding and }\tau|_{\mathbb{F}} = \sigma\} \leftrightarrow \{\beta \in \mathbb{L} \mid p^{\sigma}(\beta) = 0\}.
    \end{equation*} \hfill\hyperref[prop:rootsandextensions2]{\downsym}
\end{restatable}

\begin{rem}
    The above proposition says that the number of ways to extend from $\mathbb{F}$ to $\mathbb{F}(\alpha)$ is precisely the number of roots of that $p(x)$ has in $\mathbb{L}.$ (Not exactly, we need to apply $\sigma$ to the coefficients. This is essentially saying that we consider $\mathbb{F}$ as a subfield under $\mathbb{L}.$) In particular, this set is non-empty since $\mathbb{L}$ is algebraically closed. \\
    Note that this number need not be $\deg(f(x)).$ We shall see in the next chapter that a polynomial may be irreducible but still have repeated roots in its splitting field.
\end{rem}

\begin{restatable}[]{thm}{extendtoalgextension}
\label{thm:extendtoalgextension}
    Let $\sigma : \mathbb{F} \to \mathbb{L}$ be an embedding where $\mathbb{L}$ is algebraically closed. Let $\mathbb{K}/\mathbb{F}$ be an algebraic extension. Then, there exists an embedding $\tau : \mathbb{K} \to \mathbb{L}$ extending $\sigma.$ \\
    Moreover, if $\mathbb{K}$ is an algebraic closure of $\mathbb{F}$ and $\mathbb{L}$ of $\sigma(\mathbb{K}),$ then $\tau$ is an isomorphism extending $\sigma.$ \hfill\hyperref[thm:extendtoalgextension2]{\downsym}
\end{restatable}

\begin{cor}[Isomorphism of algebraic closures]
    If $\mathbb{K}_1$ and $\mathbb{K}_2$ are two algebraic closures of $\mathbb{F},$ then they are $\mathbb{F}$-isomorphic.
\end{cor}
\begin{proof} 
    Apply previous proposition to the inclusion $i : \mathbb{F} \hookrightarrow \mathbb{E}_2$ to extend it to an $\mathbb{F}$-isomorphism $\tau : \mathbb{E}_1 \to \mathbb{E}_2.$
\end{proof}

\begin{defn}%[]
    Given a field $\mathbb{F},$ we use $\overline{\mathbb{F}}$ to denote an algebraic closure of $\mathbb{F}.$ 
\end{defn}

\begin{restatable}[Isomorphism of splitting fields]{thm}{isosplitting}
\label{thm:isosplitting}
    Let $\mathbb{E}$ and $\mathbb{E}'$ be two splitting fields of a non-constant polynomial $f(x) \in \mathbb{F}[x]$ over $\mathbb{F}.$ Then, they are $\mathbb{F}$-isomorphic. \hfill\hyperref[thm:isosplitting2]{\downsym}
\end{restatable}