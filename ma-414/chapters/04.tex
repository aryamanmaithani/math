\chapter{Separable extensions}

\section{Derivatives}

\begin{defn}%[]
    Let $\mathbb{F}$ be a field. Define the $\mathbb{F}$-linear map $\D_{\mathbb{F}} : \mathbb{F}[x] \to \mathbb{F}[x]$ by
    \begin{equation*} 
        \D_{\mathbb{F}}\left(\sum_{i = 0}^{n} a_ix^i\right) = \sum_{i = 1}^{n}ia_i x^{i - 1}.
    \end{equation*}
    Given $f(x) \in \mathbb{F}[x],$ we call $\D_{\mathbb{F}}(f(x))$ the \deff{derivative} of $f(x)$ and also denote it by $f'(x).$
\end{defn}

\begin{rem}
    Note that the above definition requires no notion of limits. For the case of $\mathbb{F} = \mathbb{R}$ or $\mathbb{C},$ then it coincides with the usual definition if we identify a polynomial with the function it represents. We shall not require this, however.
\end{rem}

We have the follow easy-to-check proposition.
\begin{prop}
    Let $f(x), g(x) \in \mathbb{F}[x]$ and $a \in \mathbb{F}$ be arbitrary. Then,
    \begin{enumerate}
        \item $(f \pm ag)'(x) = f'(x) \pm ag'(x),$
        \item $(fg)'(x) = f'(x)g(x) + f(x)g'(x).$
    \end{enumerate}
\end{prop}
The first point is just verifying that $\D_{\mathbb{F}}$ is indeed $\mathbb{F}$-linear.

\begin{prop}
    Let $\mathbb{F} \subset \mathbb{E}$ be a field extension. Then, $\D_{\mathbb{E}}|_{\mathbb{F}} = \D_{\mathbb{F}}.$ Thus, the notation $f'(x)$ is unambiguous.
\end{prop}

\begin{defn}%[]
    Let $f(x) \in \mathbb{F}[x]$ be a non-constant monic polynomial. Let $\mathbb{E}$ be a splitting field of $f(x)$ over $\mathbb{F}.$ In $\mathbb{E}[x],$ factorise $f(x)$ uniquely as
    \begin{equation*} 
        f(x) = (x - r_1)^{e_1} \cdots (x - r_g)^{e_g},
    \end{equation*}
    where $r_1, \ldots, r_g \in \mathbb{E}$ are distinct and each $e_i \in \mathbb{N}.$ 

    The numbers $e_1, \ldots, e_g$ are called the \deff{multiplicities} of the roots $r_1, \ldots, r_n.$ \\
    If $e_i = 1$ for some $i,$ then $r_i$ is called a \deff{simple root} and a \deff{repeated root} otherwise.

    If each $e_i = 1,$ then $f(x)$ is said to be a \deff{separable polynomial}.

    If $f$ is not monic, we have the same definitions upon division by the leading coefficient.
\end{defn}

\begin{rem}
    Note that the definition of ``separable polynomial'' is ad hoc since the separability presumably depends on the splitting field. However, in view of \Cref{rem:discrepeatedroots}, we see that separability depends only on $\disc(f(x)),$ which we have seen to be independent of the splitting field.\\
    The next proposition shows something even stronger.

    Also, note that one might think that an irreducible polynomial is always separable. We will see an example of how that is not true, in general. Over fields of characteristic $0,$ however, it is true. We shall prove that as well.
\end{rem}

\begin{restatable}[]{prop}{multindepsplitting}
\label{prop:multindepsplitting}
    The number of roots and their multiplicities are independent of the splitting field chosen for $f(x)$ over $\mathbb{F}.$ \hfill\hyperref[prop:multindepsplitting2]{\downsym}
\end{restatable}

\begin{restatable}[]{prop}{derivcritreproot}
\label{prop:derivcritreproot}
    Let $f(x) \in \mathbb{F}[x]$ be a monic and let $r \in \mathbb{E} \supset \mathbb{F}$ be a root of $f(x).$ \\
    Then, $r$ is a repeated root iff $f'(r) = 0.$ \hfill\hyperref[prop:derivcritreproot2]{\downsym}
\end{restatable}

\begin{restatable}[The Derivative Criterion for Separability]{thm}{derivcritsep}
\label{thm:derivcritsep}
    Let $f(x) \in \mathbb{F}[x]$ be a monic polynomial.
    \begin{enumerate}
         \item If $f'(x) = 0,$ then every root of $f(x)$ is a multiple root.
         \item If $f'(x) \neq 0,$ then $f(x)$ has simple roots iff $\gcd(f(x), f'(x)) = 1.$ \hfill\hyperref[thm:derivcritsep2]{\downsym}
     \end{enumerate} 
\end{restatable}

\begin{restatable}[]{prop}{irredsepderiv}
\label{prop:irredsepderiv}
    Let $f(x) \in \mathbb{F}[x]$ be irreducible and non-constant.
    \begin{enumerate}
         \item $f(x)$ is separable iff $f'(x) \neq 0.$
         \item If $\chr(\mathbb{F}) = 0,$ then $f(x)$ is separable. 
     \end{enumerate} 
     In other words, irreducible polynomials over fields of characteristic $0$ are separable. \hfill\hyperref[prop:irredsepderiv2]{\downsym}
\end{restatable}

\begin{ex}
    Let $p \in \mathbb{N}$ be a prime. Consider the field $\mathbb{F}_p(X)$ and the polynomial $f(T) = T^p - X \in \mathbb{F}_p(X)[T].$ \\
    Then, $f(T)$ is irreducible, by applying Eisenstein at the prime $X.$ However, $f'(T) = 0$ and hence, not separable.

    In fact, as we shall see, the existence of $p$-th roots will play an important role.
\end{ex}

\begin{defn}%[]
    Let $\mathbb{F}$ be a field of prime characteristic $p.$ Define
    \begin{equation*} 
        \mathbb{F}^p \vcentcolon= \{\alpha^p \in \mathbb{F} : \alpha \in \mathbb{F}\}.
    \end{equation*}
    That is, $\mathbb{F}^p$ is the set of all $p$-th powers of elements of $\mathbb{F}.$
\end{defn}

\begin{prop}
    $\mathbb{F}^p$ is a subfield of $\mathbb{F}.$
\end{prop}
\begin{proof} 
    Only closure under addition is not so obvious. Note that $(x + y)^p = x^p + y^p$ for all $x, y \in \mathbb{F}.$
\end{proof}

\begin{restatable}[]{prop}{xppolyirredorroot}
\label{prop:xppolyirredorroot}
    Let $\mathbb{F}$ be a field with $\chr(\mathbb{F}) = p > 0.$ Then, $x^p - a \in \mathbb{F}[x]$ is either irreducible in $\mathbb{F}[x]$ or $a \in \mathbb{F}^p.$ \hfill\hyperref[prop:xppolyirredorroot2]{\downsym}
\end{restatable}

\begin{restatable}[]{prop}{nonseppowerp}
\label{prop:nonseppowerp}
    Let $f(x) \in \mathbb{F}[x]$ be an irreducible polynomial and let $p \vcentcolon= \chr(\mathbb{F}) > 0.$ If $f(x)$ is not separable, then there exists $g(x) \in \mathbb{F}[x]$ such that $f(x) = g(x^p).$ \hfill\hyperref[prop:nonseppowerp2]{\downsym}
\end{restatable}

\section{Perfect fields}
\begin{defn}%[]
    Let $\mathbb{F} \subset \mathbb{K}$ be a field extension. 

    An algebraic element $\alpha \in \mathbb{K}$ over $\mathbb{F}$ is called a \deff{separable element over $\mathbb{F}$} if $\irr(\alpha, \mathbb{F})$ is separable over $\mathbb{F}.$

    We say that $\mathbb{K}/\mathbb{F}$ is a \deff{separable field extension} if every $\alpha \in \mathbb{K}$ is separable (and in particular, algebraic). 

    We say that $\mathbb{F}$ is a \deff{perfect field} if every algebraic extension of $\mathbb{F}$ is separable. Equivalently, every irreducible polynomial in $\mathbb{F}[x]$ is separable.
\end{defn}

\begin{ex}
    \phantom{hi}
    \begin{enumerate}
        \item We had seen that $\mathbb{F}_p(X)$ is not perfect for any prime $p.$
        \item By \Cref{prop:irredsepderiv}, we have that every field of characteristic $0$ is perfect.
    \end{enumerate}
\end{ex}

\begin{restatable}[]{thm}{perfectiffppower}
\label{thm:perfectiffppower}
    Let $\mathbb{F}$ be a field with characteristic $p > 0.$ Then, $\mathbb{F}$ is perfect iff $\mathbb{F} = \mathbb{F}^p.$ \hfill\hyperref[thm:perfectiffppower2]{\downsym}
\end{restatable}

\begin{restatable}[]{cor}{finitefieldperfect}
\label{cor:finitefieldperfect}
    Every finite field is perfect. \hfill\hyperref[cor:finitefieldperfect2]{\downsym}
\end{restatable}

\section{Extensions of embeddings}
\begin{restatable}[]{prop}{samemultirredpoly}
\label{prop:samemultirredpoly}
    Let $f(x) \in \mathbb{F}[x]$ be an irreducible monic polynomial. Then, all roots of $f(x)$ have equal multiplicity (in any splitting field). \\
    If $\chr(\mathbb{F}) = 0,$ then all roots are simple. \\
    If $\chr(\mathbb{F}) =\vcentcolon p > 0,$ then all roots have multiplicity $p^n$ for some $n \in \mathbb{N}_0.$ \hfill\hyperref[prop:samemultirredpoly2]{\downsym}
\end{restatable}
Note that by \Cref{prop:multindepsplitting}, the $n$ also does not depend on choice of splitting field.

\begin{restatable}[]{thm}{separabledegreedef}
\label{thm:separabledegreedef}
    Let $\sigma : \mathbb{F} \to \mathbb{L}$ be an embedding of fields where $\mathbb{L}$ is an algebraic closure of $\sigma(\mathbb{F}).$ Similarly, let $\tau : \mathbb{F} \to \mathbb{L}'$ be an embedding of fields where $\mathbb{L}'$ is an algebraic closure of $\tau(\mathbb{F}).$ Let $\mathbb{E}$ be an algebraic extension of $\mathbb{F}.$

    Let $S_\sigma$ (resp. $S_\tau$) denote the set of extensions of $\sigma$ (resp. $\tau$) to embeddings of $\mathbb{E}$ into $\mathbb{L}$ (resp. $\mathbb{L}'$). Let $\lambda : \mathbb{L} \to \mathbb{L}'$ be an isomorphism extending $\tau \circ \sigma^{-1} : \sigma(\mathbb{F}) \to \tau(\mathbb{F}).$  

    The map $\psi : S_\sigma \to S_\tau$ given by $\psi(\widetilde{\sigma}) = \lambda \circ \widetilde{\sigma}$ is a bijection. \hfill\hyperref[thm:separabledegreedef2]{\downsym}
\end{restatable}

\begin{center}
        \begin{tikzcd}
        \mathbb{L}' \arrow[dd, no head]                  &  &                                                                                                     &  & \mathbb{L} \arrow[llll, "\lambda"'] \arrow[dd, no head] \\
                                                         &  &                                                                                                     &  &                                                         \\
        \widetilde{\tau}(\mathbb{E}) \arrow[dd, no head] &  & \mathbb{E} \arrow[ll, "\widetilde{\tau} \in S_\tau"'] \arrow[rr, "\widetilde{\sigma} \in S_\sigma"] &  & \widetilde{\sigma}(\mathbb{E}) \arrow[dd, no head]      \\
                                                         &  &                                                                                                     &  &                                                         \\
        \tau(\mathbb{F})                                 &  & \mathbb{F} \arrow[rr, "\sigma"] \arrow[ll, "\tau"']                                                 &  & \sigma(\mathbb{F})                                     
        \end{tikzcd}
\end{center}  

\begin{rem}
    What the above proposition is really saying is that the ``number'' (cardinality) of extensions does not depend on $\mathbb{L}$ \textbf{\emph{or}} on the embedding $\sigma.$ Note that since $\mathbb{E}$ is an arbitrary algebraic extension, the set $S_\sigma$ need not be finite. 

    Thus, we may assume $\mathbb{L} \supset \mathbb{F}$ to be an algebraic closure and $\sigma$ to be the natural inclusion.
\end{rem}

\begin{defn}%[]
    If $\mathbb{E}/\mathbb{F}$ is an algebraic extension, then the cardinality of $S_\sigma$ (as in \Cref{thm:separabledegreedef}) is called the \deff{separable degree} of $\mathbb{E}/\mathbb{F}$ and is denoted $[\mathbb{E} : \mathbb{F}]_s.$
\end{defn}

\begin{rem}
    Note that if $\sigma : \mathbb{F} \to \mathbb{L}$ is an embedding into an algebraically closed field $\mathbb{L},$ and $\widetilde{\sigma} : \mathbb{E} \to \mathbb{L}$ is an extension of $\sigma,$ where $\mathbb{E}/\mathbb{F}$ is algebraic, then $\widetilde{\sigma}(\mathbb{E})$ is actually contained in the algebraic closure of $\sigma(\mathbb{F})$ within $\mathbb{L}.$ Thus, it is fine even if $\mathbb{L}$ is not an algebraic closure of $\sigma(\mathbb{F}).$
\end{rem}

\begin{prop} \label{prop:sepdeglessthannordeg}
    Let $\alpha \in \mathbb{E} \supset \mathbb{F}$ be algebraic over $\mathbb{F}$ and $n \vcentcolon= \deg(\irr(\alpha, \mathbb{F})).$ Then, $[\mathbb{F}(\alpha) : \mathbb{F}]_s \le n = [\mathbb{F}(a) : \mathbb{F}]$ with equality iff $\alpha$ is separable over $\mathbb{F}.$
\end{prop}
\begin{proof} 
    By \Cref{prop:rootsandextensions}, we know that $[\mathbb{F}(\alpha) : \mathbb{F}]_s$ is exactly the number of roots of $p(x) = \irr(\alpha, \mathbb{F})$ in $\overline{\mathbb{F}}.$ This is at most $n = \deg(p(x)).$ Moreover, equality implies that all roots are distinct and hence, $\alpha$ is separable.
\end{proof}

\begin{restatable}[Tower Law for separable degree]{thm}{towerlawsep}
\label{thm:towerlawsep}
    Let $\mathbb{F} \subset \mathbb{E} \subset \mathbb{K}$ be a tower of finite algebraic extensions. Then, $[\mathbb{E} : \mathbb{F}]_s \le [\mathbb{E} : \mathbb{F}]$ and
    \begin{equation*} 
        [\mathbb{K} : \mathbb{F}]_s = [\mathbb{K} : \mathbb{E}]_s [\mathbb{E} : \mathbb{F}]_s.
    \end{equation*} \hfill\hyperref[thm:towerlawsep2]{\downsym}
\end{restatable}

\begin{cor}
    Let $\mathbb{F} \subset \mathbb{E} \subset \mathbb{K}$ be a tower of finite algebraic extensions. Then, $[\mathbb{K} : \mathbb{F}] = [\mathbb{K} : \mathbb{F}]_s$ iff equality holds at each stage.
\end{cor}

\begin{restatable}[]{thm}{sepiffdegequal}
\label{thm:sepiffdegequal}
    Let $\mathbb{E}/\mathbb{F}$ be a finite extension. Then, $\mathbb{E}/\mathbb{F}$ is separable iff $[\mathbb{E} : \mathbb{F}]_s = [\mathbb{E} : \mathbb{F}].$ \hfill\hyperref[thm:sepiffdegequal2]{\downsym}
\end{restatable}

\begin{cor}
    Let $\alpha \in \mathbb{E} \supset \mathbb{F}$ be separable over $\mathbb{F}.$ Then, $\mathbb{F}(\alpha)/\mathbb{F}$ is a separable extension.
\end{cor}
\begin{proof}
    By \Cref{prop:sepdeglessthannordeg}, we have $[\mathbb{F}(\alpha) : \mathbb{F}]_s = [\mathbb{F}(\alpha) : \mathbb{F}].$ By \Cref{thm:sepiffdegequal}, this means that $\mathbb{F}(\alpha)/\mathbb{F}$ is separable.
\end{proof}

\begin{restatable}[]{prop}{compdecompsep}
\label{prop:compdecompsep}
    Let $\mathbb{F} \subset \mathbb{E} \subset \mathbb{K}$ be a tower of fields. Then,\\
    $\mathbb{K}/\mathbb{F}$ is separable iff $\mathbb{K}/\mathbb{E}$ and $\mathbb{E}/\mathbb{F}$ are separable. \hfill\hyperref[prop:compdecompsep2]{\downsym}
\end{restatable}

\begin{restatable}[]{prop}{sepdegdividesdeg}
\label{prop:sepdegdividesdeg}
    Let $\mathbb{E}/\mathbb{F}$ be a finite extension. Then, $[\mathbb{E} : \mathbb{F}]_s$ divides $[\mathbb{E} : \mathbb{F}].$ If $\chr(\mathbb{F}) =\vcentcolon p > 0,$ then quotient $\dfrac{[\mathbb{E} : \mathbb{F}]}{[\mathbb{E} : \mathbb{F}]_s}$ is a power of $p.$ \hfill\hyperref[prop:sepdegdividesdeg2]{\downsym}
\end{restatable}
