\chapter{Normal extensions}

\begin{defn}%[]
    An algebraic extension $\mathbb{E}/\mathbb{F}$ is called a \deff{normal extension} if whenever $f(x) \in \mathbb{F}[x]$ is irreducible and has a root in $\mathbb{E},$ then $f(x)$ splits into linear factors in $\mathbb{E}[x].$ 
\end{defn}

\begin{defn}%[]
    Let $\mathbb{E}/\mathbb{F}$ be an extension and $\mathcal{F} = \{f_i(x)\}_{i \in I}$ be a (possibly infinite) family of non-constant polynomials in $\mathbb{F}[x].$ Then, $\mathbb{E}$ is said to be a \deff{splitting field for the family $\mathcal{F}$ over $\mathbb{F}$} if each $f_i(x)$ splits as a product of linear factors in $\mathbb{E}[x]$ and is generated by the roots of the polynomials.
\end{defn}

\begin{rem} \label{rem:splitfamilyexists}
    Note that a splitting field of any family always exists, since an algebraic closure always exists. So, we consider $A \subset \overline{\mathbb{F}}$ to be the set of roots of all the polynomials of the family $\mathcal{F}$ and then put $\mathbb{E} \vcentcolon= \mathbb{F}(A) \subset \overline{\mathbb{F}}.$
\end{rem}

\begin{restatable}[]{prop}{seppolysplittingfields}
\label{prop:seppolysplittingfields}
    Let $\mathbb{F}$ be a field, and $\mathcal{F} \subset \mathbb{F}[x]$ be a family of separable polynomials. Let $\mathbb{E} \subset \overline{\mathbb{F}}$ be the splitting field of $\mathcal{F}$ over $\mathbb{F}.$ Then, $\mathbb{E}/\mathbb{F}$ is a separable extension. \hfill\hyperref[prop:seppolysplittingfields2]{\downsym}
\end{restatable}

\begin{restatable}[]{lem}{algebraicautomorphism}
\label{lem:algebraicautomorphism}
    Let $\mathbb{E}/\mathbb{F}$ be an algebraic extension. Let $\sigma : \mathbb{E} \to \mathbb{E}$ be an $\mathbb{F}$-embedding. Then, $\sigma$ is an automorphism of $\mathbb{E}.$ \hfill\hyperref[lem:algebraicautomorphism2]{\downsym}
\end{restatable}

\begin{restatable}[]{thm}{normalequivalent}
\label{thm:normalequivalent}
    Let $\mathbb{F}$ be a field and fix an algebraic closure $\overline{\mathbb{F}}$ of $\mathbb{F}.$ Let $\mathbb{F} \subset \mathbb{E} \subset \overline{\mathbb{F}}$ be fields. Then, the following are equivalent:
    \begin{enumerate}
         \item \label{item:001} Every $\mathbb{F}$-embedding $\sigma : \mathbb{E} \to \overline{\mathbb{F}}$ is an automorphism of $\mathbb{E}.$
         \item \label{item:002} $\mathbb{E}$ is a splitting field of a family of polynomials in $\mathbb{F}[x].$
         \item \label{item:003} $\mathbb{E}/\mathbb{F}$ is a normal extension. \hfill\hyperref[thm:normalequivalent2]{\downsym}
     \end{enumerate} 
\end{restatable}

\begin{restatable}[]{prop}{operationsonnormalexts}
\label{prop:operationsonnormalexts}
    Let $\mathbb{F} \subset \mathbb{E}_1, \mathbb{E}_2 \subset \mathbb{K}$ be fields. Suppose that $\mathbb{E}_i/\mathbb{F}$ are normal. Then, so are $\mathbb{E}_1\mathbb{E}_2/\mathbb{F}$ and $(\mathbb{E}_1 \cap \mathbb{E}_2)/\mathbb{F}.$ \hfill\hyperref[prop:operationsonnormalexts2]{\downsym}
\end{restatable}

\begin{ex}
    Quadratic extensions are always normal. Indeed, pick $\alpha \in \mathbb{E} \setminus \mathbb{F}.$ Then, $\mathbb{E} = \mathbb{F}(\alpha)$ is a splitting field of $\irr(\alpha, \mathbb{F})$ over $\mathbb{F}.$
\end{ex}

\begin{rem}
    Unlike the ``tower laws'' for algebraic and separable extensions, the ``composition'' of normal extensions need not be normal. For example, consider the chain
    \begin{equation*} 
        \mathbb{Q} \subset \mathbb{Q}(\sqrt{2}) \subset \mathbb{Q}(\sqrt[4]{2}).
    \end{equation*}
    Each successive extension is quadratic and hence, normal. However, $\mathbb{Q}(\sqrt[4]{2})/\mathbb{Q}$ is not normal since the irreducible (via Eisenstein) polynomial $x^4 - 2 \in \mathbb{Q}[x]$ has a root in $\mathbb{Q}(\sqrt[4]{2})$ but does not factor completely.

    On the other hand, consider
    \begin{equation*} 
        \mathbb{Q} \subset \mathbb{Q}(\sqrt[4]{2}) \subset \mathbb{Q}(\sqrt[4]{2}, \iota).
    \end{equation*}
    Then, $\mathbb{Q}(\sqrt[4]{2}, \iota)/\mathbb{Q}$ is normal since $(\sqrt[4]{2}, \iota)$ is the splitting field for $x^4 - 2$ over $\mathbb{Q}$ but $(\sqrt[4]{2})/\mathbb{Q}$ is not.
\end{rem}

However, one part of the ``tower property'' \emph{does} hold, as can be easily verified, either directly from the definition or using one of the equivalences proven above.

\begin{prop} \label{prop:decompnormal}
    Let $\mathbb{F} \subset \mathbb{E} \subset \mathbb{K}$ be fields such that $\mathbb{K}/\mathbb{F}$ is normal. Then, $\mathbb{K}/\mathbb{E}$ is normal.
\end{prop}

\begin{rem}
    The above phenomenon is related (at least in the case of finite extensions) to the phenomenon that ``is a normal subgroup'' is not transitive either. Given groups $H \le K \le G,$ it is possible that $H$ is normal in $K$ and $K$ in $G$ but $H$ is not normal in $G.$ 

    Similarly, if we know that $H$ is normal in $G,$ then we can conclude that $H$ is normal in $K$ but $K$ need not be normal in $G.$
\end{rem}