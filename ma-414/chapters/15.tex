\chapter{Galois Groups of Quartic Polynomials}

\section{Galois group as a group of permutations}

In this chapter, we shall frequently consider the Galois group of a separable polynomial of degree $n$ as a subgroup of $S_n.$ To recall how this is done: Let $f(x) \in \mathbb{F}[x]$ be a monic separable polynomial with (distinct) roots $r_1, \ldots, r_n \in \overline{\mathbb{F}}$ in a splitting field $\mathbb{E} = \mathbb{F}(r_1, \ldots, r_n).$ Let $G \vcentcolon= \Gal(\mathbb{E}/\mathbb{F})$ be its Galois group. Note that any $\sigma \in G$ is a permutation of $R = \{r_1, \ldots, r_n\}.$ Identifying $R$ with $[n],$ we see that $\sigma|_R \in S_n.$ \\
Define $\psi : G \to S_n$ by $\sigma \mapsto \sigma|_R.$ This is an injective homomorphism since $\sigma$ is completely determined by its action on $R$ since $\mathbb{E} = \mathbb{F}(R).$ We denote the image of $\psi$ by $\G_f,$ the Galois group of $f(x).$

By \hyperref[thm:FTGT]{FTGT}, there is an intermediate subfield of $\mathbb{E}/\mathbb{F}$ corresponding to $\G_f \cap A_n.$

\begin{restatable}[]{thm}{alternatingsubgroupdiscriminantroot}
\label{thm:alternatingsubgroupdiscriminantroot}
	Let $\mathbb{F}$ be a field with $\chr(\mathbb{F}) \neq 2$ and $f(x) \in \mathbb{F}[x],$ a monic separable polynomial with (distinct) roots $r_1, \ldots, r_n \in \overline{\mathbb{F}}.$ Put $\mathbb{E} = \mathbb{F}(r_1, \ldots, r_n)$ and 
	\begin{equation*} 
		\delta = \prod_{1 \le i < j \le n} (r_i - r_j).
	\end{equation*}
	Then, $E^{\G_f \cap A_n} = \mathbb{F}(\delta).$ \hfill\hyperref[thm:alternatingsubgroupdiscriminantroot2]{\downsym}
\end{restatable}

\begin{defn}
	A subgroup $H \le S_n$ is called a \deff{transitive subgroup} if $H$ acts transitively on $\{1, \ldots, n\}.$ \\
	In other words, given any $i, j \in \{1, \ldots, n\},$ there exists $\sigma \in H$ with $\sigma(i) = j.$
\end{defn}

\begin{restatable}[]{thm}{transitivegaloisgroupiffirreducible}
\label{thm:transitivegaloisgroupiffirreducible}
	Let $f(x) \in \mathbb{F}[x]$ be a separable polynomial of degree $n.$ Then, $f(x)$ is irreducible if and only if $\G_f$ is a transitive subgroup of $S_n.$ \hfill\hyperref[thm:transitivegaloisgroupiffirreducible2]{\downsym}
\end{restatable}

\section{Transitive subgroups of \texorpdfstring{$S_4$}{S4}}

Let $H \le S_n$ be a transitive subgroup. Then, there is only one orbit of $H$ on $[n].$ In particular, this orbit has order $n.$ By the orbit-stabiliser theorem, it follows that $n \mid \md{H}.$

By \Cref{thm:transitivegaloisgroupiffirreducible}, the orders of possible Galois groups of irreducible separable quartics are $4,$ $8,$ $12,$ and $24.$ These groups are listed below.

\begin{enumerate}
	\item Isomorphic to $C_4.$ \\
	These are the groups generated by an element of order $4.$ Since we are in $S_4,$ these are the groups generated by a $4$-cycle. There are six $4$-cycles in $S_4$ and in turn, there are three subgroups of $S_4$ isomorphic to $C_4.$
	%
	\item Isomorphic to $V,$ the Klein-$4$ group. \\
	This must contain three elements of order $2.$ Thus, it is forced to be 
	\begin{equation*} 
		V = \{(1), (12)(34), (13)(24), (14)(23)\}.
	\end{equation*}
	Looking at the cycle types, we see that $V \unlhd S_4.$
	%
	\item Order $8.$ This is a Sylow $2$-subgroup of $S_4$ and thus, all of these are isomorphic. The isomorphism type turns out to be that of $D_8.$ \\
	These are $H_1 = \langle V, (12)\rangle,$ $H_2 = \langle V, (13)\rangle,$ and $H_3 = \langle V, (14)\rangle.$
	%
	\item $A_4$ is the only subgroup of order $12$ in $S_4$ and $A_4 \unlhd S_4.$
	%
	\item $S_4$ is the only subgroup of order $24$ in $S_4.$
\end{enumerate}

\section{Calculation of Galois group of quartic polynomials}

Let $\mathbb{F}$ be a field of characteristic not $2.$ Let $f(x) = x^4 + b_1x^3 + b_2x^2 + b_3x + b_4 \in \mathbb{F}[x]$ be separable. By the change $x' = x + \frac{b_1}{4},$ we may assume that there is no $x^3$ term. This change only changes the roots of $f(x)$ by addition of a constant. Thus, the discriminant is unchanged. Moreover, the constant is in $\mathbb{F}$ and thus, the splitting field is unchanged and hence, so is the Galois group. \\
So, let $f(x) = x^4 + bx^2 + cx + d \in \mathbb{F}[x]$ be a separable polynomial with roots $r_1, \ldots, r_4$ in a splitting field $\mathbb{E}$ of $f(x)$ over $\mathbb{F}.$ As before, we consider $\G_f \le S_4.$ Set
\begin{equation*} 
	\underline{t} \vcentcolon= \{t_1 = r_1r_2 + r_3r_4,\, t_2 = r_1r_3 + r_2r_4,\, t_3 = r_1r_4 + r_2r_3\}.
\end{equation*}

\begin{defn}
	The monic cubic having $t_1, t_2, t_3$ as roots is called the \deff{resolvent} of $f(x).$
\end{defn}
\begin{rem}
	We had defined resolvent in \Cref{chap:solutionscubicandquartics} in a different manner. For this chapter, we shall use the above definition. 

	As earlier, it can be shown that the resolvent is actually an element of $\mathbb{F}[x]$ and is explicitly given as
	\begin{equation*} 
		x^3 - bx^2 + 4dx + 2bd - c^2.
	\end{equation*}

	By computing the differences $t_i - t_j,$ it is also clear that the $f(x)$ has the same discriminant as its resolvent.
\end{rem}

Also, recall that there is a unique subgroup of $S_4$ isomorphic to the Klein-$4$ group. We denote it by $V.$ Moreover, $V \unlhd S_4.$ It is also visible that $V$ fixes each element of $\underline{t}.$

Lastly, define as before $H_1 = \langle V, (12)\rangle,$ $H_2 = \langle V, (13)\rangle,$ and $H_3 = \langle V, (14)\rangle.$

\begin{restatable}[]{prop}{stabiliserofti}
\label{prop:stabiliserofti}
	$\Stab t_i = H_i.$ \hfill\hyperref[prop:stabiliserofti2]{\downsym}
\end{restatable}

\begin{restatable}[]{prop}{galoisintersectklein}
\label{prop:galoisintersectklein}
	$\mathbb{E}^{\G_f \cap V} = \mathbb{F}(\underline{t})$ and $\Gal(\mathbb{F}(\underline{t})/\mathbb{F}) = \G_f/\G_f \cap V.$ \hfill\hyperref[prop:galoisintersectklein2]{\downsym}
\end{restatable}

\begin{restatable}[]{prop}{resolventquarticrootinF}
\label{prop:resolventquarticrootinF}
	The resolvent cubic of a separable quartic has a root in $\mathbb{F}$ if and only if $\G_f \subset H_i$ for some $i.$ \hfill\hyperref[prop:resolventquarticrootinF2]{\downsym}
\end{restatable}

\begin{restatable}[]{thm}{classifyingirreduciblequarticgalois}
\label{thm:classifyingirreduciblequarticgalois}
	Let $f(x) \in \mathbb{F}[x]$ an irreducible separable quartic with $\chr(\mathbb{F}) \neq 2.$ Let $r(x)$ denote the resolvent cubic of $f(x).$
	\begin{enumerate}
		\item If $r(x)$ is irreducible in $\mathbb{F}[x]$ and $\disc(f(x)) \notin \mathbb{F}^2,$ then $\G_f \cong S_4.$
		\item If $r(x)$ is irreducible in $\mathbb{F}[x]$ and $\disc(f(x)) \in \mathbb{F}^2,$ then $\G_f \cong A_4.$
		\item If $r(x)$ splits completely in $\mathbb{F}[x],$ then $\G_f \cong V.$
		\item Suppose $r(x)$ has exactly one root in $\mathbb{F}.$ 
		\begin{enumerate}
			\item If $f(x)$ is irreducible in $\mathbb{F}(\underline{t})[x],$ then $\G_f \cong D_8.$
			\item If $f(x)$ is reducible in $\mathbb{F}(\underline{t})[x],$ then $\G_f \cong C_4.$ \hfill\hyperref[thm:classifyingirreduciblequarticgalois2]{\downsym}
		\end{enumerate}
	\end{enumerate}
\end{restatable}

\begin{ex}
	Let us now show that all the above possibilities do happen over $\mathbb{F} = \mathbb{Q}.$
	\begin{enumerate}
		\item ($\G_f = C_4$) Let $f(x) = x^4 + 5x^2 + 5.$ Then,
		\begin{equation*} 
			r(x) = x^3 - 5x^2 - 20x + 100 = (x - 5)(x - 2\sqrt{5})(x + 2\sqrt{5}).
		\end{equation*}
		Thus, $\mathbb{F}(\underline{t}) = \mathbb{Q}(\sqrt{5}).$ $f(x)$ is irreducible over $\mathbb{Q},$ by Eisenstein but not over $\mathbb{F}(\underline{t})$ as seen by
		\begin{equation*} 
			f(x) = \left(x^2 + \frac{5 + \sqrt{5}}{2}\right)\left(x^2 - \frac{5 - \sqrt{5}}{2}\right).
		\end{equation*}
		Thus, $\G_f \cong C_4.$
		%
		\item ($\G_f = V$) Let $f(x) = x^4 + 1 \in \mathbb{Q}[x].$ Then, the resolvent is $r(x) = x(x - 2)(x + 2).$ Thus, $\G_f = V.$
		%
		\item ($\G_f = D_8$) Let $f(x) = x^4 - 3.$ Then,
		\begin{equation*} 
			r(x) = x(x + 2\iota\sqrt{3})(x - 2\iota\sqrt{3}).
		\end{equation*}
		Thus, $\mathbb{F}(\underline{t}) = \mathbb{Q}(\iota\sqrt{3}).$ Note that $f(x)$ factors in $\overline{\mathbb{Q}}$ as
		\begin{equation*} 
			f(x) = (x - \iota\sqrt[4]{3})(x + \iota\sqrt[4]{3})(x - \sqrt[4]{3})(x + \sqrt[4]{3}).
		\end{equation*}
		Thus, $f(x)$ has no root in $\mathbb{F}(\underline{t})$ but is irreducible over $\mathbb{Q}$ and thus, $\G_f \cong D_8.$
		%
		\item ($\G_f = A_4$) Let $f(x) = x^4 - 8x + 12.$ Then, $r(x) = x^3 - 48x - 64.$ By the rational root test, we see that $r(x)$ has no roots in $\mathbb{Q}$ and hence, is irreducible. Moreover, so is $f(x),$ by Eisenstein. Now, $\disc(f(x)) = \disc(r(x)) = 2^{12}3^4$ is a square in $\mathbb{Q}$ and thus, $\G_f = A_4.$
		%
		\item ($\G_f = S_4$) Let $f(x) = x^4 - x + 1.$ Then, $r(x) = x^3 - 4x - 1.$ Both are irreducible over $\mathbb{Q}.$ (For $f(x),$ go modulo $2$ and for $r(x),$ use the rational root test.) Now, $\disc(f(x)) = \disc(r(x)) = 229 \notin \mathbb{Q}^2$ and thus, $\G_f \cong S_4.$
	\end{enumerate}
\end{ex}