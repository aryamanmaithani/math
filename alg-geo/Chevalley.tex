\documentclass[dvipsnames]{beamer}
\mode<presentation>{}
\usepackage[utf8]{inputenc}
\usepackage{amsmath, amssymb, amsfonts, amsthm, mathtools, mathrsfs}
\setbeamertemplate{theorems}[numbered]
\title{Morphisms of Schemes: Chevalley's Theorem}
\author[Aryaman Maithani]{\texorpdfstring{Aryaman Maithani\\Mentor: Prof. Arvind Nair}{Aryaman Maithani}}
\date[14-06-2021]{June 14, 2021}
% \institute[IITB]{\texorpdfstring{Department of Mathematics\\IIT Bombay}{IIT Bombay}}
\usetheme{Warsaw}
\usepackage{parskip}
\usepackage{tcolorbox}
\usepackage{tikz-cd}
\usepackage{commands}
\tikzset{
    invisible/.style={opacity=0},
    visible on/.style={alt={#1{}{invisible}}},
    alt/.code args={<#1>#2#3}{%
      \alt<#1>{\pgfkeysalso{#2}}{\pgfkeysalso{#3}}%
  }
}
\setbeamercolor{footline}{fg=blue}
\setbeamerfont{footline}{series=\bfseries}
\addtobeamertemplate{navigation symbols}{}{%
    \usebeamerfont{footline}%
    \usebeamercolor[fg]{footline}%
    \hspace{1em}%
    \insertframenumber/\inserttotalframenumber
}
% \usepackage{tikz}
% \usecolortheme{beetle}
% \usepackage{graphicx}

\theoremstyle{definition}
\newtheorem{thm}{Theorem}
\newtheorem{defn}[thm]{Definition}
\newtheorem{prop}[thm]{Proposition}
\newtheorem{cor}[thm]{Corollary}
\newtheorem{caution}[thm]{Caution}
\newtheorem{slogan}[thm]{Slogan}
\newtheorem{ex}[thm]{Example}
\begin{document}
\begin{frame}
    \titlepage
\end{frame}
\begin{frame}{Notations}
    \begin{enumerate}
        \uncover<2->{\item $X$ and $Y$ will denote topological spaces. }
        \uncover<3->{\item $U, V, W$ will denote open subsets of the ambient topological space. }
        \uncover<4->{\item By a cover $\{U_i\}$ of $U$, we mean that $U = \bigcup_{i} U_i.$ }\uncover<5->{In particular, $U_i \subset U$ for all $i.$ }
        \uncover<6->{\item $A$ and $B$ will denote a commutative ring with $1.$ (All our rings will be of this form!) }
        \uncover<7->{\item $\Spec A$ will denote the set of prime ideals of $A.$ }
        \uncover<8->{\item Given $S \subset A$, $\langle S\rangle$ will denote the ideal generated by $S.$ }
        \uncover<9->{\item Given $f \in A$, $A_f$ will denote the localisation of $A$ at the multiplicative set $\{1, f, f^2, \ldots\}.$ }
    \end{enumerate}
\end{frame}
\begin{frame}{Presheaves}
    \begin{defn}[Presheaf]
        \uncover<2->{Let $X$ be a topological space. }\uncover<3->{A \deff{presheaf (of rings) $\mathscr{F}$ on $X$} is the following collection of data: }\\
        \begin{enumerate}
            \uncover<4->{\item For each open set $U \subset X$, we are given a ring $\mathscr{F}(U).$ }
            \uncover<5->{\item For open sets $U \subset V \subset X$, we have a ring map $\res_{V, U} : \mathscr{F}(V) \to \mathscr{F}(U)$, called the \deff{restriction map}.}
        \end{enumerate}
        \uncover<6->{The above data is required to satisfy the following conditions: }
        \begin{enumerate}
            \uncover<7->{\item $\res_{U, U} = \id_{\mathscr{F}(U)}$ for all open $U \subset X.$ }
            \uncover<8->{\item If $U \subset V \subset W$ are open sets, then the following diagram commutes }
            \uncover<9->{
            \begin{equation*} 
                \begin{tikzcd}[ampersand replacement=\&]
                    \mathscr{F}(W) \arrow[visible on=<10->, rr, "\res_{W, V}"] \arrow[visible on=<12->, rrdd, "\res_{W, U}"'] \& \& |[visible on=<10->]|{\mathscr{F}(V)} \arrow[visible on=<11->, dd, "\res_{V, U}"] \\
                    \& \& \\
                    \& \& |[visible on=<11->]|{\mathscr{F}(U)}
                \end{tikzcd}\uncover<12->{. }
            \end{equation*} }
        \end{enumerate}
    \end{defn}
\end{frame}
\begin{frame}{Sheaves}
    \begin{defn}[Sheaf]
        \uncover<2->{Let $X$ be a topological space. }\uncover<3->{A \deff{sheaf (of rings) $\mathscr{F}$ on $X$} }\uncover<4->{is a presheaf $\mathscr{F}$ on $X$ satisfying the following: }

        \uncover<5->{Given an open set $U \subset X$, }\uncover<6->{an open cover $\{U_i\}$ of $U$, }\uncover<7->{and elements $f_i \in \mathscr{F}(U_i)$ }\uncover<8->{such that $\res_{U_{i}, U_{i} \cap U_{j}}(f_{i}) = \res_{U_{j}, U_{i} \cap U_{j}}(f_{j})$ for all $i, j$, }\uncover<9->{there {\color<13->{purple}exists} a {\color<14->{purple}unique} $f \in \mathscr{F}(U)$ }\uncover<10->{such that 
        \begin{equation*} 
            \res_{U, U_i}(f) = f_i
        \end{equation*} }\uncover<11->{for all $i.$ }
    \end{defn}

    \uncover<12->{
    \begin{slogan}
        Given elements on patches which are compatible, we {\color<13->{purple}can} glue them {\color<14->{purple}uniquely}.
    \end{slogan} }
\end{frame}
\begin{frame}{Ringed spaces}
    \begin{defn}[Ringed space]
        \uncover<2->{A \deff{ringed space} is a tuple $(X, \mathscr{O}_X)$, }\uncover<3->{where $X$ is a topological space and $\mathscr{O}_X$ is a sheaf on $X.$ }
    \end{defn}
    % \uncover<3->{\begin{ex}
    %     \uncover<4->{Let $X$ be your favourite topological space. }\uncover<5->{Consider the sheaf $\mathscr{O}_X$ of \emph{real continuous functions}. }\uncover<6->{That is, given an open $U \subset X$, $\mathscr{O}_X(U)$ is the ring of continuous functions $U \to X.$ }\uncover<7->{The restriction map $\res_{V, U}$ is the literal restriction, }\uncover<8->{$f \mapsto f|_{U}.$ }

    %     \uncover<9->{This is a sheaf, thanks to the pasting lemma. }
    % \end{ex} }
% \end{frame}
% \begin{frame}{Morphisms of ringed spaces}
    \begin{defn}[Morphism of ringed spaces]
        \uncover<4->{Let $(X, \mathscr{O}_X)$ and $(Y, \mathscr{O}_Y)$ be ringed spaces. }\uncover<5->{A \deff{morphism} $\pi : (X, \mathscr{O}_X) \to (Y, \mathscr{O}_Y)$ is the following data: }
        \begin{enumerate}
            \uncover<6->{\item A continuous map $\pi : X \to Y.$ }
            \uncover<7->{\item For every open $V \subset Y$, we have a ring map }\uncover<8->{
            \vspace{-2mm}
            \begin{equation*} 
                \mathscr{O}_Y(V) \to \mathscr{O}_X(\pi^{-1}(V)).
            \end{equation*} }
        \end{enumerate}
        \vspace{-8mm}
        \uncover<9->{Moreover, the ``obvious diagrams'' must commute. }
    \end{defn}
\end{frame}
\begin{frame}{Zariski topology}
    \uncover<1->{Goal: Turn $\Spec A$ into a ringed space. }\uncover<2->{First, we need a topology. }
    \begin{defn}[Distinguished and Vanishing sets]
        \uncover<3->{Let $A$ be a ring, }\uncover<4->{and $f \in A.$ }\uncover<5->{Define 
        \begin{equation*} 
            D(f) \vcentcolon= \{\mathfrak{p} \in \Spec A : f \notin \mathfrak{p}\}.
        \end{equation*} }
        \uncover<6->{Given a subset $S \subset A$, define }\uncover<7->{
        \begin{equation*} 
            V(S) \vcentcolon= \{\mathfrak{p} \in \Spec A : S \subset \mathfrak{p} \}.
        \end{equation*} }
        \uncover<8->{(Check: $D(f) = \Spec A \setminus V(f).$) }
    \end{defn}
    \uncover<9->{Simple check 1: Given $S \subset A$, we have $V(S) = V(\langle S\rangle).$ }\\
    \uncover<10->{Simple check 2: If $D(g) \subset D(f)$, then $f$ is invertible in $A_g.$ }\uncover<11->{Thus, there is a natural map $A_f \to A_g.$ }
\end{frame}
\begin{frame}{Zariski topology}
    \begin{defn}[Zariski topology]
        \uncover<2->{Let $A$ be a ring. }\uncover<3->{Then, the collection
        \begin{equation*} 
            \{V(I) : I \subset A \text{ is an ideal}\}
        \end{equation*} }\uncover<4->{describes a topology on $\Spec A$ by denoting the collection of \emph{closed} subsets. }\uncover<5->{This is called the \deff{Zariski topology} on $\Spec A.$ }
    \end{defn}
    \begin{prop}[A basis for the Zariski topology]
        \uncover<6->{The collection $\{D(f) : f \in A\}$ forms a basis for the above topology. }
    \end{prop}
\end{frame}
\begin{frame}{A {\color{brown}Helper} Example}
    \uncover<2->{Let $k$ be a field. }\uncover<3->{We denote $\Spec k[x]$ by $\mathbb{A}_{k}^{1}.$ }

    \uncover<4->{Since $k[x]$ is a PID, }\uncover<5->{the prime ideals are $\langle 0\rangle$ and the maximal ideals. }

    \uncover<6->{The set $\{\langle 0\rangle\}$ is dense in $\mathbb{A}_{k}^{1}.$ }

    \uncover<7->{The closed sets are given precisely as: }
    \begin{enumerate}
        \uncover<8->{\item The empty set. }
        \uncover<9->{\item The whole space. }
        \uncover<10->{\item Sets containing finitely many maximal ideals. }
    \end{enumerate}
    \uncover<11->{In particular, maximal ideals are \emph{closed points}, }\uncover<12->{i.e., $\{\mathfrak{m}\}$ is closed. }\uncover<13->{Consequently, $\{\mathfrak{m}\}$ is not dense in $\mathbb{A}_{k}^{1}.$ }

    \uncover<14->{To conclude, the only dense singleton subset of $\mathbb{A}_{k}^{1}$ is $\{\langle 0\rangle\}.$ }
\end{frame}
\begin{frame}{Structure sheaf}
    \uncover<2->{We now describe a sheaf $\mathscr{O}_{\Spec A}.$ }\uncover<3->{However, we shall cheat a bit. }\uncover<4->{We only define the objects and arrows on the level of basis elements. }\uncover<5->{One must check that this does indeed a sheaf on the whole space. }
    \begin{defn}[Structure sheaf]
        \uncover<6->{Let $A$ be a ring. }\uncover<7->{Given $f \in A$, we define
        \begin{equation*} 
            \mathscr{O}_{\Spec A}(D(f)) \vcentcolon= A_{f}.
        \end{equation*} }
        \uncover<8->{Given $D(g) \subset D(f)$, }\uncover<9->{the restriction map is the natural map $A_f \to A_g.$ }

        \uncover<10->{This is called the \deff{structure sheaf} on $\Spec A$. }
    \end{defn}
\end{frame}
\begin{frame}{Schemes}
    \begin{defn}[Affine scheme]
        \uncover<2->{An \deff{affine scheme} }\uncover<3->{is a ringed space }\uncover<4->{which is isomorphic to some }\uncover<5->{$(\Spec A, \mathscr{O}_{\Spec A}).$ }
    \end{defn}
    \begin{defn}[Scheme]
        \uncover<6->{A \deff{scheme} }\uncover<7->{is a ringed space $(X, \mathscr{O}_X)$ }\uncover<8->{such that every $p \in X$ }\uncover<9->{has an open neighbourhood $U$ }\uncover<10->{such that $(U, \mathscr{O}_X|_{U})$ is an affine scheme. }
    \end{defn}
    \uncover<11->{
    \begin{slogan}
        A scheme can be covered by \deff{affine opens}.
    \end{slogan}}
    \uncover<12->{In fact, (it follows that) the affine opens form a basis for $X.$ }
\end{frame}
\begin{frame}{Morphisms of affine schemes}
    \uncover<2->{Let $\pi^{\sharp} : A \to B$ a map of rings. }\uncover<3->{This induces a map $\pi : \Spec B \to \Spec A$ }\uncover<4->{given by $\mathfrak{p} \mapsto (\pi^{\sharp})^{-1}(\mathfrak{p}).$ }\uncover<5->{This is continuous. }

    \uncover<6->{Moreover, this also induces a morphism of ringed spaces. }\uncover<7->{More explicitly, given $f \in A$, }\uncover<8->{we have the map
    \begin{equation*} 
        \begin{tikzcd}[ampersand replacement=\&]
        \mathscr{O}_{\Spec A}(D(f)) \arrow[r] \arrow[visible on=<10->, d, equals] \& \mathscr{O}_{\Spec B}(\pi^{-1}(D(f))) \arrow[visible on=<9->, r, equals] \& |[visible on=<9->]|{\mathscr{O}_{\Spec B}(D(\pi^{\sharp}f))} \arrow[visible on=<10->, d, equals] \\
        |[visible on=<10->]|{A_f} \arrow[visible on=<11->, rr] \& \& |[visible on=<10->]|{B_{\pi^{\sharp}f}} 
        \end{tikzcd}\uncover<11->{. }
    \end{equation*} }
    \uncover<12->{The above is a \deff{morphism of affine schemes}. }\uncover<13->{That is, a morphism of affine schemes is a morphism of ringed spaces that is induced by some ring map as above. }
\end{frame}
\begin{frame}{Morphisms of schemes}
    \begin{defn}[Morphism of schemes]
        \uncover<2->{A \deff{morphism of schemes} $\pi : (X, \mathscr{O}_X) \to (Y, \mathscr{O}_Y)$ }\uncover<3->{is a morphism of ringed spaces that ``locally looks like'' a morphism of affine schemes. }

        \uncover<4->{More precisely, for each choice of affine open sets $\Spec A \subset X$, $\Spec B \subset Y$, }\uncover<5->{such that $\pi(\Spec A) \subset \Spec B$, }\uncover<6->{the restricted morphism is one of affine schemes. }
    \end{defn}
\end{frame}
\begin{frame}{Some definitions}
    \begin{defn}[Compact morphism]
        \uncover<2->{A morphism $\pi : (X, \mathscr{O}_X) \to (Y, \mathscr{O}_Y)$ of schemes is \deff{compact} }\uncover<3->{if the preimage of any compact open subset is compact. }
    \end{defn}
    \begin{defn}[Finite type morphism]
        \uncover<4->{A \emph{compact} morphism $\pi : (X, \mathscr{O}_X) \to (Y, \mathscr{O}_Y)$ of schemes is \deff{of finite type} }\uncover<5->{if for every affine open $\Spec B \subset Y$, }\uncover<6->{$\pi^{-1}(\Spec B)$ can be covered by affine open subsets $\Spec A_i$, }\uncover<7->{so that each $A_i$ is a finitely generated $B$-algebra. }
    \end{defn}
    \begin{defn}[Noetherian schemes]
        \uncover<8->{A scheme $(X, \mathscr{O}_X)$ is said to be \deff{Noetherian} }\uncover<9->{if $X$ can be covered by finitely many affine opens $\Spec A_i$ }\uncover<10->{such that each $A_i$ is a Noetherian ring. } 
    \end{defn}
\end{frame}
\begin{frame}{Some topology}
    \begin{defn}[Locally closed set]
        \uncover<2->{A subset of a topological space $X$ is said to be \deff{locally closed} }\uncover<3->{if it is the intersection of an open subset and a closed subset. }
    \end{defn}
    \begin{defn}[Constructible set]
        \uncover<4->{A subset of a topological space $X$ is said to be \deff{constructible} }\uncover<5->{if it can be written as a finite disjoint union of locally closed sets. }
    \end{defn}
    \uncover<6->{
    \begin{ex}[{\color{red}Simple} example]
        $X \subset X$ is a constructible subset. \uncover<7->{$\{\langle 0\rangle\} \subset \mathbb{A}_{k}^{1}$ is not. }
    \end{ex}}
    \uncover<8->{
    \begin{caution}
        \uncover<8->{What we call ``compact'' is usually called \emph{quasicompact}. } \\
        \uncover<9->{The definition of ``constructible set'' above is not the standard one. }\uncover<10->{However, for Noetherian topological spaces (whatever those are), the two are equivalent. }
    \end{caution}}
\end{frame}
\begin{frame}{Chevalley's Theorem}
    \begin{thm}[Chevalley]
        \uncover<2->{If $\pi : (X, \mathscr{O}_X) \to (Y, \mathscr{O}_Y)$ is a }\uncover<3->{finite type morphism }\uncover<4->{of Noetherian schemes, }\uncover<5->{then the image of any constructible set is constructible. }

        \uncover<6->{In particular, the image of $\pi$ is constructible. }
    \end{thm}
\end{frame}
\begin{frame}{A consequence}
    \begin{cor}[Nullstellensatz]
        \uncover<2->{Let $k \subset K$ be a field extension. }\uncover<3->{Suppose $K$ is a finitely generated $k$-algebra. }\uncover<4->{Then, $K$ is a finite extension of $k.$ }
    \end{cor}
    \uncover<5->{
    \begin{proof} 
        \uncover<5->{Let $K$ be generated by $x_1, \ldots, x_n$, as a $k$-algebra. }\uncover<6->{It suffices to show that each $x_i$ is algebraic over $k.$ }\uncover<7->{Suppose some $x_i$ is not. } \\
        \uncover<8->{Then, we have an inclusion of rings $k[x_i] \into K$, }\uncover<9->{and $k[x_i]$ is isomorphic to the polynomial ring over $k.$ }\\
        \uncover<10->{This corresponds to a dominant morphism $\pi : \Spec K \to \mathbb{A}_{k}^{1}.$ }

        \uncover<11->{Since $\Spec K$ is a singleton, so is the image of $\pi.$ }\uncover<12->{By dominance of $\pi$ (and the {\color{brown}Helper} example), the image is $\{\langle 0\rangle\}.$ }\uncover<13->{But this is not constructible ({\color{red}Simple} example). }\uncover<14->{This contradicts Chevalley's Theorem. }\uncover<15->{\qedhere }
    \end{proof}}
\end{frame}
\begin{frame}{The End}
    Thank you for attending!

    The reference for this talk has been Professor Ravi Vakil's (excellent) notes:

    \url{http://math.stanford.edu/~vakil/216blog/FOAGnov1817public.pdf}
\end{frame}
\end{document}