\documentclass[12pt]{article}
\usepackage[lmargin=1in,rmargin=1in,tmargin=1in,bmargin=1in]{geometry}

\def\univname{}
\def\coursenum{}
\def\coursename{Differential Topology}
\def\professor{}
\def\student{Aryaman Maithani}
\def\semesteryear{Spring 2022}
\def\imagename{../icon.pdf}		  
\def\scalesize{2}
\usepackage{../aryaman}
\setcounter{tocdepth}{2}

\newcommand{\opsub}{\subset_{\operatorname{op}}}
\newcommand{\T}{\mathscr{T}}
\DeclareMathOperator{\Alt}{Alt}

\newcommand{\trans}{\mathrel{\text{\tpitchfork}}}
\makeatletter
\newcommand{\tpitchfork}{%
  \vbox{
    \baselineskip\z@skip
    \lineskip-.52ex
    \lineskiplimit\maxdimen
    \m@th
    \ialign{##\crcr\hidewidth\smash{$-$}\hidewidth\crcr$\pitchfork$\crcr}
  }%
}
\makeatother

\begin{document}

\pagenumbering{Alph}
\coverpage
\thispagestyle{empty}
\updated{\today}
\thispagestyle{empty}
% \setcounter{tocdepth}{1}
\pagenumbering{arabic}
\tableofcontents
\pagestyle{fancy}
% \setcounter{page}{1}
\setcounter{section}{-1}

\section{Preface}

I am making this while I study \emph{Differential Topology} by Victor Guillemin and Alan Pollack. These notes will likely not be helpful to anyone who is looking to learn this material from scratch. I am just going to be noting down the theorems and definitions from the book, assuming the definitions and notations I want to assume. I also skip proofs.

\textbf{Notations}
\begin{enumerate}
	\item ``$U \opsub X$'' stands for ``$U$ is a nonempty open subset of $X$''.
	\item Given a function $f : X \to \mathbb{R}^{m}$, we can write $f = (f_{1}, \ldots, f_{m})$ for functions $f_{i} : X \to \mathbb{R}$ ($i = 1, \ldots, m$). These $f_{i}$ will be referred to as \deff{component functions (of $f$)}.
	\item $S^{k} \subset \mathbb{R}^{k + 1}$ is the unit sphere.
	\item $B_{r}(p)$ will denote the open ball of radius $r$ around the point $p$. The ambient metric space will be clear from context.
	\item Given a function $T : V \to W$, $\im(T)$ denotes the image of $T$.
	\item $\mathbb{H}^{k}$ denotes the upper half-space in $\mathbb{R}^{k}$, i.e.,
	\begin{equation*} 
		\mathbb{H}^{k} \vcentcolon= \{(x_{1}, \ldots, x_{k}) \in \mathbb{R}^{k} : x_{k} \ge 0\}.
	\end{equation*}
	$\partial \mathbb{H}^{k}$ consists of those points in $\mathbb{R}^{k}$ with final coordinate $0$.
	\item For $n \in \mathbb{N}$, $[n]$ will denote the set $\{1, \ldots, n\}$.
\end{enumerate}

\begin{thm}[Invariance of Domain] \label{thm:invariance-of-domain}
	If $U \subset \mathbb{R}^{n}$ is open and $f : U \to \mathbb{R}^{n}$ is an injective continuous function, then $f(U)$ is open in $\mathbb{R}^{n}$.
\end{thm}

\begin{cor}
	Let $U \opsub \mathbb{R}^{n}$ and $V \opsub \mathbb{R}^{m}$. If $U$ and $V$ are homeomorphic, then $n = m$. In particular, $\mathbb{R}^{n}$ and $\mathbb{R}^{m}$ are not homeomorphic for $n \neq m$.
\end{cor}
Note that our notation forces $U$ and $V$ above to be nonempty.
\begin{proof} 
	Suppose $n \neq m$. Without loss of generality, $m > n$. Let $f : V \to U$ be a homeomorphism. Let $g : U \to \mathbb{R}^{m}$ denote the restriction of the inclusion $\mathbb{R}^{n} \into \mathbb{R}^{m}$ given by $\mathbf{x} \mapsto (\mathbf{x}, \mathbf{0})$. \newline
	Thus, $g \circ f : V \to \mathbb{R}^{m}$ is an injective continuous function mapping $V$ onto $g(U) \subset \mathbb{R}^{m}$. But $g(U)$ is not open in $\mathbb{R}^{m}$, contradicting \mynameref{thm:invariance-of-domain}.
\end{proof}

\begin{cor} \label{cor:boundary-Hk-intrinsic}
	Let $x, y \in \mathbb{H}^{k}$. Let $U$ and $V$ be open neighbourhoods of $x$ and $y$ in $\mathbb{H}^{k}$ respectively. If $U$ and $V$ are homeomorphic, then either both $x$ and $y$ belong to $\partial \mathbb{H}^{k}$ or neither does.
\end{cor}
The above corollary is useful later in developing a consistent definition of ``boundary''. It says that the boundary points have topologically different neighbourhoods.
\begin{proof} 
	Suppose that the statement were false. Without loss of generality, $y \in \partial \mathbb{H}^{k}$ and $x \notin \mathbb{H}^{k}$. Note that $f$ restricted to a subset of $U$ is again homeomorphism onto its image. Thus, by shrinking $U$, we may assume that $U$ is an open ball in $\mathbb{R}^{k}$ contained in $\mathbb{H}^{k}$. 

	Let $f : U \to V$ be a homeomorphism. Pick $r > 0$ such that $B_{r}(y) \cap \mathbb{H}^{k} \subset V$. (Here, $B_{r}(y)$ denotes the ball in $\mathbb{R}^{k}$.) 

	Let $V' \vcentcolon= B_{r}(y) \cap \mathbb{H}^{k}$. Note that $V'$ is not open in $\mathbb{R}^{k}$. But $f$ restricted to $f^{-1}(V')$ is an injective continuous map from an open subset of $\mathbb{R}^{k}$ onto the non-open subset $V'$, contradicting \mynameref{thm:invariance-of-domain}.
\end{proof}

\section{Manifolds and Smooth Maps}
\subsection{Definitions}

\begin{defn}
	A function $f$ from $U \opsub \mathbb{R}^{n}$ into $\mathbb{R}^{m}$ is called \deff{smooth} if each component function $f_{i}$ has partial derivatives of all orders.

	More generally, if $X \subset \mathbb{R}^{n}$, then a map $f : X \to \mathbb{R}^{m}$ is called \deff{smooth} if for each point $x \in X$, there exists an open set $U \opsub \mathbb{R}^{n}$ containing $x$ and a smooth function $F : U \to \mathbb{R}^{m}$ such that $F = f$ on $U \cap X$.
\end{defn}

\begin{defn}
	A map $f : X \to Y$ between subsets of Euclidean spaces is called a \deff{diffeomorphism} if $f$ is smooth and bijective with $f^{-1}$ also smooth.

	$X$ and $Y$ are said to be \deff{diffeomorphic} if such a map exists.
\end{defn}

\begin{exe}
	Show that if $f : X \to Y$ is smooth, then $f$ is continuous. \newline
	In particular, diffeomorphic spaces are homeomorphic.
\end{exe}

\begin{defn}
	Let $X \subset \mathbb{R}^{N}$. $X$ is said to be a \deff{$k$-dimensional manifold} if each $x \in X$ possesses a neighbourhood $V \opsub X$ which is diffeomorphic to an open subset $U \opsub \mathbb{R}^{k}$. We define the \deff{dimension} of $X$ as $\dim(X) = k$.

	A diffeomorphism $\phi : U \to V$ is called a \deff{parametrisation} of the neighbourhood $V$. The inverse diffeomorphism $\phi^{-1} : V \to U$ is called a \deff{coordinate system} on $V$. \newline
	Writing $\phi^{-1} = (x_{1}, \ldots, x_{k})$, the component functions $x_{1}, \ldots, x_{k}$ are called \deff{coordinate functions}.
\end{defn}

Note that $\dim X = k$ is well-defined. Indeed, if $U \opsub \mathbb{R}^{n}$ and $U' \opsub \mathbb{R}^{m}$ are homeomorphic, then $n = m$.

\begin{ex}
	Any nonempty open subset of $\mathbb{R}^{N}$ is an $N$-dimensional manifold.
\end{ex}

\begin{ex}
	The circle $S^{1} \subset \mathbb{R}^{2}$ is a $1$-dimensional manifold.

	The open disc $B_{1}(0) = \{x \in \mathbb{R}^{2} : \|x\| < 1\}$ is a $2$-dimensional manifold but the closed disc $\overline{B_{1}(0)}$ is not. 

	Similarly, $(0, 1)$ is a $1$-manifold but $[0, 1]$ is not. This is due to the boundary points.

	Later, we shall see the concept of \emph{manifold with boundary}.
\end{ex}

\begin{ex}
	If $X \subset \mathbb{R}^{N}$ and $Y \subset \mathbb{R}^{M}$ are manifolds, then so is $X \times Y$ with
	\begin{equation*} 
		\boxed{\dim(X \times Y) = \dim(X) + \dim(Y).}
	\end{equation*}

	Indeed, let $k \vcentcolon= \dim(X)$, $l \vcentcolon= \dim(Y)$, and let $(x, y) \in X \times Y$ be arbitrary. Let $U \opsub \mathbb{R}^{k}$ (resp. $W \opsub \mathbb{R}^{l}$) be open and $\phi : U \to X$ (resp. $\psi : W \to Y$) be a parametrisation around $x$ (resp. $y$). 

	Define $\phi \times \psi : U \times W \to X \times Y$ by
	\begin{equation*} 
		(\phi \times \psi)(u, w) \vcentcolon= (\phi(u), \psi(w)).
	\end{equation*}

	Note that $(U \times W) \opsub \mathbb{R}^{k + l}$ and $f \vcentcolon= \phi \times \psi$ is smooth (the component functions of $f$ are the component functions of $\phi$ followed by those of $\psi$). We only need to verify that this is indeed a local parametrisation.

	Note that $\phi$ and $\psi$ are diffeomorphisms onto their images (and the images are open in $X$ and $Y$ respectively). Thus, $V \vcentcolon= \phi(U) \times \psi(W)$ is an open neighbourhood of $(x, y)$ in $X \times Y$. Moreover, $g : V \to U \times W$ by $(x', y') \mapsto (\phi^{-1}(x'), \psi^{-1}(y'))$ is the inverse of $f$. \newline
	The only check that needs to be done is that $g$ is smooth. We leave this to the reader. (Use the smoothness of $\phi^{-1}$ and $\psi^{-1}$ defined in the more general sense.)
\end{ex}

\begin{defn}
	If $X$ and $Y$ are both manifolds in $\mathbb{R}^{N}$ and $Z \subset X$, then $Z$ is a \deff{submanifold} of $X$.

	The \deff{codimension} of $Z$ (in $X$) is defined by $\codim_{X}(Z) = \dim(X) - \dim(Z)$.
\end{defn}
If the ambient manifold $X$ is clear, we will simply write $\codim(Z)$.

\begin{ex}
	$S^{1}$ is a submanifold of $B_{2}(0) \subset \mathbb{R}^{2}$ of codimension $1$.
\end{ex}

\begin{rem}
	We have defined manifolds only as subsets of Euclidean spaces.
\end{rem}

\begin{rem}
	Note that any open ball in $\mathbb{R}^{k}$ is diffeomorphic to $\mathbb{R}^{k}$ (check). Thus, the domains of local parametrisations may be assumed to be $\mathbb{R}^{k}$.
\end{rem}

\subsection{Derivatives and Tangents}

\begin{defn}
	Let $U \opsub \mathbb{R}^{n}$, $f : U \to \mathbb{R}^{m}$ be smooth, and $x \in U$. The \deff{derivative of $f$ at $x$} is the function
	\begin{equation*} 
		df_{x} : \mathbb{R}^{n} \to \mathbb{R}^{m}
	\end{equation*}
	defined by
	\begin{equation*} 
		df_{x}(v) \vcentcolon= \lim_{t \to 0} \frac{f(x + tv) - f(x)}{t}.
	\end{equation*}
\end{defn}
Note that $df_{x}$ is defined on all of $\mathbb{R}^{n}$ even if $U \neq \mathbb{R}^{n}$.

\begin{rem}
	$df_{x}$ is a linear map. In particular, we may represent $df_{x}$ as a matrix using the standard bases. If $f = (f_{1}, \ldots, f_{m})$, then we have
	\begin{equation*} 
		df_{x} = 
		\begin{bmatrix}
			\dfrac{\partial f_{1}}{\partial x_{1}}(x) & \cdots & \dfrac{\partial f_{1}}{\partial x_{n}}(x) \\
			\vdots & \ddots & \vdots \\
			\dfrac{\partial f_{m}}{\partial x_{1}}(x) & \cdots & \dfrac{\partial f_{m}}{\partial x_{n}}(x)
		\end{bmatrix}.
	\end{equation*}
\end{rem}

\begin{ex}
	If $L : \mathbb{R}^{n} \to \mathbb{R}^{m}$ is a linear map, then $dL_{x} = L$ for all $x \in \mathbb{R}^{n}$. \newline
	In particular, if $i : U \into \mathbb{R}^{n}$ is the inclusion map, then $di_{x} = \id_{\mathbb{R}^{n}}$ for all $x \in U$.
\end{ex}

\begin{thm}[Chain rule]
	Let $U \opsub \mathbb{R}^{n}$, $V \opsub \mathbb{R}^{m}$. Suppose $f : U \to V$ and $g : V \to \mathbb{R}^{l}$ are smooth. For all $x \in U$, we have
	\begin{equation*} 
		d(g \circ f)_{x} = dg_{f(x)} \circ df_{x}.
	\end{equation*}
\end{thm}

\begin{defn}
	Let $X \subset \mathbb{R}^{N}$, $x \in X$, $U \opsub \mathbb{R}^{k}$, and $\phi : U \to X$ be a local parametrisation around $x$. For convenience, assume that $0 \in U$ and $\phi(0) = x$.

	The \deff{tangent space} of $X$ at $x$ to be the image of the map $d \phi_{0} : \mathbb{R}^{k} \to \mathbb{R}^{N}$. This is denoted by $T_{x}(X)$. \newline
	A \deff{tangent vector} to $X$ at $x$ is a point $v \in T_{x}(X) \subset \mathbb{R}^{N}$.
\end{defn}

Note that in the above, we are making use of the fact that $X$ is a subset of $\mathbb{R}^{N}$. Also note that $T_{x}(X)$ is a very concrete subspace of $\mathbb{R}^{N}$ -- it is not ``just defined up to isomorphism''. No matter what $\phi$ we choose, we always get the same explicit subspace of $\mathbb{R}^{N}$. The translate $x + T_{x}(X)$ will pass through $x$ and will be ``tangent'' (in a geometric manner) to $X$. Similarly, a tangent vector $v \in T_{x}(X)$ is to be imagined as a segment from $x$ to $x + v$.

The issue to clarify above is whether the above (concrete!) subspace $T_{x}(X)$ depends on $\phi$ or not. Suppose that $\psi : V \to X$ is another local parametrisation with $\psi(0) = x$. Note that $\phi(U)$ and $\psi(V)$ are both (relatively) open neighbourhoods of $x$. By passing to a subset, we may assume $\phi(U) = \psi(V)$. Thus, $h = \psi^{-1} \circ \phi : U \to V$ is a diffeomorphism. Using the chain rule on the relation $\phi = \psi \circ h$ gives
\begin{equation*} 
	d \phi_{0} = d \psi_{0} \circ d h_{0}.
\end{equation*}
Thus, $\im(d \phi_{0}) \subset \im(d \psi_{0})$. By symmetry, the converse is true too, as desired.

\begin{thm}
	With above notations,
	\begin{equation*} 
		\boxed{\dim(T_{x}(X)) = \dim(X),}
	\end{equation*}
	where the dimension on the left is the dimension as a vector space over $\mathbb{R}$.

	In particular, $d\phi_{0} : \mathbb{R}^{k} \to T_{x}(X)$ is an isomorphism.
\end{thm}

We now define derivative for an arbitrary smooth map $f : X \to Y$.

\begin{defn}
	Let $f : X \to Y$ be a smooth map of arbitrary manifolds. Let $x \in X$ and $y \vcentcolon= f(x)$. The \deff{derivative} of $f$ at $x$ is a linear map
	\begin{equation*} 
		df_{x} : T_{x}(X) \to T_{y}(Y)
	\end{equation*}
	defined as follows: Fix parametrisations $\phi : U \to X$ and $\psi : V \to Y$ around $x$ and $y$. ($U \opsub \mathbb{R}^{k}$ and $V \opsub \mathbb{R}^{l}$.) Assume $\phi(0) = x$ and $\psi(0) = y$.

	After passing to a subset of $U$ (so that $f(\phi(U)) \subset \psi(V)$), we have the following commutative square:
	\begin{equation*} 
		\begin{tikzcd}
			X \arrow[rr, "f"] & & Y \\
			U \arrow[u, "\phi"] \arrow[rr, "h = \psi^{-1} \circ f \circ \phi"'] & & V \arrow[u, "\psi"']
		\end{tikzcd}.
	\end{equation*}

	We define $df_{x}$ to be the unique map making the following square commute:
	\begin{equation*} 
		\begin{tikzcd}
			T_{x}(X) \arrow[rr, "df_{x}"] & & T_{y}(Y) \\
			\mathbb{R}^{k} \arrow[u, "d\phi_{0}"] \arrow[rr, "dh_{0}"'] & & \mathbb{R}^{l} \arrow[u, "d\psi_{0}"']
		\end{tikzcd}.
	\end{equation*}
\end{defn}
Note that $d\phi_{0}$ is an isomorphism and thus, $df_{x}$ is uniquely determined as
\begin{equation*} 
	df_{x} = d\psi_{0} \circ dh_{0} \circ d\phi_{0}^{-1}.
\end{equation*}

\begin{exe}
	Check that the above does not depend on choice of $\phi$ or $\psi$.
\end{exe}

One way of the doing the above exercise is to consider this alternate definition of $df_{x}$. Let $X \subset \mathbb{R}^{N}$, $Y \subset \mathbb{R}^{M}$, $f$, $x$, $y$, $U \opsub \mathbb{R}^{k}$, $V \opsub \mathbb{R}^{l}$, $\phi : U \to X$, $\psi : V \to Y$ have the usual meanings. \newline
By definition of $f$ being smooth, there exists an open subset $U' \opsub \mathbb{R}^{N}$ and $F : U' \to \mathbb{R}^{M}$ smooth such that $F = f$ on $U' \cap X$. Now, by shrinking $U$ and $U'$ if necessary, we get a diagram as follows:
\begin{equation*} 
	\begin{tikzcd}
		U' \arrow[rr, "F"] & & \mathbb{R}^{M} \\
		\phi(U) \arrow[u, hook] \arrow[rr, "f"] & & \psi(V) \arrow[u, hook] \\
		U \arrow[u, "\phi"] \arrow[rr, "h = \psi^{-1} \circ f \circ \phi"'] & & V \arrow[u, "\psi"']
	\end{tikzcd}.
\end{equation*}
The upper and lower squares commute and thus, the big outer rectangle commutes. In turn, the usual chain on open subsets of Euclidean spaces tells us that the following big rectangle commutes:
\begin{equation*} 
	\begin{tikzcd}
		\mathbb{R}^{N} \arrow[rr, "dF_{x}"] & & \mathbb{R}^{M} \\
		T_{x}(X) \arrow[u, hook] & & T_{y}(Y) \arrow[u, hook] \\
		\mathbb{R}^{k} \arrow[u, "d\phi_{0}"] \arrow[rr, "dh_{0}"'] & & \mathbb{R}^{l} \arrow[u, "d\psi_{0}"']
	\end{tikzcd}.
\end{equation*}

A simple diagram chase shows that the image of $dF_{X}$ restricted to $T_{x}(X)$ lands within $T_{y}(Y)$. Thus, we get an induced map
\begin{equation*} 
	\begin{tikzcd}
		\mathbb{R}^{N} \arrow[rr, "dF_{x}"] & & \mathbb{R}^{M} \\
		T_{x}(X) \arrow[u, hook] \arrow[rr, dashed, "dF_{X}|_{T_{x}(X)}"] & & T_{y}(Y) \arrow[u, hook] \\
		\mathbb{R}^{k} \arrow[u, "d\phi_{0}"] \arrow[rr, "dh_{0}"'] & & \mathbb{R}^{l} \arrow[u, "d\psi_{0}"']
	\end{tikzcd}
\end{equation*}
which makes the lower square commute. But we had already checked that there is a unique such map. Thus, we have
\begin{equation*} 
	dF_{x} = d\psi_{0} \circ dh_{0} \circ d\phi_{0}^{-1}.
\end{equation*}

Note that the left side is independent of parametrisations and the right side is independent of the extension $F$. In turn, both sides are independent of both and hence, we have our well-defined candidate for $df_{x}$. 

The above definition also makes it easy to solve the following exercise. 

\begin{exe}
	For a submanifold $X$ of $Y$, let $i : X \to Y$ be the inclusion map. Check that $di_{x}$ is the inclusion map of $T_{x}(X)$ into $T_{x}(Y)$.
\end{exe}
\begin{soln} 
	Let $X \subset Y \subset \mathbb{R}^{N}$. Consider the extension $I : \mathbb{R}^{N} \to \mathbb{R}^{N}$ which is the identity map. Then, $di_{x} = dI_{x}|_{T_{x}(X)}$. As noted earlier, $dI_{x} = I$ and we are done.
\end{soln}
I got the above from \url{https://math.stackexchange.com/a/861132/427810}.

Note that in particular, the above exercise implies that if $Z \subset X$ is a submanifold and $z \in Z$, then $T_{z}(Z) \subset T_{z}(X)$. (Once again, we emphasise that it makes sense to talk about these inclusions since the tangent spaces are concrete subspaces.)

\begin{thm}[Chain rule]
	If $X \xrightarrow{f} Y \xrightarrow{g} Z$ are smooth maps of manifolds, then
	\begin{equation*} 
		d(g \circ f)_{x} = dg_{f(x)} \circ df_{x},
	\end{equation*}
	for all $x \in X$.
\end{thm}

\subsection{The Inverse Function Theorem and Immersions}

\begin{defn}
	Let $f : X \to Y$ be a smooth map of manifolds, and $x \in X$. $f$ is called a \deff{local diffeomorphism at $x$} if $f$ maps a neighbourhood of $x$ diffeomorphically onto a neighbourhood of $y \vcentcolon= f(x)$.

	$f$ is called a \deff{local diffeomorphism} if it is a local diffeomorphism at $x$ for every $x \in X$.
\end{defn}

\begin{thm}[Inverse Function Theorem]
	Suppose that $f : X \to Y$ is a smooth map of manifolds, and let $x \in X$. \newline
	$f$ is a local diffeomorphism at $x$ iff $df_{x}$ is an isomorphism.
\end{thm}

\begin{rem}
	If $df_{x}$ is an isomorphism, one can choose local coordinates around $x$ and $y$ so that $f$ appears to be the identity $f(x_{1}, \ldots, x_{k}) = (x_{1}, \ldots, x_{k})$ on some neighbourhood of $x$.

	More precisely: there exists $U \opsub \mathbb{R}^{k}$ and local parametrisations $\phi : U \to X$, $\psi : U \to Y$ such that the following diagram commutes:
	\begin{equation*} 
		\begin{tikzcd}
			X \arrow[rr, "f"] & & Y \\
			U \arrow[u, "\phi"] \arrow[rr, "\text{identity}"'] & & U \arrow[u, "\psi"']
		\end{tikzcd}.
	\end{equation*}
	Note that the same $U$ is being used to parametrise.
\end{rem}

\begin{defn}
	Two maps $f : X \to Y$ and $f' : X' \to Y'$ are said to be \deff{equivalent} (or \deff{same up to diffeomorphism}) if there exist diffeomorphisms $\alpha$ and $\beta$ making the following diagram commute:
	\begin{equation*}  
		\begin{tikzcd}
			X \arrow[rr, "f"] & & Y \\
			X' \arrow[u, "\alpha"] \arrow[rr, "f'"'] & & Y' \arrow[u, "\beta"']
		\end{tikzcd}.		
	\end{equation*}
\end{defn}

\begin{defn} \label{defn:immersions}
	$f : X \to Y$ smooth map of manifolds, $x \in X$, $y \vcentcolon= f(x)$.

	$f$ is said to be an \deff{immersion at $x$} if $df_{x} : T_{x}(X) \to T_{y}(Y)$ is injective. \newline
	$f$ is said to be an \deff{immersion} if $f$ is an immersion at $x$ for all $x \in X$. 

	The \deff{canonical immersion} is the standard inclusion $\mathbb{R}^{k} \into \mathbb{R}^{l}$ for $k \le l$ given by $(x_{1}, \ldots, x_{k}) \mapsto (x_{1}, \ldots, x_{k}, 0, \ldots, 0)$.
\end{defn}

Check that the canonical immersion is an immersion.

\begin{thm}[Local immersion theorem]
	Suppose that $f : X \to Y$ is an immersion at $x \in X$, and $y = f(x)$. Then, there exist local coordinates around $x$ and $y$ such that
	\begin{equation*} 
		f(x_{1}, \ldots, x_{k}) = (x_{1}, \ldots, x_{k}, 0, \ldots, 0).
	\end{equation*}

	In other words, $f$ is locally equivalent to the canonical immersion around $x$.
\end{thm}

\begin{cor}
	If $f$ is an immersion at $x$, then it is an immersion in a neighbourhood of $x$.
\end{cor}

\begin{rem}
	If $\dim(X) = \dim(Y)$, then local immersions and local diffeomorphisms are the same.
\end{rem}

\begin{rem} \label{rem:immersion-injective-not-embedding}
	If $f : X \to Y$ is a smooth map, it is not necessary that $f(X)$ is a manifold. This is not true even if $f$ is assumed to an immersion and injective. 

	One can construct a smooth map $f : \mathbb{R} \to \mathbb{R}^{2}$ which is an injective immersion but the image of $f$ is the figure eight.
\end{rem}

\begin{defn}
	$f : X \to Y$ is called \deff{proper} if the preimage of every compact set in $Y$ is a compact subset of $X$. An immersion which is injective and proper is called an \deff{embedding}.
\end{defn}

\begin{thm}
	An embedding $f : X \to Y$ maps $X$ diffeomorphically onto a submanifold of $Y$.
\end{thm}

\begin{rem} \label{rem:compact-injective-immersion-embedding}
	If $X$ is compact, then every $f : X \to Y$ is proper. In this case, embeddings are same as injective immersions.

	To see why $f$ is proper, note that if $K \subset Y$ is compact, then $K$ is closed and hence, $f^{-1}(K)$ is closed in $X$ by continuity. Any closed subset of a compact space is compact and we are done.
\end{rem}

\subsection{Submersions}

We now study the case where $\dim(X) \ge \dim(Y)$. If $f : X \to Y$ carries $x$ to $y$, then we can demand surjectivity of $df_{x} : T_{x}(X) \to T_{y}(Y)$. Identical to the definitions in \Cref{defn:immersions}, we have the following.

\begin{defn} 
	$f : X \to Y$ smooth map of manifolds, $x \in X$, $y \vcentcolon= f(x)$.

	$f$ is said to be a \deff{submersion at $x$} if $df_{x} : T_{x}(X) \to T_{y}(Y)$ is surjective. \newline
	$f$ is said to be a \deff{submersion} if $f$ is an submersion at $x$ for all $x \in X$. 

	The \deff{canonical submersion} is the standard inclusion $\mathbb{R}^{k} \into \mathbb{R}^{l}$ for $k \ge l$ given by $(x_{1}, \ldots, x_{k}) \mapsto (x_{1}, \ldots, x_{l})$.
\end{defn}

\begin{thm}[Local submersion theorem]
	Suppose that $f : X \to Y$ is a submersion at $x \in X$, and $y = f(x)$. Then, there exist local coordinates around $x$ and $y$ such that
	\begin{equation*} 
		f(x_{1}, \ldots, x_{k}) = (x_{1}, \ldots, x_{l}).
	\end{equation*}

	In other words, $f$ is locally equivalent to the canonical submersion around $x$.
\end{thm}

\begin{cor}
	If $f$ is a submersion at $x$, then it is a submersion in a neighbourhood of $x$.
\end{cor}

\begin{defn}
	For a smooth map $f : X \to Y$, a point $y \in Y$ is called a \deff{regular value for $f$} if $df_{x} : T_{x}(X) \to T_{y}(Y)$ is surjective for every $x \in f^{-1}(y)$. Else, $y$ is called a \deff{critical value}.
\end{defn}
Points not in the image of $f$ are also regular values. In fact, if $\dim(X) < \dim(Y)$, then the regular values are precisely the points in $Y \setminus f(X)$.

\begin{thm}[Preimage Theorem]
	If $y$ is a regular value for $f : X \to Y$, then $f^{-1}(y)$ is a submanifold of $X$, with
	\begin{equation*} 
		\boxed{\dim(f^{-1}(y)) = \dim(X) - \dim(Y).}
	\end{equation*}
\end{thm}

\begin{exe}
	Use the above to show that $S^{k - 1}$ is a manifold of dimension $k - 1$. (Use the map $f : \mathbb{R}^{k} \to \mathbb{R}$ defined by $x \mapsto \|x\|^{2}$ and check that $1$ is a regular value.)
\end{exe}

\begin{exe}
	Note that $M(n)$ -- the space of all $n \times n$ matrices -- can be identified with $\mathbb{R}^{n^{2}}$ in a natural way and is thus a manifold of dimension $n^{2}$. Check that that subset $S(n)$ of symmetric matrices is a submanifold diffeomorphic to $\mathbb{R}^{k}$ where $k = n(n + 1)/2$.

	Check that we have a map $f : M(n) \to S(n)$ given by $A \mapsto AA^{\top}$ which is smooth. Show that the identity matrix $I \in S(n)$ is a regular value. Conclude that $O(n)$ -- the subspace of orthogonal matrices -- is a submanifold of $M(n)$ with dimension $n(n - 1)/2$.
\end{exe}

\begin{defn}
	A group that is a manifold such that the basic operations are smooth is called a \deff{Lie group}.
\end{defn}
By ``basic operations'', we mean the maps $(a, b) \mapsto ab$ and $a \mapsto a^{-1}$.

\begin{ex}
	$O(n)$ is a Lie group.
\end{ex}

\begin{defn}
	Let $g_{1}, \ldots, g_{l} : X \to \mathbb{R}$ be smooth functions, and $x \in X$. \newline
	$g_{1}, \ldots, g_{l}$ are said to be \deff{independent at $x$} if $d(g_{1})_{x}, \ldots, d(g_{l})_{x}$ are linearly independent functionals on $T_{x}(X)$, i.e., linearly independent as elements of the dual $T_{x}(X)^{\ast}$.
\end{defn}

\begin{prop}
	Let $g_{1}, \ldots, g_{l}$ be as in the above definition. Define the function
	\begin{equation*} 
		g : X \to \mathbb{R}^{l}
	\end{equation*}
	by $g \vcentcolon= (g_{1}, \ldots, g_{l})$.

	Then, $dg_{x}$ is a surjection iff $g_{1}, \ldots, g_{l}$ are independent at $x$.
\end{prop}

Note that in the above notation, $g^{-1}(0)$ is the set of common zeroes of $g_{1}, \ldots, g_{l}$. This gives us the following.

\begin{thm}
	If the smooth, real-valued functions $g_{1}, \ldots, g_{l}$ on $X$ are independent at each common zero, then the set $Z$ of common zeroes is a submanifold of $X$ with $\codim(Z) = l$.
\end{thm}

There are two partial converses to the above.

\begin{prop}
	If $y$ is a regular value of a smooth map $f : X \to Y$, then the preimage submanifold $f^{-1}(y)$ can be cut out by independent functions.
\end{prop}

\begin{prop}
	Every submanifold of $X$ is \emph{locally} cut out by independent functions. 

	More precisely: let $Z$ be a submanifold of codimension $l$, and $z \in Z$. Then, there exist $l$ independent functions $g_{1}, \ldots, g_{l}$ defined on some neighbourhood $W \opsub X$ of $z$ such that $Z \cap W$ is the common vanishing set of the $g_{i}$.
\end{prop}

\begin{prop}
	Let $Z$ be the preimage of a regular value $y \in Y$ under the smooth map $f : X \to Y$. Then the kernel of the derivative $df_{x} : T_{x}(X) \to T_{y}(Y)$ at any point $x \in Z$ is precisely the tangent space $T_{x}(Z)$.
\end{prop}

\subsection{Transversality}

So far we discussed the problem of when $f^{-1}(y)$ is a manifold, given a smooth map $f : X \to Y$. Now, we wish to study the more general case of when $f^{-1}(Z)$ is a manifold, where $Z \subset Y$ is a submanifold. Note that being a manifold is a local problem. More precisely: $f^{-1}(Z)$ is a manifold iff every $x \in f^{-1}(Z)$ has a neighbourhood $U \opsub X$ such that $f^{-1}(Z) \cap U$ is a manifold. This observation along with some calculations leads to the following definition and theorem.

\begin{defn}
	Let $f : X \to Y$ be a smooth map, and $Z \subset Y$ a manifold. The map $f$ is said to be \deff{transversal} to the submanifold $Z$, abbreviated $f \trans Z$, if the following equation holds true at each point $x \in f^{-1}(Z)$:
	\begin{equation*} 
		\boxed{\im(df_{x}) + T_{y}(Z) = T_{y}(Y),}
	\end{equation*}
	where $y \vcentcolon= f(x)$.
\end{defn}

\begin{thm}
	If the smooth map $f : X \to Y$ is transversal to a submanifold $Z \subset Y$, then $f^{-1}(Z)$ is a submanifold of $X$. Moreover,
	\begin{equation*} 
		\codim(f^{-1}(Z)) = \codim(Z).
	\end{equation*}
\end{thm}

Note that when $Z$ is a single point, then $T_{y}(Z)$ is the zero subspace and the transversality condition reduces to $\im(df_{x}) = T_{y}(Y)$, i.e., $df_{x}$ is surjective. Thus, the case of $y$ being a regular value was a special case.

\begin{defn}
	Let $Y$ be a manifold, and let $X$ and $Z$ be submanifolds of $Y$. We say that $X$ and $Z$ are \deff{transversal}, abbreviated $X \trans Z$ if
	\begin{equation*} 
		\boxed{T_{x}(X) + T_{x}(Z) = T_{x}(X \cap Z),}
	\end{equation*}
	for every $x \in X \cap Z$.
\end{defn}
The above is simply saying that $i \trans Z$, where $i$ is the inclusion $i : X \into Y$. By symmetry, it is also the same as $j \trans X$ for $j : Z \into Y$. The earlier theorem then specialises to the following.

\begin{thm}
	The intersection of two transversal submanifold manifolds of $Y$ is again a submanifold. Moreover,
	\begin{equation*} 
		\codim(X \cap Z) = \codim(X) + \codim(Z).
	\end{equation*}
\end{thm}
In terms of dimensions, the above says
\begin{equation*} 
	\dim(X \cap Z) = \dim(X) + \dim(Z) - \dim(Y).
\end{equation*}
In particular, note that if $\dim(X) + \dim(Z) < \dim(Y)$, then $X$ and $Z$ can only intersect transversally by not intersecting at all. 

\begin{ex}
	The two coordinate axes intersect transversally in $\mathbb{R}^{2}$ but not when considered submanifolds of $\mathbb{R}^{3}$.
\end{ex}

\begin{exe}
	Suppose $A : \mathbb{R}^{k} \to \mathbb{R}^{n}$ is a linear map and $V$ is a vector space of $\mathbb{R}^{n}$. Check that $A \trans V \Leftrightarrow A(\mathbb{R}^{k}) + V = \mathbb{R}^{n}$. 

	In particular, if $W$ is another subspace of $\mathbb{R}^{n}$, then $V \trans W \Leftrightarrow V + W = \mathbb{R}^{n}$.
\end{exe}

\subsection{Homotopy and Stability}

Let $I$ denote the unit interval $[0, 1] \subset \mathbb{R}$.

\begin{defn}
	Let $f_{0}, f_{1} : X \to Y$ be smooth maps. We say that $f_{0}$ and $f_{1}$ are \deff{homotopic}, abbreviated $f_{0} \sim f_{1}$, if there exists a smooth map $F : X \times I \to Y$ such that $F(x, 0) = f_{0}(x)$ and $F(x, 1) = f_{1}(x)$ for all $x \in X$. \newline
	$F$ is called a \deff{homotopy} between $f_{0}$ and $f_{1}$.
\end{defn}

It is easily checked that $\sim$ is an equivalence relation on smooth maps from $X$ to $Y$, and the equivalence class to which a mapping belongs is its \deff{homotopy class}. We also get a family of smooth maps $(f_{t})_{t \in [0, 1]}$ from $X$ to $Y$ given by
\begin{equation*} 
	f_{t}(x) \vcentcolon= F(x, t).
\end{equation*}

We sometimes also say ``$f_{t} : X \to Y$'' is a homotopy to denote a family $(f_{t})_{t \in I}$ of smooth maps from $X$ to $Y$ such that the map $F : X \times I \to Y$ defined by $(x, t) \mapsto f_{t}(x)$ is a homotopy between $f_{0}$ and $f_{1}$.

\begin{defn}
	A property of smooth maps is said to be \deff{stable} provided that whenever $f_{0} : X \to Y$ possesses the property and $f_{t} : X \to Y$ is a homotopy of $f_{0}$, then, for some $\varepsilon > 0$, each $f_{t}$ with $t < \varepsilon$ also possesses the property.

	The collection of maps that posses a particular property may be referred to as a \deff{stable class} of maps. 
\end{defn}

\begin{ex}
	Consider curves in the planes, i.e., smooth maps from $\mathbb{R}^{1}$ to $\mathbb{R}^{2}$. The property that a curve pass through the origin is not stable. Nor is the property of intersecting the $x$-axis stable. (Think about the mapping $x \mapsto (x, x^{2})$ and the homotopy $(x, t) \mapsto (x, x^{2} + t)$ in both cases.)

	However, transversal intersection with the $x$-axis \emph{is} a stable property.
\end{ex}

\begin{exe}
	Verify the last line above.
\end{exe}

\begin{thm}[Stability theorem]
	Let $X$ be a compact manifold, and $Y$ an arbitrary manifold. The following classes of smooth maps are stable classes:
	\begin{enumerate}
		\item local diffeomorphisms,
		\item immersions,
		\item submersions,
		\item maps transversal to a fixed submanifold $Z \subset Y$,
		\item embeddings,
		\item diffeomorphisms.
	\end{enumerate}
\end{thm}

\section{Sard's Theorem and Morse Functions}

Recall when a subset of a Euclidean space is said to have measure zero.

\begin{defn}
	Let $Y$ be a manifold and $C \subset Y$. $C$ is said to have \deff{measure zero} if for every local parametrisation $\psi$ of $Y$, the preimage $\psi^{-1}(C)$ has measure zero in Euclidean space.
\end{defn}

It can be checked that the following is reducible to the following: there exists a collection of local parametrisations $\{\psi_{\alpha}\}_{\alpha}$ such that $C$ is covered by the images of $\psi_{\alpha}$ and $\psi_{\alpha}^{-1}(C)$ has measure zero for all $\alpha$.

\begin{thm}[Sard's Theorem]
	Let $f : X \to Y$ be any smooth map of manifolds. \newline
	Almost every point in $Y$ is a regular value, i.e., the set of critical values has measure zero.
\end{thm}

\begin{cor}
	The regular values of any smooth map $f : X \to Y$ are dense in $Y$. In fact, if $f_{i} : X \to Y$ are any countable number of smooth maps, then the points of $f_{i}$ that are simultaneously regular values for all of the $f_{i}$ are dense.
\end{cor}

\begin{defn}
	Let $f : X \to Y$ be a smooth map, and $x \in X$. \newline
	If $df_{x}$ is surjective, then we say that \deff{$f$ is regular at $x$} or that \deff{$x$ is a regular point of $f$}. Else, we say that \deff{$x$ is a critical point of $x$}.
\end{defn}

\begin{rem}
	Let $f : X \to Y$ be smooth, and $y \in Y$. \newline
	$y$ is a regular value $\Leftrightarrow$ \emph{every} point in $f^{-1}(y)$ is a regular point. \newline
	$y$ is a critical value $\Leftrightarrow$ \emph{some} point in $f^{-1}(y)$ is a critical point. 
\end{rem}

\begin{defn}
	Let $f : \mathbb{R}^{k} \to \mathbb{R}$ be a smooth function. The \deff{Hessian matrix of $f$} is the $k \times k$ matrix
	\begin{equation*} 
		H = \left(\frac{\partial^{2} f}{\partial x_{i} \partial x_{j}}\right)_{i, j}.
	\end{equation*}
	If $H$ is nonsingular at a critical point $x$, then $x$ is said to be a \deff{nondegenerate critical point of $f$}.
\end{defn}

\begin{prop}
	Nondegenerate critical points are isolated from other critical points of $f$.
\end{prop}
\begin{proof}[Sketch]
	Define $g : \mathbb{R}^{k} \to \mathbb{R}^{k}$ by $g = \left(\frac{\partial f}{\partial x_{1}}, \ldots, \frac{\partial f}{\partial x_{k}}\right)$. \newline
	Then $dg_{x}$ is nonsingular and thus, $g \neq 0$ on a punctured neighbourhood of $x$.
\end{proof}

\begin{thm}[Morse Lemma]
	Suppose $a \in \mathbb{R}^{k}$ is a nondegenerate critical point of $f : \mathbb{R}^{k} \to \mathbb{R}$, and
	\begin{equation*} 
		(h_{ij}) = \left(\frac{\partial^{2} f}{\partial x_{i} \partial x_{j}}(a)\right)
	\end{equation*}
	is the Hessian of $f$ at $a$. Then there exists a local coordinate system $(x_{1}, \ldots, x_{k})$ around $a$ such that
	\begin{equation*} 
		f = f(a) + \sum h_{ij} x_{i} x_{j}
	\end{equation*}
	near $a$.
\end{thm}

We now extend the concept of nondegeneracy of critical points to arbitrary manifolds.

\begin{defn}
	Suppose that $f : X \to \mathbb{R}$ has a critical point at $x$ and that $\phi$ is a local parametrisation carrying the origin to $x$. (Then, $0$ is a critical point for $f \circ \phi$: use chain rule.) \newline
	We declare $x$ to be \deff{nondegenerate for $f$} if $0$ is nondegenerate for $f \circ \phi$.
\end{defn}

One can check that the above does not depend on $\phi$.

\begin{defn}
	A function $X \to \mathbb{R}$ whose critical points are all nondegenerate is called a \deff{Morse function}.
\end{defn}

\textbf{Notation:} If $f$ is a function on $X \subset \mathbb{R}^{N}$ and $a = (a_{1}, \ldots, a_{N}) \in \mathbb{R}^{N}$, we define a new function $f_{a} : X \to \mathbb{R}$ by
\begin{equation*} 
	f_{a}(x_{1}, \ldots, x_{N}) = f + a_{1} x_{1} + \cdots + a_{N} x_{N}.
\end{equation*}

\begin{thm}
	Let $f : X \to \mathbb{R}$ be an arbitrary smooth function. For almost every $a \in \mathbb{R}^{N}$, the function $f_{a}$ is a Morse function on $X$.
\end{thm}

\subsection{Embedding Manifolds in Euclidean Space}

We had defined a $k$-dimensional manifold as some subset of $\mathbb{R}^{N}$. We now ``show'' that we may assume $N = 2k + 1$. That is, if $X$ is a $k$-dimensional manifold and $n \ge 2k + 1$, then there exists an embedding $X \to \mathbb{R}^{n}$. (Note that $\mathbb{R}^{n}$ naturally embeds inside $\mathbb{R}^{m}$ for $m \ge n$.)

\begin{defn}
	Let $X \subset \mathbb{R}^{N}$ be a manifold. The \deff{tangent bundle of $X$}, abbreviated $T(X)$, is defined as
	\begin{equation*} 
		T(X) \vcentcolon= \{(x, v) \in X \times \mathbb{R}^{N} : v \in T_{x}(X)\} \subset X \times \mathbb{R}^{2N}.
	\end{equation*}
\end{defn}
$T(X)$ contains a copy of $X$: $X_{0} = \{(x, 0) : x \in X\}$. Note that $T(X)$ is again a concrete subset of $\mathbb{R}^{2N}$.

\begin{defn}
	Any smooth map $f : X \to Y$ induces a global \deff{derivative map} $df : T(X) \to T(Y)$ defined by
	\begin{equation*} 
		df(x, v) = (f(x), df_{x}(v)).
	\end{equation*}
\end{defn}

Note that if $X \subset \mathbb{R}^{N}$ and $Y \subset \mathbb{R}^{M}$, then $df$ is a map from a subset of $\mathbb{R}^{2N}$ into a subset of $\mathbb{R}^{2M}$, and it makes sense to talk about smoothness of $df$.

\begin{prop}
	$df$ is a smooth map. Moreover, $d(g \circ f) = dg \circ df$. Consequently, if $f$ is a diffeomorphism, then so is $df$.
\end{prop}
In particular, the tangent bundle $T(X)$ is determined up to diffeomorphism, and does not depend on the ambient $\mathbb{R}^{N}$.

\begin{prop}
	The tangent bundle of a manifold is another manifold, with twice the dimension.

	In symbols: If $X$ is a manifold, then so is $T(X)$ and $\dim(T(X)) = 2 \dim(X)$.
\end{prop}

\begin{thm}
	Every $k$-dimensional manifold admits a one-to-one immersion in $\mathbb{R}^{2k + 1}$.
\end{thm}
Note that one-to-one immersions are weaker than embeddings (\Cref{rem:immersion-injective-not-embedding}) but the two coincide for compact spaces (\Cref{rem:compact-injective-immersion-embedding}). To go from compact to arbitrary manifolds, we use partitions of unity.

\begin{thm}
	Let $X$ be an \underline{arbitrary subset} of $\mathbb{R}^{N}$. For any covering of $X$ by (relatively) open subsets $(U_{\alpha})_{\alpha}$, there exists a sequence $(\theta_{i})_{i \ge 1}$ of smooth functions on $X$, called a \deff{partition of unity subordinate to $(U_{\alpha})_{\alpha}$}, with the following properties:
	\begin{enumerate}
		\item $0 \le \theta_{i}(x) \le 1$ for each $x \in X$ and all $i \ge 1$.
		\item Each $x \in X$ has a neighbourhood on which all but finitely many $\theta_{i}$ are zero.
		\item Each function $\theta_{i}$ is zero outside some closed set contained in some $U_{\alpha}$.
		\item For each $x \in X$, $\sum \theta_{i}(x) = 1$.	
	\end{enumerate}
\end{thm}
Note that the last sum is finite for all $x$, in view of the second point.

\begin{cor}
	On any manifold $X$, there exists a proper (smooth) map $\rho : X \to \mathbb{R}$.
\end{cor}

\begin{thm}[Whitney Theorem]
	Every $k$-dimensional manifold embeds in $\mathbb{R}^{2k + 1}$.
\end{thm}


\section{Transversality and Intersection}
\subsection{Manifolds with Boundary}

\begin{defn}
	A subset $X \subset \mathbb{R}^{N}$ is a \deff{$k$-dimensional manifold with boundary} if every point of $X$ possesses a (relative) neighbourhood diffeomorphic to an open subset of $\mathbb{H}^{k}$. A \deff{local parametrisation} is defined similarly as before.

	The \deff{boundary of $X$}, denoted $\partial X$, consists of those points that belong to the image of $\partial \mathbb{H}^{k}$ under some local parametrisation. Its complement is called the \deff{interior of $X$}, denoted $\Int(X) = X \setminus \partial X$.
\end{defn}
Note that \Cref{cor:boundary-Hk-intrinsic} shows that if $x$ belongs to the image of $\partial \mathbb{H}^{k}$ under \emph{some} local parametrisation, then it does so under \emph{any}.

From now on, when we mean interior and boundary, we mean so in the manifold-with-boundary sense and not as a subset of $\mathbb{R}^{N}$. 

\begin{rem}
	Manifolds considered in the earlier chapter are also ``manifolds with boundary'' with $\partial X = \emptyset$.

	We shall continue to use the word ``manifold'' in the sense of the first chapter. For emphasis, we may sometimes refer to them as \emph{boundaryless}.
\end{rem}

\begin{ex} \label{ex:product-not-manifold-boundary}
	$X \vcentcolon= [0, 1]$ is a manifold with boundary. We have $\partial X = \{0, 1\}$. \newline
	Similarly, the closed disc $X' \vcentcolon= \overline{B_{1}(0)} \subset \mathbb{R}^{2}$ is a manifold with boundary. We have $\partial X' = S^{1}$.

	The product of two manifolds with boundary need not again be a manifold with boundary. For example, $[0, 1] \times [0, 1]$ is not a manifold with boundary (the corners are too pointy!).
\end{ex}

\begin{exe}
	Show that $\partial X$ is a closed subset of $X$.
\end{exe}

The failure in \Cref{ex:product-not-manifold-boundary} is essentially because $\mathbb{H}^{k} \times \mathbb{H}^{l}$ is not open in $\mathbb{H}^{k + l}$ (look at the case $k = l = 1$). However, $\mathbb{R}^{k} \times \mathbb{H}^{l}$ \emph{is} open in $\mathbb{H}^{k + l}$, giving us the next proposition. (We actually have $\mathbb{R}^{k} \times \mathbb{H}^{l} = \mathbb{H}^{k + l}$.)

\begin{prop}
	Let $X$ be a boundaryless manifold, and $Y$ be a manifold with boundary. \newline
	Then, $X \times Y$ is a manifold with boundary with
	\begin{equation*} 
		\partial(X \times Y) = X \times \partial Y \andd \dim(X \times Y) = \dim(X) + \dim(Y).
	\end{equation*}
\end{prop}

The earlier notions of derivatives and tangent spaces go as usual.

First, suppose that $U \opsub \mathbb{H}^{k}$ and $g : U \to \mathbb{R}^{l}$ is smooth. If $u$ is an interior point of $U$, then $dg_{u}$ is already defined. If $u \in \partial U$ (i.e., $u \in U \cap \partial \mathbb{H}$), then smoothness of $g$ allows an extension $\widetilde{g}$ to an open neighbourhood of $u$ in $\mathbb{R}^{k}$. $dg_{u}$ is then $d\widetilde{g}_{u} : \mathbb{R}^{k} \to \mathbb{R}^{l}$. \newline
Note that $dg_{u}$ is still a map from $\mathbb{R}^{k}$ to $\mathbb{R}^{l}$.

Now, if $X \subset \mathbb{R}^{N}$ is a manifold with boundary, and $x \in X$, then the \deff{tangent space} $T_{x}(X)$ is defined to be the image of any local parametrisation around $x$. As before, $T_{x}(X)$ is a $k$-dimensional (``concrete'') subspace, even when $x \in \partial X$. 

Lastly, given a smooth map $f : X \to Y$ between manifolds with boundary, the derivative may be defined as a linear transformation $df_{x} : T_{x}(X) \to T_{f(x)}(Y)$.

The chain rule remains valid as before.

\begin{prop}
	Let $X$ be a $k$-dimensional manifold with boundary. \newline
	$\Int(X)$ is a $k$-dimensional manifold (without boundary). \newline
	$\partial X$ is a $(k - 1)$-dimensional manifold (without boundary).
\end{prop}

All definitions from earlier formulated in terms of derivatives of mappings make sense verbatim when the manifolds have boundaries (immersions, submersions, regular values, transversality).

\begin{ex}
	Consider the smooth map $f : \mathbb{H}^{2} \to \mathbb{R}$ given by $(x_{1}, x_{2}) \mapsto x_{2}$. Note that $\mathbb{H}^{2}$ is a manifold with boundary. \newline
	Let $Z = \{0\}$. Then, we have $f \trans Z$, i.e., $0$ is a regular value of $f$. Indeed, let $x \in f^{-1}(Z)$. We have $\im(df_{x}) = \spn\left\{\begin{bmatrix}
		0 & 1
	\end{bmatrix}\right\}$.

	We have $f^{-1}(Z) = \partial \mathbb{H}^{2}$ -- this is a one dimensional manifold without boundary. Note that the equation
	\begin{equation*} 
		\partial(f^{-1}(Z)) = f^{-1}(Z) \cap \partial X
	\end{equation*}
	does not hold.
\end{ex}

\begin{obs}
	If $x \in \partial X$, then $T_{x}(\partial X)$ is a linear subspace of $T_{x}(X)$ of codimension $1$. \newline
	Given a map $f : X \to Y$, let $\partial f : \partial X \to Y$ denote the restriction of $f$ to $\partial X$. \newline
	The derivative of $\partial f$ at $x$ is just the restriction of $df_{x}$ to the subspace $T_{x}(\partial X)$.
\end{obs}

\begin{thm}
	Let $f$ be a smooth map of a manifold with boundary $X$ into a boundaryless manifold $Y$, and suppose that both $f : X \to Y$ and $\partial f : \partial X \to Y$ are transversal with respect to a boundaryless manifold $Z \subset Y$. Then, the preimage $f^{-1}(Z)$ is a manifold with boundary
	\begin{equation*} 
		\partial(f^{-1}(Z)) = f^{-1}(Z) \cap \partial X,
	\end{equation*}
	and $\codim_{X}(f^{-1}(Z)) = \codim_{Y}(Z)$.
\end{thm}

\begin{cor}
	Suppose that $S$ is a manifold without boundary and that $\pi : S \to \mathbb{R}$ is a smooth function with regular value $0$. Then the subset $\{s \in S : \pi(s) \ge 0\}$ is a manifold with boundary, and the boundary is $\pi^{-1}(0)$.
\end{cor}
As an application of the above corollary, we see that the closed unit ball in $\mathbb{R}^{n}$ is a manifold with boundary.

\begin{thm}[Sard's Theorem] \label{thm:sard-boundary}
	Let $f : X \to Y$ be a smooth map, where $X$ is a boundary with manifold and $Y$ is boundaryless. Almost every point of $Y$ is a regular value for both $f$ and $\partial f$.
\end{thm}

\subsection{One-Manifolds and Some Consequences}

\begin{thm}
	Every compact, connected, one-dimensional manifold with boundary is diffeomorphic to either $[0, 1]$ or $S^{1}$.
\end{thm}

\begin{cor}
	The boundary of any compact one-dimensional manifold with boundary consists of an even number of points.
\end{cor}

\begin{thm}
	If $X$ is any compact manifold with boundary, then there exists no smooth map $g : X \to \partial X$ such that $\partial g : \partial X \to \partial X$ is the identity.
\end{thm}

\begin{thm}[Brouwer Fixed-Point Theorem]
	Any smooth map $f$ of the closed unit ball into itself has a fixed point.
\end{thm}

\subsection{Transversality}

Earlier we saw that transversality is a stable property (\mynameref{thm:stability}), when the domain is compact. We now see something more valuable.

\begin{rem}
	One may generalise the idea of homotopies slightly by considering the parameter to come from a more general set than just $I$.

	Suppose $f_{s} : X \to Y$ is a family of smooth maps, indexed by a parameter $s$ that ranges over some set $S$. As we did with homotopies, consider the map $F : X \times S \to Y$ defined by $(x, s) \mapsto f_{s}(x)$. We require that the family vary smoothly by assuming $S$ to be a manifold and $F$ to be smooth.
\end{rem}

\begin{thm}[The Transversality Theorem]
	Suppose that $F : X \times S \to Y$ is a smooth map of manifolds, where only $X$ has boundary, and let $Z \subset Y$ be a boundaryless submanifold. If both $F$ and $\partial F$ are transversal to $Z$, then for almost every $s \in S$, both $f_{s}$ and $\partial f_{s}$ are transversal to $Z$.
\end{thm}
\begin{rem}
	Let $W \subset X \times S$ be the preimage of $Z$. (By the earlier results, $W$ is a manifold with boundary $\partial W = W \cap \partial(X \times S)$.) Let $\pi : W \to S$ be the projection $(x, s) \mapsto s$.

	The proof shows that if $s$ is a regular value for $\pi$, then we have $f_{s} \trans Z$ and that if $s$ is a regular value for $\partial \pi$, then we have $\partial f_{s} \trans Z$. By \mynameref{thm:sard-boundary}, the result then follows.
\end{rem}

As a consequence, we have the following: Let $f : X \to \mathbb{R}^{M}$ be a smooth map, take $S$ to be any open ball of $\mathbb{R}^{M}$, and define $F : X \times S \to \mathbb{R}^{M}$ by $F(x, s) = f(x) + s$. Fixing $x$, the map $F(x, -)$ is a submersion of $S$. In turn, $F$ is a submersion and hence, transversal to any manifold $Z \subset \mathbb{R}^{M}$. The same is true for $\partial F$ as well. (Note that $\partial(X \times S) = (\partial X) \times S$.) Fix a manifold $Z \subset \mathbb{R}^{M}$. \newline
By the transversality theorem, we have that $x \mapsto f(x) + s$ is transversal to $Z$ for almost every $s$. Thus, $f$ may be deformed into a transversal map by the addition of an arbitrarily small quantity $s$.

Some extra work needs to be done to replace the target $\mathbb{R}^{M}$ with an arbitrary manifold $Y$.	

\begin{thm}[$\epsilon$-neighbourhood Theorem]
	For a compact boundaryless manifold $Y \subset \mathbb{R}^{M}$ and a positive number $\epsilon$, let $Y^{\epsilon}$ be the open set of points in $\mathbb{R}^{M}$ with distance less than $\epsilon$ from $Y$. If $\epsilon$ is sufficiently small, then each point $w \in Y^{\epsilon}$ possesses a unique closest point in $Y$, denoted $\pi(w)$. Moreover, the map $\pi : Y^{\epsilon} \to Y$ is a submersion.

	When $Y$ is not compact, there still exists a submersion $\pi : Y^{\epsilon} \to Y$ that is the identity on $Y$, but now $\epsilon$ must be allowed to be a smooth positive function on $Y$, and $Y^{\epsilon}$ is defined as $\{w \in \mathbb{R}^{M} : \|w - y\| < \epsilon(y) \text{ for some } y \in Y\}$.
\end{thm}

\begin{cor}
	Let $f : X \to Y$ be a smooth map with $Y$ boundaryless. Then, there is an open ball $S$ in the ambient space of $Y$ and a smooth map $F : X \times S \to Y$ such that $F(-, 0) = f$, and for any fixed $x \in X$, the map $s \mapsto F(x, s)$ is a submersion $S \to Y$. In particular, both $F$ and $\partial F$ are submersions.
\end{cor}

\begin{thm}[Transversality Homotopy Theorem]
	Let $X, Y, Z$ be manifolds with $Z \subset Y$ and only $X$ having boundary. 

	For any smooth map $f : X \to Y$, there exists a smooth map $g : X \to Y$ homotopic to $f$ such that $g \trans Z$ and $\partial g \trans Z$.
\end{thm}

\begin{defn}
	Let $Y \subset \mathbb{R}^{M}$ be a manifold. For $y \in Y$, the \deff{normal space of $Y$ at $y$}, denoted $N_{y}(Y)$, is defined to be the orthogonal complement of $T_{y}(Y)$ in $\mathbb{R}^{M}$. The \deff{normal bundle of $Y$}, denoted $N(Y)$, is defined as
	\begin{equation*} 
		\{(y, v) \in Y \times \mathbb{R}^{M} : v \in N_{y}(Y)\} \subset R^{2M}.
	\end{equation*}
\end{defn}

Note that $N(Y)$ depends on the ambient space $\mathbb{R}^{M}$. There is a natural projection $\sigma : N(Y) \to Y$ defined by $(y, v) \mapsto y$.

\begin{prop}
	If $Y \subset \mathbb{R}^{M}$, then $N(Y)$ is a manifold of dimension $M$ and the projection $N(Y) \to Y$ is a submersion.
\end{prop}

\begin{defn}
	Let $C \subset X$ be an arbitrary subset. A map $f : X \to Y$ is \deff{transversal to $Z$ on $C$} if the transversality condition
	\begin{equation*} 
		\im(df_{x}) + T_{f(x)}(Z) = T_{f(x)}(Y)
	\end{equation*}
	holds for all $x \in C \cap f^{-1}(Z)$.

	This is denoted as ``$f \trans Z$ on $C$''.
\end{defn}

\begin{prop}
	If $U$ is an open neighbourhood of a closed set $C$ in $X$, then there exists a smooth function $\gamma : X \to [0, 1]$ that is identically equal to $1$ outside $U$ but is $0$ on a neighbourhood of $C$.
\end{prop}

\begin{thm}[Extension Theorem]
	Suppose that $Z$ is a \underline{closed} submanifold of $Y$, both boundaryless, and $C$ is a \underline{closed} subset of $X$. 

	Let $f : X \to Y$ be a smooth map with $f \trans Z$ on $C$ and $\partial f \trans Z$ on $C \cap \partial X$. \newline
	Then, there exists a smooth map $g : X \to Y$ homotopic to $f$, such that $g \trans Z$ and $\partial g \trans Z$, and on a neighbourhood of $C$, we have $g = f$.
\end{thm}
In less precise language: We have modified $f$ homotopically outside a neighbourhood of $C$ to make the transversality conditions hold on all of $X$. 

Picking $C = \partial X$ (this is always closed!) gives us the following corollary.

\begin{cor}
	If, for $f : X \to Y$, the boundary map $\partial f : \partial X \to Y$ is transversal to a closed submanifold $Z \subset Y$, then there exists a map $g : X \to Y$ homotopic to $f$ such that $\partial g = \partial f$ and $g \trans Z$.
\end{cor}
A useful restatement: Supposed $h : \partial X \to Y$ is transversal to $Z$. If $h$ extends to a map $X \to Y$, then $h$ extends to a map that is transversal to $Z$.

\section{Oriented Intersection Theory}

\subsection{Orientation}

Let $V$ be a vector space, and $\beta = (v_{1}, \ldots, v_{k})$ and $\beta' = (v_{1}', \ldots, v_{k}')$ be two ordered bases of $V$. Let $A : V \to V$ be the unique linear isomorphism such that $Av_{i} = v_{i}'$ for all $i \in [k]$.

\begin{defn}
	$\beta$ and $\beta'$ are said to be \deff{equivalently oriented} if $\det(A) > 0$.
\end{defn}

The above defines an equivalence relation on the set of all ordered bases of $V$. There are precisely two equivalence classes.

\begin{defn}
	An \deff{orientation of $V$} is an arbitrary decision to affix a positive sign to the elements of one equivalence class and a negative sign to the others.

	The sign given an ordered basis $\beta$ is called its \deff{orientation}, so $\beta$ is either \deff{positively oriented} or \deff{negatively oriented}, depending on which equivalence class it belongs to. 
\end{defn}

There are precisely two possible orientations for any (nonzero finite-dimensional) vector space $V$. In order to fix an orientation, it suffices to pick a standard ordered basis and assign a sign to it.

The \deff{standard orientation of Euclidean space} is the one for which the standard ordered basis is positively oriented.

For the zero vector space, an orientation is simply a choice of $+1$ or $-1$ (associated to the empty basis, if you wish).

% \textbf{Notation:} Given an ordered subset $\beta = (v_{1}, \ldots, v_{k})$ of $V$ and a map $A : V \to W$, we denote by $A\beta$ the ordered subset $(Av_{1}, \ldots, Av_{k}) \subset W$.

\begin{defn}
	Let $V$ and $W$ be oriented vector spaces, and $A : V \to W$ be a linear isomorphism. $A$ is said to \deff{preserve orientation} if $A \beta$ is a positively oriented basis of $W$ for any choice of positively oriented basis $\beta \subset V$.
\end{defn}
Note: If $\beta$ and $\beta'$ are ordered bases of $V$ from the same orientation class, then $A \beta$ and $A \beta'$ will also have the same orientation class in $W$. Thus, the above definition can be verified by checking the effect on $A$ on some fixed basis $\beta$.

\section{Integration on Manifolds}

\subsection{Exterior algebra}

We quickly recall some linear algebra. Throughout, $V$ will denote a $k$-dimensional vector space over $\mathbb{R}$ and $p$ is a positive integer.

\begin{enumerate}[leftmargin=*]
	% \item $V^{p} = \underbrace{V \times \cdots \times V}_{p \text{ times}}$.
	\item A \deff{$p$-tensor on $V$} is a multilinear function $V^{p} \to \mathbb{R}$.
	%
	\item The collection of all $p$-tensors is a vector space, denoted $\T^{p}(V^{\ast})$. \newline
	Note that $\T^{1}(V^{\ast})$ is simply the dual of $V$, denoted $V^{\ast}$.
	%
	\item If $T$ is a $p$-tensor and $S$ a $q$-tensor, then the \deff{tensor product of $T$ with $S$}, denoted $T \otimes S$, is the $(p + q)$-tensor defined by
	\begin{equation*} 
		(T \otimes S)(v_{1}, \ldots, v_{p}, v_{p + 1}, \ldots, v_{p + q}) \vcentcolon= T(v_{1}, \ldots, v_{p}) \cdot S(v_{p + 1}, \ldots, v_{p + q}).
	\end{equation*}
	%
	\item Tensor product is associative and distributes over addition. However, it is not commutative in general.
	%
	\item If $\{\phi_{1}, \ldots, \phi_{k}\}$ is a basis for $V^{\ast}$, then $\{\phi_{i_{1}} \otimes \cdots \otimes \phi_{i_{p}} : 1 \le i_{1}, \ldots, i_{p} \le k\}$ is a basis for $\T^{p}(V^{\ast})$. Consequently, $\dim(\T^{p}(V^{\ast})) = (\dim(V))^{p}$.
	%
	\item For $\sigma \in S_{p}$, we denote the sign of $\sigma$ by $(-1)^{\sigma}$. \newline
	For a $p$-tensor $T$, define $T^{\sigma}$ to be the tensor given by
	\begin{equation*} 
		T^{\sigma}(v_{1}, \ldots, v_{p}) \vcentcolon= T(v_{\sigma(1)}, \ldots, v_{\sigma(p)}).
	\end{equation*}
	If $\tau \in S_{p}$ is any permutation, one has $(T^{\sigma})^{\tau} = T^{\tau \circ \sigma}$.
	%
	\item A $p$-tensor $T$ is said to be \deff{alternating} if $T^{\pi} = (-1)^{\pi} T$ for all transpositions $\pi \in S_{p}$ (and consequently, for \emph{all} $\pi \in S_{p}$).
	%
	\item If $T$ is any $p$-tensor, we define
	\begin{equation*} 
		\Alt(T) \vcentcolon= \frac{1}{p!} \sum_{\sigma \in S_{p}} (-1)^{\sigma} T^{\sigma}.
	\end{equation*}
	Then, $\Alt(T)$ is an alternating tensor. If $T$ were alternating to begin with, then $\Alt(T) = T$.
	%
	\item The alternating $p$-tensors form a vector subspace of $\T^{p}(V^{\ast})$, denoted $\Lambda^{p}(V^{\ast})$. 
	%
	\item If $T \in \Lambda^{p}(V^{\ast})$ and $S \in \Lambda^{q}(V^{\ast})$, we define their \deff{wedge product} $T \wedge S \in \Lambda^{p + q}(V^{\ast})$ by
	\begin{equation*} 
		T \wedge S \vcentcolon= \Alt(T \otimes S).
	\end{equation*}
	Note that the above definition would make sense even if $T$ and $S$ were not alternating. However, we reserve the use of $\wedge$ for alternating tensors.
	%
	\item If $T \in \T^{p}(V^{\ast})$ and $S \in \T^{q}(V^{\ast})$ and $\Alt(T) = 0$, then $\Alt(T \otimes S) = 0 = \Alt(S \otimes T)$. 
	%
	\item Wedge product is associative. We have
	\begin{equation*} 
		(T \wedge S) \wedge R = \Alt(T \otimes S \otimes R) = T \wedge (S \wedge R)
	\end{equation*}
	for all alternating tensors $T, S, R$. \newline
	The wedge product also distributes over addition. \newline
	The wedge product is \emph{anticommutative}: if $T \in \Lambda^{p}(V^{\ast})$ and $S \in \Lambda^{q}(V^{\ast})$, then
	\begin{equation*} 
		T \wedge S = (-1)^{pq} S \wedge T.
	\end{equation*}
	In particular, if $p$ is odd, then $T \wedge T = 0$. As a special case, we have $\phi \wedge \phi = 0$ for all $\phi \in V^{\ast}$.
	%
	\item If $\{\phi_{1}, \ldots, \phi_{k}\}$ is a basis for $V^{\ast}$, then $\{\phi_{i_{1}} \wedge \cdots \wedge \phi_{i_{p}} : 1 \le i_{1} < \cdots < i_{p} \le k\}$ is a basis for $\T^{p}(V^{\ast})$. Consequently, 
	\begin{equation*} 
		\dim(\T^{p}(V^{\ast})) = \binom{k}{p} = \frac{k!}{p! (k - p)!}.
	\end{equation*}
	(Here $k = \dim(V)$.)
	%
	\item We define $\Lambda^{0}(V^{\ast}) = \mathbb{R}$. The wedge product is extended by defining the product with an element $\lambda \in \mathbb{R}$ as scalar multiplication. The wedge product then makes the direct sum
	\begin{equation*} 
		\Lambda(V^{\ast}) = \Lambda^{0}(V^{\ast}) \oplus \Lambda^{1}(V^{\ast}) \oplus \cdots \oplus \Lambda^{k}(V^{\ast})
	\end{equation*}
	a noncommutative algebra, called the \deff{exterior algebra of $V^{\ast}$}.
	%
	\item Suppose $A : V \to W$ is a linear map. Then, the transpose map $T^{\ast} : W^{\ast} \to V^{\ast}$ extends to the exterior algebras, $A^{\ast} : \Lambda^{p}(W^{\ast}) \to \Lambda^{p}(V^{\ast})$ for all $p \ge 1$ by
	\begin{equation*} 
		(A^{\ast}T)(v_{1}, \ldots, v_{p}) \vcentcolon= T(Av_{1}, \ldots, Av_{p})
	\end{equation*}
	for $T \in \Lambda^{p}(W^{\ast})$ and $(v_{1}, \ldots, v_{p}) \in V^{p}$. For $p = 0$, we define $A^{\ast}(\lambda) = \lambda$.

	Then, $A^{\ast}$ is linear, $A^{\ast}(T \wedge S) = (A^{\ast} T) \wedge (A^{\ast} S)$, and $A^{\ast}(1) = 1$. Thus, $A^{\ast}$ is an algebra homomorphism. \newline
	If $B : V \to W$ is another linear map, then $(AB)^{\ast} = B^{\ast} A^\ast$.
\end{enumerate}

\begin{thm}[Determinant Theorem] \label{thm:determinant-theorem}
	Let $V$ be a $k$-dimensional vector space.

	If $A : V \to V$ is a linear isomorphism, then $A^{\ast} : \Lambda^{k}(V^{\ast}) \to \Lambda^{k}(V^{\ast})$ is given by multiplication by the scalar $\det(A)$. 

	In particular, if $\phi_{1}, \ldots, \phi_{k} \in \Lambda^{1}(V^{\ast})$, then
	\begin{equation*} 
		A^{\ast} \phi_{1} \wedge \cdots \wedge A^{\ast} \phi_{k} = (\det(A)) \phi_{1} \wedge \cdots \wedge \phi_{k}.
	\end{equation*}
\end{thm}

\begin{defn}
	Let $V$ be an oriented $k$-dimensional vector subspace of $\mathbb{R}^{N}$. There is a unique alternating $k$-tensor $T \in \Lambda^{k}(V^{\ast})$ such that $T(v_{1}, \ldots, v_{k}) = 1/k!$ for all positively oriented ordered orthonormal bases $(v_{1}, \ldots, v_{k})$. $T$ is called the \deff{volume element of $V$}.
\end{defn}

\begin{exe}
	Show that the volume element is given as follows: Let $\phi_{1}, \ldots, \phi_{k} \in V^{\ast}$ be the (ordered) dual basis corresponding to some positively oriented ordered orthonormal basis for $V$. Then, the volume element is $\phi_{1} \wedge \cdots \wedge \phi_{k}$.
\end{exe}
\begin{soln}
	By solving this exercise, we shall show the existence of a volume element. The uniqueness then follows since $\dim(\Lambda^{k}(V^{\ast})) = 1$.

	Let $(v_{1}, \ldots, v_{k})$ be a positively oriented ordered orthonormal basis (POOOB) for $V$ and $(\phi_{1}, \ldots, \phi_{k})$ denote the dual basis. Let $(w_{1}, \ldots, w_{k})$ be any other POOOB for $V$. Then, there exists an orthogonal linear map $A : V \to V$ such that $Av_{i} = w_{i}$ for all $i$. Furthermore, since $(w_{i})_{i}$ is also positively oriented, we have $\det(A) = 1$. Now, by the \mynameref{thm:determinant-theorem}, we have
	\begin{align*} 
		1 \cdot  &= \left[(A^{\ast} \phi_{1}) \wedge \cdots \wedge (A^{\ast} \phi_{k})\right](w_{1}, \ldots, w_{k}) \\
		&= \Alt(A^{\ast}\phi_{1} \otimes \cdots \otimes A^{\ast}\phi_{k})(w_{1}, \ldots, w_{k}) \\
		&= \frac{1}{k!} \sum_{\sigma \in S_{k}} (-1)^{\sigma} (A^{\ast}\phi_{1} \otimes \cdots \otimes A^{\ast}\phi_{k})^{\sigma}(w_{1}, \ldots, w_{k}) \\
		&= \frac{1}{k!} \sum_{\sigma \in S_{k}} (-1)^{\sigma} (A^{\ast}\phi_{1} \otimes \cdots \otimes A^{\ast}\phi_{k})(w_{\sigma(1)}, \ldots, w_{\sigma(k)}) \\
		&= \frac{1}{k!} \sum_{\sigma \in S_{k}} (-1)^{\sigma} \phi_{1}(Aw_{\sigma(1)}) \cdots \phi_{k}(Aw_{\sigma(k)}).
	\end{align*}
	The only nonzero summand is when $\sigma = \id$ for otherwise we have $\sigma(i) \neq i$ for some $i$ and consequently $\phi_{i}(Aw_{\sigma(i)}) = \phi_{i}(v_{\sigma(i)}) = 0$. Thus, we are left with
	\begin{equation*} 
		(\phi_{1} \wedge \cdots \wedge \phi_{k})(w_{1}, \ldots, w_{k}) = \frac{1}{k!}. \qedhere
	\end{equation*}
\end{soln}

\subsection{Differential forms}

\begin{defn}
	Let $X$ be a smooth manifold with or without boundary. A \deff{$p$-form on $X$} is a function $\omega$ defined on $X$ such that $\omega(x) \in \Lambda^{p}[T_{x}(X)^{\ast}]$ for all $x \in X$.
\end{defn}
Note that there is no condition on $\omega$. For example, $0$-forms on $X$ are \emph{arbitrary} real-valued functions on $X$ (need not even be continuous).

The sum of two $p$-forms is defined pointwise. Similarly, if $\omega$ is a $p$-form and $\theta$ a $q$-form, then their wedge product is defined pointwise and satisfies $\omega \wedge \theta = (-1)^{pq} \theta \wedge \omega$.

\begin{defn}
	If $\phi : X \to \mathbb{R}$ is a smooth function, then $d\phi_{x} : T_{x}(X) \to \mathbb{R}$ is a linear map at each point $x$. Thus, the assignment $x \mapsto d\phi_{x}$ defines a $1$-form $d\phi$ on $X$, called the \deff{differential of $\phi$}.
\end{defn}

\begin{ex}
	Let $x_{1}, \ldots, x_{k} : \mathbb{R}^{k} \to \mathbb{R}$ denote the projection functions. Then, for each $z \in \mathbb{R}^{k}$, we have $T_{z}(\mathbb{R}^{k}) = \mathbb{R}^{k}$ and
	\begin{equation*} 
		dx_{i}(z)(a_{1}, \ldots, a_{k}) = a_{i}.
	\end{equation*}
	Thus, at each $z \in \mathbb{R}^{k}$, the linear functionals $dx_{1}(z), \ldots, dx_{k}(z)$ just form the standard basis of $(\mathbb{R}^{k})^{\ast}$.
\end{ex}

For each strictly increasing sequence $I = (i_{1}, \ldots, i_{p})$ of integers, define
\begin{equation*} 
	dx_{I} = dx_{i_{1}} \wedge \cdots \wedge dx_{i_{p}}.
\end{equation*}
The above is a $p$-form on $\mathbb{R}^{k}$.

\begin{prop}
	Every $p$-form on an open set $U \opsub \mathbb{R}^{k}$ may be uniquely expressed as a sum $\sum f_{I} dx_{I}$, over increasing index sequences with $f_{I}$ being real-valued functions on $U$.
\end{prop}

\begin{exe} \label{exe:d-phi-summation-partials}
	Show that if $\phi$ is a smooth function on $\mathbb{R}^{k}$, then
	\begin{equation*} 
		d \phi = \sum \frac{\partial \phi}{\partial x_{i}} dx_{i}.
	\end{equation*}
\end{exe}
\begin{soln}
	Both elements written above are $p$-forms and we check that they are equal by evaluating them both at an arbitrary $z \in \mathbb{R}^{k}$.

	By definition, we have $d \phi(z) = d\phi_{z}$. Note that $d\phi_{z}$ is a function from $T_{z}(\mathbb{R}^{k})$ to $\mathbb{R}$. Moreover this function is given as follows:
	\begin{equation*} 
		d \phi(z)(y_{1}, \ldots, y_{k}) = 
		\begin{bmatrix}
			\dfrac{\partial \phi}{\partial x_{1}}(z) & \cdots & \dfrac{\partial \phi}{\partial x_{k}}(z)
		\end{bmatrix}
		\begin{bmatrix}
			y_{1} \\
			\vdots \\
			y_{k} \\
		\end{bmatrix}
		= \frac{\partial \phi}{\partial x_{1}}(z) y_{1} + \cdots + \frac{\partial \phi}{\partial x_{k}}(z) y_{k}.
	\end{equation*}
	On the other hand, we noted that
	\begin{equation*} 
		dx_{i}(y_{1}, \ldots, y_{k}) = y_{i}
	\end{equation*}
	for all $i$. Thus, we get
	\begin{equation*} 
		d \phi(z)(y_{1}, \ldots, y_{k}) = \frac{\partial \phi}{\partial x_{1}}(z) dx_{1}(y_{1}, \ldots, y_{k}) + \cdots + \frac{\partial \phi}{\partial x_{k}}(z) dx_{k}(y_{1}, \ldots, y_{k}),
	\end{equation*}
	as desired.
\end{soln}

\begin{defn}
	Let $f : X \to Y$ be a smooth map, and $\omega$ a $p$-form on $Y$. The \deff{pullback of $\omega$ by $f$}, denoted $f^{\ast} \omega$, is a $p$-form on $X$ defined by
	\begin{equation*} 
		f^{\ast} \omega(x) \vcentcolon= (df_{x})^{\ast} \omega[f(x)].
	\end{equation*}
\end{defn}
Unraveling: Let $x \in X$, and set $y \vcentcolon= f(x)$. Then, we have a linear map $df_{x} : T_{x}(X) \to T_{x}(Y)$. This induces the transpose map $(df_{x})^{\ast} : \Lambda^{p}(T_{x}(Y)) \to \Lambda^{p}(T_{x}(X))$. Since $\omega(y)$ is an element of $\Lambda^{p}(T_{x}(Y))$, the above definition gives us an element of $\Lambda^{p}(T_{x}(X))$, as desired.

\begin{ex}
	When $\omega$ is a $0$-form, i.e., a function on $Y$, then the pullback $f^{\ast} \omega$ is simply the function $\omega \circ f$.
\end{ex}

\begin{exe}
	Prove the following:
	\begin{enumerate}
		\item $f^{\ast}(\omega_{1} + \omega_{2}) = f^{\ast}\omega_{1} + f^{\ast}\omega_{2}$,
		\item $f^{\ast}(\omega \wedge \theta) = (f^{\ast}\omega) \wedge (f^{\ast}\theta)$,
		\item $(f \circ h)^{\ast} \omega = h^{\ast} f^{\ast} \omega$.
	\end{enumerate}
\end{exe}

\begin{ex}
	Let $U \opsub \mathbb{R}^{k}$ and $V \opsub \mathbb{R}^{l}$, and $f : V \to U$ be smooth. Use $x_{1}, \ldots, x_{k}$ (resp. $y_{1}, \ldots, y_{l}$) for the standard coordinates on $\mathbb{R}^{k}$ (resp. $\mathbb{R}^{l}$). \newline
	Write $f = (f_{1}, \ldots, f_{k})$. The derivative $df_{y}$ at a point $y \in V$ is represented by the matrix
	\begin{equation*} 
		\left[\frac{\partial f_{i}}{\partial y_{j}}(y)\right]_{i, j}.
	\end{equation*}
	The transpose $(df_{y})^{\ast}$ on $1$-forms is given by the transpose of the above matrix.

	Consequently, given the $1$-form $dx_{i}$ (where $1 \le i \le k$), we have
	\begin{equation*} 
		f^{\ast} dx_{i}(y) = (df_{y})^{\ast} dx_{i}(f(y)).
	\end{equation*}
	The above quantity is an element of $\Lambda^{1}(T_{y}(V)) = \Lambda^{1}(\mathbb{R}^{l})$. Given $z = (z_{1}, \ldots, z_{l}) \in \mathbb{R}^{l}$, the right hand side evaluates as
	\begin{align*} 
		(df_{y})^{\ast} dx_{i}(f(y))(z) &= dx_{i}(f(y)) (df_{y}(z)).
	\end{align*}
	Now, recall that $dx_{i}(f(y))$ is simply the functional that is dual to $x_{i}$. Thus, the above quantity is simply the $i$-th coordinate of $df_{y}(z)$ and is given as
	\begin{equation*} 
		\sum_{j = 1}^{l} \frac{\partial f_{i}}{\partial y_{j}}(y) z_{j}.
	\end{equation*}
	Note that the $z_{j}$ above can in turn be written as $dy_{j}(y)(z)$. Thus, we have shown
	\begin{equation*} 
		f^{\ast} dx_{i}(y)(z) = \sum_{j = 1}^{l} \frac{\partial f_{i}}{\partial y_{j}}(y) dy_{j}(y)(z).
	\end{equation*}
	In concise notation, we have
	\begin{equation*} 
		f^{\ast} dx_{i} = \sum_{j = 1}^{l} \frac{\partial f_{i}}{\partial y_{j}} dy_{j} = d f_{i}
	\end{equation*}
	for all $1 \le i \le k$. (The second equality above is simply \Cref{exe:d-phi-summation-partials}.)
\end{ex}

\end{document}