\documentclass[11pt,leqno,landscape,semhelv]{seminar}
\usepackage{amsmath, amssymb, amsfonts, amsthm, mathtools}
\usepackage{fancybox}
\usepackage[inline]{enumitem}
\usepackage{cancel}
\usepackage{soul}
\usepackage{centernot}
\usepackage{tikz-cd}
\usepackage{datetime2}

\theoremstyle{definition}
\newtheorem{joke}{Joke}
\newtheorem{thm}{Theorem}
\newtheorem{lem}[thm]{Lemma}
\newtheorem{cor}[thm]{Corollary}
\newtheorem{prop}[thm]{Proposition}
\newtheorem{defn}[thm]{Definition}

\usepackage{chngcntr}
\numberwithin{joke}{section}
\numberwithin{thm}{section}
\numberwithin{equation}{section}

\newcommand{\example}[1]{\refstepcounter{thm}\par\medskip
   {\textbf{Example \thethm.} #1} \rmfamily}
\newcommand{\remark}[1]{\refstepcounter{thm}\par\medskip
   {\textit{Remark \thethm.} #1} \rmfamily}
\usepackage{titlesec}
\titleformat{\section}[block]
  {}{\centering\huge\S\thesection}{0.25cm}{\centering\Huge\textsc}
\titleformat{\subsection}[block]
  {}{\S\S\thesubsection}{0.25cm}{\Large}
  
\renewcommand{\sec}[1]{%
\begin{slide}
\begin{center}
    \begin{center}
        \section{#1}
    \end{center}
\end{center}
\end{slide}}
\newcommand{\cd}[1]{
\begin{center}
	\begin{tikzcd}
		#1
	\end{tikzcd}
\end{center}}
\newcommand{\cod}{\operatorname{cod}}
\newcommand{\dom}{\operatorname{dom}}
\newcommand{\id}{\operatorname{id}}
\newcommand{\Hom}{\operatorname{Hom}}
\newcommand{\op}{^{\operatorname{op}}}
\newcommand{\mono}{\rightarrowtail}
\newcommand{\epi}{\twoheadrightarrow}
\newcommand{\tto}{\rightrightarrows}
\let\emptyset\varnothing

\DeclareSymbolFont{matha}{OML}{txmi}{m}{it}
\DeclareMathSymbol{\varv}{\mathord}{matha}{35}

\setlength\parindent{0pt}

\usepackage{xcolor}
\definecolor{mybgcolor}{RGB}{50, 50, 50} %46, 51, 63
\definecolor{mylinkcolor}{RGB}{0, 255, 255} %46, 51, 63

\usepackage{pagecolor}
%\pagecolor{mybgcolor}
%\color{white}

\usepackage[colorlinks=true,
	%linkcolor=mylinkcolor
	]
	{hyperref}


\title{\vspace{1cm} Category Theory}
\author{Aryaman Maithani\\\url{https://aryamanmaithani.github.io/}}
\date{\DTMnow}

\begin{document}
\maketitle
\newpage
\setcounter{section}{-2}
\tableofcontents
\sec{Introduction}
This is essentially a way to force myself to be productive and self-study Category Theory. I try to write these notes in a way that I'm teaching them to someone to help me understand it better.\\
That being said, I don't really think that these notes would be better than reading the book which I'm reading itself. The book being - 
\begin{center}
	\emph{Category Theory} by \emph{Steve Awodey}.
\end{center}
I do skip some of the more advanced examples but in places I do add extra explanations as I feel necessary. So yeah, have fun.\\
A personal suggestion: whenever you see a proposition or lemma or anything that is followed by a proof, try doing it yourself. When you struggle with it, that's when the definitions \emph{really} seep in. Due to this, you would find many proofs that are not exactly the same as the ones given in the book. \\~\\
These notes will keep getting updated as I read more, possibly over the next two years.\\
I'm also not proof-reading these, so there are bound to be many errors. If you find any, \emph{please} let me know. You shall then get to feature in the last section - \nameref{sec:ack}, if you wish. Apart from typos, you may also give suggestions if you think that something was ambiguously phrased. This will help me in wording things better for more clarity. That's how these notes could \emph{really} end up adding something of value.\\
Lastly, if you want to view this in a way that you can select text or click hyperlinks, you may use the following link - 
\begin{center}
	\url{bit.ly/raw-cat}
\end{center}
\sec{Preliminaries}
These are not strictly preliminaries, in the sense that you do not \emph{need} to know these to understand the text. However, since the examples do rely on these, it's certainly beneficial to know these.\\
The reader may skip these at the beginning and return to these whenever the text demands it.
\subsection{Axiom of Choice}\label{axiomofchoice}
Given a collection $\mathcal{B}$ of nonempty sets, there exists a function
\begin{equation*} 
	f:\mathcal{B} \to \bigcup_{B \in \mathcal{B}}B
\end{equation*}
such that $f(B) \in B,$ for each $B \in \mathcal{B}.$\\
The function $f$ is called a \emph{\textbf{choice function}} for the collection $\mathcal{B}.$\\
Note that the above can be stated in an equivalent way where the collection of (nonempty) sets is being indexed by some set. This is done in the following manner:\\
Let $X$ be a set and $(E_x)_{x \in X}$ be an arbitrary collection of nonempty sets. \\
Set $E = \displaystyle\bigcup_{x \in X}E_x.$ \\
Then, there exists a function
\begin{equation*} 
 	f : X \to E
\end{equation*}
such that $f(x) \in E_x,$ for all $x \in X.$\\
As before, $f$ is called a \emph{\textbf{choice function}}.\\~\\
Informally, the definition above states that:\\
Given a collection $\mathcal{B}$ of nonempty sets, there is a way to ``choose'' exactly one element from each set. (Note that we do not demand the sets to be disjoint, so it is possible that the same element can picked from two different sets. That is okay.) \\
One may also note that there is no uniqueness demanded of the choice function.
\subsection{Monoids}
A \emph{monoid} is a set $M$ equipped with a binary operation $*$ and a distinguished element $u \in M$ satisfying:
\begin{itemize}
	\item $a*(b*c) = (a*b)*c,$ for all $a, b, c \in M,$ and
	\item $a*u = a = u*a$ for all $a \in M.$
\end{itemize}
We may sometimes also refer to the monoid as $(M, *, u).$\\
A monoid homomorphism $h:(M, *, u_M) \to (N, \cdot, u_N)$ is a function $h:M\to N$ satisfying
\begin{itemize}
	\item $h(a*b) = h(a)\cdot h(b),$ for all $a, b \in M,$ and
	\item $h(u_M) = h(u_N).$
\end{itemize}
\subsection{Groups}
With the same notation as earlier, a monoid $M$ is said to be \emph{group} if for every $a \in M,$ there exists some $b \in M$ such that $a * b = u = b * a.$\\
A group homomorphism is a monoid homomorphism between groups.
\subsection{Preorders}
A \emph{preorder} is a set $P$ equipped with a binary relation $\le$ satisfying:
\begin{itemize}
	\item $a \le a$ for all $a \in P,$ and
	\item $a \le b$ and $b \le c$ implies $a \le c$ for all $a, b, c \in P.$
\end{itemize}
That is, the relation is reflexive and transitive.
\subsection{Posets}
A \emph{poset} is a preorder $P$ with the additional condition that $\le$ is antisymmetric, that is,
	\begin{equation*} 
		a \le b \text{ and } b \le a \implies a = b \text{ for all } a, b \in P.
	\end{equation*}
An order-preserving map between posets $(P, \le_P)$ and $(Q, \le_Q)$ is a function $f:P\to Q$ satisfying
\begin{align*} 
	p \le_P q \implies f(p) \le_Q f(q). 
\end{align*}
\subsection{Boolean algebra} \label{boolalg}
A \emph{Boolean algebra} is a poset $B$ equipped with distinguished elements $0, 1,$ binary operations $a\vee b$ of ``join'' and $a\wedge b$ of ``meet,'' and a unary operation $\neg b$ of ``complementation.'' These are required to satisfy the conditions
\begin{align*} 
	0 &\le a\\
	a &\le 1\\
	a \le c \text{ and } b \le c \quad &\text{iff} \quad a \vee b \le c \\
	c \le a \text{ and } c \le b \quad &\text{iff} \quad c \le a \wedge b \\
	a \le \neg b \quad &\text{iff} \quad a \wedge b = 0\\
	\neg\neg a  &= a.
\end{align*}
A typical example of a Boolean algebra is the power-set $\mathcal{P}(X)$ of $X.$ We have the following identifications:
\begin{equation*} 
	0 = \emptyset, 1 = X, A \vee B = A \cup B, A \wedge B = A \cap B, \neg A = X\setminus A.
\end{equation*}
A Boolean homomorphism is a function $h:B\to B'$ between Boolean algebras that is an order-preserving map which preserves the addition structure, id est, 
\begin{align*} 
	h(0) &= 0, \\
	h(1) &= 1, \\
	h(a \vee b) &= h(a) \vee h(b), \\
	h(a \wedge b) &= h(a) \wedge h(b), \text{ and} \\
	h(\neg a) &= \neg h(a).
\end{align*}
%
\subsection{Topological spaces}
A \textbf{\emph{topology}} on a set $X$ is a collection $\mathcal{T}$ of subsets of $X$ having the following properties:
\begin{itemize} 
\item $\emptyset$ and $X$ are in $\mathcal{T}.$
\item The union of elements of any subcollection of $\mathcal{T}$ is in $\mathcal{T}.$
\item The intersection of the elements of any finite subcollection of $\mathcal{T}$ is in $\mathcal{T}.$
\end{itemize}
If $X$ is a topological space with topology $\mathcal{T}$, we say that a subset $U$ of $X$ is an \textbf{\emph{open set}} of $X$ if $U$ belongs to the collection $\mathcal{T}.$\\
In this text, we shall write $O(X)$ to denote $\mathcal{T},$ id est, the collection of open sets.

Given topological spaces $X$ and $Y,$ a function $f:X\to Y$ is said to be open if the preimage of every open set is open. That is to say, if $V \in O(Y),$ then $f^{-1}(V) \in O(X).$

(To recall, $f^{-1}(V) = \{x \in X \mid f(x) \in V\}.$)
%
\subsection{Equivalence relations and Quotients} \label{equivrel}
Let $B$ be a set. A \emph{relation} $R$ \emph{on} $B$ is a subset $R \subset B \times B.$ We write ``$aRb$'' to mean ``$(a, b) \in R$.''\\
Let ${\sim}$ be a relation on $B.$ We say that ${\sim}$ is an \emph{equivalence relation} on $B$ if the following holds for all $x, y, z \in B:$
\begin{itemize}
	\item (Reflexive) $x \sim x,$
	\item (Symmetric) $x \sim y$ implies $y \sim x,$
	\item (Transitive) $x \sim y$ and $y \sim z$ implies $x {\sim} z.$
\end{itemize}
Given such a relation, one defines the \emph{equivalence class} $[x]$ of an element $x \in X$ by

\begin{equation*} 
	[x] = \{y \in  B \mid x \sim y\}.
\end{equation*}

The various different equivalence classes $[x]$ then form a \emph{partition} of $B,$ that is:
\begin{itemize}
	\item Each equivalence class is nonempty,
	\item Any two distinct equivalence classes are disjoint, and
	\item The union of all equivalence classes is $B.$
\end{itemize}
(That the above is true is an exercise.)\\
In particular, given any $x \in B,$ there exists a unique equivalence class to which $x$ belongs.\\
The set of all equivalence classes

\begin{equation*} 
	B/{\sim} = \{[x] \mid x \in B\}
\end{equation*}
is called the \emph{quotient} of $X$ by ${\sim}.$ it is used in place of $B$ when one wants to ``abstract away'' the difference between equivalent elements $x \sim y,$ in the sense that in $B/{\sim}$ such (and only such) elements are identified, since

\begin{equation*} 
	[x] = [y] \; \text{iff} \; x \sim y.
\end{equation*}
We have a natural (surjective) map $B \to B/{\sim}$ called the \emph{quotient mapping},

\begin{equation*} 
	q:B \to B/{\sim}
\end{equation*}
defined as $x \mapsto [x].$ 

\begin{defn} 
	Let $R$ be a relation on a set $B$ and $f:B\to Z$ be a function.\\
	We say that ``$f$ respects $R$'' if $f(b) = f(b')$ whenever $(b, b') \in R.$
\end{defn}

\begin{prop} \label{prop:fextend}
The quotient mapping has the property that a map $f:B\to Z$ extends (uniquely) as,
\begin{equation*} 
	\begin{tikzcd}
		B \arrow[rr, "q"] \arrow[rrddd, "f"'] && B/{\sim} \arrow[ddd, "\bar{f}" ,dotted]\\
		&&\\&&\\
		&& Z
	\end{tikzcd}
\end{equation*}
precisely when $f$ respects the equivalence relation.
\end{prop}
\begin{proof} 
	Note that if the map $f$ extends, then it must respect the relation. For if $x \sim y,$ then $q(x) = q(y)$ and $f(x) = \bar{f}(q(x)) = \bar{f}(q(y)) = f(y).$\\~\\
	Conversely, suppose that $f$ respects $\sim.$ Then, define $\bar{f}:B/{\sim} \to Y$ as follows:
	\begin{equation*} 
		\bar{f}([x]) = f(x).
	\end{equation*}
	We must verify that the above is indeed well-defined, id est, $f(x) = f(y)$ whenever $[x] = [y].$ However, this is precisely what it means to say that $f$ respects $\sim.$\\
	The uniqueness follows from the fact that $q$ is surjective.
\end{proof}

%This concept extends to many other cases of taking more sophisticated quotients in the cases of monoids, groups, topological spaces, et cetera.\\~\\

\textbf{Equivalence closure}: Suppose we are given an arbitrary relation $R$ on some set $B.$ We can construct a relation ${\sim}$ on $B$ which is the \emph{smallest} equivalence relation containing $R$. This is constructed as follows:\\
Let $\mathcal{E}$ be the set of all equivalence relations on $B$ containing $R,$ id est,
\begin{equation*} 
	\mathcal{E} = \{E \subset B \times B \mid R \subset E \text{ and } E \text{ is an equivalence relation}\}.
\end{equation*}

Then, 
\begin{equation*} 
	{\sim} = \bigcap_{E \in \mathcal{E}}E.
\end{equation*}
It can be verified that ${\sim}$ is an equivalence relation. (Intersection of any family of equivalence relation is an equivalence relation.) Moreover, $R \subset {\sim}.$ We also have that ${\sim} \subset E$ for any equivalence relation $E \supset R.$\\~\\
We say that ${\sim}$ is the equivalence relation \emph{generated} by $R.$

We now observe a crucial property of this closure which is later used in \S\S\ref{ssec:coeq}.\\
Let the notations be the same as earlier. Let $f:B\to Z$ be a function.\\
As before, we let ${\sim}$ denote the equivalence relation generated by $R$ and $q:B \to B/{\sim}$ be the projection mapping, all as depicted in the following diagram.

\begin{equation*} 
	\begin{tikzcd}
		B \arrow[rr, "q"] \arrow[rrddd, "f"'] && B/{\sim}\\
		&&\\&&\\
		&& Z
	\end{tikzcd}
\end{equation*}

\begin{prop} \label{prop:fextends2}
	If $f:B\to Z$ respects $R,$ then there exists a (unique) function $\bar{f}:B/{\sim} \to Z$ such $\bar{f}\circ q = f.$
\end{prop}
\begin{proof} 
	We first define an equivalence relation $\approx$ on $B$ as 
	\begin{equation*} 
		b \approx b' \text{ iff } f(b) = f(b').
	\end{equation*}
	(This can easily be verified to be an equivalence relation.)\\
	\textbf{Claim 1.} $R \subset \ \approx.$\\
	\emph{Proof.} $(b, b)' \in R \implies f(b) = f(b') \implies (b, b') \in \ \approx.$ \\
	(The first implication follows from the fact that $f$ respects $R.$)\\~\\
	Now, from the definition of $\sim,$ it follows that ${\sim} \subset {\approx}.$\\
	However, this means that $f$ respects $\sim$ as well. Indeed,
	\begin{align*} 
		b \sim b' &\implies b \approx b'\\
		&\implies f(b) = f(b').
	\end{align*}
	The result now follows by Proposition \ref{prop:fextend}.
\end{proof}
\sec{Categories}
\subsection{Definition - Category}
We avoid any technicalities of set theory and define what a category is.
\begin{defn} 
	A \emph{category} consists of the following data:
	\begin{itemize}
		\item Objects: $A, B, C, \ldots$
		\item Arrows (or \emph{morphisms}): $f, g, h, \ldots$
		\item For each arrow $f,$ there are given objects
		\[\dom(f), \cod(f)\]
		called the \emph{domain} and \emph{codomain} of $f.$ We write 
		\[f:A \to B\]
		to indicate that $\cod(f) = A$ and $\dom(f) = B.$
		\item Given arrows $f:A\to B$ and $g:B\to C,$ there is given an arrow
		\[g\circ f : A \to C\]
		called the \emph{composite} of $f$ and $g.$
		\item For each object $A,$ there is given an arrow
		\[1_A : A \to A\]
		called the \emph{identity arrow} of $A.$
	\end{itemize}
	Additionally, we require the above data to follow the following laws:
	\begin{itemize}
		\item Associativity:
		\[h \circ (g \circ f) = (h \circ g) \circ f\]
		for all $f : A \to B, g: B \to C, h : C \to D.$
		\item Unit:
		\[f\circ 1_A = f = 1_B \circ f\]
		for all $f:A\to B.$
	\end{itemize}
\end{defn}
The above highly imitates the behaviour of that of sets and functions. Indeed, that shall be our first example of a category. Many of our other examples will also consist of categories where the objects are ``structured sets'' and arrows ``structure-preserving functions.'' However, these are not all. A category is \emph{anything} that satisfies the above definition.\\
Note that technically, it isn't correct to write $C \in \mathbf{C}$ or $(f:C \to D) \in \mathbf{C};$ however, we shall abuse notation and write so as the alternative is too annoying.
\subsection{Examples}
\begin{enumerate} 
	\item The category $\mathsf{Set}.$ The objects of this category are (all) sets and the arrows are (all) functions between sets. The composite of two arrows is defined as composition of functions. It can be verified that this is indeed a well defined composition. Lastly, given any object $X$ (a set), the identity arrow of $X$ is simply the identity function $1_X = \id_X:X\to X.$\\
	The above is all the data that's required to define a category. Now, it is be verified that $\mathsf{Set}$ is indeed a category with the above data. These are basic results from set theory and we leave this to the reader.
	\item We may also consider the category $\mathsf{Set}_{\text{fin}},$ the category consisting of finite sets and functions between them.\\
	In the same spirit, we may take other restrictions as well. For example, we may take only injective functions instead of functions. By noting that that the composition of two injections is again an injection and that the identity map is an injection, we get that this is indeed a category.\\
	Similarly, one may restrict the arrows to surjections.
	\item As mentioned before, many categories arise from other mathematical structures. In these categories, the objects are ``structured sets'' and arrows ``structure-preserving functions.'' The following table lists these.

	\begin{center}
		\begin{tabular}{|c|c|c|}
		\hline
		Category & Objects & Arrows (Morphisms)\\
		\hline
		$\mathsf{Top}$ & Topological spaces & Continuous maps\\
		$\mathsf{Mon}$ & Monoids & Monoid homomorphisms\\
		$\mathsf{Grp}$ & Groups & Group homomorphisms\\
		$\mathsf{Ring}$ & Rings & Ring homomorphisms\\
		$\mathsf{Field}$ & Fields & Field extensions\\
		$\mathsf{Vec}_\Bbbk$ & Vector spaces over $\Bbbk$ & Linear maps\\
		$\mathsf{Pos}$ & Posets & Order-preserving maps\\
		$\mathsf{BA}$ & Boolean algebra & Boolean homomorphisms\\
		\hline
		\end{tabular}
	\end{center}
	\item Preorders. A preorder is a set $P$ equipped with a binary relation $\le$ satisfying:
	\begin{itemize}
		\item $a \le a$ for all $a \in P,$ and
		\item $a \le b$ and $b \le c$ implies $a \le c$ for all $a, b, c \in P.$
	\end{itemize}
	That is, the relation is reflexive and transitive.\\
	Any preorder $P$ can be recognised as a category with objects being the elements of $P$ and a unique arrow 
	\begin{equation} \label{eq:precat}
		a \to b \text{ iff } a \le b.
	\end{equation}
	The compositions of arrows is defined in the way that will be forced from construction. One can verify that this is indeed a category.\\
	Note very carefully these this is quite different from the earlier examples where the objects were ``sets'' and  arrows were functions.\\
	Conversely, given a category where there is at most one arrow between any two objects, one may define a preorder using (\ref{eq:precat}).
	\item A poset is a preorder $P$ with the additional condition that $\le$ is antisymmetric, that is,
	\begin{equation*} 
		a \le b \text{ and } b \le a \implies a = b \text{ for all } a, b \in P.
	\end{equation*}
	With the same construction as before, we see that a poset is also a category.
	\item Finite categories. These are categories with finitely many arrows (and thus, finitely many objects as well). As the previous example shows posets to be categories, we can already get many examples. Let's look at some of them in particular.
	\begin{itemize}
		\item The category $\mathbf{1}$ looks like this:
		\begin{center}
			\begin{tikzcd} 
				*
			\end{tikzcd}
		\end{center}
		It has one object and its identity arrow, which we do not draw.
		\item The category $\mathbf{2}$ looks like this:
		\begin{center}
			\begin{tikzcd} 
				* \arrow[r] & \star
			\end{tikzcd}
		\end{center}
		It has two objects, their identity arrows, and exactly one arrow between them. Once again, the law of composition is clear.
		\item The category $\mathbf{3}$ looks like this:
		\begin{center}
			\begin{tikzcd} 
				* \arrow[r] \arrow[dr] 	& \star \arrow[d]\\
									 	& \odot
			\end{tikzcd}
		\end{center}
		This is the category recognised by the poset $\{1, 2, 3\}$ with the typical ordering.
		\item The category $\mathbf{0}$ looks like this:
		\begin{center}
			\begin{tikzcd}
				
			\end{tikzcd}
		\end{center}
		It has no objects and no arrows.\\
		It is easy to specify finite categories - just take some objects and start putting arrows between them, but makes sure to put in the necessary identities and composites, as required by the axioms for a category. Also, if there are any any loops, then the need to be cut off by equations to keep the category finite. For example, consider the following specification:
		\begin{equation} \label{diag:looptwo}
		\begin{tikzcd}
			A \arrow[rr, shift left = 1, "e"] && B \arrow[ll, shift left = 1, "f"]
		\end{tikzcd}
	\end{equation}
	Unless we stipulate an equation like $ef = 1_A,$ we will end up with infinitely many arrows $ef, efef, efefefef, \ldots$ This is still a category, just not a \emph{finite} one. We will see this situation later again when we discuss free categories. 
	\end{itemize}
	\item Let $X$ be a set. We can regard $X$ as a category $\mathbf{Dis}(X)$ by taking the objects to be the elements of $X$ and taking the arrows to just be the required identity arrows, one for each $x \in X.$ Such categories, in which the arrows are only identities, are called \emph{discrete}. These categories are just special posets.
\end{enumerate}
Before returning to more examples, we now define a ``structure preserving'' ``function" between categories.
\newpage
\subsection{Definition - Functors}
\begin{defn} 
	A functor \[F:\mathbf{C} \to \mathbf{D}\] between categories $\mathbf{C}$ and $\mathbf{D}$ is a mapping of objects (of $\mathbf{C}$) to objects (of $\mathbf{D}$) and arrows (of $\mathbf{C}$) to arrows (of $\mathbf{D}$), in such a way that
	\begin{enumerate}
		\item $F(f:A\to B) = F(f):F(A)\to F(B),$
		\item $F(1_A) = 1_{F(A)},$ and
		\item $F(g\circ f) = F(g)\circ F(f).$
	\end{enumerate}
\end{defn}
Thus, an arrow between objects gets mapped to an arrow between the images of the two objects. That is to say, a functor preserves domain and codomain. Similarly, it preserved identities and compositions.\\
This can be illustrated via the following picture:
\begin{center}
    \begin{tikzcd}
		& A \arrow[rr, "f"] \arrow[rrdd, "g \circ f"']&& B \arrow[dd, "g"]\\
		\mathbf{C} \arrow[dddd, "F"']&&&\\
		 &&& C\\
		&&&\\
		& F(A) \arrow[rr, "F(f)"] \arrow[rrdd, "F(g \circ f)"'] && F(B) \arrow[dd, "F(g)"] \\\mathbf{D}&&&\\
		&&& F(C)                  
	\end{tikzcd}
\end{center}
Functors compose in the expected way. Moreover, this composition is associative. Also, every category $\mathbf{C}$ has an identity functor $1_{\mathbf{C}} : \mathbf{C} \to \mathbf{C}.$\\
This gives us another category $\mathsf{Cat},$ the category with categories as objects and functors as arrows.\\
Let us now turn back to examples.
\subsection{Some more examples}
\begin{enumerate}
	\item We had previously seen that a poset $P$ is in fact a category. The following question is natural to ask: Given posets $P$ and $Q,$ what is a functor between them?\\
	As it turns out and one can easily see, a functor $F:P\to Q$ is precisely an order preserving function from $P$ to $Q.$\\
	The reader may compare this with the category $\mathsf{Pos}.$
	\item One may also consider monoids to be categories in the following way:\\
	A monoid is just a category with one element. The arrows of the category are the elements of the monoid. In particular, the identity arrow is the unit element. Composition of arrows is determined by the ``product" (binary operation) $m\cdot n$ of the monoid.\\~\\
	For any set $X,$ the set of functions from $X$ to $X,$ written as
	\begin{equation*} 
		\Hom_{\mathsf{Set}}(X, X)
	\end{equation*}
	is a monoid under the operation of composition. More generally, for any object $C$ in any category $\mathbf{C},$ the set of arrows from $C$ to $C,$ written as $\Hom_\mathbf{C}(C, C),$ is a monoid under the composition operation of $\mathbf{C}.$ The unit is the identity $1_C.$\\~\\
	Once again, we have a correspondence between the arrows in $\mathsf{Mon}$ and functors between monoids regarded as categories. The correspondence is that they are the same!
	\item \emph{Forgetful functors.}\\
	Consider the functor $U:\mathsf{Grp} \to \mathsf{Set}$ which maps a group to its underlying set and a group homomorphism to the corresponding function between sets. This is clearly a functor. This is called the ``forgetful functor'' as it ``forgets'' the structure of the group. Similarly, one has forgetful functors from various different categories where the objects are structured sets.
\end{enumerate}
\subsection{Isomorphisms}
\begin{defn} 
	In any category $\mathbf{C},$ an arrow $f:A\to B$ is called an \emph{isomorphism}, if there is an arrow $g:B\to A$, called the \emph{inverse} of $f$ such that
	\begin{equation*} 
		g\circ f = 1_A \text{ and } f\circ g = 1_B.
	\end{equation*}
\end{defn}
\begin{lem} 
	Inverses are unique.
\end{lem}
\begin{proof} 
	Let $f:A\to B$ be an arrow and let $g_1, g_2:B\to A$ be inverses of $f.$ Observe that
	\begin{equation*} 
		g_1 = g_1 \circ 1_B = g_1 \circ (f \circ g_2) = (g_1 \circ f) \circ g_2 = 1_A \circ g_2 = g_2.
	\end{equation*}
\end{proof}
Since inverses are unique, we write the inverse of $f$ as $f^{-1}.$ (This is, of course, in the case that $f$ is indeed an isomorphism.)\\
\begin{defn} 
	For objects $A, B$ of $\mathbf{C},$ we say that $A$ and $B$ are \emph{isomorphic} if there exists an isomorphism between them.\\
	This is denoted by writing $A \cong B.$
\end{defn}
\example{} Let us consider some familiar categories where the objects are structured sets.
\begin{enumerate}
	\item $\mathsf{Set}.$ Here, the isomorphisms are precisely the bijections.
	\item $\mathsf{Mon}, \mathsf{Grp}, \mathsf{Ring}, \mathsf{Field}, \mathsf{Vec}_\Bbbk.$ Here, the isomorphisms are precisely the (appropriate) homomorphisms\footnote{By ``appropriate,'' we mean that one must consider \emph{monoid} homomorphisms in the case of $\mathsf{Mon}$ and so on.} which are bijective.
	\item $\mathsf{Pos}.$ Here, the isomorphisms are \emph{\textbf{not}} the same as bijective order-preserving maps. While every isomorphism is indeed a bijective order-preserving map, the converse is not true.\\
	For example, consider the posets $P_1$ and $P_2$ defined on $\{a, b\}.$ Let the order on $P_1$ be the discrete one. Let the order on $P_2$ be $a \le b.$ Consider the identity function as an arrow from $P_1$ to $P_2.$ This is clearly a bijection that preserves the (trivial) order relations. However, this clearly is not an isomorphism of posets. (One may manually verify that the set-theoretic inverse is not an arrow in $\mathsf{Pos}$. Later, we shall see a method that utilises the structure.)
	\item $\mathsf{Top}.$ Once again, it is \textbf{\emph{not}} the case that continuous bijections are isomorphisms.
\end{enumerate}
The last two examples illustrate how isomorphisms need not always take the same form. The definition of isomorphism is our first example of an \emph{abstract}, category theoretic definition of an important notion. It is abstract in the sense that it makes use only of the category theoretic notions, rather than some additional information about the objects and arrows.\\
As the reader may be familiar that one often defines a ``group isomorphism'' to be a bijective homomorphism. This definition makes use of the following fact - the inverse of a bijective homomorphism is once again a homomorphism. The advantage of our definition is that applies in \emph{any} category.
%
\subsection{Categories - New from Old}
We now consider some constructions that help us create new categories from old.
\begin{enumerate}
	\item \emph{Product.} The product of two categories $\mathbf{C}$ and $\mathbf{D},$ written as
	\begin{equation*} 
		\mathbf{C} \times \mathbf{D}
	\end{equation*}
	has objects of the form $(C, D),$ for $C \in \mathbf{C}$ and $D \in \mathbf{D},$ and arrows of the form
	\begin{equation*} 
		(f, g) : (C, D) \to (C', D')
	\end{equation*}
	for $f: C \to C' \in \mathbf{C}$ and $g:D\to D\in \mathbf{D}.$ Composition and units are defined componentwise, that is,
	\begin{align*} 
		(f, g) \circ (f', g') &= (f\circ f', g\circ g'),\\
		1_{(C, D)} &= (1_C, 1_D).
	\end{align*}
	There are two obvious \emph{projection functors}
	\begin{equation*} 
		\begin{tikzcd}
			\mathbf{C} && \mathbf{C}\times\mathbf{D} \arrow[rr, "\pi_2"] \arrow[ll, "\pi_1"'] && \mathbf{D}
		\end{tikzcd}
	\end{equation*}
	defined by $\pi_1(C, D) = C$ and $\pi_1(f, g) = f,$ and similarly for $\pi_2.$\\
	As groups are monoids, one may recognise them as categories. In this case, the reader familiar with groups may recognise that for groups $G$ and $H,$ the product category $G\times H$ is the same as the (direct) group product $G \times H$ interpreted as a category.
	\item The \emph{opposite} (or ``dual'') category $\mathbf{C}\op$ of a category $\mathbf{C}$ has the same objects and arrow $f:C\to D$ in $\mathbf{C}\op$ is an arrow $f:D\to C$ in $\mathbf{C}.$ That is, $\mathbf{C}\op$ is just $\mathbf{C}$ with all of the arrows formally turned around.\\
	For easier notation, we shall write
	\begin{equation*} 
		f^* : D^* \to C^*
	\end{equation*}
	in $\mathbf{C}\op$ for $f:C\to D$ in $\mathbf{C}.$ With this notation, we have the units and composition rule as
	\begin{align*} 
		1_{C^*} &= (1_C)^*\\
		f^*\circ g^* &= (g \circ f)^*.
	\end{align*}
	Thus, a diagram in $\mathbf{C}$
	\begin{equation*} 
		\begin{tikzcd}
			A \arrow[rr, "f"] \arrow[rrdd, "g\circ f"'] && B \arrow[dd, "g"]\\
			&&\\
			&& C
		\end{tikzcd}
	\end{equation*}
	looks like this in $\mathbf{C}\op$
	\begin{equation*} 
		\begin{tikzcd}
			A^* && \arrow[ll, "f^*"'] B^* \\
			&&\\
			&& \arrow[lluu, "f^*\circ g^*"]\arrow[uu, "g^*"'] C^*
		\end{tikzcd}
	\end{equation*}
	\item The \emph{arrow category} $\mathbf{C}^\to$ of a category $\mathbf{C}$ has the arrows of $\mathbf{C}$ as objects and arrow $g$ from $f:A\to B$ to $f':A\to B'$ in $\mathbf{C}^\to$ is a ``commutative square''
	\begin{equation*} 
		\begin{tikzcd}
		A \arrow[dd, "f"'] \arrow[rr, "g_1"] &  & A' \arrow[dd, "f'"] \\
		                                     &  &                     \\
		B \arrow[rr, "g_2"']                 &  & B'                 
		\end{tikzcd}
	\end{equation*}
	where $g_1$ and $g_2$ are arrows in $\mathbf{C}.$ That is, such an arrow is a pair of arrows $g = (g_1, g_2)$ in $\mathbf{C}$ such that
	\begin{equation*} 
		g_2\circ f = f'\circ g_1.
	\end{equation*}
	(A possible confusion that may arise in the mind of the reader is - why should such a commutative square exist? However, note that we are not claiming the existence of such a square. It could very well be possible that there is no arrow between two given objects in $\mathbf{C}^\to$.)\\
	The identity arrow $1_f$ on an object $f:A\to B$ is the pair $(1_A, 1_B).$\\
	Composition of arrows is componentwise:
	\begin{equation*} 
		(h_1, h_2)\circ (g_1, g_2) = (h_1\circ g_1, h_2\circ g_2).
	\end{equation*}
	That this works can be verified using the diagram
	\begin{equation*} 
		\begin{tikzcd}
		A \arrow[dd, "f"'] \arrow[rr, "g_1"] &  & A' \arrow[dd, "f'"] \arrow[rr, "h_1"] &  & A'' \arrow[dd, "f''"] \\
		                                     &  &                                       &  &                       \\
		B \arrow[rr, "g_2"']                 &  & B' \arrow[rr, "h_2"']                 &  & B''                  
		\end{tikzcd}
	\end{equation*}
	\item The \emph{slice category} $\mathbf{C}/C$ of a category $\mathbf{C}$ over an object $C \in \mathbf{C}$ has
	\begin{itemize}
		\item Objects: all arrows $f \in \mathbf{C}$ such that $\cod(f) = C,$
		\item Arrows: an arrow $a$ from $f:X\to C$ to $f':X' \to C$ is an arrow $a: X\to X'$ in $\mathbf{C}$ such that $f'\circ a = f,$ as indicated in
		\begin{equation*} 
			\begin{tikzcd}
			X \arrow[rddd, "f"'] \arrow[rr, "a"] &     & X' \arrow[lddd, "f'"] \\
			                                       &     &                   \\
			                                       &     &                   \\
			                                       &  C  &                      
			\end{tikzcd}
		\end{equation*}
	\end{itemize}
	The identity arrows and compositions are induced from those of $\mathbf{C},$ just like the case of arrow category. Note that there is a functor $U:\mathbf{C}/C \to \mathbf{C}$ that ``forgets about the base object $C.$''\\
	If $g:C\to D$ is a any arrow, then there is a composition functor,
	\begin{equation*} 
		g_*:\mathbf{C}/C \to \mathbf{C}/D
	\end{equation*}
	defined by $g_*(f) = g\circ f,$ and similarly for the arrows in $\mathbf{C}/C.$ (The arrows just go to themselves.)\\
	The following diagram may be useful for the reader.
	\begin{equation*} 
		\begin{tikzcd}
X \arrow[rrdd, "f"'] \arrow[rrrr, "a"] \arrow[rrdddd, "g\circ f"', bend right] &%
&&&X' \arrow[lldd, "f'"] \arrow[lldddd, "g\circ f'", bend left] \\
&&&&\\&&C\arrow[dd, "g"] &&\\&&&&\\&&D&&                                                             
\end{tikzcd}
	\end{equation*}
	%
	To repeat, we saw that given an object $C \in \mathbf{C},$ we get a category $\mathbf{C}/C.$ Moreover, given an arrow $g:C\to D\in \mathbf{C},$ we get a functor $g_*:\mathbf{C}/C \to \mathbf{C}/D.$\\
	Recalling that categories and functors are nothing but objects and arrows in $\mathsf{Cat},$ this suggests that the above construction is in fact a functor. In fact, this is true as the reader may verify
	\begin{equation*} 
		\mathbf{C}/(-) : \mathbf{C} \to \mathsf{Cat}
	\end{equation*}
	to be a functor. 
	If $\mathbf{C} = \mathbf{P}$ is a poset category and $p \in \mathbf{P},$ then
	\begin{equation*} 
		\mathbf{P}/p \cong\ {\downarrow}(p),
	\end{equation*}
	that is, the slide category $\mathbf{P}/p$ is just the \emph{principal ideal} ${\downarrow}(p)$ consisting of elements $q \in \mathbf{P}$ with $q \le p.$
	\item Similarly, the \emph{coslice} category $C/\mathbf{C}$ of a category $\mathbf{C}$ under an object $C$ of $\mathbf{C}$ has as objects all arrows $f$ of $\mathbf{C}$ such that $\dom(f) = C,$ and an arrow from $f:C\to X$ to $f':C'\to X$ is an an arrow $h:X\to X'$ such that $h\cdot f = f'.$
	\begin{equation*} 
		\begin{tikzcd}
		X \arrow[rr, "h"] &                                         & X' \\
						  &											&	 \\
		                  &                                         &    \\
		                  &  C \arrow[luuu, "f"] \arrow[ruuu, "f'"']  &   
		\end{tikzcd}
	\end{equation*}
\end{enumerate}
\example{}The category $\mathsf{Set}_*$ of \emph{pointed sets} consists of sets $A$ with distinguished elements $a \in A,$ and arrows $f:(A, a) \to (B, b)$ are functions $f:A\to B$ that preserves the ``points,'' $f(a) = b.$ This is isomorphic to the coslice category,
\begin{equation*} 
	\mathsf{Set}_* \cong 1/\mathsf{Set}
\end{equation*}
of $\mathsf{Set}$ under any singleton $1 = \{*\}.$ To see this, note that the functions $a:1\to A$ correspond uniquely to elements, $a(*) = a \in A,$ and arrows $f:(A, a) \to (B, b)$ correspond exactly to commutative triangles:
\begin{equation*} 
	\begin{tikzcd}
1 \arrow[rr, "a"] \arrow[rrdd, "b"'] &  & A \arrow[dd, "f"] \\
                                     &  &                   \\
                                     &  & B                
\end{tikzcd}
\end{equation*}
\subsection{Free categories}\label{ssec:free}
\textbf{Free monoid.} First we look at the concept of a free monoid.\\
First, we start with a set $A$ which we shall call an ``alphabet.'' We shall denote its elements as $a, b, c, \ldots$ and call them ``letters,'' that is,
\begin{equation*} 
	A = \{a, b, c, \ldots\}.
\end{equation*}
A \emph{word} over $A$ is a finite sequence of letters:
\begin{equation*} 
	aryaman,\;integral,\;lenny,\;face,\;alkmdslkd,\ldots.
\end{equation*}
We write $\varepsilon$ for the empty word, that is, the unique word over $A$ of length zero. The ``Kleene closure'' of $A$ is defined to be the set 
\begin{equation*} 
	A^* = \{\text{words over }A\}.
\end{equation*}
The above can easily be made into a monoid with composition (denoted by $*$) as concatenation. Accordingly, $\varepsilon$ is the unit. Thus, $A^*$ is a monoid, called the \emph{free monoid} over the set $A.$\\
Any element $a \in A$ can also be regarded as a word of length one and hence, we have a function
\begin{equation*} 
	i:A \to A^*
\end{equation*}
defined by $i(a) = a,$ and called the ``insertion of generators.'' The elements of $A$ ``generate'' the free monoid, in the sense that every $w \in A^*$ is a $*-$product of $a$s, that is, $w = a_1*a_2*\cdots *a_n$ for some $a_1, \ldots, a_n \in A.$ One usually defines the property of being ``free'' in the following manner:\\
A monoid $M$ is freely generated by $A \subset M,$ if the following conditions hold:
\begin{enumerate}
	\item Every element $m \in M$ can be written as a product of elements of $A:$
	\begin{equation*} 
		m = a_1 \cdot_M \cdots \cdot_M a_n, \quad a_i \in A.
	\end{equation*}
	\item No ``nontrivial'' relations hold in $M,$ that is, if $a_1\cdots a_j = a_1'\cdots a_k',$ then this is required by the axioms for monoids.
\end{enumerate}
The second condition might seem a little vague and imprecise. We give a precise definition of ``free'' - capturing what is meant in the above - which avoids vagueness.\\
First, every monoid $N$ has an underlying set $|N|,$ and every monoid homomorphism $f:N\to M$ has an underlying function $|f|:|N| \to |M|.$ This is nothing but the action of the forgetful functor seen earlier. ``The'' free monoid on a set $A$ is, by definition, ``the'' monoid with the following so-called \emph{universal mapping property} or UMP.\\\\
\emph{Universal Mapping Property of }$M(A)$\\
There is a function $i:A\to |M(A)|,$ and given any monoid $N$ and any function $f:A\to |N|,$ there is a \emph{unique} monoid homomorphism $\bar{f}:M(A) \to N$ such that $|\bar{f}|\circ i = f,$ all as indicated in the following diagram:\\
in $\mathsf{Mon}$:
\begin{equation*} 
	\begin{tikzcd}
		M(A) \arrow[rr, dotted, "\bar{f}"] && N
	\end{tikzcd}
\end{equation*}
in $\mathsf{Set}$:
\begin{equation*} 
	\begin{tikzcd}
	{\vert}M(A){\vert} \arrow[rr, "{\vert}\bar{f}{\vert}"]&  & {\vert}N{\vert} \\
	                                     &  &     \\
	A \arrow[uu, "i"] \arrow[rruu, "f"'] &  &    
	\end{tikzcd}
\end{equation*}
The reader is encouraged to prove the following proposition on their own.
\begin{prop}
	$A^*$ has the UMP of the free monoid on $A.$
\end{prop}
\begin{proof} 
	Given $f:A \to |N|,$ define $f:A^* \to N$ by
	\begin{align*} 
		f(\varepsilon) = u_N, \text{ the unit of }N,\\
		f(a_1\cdots a_i) = f(a_1)\cdot_N\cdots\cdot_Nf(a_i).
	\end{align*}
	Then, $\bar{f}$ is clearly a homomorphism with 
	\begin{equation*} 
		\bar{f}(a) = f(a) \quad \text{for all } a \in A.
	\end{equation*}
	Now, we prove the uniqueness of $\bar{f}.$\\
	If $g:A^* \to N$ also satisfies $g(a) = f(a)$ for all $a \in A,$ then for all $a_1\cdots a_i \in A^*:$
	\begin{align*} 
		g(a_1\cdots a_i) &= g(a_1*\cdots *a_i)\\
		&= g(a_1)\cdot_N\cdots\cdot_Ng(a_i)\\
		&= f(a_1)\cdot_N\cdots\cdot_Nf(a_i)\\
		&= \bar{f}(a_1)\cdot_N\cdots\cdot_N\bar{f}(a_i)\\
		&= \bar{f}(a_1)\cdot_N\cdots\cdot_Ng(a_i)\\
		&= \bar{f}(a_1\cdots a_i)
	\end{align*}
	Thus, $g = \bar{f}$ and hence, uniqueness is proved.
\end{proof}
Using the UMP, it is also easy to show that the free monoid $M(A)$ is determined uniquely, up to isomorphism, in the following sense.
\begin{prop}
	Given monoids $M$ and $N$ with functions $i:A\to |M|$ and $j:A\to|N|,$ each with the UMP of the free monoid of $A,$ there is a (unique) monoid isomorphism $h:M\to N$ such that $|h|i = j$ and $|h^{-1}|j = i.$
\end{prop}
Once again, the reader is encouraged to prove this on their own as it's a fun exercise.
\begin{proof} 
	Using $j$ and the UMP of $M,$ we get a monoid homomorphism $\bar{j}:M \to N$ with the following diagrams:\\
	in $\mathsf{Set}$:
	\begin{equation*} 
		\begin{tikzcd}
			M \arrow[rr, dotted, "\bar{j}"] && N
		\end{tikzcd}
	\end{equation*}
	in $\mathsf{Mon}$:
	\begin{equation*} 
		\begin{tikzcd}
{\vert}M{\vert} \arrow[rr, "{\vert}\bar{j}{\vert}"] &  & {\vert}N{\vert}                     \\
                                                    &  &                                     \\
                                                    &  & A \arrow[uu, "j"] \arrow[lluu, "i"]
\end{tikzcd}
	\end{equation*}
	On the other hand, using $i$ and the UMP of $N$ we get a monoid homomorphism $\bar{i}:N \to M$ with the following diagrams:\\
	in $\mathsf{Set}$:
	\begin{equation*} 
		\begin{tikzcd}
			N \arrow[rr, dotted, "\bar{j}"] && M
		\end{tikzcd}
	\end{equation*}
	in $\mathsf{Mon}$:
	\begin{equation*} 
		\begin{tikzcd}
{\vert}N{\vert} \arrow[rr, "{\vert}\bar{i}{\vert}"] &  & {\vert}M{\vert} \\
                                                    &  &                 \\
A \arrow[uu, "j"] \arrow[rruu, "i"']                &  &                
		\end{tikzcd}
	\end{equation*}
Composing the above arrows gives us the diagrams:
in $\mathsf{Set}$:
	\begin{equation*} 
		\begin{tikzcd}
			M \arrow[rr, dotted, "\bar{j}"] && N \arrow[rr, dotted, "\bar{j}"] && M
		\end{tikzcd}
	\end{equation*}
	in $\mathsf{Mon}$:
	\begin{equation*} 
		\begin{tikzcd}
{\vert}M{\vert} \arrow[rr, "{\vert}\bar{j}{\vert}"] &  & {\vert}N{\vert} \arrow[rr, "{\vert}\bar{i}{\vert}"]    &  & {\vert}M{\vert} \\
                                                    &  &                                                        &  &                 \\
                                                    &  & A \arrow[uu, "j"] \arrow[rruu, "i"'] \arrow[lluu, "i"] &  &                
\end{tikzcd}
	\end{equation*}
	Now, look at the monoid homomorphism $\bar{i}\circ\bar{j}:M\to M.$\\
	It has the property that $|\bar{i}\circ\bar{j}|i = i.$ As $1_M:M\to M$ has this property, by the uniqueness part of the UMP of $M,$ it follows that $\bar{i}\circ\bar{j} = 1_M.$ Exchanging the roles of $M$ and $N$ shows that $\bar{j}\circ\bar{i} = 1_N.$\\
	This finishes our proof.
\end{proof}
(How do we get that this isomorphism is unique?)\\~\\
\textbf{Free category.} Now, we want to do the same thing for categories in general. Instead of underlying sets, categories have underlying graphs, so let us review these first.\\
A \emph{directed graph} consists of vertices and edges, each of which is directed, that is, each edge has a ``source'' and a ``target'' vertex.
\begin{equation*} 
	\begin{tikzcd}
A \arrow[dd, "u"']                   &  & B                  \\
                                     &  &                    \\
C \arrow[rr, "x"'] \arrow[rruu, "v"] &  & D \arrow[uu, "w"']
\end{tikzcd}
\end{equation*}
We draw graphs just like categories, but there is no composition of edges, and there are no identities.\\
Thus, a graph consists of two sets, $E$ (edges) and $V$ (vertices), and two functions, $s:E\to V$ (source) and $t:E\to V$ (target). Thus, in $\mathsf{Set},$ a graph is just a configuration of objects and arrows of the form
\begin{equation*} 
	\begin{tikzcd}
		E \arrow[rr, "t"', shift right] \arrow[rr, "s", shift left] &  & V
	\end{tikzcd}
\end{equation*}
Now, every graph ``generates'' a category, the \emph{free category} on $G.$ This is similar in spirit to the construction of the free monoid on a set. There we created the words by writing letters one after the other. We do the same here but adjoining arrows only if the source and target matches. To be more precise, the free category $\mathbf{C}(G)$ is defined by taking the vertices of $G$ as objects, and the \emph{paths} in $G$ as arrows, where a path is a finite sequence of edges $e_1, \ldots, e_n$ such that $t(e_i) = s(e_{i+1}),$ for all $i = 1, \ldots, n.$ We write the arrows of $\mathbf{C}(G)$ in the form $e_ne_{n-1}\ldots e_1.$
\begin{equation*} 
	\begin{tikzcd}
		v_0 \arrow[rr, "e_1"] && v_1 \arrow[rr, "e_2"] && \ldots \arrow[rr, "e_n"] && v_n
	\end{tikzcd}
\end{equation*}
Put
\begin{align*} 
	\dom(e_n\ldots e_1) &= s(e_1)\\
	\cod(e_n\ldots e_1) &= t(e_n)
\end{align*}
and define composition by concatenation:
\begin{equation*} 
	e_n\ldots e_1 \circ e_m' \ldots e_1' = e_n\ldots e_1e_m'\ldots e_1'.
\end{equation*}
For each vertex $v,$ we have an ``empty path'' denoted by $1_v,$ which is to be the identity arrow at $v.$\\
We will show that the $\mathbf{C}(G)$ so defined also has a UMP. Before that, we make a slight digression.\\
First, we observe that any category $\mathbf{C}$ can be described with a diagram like this:
\begin{equation*} 
	\begin{tikzcd}
	C_2 \arrow[rr, "\circ"] &  & C_1 \arrow[rr, "\cod", shift left=2] \arrow[rr, "\dom"', shift right=2] &  & C_0 \arrow[ll, "i" description]
	\end{tikzcd}
\end{equation*}
where $C_0$ is the collection of objects of $\mathbf{C},$ $C_1$ the arrows, $i$	 is the identity operation, and $C_2$ is the collection $\{(f, g) \in C_1 \times C_1 : \cod(f) = \dom(g)\}.$
Then, a functor $F:\mathbf{C} \to \mathbf{D}$ from $\mathbf{C}$ to another category $\mathbf{D}$ is a pair of functions
\begin{align*} 
	F_0 &: C_0 \to D_0\\
	F_1 &: C_1 \to D_1
\end{align*}
such that each similarly labeled square in the following diagram commutes:
\begin{equation} \label{diag:func}
	\begin{tikzcd}
C_2 \arrow[rr, "\circ"] \arrow[ddd, "F_2"'] &  & C_1 \arrow[rr, "\cod", shift left=2] \arrow[rr, "\dom"', shift right=2] \arrow[ddd, "F_1"'] &  & C_0 \arrow[ll, "i" description] \arrow[ddd, "F_0"] \\ &&&&\\ &&&&\\
D_2 \arrow[rr, "\circ"']&  & D_1 \arrow[rr, "\cod", shift left=2] \arrow[rr, "\dom"', shift right=2]&  & D_0 \arrow[ll, "i" description]                  
\end{tikzcd}
\end{equation}
where $F_2(f, g) = (F_1(f), F_1(g)).$\\
Now let us describe a \emph{homomorphism of graphs},
\begin{equation*} 
	h:G \to H.
\end{equation*}
We need a pair of functions $h_0 : G_0 \to H_0, h_1 : G_1 \to H_1$ making the two squares commute (once with $t$s and once with $s$s) in the following diagram commute:
\begin{equation} \label{diag:graphhomo}
	\begin{tikzcd}
		G_1 \arrow[rr, "t", shift left = 1] \arrow[rr, "s"', shift right = 1] \arrow[ddd, "h_1"'] && G_0 \arrow[ddd, "h_0"]\\
		&&\\
		&&\\
		H_1 \arrow[rr, "t", shift left = 1] \arrow[rr, "s"', shift right = 1] && H_0
	\end{tikzcd}
\end{equation}
With this in place, we can now describe the forgetful functor
\begin{equation*} 
	U:\mathsf{Cat} \to \mathsf{Graphs}
\end{equation*}
as sending the category
\begin{equation*} 
	\begin{tikzcd}
	C_2 \arrow[rr, "\circ"] &  & C_1 \arrow[rr, "\cod", shift left=2] \arrow[rr, "\dom"', shift right=2] &  & C_0 \arrow[ll, "i" description]
	\end{tikzcd}
\end{equation*}
to the underlying graph
\begin{equation*} 
	\begin{tikzcd}
	C_1 \arrow[rr, "\cod", shift left=2] \arrow[rr, "\dom"', shift right=2] &  & C_0.
	\end{tikzcd}
\end{equation*}
And similarly, the effect of $U$ is described functors by erasing some of the arrows of (\ref{diag:func}) to get a diagram like (\ref{diag:graphhomo}).\\
Recall how we had defined the UMP of free monoid using the forgetful functor. We shall do the same in this case. Borrowing the same notation, we shall write $|\mathbf{C}| = U(\mathbf{C}),$ et cetera, for the underlying graph of a category $\mathbf{C}.$\\
The free category on a graph now has the following UMP.\\\\
\emph{Universal mapping property of }$\mathbf{C}(G).$\\
There is a graph homomorphism $i:G \to |\mathbf{C}(G)|,$ and given any category $\mathbf{D}$ and any graph homomorphism $h:G \to |\mathbf{D}|,$ there is a unique functor $\bar{h}:\mathbf{C}(G) \to \mathbf{D}$ with $|\bar{h}|\circ i = h.$\\
in $\mathsf{Cat}$:
\begin{equation*} 
	\begin{tikzcd}
		\mathbf{C}(G) \arrow[rr, dotted, "\bar{h}"] && \mathbf{D}
	\end{tikzcd}
\end{equation*}
in $\mathsf{Graph}$:
\begin{equation*} 
	\begin{tikzcd}
	{\vert}\mathbf{C}(G){\vert} \arrow[rr, "{\vert}\bar{h}{\vert}"] &  & {\vert}\mathbf{D}{\vert} \\&  &\\&  &\\
	G \arrow[uuu, "i"] \arrow[rruuu, "h"']&&                        
	\end{tikzcd}
\end{equation*}
\example{} The free category on a graph with just one vertex is just a free monoid on the set of edges. The free category on a graph with only vertices (no edges) is the discrete category on the set of vertices of $G.$ The free category on a graph with two vertices and one edge between them is the finite category $\mathbf{2}.$ The free category on a graph of the form
\begin{equation*} 
	\begin{tikzcd}
		A \arrow[rr, shift left = 1, "e"] && B \arrow[ll, shift left = 1, "f"]
	\end{tikzcd}
\end{equation*}
has (in addition to the identity arrows) the infinitely many arrows:
\begin{equation*} 
	e, f, ef, fe, efe, fef, \ldots
\end{equation*}
Recall we had seen the above graph earlier ((\ref{diag:looptwo})) when we were discussing finite categories.
\sec{Abstract Structures}
\subsection{Epis and Monos}
\begin{defn} 
	In any category $C,$ an arrow
	\begin{equation*} 
		f:A\to B
	\end{equation*}
	is called a 
	\begin{itemize}
		\item \emph{monomorphism}, if given any $g, h:C\to A,$ $fg = fh$ implies $g = h.$
		\item \emph{epimorphism}, if given any $i, j:B\to D,$ $if = jf$ implies $i = j.$
	\end{itemize}
	\begin{equation*} 
		\begin{tikzcd}
		C \arrow[rr, "h"', shift right] \arrow[rr, "g", shift left] &  & A \arrow[rr, "f"] &  & B \arrow[rr, "i", shift left] \arrow[rr, "j"', shift right] &  & D
		\end{tikzcd}
	\end{equation*}
\end{defn}	
We often write $f:A\mono B$ if $f$ is a monomorphism and $f:A \epi B$ if $f$ is an epimorphism.
\begin{prop} \label{prop:monin}
	A function $f:A\to B$ between is monic iff $f$ is injective.
\end{prop}
\begin{proof} 
	$(\implies)$ Suppose that $f$ is monic. We show that $f$ is injective. \\
	Let $a, a' \in A$ be such that $f(a) = f(a').$\\
	Consider $g:A \to A$ defined as
	\begin{align*} 
		g(x) = \begin{cases}
			x & x \neq a\\
			a' & x = a
		\end{cases}
	\end{align*}
	and let $h:A \to A = 1_A.$\\
	Clearly, one sees that $fg = f = f1_A.$ As $f$ is monic, we have that $1_A = g.$ In particular, $g(a) = 1_A(a)$ which yields $a' = a.$\\
	$(\impliedby)$ Suppose that $f$ in injective. We show that $f$ is monic.\\
	Let $g, h:C\to A$ be arrows such that $fg = fh.$ Let $a \in A$ be arbitrary. Then, $fg(a) = fh(a).$ As $f$ is injective, this yields $g(a) = h(a).$
\end{proof}
Before going ahead, we may make the following observation for proving that $f:A\to B$ is injective.
\begin{prop} \label{prop:injec}
	Let $f:A\to B$ be a function. Let $1 = \{*\}$ be any one-element set.\\
	Suppose that $fg \neq fh$ whenever $g, h: 1 \to A$ are distinct functions.\\
	Then, $f$ is injective.
\end{prop}
\begin{proof} 
	Let $a, a' \in A$ be such that $a \neq a'.$ Consider the functions
	\begin{equation*} 
		\bar{a}, \bar{a'} : 1 \to A
	\end{equation*}
	where
	\begin{equation*} 
		\bar{a}(*) = a, \quad \bar{a'}(x) = a'.
	\end{equation*}
	Since $\bar{a} \neq \bar{a'},$ it follows from our hypothesis that $f\bar{a} \neq f\bar{a'}.$ Thus, $f(a) = (f\bar{a})(x) \neq (f\bar{a'})(x) = f(a').$ Hence, it follows that $f$ is injective.
\end{proof}
Using this proposition, one may give a slightly simpler proof of $(\implies)$ of Proposition \ref{prop:monin}.\\
\example{} In many categories of ``structured sets'' like monoids, the monos are exactly the ``injective homomorphisms''. More precisely, a homomorphism $h:M\to N$ is monic precisely if the underlying function $|h|:M\to N$ is injective. (By the above, it is the same as saying $|h|$ is monic.)\\
To see that $|h|$ being injective $\implies$ $h$ is monic, one may consider the earlier proof.\\
Conversely, let $h:M \to N$ be monic. We show that $|h|$ is monic. \\
Suppose $x, y:1 \to |M|$ are two different ``elements'' (arrows), where $1 = \{*\},$ any one-element set. By Proposition \ref{prop:injec}, it suffices to prove that $|h|x, |h|y : 1 \to N$ are also distinct.\\
By the UMP of the free monoid $M(1),$ there are distinct corresponding homomorphisms $\bar{x}, \bar{y}:M(1) \to M,$ with distinct composites $h\circ \bar{x}, h\circ \bar{y}:M(1) \to N,$ since $h$ is monic. Thus, the corresponding ``elements'' $|h|\circ x, |h|\circ y:1 \to N$ are also distinct, again by the UMP of $M(1).$\\\\
\example{} In a poset $\mathbf{P},$ every arrow $p \le q$ is both monic and epic. This follows trivially from the fact that given any two objects, there is at most one arrow from one to the other.\\\\
Dually to the foregoing, the epis in $\mathsf{Set}$ are exactly the surjective functions. This is a fun exercise that is left to the reader. However, unlike the previous case in $\mathsf{Mon},$ the epis are not just the surjective homomorphisms!\\
\example{\label{eg:monepi}} In the category $\mathsf{Mon}$ of monoids and monoid homomorphisms, there is a monoid homomorphism
\begin{equation*} 
	\mathbb{N} \mono \mathbb{Z}
\end{equation*}
where $\mathbb{N}$ is the additive monoid of nonnegative integers and $\mathbb{Z}$ is the additive monoid of integers. We show that this map, given by the inclusion, is also epic in $\mathsf{Mon}$ by showing that the following holds:\\
Given any monoid homomorphisms $f, g: (|\mathbb{Z}|, +, 0) \to (M, *, u),$ if the restrictions to $|\mathbb{N}|$ are equal, then $f = g.$\\
The restrictions to $|\mathbb{N}|$ being equal already tells us that $f(n) = g(n)$ whenever $n \ge 0.$ Let us assume that $n < 0.$ One then notes

\begin{align*} 
	f(n) &= f(n)*u\\
	&= f(n)*g(0)\\
	&= f(n)*g(-n + n)\\
	&= f(n)*g(-n)*g(n)\\
	&= f(n)*f(-n)*g(n)\\
	&= f(0)*g(n)\\
	&= u*g(n)\\
	&= g(n)
\end{align*}
From an algebraic point of view, a morphism $e$ is epic if and only if $e$ cancels on the right: $xe = ye$ implies $x = y.$ Dually, $m$ is monic if and only if $m$ cancels on the left: $mx = my$ implies $x = y.$\\
Note that, of course, this does not mean that either $e$ or $m$ is invertible. (Even if they're cancel-able from both sides.)\\
However, \emph{if} a morphism \emph{is} invertible, then it is clearly a mono and an epi. This leads to the next proposition.
\begin{prop}
	Every iso is both monic and epic.
\end{prop}
\begin{proof} 
	Let $e$ be an isomorphism with $m$ as its inverse. Let $x$ and $y$ be arrows such that $xe = ye.$ Then, one has $xem = yem$ or $x = y,$ as desired. Similarly, $ef = eg$ implies that $f = g.$ 
\end{proof}
In $\mathsf{Set},$ the converse also holds: every mono-epi is iso. But this is not true in the general case, as Example \ref{eg:monepi} showed us.

\subsubsection{Sections and retractions}
We just noted that any iso is both monic and epic. However, carefully looking at the above proof tells us the following more general result:\\
Let $f:A\to B$ and $g:B\to A$ be arrows such that $gf = 1_A.$ Then, $f$ is monic and $g$ epic.\\
This leads to the following definition.
\begin{defn} 
	A \emph{split} mono (epi) is an arrow with a left (right) inverse. Given arrows $e:X \to A$ and $s:A \to X$ such that $es = 1_A,$ the arrow $s$ is called a \emph{section} or a \emph{splitting} of $e,$ and the arrow $e$ is called a \emph{retraction} of $s.$ The object $A$ is called a \emph{retract} of $X.$
\end{defn}
Clearly every \emph{split} mono (epi) is also a mono (epi) but the converse is not true in general.\\
Since functors preserve identities, they also preserve \emph{split} epis and monos. Compare this with Example \ref{eg:monepi} where the forgetful functor $\mathsf{Mon}\to \mathsf{Set}$ clearly does not preserve the epi $\mathbb{N}\mono \mathbb{Z}.$ (Thus, this also serves as an example of an epi that does not split.)\\
\example{} In $\mathsf{Set},$ every mono splits except those of the form
\begin{equation*} 
	\emptyset \mono A.
\end{equation*}
To see this, let $f:X \mono A$ be a mono with $X \neq \emptyset.$ Pick some $x_0 \in X$ and define $g:A \epi X$ as
\begin{equation*} 
	g(a) = \begin{cases}
		b & a = f(b)\\
		x_0	& a \neq f(b) \text{ for any } b \in X
	\end{cases}
\end{equation*}	
In view of $f$ being monic (and hence, injective), the above function is well defined and the reader may verify that $gf = 1_X.$\\
Let us look at the more interesting condition of an epi splitting. (Recall \nameref{axiomofchoice}.)
\begin{lem} \label{lem:episplit}
	In $\mathsf{Set},$ the condition that every epi splits is equivalent to the axiom of choice.
\end{lem}
\begin{proof} 
	First, let us assume the axiom of choice. Consider an epi
	\begin{equation*} 
		e : E \epi X.
	\end{equation*}
	We have the following nonempty collection of nonempty sets:
	\begin{equation*} 
		E_x = e^{-1}\{x\}, \quad x \in X.
	\end{equation*}
	A choice function for this family $(E_x)_{x \in X}$ is exactly a splitting of $e,$ that is, a function $s:X \to E$ such that $es = 1_X,$ since that means that $s(x) \in E_x$ for all $x \in X.$\\
	Conversely, assume that every epi splits. Given a nonempty collection of nonempty sets,
	\begin{equation*} 
		(E_x)_{x\in X}
	\end{equation*}
	take $E = \{(x, y) \mid x \in X, y \in E_x\}$ and define the epi $e:E\epi X$ by $(x, y) \mapsto x.$ A splitting $s$ of $e$ then determines a choice function for the collection.
\end{proof}
\begin{cor}
	The Axiom of Choice is equivalent to the following statement:\\
	Given any surjective map $s:A \epi B,$ there exists a map $f:B\to A$ such that $sf = 1_B.$
\end{cor}
\subsubsection{Projective objects}
A notion related to the existence of ``choice functions'' is that of being ``projective.''
\begin{defn} \label{def:projec}
	An object $P$ is said to be \emph{projective} if for any epi $e:E\epi X$ and arrow $f:P\to X$ there is some (not necessarily unique) arrow $\bar{f}:P \to E$ such that $e\circ \bar{f} = f,$ as indicated in the following diagram:
\end{defn}
\begin{equation} \label{diag:projec}
	\begin{tikzcd}
		&  & E \arrow[ddd, "e", two heads] \\&&\\&&\\
		P \arrow[rr, "f"'] \arrow[rruuu, "\bar{f}", dotted] &  & X
	\end{tikzcd}
\end{equation}
One says that $f$ \emph{lifts across} $e.$
\begin{prop}
	Any epi into a projective object splits.
\end{prop}
\begin{proof} 
	Let $P$ be a projective object and $e:E \epi P$ an epi. Consider (\ref{diag:projec}) with $X = P$ and $f = 1_P.$ The lift $\overline{1_P}$ is clearly a right inverse of $e.$
\end{proof}
\begin{prop}
	The axiom of choice implies that all sets are projective.
\end{prop}
\begin{proof} 
	Consider $E, P, X, e, f$ as in the Definition \ref{def:projec}. By Lemma \ref{lem:episplit}, we see that there exists $s:X \to E$ such that $es = 1_X.$ One sees that $\bar{f} = sf$ has the desired property.
\end{proof}
\begin{prop}
	Any retract of a projective object is also projective.
\end{prop}
\begin{proof} 
	Let $P$ be a projective object and $R$ a retract of $P.$ Let $p, s$ be as pictured.
	\begin{equation*} 
		\begin{tikzcd}
		R \arrow[rr, "s", tail] \arrow[rrddd, "1"'] &  & P \arrow[ddd, "p", two heads] \\&&\\&&\\&&R                            
		\end{tikzcd}
	\end{equation*}
	Let $e: E \epi X$ be an epi and let $f:R\to X$ be an arrow. We show that $e$ lifts across $f.$ The information so far can be pictured as follows:
	\begin{equation*} 
		\begin{tikzcd}
			R \arrow[rr, "s", tail] \arrow[rrddd, "1"'] &%
			& P \arrow[ddd, "p", two heads] &  & E \arrow[ddd, "e", two heads] \\
			&&&&\\&&&&\\&&R \arrow[rr, "f"]&&X
		\end{tikzcd}	
	\end{equation*}
	Now, since $P$ is projective, we get that $e$ lifts across $fp:P\to E.$ Let this lift be $\overline{fp}$ as shown.
	\begin{equation*} 
		\begin{tikzcd}
			R \arrow[rr, "s", tail] \arrow[rrddd, "1"'] &%
			& P \arrow[ddd, "p", two heads] \arrow[rr, "\overline{fp}", dotted]  &  & E \arrow[ddd, "e", two heads] \\
			&&&&\\&&&&\\&&R \arrow[rr, "f"]&&X
		\end{tikzcd}	
	\end{equation*}
	Then, $\bar{f} = \overline{fp}s$ is the desired lift of $e$ across $f.$ To see this, one observes that the above diagram commutes, that is,
	\begin{align*} 
		e(\overline{fp}s) &= (e \overline{fp})s\\
		&= (fp)s\\
		&= f(ps)\\
		&= f(1)\\
		&= f.
	\end{align*}
\end{proof}
%
%
\subsection{Initial and terminal objects}
\begin{defn} 
	In any category $\mathbf{C},$ an object
	\begin{itemize}
		\item $0$ is \emph{initial} if for any object $C$ there is a unique morphism,
		\begin{equation*} 
			0 \to C,
		\end{equation*}
		\item $1$ is \emph{terminal} if for any object $C$ there is a unique morphism,
		\begin{equation*} 
			C \to 1.
		\end{equation*}
	\end{itemize}
\end{defn}
One may note that a terminal object in $\mathbf{C}$ is precisely an initial object in $\mathbf{C}\op.$ Note that a category need not have an terminal or initial object. A category may also have one more than of either. However, the following proposition tells us that they must be isomorphic.
\begin{prop}
	Initial (terminal) objects are unique up to isomorphism.
\end{prop}
\begin{proof} 
	We shall prove the statement for initial objects.\\
	Let $C$ and $C'$ be initial objects. Let $i:C\to C'$ and $i':C'\to C$ be the unique morphisms between these objects.\\
	Consider the morphism $i\circ i'.$ It is a morphism from $C'$ to $C'.$ Note that $1_{C'}$ is a morphism from $C'$ to $C'.$ As $C'$ is an initial object, there is only one morphism from $C'$ to itself. Thus, $i\circ i' = 1_{C'}.$ A similar argument shows that $i' \circ i = 1_C$ and thus, $i$ is an isomorphism.
\end{proof}
\example{}
\begin{enumerate}
	\item In $\mathsf{Set},$ the empty set is initial and any singleton set $\{x\}$ is terminal. Observe that $\mathsf{Set}$ has just one initial object but many terminal objecs.
	\item In $\mathsf{Cat},$ the category $\mathbf{0}$ (no objects and no arrows) is initial and the category $\mathbf{1}$ (one object and its identity arrow) is terminal.
	\item In $\mathsf{Grp},$ the one-element group is both initial and terminal. (Same for $\mathsf{Vec}_\Bbbk$ and $\mathsf{Mon}.$) But in $\mathsf{Ring}$ (commutative with unit), the ring $\mathbb{Z}$ is initial and the one-element ring is terminal.
	\item Recall \nameref{boolalg}.
	Another familiar example of the two-element Boolean algebra $\mathbf{2} = \{0, 1\}$ (which may be taken to be the power-set $\mathcal{P}(1)$). It is an initial object in the category $\mathsf{BA}$ of Boolean algebras and boolean homomorphisms.\\
	The one-element structure (i.e., $\mathcal{P}(0)$) is terminal
	\item In a poset (viewed as a category), an object is initial iff is the least and terminal iff is the greatest. Thus, for instance, any Boolean algebra has both. (Note that we are viewing the Boolean algebra as a category in itself. Different from what we did in the previous example.)\\
	On the other hand, the poset $(\mathbb{Z}, \le)$ has neither.
	\item For any category $\mathbf{C}$ and any object $X \in \mathbf{C},$ the identity arrow $1_X:X\to X$ is a terminal object in the slice category $\mathbf{C}/X$ and an initial object in the coslice category $X/\mathbf{C}.$

%
%
\subsection{Generalised elements}
Let us consider arrows into and out of initial and terminal objects.\\
A set $A$ has an arrow into the initial object $0$ precisely if it is itself initial and the same is true for poset categories. In monoids and groups, by contrast, every object has a unique arrow to the initial object, since it is also terminal.\\
Let us consider some arrows from terminal objects. For any set $X,$ for instance, we have an isomorphism
\begin{equation*} 
	X \cong \Hom_{\mathsf{Set}}(1, X)
\end{equation*}
between elements $x \in X$ and arrows $\bar{x}:1\to X,$ determined $\bar{x}(*) = x,$ from a terminal object $1 = \{*\}.$ A similar situation holds in posets and in topological spaces, where the arrows $1\to P$ correspond to elements of the underlying set of a poset or topological space. In any category with terminal objects $1,$ such arrows $1 \to A$ are often called \emph{global elements}, or \emph{points}, or \emph{constants} of $A.$ In sets, posets, and spaces, the general arrows $A \to B$ are determined by what they do to the points of $A,$ in the sense that two arrows $f, g: A \to B$ are equal if for every point $a:1 \to A$ the composites $fa$ and $ga$ are equal.\\
However, this is not always the case! Recall that in $\mathsf{Mon},$ the terminal object $1$ is also an initial object and hence, given any monoid $M,$ there is a unique arrow $1\to M.$ (This is to say that $M$ has one point.) In the category $\mathsf{BA},$ there is no arrow $1 \to B$ if $B \neq 1.$ (This is to say that $B$ has no points.)\\
Thus, in general, an object is not determined by its points. For this reason, it is convenient to introduce the device of \emph{generalised elements}. There are arbitrary arrows
\begin{equation*} 
	x : X \to A
\end{equation*}
(with arbitrary domain $X$), which can be regarded as \emph{generalised} or \emph{variable elements} of $A.$\\
We summarise the above in the following list of examples:\\
\example{}
\begin{itemize}
	\item Consider arrows $f, g: P\to Q$ in $\mathsf{Pos}.$ Then, $f = g$ iff for all $x:1 \to P,$ we have $fx = gx.$ In this sense, posets ``have enough points'' to separate the arrows.
	\item By contrast, in $\mathsf{Mon},$ for homomorphisms $h, j: M \to N,$ we always have $hx = jx,$ for all $x:1 \to M,$ since there is just one such point $x.$ However, we do have examples of distinct homomorphisms between two monoids. Thus, monoids do not ``have enough points.''
	\item But in any category $\mathbf{C},$ and for any arrows $f, g:C\to D,$ we always have $f = g$ iff for all $x:X\to C,$ it holds that $fx = gx.$\\
	The ``only if'' part is trivial. To see the ``if'' part, consider $X = C$ and $x = 1_C.$\\
	Thus, all objects have enough generalised elements.
	\item In fact, it often happens that it is enough to consider generalised elements of just a certain form $T \to A,$ that is, for certain ``test'' objects $T.$ We shall consider this presently.
\end{itemize}
Generalised elements are also good for ``testing'' for various conditions. Consider, for instance, diagrams of the following shape:
\begin{equation*} 
	\begin{tikzcd}
		X \arrow[rr, "x", shift left = 1] \arrow[rr, "x'"', shift right = 1] && A \arrow[rr, "f"] && B 
	\end{tikzcd}
\end{equation*}
The arrow $f$ is monic iff $x \neq x'$ implies $fx \neq fx'$ for all $x, x',$ that is, just if $f$ is ``injective on generalised elements.''\\
Similarly, in any category $\mathbf{C},$ to test whether a square commutes,
\begin{equation*} 
	\begin{tikzcd}
	A \arrow[rr, "f"] \arrow[ddd, "g"'] &  & B \arrow[ddd, "\alpha"] \\&&\\&&\\D \arrow[rr, "\beta"']&&C
	\end{tikzcd}
\end{equation*}	
we shall have $\alpha f = \beta g$ just if $\alpha fx = \beta gx$ for all generalised elements $x:X \to A.$ (why?)\\\\
\end{enumerate}
\example{} Generalised elements can also be used to ``reveal more structure'' than do the constant elements. For example, consider the following posets $X$ and $A:$
\begin{align*} 
	X &= \{x\le y, y\le z\},\\
	A &= \{a \le b \le c\}.
\end{align*}
There is an order preserving bijection $f:X \to A$ defined by
\begin{align*} 
	f(x) = a, \quad f(y) = b, \quad f(z) = c.
\end{align*}
It is easy to see that $f$ is both monic and epic in the category $\mathsf{Pos};$ however, it is clearly not an iso. In fact, no map between them is an iso. However, how would one \emph{prove} this? In this case, it is not that difficult to do this via ``brute force.''\\
One way to prove that two objects are not isomorphic is to use ``invariants'': attributes that are preserved by isomorphisms. If two objects differ by an invariant, they cannot be isomorphic. For instance, the number of global elements of $X$ and $A$ is the same, namely the three elements of the set. But consider instead the ``$\mathbf{2}-$elements'' $\mathbf{2} \to X,$ from the poset $\mathbf{2} = \{0 \le 1\}$ as a ``test-object.'' Then $X$ has $5$ such elements, and $A$ has $6.$ Since these numbers are invariants (why?), the posets cannot be isomorphic. In more detail, we can define for any poset $P$ the numerical invariant
\begin{equation*} 
	|\Hom(\mathbf{2}, P)| = \text{ the number of elements of }\Hom(\mathbf{2}, P).
\end{equation*}
Then if $P \cong Q,$ it is easy to see that $|\Hom(\mathbf{2}, P)| = |\Hom(\mathbf{2}, Q)|,$ since any isomorphism
\begin{equation*} 
	\begin{tikzcd}
		P \arrow[rr, "i", shift left = 1] && Q \arrow[ll, "j", shift left = 1]
	\end{tikzcd}
\end{equation*}
also gives an isomorphism
\begin{equation*} 
	\begin{tikzcd}
		\Hom(\mathbf{2}, P) \arrow[rr, "i_*", shift left = 1] && \Hom(\mathbf{2}, Q) \arrow[ll, "j_*", shift left = 1]
	\end{tikzcd}
\end{equation*}
defined by composition:
\begin{align*} 
	i_*(f) &= if,\\
	j_*(g) &= jg,
\end{align*}
for all $f:\mathbf{2} \to P$ and $g:\mathbf{2} \to Q.$ Indeed, this is a special case of the very general fact that $\Hom(X,- )$ is always a functor, and functors always preserve isomorphisms.
\example{} As in the foregoing example, it is often the case that generalised elements $t:T\to A$ ``based at'' certain objects $T$ are especially revealing. We can think of such elements geometrically as ``figures of shape $T$ in $A$,'' just as an arrow $\mathbf{2} \to P$ in posets is a figure of shape $p \le p'$ in $P.$ For instance, as we have already seen, in the category of monoids, the arrows from the terminal monoid are entirely uninformative, but those from the free monoid on one generator $M(1)$ suffice to distinguish homomorphisms, in the sense that two homomorphisms $f, g:M \to M'$ are equal if their composites with all such arrows are equal. Since we know that $M(1) = \mathbb{N},$ the monoid of natural numbers, we can think of generalised elements $M(1) \to M$ based at $M(1)$ as ``figures of shape $\mathbb{N}$'' in $M.$ In fact, by the UMP of $M(1),$ the underlying set $|M|$ is therefore (isomorphic to) the collection $\Hom_{\mathsf{Mon}}(\mathbb{N}, M)$ of all such figures, since
\begin{equation*} 
	|M| \cong \Hom_{\mathsf{Set}}(1, |M|) \cong \Hom_{\mathsf{Mon}}(\mathbb{N}, M).
\end{equation*}
In this sense, a map from a monoid is determined by its effect on all of the figures of shape $\mathbb{N}$ in the monoid.
%
%
\subsection{Products}
\begin{defn} \label{def:prod}
	In any category $\mathbf{C},$ a \emph{product diagram} for the objects $A$ and $B$ consists of an object $P$ and arrows
	\begin{equation*} 
		\begin{tikzcd}
			A && P \arrow[ll, "p_1"'] \arrow[rr, "p_2"] && B
		\end{tikzcd}
	\end{equation*}
	satisfying the following UMP:\\
	Given any diagram of the form
	\begin{equation*} 
		\begin{tikzcd}
			A && X \arrow[ll, "x_1"'] \arrow[rr, "x_2"] && B
		\end{tikzcd}
	\end{equation*}
	there exists a unique $u:X\to P$ making the diagram
	\begin{equation*} 
		\begin{tikzcd}
			&& X \arrow[llddd, "x_1"'] \arrow[rrddd, "x_2"] \arrow[ddd, dotted, "u"] &&\\
			&&&&\\&&&&\\
			A && P \arrow[ll, "p_1"'] \arrow[rr, "p_2"] && B
		\end{tikzcd}
	\end{equation*}
	commute, that is, such that $x_1 = p_1u$ and $x_2 = p_2u.$
\end{defn}
\remark{\label{rem:UMP-prod}} As in other UMPs, there are two parts:
\begin{itemize}
	\item \emph{Existence}: There is some $u:X\to P$ such that $x_1 = p_1u$ and $x_2 = p_2u.$
	\item \emph{Uniqueness}: Given any $v:X\to P,$ if $p_1v = x_1$ and $p_2v = x_2,$ then $v = u.$ 
\end{itemize}
\begin{prop} \label{prop:prodiso}
	Products are unique up to isomorphism.
\end{prop}
\begin{proof}
	Suppose
	\begin{equation*} 
		\begin{tikzcd}
			A && P \arrow[ll, "p_1"'] \arrow[rr, "p_2"] && B
		\end{tikzcd}
	\end{equation*}
	and
	\begin{equation*} 
		\begin{tikzcd}
			A && Q \arrow[ll, "q_1"'] \arrow[rr, "q_2"] && B
		\end{tikzcd}
	\end{equation*}
	are products of $A$ and $B.$ Since $Q$ is a product, we get a (unique) morphism $i:P \to Q$ making the upper triangle of (\ref{diag:prodiso}) commute. Similarly, since $P$ is a product we get a (unique) morphism $j:Q\to P$ making the lower triangle of (\ref{diag:prodiso}) commute.
	\begin{equation} \label{diag:prodiso}
		\begin{tikzcd}
			&& P \arrow[llddd, "p_1"'] \arrow[rrddd, "p_2"] \arrow[ddd, dotted, "i"] &&\\
			&&&&\\&&&&\\
			A && Q \arrow[ll, "q_1"'] \arrow[rr, "q_2"] \arrow[ddd, dotted, "j"] && B\\
			&&&&\\&&&&\\
			&& P \arrow[lluuu, "p_1"'] \arrow[rruuu, "p_2"] &&
		\end{tikzcd}
	\end{equation}
	Note that the composite $j\circ i$ is a morphism from $P$ to $P$ which has the following property: $p_1 \circ j \circ i = p_1$ and $p_2 \circ j \circ i = p_2.$ Note that $1_P$ also has the same property, id est, $p_1 \circ 1_P = p_1$ and $p_2 \circ 1_P = p_2.$ Since $P$ is a product, there is a such unique morphism and thus, $j\circ i = 1_P.$ Interchanging the roles of $P$ and $Q$ gives us $i \circ j = 1_Q$ and hence, $i$ and $j$ are the desired isomorphisms.
\end{proof}
If $A$ and $B$ have a product, we write
\begin{equation} 
	\begin{tikzcd}
		A && A \times B \arrow[ll, "p_1"'] \arrow[rr, "p_2"] && B
	\end{tikzcd}
\end{equation}
for one such product. Then given $X, x_1, x_2$ as in the definition (\ref{def:prod}), we write
\begin{equation} \label{eq:prodpair}
	\langle x_1, x_2\rangle \text{ for } u:X\to A\times B.
\end{equation}
Note, however, that a pair of objects may have many different products in a category. For example, given a product $A \times B, p_1, p_2,$ and any iso $h:A\times B \to Q,$ the diagram $Q, p_1\circ h, p_2\circ h$ is also a product of $A$ and $B.$\\
Now an arrow \emph{into} a product
\begin{equation*} 
	f : X \to A \times B
\end{equation*}
is ``the same thing'' as a pair of arrows
\begin{equation*} 
	f_1:X\to A, \quad f_2: X\to B.
\end{equation*}
(This follows from the UMP.)\\
So, we can essentially forget about such arrows, in that they are uniquely determined by pairs of arrows. But something useful \emph{is} gained if a category has products; namely,, consider arrows \emph{out} of the product,
\begin{equation*} 
	g:A\times B \to Y.
\end{equation*}
Such a $g$ is a ``function of two variables''; given any two generalised elements $f_1:X\to A$ and $f_2:X\to B,$ we have an element $g\langle f_1, f_2\rangle:X \to Y.$ Such arrows $g:A \times B \to Y$ are not ``reducible'' to anything more basic, the way that products into arrows were.
%
%
\subsection{Examples of Products}
\begin{enumerate}
	\item $\mathsf{Set}.$ In this category, every pair of objects does have a product. It is the usual Cartesian product (along with the projections). Given sets $A$ and $B,$ the \emph{cartesian product} of $A$ and $B$ is the set of ordered pairs
	\begin{align*} 
		A \times B = \{(a, b) \mid a \in A, b \in B\}.
	\end{align*}
	The projections are the following two maps
	\begin{equation*} 
		\begin{tikzcd}
			A && A \times B \arrow[ll, "\pi_1"'] \arrow[rr, "\pi_2"] && B
		\end{tikzcd}
	\end{equation*}
	defined as
	\begin{equation*} 
		\begin{tikzcd}
			a && (a, b) \arrow[ll, maps to, "\pi_1"'] \arrow[rr, maps to, "\pi_2"] && b
		\end{tikzcd}
	\end{equation*}
	Moreover, given any pair of functions $f:X\to A$ and $g:X\to B,$ there is a unique $h:X\to A\times B$ that makes the required diagram commute. It is given by
	\begin{equation*} 
		h(x) = (f(x), g(x)).
	\end{equation*}
	Note that if we choose a different definition of ordered pairs, we get different sets. They will, of course, be isomorphic.
	\item Products of ``structured sets'' like monoids and groups can \emph{often} be constructed as products of the underlying sets with \emph{componentwise} operations: If $G$ and $H$ are groups, for instance, $G\times H$ can be constructed by taking the underlying set of $G \times H$ to be the set $\{\langle g, h \mid g \in G, h \in H\rangle\}$ and defining the binary operation by
	\begin{equation*} 
		\langle g, h\rangle\cdot\langle g', h'\rangle = \langle g\cdot g', h\cdot h'\rangle
	\end{equation*}
	the unit by
	\begin{equation*} 
		u = \langle u_G, u_H\rangle
	\end{equation*}
	and inverses by
	\begin{equation*} 
		\langle a, b\rangle^{-1} = \langle a^{-1}, b^{-1}\rangle.
	\end{equation*}
	The projection homomorphisms $G\times H \to G$ (or $H$) are the evident one $\langle g, h\rangle \mapsto g$ (or $h$).\\
	Note that this need not always work in all categories with ``structured sets.'' For example, the cartesian product of two fields with componentwise operations is not a field. (Not just ``not necessarily,'' it's \emph{never} a field.)
	%
	\item For categories $\mathbf{C}$ and $\mathbf{D},$	we already defined the category of objects and arrows,
	\begin{equation*} 
		\mathbf{C} \times \mathbf{D}.
	\end{equation*}
	Together with the evident projection functors, this is indeed a product in $\mathsf{Cat}$ (when $\mathbf{C}$ and $\mathbf{D}$ are ``small'').\\
	As special cases, we also get the products of posets and of monoids as product of categories. For them to indeed be the products in $\mathsf{Pos}$ and $\mathsf{Mon},$ one has to check that the product category does indeed take the form of a poset-category or a monoid-category. Moreover, the projections and the unique paired functions have to be checked to be monotone/monoid homomorphisms.
	%
	\item Let $P$ be a poset (considered as a category) and consider a product of elements $p, q \in P.$ We must have projections
	\begin{align*} 
		p \times q &\le p,\\
		p \times q &\le q,
	\end{align*}
	and if for any element $x,$ we have
	\begin{equation*} 
		x \le p, \quad \text{and} \quad x \le q,
	\end{equation*}
	then we need
	\begin{equation*} 
		x \le p \times q.
	\end{equation*}
	The above operation $\times$ can be recognised as the ``meet'' operation. $p \times q$ is just what is usually the \emph{greatest lower bound} of $p$ and $q.$ In more familiar notation, we have $p\times q = p \wedge q.$
	\item One can show that the product of topological spaces, as usually defined, is indeed the product in $\mathsf{Top}.$
\end{enumerate}
%
%
\subsection{Categories with products}
Let $\mathbf{C}$ be a category that has a product diagram for every pair of objects. Suppose we have objects and arrows
\begin{equation*} 
	\begin{tikzcd}
		A \arrow[ddd, "f"'] && A \times A' \arrow[ll, "p_1"'] \arrow[rr, "p_2"] && A' \arrow[ddd, "f'"]\\
		&&&&\\&&&&\\
		B && B \times B' \arrow[ll, "q_1"'] \arrow[rr, "q_2"] && B'
	\end{tikzcd}
\end{equation*}
with indicated products. Then, we write
\begin{equation*} 
	f \times f' : A \times A' \to B \times B'
\end{equation*}
for the arrow $f\times f'$ obtained by the UMP of $B \times B'$ and the arrows $f\circ p_1:A\times A' \to B$ and $f\circ p_2:A\times A' \to B.$ That is, the unique $f\times f'$ making both squares in the following diagram commute:
\begin{equation*} 
	\begin{tikzcd}
		A \arrow[ddd, "f"'] && A \times A' \arrow[ll, "p_1"'] \arrow[rr, "p_2"] \arrow[ddd, "f\times f'", dotted] && A' \arrow[ddd, "f'"]\\
		&&&&\\&&&&\\
		B && B \times B' \arrow[ll, "q_1"'] \arrow[rr, "q_2"] && B'
	\end{tikzcd}
\end{equation*}
In this way, if we choose a product for each pair of objects, we get a functor
\begin{equation*} 
	\times : \mathbf{C} \times \mathbf{C} \to \mathbf{C}.
\end{equation*}
This is defined on objects by sending $(A, A')$ to $A \times A'$ and $(f, f'):(A, A') \to (B, B')$ to $f\times f':A \times A' \to B \times B'.$\\
Let us show that this is indeed a functor.\\
The domain and codomain is clearly preserved, by construction.\\
Let us show that it preserves units. Let $(A, A') \in \mathbf{C} \times \mathbf{C}.$ We wish to show that $1_A\times 1_{A'}:A\times A' \to A\times A'$ is the identity arrow. For this, we note that the following diagram commutes (why?):
\begin{equation*} 
	\begin{tikzcd}
	A \arrow[ddd, "1_A"'] &  & A \times A' \arrow[ll, "p_1"'] \arrow[rr, "p_2"] \arrow[ddd, "1_{A \times A'}"] &  & A' \arrow[ddd, "1_A"] \\&&&&\\&&&&\\
	A &  & A\times A' \arrow[ll, "p_1"]\arrow[rr, "p_2"']&&A'                  
	\end{tikzcd}
\end{equation*}	
By the uniqueness clause of the UMP, we get that $1_A \times 1_{A'} = 1_{A\times A'}.$\\
To show that this map preserves preserves composition, one may observe diagram (\ref{diag:prodcomp}) and use the UMP of $C \times C'.$

\begin{equation} \label{diag:prodcomp}
	\begin{tikzcd}
	A \arrow[ddd, "f"'] &  & A \times A' \arrow[ll, "p_1"'] \arrow[rr, "p_2"] \arrow[ddd, "f\times f'", dashed] &  & A' \arrow[ddd, "f'"] \\
	&&&&\\&&&&\\
	B \arrow[ddd, "g"'] &  & B\times B' \arrow[ll, "q_1"] \arrow[rr, "q_2"'] \arrow[ddd, "g\times g'", dashed]  &  & B' \arrow[ddd, "g'"] \\
	&&&&\\&&&&\\
	C&& C\times C' \arrow[ll, "r_1"'] \arrow[rr, "r_2"]&& C'                  
	\end{tikzcd}
\end{equation}
One may now define a ternary product
\begin{equation*} 
	A_1 \times A_2 \times A_3
\end{equation*}
with an analogous UMP: there are three projections $p_i:A_1 \times A_2 \times A_3 \to A_i,$ and for any object $X$ and three arrows $x_i:X\to A_i,$ there is a unique arrow $u:X\to A_1\times A_2 \times A_3$ such that $p_iu = x_i$ for $i = 1, 2, 3.$ Clearly, such a condition can be formulated for any (finite?) number of factors.\\
It is clear that if a category has binary products, then it has all finite products with two or more factors; for instance, one could set
\begin{equation*} 
	A \times B \times C = (A \times B) \times C
\end{equation*}
with the appropriate projections. On the other hand, one could have taken $A \times (B \times C)$ as well. It can be seen that both of these satisfy the ternary UMP. However, by the ternary UMP, we also then have
\begin{align*} 
	(A \times B) \times C \cong A \times (B \times C),
\end{align*}
id est, the binary product operation is associative up to isomorphism.\\
Observe that a terminal object is a ``nullary'' product, a product of zero objects:\\
Given no objects, there is an object $1$ with no maps, and given any other object $X$ and no maps, there is a unique arrow
\begin{equation*} 
	!:X \to 1
\end{equation*}
making nothing further commute.\\
Similarly, any object $A$ is the \emph{unary product} of $A$ with itself one time.\\~\\
Finally, one may also define the product of a family of objects $(C_i)_{i \in I}$ indexed by \emph{any} set $I$ in the following manner:\\
The product $(C_i)_{i \in I}$ is an object $P$ along with a family $(p_i)_{i \in I}$ of arrows
\begin{equation*} 
	p_i:P \to C_i
\end{equation*}
satisfying the following UMP:\\
Given any object $X$ and family $(x_i)_{i \in I}$ of arrows
\begin{equation*} 
	x_i:X \to C_i,
\end{equation*}
there exists a unique $u:X \to P$ such that $p_iu = x_i$ for every $i \in I.$
\begin{defn} 
	A category $\mathbf{C}$ is said to \emph{have all finite products}, if it has a terminal object and all binary products (and therewith products of any finite cardinality). The category $\mathbf{C}$ \emph{has all products} if every set of objects in $\mathbf{C}$ has a product.
\end{defn}
%
%
\subsection{Hom-sets}
In this (sub)section, we assume that all categories are locally small, that is given any two objects in this category, the collection of morphisms from one to the other is in fact a set.\\
Recall that in any category $\mathbf{C},$ given any objects $A$ and $B,$ we write
\begin{equation*} 
	\Hom(A, B) = \{f \in \mathbf{C} \mid f:A\to B\}
\end{equation*}
and call such a set of arrows a \emph{$\Hom$-set}. Clearly, any element in the above set is an arrow with domain $A$ and codomain $B.$ Thus, we may compose it with an arrow $g : B \to B'$ to get a an arrow $g\circ f:A \to B',$ and arrow with domain $A$ and codomain $B'.$ This is just an elaborate of saying that any arrow $g:B\to B'$ induces a function
\begin{equation*} 
	\Hom(A, g) : \Hom(A, B) \to \Hom(A, B')
\end{equation*}
defined in the above manner, id est,
\begin{equation*} 
	(f: A \to B) \mapsto (g \circ f:A \to B').
\end{equation*}
(Note very carefully that the above sends an arrow to another arrow.)\\
Thus, we now have seen two things of the form $\Hom(A, -);$ when $-$ is an object $B,$ we get a \emph{set} $\Hom(A, B),$ an object in $\mathsf{Set},$ when $-$ is an arrow $g:B\to B',$ we get a \emph{function} $\Hom(A, g),$ an arrow in $\mathsf{Set}.$ Moreover, it's a function from $\Hom(A, \dom g)$ to $\Hom(A, \cod g).$ The attentive reader should now see what this is screaming - the creation of a functor!\\
Indeed, now we show that
\begin{equation*} 
	\Hom(A, -) : \mathbf{C} \to \mathsf{Set}
\end{equation*}
is a functor. (The definition of this map is precisely what was written above.)\\
That it preserves domain and codomain follows from construction. We now show that it preserves units and compositions.\\
Units:\\
Let $B \in \mathbf{C}$ be an object and $1_B$ its identity arrow. We show that $\Hom(A, 1_B):\Hom(A, B) \to \Hom(A, B)$ is the identity map $1_{\Hom(A, B)}.$\\
This is direct; indeed, let $f \in \Hom(A, B).$ Then,
\begin{align*} 
	\Hom(A, 1_B)(f) &= 1_B \circ f\\
	&= f.
\end{align*}

Compositions:\\
Let $g:B\to C$ and $h:C\to D$ be arrows in $\mathbf{C}.$ We show that
\begin{equation*} 
	\Hom(A, h\circ g) = \Hom(A, h)\circ\Hom(A, g).
\end{equation*}
Note that both sides are functions from $\Hom(A, B)$ to $\Hom(A, C).$ Let $f \in \Hom(A, B).$ Then,
\begin{align*} 
	\Hom(A, h\circ g)(f) &= (h\circ g)\circ f\\
	&= h \circ (g \circ f)\\
	&= h \circ (\Hom(A, g)(f))\\
	&= \Hom(A, h)(\Hom(A, g)(f))\\
	&= (\Hom(A, h)\circ\Hom(A, b))(f).
\end{align*}
This completes the proof.\\
Thus, $\Hom(A, -)$ is indeed a functor. We shall call this the (covariant) \emph{representable functor} of $A.$ We will study such representable functors later. For now, we see how one can use $\Hom$-sets to give another formulation of the definition of products.\\
For \emph{any} object $P,$ a pair of arrows of arrows $p_1:P\to A$ and $p_2:P\to B$ determine an element of the set
\begin{align*} 
	\Hom(P, A) \times \Hom(P, B).
\end{align*}
Now, given any arrow
\begin{equation*} 
	x:X \to P
\end{equation*}
composing with $p_1$ and $p_2$ gives a pair of arrows $x_1 = p_1\circ x: X \to A$ and $x_2 = p_2 \circ x:X \to B,$ as indicated in the following diagram:
\begin{equation*} 
	\begin{tikzcd}
	&& X \arrow[llddd, "x_1"'] \arrow[ddd, "x"] \arrow[rrddd, "x_2"] &&\\
	&&&&\\&&&&\\
	A && P \arrow[ll, "p_1"] \arrow[rr, "p_2"'] && B
	\end{tikzcd}
\end{equation*}
(We emphasise that in the above discussion, $P$ is \emph{any} object. Not necessarily a product.)\\
In this way, we have a function
\begin{equation*} 
	\varv_X = (\Hom(X, p_1), \Hom(X, p_2)) : \Hom(X, P) \to \Hom(X, A) \times \Hom(X, B)
\end{equation*}
defined by
\begin{equation} \label{eq:vx}
	\varv_X(x) = (x_1, x_2).
\end{equation}
Note very carefully that the above function depends on $P, p_1, p_2.$\\
Using this function, we can express the condition of being a product concisely as follows.
\begin{prop} \label{prop:canonprod}
	A diagram of the form
	\begin{equation*} 
		\begin{tikzcd}
			A && P \arrow[ll, "p_1"'] \arrow[rr, "p_2"] && B
		\end{tikzcd}
	\end{equation*}
	is a product for $A$ and $B$ iff for every object $X,$ the canonical function $\varv_X$ given in (\ref{eq:vx}) is an isomorphism.
\end{prop}
\begin{proof} 
	The proof follows from observing that the above condition is just a rephrasing of the UMP of the product. Let $(x_1, x_2) \in \Hom(X, A) \times \Hom(X, B)$ be arbitrary. We recall from Remark \ref{rem:UMP-prod}, the two conditions of product:
	\begin{itemize}
		\item \emph{Existence}: There is some $x:X\to P,$ id est, $x \in \Hom(X, P)$ such that $x_1 = p_1x$ and $x_2 = p_2x.$ This is iff $\varv_X$ is surjective.
		\item \emph{Uniqueness}: Given any $v:X\to P,$ if $p_1v = x_1$ and $p_2v = x_2,$ then $v = x.$  This is iff $\varv_X$ is injective.
	\end{itemize}
	The result follows. (Recall that an isomorphism in $\mathsf{Set}$ is the same as a bijection.)
\end{proof}
\begin{defn} 
	Let $\mathbf{C}, \mathbf{D}$ be categories with binary products. A function $F:\mathbf{C} \to \mathbf{D}$ is said to \emph{preserve binary products} if it takes every product
	\begin{equation*} 
		\begin{tikzcd}
			A && A \times B \arrow[ll, "p_1"'] \arrow[rr, "p_2"] && B
		\end{tikzcd}
		\quad \text{in }\mathbf{C}
	\end{equation*}
	to a product diagram
	\begin{equation} \label{diag:funcprod}
		\begin{tikzcd}
			FA && F(A \times B) \arrow[ll, "p_1"'] \arrow[rr, "p_2"] && FB
		\end{tikzcd}
		\quad \text{in }\mathbf{D}.
	\end{equation}
\end{defn}
Note that in the above, it is not sufficient that $F(A \times B)$ is isomorphic to $FA \times FB.$ We also require $Fp_1$ and $Fp_2$ to act like the arrows that give us the UMP. Thus, we must have an isomorphism that is ``good enough''.\\
Let us see more elaborately when the above is indeed a product diagram in $\mathbf{D}.$ As $\mathbf{D}$ has binary products, consider the product $FA \times FB$ with the projections $q_1$ and $q_2$ to $FA$ and $FB,$ respectively.\\
Consider the unique arrow $i:F(A\times B) \to FA \times FB$ obtained via the UMP of $FA \times FB$ using the arrows $Fp_1$ and $Fp_2.$ Then, (\ref{diag:funcprod}) is a product diagram iff $i$ is an isomorphism.\\
A rough sketch of the proof is as follows: For the forward direction, it will follow from the UMP as in the proof of Proposition \ref{prop:prodiso}.\\
For the reverse direction, let $X$ be an arbitrary object with arrows into $A$ and $B,$ use $i^{-1}$ and UMP of $FA \times FB$ to get the arrow $x:X\to F(A \times B)$ as desired by the definition of product.\\~\\
To summarise the discussion, I shall write what the book writes concisely as:\\
It follows that $F$ preserves products just if
\begin{equation*} 
	F(A\times B) \cong FA \times FB
\end{equation*}
``canonically,'' that is, iff the canonical ``comparison arrow''
\begin{equation*} 
	\langle Fp_1, Fp_2\rangle : F(A \times B) \to FA \times FB
\end{equation*}
in $\mathbf{D}$ is an isomorphism.\\
(Recall the definition of $\langle Fp_1, Fp_2\rangle$ from (\ref{eq:prodpair}).)\\
For example, the forgetful functor $U:\mathsf{Mon}\to\mathsf{Set}$ preserves products.
\begin{cor}
	For any object $X$ in a category $\mathbf{C}$ with products, the (covariant) representable functor
	\begin{equation*} 
		\Hom_{\mathbf{C}}(X, -) : \mathbf{C} \to \mathsf{Set}
	\end{equation*}
	preserves products.
\end{cor}
\begin{proof} 
	By our above observation, we want to show that: \\
	for any $A, B \in \mathbf{C},$ $\Hom(X, A\times B)$ and $\Hom(X, A) \times \Hom(X, B)$ are canonically isomorphic (in $\mathsf{Set}$). \\
	By Proposition \ref{prop:canonprod}, we precisely have that there is a canonical isomorphism
	\begin{equation*} 
		\Hom(X, A \times B) \cong \Hom(X, A) \times \Hom(X, B).
	\end{equation*}
	We have used that $\varv_X$ (defined in (\ref{eq:vx})) is simply $\langle \Hom(X, p_1), \Hom(X, p_2)\rangle$ (as defined in (\ref{eq:prodpair})).\\

\end{proof}
\sec{Duality}
\subsection{The Duality Principle}
Let us recall the definition of a category: There are two kinds of \emph{things}, objects $A, B, C,\ldots$ and arrows $f, g, h, \ldots;$ four operations $\dom(f), \cod(f), 1_A, g \circ f;$ and these satisfy the following seven axioms:
\begin{align} 
  \dom(1_A) = A &\qquad \cod(1_A) = A\nonumber\\
  f\circ1_{\dom(f)} = f &\qquad 1_{\cod(f)}\circ f = f \label{catax}\\
  \dom(g\circ f) = \dom(f) &\qquad \cod(g\circ f) = \cod(g)\nonumber\\
  h\circ(g\circ f) &= (h\circ g) \circ f\nonumber
\end{align}
Where the operation ``$g\circ f$'' is defined precisely when
\begin{equation*} 
  \dom(g) = \cod(f),
\end{equation*}
so a suitable form of this should occur as a condition on each equation containing $\circ,$ as in $\dom(g) = \cod(f) \implies \dom(g\circ f) = \dom(f).$\\
Now, given any sentence $\Sigma$ in the elementary language of category theory, we can form the ``dual statement'' $\Sigma^*$ by making the following replacements:
\begin{align*} 
  f\circ g \; &\text{for} \; g \circ f,\\
  \cod \; &\text{for} \; \dom,\\
  \dom \; &\text{for} \; \cod.
\end{align*}
It is easy to see that after these replacements, the statement will again be well formed. Next, suppose that we have shown a statement $\Sigma$ to entail one $\Delta,$ that is, $\Sigma \implies \Delta,$ without using any of the category axioms. Then, it follows that $\Sigma^* \implies \Delta^*.$ This is because the substituted terms are mere undefined constants if we don't use any category axioms.\\
However, now observe that the axioms (\ref{catax}) for category theory (CT) are themselves ``self-dual,'' in the sense that we have,
\begin{equation*} 
  \text{CT}^* = \text{CT}.
\end{equation*}
We now have the following \emph{duality principle}.
%
\begin{prop}[formal duality]
  For any sentence $\Sigma$ in the language of category theory, if $\Sigma$ follows from the axioms of categories, then do foes its dual $\Sigma^*$:
  \begin{equation*} 
    \text{CT} \implies \Sigma \; \text{implies} \; \text{CT} \implies \Sigma^*.
  \end{equation*}
\end{prop}
Taking a more conceptual point of view, note that if a statement $\Sigma$ involves some diagram of objects and arrows,
\begin{equation*} 
  \begin{tikzcd}
    A \arrow[rr, "f"] \arrow[rrddd, "g\circ f"'] && B\arrow[ddd, "g"]\\
    &&\\&&\\
    &&C
  \end{tikzcd}
\end{equation*}
then the dual statement $\Sigma^*$ involves the diagram obtained from it by reversing the direction and the order of composition of arrows.
\begin{equation*} 
  \begin{tikzcd}
  A &  & B \arrow[ll, "f"']                            \\
    &  &                                               \\
    &  &                                               \\
    &  & C \arrow[lluuu, "f\circ g"] \arrow[uuu, "g"']
  \end{tikzcd}
\end{equation*}
Recalling the opposite category $\mathbf{C}\op$ of a category $\mathbf{C},$ we see that an interpretation of a statement $\Sigma$ in $\mathbf{C}$ automatically gives an interpretation of $\Sigma^*$ in $\mathbf{C}\op.$\\
Now suppose that a statement $\Sigma$ holds for all categories $\mathbf{C}.$ Then, it also holds in all categories $\mathbf{C}\op,$ and so $\Sigma^*$ holds in all categories $(\mathbf{C}\op)\op.$ But since for every category $\mathbf{C},$
\begin{equation*} 
  (\mathbf{C}\op)\op= \mathbf{C},
\end{equation*}
we see that $\Sigma^*$ also holds in all categories $\mathbf{C}.$ We therefore have the following form of conceptual form of the duality principle.
\begin{prop}[conceptual duality]
  For any statement $\Sigma$ about categories, if $\Sigma$ holds for all categories, then so does the dual statement $\Sigma^*.$
\end{prop}
%
\subsection{Coproducts}
Let us consider the example of products and see what the dual notion must be. We first recall the definition of product.
\begin{defn} 
  A diagram \begin{tikzcd}A & P \arrow[l, "p_1"'] \arrow[r, "p_2"] & B\end{tikzcd} is a \emph{product} of $A$ and $B,$ if for any $Z$ and \begin{tikzcd}A & Z \arrow[l, "z_1"'] \arrow[r, "z_2"] & B\end{tikzcd} there is a unique $u:Z\to P$ with $p_i \circ u = z_i,$ all as indicated in
  \begin{equation*} 
    \begin{tikzcd}
    && Z \arrow[llddd, "z_1"'] \arrow[rrddd, "z_2"] \arrow[ddd, "u", dotted] &&\\
    &&&&\\&&&&\\
    A && P \arrow[ll, "p_1"] \arrow[rr, "p_2"'] && B
    \end{tikzcd}
  \end{equation*}
\end{defn}
Now what is the dual statement? The reader is encouraged to write the dual statement themselves and compare it with the next definition. The convention is to use the prefix ``co-'' to indicate the dual notion. Thus, we get the definition of a \emph{co-}product as follows.
\begin{defn} 
A diagram \begin{tikzcd}A \arrow[r, "q_1"] & Q & B \arrow[l, "q_2"']\end{tikzcd} is a \emph{coproduct} of $A$ and $B,$ if for any $Z$ and \begin{tikzcd}A \arrow[r, "z_1"] & Z & B \arrow[l, "z_2"']\end{tikzcd} there is a unique $u:Q\to Z$ with $u \circ q_i = z_i,$ all as indicated in

  \begin{equation*} 
    \begin{tikzcd}
    && Z&&\\
    &&  &&\\&&  &&\\
    A \arrow[rr, "q_1"'] \arrow[rruuu, "z_1"] &  & Q \arrow[uuu, "u"', dotted] &
    & B \arrow[ll, "q_2"] \arrow[lluuu, "z_2"']
    \end{tikzcd} 
  \end{equation*}
\end{defn}
We usually write \begin{tikzcd}A \arrow[r, "i_1"] & A+B & B \arrow[l, "i_2"']\end{tikzcd} for the coproduct and $[f, g]$ for the uniquely determined arrow $u:A+B \to Z.$ The ``coprojections'' $i_1:A \to A+B$ and $i_2:B\to A+B$ are usually called \emph{injections}, with no deeper meaning.

A coproduct is therefore, precisely the product of the objects in the opposite category. This immediately gets a lot of examples of coproducts. However, the opposite category of a familiar category is not really very familiar. Let us look at some more familiar categories and coproducts there.

\hrulefill

However, before we see examples, a joke:
\begin{joke}
A mathematician is a person who turns coffee into theorems.\\
A comathematician is a coperson who turns cotheorems into ffee.
\end{joke}
\hrulefill

\example{}  In $\mathsf{Set},$ the coproduct $A+B$ of two sets is their disjoint union which can be constructed, for example, as
\begin{equation*} 
  A + B = \{(a, 1) \mid a \in A\} \cup \{(b, 2) \mid b \in B\}
\end{equation*}
with evident coproduct injections as
\begin{equation*} 
  i_1(a) = (a, 1), \quad i_2(b) = (b, 2).
\end{equation*}
Given any functions $f$ and $g$ as in
\begin{equation*} 
  \begin{tikzcd}
    A \arrow[rr, "f"] && Z && B \arrow[ll, "g"']
  \end{tikzcd},
\end{equation*}
we define $[f, g]:A+B \to Z$ as
\begin{equation*} 
  [f, g](x, \delta) = \begin{cases}
    f(x) & \delta = 1\\
    g(x) & \delta = 2.
  \end{cases}
\end{equation*}
It can be verified that $[f, g]\circ i_1 = f$ and $[f, g]\circ i_2 = g.$\\
Moreover, given any $h:A+B \to Z$ with $h\circ i_1 = f$ and $h\circ i_2 = g,$ we must have $h = [f, g].$\\
Thus, every pair of objects in $\mathsf{Set}$ does have a coproduct.\\
Also, note that in $\mathsf{Set},$ every finite set is a coproduct:
\begin{equation*} 
  A \cong \underbrace{1 + 1 + \cdots + 1}_{n \text{ times}}
\end{equation*}
for $n = \operatorname{card}(A).$ This is because a function $f:A \to Z$ is uniquely determined by its values $f(a)$ for all $a \in A.$ (This also encapsulates the fact that one may define $f(a)$ in \emph{any} way for each $a \in A$ and still get a function $f:A\to Z.$ This is in contrast to something more structured like a monoid where the arrows must satisfy some further constraints.)

\example{\label{ex:freemoncop}} If $M(A)$ and $M(B)$ are \emph{free} monoids on sets $A$ and $B,$ then in $\mathbf{Mon},$ we can construct there coproduct as
\begin{equation*} 
  M(A) + M(B) \cong M(A+B),
\end{equation*}
where $A+B$ is the coproduct of sets, id est, their disjoint union as defined above.\\
The injections are the natural inclusions.\\
One can see that this is a coproduct directly by considering words over $A + B,$ but it also follows abstractly by using the diagram.
\begin{equation} \label{diag:moncoprod}
  \begin{tikzcd}
    && N&&\\&&&&\\&&&&\\&&&&\\
    M(A) \arrow[rruuuu] \arrow[rr]&  & 
    M(A+B) \arrow[uuuu, dotted]    &  & M(B) \arrow[lluuuu] \arrow[ll]
    \\&&&&\\&&&&\\
    A \arrow[uuu, "\eta_A"] \arrow[rr] &  & A+B \arrow[uuu, "\eta_{A+B}"] 
    && B \arrow[ll] \arrow[uuu, "\eta_B"']
  \end{tikzcd}
\end{equation}
in which the $\eta$s are the respective insertion of generators. (Recall this from \S\S\ref{ssec:free}.)\\
(Note that there's actually an abuse of notation in the above diagram as we have objects from both $\mathsf{Set}$ and $\mathsf{Mon}$ in it. This will carry on for the rest of this example.)\\
The UMPs of $M(A), M(B), A+B,$ and $M(A+B)$ then imply that $M(A+B)$ has the required UMP of $M(A) + M(B).$ \\~\\
%
Let us look at this in more detail:\\
The injections $i_1 : M(A) \to M(A + B)$ and $i_2 : M(B) \to M(A + B)$ are defined to be precisely those that make the squares in (\ref{diag:moncoprod}) commute. (Their existence and uniqueness are given by the UMP of free monoids.)\\
Now we show that $M(A + B)$ has the desired UMP given these injections.\\~\\
%
Let $f:M(A) \to N$ and $g:M(B) \to N$ be monoid homomorphisms. We want to show the existence of a unique monoid homomorphism $u:M(A + B) \to N$ that makes the two triangles commute.\\~\\
\textbf{Existence}: Consider the arrows $f\circ\eta_A:A\to N$ and $g\circ\eta_B:B\to N$ (in $\mathsf{Set}$). By the UMP of $A + B,$ there exists $h:A+B \to N$ making the following diagram commute.

\begin{equation} \label{diag:coprodmon2}
  \begin{tikzcd}
  && N &&\\&&&&\\&&&&\\
  M(A) \arrow[rruuu, "f"]            &  &                         &  & M(B) \arrow[lluuu, "g"']\\&&&&\\&&&&\\
  A \arrow[uuu, "\eta_A"] \arrow[rr, "i_1'"'] &  & A+B \arrow[uuuuuu, "h"] &  & B \arrow[ll, "i_2'"] \arrow[uuu, "\eta_B"']
  \end{tikzcd}
\end{equation}
Now, using the UMP of the free monoid $M(A + B),$ get a monoid homomorphism $u:M(A + B) \to N$ such that $u\circ\eta_{A+B} = h.$\\
Now, we show that this $u$ makes the triangles commute. We show this for the left triangle. We first observe that $f\circ \eta_A = u \circ i_1 \circ \eta_A.$ This was because $h = u \circ \eta_{A+B}$ and the fact that the left square commuted. And also note that
\begin{align*} 
  f\circ \eta_A = u \circ i_1 \circ \eta_A \implies f = u \circ i_1.
\end{align*}
This above follows from the UMP of $M(A).$\\
Similarly, we get that the right triangle commutes.\\~\\
%
\textbf{Uniqueness}: Let $v:M(A+B) \to N$ be another monoid homomorphism making (\ref{diag:moncoprod}) commute. Then (\ref{diag:coprodmon2}) also commuted with $v\circ \eta_{A+B}$ instead of $h.$ However, by the UMP of $M(A+B),$ this forces $h = v\circ \eta_{A+B},$ id est, $u \circ \eta_{A+B} = v \circ \eta_{A+B}.$ This clearly forces $u = v,$ as desired.
  
Note: twice in the above have we used that $f\circ \eta = g\circ\eta \implies f = g.$ This had not been proven earlier but is an easy consequence of the UMP. This is left to the reader.\\~\\
%
The foregoing examples says precisely that the free monoid functor $M:\mathsf{Set} \to \mathsf{Mon}$ preserves coproducts. However, note that the underlying set of $M(A + B)$ is \textbf{not} the coproduct of the underlying sets of $M(A)$ and $M(B).$ 

\example{} In $\mathsf{Top},$ the coproduct of two topological spaces $X$ and $Y$ is the space $X + Y$ defined as follows:\\
As a set, $X + Y$ is simply the disjoint union of $X$ and $Y,$ id est, the coproduct in $\mathsf{Set}.$\\
A set $U \subset X + Y$ is open iff $U \cap X$ is open and $U \cap Y$ is open. (Considering our previous construction of coproduct in $\mathsf{Set},$ we should write $U \cap (X \times \{1\})$ with the understanding that $X \times \{1\}$ has the topology $O(X) \times \{1\}.$)\\
The injections are the same as in $\mathsf{Set}.$ It is an easy verification that these injections are indeed arrows in $\mathsf{Top},$ id est, these are continuous.\\
Moreover, given any $z_1, z_2, Z$ as in the definition, it can be verified that the arrow $u:X+Y \to Z$ obtained in $\mathsf{Set}$ is indeed an arrow in $\mathsf{Top},$ id est, it is continuous.

\example{\label{ex:rootedpos}} Coproducts of posets are similarly constructed from the coproducts of the underlying sets, by ``putting them side by side.'' That is, given posets $P$ and $Q,$ the poset $P+Q$ is simply a poset on the disjoint union $P + Q$ with the relation as inherited from earlier without any additional ones.\\
What about ``rooted posets'', id est, posets with a distinguished initial element $0?$ In the category $\mathsf{Pos}_0$ of such posets and monotone maps that preserve $0,$ one constructs the coproduct of two such posets $P$ and $Q$ from the coproduct $P + Q$ in the category $\mathsf{Pos},$ by ``identifying'' the two different $0$s,

\begin{equation*} 
  A +_{\mathsf{Pos}_0} B = (A +_{\mathsf{Pos}} B)/\text{``}0_A = 0_B\text{''}.
\end{equation*}
(Recall \nameref{equivrel}.)

Recall the example of product in a poset (viewed as a category). There we had gotten the product to be the greatest lowest bound of two elements. Dually, one can consider the question of coproduct in a poset. The answer is not surprising.

\example{} Let $P$ be a fixed poset and $p, q \in P.$ Suppose $p + q$ exists. Then we have

\begin{equation*} 
  p \le p + q \quad \text{and} \quad q \le p + 1
\end{equation*}
and if 

\begin{equation*} 
  p \le z \quad \text{and} \quad p \le z
\end{equation*}
then

\begin{equation*} 
  p + q \le z.
\end{equation*}
So, $p + q = p \vee q$ is the join, or \emph{least upper bound} of $p$ and $q.$\\
(Of course, it is not necessary that joins exist.)

\example{} Two monoids $A, B$ have a coproduct of the form
\begin{equation*} 
  A + B = M(|A| + |B|)/\sim
\end{equation*}
where, as before, the free monoid $M(|A| + |B|)$ is strings (words) over the disjoint union $|A| + |B|$ of the underlying sets - that is, the elements of $A$ and $B$ - and the equivalence relation $v \sim w$ is the smallest one containing all instances of the following equations:
\begin{align} 
  (\ldots x u_A y \ldots) &= (\ldots xy\ldots)\nonumber\\
  (\ldots x u_B y \ldots) &= (\ldots xy\ldots) \label{eq:coprod}\\
  (\ldots aa' \ldots) &= (\ldots a\cdot_Aa'\ldots)\nonumber\\
  (\ldots bb' \ldots) &= (\ldots b\cdot_Ab'\ldots)\nonumber
\end{align}

The idea is informally to have only those words with letters from $A$ and $B$ such that the letters alternate and that $u_A$ and $u_B$ never appear.

The unit is the equivalence class $[\epsilon]$ of the empty word (which is the same as $[u_A]$ and $[u_B]$). Multiplication of equivalence classes is as expected:
\begin{equation*} 
  [x\ldots y]\cdot[x'\ldots y'] = [x\ldots yx' \ldots y'].
\end{equation*}
It would have to be verified that the above is indeed well defined. That is to say:
\begin{equation*} 
  [x\ldots y] = [a\ldots b] \text{ and } [x'\ldots y'] = [a'\ldots b'] \text{ implies } [x\ldots yx' \ldots y'] = [a\ldots ba' \ldots b'].
\end{equation*}
We leave this to the reader.

The coproduct injections $i_A:A\to A+B$ and $i_B:B\to A+B$ are simply
\begin{equation*} 
  i_A(a) = [a], \quad i_B(b) = [b],
\end{equation*}
which are easily seen to be homomorphisms. Given any homomorphism $f:A\to M$ and $g:B\to M$ into a monoid $M,$ the unique homomorphism
\begin{equation*} 
  [f, g] : A + B \to M
\end{equation*}
is defined as follows:
\begin{enumerate}
  \item We have the functions $|f| : |A| \to |M|,$ $|g| : |B| \to |M|.$ We use these to obtain the function $[|f|, |g|] : |A| + |B| \to |M|.$
  \item The function $[|f|, |g|] : |A| + |B| \to |M|$ is extended to one $[f, g]'$ on the free monoid $M(A + B)$ using the UMP.
  
  \begin{equation*} 
    \begin{tikzcd}
     M({\vert}A{\vert} + {\vert}B{\vert}) \arrow[rr, "{[f, g]'}"] \arrow[ddd, two heads] &  & M \\
      &&\\&&\\
      M({\vert}A{\vert} + {\vert}B{\vert})/{\sim} &&
    \end{tikzcd}
  \end{equation*}
  \item We then observe that $[f, g]'$ ``respects the equivalence relation $\sim$,'' the sense that if $v \sim w$ in $M(|A| + |B|),$ then $[f, g]'(v) = [f, g]'(w).$ This is a consequence of the fact that the equations (\ref{eq:coprod}) are respected by homomorphisms.\\
  Thus, the map $[f, g]'$ extends to the quotient to yield the desired map $[f, g]:M(|A| + |B|)/\sim \to M.$

  \begin{equation*} 
    \begin{tikzcd}
     M({\vert}A{\vert} + {\vert}B{\vert}) \arrow[rr, "{[f, g]'}"] \arrow[ddd, two heads] &  & M \\
      &&\\&&\\
      M({\vert}A{\vert} + {\vert}B{\vert})/{\sim} \arrow[rruuu, "{[f, g]}"', dotted] &&
    \end{tikzcd}
  \end{equation*}
  That this is the unique homomorphism is to be verified.
\end{enumerate}

This construction also works to give coproducts in $\mathsf{Grp},$ where it is usually called the \emph{free product} of $A$ and $B$ and written as $A \oplus B,$ we well as other categories of ``algebras,'' id est, sets equipped with operations. \\
As we had seen before in Example \ref{ex:freemoncop}, the underlying set of $A+B$ is \emph{not} the coproduct of $|A|$ and $|B|.$ This is to say that the forgetful functor $U:\mathsf{Mon} \to \mathsf{Set}$ does not preserve coproducts.

\example{} For \emph{abelian groups} $A, B,$ the free product $A \oplus B$ need not be abelian. One could take a further quotient to get an abelian group which would give us the coproduct in the category $\mathsf{Ab}$ of abelian groups (and group homomorphisms). However, there is a more convenient presentation which we now consider.

As we will work in the category $\mathsf{Ab}$ of abelian groups, we shall use additive notation to represent the group operation.\\
In the category $\mathsf{Ab},$ the coproduct $A \oplus B$ must be forced to satisfy the commutativity conditions
\begin{equation*} 
  (a_1b_1a_2b_2\ldots) \sim (a_1a_2\ldots b_1b_2\ldots)
\end{equation*}
we can shuffle all the $a$s to the front and all the $b$s to the back, of the words. However, we already have
\begin{equation*} 
  (a_1a_2\ldots b_1b_2\ldots) \sim (a_1 + a_2 + \cdots + b_1 + b_2 + \cdots) \sim (a + b),
\end{equation*}
where $a = (a_1 + a_2 + \cdots) \in A$ and $b = (b_1 + b_2 + \cdots) \in B.$\\
Thus, in effect, we have pairs $(a, b).$ \\
This motivates the fact that we can take the typical product of groups as the coproduct. That is, as a set, it is $|A| \times |B|.$ The operation is component wise.\\
As injections, we use the homomorphisms
\begin{equation*} 
  i_A(a) = (a, 0_B), \quad i_B(b) = (0_A, b).
\end{equation*}
Then, given any homomorphisms $A \overset{f}{\longrightarrow}X \overset{g}{\longleftarrow} B,$ we let $[f, g]:A + B \to X$ be defined as
\begin{equation*} 
  [f, g](a, b) = f(a) +_X g(b).
\end{equation*}
The fact that $[f, g]$ is indeed a homomorphism follows from the commutativity of $+_X.$ Indeed, one can see
\begin{align*} 
  [f, g]((a, b) + (a', b')) &= [f, g](a + a', b + b')\\
  &= f(a + a') +_X + g(b + b')\\
  &= f(a) +_X f(a') +_X g(b) +_X g(b')\\
  &= f(a) +_X g(b) +_X f(a') +_X g(b')\\
  &= [f, g](a, b) +_X [f, g](a', b').
\end{align*}
Now, we verify that $f = [f, g]i_A.$ Let $a \in A,$ then
\begin{align*} 
  ([f, g]\circ i_A)(a) &= [f, g](i_A(a))\\
  &= [f, g](a, 0_B)\\
  &= f(a) +_X g(0_B)\\
  &= f(a) +_X 0_X\\
  &= f(a).
\end{align*}
Similarly, $g = [f, g]i_B.$\\
Now, let $h:A + B \to X$ be such that $hi_A = f$ and $hi_B = g.$\\
First we show that $h$ and $[f, g]$ agree on $A \times \{0_B\}.$
\begin{align*} 
  h(a, 0_B) &= h(i_A(a))\\
  &= f(a)\\
  &= [f, g](i_A(a))\\
  &= [f, g](a, 0_B).
\end{align*}
Similarly, we see that $[f, g](0_A, b) = h(0_A, b).$
Now, let $(a, b) \in A + B.$ Then,
\begin{align*} 
  h(a, b) &= h(a, 0_B) +_X h(0_A, b)\\
  &= [f, g](a, 0_B) +_X [f, g](0_A, b)\\
  &= [f, g](a, b).
\end{align*}
Thus, the uniqueness is also proved.

\begin{prop}
  In the category $\mathsf{Ab}$ of abelian groups and group homomorphisms, there is a canonical isomorphism between the binary coproduct and product,
  \begin{align*} 
    A + B \cong A \times B.
  \end{align*}
\end{prop}
\begin{proof}
  To define an arrow $\varv:A+B \to A\times B,$ we need one $A \to A \times B$ (and one $B \to A \times B$), so we need arrows $A\to A$ and $A \to B$ (and $B \to A$ and $B \to B$). For these, we take $1_A : A \to A$ and the zero homomorphism $0_B : A \to B$ (and $0_A : B \to A$ and $1_B : B \to B$). Thus, all together, we get
  \begin{equation*} 
    \varv = [\langle 1_A, 0_B\rangle, \langle 0_A, 1_B\rangle]:A+B \to A \times B.
  \end{equation*}
  Then, given any $(a, b) \in A + B,$ we have
  \begin{align*} 
    \varv(a, b) &= \langle 1_A, 0_B\rangle(a) + \langle 0_A, 1_B\rangle(b)\\
    &= (1_A(a), 0_B(a)) + (0_A(b), 1_B(b))\\
    &= (a, 0) + (0, b)\\
    &= (a, b).
  \end{align*}
\end{proof}

Just as with products, one can consider the empty coproduct, which is an initial object $0$, as well as coproduct of several factors, and the coproduct of two arrows,
\begin{equation*} 
  f + f' : A + A' \to B + B'
\end{equation*}
which leads to a coproduct functor $+ : \mathbf{C} \times \mathbf{C} \to \mathbf{C}$ on categories $\mathbf{C}$ with binary coproducts. All of these facts follow simply by duality; that is, by considering the dual notions in the opposite category. Similarly, we have the following proposition.
\begin{prop}
  Coproducts are unique up to isomorphism.
\end{prop}
\begin{proof} 
  Use duality, Proposition \ref{prop:prodiso} and the fact that the dual of ``isomorphism'' is ``isomorphism.''
\end{proof}
In just the same way, it follows that $(A + B) + C \cong A + (B + C).$

Thus, in general, it will suffice to introduce new notions once and then simply observe that the dual notions have analogous (but dual) properties.

\subsection{Equalizers}
\begin{defn} 
  In any category $\mathbf{C},$ given parallel arrows
  \begin{equation*} 
    \begin{tikzcd}
      A \arrow[rr, "g"', shift right] \arrow[rr, "f", shift left]&&B
    \end{tikzcd}
  \end{equation*}
  an \emph{equalizer} of $f$ and $g$ consists of an object $E$ and arrow $e:E\to A,$ \textbf{universal} such that
  \begin{equation*} 
    f\circ e = g \circ e.
  \end{equation*}
  That is, given any $z : Z \to A$ with $f \circ z = g \circ z,$ there is a \emph{unique} $u:Z \to E$ with $e \circ u = x,$ all as in the diagram

  \begin{equation*} 
    \begin{tikzcd}
    E \arrow[rr, "e"] &  & A \arrow[rr, "g"', shift right] \arrow[rr, "f", shift left]&&B\\
    &&&&\\&&&&\\
    Z \arrow[rruuu, "z"'] \arrow[uuu, "u", dotted] &&&&  
    \end{tikzcd}
  \end{equation*}
\end{defn}

Let us consider some examples.
\example{} Consider the category $\mathbf{C} = \mathsf{Set}.$ Let $f, g:A \tto B$ be a pair of parallel arrows. Consider the equationally defined subset $E = \{x \in A \mid f(x) = g(x)\} \subset A$ along with the inclusion map $i:E \hookrightarrow A$ define as $x\mapsto x.$ Then $E$ and $i$ comprise the equalizer. This can be concisely written as
\begin{equation*} 
  \{x \in A \mid f(x) = g(x)\} \hookrightarrow A.
\end{equation*}
Let us show that this has the properties required. First, we show that $f\circ i = g\circ i.$ Let $x \in E,$ then
\begin{align*} 
  (f \circ i)(x) &= f(i(x))\\
  &= f(x)\\
  &= g(x)\\
  &= g(i(x))\\
  &= (g\circ i)(x).
\end{align*}
Now, let $z:Z \to A$ be any function with $f \circ z = g \circ z$ as depicted by

\begin{equation*} 
  \begin{tikzcd}
    E \arrow[rr, "i", hook] &  & A \arrow[rr, "g"', shift right] \arrow[rr, "f", shift left]&&B\\
    &&&&\\&&&&\\
    Z \arrow[rruuu, "z"'] &&&&  
    \end{tikzcd}
\end{equation*}
From the above, we get that $f(z(x)) = g(z(x))$ for all $x \in Z.$ This is the same as saying that $z(x) \in E$ for all $x \in Z.$\\
Which, in turn, is the same as saying that the function $z$ ``restricts''\footnote{``Restriction'' usually refers to restricting the map to a subset of the domain. This is not what we mean here.} to a function $\bar{z}:Z \to E$ defined as $x \mapsto z(x).$ (As the image lies within $E.$)\\
This is the desired $u = \bar{z}$ that makes the diagram commute.

\begin{equation*} 
  \begin{tikzcd}
    E \arrow[rr, "i", hook] &  & A \arrow[rr, "g"', shift right] \arrow[rr, "f", shift left]&&B\\
    &&&&\\&&&&\\
    Z \arrow[rruuu, "z"']  \arrow[uuu, "u", dotted] &&&&  
    \end{tikzcd}
\end{equation*}

The uniqueness of $u$ follows from $i$ being monic.

\example{} Let us take a more explicit version of the above example. Suppose we have the functions $f, g: \mathbb{R}^2 \tto \mathbb{R}$ where
\begin{align*} 
  f(x, y) &= x^2 + y^2,\\
  g(x, y) &= 1
\end{align*}
and we take the equalizer in say, $\mathsf{Top}.$ As before, this is the subspace (along with the inclusion map)
\begin{equation*} 
  S^1 = \{(x, y) \in \mathbb{R}^2 \mid x^2 + y^2 = 1\} \hookrightarrow \mathbb{R}^2,
\end{equation*}
that is, the unit circle in the plane. (Note that the inclusion is indeed continuous.)\\
Once again, given any generalised element $z:Z\to \mathbb{R}^2$ with $fz = gz,$ we have that $z$ actually maps into $S^1$ and we note that the ``restriction'' $\bar{z}:Z\to S^1$ is in fact continuous. The uniqueness again follows from the inclusion being monic.

Before moving ahead, we note that every subset $U \subset A$ is of this ``equational'' form, that is, every subset is an equalizer for some pair of functions. (We are back in $\mathsf{Set}.$)\\
Indeed, one can do this in a very canonical way. First, let us put 
\begin{equation*} 
  2 = \{\bot, \top\}.
\end{equation*}
The above can be thought of as a set of ``truth values.'' Then, consider the \emph{characteristic function}
\begin{equation*} 
  \chi_U : A \to 2,
\end{equation*}
defined for $x \in A$ by
\begin{align*} 
  \chi_U(x) = \begin{cases}
    \top & x \in U\\
    \bot & x \notin U.
  \end{cases}
\end{align*}
Thus, we have
\begin{equation*} 
  U = \{x \in A \mid \chi_U(x) = \top\}.
\end{equation*}
So the following is an equalizer:
\begin{equation*} 
  \begin{tikzcd}
    U \arrow[rr] && A \arrow[rr, "\top!", shift left] \arrow[rr, "\chi_U"', shift right] && 2
  \end{tikzcd}
\end{equation*}
where $\top! = \top\circ !:A \overset{!}{\longrightarrow}1 \overset{\top}{\longrightarrow}2,$ id est, $\top!(x) = \top$ for all $x \in A.$

Moreover, for every function,
\begin{equation*} 
  \varphi:A \to 2,
\end{equation*}
we can form the ``variety'' (id est, equational subset)
\begin{equation*} 
  V_\varphi = \{x \in A \mid \varphi(x) = \top\}
\end{equation*}
as an equalizer, in the same way.

Now, it is easy to see that these operations are mutually inverses, id est, $\chi_{V_{\varphi}} = \varphi$ and $V_{\chi_{U}} = U.$ To see, we first note that
\begin{align*} 
  \chi_{V_{\varphi}}(x) &= \begin{cases}
    \top & x \in V_{\varphi}\\
    \bot & x \notin V_{\varphi}
  \end{cases}\\
  &= \begin{cases}
    \top & \phi(x) = \top\\
    \bot & \phi(x) \neq \top
  \end{cases}\\
  &= \begin{cases}
    \top & \phi(x) = \top\\
    \bot & \phi(x) = \bot
  \end{cases}\\
  &= \varphi(x).
\end{align*}
Then, we note that
\begin{align*} 
  V_{\chi_{U}} &= \{x \in A \mid \chi_U(x) = \top\}\\
  &= \{x \in A \mid x \in U\}\\
  &= U.
\end{align*}
Thus, we have the familiar isomorphism
\begin{equation*} 
  \Hom(A, 2) \cong P(A).
\end{equation*}
\begin{prop} \label{prop:eqmonic}
  In any category, if $e:E\to A$ is an equalizer of some pair of arrows, then $e$ is monic.
\end{prop}
\begin{proof} 
  Let $e$ be the equalizer of $f, g:A\tto B.$\\
  Let $x, y:Z \tto E$ be a pair of arrows such that $ex = ey.$ We wish to show that $x = y.$
  \begin{equation*} 
    \begin{tikzcd}
      Z \arrow[rr, "x", shift left]\arrow[rr, "y"', shift right] &&
      E \arrow[rr, "e"] &&
      A \arrow[rr, "f", shift left]\arrow[rr, "g"', shift right] && B
    \end{tikzcd}
  \end{equation*}
  Consider $z = ex = ey.$ Then, we have
  \begin{align*} 
    fz &= fex\\
    &= gex & (\because fe = ge)\\
    &= gz.
  \end{align*}
  Thus, we have a diagram of the form
  \begin{equation*} 
    \begin{tikzcd}
    E \arrow[rr, "e"] &  & A \arrow[rr, "g"', shift right] \arrow[rr, "f", shift left]&&B\\
    &&&&\\&&&&\\
    Z \arrow[rruuu, "z"'] \arrow[uuu, "u", dotted] &&&&  
    \end{tikzcd}
  \end{equation*}
  Now, using the universal property of the equalizer, we get that there is a unique $u:Z \to E$ making the following diagram commute
  \begin{equation*} 
    \begin{tikzcd}
    E \arrow[rr, "e"] &  & A \arrow[rr, "g"', shift right] \arrow[rr, "f", shift left]&&B\\
    &&&&\\&&&&\\
    Z \arrow[rruuu, "z"'] \arrow[uuu, "u", dotted] &&&&  
    \end{tikzcd}
  \end{equation*}
  However, one can observe that both $u = x$ and $u = y$ make the diagram commute. Uniqueness forces $x = y.$
\end{proof}

\example{} In many other categories, such as posets and monoids, the equalizer of a parallel pair of arrows $f, g:A\tto B$ can be constructed by taking the equalizer of the underlying functions as above, id est, the subset $A(f = g) \subset A$ of elements $x \in A$ where $f$ and $g$ agree, $f(x) = g(x),$ and then restricting the structure of $A$ to $A(f = g).$ \\~\\
%
%
For instance, in posets one takes the ordering from $A$ restricted to this subset $A(f = g),$ and in topological spaces one takes the subspace topology.\\~\\
%
In monoids, the subset $A(f = g)$ is then again a monoid with the operations from $A,$ id est, it contains the unit and is closed under multiplication. To see this, we first note that $f(u_A) = u_B = g(u_A)$ and thus $u_A \in A(f = g).$ Secondly, if $f(a) = g(a)$ and $f(a') = g(a'),$ then $f(aa') = f(a)f(a') = g(a)g(a') = g(aa')$ and thus, $A(f = g)$ is closed under multiplication.\\
This shows that $A(f = g)$ is a submonoid of $A$ and hence, the inclusion is a homomorphism.\\~\\
%
In abelian groups, for instance, one has an alternate description of the equalizer using the fact that,
\begin{equation*} 
  f(x) = g(x) \quad \text{iff} \quad (f - g)(x) = 0.
\end{equation*}
Now, since we're in $\mathsf{Ab},$ $f - g$ is again a homomorphism. Thus, the equalizer of $f$ and $g$ is the same as that of the homomorphism $f - g$ and the zero homomorphism $0:A\to B,$ so it suffices to consider equalizers of the special form $A(h = 0) \mono A$ for arbitrary homomorphisms $h:A\to B.$ \\
This subgroup of $A$ is called the \emph{kernel} of $h,$ written $\ker(h).$ Thus, we have the equalizer
\begin{equation*} 
  \begin{tikzcd}
    \ker(f - g) \arrow[rr, hook] && A \arrow[rr, "f", shift left] \arrow[rr, "g"', shift right] && B
  \end{tikzcd}.
\end{equation*}

\subsection{Coequalizers} \label{ssec:coeq}
We consider the notion dual to that of equalizer.
\begin{defn} 
  For any parallel arrows $f, g:A \tto B,$ in a category $\mathbf{C},$ a \emph{coequalizer} consists of $Q$ and $q:B \to Q,$ universal with the property $qf = qg,$ as in
  \begin{equation*} 
    \begin{tikzcd}
    A \arrow[rr, "g"', shift right] \arrow[rr, "f", shift left] && 
    B \arrow[rr, "q"] \arrow[rrddd, "z"'] &  & Q \arrow[ddd, "u", dotted] \\
    &&&&\\&&&&\\
    &&&& Z
    \end{tikzcd}
  \end{equation*}
  That is, given any $Z$ and $z:B\to Z,$ if $zf = zg,$ then there exists a unique $u:Q \to Z$ such that $uq = z.$
\end{defn}
Using duality, we directly get the following proposition.
\begin{prop}
  In any category, if $q:B\to Q$ is a coequalizer of some pair of arrows, then $q$ is epic.
\end{prop}
\begin{proof} 
  Duality and Proposition \ref{prop:eqmonic}.
\end{proof}
Before the next examples, recall \S\S\ref{equivrel}.
\example{} Consider an equivalence relation $R \subset B \times B$ and the diagram
\begin{equation*} 
  \begin{tikzcd}
    R \arrow[rr, "r_1", shift left] \arrow[rr, "r_2"', shift right] && B
  \end{tikzcd}
\end{equation*}
where the two $r$s are the two compositions of the projections with the inclusion

\begin{equation*} 
  \begin{tikzcd}
  && R \arrow[ddd, hook] \arrow[llddd, "r_1"'] \arrow[rrddd, "r_2"] &&\\
  &&&&\\&&&&\\
  B && B \times B \arrow[ll, "p_1"] \arrow[rr, "p_2"'] && B
  \end{tikzcd}
\end{equation*} 

The quotient projection 
\begin{equation*} 
  \pi:B \to B/R
\end{equation*}
defined as $x \mapsto [x]$ is then a coequalizer of $r_1$ and $r_2.$\\
To see this, consider any $f:X \to Y$ as in

\begin{equation*} 
  \begin{tikzcd}
  R \arrow[rr, "r_1", shift left] \arrow[rr, "r_2"', shift right]&& 
  B \arrow[rr, "\pi"] \arrow[rrddd, "f"'] &  & X/R \arrow[ddd, "\bar{f}", dotted] \\&&&&\\&&&&\\
  &&&& Y  
  \end{tikzcd}
\end{equation*}

Then, there exists a unique $\bar{f}$ as indicated in the above diagram, id est, 
\begin{equation*}   
  \bar{f}\pi(b) = f(b) \quad \text{for all } b \in B.
\end{equation*}
As we had noted in Proposition \ref{prop:fextend}, this happens precisely when $f$ respects $R.$ As $fr_1 = fr_2,$ this is indeed true. To see this, let $(b, b') \in R.$ Then,
\begin{align*} 
  f(b) &= fr_1(b, b')\\
  &= fr_2(b, b')\\
  &= f(b').
\end{align*}

\exapmle{} The coequalizer of any arbitrary pair of arrows $f, g:A\tto B$ in $\mathsf{Set}$ can be constructed similarly as follows:
\begin{enumerate}
  \item Define a relation $R$ on $B$ as follows:
  \begin{equation*} 
    R = \{(f(a), g(a)) \mid a \in A\}.
  \end{equation*}
  \item Define $\sim$ to be the equivalence relation generated by $R.$\\
  Thus, $\sim$ is the equivalence relation generated by the equations $f(a) = g(a)$ for all $a \in A.$
  \item Then, $\pi:X \to X/R$ is a coequalizer of $f$ and $g.$ To see this, consider any $z:B \to Z$ as in
  \begin{equation*} 
    \begin{tikzcd}
    A \arrow[rr, "f", shift left] \arrow[rr, "g"', shift right] &  & B \arrow[rr, "\pi"] \arrow[rrddd, "z"'] &  & B/{\sim} \arrow[ddd, "\bar{z}", dotted] \\
    &&&&\\&&&&\\
    &&&& Z       
    \end{tikzcd}
  \end{equation*}

  As before, one sees that $z$ respects $R.$ Then, by Proposition \ref{prop:fextends2}, we get the existence of the unique $\bar{z},$ as desired.
\end{enumerate}

\example{} In Example \ref{ex:rootedpos}, we considered the coproduct of ``rooted posets'' $P$ and $Q$ by first making $P + Q$ in $\mathsf{Pos}$ and then ``identifying'' the resulting two different $0$-elements $0_P$ and $0_Q$ (id est, the images of these under the coproduct inclusions). We can now describe this ``identification'' as a coequalizer taken in  $\mathsf{Pos},$
\begin{equation*} 
  \begin{tikzcd}
    1 \arrow[rr, "0_P", shift left] \arrow[rr, "0_Q"', shift right] && P + Q \arrow[rr] && P + Q/(0_P = 0_Q)
  \end{tikzcd}
\end{equation*}

\sec{Acknowledgments\label{sec:ack}}
Here's a list of the people who have helped me make the notes better. I'm thankful to them. The count after the name denotes the number of changes made due to their suggestions - these include both typos and pointing out places where the phrasing could be improved.\\
The names are listed in chronological order based on the first suggestion.\\

\begin{itemize}[nosep]
	\item Ishan Kapnadak: 1
	\item Amit Rajaraman: 11
	\item Atharva Pangarkar: 1
	\item[$\star$] Divyanka Chaudhari: 1
\end{itemize}
\end{document}