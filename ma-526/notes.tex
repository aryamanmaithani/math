\documentclass[12pt]{article}	% Always compile at least twice.
\usepackage[lmargin=1in,rmargin=1in,tmargin=1in,bmargin=1in]{geometry}
\usepackage[dvipsnames]{xcolor}
% \usepackage{pdfpages}

% Cover Information	
\def\univname{}
\def\coursenum{MA 526}
\def\coursename{Commutative Algebra}
\def\professor{}
\def\student{Aryaman Maithani}
\def\semesteryear{Spring 2020-21}
\def\imagename{iitb.png}		    % Replace with University Seal
\def\scalesize{2}					% Scale Logo Size 

% Style Package (Load After Cover Information)
\usepackage{lecture-notes}	% Change to match style file
\DeclareMathOperator{\Ass}{Ass}
\DeclareMathOperator{\mSpec}{mSpec}
\DeclareMathOperator{\Supp}{Supp}
\DeclareMathOperator{\ann}{ann}
\DeclareMathOperator{\nil}{nil}
\DeclareMathOperator{\htt}{height}

% -------------------
% Content
% -------------------
\begin{document}

\tikzset{
  symbol/.style={
    draw=none,
    every to/.append style={
      edge node={node [sloped, allow upside down, auto=false]{$#1$}}}
  }
}


% Cover Page
\coverpage

% Last Updated
\updated{\today}

% Table of Contents
\thispagestyle{empty}
\tableofcontents
\newpage
\pagestyle{fancy}
\setcounter{page}{1}

\textbf{Notation}
\begin{enumerate}
	\item $\mathcal{N}(R)$ denotes the nilradical of $R.$
	\item $\mathcal{J}(R)$ denotes the Jacobson radical of $R.$
	\item $\Spec(R)$ denotes the set of prime ideals of $R.$
	\item $\mSpec(R)$ denotes the set of maximal ideals of $R.$
	\item $N \le M$ is read as ``$N$ is a submodule of $M$.''
	\item $I \unlhd R$ is read as ``$I$ is an ideal of $R$.''
	\item For an integral domain $R,$ $Q(R)$ denotes its field of fractions.
	\item $\mathsf{k}$ denotes a field. If $\mathsf{k}$ is algebraically closed, we write this as $\mathsf{k} = \overline{\mathsf{k}}.$ 
\end{enumerate}

\section{Associated primes of ideals and modules} % Lecture 1
\begin{defn}%[]
	Suppose $M, N$ are $R$-submodules of some $R$-module $M'.$ Then,
	\begin{equation*} 
		M:_RN \vcentcolon= \{r \in R \mid rN \subset M\}.
	\end{equation*}
\end{defn}
\begin{defn}%[]
	Let $M$ be an $R$-module and $0 \neq x \in M.$ If $\mathfrak{p} = 0:_Rx$ is a prime in $R,$ then we say that $\mathfrak{p}$ is an \deff{associated prime} of $M.$
	\begin{equation*} 
		\Ass_R(M) \vcentcolon= \{\mathfrak{p} \in \Spec(R) \mid \mathfrak{p} = 0 :_R x \text{ for some } x \in M \setminus \{0\}\}.
	\end{equation*}
\end{defn}
\begin{defn}%[]
	The elements of $\Ass(M)$ which are not minimal in $\Ass(M)$ are called \deff{embedded primes}.
\end{defn}
\begin{defn}%[]
	Fix $x \in X.$ The map $\mu_x : R \to M$ defined by $r \mapsto rx$ is called the \deff{homothety} by $x.$
\end{defn}
Note that $\ker \mu_x = 0:_Rx.$

\begin{prop}
	A prime $\mathfrak{p}$ is an associated prime of $M$ iff $R/\mathfrak{p}$ is isomorphic to a submodule of $M.$
\end{prop}

\begin{defn}%[]
	$a \in R$ is a \deff{zero divisor} on $M$ if $ax = 0$ for some $0 \neq x \in M.$
	\begin{equation*} 
		\mathcal{Z}(M) \vcentcolon= \{a \in R \mid a \text{ is a zero divisor on }M\}.
	\end{equation*} 
	If $a$ is not a zero divisor, then $\mu_a$ is called a \deff{non zero divisor} on $M$ or \deff{$M$-regular}.
\end{defn}

Note that $a$ is a zero divisor iff $\mu_a$ is not injective.

\begin{prop}
	Let $R$ be Noetherian and $M \neq 0$ finitely generated $R$-module. Then,
	\begin{enumerate}
		\item the maximal elements among $\{(0 : x) \mid x \neq 0\}$ are prime. In particular, $\Ass M \neq \emptyset.$
		\item $\mathcal{Z}(M) = \bigcup_{\mathfrak{p} \in \Ass(M)}\mathfrak{p}.$
	\end{enumerate}
\end{prop}

\begin{ex}
	Let $R = \mathsf{k}[x, y]$ for a field $\mathsf{k}$ and put $I = \langle x^2, xy\rangle.$ Then, $\Ass(R/I) = \{\langle x\rangle, \langle x, y\rangle\}.$ Note that $\langle x\rangle$ is not maximal among the annihilators.
\end{ex}

\begin{prop} \label{prop:assoflocalisation}
	Let $S \subset R$ be a multiplicatively closed set. Then,
	\begin{enumerate}
		\item $\Ass_{S^{-1}R}(S^{-1}M) = \{S^{-1}\mathfrak{p} \mid \mathfrak{p} \in \Ass(M),\;\mathfrak{p} \cap S = \emptyset\}.$
		\item $\mathfrak{p} \in \Ass_R(M) \iff \mathfrak{p}R_{\mathfrak{p}} \in \Ass_{R_{\mathfrak{p}}}(M_{\mathfrak{p}}).$
	\end{enumerate}
\end{prop}

\begin{defn}%[]
	$\Supp(M) \vcentcolon= \{\mathfrak{p} \in \Spec(R) \mid M_{\mathfrak{p}} \neq 0\}.$
\end{defn}
\begin{prop}
	If $M$ is finitely generated, then $\Supp(M) = V(\ann M).$
\end{prop}
\begin{prop}
	If $0 \to N \to M \to L \to 0$ is exact, then $\Supp M = \Supp N \cup \Supp L.$
\end{prop}
\begin{prop}
	Let $L, K$ be f.g. $R$-modules. Then, $\Supp(K \otimes_R L) = \Supp L \cap \Supp K.$\\
	In particular, $\Supp(M/IM) = \Supp M \cap V(I).$
\end{prop}

\begin{prop}
	$\Ass(M) \subset \Supp(M).$
\end{prop}
Note that if $R$ is Noetherian and $I \unlhd R$ is an ideal, then $\Ass(R/I) \subset \Supp(R/I) = V(I).$

\begin{disc}
	Assume that $R$ and $M$ are Noetherian from now.
\end{disc}

\begin{prop}
	$\Ass M$ and $\Supp M$ have the same set of minimal primes.
\end{prop}
\begin{rem}
	Note that $\mathfrak{p}$ is a minimal prime over $\mathfrak{p}^n.$ That is, it is a minimal element of $V(\mathfrak{p}^n) = \Supp(R/\mathfrak{p}^n)$ and hence, an element of $\Ass(M/\mathfrak{p}^n).$

	Note that $V(\mathfrak{p}^n) = \Supp(R/\mathfrak{p}^n)$ is true because of the Noetherian assumption.
\end{rem}

\begin{thm}
	\begin{enumerate}
		\item There exists a sequence of $R$-submodules of $M$
		\begin{equation*} 
			(0) = M_0 \subset M_1 \subset \cdots \subset M_{n-1} \subset M_n = M
		\end{equation*}
		such that $M_{i+1}/M_i \cong R/\mathfrak{p_i}$ for $\mathfrak{p}_i \in \Spec(R).$
		\item Given any sequence as above, we have
		\begin{equation*} 
			\Ass M \subset \{\mathfrak{p}_1, \ldots, \mathfrak{p}_n\} \subset \Supp M.
		\end{equation*}
	\end{enumerate}
	In particular, $\Ass M$ is always finite and hence, the set of minimal primes over any ideal is finite.
\end{thm}

\begin{defn}%[]
	Let $N \le M$ be a submodule such that $\Ass(M/N) = \{\mathfrak{p}\}.$ Then, $M$ is called \deff{$\mathfrak{p}$-primary}.
\end{defn}
\begin{defn}%[]
	Let $M$ be a module such that $\Ass M = \{\mathfrak{p}\}.$ Then, $M$ is called \deff{$\mathfrak{p}$-coprimary}.
\end{defn}

\begin{ex}
	If $\mathfrak{m} \subset R$ is maximal, then $\mathfrak{m}^n$ is $\mathfrak{m}$-primary for all $n \ge 1.$\\
	If $\mathfrak{p} \subset R$ is prime, then $\mathfrak{p}^n$ need not be $\mathfrak{p}$-primary.
\end{ex}

\begin{prop}
	If $\mathfrak{q}$ is a $\mathfrak{p}$-primary ideal of $R,$ then $\mathfrak{q}R_{\mathfrak{p}}$ is a $\mathfrak{p}R_{\mathfrak{p}}$-primary ideal.
\end{prop}
\begin{proof} 
	Note that $(R/\mathfrak{q})_{\mathfrak{p}} \cong R_{\mathfrak{p}}/\mathfrak{q}R_{\mathfrak{p}}$ as $\mathbb{R}_{\mathfrak{p}}$-modules. By \Cref{prop:assoflocalisation}, we see that
	\begin{align*} 
		\mathfrak{a}R_{\mathfrak{p}} \in \Ass_{R_{\mathfrak{p}}}(R_{\mathfrak{p}}/\mathfrak{q}R_{\mathfrak{p}}) &\iff \mathfrak{a}R_{\mathfrak{p}} \in \Ass_{R_{\mathfrak{p}}}\left((R/\mathfrak{q})_{\mathfrak{p}}\right)\\
		&\iff \mathfrak{a} \in \Ass_R(R/\mathfrak{q}) = \{\mathfrak{p}\}\\
		&\iff \mathfrak{a} = \mathfrak{p}
	\end{align*}
	and hence, $\mathfrak{q}R_{\mathfrak{p}}$ is $\mathfrak{p}R_{\mathfrak{p}}$-primary.
\end{proof}

\begin{defn}%[]
	For $a \in R,$ define $\mu_a : M \to M$ as $x \mapsto ax.$
\end{defn}

\begin{defn}%[]
	\begin{align*} 
		\nil(M) \vcentcolon=&\; \{a \in R \mid \mu_a \text{ is nilpotent}\}\\
		=&\; \{a \in R \mid a^nM = 0 \text{ for some } n\}\\
		=&\; \sqrt{\ann M}
	\end{align*}
\end{defn}

\begin{prop}
	If $\Ass(M) = \{\mathfrak{p}\},$ then $\mathcal{Z}(M) = \nil M = \sqrt{\ann M}.$
\end{prop}
\begin{thm}
	$\md{\Ass M} = 1 \iff \mathcal{Z}(M) = \nil M.$\\
	If either condition holds, we have $\Ass M = \{\sqrt{\ann M}\}.$
\end{thm}

\begin{cor}
	If $N \le M$ is $\mathfrak{p}$-primary, then $\Ass(M/N) = \{\sqrt{\ann M/N}\}.$
\end{cor}
\begin{cor}
	$I$ is $\mathfrak{p}$-primary implies $\mathfrak{p} = \sqrt{I}.$
\end{cor}
\begin{rem}
	Note that if $\sqrt{I}$ is prime, it does not imply that $I$ is $\sqrt{I}$-primary.
\end{rem}
\begin{cor}
	$I$ is $\mathfrak{p}$-primary iff $\bigcup_{\mathfrak{p} \in \Ass(R/I)}\mathfrak{p} = \mathcal{Z}(R/I) = \nil(R/I) = I.$
\end{cor}

\begin{prop}
	If $N_1$ and $N_2$ are $\mathfrak{p}$-primary, so is $N_1 \cap N_2.$
\end{prop}

\begin{defn}%[]
	A submodule $N \le M$ is called \deff{reducible} if $N = N_1 \cap N_2$ with $N_1 \neq N \neq N_2.$ It is called \deff{irreducible} otherwise.
\end{defn}
\begin{prop}
	Prime ideals are irreducible.
\end{prop}

\begin{thm}
	Proper irreducible submodules are primary.
\end{thm}
\begin{thm}
	Any proper submodule can be written as an intersection of finitely many irreducible submodules.
\end{thm}

\begin{cor}
	Let $R$ be a Noetherian ring and $M$ a Noetherian $R$-module. If $N \lneq M$ is a proper $R$-submodule, then $N$ can be written as
	\begin{equation*} 
		N = N_1 \cap \cdots \cap N_r,
	\end{equation*}
	where $N_1, \ldots, N_r$ are primary submodules. 

	The above is called a \deff{primary decomposition} of $N.$
\end{cor}

\begin{defn}
	A primary decomposition is called \deff{minimal} if $\Ass(M/N_i) \neq \Ass(M/N_j)$ for $i \neq j.$

	It is called \deff{irredundant} if $N_i$ can be removed.
\end{defn}

\begin{thm}
	If $N = N_1 \cap \cdots \cap N_r$ is an irredundant primary decomposition and $\Ass(M/N_i) = \{\mathfrak{p}_i\},$ then $\Ass(M/N) = \{\mathfrak{p}_1, \ldots, \mathfrak{p}_r\}.$ 
\end{thm}

\begin{thm}
	If $\mathfrak{p}$ is a minimal associated prime of $M/N,$ then the $\mathfrak{p}$-primary component of $N$ is $\varphi_{\mathfrak{p}}^{-1}(N\mathfrak{p}),$ where $\varphi_{\mathfrak{p}} : M \to M_\mathfrak{p}$ is the natural map $x \mapsto \frac{x}{1}.$

	In particular, the component corresponding to the minimal prime is uniquely determined.
\end{thm}

\section{Artinian rings and Artinian modules}

We now drop the assumption from the previous chapter of rings and modules being Noetherian.

\begin{defn}%[]
	An $R$-module $M$ is called \deff{Artinian} if every descending chain of $R$-submodules of $M$ stabilises.

	$R$ is said to be an \deff{Artinian ring} if $R$ is Artinian as an $R$-module.
\end{defn}

\begin{prop}
	Let $\mathsf{k}$ be a field and $V$ a $\mathsf{k}$-module, i.e., a $\mathsf{k}$-vector space. Then, $V$ is Artinian iff $V$ is finite dimensional iff $V$ is Noetherian.
\end{prop}

\begin{prop}
	Let $R$ be an Artinian ring.
	\begin{enumerate}
		\item If $I$ is an ideal of $R,$ then $R/I$ is an Artinian ring.
		\item If $R$ is an integral domain, then $R$ is a field.
		\item More generally, every non zero divisor of $R$ is a unit.
		\item If $\mathfrak{p} \in \Spec(R),$ then $\mathfrak{p}$ is maximal. That is, $\Spec(R) = \mSpec(R).$ \\
		Thus, $\mathcal{N}(R) = \mathcal{J}(R).$
		\item $R$ has finitely many maximal ideals. (It may have infinitely many ideals, however.)
		\item If $I \unlhd R,$ then $\Ass(R/I) = \Supp(R/I) = V(I).$
		\item If $N = \mathcal{N}(R),$ then there exists $k$ such that $N^k = 0.$
		\item Let $0 \to N \to M \to L \to 0$ be an exact sequence. Then $M$ is Artinian iff $N$ and $L$ are Artinian.\\
		In particular, $\bigoplus_{i = 1}^n M_i$ is Artinian iff each $M_i$ is.
		\item If $M$ is a finitely generated $R$-module, then $M$ is an Artinian $R$-module and $R/\ann M$ is an Artinian ring.
	\end{enumerate}
\end{prop}

\begin{prop}
	Let $M$ be an $R$-module and $\mathfrak{m}_1, \ldots, \mathfrak{m}_n \in \mSpec R$ are maximal ideals such that $\mathfrak{m}_1\cdots\mathfrak{m}_nM = 0.$ Then,\\
	$M$ is Noetherian $\iff$ $M$ is Artinian.
\end{prop}
Note that the maximal ideals above need not be distinct. Moreover, $R$ is not assumed to be Artinian.

\begin{prop}
	Let $R$ be an Artinian ring. Then, $R$ is Noetherian ring.
\end{prop}
\begin{prop}
	Let $R$ be a Noetherian ring with $\Spec R = \mSpec R.$ Then, $R$ is an Artinian ring.
\end{prop}

\begin{prop}
	If $R$ is Artinian and $M$ an Artinian $R$-module, then $M$ is a Noetherian $R$-module. In particular, $M$ is finitely generated.
\end{prop}

\begin{thm}
	Let $R$ be an Artinian ring. Then, there exist uniquely determined Artinian local rings $R_1, \ldots, R_n$ such that
	\begin{equation*} 
		R \cong R_1 \times \cdots \times R_n.
	\end{equation*}
\end{thm}

\begin{defn}%[]
	An $R$-module $M \neq 0$ is called \deff{simple} if the only $R$-submodules of $M$ are $0$ and $M.$
\end{defn}
\begin{prop}
	An $R$-module $M$ is simple iff $M \cong R/\mathfrak{m}$ for some $\mathfrak{m} \in \mSpec R.$ The isomorphism is as $R$-modules. In particular, $M$ is cyclic.
\end{prop}
\begin{lem} 
	A simple module is both Noetherian and Artinian.
\end{lem}

\begin{defn}%[]
	Let $M$ be an $R$-module. A series of the form
	\begin{equation*} 
		0 = M_0 \subsetneq M_1 \subsetneq \cdots \subsetneq M_{n-1} \subsetneq M_n = M
	\end{equation*}
	is called a \deff{composition series} if $M_{i + 1}/M_i$ is simple for each $i.$ $n$ is called the \deff{length} of this composition series.
\end{defn}
Note that a composition series has finite length, by definition.

\begin{thm}
	$M$ has a composition series $\iff$ $M$ is Artinian and Noetherian.
\end{thm}

\begin{defn}
	Let $M \neq 0$ be an $R$-module. Define
	\begin{equation*} 
		l_R(M) \vcentcolon= \min \{n \mid M \text{ has a composition series of length }n\}.
	\end{equation*}
	$l_R(M) = \infty$ if the set on the right is empty. $l_R(M)$ is called the \deff{length} of $M$ over $R.$
\end{defn}
Note that if $R = \mathsf{k}$ is a field, then the length of $M$ is simply the dimension.
\begin{defn}%[]
	If $l_R(M) < \infty,$ then $M$ is called a \deff{finite length} module.
\end{defn}
\begin{prop}
	$M$ is a finite length module iff $M$ is Artinian and Noetherian.
\end{prop}
\begin{prop}
	Let $R$ be a Noetherian ring and $M$ a finite length $R$-module. Then, $\Ass(M) \subset \mSpec(R).$
\end{prop}

\begin{prop}
	Let $M$ be a finite length module and $N \le M.$ Then, $N$ also has finite length and $l_R(N) \le l_R(M)$ with equality iff $N = M.$
\end{prop}
\begin{thm}[Jordan-H\"older]
	Every composition series of a finite length module $M$ has the same length.

	Now, if
	\begin{align*} 
		0 = M_0 \subsetneq M_1 \subsetneq \cdots \subsetneq M_{n-1} \subsetneq M_n = M,\\
		0 = N_0 \subsetneq N_1 \subsetneq \cdots \subsetneq N_{n-1} \subsetneq N_n = M
	\end{align*}
	are two composition series of $M,$ then there exists a permutation $\sigma \in S_n$ such that
	\begin{equation*} 
		M_{i}/M_{i - 1} \cong N_{\pi(i)}/N_{\pi(i) - 1}
	\end{equation*}
	for all $1 \le i \le n.$ In other words, the quotients that appear are the same.
\end{thm}

\begin{prop}
	Let $M$ be a finite length module. Any series of the form
	\begin{equation*} 
		0 = M_0 \subsetneq M_1 \subsetneq \cdots \subsetneq M_{n-1} \subsetneq M_n = M
	\end{equation*}
	is actually a composition series.
\end{prop}

\begin{prop}
	Let $0 \to N \to M \to L \to 0$ be an exact sequence. Then, $l_R(M) = l_R(N) + l_R(L).$
\end{prop}
Note that $M$ is finite length iff $N$ and $L$ both are.

\begin{prop}
	If $R$ is Noetherian and $M$ a finite length $R$-module, then $\Ass(M) \subset \mSpec(R).$
\end{prop}

\section{Integral Extensions of Rings}

\begin{defn}%[]
	Let $R \subset S$ be non-zero commutative rings. An element $s \in S$ is called \deff{integral} over $R$ if there exists a \underline{monic} polynomial $f(x) \in R[x]$ such that $f(s) = 0.$

	Let
	\begin{equation*} 
		T = \{s \in S \mid s \text{ is integral over }R\}.
	\end{equation*}
	$T$ is called the \deff{integral closure} of $R$ in $S.$

	If $R$ is an integral domain and $S = Q(R),$ then $T$ is called the \deff{normalisation} of $R.$ $R$ is called \deff{normal} or \deff{integrally closed} if $T = R.$
\end{defn}
Recall that if $R$ is an integral domain, then $Q(R)$ denotes the field of fractions of $R.$

We shall shortly show that $T$ is a subring of $S.$

\begin{thm}
	If $R$ is a UFD, then $R$ is integrally closed. In other words, UFDs are normal.
\end{thm}
The converse is not true.

\begin{thm}[Cayley-Hamilton]
	Let $I \unlhd R$ be an ideal and $M$ a finitely generated $R$-module. Let $\varphi : M \to M$ be an $R$-endomorphism such that $\varphi(M) \subset IM.$ Then, $\varphi$ satisfies a monic polynomial of the form
	\begin{equation*} 
		x^n + a_1x^{n - 1} + \cdots + a_n
	\end{equation*}
	with $a_1, \ldots, a_n \in I.$
\end{thm}

\begin{cor}[Nakayama]
	Suppose $M$ is finitely generated over $R$ and $I \unlhd R$ is such that $M = IM.$ Then, there exists $a \in I$ such that $(1 + a)M = 0.$\\
	In particular, if $I \subset \mathcal{J}(R),$ then $M = 0.$
\end{cor}

\begin{cor}
	If $\varphi : M \to M$ is a surjective $R$-linear map, then $\varphi$ is an isomorphism. ($M$ is finitely generated as usual.)
\end{cor}

\begin{cor}
	Suppose $M \cong R^n.$ Then, any set of $n$ generators is linearly independent.
\end{cor}

\begin{cor}
	Let $R$ be a non-zero commutative ring. Then, $R^n \cong R^m$ (as $R$-modules) iff $n = m.$
\end{cor}

\begin{thm}
	Let $R \subset S$ be non-zero commutative rings and $s \in S.$ TFAE:
	\begin{enumerate}
		\item $s$ is integral over $R.$
		\item $R[s]$ is a finitely generated as an $R$-module.
		\item There exists a subring $T$ such that $R[s] \subset T \subset S$ and $T$ is a finitely generated $R$-module.
	\end{enumerate}
\end{thm}

\begin{thm}
	Let $R \subset S$ be a ring extension and $T = \overline{R}^S$ the integral closure of $R$ in $S.$ Then, $T$ is a subring of $S.$
\end{thm}

\begin{prop}
	If $R \subset T$ and $T \subset S$ are integral extensions, then so is $R \subset S.$
\end{prop}

\begin{cor}
	If $T$ is the integral closure of $R$ in $S,$ then the integral closure of $T$ in $S$ is $T.$

	Symbolically, if $T = \overline{R}^S,$ then $\overline{T}^S = T.$
\end{cor}

Note that if $R \subset S$ is a ring extension and $I \unlhd S$ is an ideal, then $I^c = R \cap I$ is an ideal of $R.$ (Called the contraction.) Also, one has the natural inclusion and projection maps as
\begin{equation*} 
	R \overset{i}{\hookrightarrow} S \overset{\pi}{\twoheadrightarrow} S/I.
\end{equation*}
Then, $I^c = \ker (\pi \circ i)$ and hence, $R/I^c$ is isomorphic to a subring of $S/I.$ We denote this inclusion by writing $R/I^c \hookrightarrow S/I.$

\begin{prop}
	If $R \subset S$ is an integral extension, then so is $R/I^c \hookrightarrow S/I.$
\end{prop}

\begin{defn}%[]
	Suppose $\varphi : R \to S$ is a ring map. Then, $\varphi$ is called \deff{integral} if $\varphi(R) \subset S$ is an integral extension.
\end{defn}

\begin{prop}
	Let $U \subset R$ be a multiplicatively closed subset and let $R \subset S$ be an integral extension. Then, $U^{-1}R \subset U^{-1}S$ is an integral extension.
\end{prop}

\begin{prop}
	Let $R$ be an integral domain. TFAE:
	\begin{enumerate}
		\item $R$ is integrally closed (normal).
		\item $R_\mathfrak{p}$ is integrally closed for all $\mathfrak{p} \in \Spec(R).$
		\item $R_\mathfrak{m}$ is integrally closed for all $\mathfrak{m} \in \mSpec(R).$
	\end{enumerate}
\end{prop}

\begin{lem} 
	Let $R \subset S$ be an integral extension of integral domains. \\
	Then, $R$ is a field $\iff$ $S$ is a field.
\end{lem}

\begin{cor}
	Let $R \subset S$ be rings (not necessarily domains) and $\mathfrak{q} \in \Spec S.$ Define $\mathfrak{p} \vcentcolon= R \cap \mathfrak{q}.$ \\
	Then, $\mathfrak{p} \in \mSpec R \iff \mathfrak{q} \in \mSpec S.$

	In particular, given an integral extension, the contraction of a maximal ideal is maximal.
\end{cor}

\begin{defn}%[]
	Let $R \subset S$ be rings. Suppose $Q \in \Spec S$ and $P \in \Spec R.$ $Q$ is said to \deff{lie over} $P$ if $Q^c = Q \cap R = P.$
\end{defn}

\begin{thm}[Lying over theorem]
	Let $R \subset S$ be an integral extension of rings and $\mathfrak{p} \in \Spec R.$ Then, there exists $\mathfrak{q} \in \Spec S$ such that $\mathfrak{q} \cap R = \mathfrak{p}.$

	In other words: Given an integral extension, there is always a prime lying over a prime.
\end{thm}

\begin{thm}[Going up theorem]
	Let $R \subset S$ be an integral extension. Let $\mathfrak{p}_1, \mathfrak{p}_2 \in \Spec R$ with $\mathfrak{p}_1 \subset \mathfrak{p}_2$ and $\mathfrak{q}_1 \in \Spec S$ be such that $\mathfrak{q}_1 \cap R = \mathfrak{p}_1.$ Then, there exists $\mathfrak{q}_2 \in \Spec S$ such that $\mathfrak{q}_1 \subset \mathfrak{q}_2$ and $\mathfrak{q}_2 \cap R = \mathfrak{p}_2.$
\end{thm}
\begin{center}
	\begin{tikzcd}
	S \arrow[d, "\text{int}"', no head] &  & \mathfrak{q}_1 \arrow[d, no head] \arrow[r,symbol=\subset] & \exists \mathfrak{q}_2 \arrow[d, no head, dotted] \\
	R                                   &  & \mathfrak{p}_1 \arrow[r,symbol=\subset]                    & \mathfrak{p}_2                           
	\end{tikzcd}
\end{center}

In fact, inductively, we see that any chain above can be ``completed.''

\begin{center}
	\begin{tikzcd}
S \arrow[d, "\text{int}"', no head] &  & \mathfrak{q}_1 \arrow[d, no head] \arrow[r,symbol=\subset] & \mathfrak{q}_2 \arrow[d, no head] \arrow[r,symbol=\subset] & \cdots \arrow[r,symbol=\subset] & \mathfrak{q}_m \arrow[d, no head] \arrow[r,symbol=\subset] & \exists \mathfrak{q}_{m+1} \arrow[d, no head, dashed] \arrow[r,symbol=\subset] & \cdots \arrow[r,symbol=\subset] & \exists \mathfrak{q}_{n} \arrow[d, no head, dashed] \\
R                                   &  & \mathfrak{p}_1 \arrow[r,symbol=\subset]                    & \mathfrak{p}_2 \arrow[r,symbol=\subset]                    & \cdots \arrow[r,symbol=\subset] & \mathfrak{p}_m \arrow[r,symbol=\subset]                    & \mathfrak{p}_{m+1} \arrow[r,symbol=\subset]                                    & \cdots \arrow[r,symbol=\subset] & \mathfrak{p}_{n}                                   
\end{tikzcd}
\end{center}

\begin{prop}[Incompatibility (INC)]
	Let $R \subset S$ be an integral extension of rings. Let $Q_1, Q_2 \in \Spec S$ lie over $P \in \Spec R.$ If $Q_1$ and $Q_2$ are distinct, then they are incomparable. \\
	That is, $Q_1 \neq Q_2 \implies Q_1 \not\subset Q_2$ and $Q_2 \not\subset Q_1.$
\end{prop}
\begin{center}
	\begin{tikzcd}
	S \arrow[dd, "\text{int}"', no head] & Q_1 \arrow[rdd, no head] &   & Q_2 \arrow[ldd, no head] \\
	                                     &                          &   &                          \\
	R                                    &                          & P &                         
	\end{tikzcd}
\end{center}	

\begin{lem} 
	Let $f : R \to S$ be any ring homomorphism and $P \in \Spec R.$ TFAE:
	\begin{enumerate}
		\item $P^{ec} = f^{-1}(f(P)S) = P,$ and
		\item $\exists Q \in \Spec S$ such that $Q^c = P.$ That is, a prime lies over $P.$
	\end{enumerate}
\end{lem}
Note that the above is a general fact, no assumptions of integral extensions.

\begin{thm}[Going down theorem]
	Let $R$ be a \underline{normal} domain, $S$ an \underline{integral} domain and $R \subset S$ be an integral extension.

	Given $P_0, P_1 \in \Spec R$ with $P_0 \supset P_1$ and a prime $Q_0 \in \Spec S$ lying over $P_0,$ there exists a prime $Q_1 \in \Spec S$ lying over $P_1$ with $Q_0 \supset Q_1.$
\end{thm}
\begin{center}
	\begin{tikzcd}
	S \arrow[d, "\text{int}"', no head] &  & \mathfrak{q}_1 \arrow[d, no head] \arrow[r,symbol=\supset] & \exists \mathfrak{q}_2 \arrow[d, no head, dotted] \\
	R                                   &  & \mathfrak{p}_1 \arrow[r,symbol=\supset]                    & \mathfrak{p}_2                           
	\end{tikzcd}
\end{center}

In fact, inductively, we see that any chain above can be ``completed.''

\begin{center}
	\begin{tikzcd}
S \arrow[d, "\text{int}"', no head] &  & \mathfrak{q}_1 \arrow[d, no head] \arrow[r,symbol=\supset] & \mathfrak{q}_2 \arrow[d, no head] \arrow[r,symbol=\supset] & \cdots \arrow[r,symbol=\supset] & \mathfrak{q}_m \arrow[d, no head] \arrow[r,symbol=\supset] & \exists \mathfrak{q}_{m+1} \arrow[d, no head, dashed] \arrow[r,symbol=\supset] & \cdots \arrow[r,symbol=\supset] & \exists \mathfrak{q}_{n} \arrow[d, no head, dashed] \\
R                                   &  & \mathfrak{p}_1 \arrow[r,symbol=\supset]                    & \mathfrak{p}_2 \arrow[r,symbol=\supset]                    & \cdots \arrow[r,symbol=\supset] & \mathfrak{p}_m \arrow[r,symbol=\supset]                    & \mathfrak{p}_{m+1} \arrow[r,symbol=\supset]                                    & \cdots \arrow[r,symbol=\supset] & \mathfrak{p}_{n}                                   
\end{tikzcd}
\end{center}

\begin{thm}
	Let $R$ be a \underline{Noetherian} normal domain with quotient field $K.$ Let $K \subset L$ be a \underline{separable} extension. Then, $\overline{R}^L$ is a finite $R$-module. In particular, it is a Noetherian ring.
\end{thm}

\section{Dimension Theory of Affine Algebra over Fields}

\begin{lem}[Artin-Tate Lemma]
	Let $R \subset S \subset T$ be rings. Suppose
	\begin{enumerate}
		\item $R$ is Noetherian,
		\item $T$ is a finitely generated $S$ module,
		\item $T$ is a finitely generated $R$ algebra.
	\end{enumerate}
	\begin{center}
		\begin{tikzcd}
		{R[t_1, \ldots, t_s] = T = St'_1 + \cdots + St'_k} \arrow[d, no head] \\
		S \arrow[d, no head]                                                  \\
		R                                                                    
		\end{tikzcd}
	\end{center}
	Then, $S$ is a finitely generated $R$-algebra. In other words, there exist $s_1, \ldots, s_n \in S$ such that $S = R[s_1, \ldots, s_n].$\\
	In particular, $S$ is Noetherian.
\end{lem}

\begin{defn}%[]
	Let $\mathsf{k}$ be a field. An \deff{affine $\mathsf{k}$-algebra} is a ring of the form $R = \mathsf{k}[x_1, \ldots, x_n]/I$ for some ideal $I \unlhd \mathsf{k}[x_1, \ldots, x_n].$
\end{defn}

\begin{lem}[Zariski] 
	Let $\mathsf{k}$ be any field and $R = \mathsf{k}[x_1, \ldots, x_n]/I$ be an affine $\mathsf{k}$-algebra which is also a field. \hfill{\color{ForestGreen}(That is, $I$ is maximal.)} \\
	Then, $R$ is an algebraic extension of $\mathsf{k}.$
\end{lem}

\begin{cor}
	Let $\varphi : R \to S$ be a ring homomorphism, where $R$ and $S$ are affine $\mathsf{k}$-algebras. Let $\mathfrak{m} \in \mSpec(S).$ Then, $\varphi^{-1}(\mathfrak{m}) \in \mSpec(R).$
\end{cor}

\begin{thm}[Weak Nullstellensatz]
	If $\mathsf{k}$ is algebraically closed, then maximal ideals $\mathfrak{m} \in \mSpec \mathsf{k}[x_1, \ldots, x_n]$ are precisely those of the form $\mathfrak{m}_a = (x_1 - a_1, \ldots, x_n - a_n)$ for some $(a_1, \ldots, a_n) \in \mathsf{k}^n.$
\end{thm}

\begin{cor}[Criterion for solvability]
	Let $p_1(x_1, \ldots, x_n), \ldots, p_s(x_1, \ldots, x_n)$ be polynomials in $\mathsf{k}[x_1, \ldots, x_n].$ Then, the polynomials have a common solution iff the ideal generated by them is not the whole ring.
\end{cor}
\begin{rem}
	In fact, one need not assume $s < \infty$ in the above.	
\end{rem}

\begin{defn}%[]
	Given a field $\mathsf{k},$ $\mathbb{A}_\mathsf{k}^n$ denotes the \deff{affine $n$-space over $\mathsf{k}$}. It is simply the set $\mathsf{k}^n$ without any vector space structure.

	Given any ideal $I \unlhd \mathsf{k}[x_1, \ldots, x_n],$ we define the \deff{zero set of $I$} as
	\begin{equation*} 
		\mathcal{Z}(I) = \{\underline{a} \in \mathbb{A}_\mathsf{k}^n : f(\underline{a}) = 0 \text{ for all } f \in I\} \subset \mathbb{A}_\mathsf{k}^n.
	\end{equation*}
	A subset of $\mathbb{A}_\mathsf{k}^n$ which is the zero set of some ideal is called an \deff{algebraic set}.

	Given an algebraic set $X \subset \mathbb{A}_\mathsf{k}^n,$ we define the \deff{ideal of $X$} as
	\begin{equation*} 
		\mathcal{I}(X) = \{f \in \mathsf{k}[x_1, \ldots, x_n] : f(x) = 0 \text{ for all } x \in X\} \subset \mathsf{k}[x_1, \ldots, x_n].
	\end{equation*}
\end{defn}

\begin{rem}
	An ideal of an algebraic set is always a radical ideal.	
\end{rem}

\begin{thm}[Strong Nullstellensatz]
	If $\mathsf{k}$ is algebraically closed and $I \unlhd \mathsf{k}[x_1, \ldots, x_n] = S$ an ideal, then $\mathcal{I}(\mathcal{Z}(I)) = \sqrt{I}.$

	In particular, there is a bijection
	\begin{align*} 
		\{\text{radical ideals in } S\} \leftrightarrow \{\text{algebraic subsets in }\mathbb{A}_\mathsf{k}^n\}.
	\end{align*}
\end{thm}

\begin{defn}%[]
	Given a polynomial $f \in \mathsf{k}[x_1, \ldots, x_n],$ we can write
	\begin{equation*} 
		f = \sum_{\alpha \in (\mathbb{N} \cup \{0\})^n} a_\alpha x^\alpha.
	\end{equation*}
	If $a_\alpha \neq 0,$ we say that $x_\alpha$ is a \deff{term} of $f.$

	Writing $\alpha = (\alpha_1, \ldots, \alpha_n),$ $\md{\alpha}$ denotes the maximum of $\alpha_1, \ldots, \alpha_n.$
\end{defn}

\begin{prop}
	Let $\mathsf{k}$ be any field. Let $f \in S = \mathsf{k}[x_1, \ldots, x_n]$ be a non-constant polynomial. Let 
	\begin{equation*} 
		N > \max\{\md{\alpha} : \alpha \in (\mathbb{N} \cup \{0\})^n,\;x^{\alpha}\text{ is a term of }f\}.
	\end{equation*}
	Without loss of generality, we may assume that $x_n$ appears non-trivially in some term of $f.$ Define the map $\Phi : S \to S$ by identity on $\mathsf{k}$ and
	\begin{align*} 
		x_i \mapsto \begin{cases}
			x_i - x_n^{N^i} & i \neq n,\\
			x_n & i = n.
		\end{cases}
	\end{align*}
	Then, $\Phi$ is an automorphism such that $\Phi(f)$ is monic in $x_n,$ up to a constant. That is,
	\begin{equation*} 
		\Phi(f) = cx_n^r + g_1x_n^{r - 1} + \cdots + g_{n},
	\end{equation*}
	where $0 \neq c \in \mathsf{k}$ and $g_1, \ldots, g_n \in \mathsf{k}[x_1, \ldots, x_{n - 1}].$
\end{prop}

\begin{thm}[Noetherian Normalisation Lemma] \label{thm:NNL}
	Let $R = \mathsf{k}[\theta_1, \ldots, \theta_n]$ be an affine $\mathsf{k}$-algebra. Then, there exist algebraically independent elements $z_1, \ldots, z_d \in R$ such that $\mathsf{k}[x_1, \ldots, x_n] \subset R$ is an integral extension.
	\begin{center}
		\begin{tikzcd}
			R \arrow[d, "\text{integral}", no head]\\
			\mathsf{k}[z_1, \ldots, z_d] = S.
		\end{tikzcd}
	\end{center}
	In particular, $R$ is a finite $S$ module.
\end{thm}

\begin{cor}
	Let $R$ be an affine $\mathsf{k}$-algebra and $I \unlhd R$ an ideal. Then
	\begin{equation*} 
		\sqrt{I} = \bigcap_{\mathfrak{m} : I \subset \mathfrak{m} \in \mSpec(R)} \mathfrak{m}.
	\end{equation*}
	In particular, $\mathcal{N}(R) = \mathcal{J}(R).$
\end{cor}

\begin{defn}%[]
	A \deff{saturated chain of prime ideals} is a chain
	\begin{equation*} 
		\mathfrak{p}_0 \subsetneq \mathfrak{p}_1 \subsetneq \cdots \subsetneq \mathfrak{p}_n
	\end{equation*}
	of prime ideals such that no prime ideal can be inserted strictly in between anywhere above. (In other words, there exists no $i \in \{0, \ldots, n - 1\}$ and no $\mathfrak{q} \in \Spec(R)$ such that $\mathfrak{p}_i \subset \mathfrak{q} \subset \mathfrak{p}_{i + 1}.$)

	The \deff{length} of the above chain is $n.$ The \deff{dimension} of $R$ is defined as
	\begin{equation*} 
		\dim(R) = \sup\{n : \exists \text{ a saturated chain of length }n\}.
	\end{equation*}
\end{defn}
$\dim(R)$ may be $\infty$ even if $R$ is Noetherian. 
\begin{defn}%[]
	Given a prime ideal $\mathfrak{p} \unlhd R,$ the \deff{height} of $\mathfrak{p}$ is defined as 
	\begin{equation*} 
		\htt(\mathfrak{p}) = \dim(R_\mathfrak{p}).
	\end{equation*}
\end{defn}

\begin{ex}
	Here are some examples.
	\begin{enumerate}
		\item If $R$ is Artinian, then $\dim(R) = 0.$ In particular, $\dim(\mathsf{k}) = 0.$
		\item $\dim(\mathbb{Z}) = 1.$
		\item $\dim(\mathsf{k}[X]) = 1.$
		\item In general, if $R$ is a PID and not a field, then $\dim(R) = 1.$
	\end{enumerate}
\end{ex}

\begin{prop}
	Let $R \subset S$ be an integral extension of rings. Then,
	\begin{enumerate}
		\item $\dim(R) = \dim(S).$
		\item If $I \lhd S$ is a proper ideal, then $\dim(S/I) = \dim(R/I \cap R).$
		\item Suppose $S$ is integral and $R$ normal. Let $Q \in \Spec(S).$ Then, $\dim(S_Q) = \dim(R_{Q \cap R}).$
	\end{enumerate}
\end{prop}

\begin{thm}
	Let $R$ be an affine algebra over a field $\mathsf{k}.$ Let $z_1, \ldots, z_d \in R$ be such that $S = \mathsf{k}[z_1, \ldots, z_d] \subset R$ is an integral extension. \hfill {\color{ForestGreen}(Exists by NNL.)}\\
	Then,
	\begin{enumerate}
		\item $\dim(R) = d = \dim(\mathsf{k}[z_1, \ldots, z_d]).$
		\item Any maximal saturated chain of prime ideals in $R$ has length $d.$
	\end{enumerate}
\end{thm}
\begin{rem}
	The above shows that the $d$ in \nameref{thm:NNL} is determined uniquely. Moreover, it shows that the dimension of polynomial ring in $d$ variables over a field is $d.$
\end{rem}
\end{document}