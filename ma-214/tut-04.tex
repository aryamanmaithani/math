\documentclass{article}
\usepackage{amsmath, amssymb, amsfonts, amsthm, mathtools}
\usepackage[utf8]{inputenc}
\usepackage[inline]{enumitem}
\usepackage{cancel}
\usepackage{soul}
\usepackage{hyperref}
\newtheorem{theorem}{Theorem}
\newtheorem{lem}{Lemma}
\newtheorem{defn}{Definition}

\setlength\parindent{0pt}

\usepackage{xcolor}
\definecolor{mybgcolor}{RGB}{50, 50, 50} %46, 51, 63

\usepackage{pagecolor}
\pagecolor{mybgcolor}
\color{white}

\usepackage{geometry}
\geometry{
    a4paper,
    total={170mm,257mm},
    left=20mm,
    top=20mm,
}

\title{MA 214: Tutorial 4}
\author{Aryaman Maithani}
\date{12-02-2020}

\begin{document}
\maketitle
4. Suppose $p(x)$ is a polynomial of degree $\le 3.$\\
Show that if we compute $T_{2N}'$ in Romberg integration, then we get the exact value of the integral.\\~\\
\textbf{Solution.}\\
Note that Romberg integration uses composite trapezoidal rule for the approximations.\\
The idea is to show that the $T_{2N}'$ computed is actually the value that is approximated by the composite Simpson's rule. However, we know that the error for Simpson's rule is $0$ when integrating a polynomial of degree $\le 3.$ (There's the $f^{(4)}(\xi)$ term.)\\~\\
The rest is just simple calculation:\\
First, we have that 
\[T_N = \dfrac{h}{2}\left[f(x_0) + 2\displaystyle\sum_{i=1}^{N-1}f(a + ih) + f(x_N)\right],\;\text{where } h = \frac{b - a}{N}, \text{ and}\]
\[T_{2N} = \dfrac{h}{4}\left[f(x_0) + 2\displaystyle\sum_{i=1}^{2N-1}f\left(a + i\frac{h}{2}\right) + f(x_{2N})\right],\]
where the $h$ is as before.\\
$T_{2N}$ can be rearranged to be better written as:
\[T_{2N} = \frac{1}{2}T_{N} + \frac{h}{2}\sum_{i=1}^{N}f\left(a + (2i - 1)\frac{h}{2}\right),\]
for the same $h = \frac{b - a}{2N}.$\\~\\
Now, using the formula
\[T_{2N}' = T_{2N} - \frac{T_N - T_{2N}}{4 - 1} = \frac{4T_{2N} - T_N}{3},\]
we get:
\begin{align*} 
	T_{2N}' &= \frac{1}{3}\left(2T_N + 2h\sum_{i=1}^{N}f\left(a + (2i - 1)\frac{h}{2}\right) - T_N\right)\\~\\
	&= \frac{1}{3}\left(\dfrac{h}{2}\left[f(x_0) + 2\displaystyle\sum_{i=1}^{N-1}f(a + ih) + f(x_N)\right] + 2h\sum_{i=1}^{N}f\left(a + (2i - 1)\frac{h}{2}\right)\right)\\~\\
	&= \frac{h}{6}\left[f(x_0) + 2\displaystyle\sum_{i=1}^{N-1}f(a + ih) + f(x_N) + 4\sum_{i=1}^{N}f\left(a + (2i - 1)\frac{h}{2}\right)\right]\\~\\
	&= \frac{h}{6}\left[f(x_0) + 2\displaystyle\sum_{i=1}^{N-1}f(a + ih) + 4\sum_{i=1}^{N}f\left(a + (2i - 1)\frac{h}{2}\right) + f(x_N)\right] & \text{where} h = \frac{b - a}{N}
\end{align*}
The above is precisely composite Simpson's rule with $N$ divisions. Thus, we are done.
\end{document}