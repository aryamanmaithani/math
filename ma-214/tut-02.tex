\documentclass{article}
\usepackage{amsmath, amssymb, amsfonts, amsthm, mathtools}
\usepackage[utf8]{inputenc}
\usepackage[inline]{enumitem}
\usepackage{cancel}
\usepackage{soul}
\usepackage{hyperref}
\newtheorem{theorem}{Theorem}
\newtheorem{lem}{Lemma}
\newtheorem{defn}{Definition}

\setlength\parindent{0pt}

\usepackage{xcolor}
\definecolor{mybgcolor}{RGB}{50, 50, 50} %46, 51, 63

\usepackage{pagecolor}
\pagecolor{mybgcolor}
\color{white}

\usepackage{geometry}
\geometry{
    a4paper,
    total={170mm,257mm},
    left=20mm,
    top=20mm,
}

\title{MA 214: Tutorial 2}
\author{Aryaman Maithani}
\date{29-01-2020}

\begin{document}
\maketitle
1. Let $x_0, x_1, \ldots, x_m$ be not necessarily distinct points. Let $f$ and $g$ be two polynomials such that $f$ and $g$ agree on $x = x_0, x_1, \ldots, x_m.$ Also, $\deg f \le m$ and $\deg g \le m.$ Show that $f = g.$\\

\textbf{Solution}.\\
Let $y_1, \ldots, y_r$ be the distinct points out of $x_0, \ldots x_m.$\\
Let $m_i$ denote the number of times that $y_i$ is repeated.\\
Thus, we can straight away note that
\[m_1 + m_2 + \cdots + m_r = m+1. \qquad \text{(I)}\]
Now, let us define $h(x) := f(x) - g(x).$ Note that $h$ has the following properties:\\
(i) $h$ is a polynomial of degree $\le m$\\
(ii) $h^{(k)}(x) = f^{(k)}(x) - g^{(k)}(x)$ for all $k \ge 0.$\\~\\
The fact that $f$ and $g$ agree on $x_0, \ldots, x_m$ gives us that:\\
\begin{align*} 
	h(y_1) = h^{(1)}(y_1) = &\cdots = h^{(m_1 - 1)}(y_1) = 0\\
	h(y_2) = h^{(1)}(y_2) = &\cdots = h^{(m_2 - 1)}(y_2) = 0\\
	&\vdots\\
	h(y_r) = h^{(1)}(y_r) = &\cdots = h^{(m_r - 1)}(y_r) = 0	
\end{align*}
Thus, $y_1$ is root of $h$, repeated $m_1$ times and so on.\\
This means that $h$ can be written as:
\[h(x) = (x - y_1)^{m_1}\cdots(x - y_r)^{m_r}q(x),\]
for some polynomial $q(x).$\\
However, note that the degree of the left hand side is $\le m,$ by (i).\\
On the other hand, the degree of $(x - y_1)^{m_1}\cdots(x - y_r)^{m_r}$ is $m+1,$ by (I).\\
Thus, if $q(x)$ is not identically zero, we'd get that the degree of the RHS is strictly greater than the degree of the LHS, a contradiction.\\
Thus, we get that $q(x) = 0$ for all $x.$ This, in turn, gives that $h(x) = 0$ for all $x.$\\
This gives us that $f = g,$ as desired.

\hrulefill

4. A function $f(x)$ has a double zero at $z_1$ and a triple zero at $z_2.$ Determine the form of the polynomial of degree $\le 5$ which interpolates $f(x)$ twice at $z_1,$ three times at $z_2$ and once at some other point $z_3.$\\

\textbf{Solution.}\\
One ``stupid'' way to do this would obviously be to construct the general divided difference table and write it. However, we can be smarter.\\
Using the same idea as earlier, we know that the form of the polynomial, say $h(x)$, must be:\\
$h(x) = (x - z_1)^2(x - z_2)^3q(x),$ where $q(x)$ is some polynomial.\\~\\
The factor $(x - z_1)^2$ appeared because $z_1$ was a double root of $f$ and $h$ is supposed to interpolate $f$ twice. Similar justification of $(x - z_2)^3.$\\~\\
Now, note that $h(x)$ has degree $\le 5$ and $(x - z_1)^2(x - z_2)^3$ already has degree $5.$ Thus, $p(x)$ must be a constant polynomial.\\
Let $p(x) = c.$ Now, we just have to determine what the constant $c$ must be.\\
To do this, we just evaluate $h(x)$ at $x = z_3$ to get
\[h(z_3) = (z_3 - z_1)^2(z_3 - z_2)^3c.\]
Note that $h(z_3)$ must equal $f(z_3)$ as $h$ interpolates $f$ at $z_3.$\\
Thus, $c = \dfrac{f(z_3)}{(z_3 - z_1)^2(z_3 - z_2)^3}.$\\~\\
This gives us the final polynomial to be
\[h(x) = f(z_3)\dfrac{(x - z_1)^2(x - z_2)^3}{(z_3 - z_1)^2(z_3 - z_2)^3}.\]
(Note that the denominator is indeed nonzero.)
\end{document}