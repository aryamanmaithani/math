\documentclass{article}
\usepackage{amsmath, amssymb, amsfonts, amsthm, mathtools}
\usepackage[utf8]{inputenc}
\usepackage[inline]{enumitem}
\usepackage{cancel}
\usepackage{soul}
\usepackage{hyperref}
\newtheorem{theorem}{Theorem}
\newtheorem{lem}{Lemma}
\newtheorem{defn}{Definition}

\setlength\parindent{0pt}

\usepackage{xcolor}
\definecolor{mybgcolor}{RGB}{50, 50, 50} %46, 51, 63

\usepackage{pagecolor}
\pagecolor{mybgcolor}
\color{white}

\usepackage{geometry}
\geometry{
    a4paper,
    total={170mm,257mm},
    left=20mm,
    top=20mm,
}

\title{MA 214: Tutorial 1}
\author{Aryaman Maithani}
\date{22-01-2020}

\begin{document}
\maketitle

2. Let $f:[a, b] \to \mathbb{R}$ be a continuous function and let $g \ge 0$ be an integrable function on $[a, b].$ Show that
\[\int_{a}^{b} f(x)g(x) \text{d}x = f(\xi)\int_{a}^{b} g(x) \text{d}x \qquad \text{for some } \xi \in [a, b].\]

\textbf{Solution.}\\
%Case 1. $g(x) = 0$ for all $x \in [a, b].$ In this case, it's trivial. One can simply choose $\xi = a$ and be done as both sides are zero.\\
%Case 2. $g(x)$ is not identically zero.\\
%As $g$ 
As $f$ is a continuous function defined on a closed and bounded interval, it is bounded and moreover, it attains these bounds. (Extreme value theorem, done in 105.)\\
In other words, there exist $m, M \in [a, b]$ such that $f(m) \le f(x) \le f(M)$ for all $x \in [a, b].$\\
(Note that it's not necessary that $m \le M$ but that's not required anyway.)\\
As $g \ge 0,$ the above gives us that
\[f(m)g(x) \le f(x)g(x) \le f(M)g(x) \quad \forall x \in [a, b].\]
Integrating on all three sides gives that $\displaystyle\int_{a}^{b} f(x)g(x) \text{d}x$ lies between $If(m)$ and $If(M)$ where $I = \displaystyle\int_{a}^{b} g(x)\text{d}x.$\\
Now, note that $h(x) := If(x)$ defined for $x \in [a, b]$ is a continuous function and thus, by intermediate value property, there exists $\xi$ between $m$ and $M$ such that $h(\xi) = \displaystyle\int_{a}^{b} f(x)g(x) \text{d}x.$ As $m, M \in [a, b],$ we also get that $\xi \in [a, b].$\\~\\
Thus, we get that $\displaystyle f(\xi)\int_{a}^{b} g(x) \text{d}x = \int_{a}^{b} f(x)g(x) \text{d}x,$ as desired.

\hrulefill

5. Prove that the $k^{\text{th}}$ divided difference $p[x_0, \ldots, x_k]$ of a polynomial $p$ of degree $\le k$ is independent of the interpolation points $x_0, x_1, \ldots, x_k.$ \\

\textbf{Solution}.\\
Let $p(x)$ be a polynomial of degree $\le k.$\\
Then, $p(x) = a_0 + a_1x + \cdots a_{k}x^k$ for some $a_0, \ldots, a_k \in \mathbb{R}.$ \hfill ($a_k$ may be zero.)\\
Now, we show that the $k^{\text{th}}$ divided difference equals $a_k$ independent of the choice of $x_0, \ldots, x_k.$ This would clearly prove the desired result.\\~\\
To see this, simply observe that $P_k(x)$ is a polynomial of degree $\le k$ such that $P_k(x_i) = p(x_i)$ for all $i \in \{0, \ldots, k\}.$ In other words, $p$ and $q$ agree at $k+1$ points. By the uniqueness theorem seen earlier, this forces $p(x) = P_k(x).$ In turn, this forces that $p(x)$ and $P_k(x)$ have the same leading coefficient. We already know the leading coefficient of $p(x)$ is $a_k,$ by definition. On the other hand, recalling the definition of $P_k(x)$ gives us:\\
\begin{align*} 
		P_k(x) := p[x_0]  +& p[x_0, x_1](x - x_0)\\
						 +& p[x_0, x_1, x_2](x - x_0)(x - x_1)\\
						 +& \ldots\\
						  &\vdots\\
						 +& p[x_0, x_1, \ldots, x_k](x - x_0)(x - x_1)\cdots(x - x_{k-1})
	\end{align*}
	Thus, the leading coefficient of $P_k(x)$ is $p[x_0, \ldots, x_k].$\\
	This completes the proof as we conclude that $a_k = p[x_0, \ldots, x_k],$ as desired.
\end{document}