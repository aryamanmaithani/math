\documentclass[12pt]{article}	% Always compile at least twice.
\usepackage[lmargin=1in,rmargin=1in,tmargin=1in,bmargin=1in]{geometry}
% \usepackage{pdfpages}

% Cover Information	
\def\univname{}
\def\coursenum{MA 408}
\def\coursename{Measure Theory}
\def\professor{}
\def\student{Aryaman Maithani}
\def\semesteryear{Spring 2020-21}
\def\imagename{../icon.pdf}
\def\scalesize{2}					% Scale Logo Size 

% Style Package (Load After Cover Information)
\usepackage{../aryaman}	% Change to match style file

\newcommand{\loc}{L^{1}_{\operatorname{loc}}}
\DeclareMathOperator{\BV}{BV}
\DeclareMathOperator{\NBV}{NBV}

% -------------------
% Content
% -------------------
\begin{document}


% Cover Page
\coverpage

% Last Updated
\updated{\today}

% Table of Contents
\thispagestyle{empty}
\tableofcontents
\newpage
\pagestyle{fancy}
\setcounter{page}{1}

\section{Measures} % Lecture 1

\subsection{Introduction}

\begin{thm}[Non existence of ideal measure]
	There is no map $\mu : \mathcal{P}(\mathbb{R}) \to [0, \infty]$ such that
	\begin{enumerate}
		\item $\mu(\emptyset) = 0,$
		\item $\mu(E) = \mu(x + E)$ for all $x \in \mathbb{R}$ and $E \in \mathcal{P}(\mathbb{R}),$\\
		where $x + E \vcentcolon= \{x + y \mid y \in E\},$
		\item for any disjoint countable collection $\{E_i\}_{i}^{\infty}$ of subsets of $\mathbb{R},$ we have
		\begin{equation*} 
			\mu\left(\bigsqcup_{i = 1}^\infty E_i\right) = \sum_{i = 1}^{\infty} \mu(E_i),
		\end{equation*}
		\item $\mu([0, 1]) = 1.$
	\end{enumerate}
\end{thm}
Note that the last is a ``normalisation'' property. Otherwise $\mu \equiv 0$ or $\mu(X) = \begin{cases}
	0 & X = \emptyset,\\
	\infty & \text{otherwise}
\end{cases}$ would also satisfy and give us ``useless'' functions.

Replacing ``countable union'' with ``finite union'' also won't do the trick in general due to the Banach-Tarski ``paradox'' (theorem).

Both the above required a use of the Axiom of Choice.

\subsection{\texorpdfstring{$\sigma$}{s}-algebras}


\begin{defn}[Algebra]
	Let $X$ be a non-empty set.\\
	An \deff{algebra (``field'')} on $X$ is a non-empty collection $\mathcal{F} \subset \mathcal{P}(X)$ satisfying
	\begin{enumerate}
		\item (Closure under complements) $A \in \mathcal{F} \implies A^c \in \mathcal{F},$
		\item (Closure under finite unions) $A_1, \ldots, A_n \in \mathcal{F} \implies \bigcup_{i = 1}^n A_i \in \mathcal{F}.$
	\end{enumerate}
\end{defn}

\begin{defn}[$\sigma$-algebra]
	Let $X$ be a non-empty set.\\
	A \deff{$\sigma$-algebra (``$\sigma$-field'')} on $X$ is a non-empty collection $\mathcal{F} \subset \mathcal{P}(X)$ satisfying
	\begin{enumerate}
		\item (Closure under complements) $A \in \mathcal{F} \implies A^c \in \mathcal{F},$
		\item (Closure under countable unions) $A_1, A_2, \ldots \in \mathcal{F} \implies \bigcup_{i = 1}^\infty A_i \in \mathcal{F}.$
	\end{enumerate}
\end{defn}

\begin{ex}[Countable-cocountable $\sigma$-algebra]
	Let $X \neq \emptyset.$ Then,
	\begin{equation*} 
		\mathcal{F} = \{E \subset X \mid E \text{ or } E^c \text{ is countable}\}
	\end{equation*}
	is a $\sigma$-algebra on $X.$
\end{ex}

\begin{defn}[$\sigma$-algebra generated by a set]
	Let $\mathcal{E} \subset \mathcal{P}(X).$ Then,
	\begin{equation*} 
		\mathcal{M}(\mathcal{E}) \vcentcolon= \bigcap_{\substack{\mathcal{E} \subset \mathcal{B}\\
		\mathcal{B} \text{ is a }\sigma-\text{algebra}}} \mathcal{B}
	\end{equation*}
	is a $\sigma$-algebra. Moreover, it is the smallest $\sigma$-algebra containing $\mathcal{B}.$

	This is called the \deff{$\sigma$-algebra} generated by $\mathcal{B}.$
\end{defn}

\begin{defn}[Borel $\sigma$-algebra]
	Let $(X, \mathcal{T})$ be a topological space. The $\sigma$-algebra generated by $\mathcal{T}$ is called the \deff{Borel $\sigma$-algebra} on $X,$ denoted $\mathcal{B}(X).$
\end{defn}
In other words, $\mathcal{B}(X)$ is the $\sigma$-algebra generated by the open sets of $X.$

\begin{prop}
	All of the following are contained in $\mathcal{B}(\mathbb{R})$:
	\begin{enumerate}
		\item All closed sets.
		\item All open sets.
		\item All $F_\sigma$ and $G_\delta$ sets.
	\end{enumerate}
\end{prop}
Recall that an $F_\sigma$ set is a set which can be written as countable union of closed sets. Similarly, $G_\delta$ as countable intersection of open sets.

\begin{prop}
	$\mathcal{B}(\mathbb{R})$ is generated by any of the following collections.
	\begin{enumerate}
		\item $\{(a, b) \mid a < b\}$ or $\{[a, b] \mid a < b\},$
		\item $\{(a, b] \mid a < b\}$ or $\{[a, b) \mid a < b\},$
		\item $\{(a, \infty) \mid a \in \mathbb{R}\}$ or $\{(-\infty, b) \mid b \in \mathbb{R}\},$
		\item $\{[a, \infty) \mid a \in \mathbb{R}\}$ or $\{(-\infty, b] \mid b \in \mathbb{R}\}.$
	\end{enumerate}
\end{prop}

\begin{defn}[Product of $\sigma$-algebras]
	Let $(X_{\alpha})_{\alpha \in A}$ be an indexed collection of nonempty sets, $X \vcentcolon= \prod_{\alpha \in A} X_{\alpha}$, and $\pi_{\alpha} : X \to X_{\alpha}$ the projection (coordinate) maps. If $\mathcal{M}_{\alpha}$ is a $\sigma$-algebra on $X_{\alpha}$ (for all $\alpha$), the \deff{product $\sigma$-algebra} on $X$ is the $\sigma$-algebra generated by
	\begin{equation*} 
		\{\pi_{\alpha}^{-1}(E_{\alpha}) : E_{\alpha} \in \mathcal{M}_{\alpha}, \alpha \in A\}.
	\end{equation*}
	This $\sigma$-algebra is denoted by $\bigotimes_{\alpha \in A} \mathcal{M}_{\alpha}$. 

	If $A = \{1, \ldots, n\}$, we also write this as $\bigotimes_{j = 1}^{n} \mathcal{M}_{j}$ or $\mathcal{M}_{1} \otimes \cdots \otimes \mathcal{M}_{n}$.
\end{defn}

\begin{prop}
	If $A$ is countable, then $\bigotimes_{\alpha} \mathcal{M}_{\alpha}$ is the $\sigma$-algebra generated by $\{\prod_{\alpha} E_{\alpha} : E_{\alpha} \in \mathcal{M}_{\alpha}\}$.
\end{prop}

\begin{prop}
	For each $\alpha \in A$, let $\mathcal{E}_{\alpha}$ be a generating set for $\mathcal{M}_{\alpha}$. Then, $\mathcal{M}_{\alpha}$ is generated by $\mathcal{F}_{1} \vcentcolon= \{\pi_{\alpha}^{-1}(E_{\alpha}) : E_{\alpha} \in \mathcal{E}_{\alpha}, \alpha \in A\}$. \newline
	Furthermore, if $A$ is countable and \underline{$X_{\alpha} \in \mathcal{E}_{\alpha}$} for all $\alpha$, then $\bigotimes_{\alpha} \mathcal{M}_{\alpha}$ is also generated by $\mathcal{F}_{2} \vcentcolon= \{\prod_{\alpha} E_{\alpha} : E_{\alpha} \in \mathcal{E}_{\alpha}\}$.
\end{prop}

With the above, we get two (possibly different) $\sigma$-algebras on $\mathbb{R}^n.$ One is the Borel $\sigma$-algebra on it, by virtue of it being a topological space, i.e., $\mathcal{B}(\mathbb{R}^n)$ and the other is the product of $\sigma$-algebra, i.e., $\prod_{i = 1}^n \mathcal{B}(\mathbb{R}).$ As it turns out, both are equal.

\begin{thm}
	Let $X_{1}, \ldots, X_{n}$ be metric spaces, and let $X \vcentcolon= \prod_{j} X_{j}$ be equipped with the product metric. Then, $\bigotimes_{j} \mathcal{B}(X_{j}) \subset \mathcal{B}(X)$. Furthermore, if each $X_{j}$ is separable, then $\bigotimes_{j} \mathcal{B}(X_{j}) = \mathcal{B}(X)$.

	In particular, $\mathcal{B}(\mathbb{R}^n) = \bigotimes_{i = 1}^{n}\mathcal{B}(\mathbb{R}).$
\end{thm}

\begin{defn}
	An \deff{elementary family} is a collection $\mathcal{E}$ of subsets of $X$ such that 
	\begin{enumerate}
		\item $\emptyset \in \mathcal{E}$,
		\item if $E, F \in \mathcal{E}$, then $E \cap F \in \mathcal{E}$,
		\item if $E \in \mathcal{E}$, then $E^{c}$ is a finite disjoint union of members of $\mathcal{E}$.
	\end{enumerate}
\end{defn}

\begin{prop} \label{prop:elementary-family-gives-algebra}
	If $\mathcal{E}$ is an elementary family, the collection $\mathcal{A}$ of finite disjoint union of members of $\mathcal{E}$ is an algebra.
\end{prop}

\subsection{Measures}

\begin{defn}[Measure]
	Suppose $X$ is a non-empty set and $\mathcal{M}$ a $\sigma$-algebra on $X.$ A \deff{measure} on $X$ is a map 
	\begin{equation*} 
		\mu : X \to [0, \infty]
	\end{equation*}
	satisfying
	\begin{enumerate}
		\item $\mu(\emptyset) = 0,$
		\item (\deff{countable additivity}) if $\{E_i\}_{1}^\infty \subset \mathcal{M}$ are pairwise disjoint, then
		\begin{equation*} 
			\mu\left(\bigsqcup_{i = 1}^\infty E_i\right) = \sum_{i = 1}^{\infty} \mu(E_i).
		\end{equation*}
	\end{enumerate}
	$(X, \mathcal{M}, \mu)$ is called a \deff{measure space}.
\end{defn}
Note that $\mu\left(\bigsqcup E_i\right)$ makes sense because $\mathcal{M}$ is a $\sigma$-algebra and hence $\bigsqcup E_i \in \mathcal{M}.$

\begin{rem}
	Countable additivity implies \deff{finite additivity}: If $E_{1}, \ldots, E_{n}$ are disjoint sets in $\mathcal{M}$, then $\mu\left(\bigcup_{j} E_{j}\right) = \sum_{j} \mu(E_{j})$. 

	If $\mu$ satisfies $\mu(\emptyset) = 0$ and finite additivity, then $\mu$ is called a \deff{finitely additive measure}. (Note that this $\mu$ need not be a measure.)
\end{rem}

\begin{defn}
	If $X$ is a set and $\mathcal{M} \subset \mathcal{P}(X)$ a $\sigma$-algebra, then $(X, \mathcal{M})$ is called a \deff{measurable space} and sets in $\mathcal{M}$ are called \deff{measurable sets}. If $\mu$ is a measure on $(X, \mathcal{M})$, then $(X, \mathcal{M}, \mu)$ is called a \deff{measure space}.
\end{defn}

\begin{defn}
	Let $(X, \mathcal{M}, \mu)$ be a measure space.
	\begin{enumerate}
		\item If $\mu(X) < \infty,$ then $\mu$ is called \deff{finite}.
		\item If $X = \bigcup_{j = 1}^\infty E_j,$ where $E_j \in \mathcal{M}$ and $\mu(E_j) < \infty$ for all $j,$ then $\mu$ is called \deff{$\sigma$-finite}.
		\item If for each $E \in \mathcal{M}$ with $\mu(E) = \infty,$ there exists $F \in \mathcal{M}$ with $F \subset E$ and $0 < \mu(F) < \infty,$ then $\mu$ is called \deff{semifinite}.
	\end{enumerate}
\end{defn}

\begin{exe}
	Every $\sigma$-finite measure is semifinite, but the converse is not true.
\end{exe}

\begin{prop}
	Suppose $(X, \mathcal{M}, \mu)$ is a measure space. All sets mentioned below are in $\mathcal{M}.$ Then,
	\begin{enumerate}
		\item (\deff{Monotonicity}) $E \subset F \implies \mu(E) \le \mu(F),$
		\item (\deff{Subadditivity}) $\mu\left(\bigcup_{1}^\infty E_i\right) \le \sum_{1}^{\infty} \mu(E_i),$
		\item (\deff{Continuity from below}) If $E_i \uparrow$ (i.e., $E_1 \subset E_2 \subset \cdots$), then
		\begin{equation*} 
			\mu\left(\bigcup_{i = 1}^\infty E_i\right) = \lim_{n\to \infty} \mu(E_i),
		\end{equation*}
		\item (\deff{Continuity from above}) If $E_i \downarrow$ (i.e., $E_1 \supset E_2 \supset \cdots$), and $\mu(E_{i}) < \infty$ for some $i$, then
		\begin{equation*} 
			\mu\left(\bigcap_{i = 1}^\infty E_i\right) = \lim_{n\to \infty} \mu(E_i).
		\end{equation*}
	\end{enumerate}
\end{prop}

\begin{defn}
	If $(X, \mathcal{M}, \mu)$ is a measure set, a set $E \in \mathcal{M}$ such that $\mu(E) = 0$ is called a \deff{null set}. If a statement about points $x \in X$ is true except for $x$ in some null set, we say that it is true \deff{almost everywhere (a.e.)}, or for \deff{almost every $x$}.

	A measure whose domain includes all subsets of null sets is said to be \deff{complete}.
\end{defn}

\begin{defn}[Completion]
	Given a measure space $(X, \mathcal{M}, \mu),$ the \deff{completion} of $\mathcal{M}$ with respect to $\mu$, denoted $\overline{\mathcal{M}}$, is the collection of all subsets of the form $E \cup N$ where $E \in \mathcal{M}$ and $N$ is a subset of a null set. 
\end{defn}
Note that the set $N$ above itself need not be in $\mathcal{M}$.

Clearly, $\mathcal{M} \subset \overline{\mathcal{M}}$ since $\emptyset$ is a null set.

\begin{prop}[Extension to completion]
	Let $(X, \mathcal{M}, \mu)$ be a measure space.
	\begin{enumerate}
		\item $\overline{\mathcal{M}}$ is a $\sigma$-algebra.
		\item There is a unique measure
		\begin{equation*} 
			\overline{\mu} : \overline{\mathcal{M}} \to [0, \infty]
		\end{equation*}
		such that $\overline{\mu}|\mathcal{M} = \mu.$
	\end{enumerate}
	$\overline{\mu}$ is called the \deff{completion} of $\mu$.
\end{prop}

\subsection{Outer Measures}

\begin{defn}[Outer measure]
	An \deff{outer measure} on a nonempty set $X$ is a map
	\begin{equation*} 
		\mu^* : \mathcal{P}(X) \to [0, \infty]
	\end{equation*}
	satisfying
	\begin{enumerate}
		\item $\mu^*(\emptyset) = 0,$
		\item $A \subset B \implies \mu^*(A) \le \mu^*(B),$
		\item $\mu\left(\bigcup_{i = 1}^\infty E_i\right) \le \sum_{i = 1}^{\infty} \mu(E_i).$
	\end{enumerate}
\end{defn}
Note that we don't demand equality even if disjoint.

\begin{prop}[A construction of an outer measure] \label{prop:constructoutermeasure}
	Suppose $\mathcal{F} \subset \mathcal{P}(X)$ and $\rho : \mathcal{F} \to [0, \infty]$ is a map such that
	\begin{enumerate}
		\item $\emptyset, X \in \mathcal{F},$
		\item $\rho(\emptyset) = 0.$
	\end{enumerate}
	For $E \in \mathcal{P}(X),$ define
	\begin{equation*} 
		\mu^*(E) \vcentcolon= \inf\left\{\sum_{i = 1}^{\infty}\rho(E_i) : E_i \in \mathcal{F},\; E \subset \bigcup_{i = 1}^\infty E_i\right\}.
	\end{equation*}
	Then, $\mu^*$ is an outer measure.
\end{prop}
Note that the above had just the bare minimum requirement for both $\rho$ and $\mathcal{F}$ and still gave us that $\mu^*$ is an outer measure.

\begin{defn}[$\mu^*$-measurable]
	Given an outer measure $\mu^*$ on a set $X,$ a set $A \subset X$ is said to be \deff{$\mu^*$-measurable} if for all $E \subset X,$ we have
	\begin{equation*} 
		\mu^*(E) = \mu^*(E \cap A) + \mu^*(E \cap A^c).
	\end{equation*}
\end{defn}
Note that $\mu^*(E) \le \mu^*(E \cap A) + \mu^*(E \cap A^c)$ holds for all $A$ and $E$, just by virtue of $\mu^{\ast}$ be an outer measure. Moreover, the reverse inequality also holds trivially if $E = \infty$. Thus, $A$ is $\mu^{\ast}$-measurable iff
\begin{equation*} 
	\mu^{\ast}(E) \ge \mu^*(E \cap A) + \mu^*(E \cap A^c) \text{ for all $E \subset X$ such that $\mu^{\ast}(E) < \infty$.}
\end{equation*}

\begin{thm}[Carathéodory]
	Let $\mu^*$ be an outer measure on $X.$ Let
	\begin{equation*} 
		\mathcal{M} \vcentcolon= \{A \subset X : A \text{ is } \mu^*\text{-measurable}\}.
	\end{equation*}
	Then,
	\begin{enumerate}
		\item $\mathcal{M}$ is a $\sigma$-algebra.
		\item $\mu^*|\mathcal{M}$ is a complete measure.
	\end{enumerate} 
\end{thm}

\begin{defn}[Premeasure]
	Suppose $\mathcal{F}$ is an algebra on $X.$ A map
	\begin{equation*} 
		\mu_0 : \mathcal{F} \to [0, \infty]
	\end{equation*}
	is called a \deff{premeasure} if
	\begin{enumerate}
		\item $\mu_0(\emptyset) = 0,$
		\item if $\{A_i\}_{i = 1}^\infty \subset \mathcal{F}$ are pairwise disjoint such that $\bigsqcup_{i = 1}^\infty A_i \in \mathcal{F},$ then
		\begin{equation*} 
			\mu_0\left(\bigsqcup_{i = 1}^\infty A_i\right) = \sum_{i = 1}^{\infty} \mu_0(A_i).
		\end{equation*}
	\end{enumerate}
\end{defn}
Note that by putting all but finitely many $A_i = \emptyset,$ the above equality holds for finite unions as well. (The finite union \emph{will} be in $\mathcal{F}$ since it's an algebra.)

\begin{prop}
	Suppose $\mu_0$ is a premeasure on an algebra $\mathcal{F}.$ Then, if $\mu^*$ is the outer measure as defined in \Cref{prop:constructoutermeasure} (with $\rho = \mu_0$), then
	\begin{enumerate}
		\item $\mu^*|\mathcal{F} = \mu_0,$
		\item every set in $\mathcal{F}$ is $\mu^*$-measurable.
	\end{enumerate}
\end{prop}

\begin{thm}
	Suppose $\mathcal{F} \subset \mathcal{P}(X)$ is an algebra and let $\mathcal{M}$ be the $\sigma$-algebra generated by $\mathcal{F}.$

	Let $\mu_0$ be a premeasure defined on $\mathcal{F}$ and let $\mu^*$ be the outer measure as before. Then
	\begin{enumerate}
		\item $\mu^*|\mathcal{M}$ is a measure on $(X, \mathcal{M}).$ Put $\mu = \mu^*|\mathcal{M}$ for the next part.
		\item If $\nu$ is any measure extending $\mu_0,$ then $\nu \le \mu$, and
		\begin{equation*} 
			\nu(E) = \mu(E)
		\end{equation*}
		whenever $\mu(E) < \infty.$ 
		\item If $\mu_{0}$ is $\sigma$-finite, then $\mu$ is the unique extension of $\mu_{0}$ to a measure on $\mathcal{M}$.
	\end{enumerate}
\end{thm}

\subsection{Borel measures on the real line}

\begin{defn}
	A \deff{half-interval} is a subset of $\mathbb{R}$ of one of the following forms:
	\begin{enumerate}
		\item $(a, b]$ for $-\infty \le a < b < \infty,$ 
		\item $(a, \infty)$ for $-\infty \le a < \infty,$
		\item $\emptyset.$
	\end{enumerate}
\end{defn}

\begin{prop}
	The collection $\mathcal{F}$ of all finite disjoint unions of half-intervals is an algebra on $\mathbb{R}$. The $\sigma$-algebra generated by $\mathcal{F}$ is $\mathcal{B}(\mathbb{R})$.
\end{prop}

\begin{prop} 
	Let $\mathcal{F}$ be the algebra consisting of finite unions of half-intervals. Let $F : \mathbb{R} \to \mathbb{R}$ be an increasing an right-continuous function. Define
	\begin{equation*} 
		\mu_0\left(\bigsqcup_{i = 1}^n (a_j, b_j]\right) \vcentcolon= \sum_{i = 1}^{n} [F(b_j) - F(a_j)],
	\end{equation*}
	and let $\mu_0(\emptyset) = 0.$
	
	Then, $\mu_0$ is a well-defined premeasure on $\mathcal{F}.$
\end{prop}

\begin{defn}[Borel measure]
	A measure $\mu$ on $(\mathbb{R}, \mathcal{B}(\mathbb{R}))$ is called a \deff{Borel measure} on $\mathbb{R}.$
\end{defn}

\begin{thm}
	If $F : \mathbb{R} \to \mathbb{R}$ is any increasing, right-continuous function, there is a unique Borel measure $\mu_F$ on $\mathbb{R}$ such that $\mu_F((a, b]) = F(b) - F(a)$ for all $a, b.$ If $G$ is another such function, we have $\mu_F = \mu_G$ iff $F - G$ is constant. Conversely, if $\mu$ is a Borel measure on $\mathbb{R}$ that is finite on all bounded Borel sets and we define
	\begin{equation*} 
		F(x) \vcentcolon= \begin{cases}
			\mu\left((0, x]\right) & x > 0,\\
			0 & x = 0,\\
			-\mu\left((-x, 0]\right) & x < 0,
		\end{cases}
	\end{equation*}
	then $F$ is increasing and right-continuous, and $\mu = \mu_F.$
\end{thm}

Let $F$ be an increasing and right-continuous function on $\mathbb{R}$. The earlier theory gives us not only a Borel measure $\mu_{F}$ but also a complete measure $\overline{\mu_{F}}$, which is the completion of $\mu_{F}$. We shall usually denote the completion also by $\mu_{F}$; it is called the \deff{Lebesgue-Stieltjes measure}associated to $F$.

\begin{prop}
	Let $\mu$ be a complete Lebesgue-Stieltjes measure on $\mathbb{R}$, associated to an increasing right-continuous function $F$. Let $\mathcal{M}_{\mu}$ denote the domain of $\mu$.

 	For any $E \in \mathcal{M}_{\mu}$, we have
 	\begin{align*} 
 		\mu(E) &= \inf\left\{\sum_{j = 1}^{\infty} \mu((a_{j}, b_{j}]) : E \subset \bigcup_{j} (a_{j}, b_{j}]\right\} \\
 		&= \inf\left\{\sum_{j = 1}^{\infty} \mu((a_{j}, b_{j})) : E \subset \bigcup_{j} (a_{j}, b_{j})\right\} \\
 		&= \inf\{\mu(U) : U \supset E \text{ and $U$ is open}\} \\
 		&= \sup\{\mu(K) : K \subset E \text{ and $K$ is compact}\}.
 	\end{align*}

 	If $\mu(E) < \infty$, then for every $\epsilon > 0$ there is a set $I \subset \mathbb{R}$ that is a finite union of open intervals such that $\mu(E \Delta I) < \epsilon$.

 	If $A \subset \mathbb{R}$, the following are equivalent.
 	\begin{enumerate}
 		\item $A \in M_{\mu}$.
 		\item $A = V \setminus N_{1}$ where $V$ is a $G_{\delta}$ set and $\mu(N_{1}) = 0$.
 		\item $A = H \cup N_{2}$ where $H$ is an $F_{\sigma}$ set and $\mu(N_{2}) = 0$.
 	\end{enumerate}
\end{prop} 
Recall that a $G_{\delta}$ set is a countable intersection of open sets, and an $F_{\sigma}$ set is a countable union of closed sets.

Now, consider $F$ to be the identity function. The associated (complete) measure is denoted by $m$ and called the \deff{Lebesgue measure}. The domain of $m$ is denoted by $\mathcal{L}$ and called the class of \deff{Lebesgue measurable} sets.

\begin{prop}[Invariance of Lebesgue measure]
	If $E \in \mathcal{L}$, then $E + s \in \mathcal{L}$ and $r E \in \mathcal{L}$ for all $s, r \in \mathbb{R}$. Moreover, $m(E + s) = m(E)$ and $m(rE) = \md{r} m(E)$.
\end{prop}

\section{Integration}

\subsection{Measurable functions}

\begin{defn}
	Let $(X, \mathcal{M})$ and $(Y, \mathcal{N})$ be measurable spaces. A function $f : X \to Y$ between is called \deff{$(\mathcal{M}, \mathcal{N})$-measurable} if $f^{-1}(E) \in \mathcal{M}$ for all $E \in \mathcal{N}$.
\end{defn}

\begin{prop}
	Composition of measurable functions is measurable. If the measure on $Y$ is generated by $\mathcal{E}$, then it suffices to check that $f^{-1}(E)$ is measurable for all $E \in \mathcal{E}$. Consequently, if $X$ and $Y$ are topological spaces with the Borel measure, then continuous functions are measurable.
\end{prop}

If $(X, \mathcal{M})$ is a measurable space, a real-valued (resp. complex-valued) function on $X$ will be called \deff{$\mathcal{M}$-measurable}, or simply \deff{measurable}, if $f$ is $(\mathcal{M}, \mathcal{B}(\mathbb{R}))$ (resp. $(\mathcal{M}, \mathcal{B}(\mathbb{C}))$) measurable. In particular, $f : \mathbb{R} \to \mathbb{R}$ is \deff{Lebesgue} (resp. \deff{Borel}) \deff{measurable} if $f$ is $(\mathcal{L}, \mathcal{B}(\mathbb{R}))$ (resp. $(\mathcal{B}(\mathbb{R}), \mathcal{B}(\mathbb{R}))$) measurable; likewise for $f : \mathbb{R} \to \mathbb{C}$. \newline
Note that if $f, g : \mathbb{R} \to \mathbb{R}$ are Lebesgue measurable, it is not necessary that $f \circ g$ is Lebesgue measurable. This is because $f$ and $g$ are $(\mathcal{L}, \mathcal{B}(\mathbb{R}))$-measurable and not (necessarily) $(\mathcal{L}, \mathcal{L})$.

\begin{cor}
	If $(X, \mathcal{M})$ is a measurable space and $f : X \to \mathbb{R}$, $f$ being measurable is equivalent to any of the following:
	\begin{enumerate}
		\item $f^{-1}((a, \infty)) \in \mathcal{M}$ for all $a \in \mathbb{R}$,
		\item $f^{-1}([a, \infty)) \in \mathcal{M}$ for all $a \in \mathbb{R}$,
		\item $f^{-1}((-\infty, a)) \in \mathcal{M}$ for all $a \in \mathbb{R}$,
		\item $f^{-1}((-\infty, a]) \in \mathcal{M}$ for all $a \in \mathbb{R}$,
	\end{enumerate}
\end{cor}

\begin{prop}[Universal property of products]
	Let $X$ and $(Y_{\alpha})_{\alpha}$ be measurable spaces. Put $Y \vcentcolon= \prod_{\alpha} Y$ and give $Y$ the product $\sigma$-algebra. Let $\pi_{\alpha} : Y \to Y_{\alpha}$ denote the projection maps.

	Then, $f : X \to Y$ is measurable iff $f \circ \pi_{\alpha}$ is measurable for all $\alpha$. Moreover, each $\pi_{\alpha}$ is measurable.
\end{prop}
\begin{cor}
	A function $X \to \mathbb{C}$ is measurable iff its real and imaginary parts are measurable functions $X \to \mathbb{R}$.
\end{cor}

Recall the extended real line $\overline{\mathbb{R}} = [-\infty, \infty]$ is a metrisable topological space. We may talk about measurable functions $X \to \overline{\mathbb{R}}$ by giving $\overline{\mathbb{R}}$ the Borel measure. Explicitly, this is given by $\mathcal{B}(\overline{\mathbb{R}}) = \{E \subset \overline{\mathbb{R}} : E \cap \mathbb{R} \in \mathcal{B}(\mathbb{R})\}$.

\begin{prop}
	If $f, g : X \to \mathbb{C}$ are measurable, so are $f + g$ and $fg$.

	If $(f_{n})_{n \ge 1}$ is a sequence of $\overline{\mathbb{R}}$-valued measurable functions, then the functions 
	\begin{align*} 
		g_{1}(x) \vcentcolon= \sup_{n} f_{n}(x), \quad g_{2}(x) \vcentcolon= \inf_{n} f_{n}(x), \\
		g_{3}(x) \vcentcolon= \limsup_{n \to \infty} f_{n}(x), \quad g_{4}(x) \vcentcolon= \liminf_{n \to \infty} f_{n}(x)
	\end{align*}
	are all measurable. 

	If $f(x) = \lim_{n \to \infty} f_{n}(x)$ exists for all $x$, then $f$ is measurable. This is also true if $f_{n}$ were complex measurable functions.

	If $f, g : X \to \overline{\mathbb{R}}$ are measurable, then so are $\min(f, g)$ and $\max(f, g)$.
\end{prop}

\begin{defn}
	If $f : X \to \mathbb{R}$, we define the \deff{positive} and \deff{negative parts} of $f$ to be
	\begin{equation*} 
		f^{+}(x) \vcentcolon= \max(f(x), 0) \andd f^{-}(x) \vcentcolon= \max(-f(x), 0).
	\end{equation*}
\end{defn}

Note that $f^{+}$ and $f^{-}$ are nonnegative valued measurable functions. Moreover, $f = f^{+} - f^{-}$. 

If $f$ is complex-valued, we have the \deff{polar decomposition} as
\begin{equation*} 
	f = (\sign(f)) \md{f},
\end{equation*}
where $\sign(z) = z/\md{z}$ for $z \neq 0$ and $\sign(0) = 0$. 

Let $(X, \mathcal{M})$ be a measurable space as usual. If $E \subset X$, the \deff{characteristic} (or \deff{indicator}) \deff{function} $\chi_{E}$ is defined on $X$ by
\begin{equation*} 
	\chi_{E}(x) \vcentcolon= 
	\begin{cases}
		1 & x \in E, \\
		0 & x \notin E.
	\end{cases}
\end{equation*}

Note that $\chi_{E}$ is measurable iff $E$ is measurable. A \deff{simple function} on $X$ is a finite linear combination, with complex coefficients, of characteristic functions of sets in $\mathcal{M}$. Equivalently, $f : X \to \mathbb{C}$ is simple iff $f$ is measurable and the image of $f$ is a finite subset of $\mathbb{C}$. Explicitly, we have
\begin{equation*} 
	f = \sum_{j = 1}^{n} z_{j} \chi_{E_{j}},
\end{equation*}
where $\im(f) = \{z_{1}, \ldots, z_{n}\}$ and $E_{j} = f^{-1}(\{z_{j}\})$. This is called the \deff{standard representation} of $f$. It exhibits $f$ as a linear combination, with distinct coefficients, of characteristic functions of disjoint sets whose union is $X$. (It is possible that one $z_{j}$ is $0$ but we still consider it as a part of the representation.)

\begin{exe}
	If $f$ and $g$ are simple, then so are $f + g$ and $fg$.
\end{exe}

\begin{thm}
	Let $(X, \mathcal{M})$ be a measurable space.
	\begin{enumerate}
		\item If $f : X \to [0, \infty]$ is measurable, then there is a sequence $(\phi_{n})_{n \ge 1}$ of real-valued simple functions such that $0 \le \phi_{1} \le \phi_{2} \le \cdots \le f$, $\phi_{n} \to f$ pointwise, and $\phi_{n} \to f$ uniformly on any set on which $f$ is bounded.
		%
		\item If $f : X \to \mathbb{C}$ is measurable, then there is a sequence $(\phi_{n})_{n \ge 1}$ of simple functions such that $0 \le \md{\phi_{1}} \le \md{\phi_{2}} \le \cdots \le \md{f}$, $\phi_{n} \to f$ pointwise, and $\phi_{n} \to f$ uniformly on any set on which $f$ is bounded.
	\end{enumerate}
\end{thm}

\begin{prop}
	Let $(X, \mathcal{M}, \mu)$ be a measure space. The following implications are valid iff $\mu$ is complete:
	\begin{enumerate}
		\item If $f$ is measurable and $f = g$ $\mu$ a.e., then $g$ is measurable.
		\item If $f_{n}$ is measurable for $n \in \mathbb{N}$ and $f_{n} \to f$ $\mu$ a.e., then $f$ is measurable.
	\end{enumerate}
\end{prop}

\begin{prop}
	Let $(X, \mathcal{M}, \mu)$ be a measure space, and let $(X, \overline{\mathcal{M}}, \overline{\mu})$ be its completion. If $f$ is an $\overline{\mathcal{M}}$-measurable function on $X$, then there is an $\mathcal{M}$-measurable function $g$ such that $f = g$ $\bar{\mu}$ almost everywhere.
\end{prop}

\subsection{Integration of nonnegative functions}

In this subsection, we fix a measure space $(X, \mathcal{M}, \mu)$. We define
\begin{equation*} 
	\mathcal{L}^{+} \vcentcolon= \text{the space of all measurable functions from $X$ to $[0, \infty]$.}
\end{equation*}
The above may be denoted by $L^{+}(X)$ or $L^{+}(\mu)$ or $L^{+}(X, \mu)$.

If $\phi \in L^{+}$ is a simple function with \emph{standard representation} $\phi = \sum a_{j} \chi_{E_{j}}$, then we define the \deff{integral} of $\phi$ with respect to $\mu$ by
\begin{equation*} 
	\int_{X} \phi \,{\mathrm{d}}\mu \vcentcolon= \sum_{j = 1}^{n} a_{j} \mu(E_{j}).
\end{equation*}

Note that there is no question of ``well-defined-ness'' since there is a unique standard representation. We make the convention $0 \cdot \infty = 0$. The usual conventions of alternate notations apply. Some are shown below.
\begin{equation*} 
	\int_{A} \phi \,{\mathrm{d}}\mu = \int_{A} \phi = \int_{A} \phi(x) \,{\mathrm d}\mu = \int_{X} \phi \chi_{A} \,{\mathrm d}\mu, \quad \int = \int_{X}.
\end{equation*}
(In the above, $A$ is any measurable subset of $X$. Note that $\phi \chi_{A}$ is a simple function on $A$.)

\begin{prop}
	Let $\phi$ and $\psi$ be simple function in $L^{+}$.
	\begin{enumerate}
		\item If $c \ge 0$, then $\int c \phi = c \int \phi$.
		\item $\int(\phi + \psi) = \int \phi + \int \psi$.
		\item If $\phi \le \psi$, then $\int \phi \le \int \psi$.
		\item $A \mapsto \int_{A} {\mathrm d}\mu$ is a measure on $\mathcal{M}$.
	\end{enumerate}
\end{prop}

Extend the definition of $\int$ to all $f \in L^{+}$ by
\begin{equation*} 
	\int f \,{\mathrm d}\mu \vcentcolon= \sup \left\{\int \phi \,{\mathrm d}\mu : 0 \le \phi \le f,\, \phi \text{ simple}\right\}.
\end{equation*}
By the previous proposition, the above agrees with the earlier definition when $f$ is simple. The definition quickly also implies
\begin{equation*} 
	\int f \le \int g \text{ whenever } f \le g, \andd \int cf = c \int f \text{ for all } c \in [0, \infty].
\end{equation*}

\begin{thm}[Monotone Convergence Theorem]
	If $(f_{n})_{n}$ is a sequence in $L^{+}$ such that $f_{j} \le f_{j + 1}$ for all $j$, and $f = \lim_{n} f_{n} = \sup_{n} f_{n}$, then $\int f = \lim_{n} \int f_{n}$.
\end{thm}
\begin{cor}
	If $(\phi_{n})_{n}$ is a sequence of simple $L^{+}$ functions increasing to $f$, then $\int f = \lim_{n} \int \phi_{n}$.

	If $(f_{n})_{n}$ is a finite or infinite sequence in $L^{+}$ and $f = \sum_{n} f_{n}$, then $\int f = \sum_{n} \int f_{n}$.
\end{cor}

\begin{prop}
	If $f \in L^{+}$, then $\int f = 0$ iff $f = 0$ a.e.

	If $(f_{n})_{n}$ is a sequence in $L^{+}$, $f \in L^{+}$, and $f_{n}(x)$ increases to $f(x)$ for a.e. $x$, then $\int f = \lim_{n} \int f_{n}$.
\end{prop}

\begin{thm}[Fatou's Lemma]
	If $(f_{n})_{n}$ is any sequence in $L^{+}$, then
	\begin{equation*} 
		\int (\liminf f_{n}) \le \liminf \int f_{n}.
	\end{equation*}
\end{thm}

\begin{ex}
	Consider $f_{n} = \chi_{(n, n + 1)}$ or $f_{n} = n \chi_{(0, 1/n)}$ to see that the inequality can be strict. Note that in either case $\int f_{n} = 1$ for all $n$ but $f_{n} \to 0$ pointwise.
\end{ex}

\begin{cor}
	If $(f_{n})_{n}$ is a sequence in $L^{+}$, $f \in L^{+}$, and $f_{n} \to f$ a.e., then $\int f \le \liminf \int f_{n}$.
\end{cor}

\begin{prop}
	If $f \in L^{+}$ and $\int f < \infty$, then $\{x : f(x) = \infty\}$ is a null set and $\{x : f(x) > 0\}$ is $\sigma$-finite.
\end{prop}

\subsection{Integration of complex functions}

We continue to work on a fixed measure space $(X, \mathcal{M}, \mu)$. Let $f : X \to \mathbb{R}$ be measurable. Note that then $f^{+}$, $f^{-}$, and $\md{f}$ are all in $L^{+}$. (In fact, $\md{f} = f^{+} + f^{-}$.) Thus, it makes sense to talk about their integrals. \newline
$f$ is said to be \deff{integrable} if either (and hence both) of the two equivalent conditions hold:
\begin{enumerate}
	\item $\int f^{+}$ and $\int f^{-}$ are finite.
	\item $\int \md{f}$ is finite.
\end{enumerate}

In this case, we define
\begin{equation*} 
	\int f \vcentcolon= \int f^{+} - \int f^{-}.
\end{equation*}

\begin{prop}
	The set of all integrable real-valued functions on $X$ is a real vector space, and the integral is a linear functional on it.
\end{prop}

Now, if $f$ is a complex-valued measurable function, we say that $f$ is \deff{integrable} if $\int \md{f} < \infty$. More generally, if $E \in \mathcal{M}$, $f$ is \deff{integrable on $E$} if $\int_{E} \md{f} < \infty$. Check that $f$ is integrable iff its real and imaginary parts are so. In this case, we define
\begin{equation*} 
	\int f \vcentcolon= \int \Re(f) + \iota \int \Im(f).
\end{equation*}
It follows that the space of complex-valued functions on $X$ is a complex vector space, and the integral is a linear functional on it. This space is denoted by $L^{1}$ (or $L^{1}(\mu)$ or ...). 

\begin{prop}
	If $f \in L^{1}$, then 
	\begin{enumerate}
	\item $\md{\int f} \le \int \md{f}$,
	\item $\{x : f(x) \neq 0\}$ is $\sigma$-finite,
	\item if $g \in L^{1}$, then $\int_{E} f = \int_{E} g$ for all $E \in \mathcal{M}$ iff $\int \md{f - g} = 0$ iff $f = g$ a.e.
	\end{enumerate} 
\end{prop}

The above tells us that it makes no difference if we alter functions on null sets. In this fashion, we can treat $\overline{\mathbb{R}}$-valued functions that are finite a.e. as real-valued functions for the purposes of integration.

We find it useful to treat $L^{1}(\mu)$ to be the set of \emph{equivalence classes} of integrable functions modulo the relation $f \sim g$ if $f = g$ a.e. This new $L^{1}$ continues to be a complex vector space. Moreover, $L^{1}$ now becomes a metric space with metric $\rho(f, g) \vcentcolon= \int \md{f - g}$.

\begin{thm}[Dominated convergence theorem]
	Let $(f_{n})_{n}$ be a sequence in $L^{1}$ such that
	\begin{enumerate}
		\item $f_{n} \to f$ a.e., and
		\item there exists $g \in L^{1}$ such that $\md{f_{n}} \le g$ a.e. for all $n$.
	\end{enumerate}
	Then, $f \in L^{1}$ and $\int f = \lim_{n} \int f_{n}$.
\end{thm}
\begin{cor}
	Suppose that $(f_{n})_{n}$ is a sequence in $L^{1}$ such that $\sum \int \md{f_{n}} < \infty$. Then, $\sum f_{n}$ converges a.e. to a function $f \in L^{1}$ and $\int f = \sum \int f_{n}$.
\end{cor}

\begin{thm}
	If $f \in L^{1}(\mu)$ and $\epsilon > 0$, then there is an integrable simple function $\phi$ such that $\int \md{f - \phi} < \epsilon$. (Simple functions are dense.)

	If $\mu$ is a Lebesgue-Stieltjes measure on $\mathbb{R}$, the sets in the definition of $\phi = \sum a_{j} \chi_{E_{j}}$ can be taken to be finite unions of open intervals; moreover, there is a continuous function $g$ that vanishes outside a bounded interval such that $\int \md{f - g} < \epsilon$.
\end{thm}

\begin{thm}
	Suppose that $f : X \times [a, b] \to \mathbb{C}$ (here $-\infty < a < b < \infty$) and that $f(-, t) : X \to \mathbb{C}$ is integrable for each $t \in [a, b]$. Let $F(t) \vcentcolon= \int_{X} f(x, t) \,{\mathrm d}\mu(x)$ for $t \in [a, b]$.

	\begin{enumerate}
		\item Suppose that there exists $g \in L^{1}(\mu)$ such that $\md{f(x, t)} \le g(x)$ for all $x, t$. \newline
		If $\lim_{t \to t_{0}} f(x, t) = f(x, t_{0})$ for all $x \in X$, then $\lim_{t \to t_{0}} F(t) = F(t_{0})$; in particular, if $f(x, -)$ is continuous for every $x$, then $F$ is continuous. 
		%
		\item Suppose that $\partial f/\partial t$ exists and there is a $g \in L^{1}(\mu)$ such that $\md{(\partial f/\partial t)(x, t)} \le g(x)$ for all $x, t$. Then, $F$ is differentiable and $F'(t) = \int (\partial f/\partial t)(x, t) \,{\mathrm d}\mu(x)$.
	\end{enumerate}
\end{thm}

In the special case that $\mu$ is the Lebesgue measure on $\mathbb{R}$, the integral developed is called the \deff{Lebesgue integral}.

\begin{thm}
	Let $f$ be a bounded real-function on $[a, b]$.
	\begin{enumerate}
		\item $f$ is Riemann integrable iff $\{x \in [a, b] : \text{$f$ is discontinuous at $x$}\}$ has Lebesgue measure zero.
		\item If $f$ is Riemann integrable, then $f$ is the Lebesgue measurable, and $\int_{a}^{b} f = \int_{[a, b]} f \,{\mathrm d}m$.
	\end{enumerate}
\end{thm}

\subsection{Modes of convergence}

\begin{defn}
	Let $(X, \mathcal{M}, \mu)$ be a measure space. Let $(f_{n})_{n}$ be a sequence of complex-valued measurable functions on $X$, and $f : X \to \mathbb{C}$ be measurable.

	\begin{enumerate}
		\item $(f_{n})_{n}$ is \deff{Cauchy in measure} if for every $\epsilon > 0$,
		\begin{equation*} 
			\mu(\{x : \md{f_{n}(x) - f_{m}(x)} \ge \epsilon\}) \to 0 \quad \text{as $n, m \to \infty$}.
		\end{equation*}
		%
		\item $f_{n} \to f$ \deff{in measure} if for every $\epsilon > 0$,
		\begin{equation*} 
			\mu(\{x : \md{f_{n}(x) - f(x)} \ge \epsilon\}) \to 0 \quad \text{ as $n \to \infty$}.
		\end{equation*}
	\end{enumerate}
\end{defn}

We already know what it means for $f_{n}$ to converge pointwise, a.e., uniformly, and in $L^{1}$.

\begin{ex}
	Consider the following examples of sequences of measurable functions on $\mathbb{R}$:
	\begin{enumerate}[label=(\roman*)]
		\item $f_{n} = \frac{1}{n} \chi_{(0, n)}$.
		\item $f_{n} = \chi_{(n, n + 1)}$.
		\item $f_{n} = n \chi_{[0, 1/n]}$.
		\item $f_{1} = \chi_{[0, 1]}$, $f_{2} = \chi_{[0, 1/2]}$, $f_{3} = \chi_{[1/2, 1]}$, and in general, $f_{n} = \chi_{j/2^{k}, (j + 1)/2^{k}}$, where $n = 2^{k} + j$ with $0 \le j < 2^{k}$.
	\end{enumerate}

	In (i), (ii), and (iii), $f_{n} \to 0$ uniformly, pointwise, and a.e., respectively, but $f_{n} \not\to 0$ in $L^{1}$ since $\int \md{f_{n}} = 1$ for all $n$. 

	In (i) and (iii), $f_{n} \to 0$ in measure (but not in $L^{1}$).

	In (iv), $f_{n} \to 0$ in $L^{1}$, but $f_{n}(x)$ converges for no $x$.
\end{ex}

\begin{prop}
	If $f_{n} \to f$ a.e. and $\md{f_{n}} \le g \in L^{1}$, then $f_{n} \to f$ in $L^{1}$.

	Suppose $(f_{n})_{n}$ is Cauchy in measure. Then, there is a measurable function $f$ such that $f_{n} \to f$ in measure, and there is a subsequence $(f_{n_{j}})_{j}$ such that $f_{n_{j}} \to f$ a.e. Moreover, if also $f_{n} \to g$ in measure, then $g = f$ a.e.

	If $f_{n} \to f$ in $L^{1}$, then $f_{n} \to f$ in measure. Moreover, there is a subsequence $(f_{n_{j}})$ such that $f_{n_{j}} \to f$ a.e.
\end{prop}

\begin{thm}[Egoroff's theorem]
	Suppose $\mu(X) < \infty$, and $f, f_{1}, f_{2}, \ldots$ are complex-valued measurable functions on $X$ such that $f_{n} \to f$ a.e. \newline
	Then, for every $\epsilon > 0$, there exists $E \subset X$ such that $\mu(E) < \epsilon$ and $f_{n} \to f$ uniformly on $E^{c}$.
\end{thm}

\subsection{Product measures}

Let $(X, \mathcal{M}, \mu)$ and $(Y, \mathcal{N}, \nu)$ be measure spaces. We define a measure on the measurable space $(X \times Y, \mathcal{M} \otimes \mathcal{N})$. 

Define a \deff{rectangle} to be a set of the form $A \times B$ where $A \in \mathcal{M}$ and $B \in \mathcal{N}$. Check that the class of rectangles is closed under finite intersections and complements. Thus, the collection $\mathcal{A}$ of finite disjoint unions of rectangles is an algebra (\Cref{prop:elementary-family-gives-algebra}) and the $\sigma$-algebra it generates is $\mathcal{M} \otimes \mathcal{N}$ (by definition).

\begin{exe}
	If $A \times B = \bigsqcup_{j} A_{j} \times B_{j}$, then $\mu(A) \nu(B) = \sum_{j} \mu(A_{j}) \nu(B_{j})$. ($j$ may run over a finite or countably infinite set.)
\end{exe}

Thus, if $E \in \mathcal{A}$ is the disjoint union of rectangles $A_{1} \times B_{1}, \ldots, A_{n} \times B_{n}$, we may define
\begin{equation*} 
	\pi(E) \vcentcolon= \sum_{j = 1}^{n} \mu(A_{j}) \nu(B_{j}).
\end{equation*}
Moreover, $\pi$ is a premeasure on $\mathcal{A}$. By our earlier theory, $\pi$ generates an outer measure on $X \times Y$ whose restriction to $\mathcal{M} \otimes \mathcal{N}$ is a measure that extends $\pi$. We call this the \deff{product measure}. If $\mu$ and $\nu$ are $\sigma$-finite, then so is $\mu \times \nu$. In this case, $\mu \times \nu$ is the unique measure on $\mathcal{M} \otimes \mathcal{N}$ such that $(\mu \times \nu)(A \times B) = \mu(A) \nu(B)$ for all rectangles $A \times B$. 

The above (and the below) constructions (and results) can be extended to more factors but we work with only two.

If $E \subset X \times Y$, for $x \in X$ and $y \in Y$, we define the \deff{$x$-section} $E_{x}$ and the \deff{$y$-section} $E^{y}$ of $E$ by
\begin{equation*} 
	E_{x} = \{y \in Y : (x, y) \in E\} \subset Y \andd E^{y} \vcentcolon= \{x \in X : (x, y) \in E\} \subset X.
\end{equation*}
Similarly, if $f$ is a function on $X \times Y$, we define the \deff{$x$-section} $f_{x}$ and the \deff{$y$-section} $f^{y}$ of $f$ by
\begin{equation*} 
	f_{x}(y) \vcentcolon= f(x, y) =\vcentcolon f^{y}(x).
\end{equation*}

Note: $(\chi_{E})_{x} = \chi_{E_{x}}$ and $(\chi_{E})^{y} = \chi_{E^{y}}$.

\begin{prop}
	\phantom{hi}
	\begin{enumerate}
		\item If $E \in \mathcal{M} \otimes N$, then $E_{x} \in \mathcal{N}$ for all $x \in X$ and $E^{y} \in \mathcal{M}$ for all $y \in Y$.
		\item If $f$ is $\mathcal{M} \otimes N$-measurable, then $f_{x}$ is $\mathcal{N}$-measurable for all $x \in X$ and $f^{y}$ is $\mathcal{M}$-measurable for all $y \in Y$.
	\end{enumerate}
\end{prop}

\begin{thm}
	Suppose that $(X, \mathcal{M}, \mu)$ and $(Y, \mathcal{N}, \nu)$ are $\sigma$-finite measure spaces. If $E \in \mathcal{M} \otimes \mathcal{N}$, then the functions $x \mapsto \nu(E_{x})$ and $y \mapsto \mu(E^{y})$ are measurable on $X$ and $Y$, respectively, and
	\begin{equation*} 
		(\mu \times \nu)(E) = \int \nu(E_{x}) \,{\mathrm d}\mu(x) = \int \mu(E^{y}) \,{\mathrm d}\nu(y).
	\end{equation*}
\end{thm}

\begin{thm}[The Fubini-Tonelli theorem]
	Suppose that $(X, \mathcal{M}, \mu)$ and $(Y, \mathcal{N}, \nu)$ are $\sigma$-finite measure spaces.
	\begin{enumerate}
		\item (Tonelli) If $f \in L^{+}(X \times Y)$, then the functions $g(x) \vcentcolon= \int f_{x} \,{\mathrm d}\nu$ and $h(y) \vcentcolon= \int f^{y} \,{\mathrm d}\mu$ are in $L^{+}(X)$ and $L^{+}(Y)$ respectively, and
		\begin{equation} \label{eq:001}
			\begin{aligned}
				\int f \,{\mathrm{d}}(\mu \times \nu) &= \int \left[\int f(x, y) \,{\mathrm{d}}\nu(y)\right] \,{\mathrm{d}}\mu(x) \\
				&= \int \left[\int f(x, y) \,{\mathrm{d}}\mu(x)\right] \,{\mathrm{d}}\nu(y).
			\end{aligned}
		\end{equation}
		%
		\item (Fubini) If $f \in L^{1}(X \times Y)$, then $f_{x} \in L^{1}(\nu)$ for a.e. $x \in X$, $f^{y} \in L^{1}(\mu)$ for a.e. $y \in Y$. The a.e.-defined functions $g(x) \vcentcolon= \int f_{x} \,{\mathrm d}\nu$ and $h(y) \vcentcolon= \int f^{y} \,{\mathrm d}\mu$ are in $L^{1}(X)$ and $L^{1}(Y)$ respectively, and \Cref{eq:001} holds.
	\end{enumerate}
\end{thm}

The product measure is almost never complete, even if $\mu$ and $\nu$ are so.

\begin{thm}[The Fubini-Tonelli Theorem for Complete Measures]
	Let $(X, \mathcal{M}, \mu)$ and $(Y, \mathcal{N}, \nu)$ be complete $\sigma$-finite measure spaces, and let $(X \times Y, \mathcal{L}, \lambda)$ be the completion of $(X \times Y, \mathcal{M} \otimes \mathcal{N}, \mu \times \nu)$. 

	If $f$ is $\mathcal{L}$-measurable and either
	\begin{enumerate}[label=(\roman*)]
		\item $f \ge 0$, or 
		\item $f \in L^{1}(\lambda)$,
	\end{enumerate}
	then $f_{x}$ is $\mathcal{N}$-measurable for a.e. $x$ and $f^{y}$ is $\mathcal{M}$-measurable for a.e. $y$, and in case (ii), $f_{x}$ and $f^{y}$ are also integrable for a.e. $x$ and $y$. Moreover, $x \mapsto \int f_{x} \,{\mathrm d}\nu$ and $y \mapsto \int f^{y} \,{\mathrm d}\mu$ are measurable, and in case (ii) also integrable, and
	\begin{equation*} 
		\int f \,{\mathrm d}\lambda = \iint f(x, y) \,{\mathrm d}\mu(x)\,{\mathrm d}\nu(y) = \iint f(x, y) \,{\mathrm d}\nu(y)\,{\mathrm d}\mu(x).
	\end{equation*}
\end{thm}

\subsection{Integration in Polar Coordinates}

If $x \in \mathbb{R}^{n} \setminus \{0\}$, the \deff{polar coordinates} of $x$ are
\begin{equation*} 
	r \vcentcolon= \|x\| \in (0, \infty), \quad x' \vcentcolon= \frac{x}{r} \in S^{n - 1}.
\end{equation*}
The map $\Phi(x) \vcentcolon= (r, x')$ is a homeomorphism from $\mathbb{R}^{n} \setminus \{0\}$ onto $(0, \infty) \times S^{n - 1}$.

$m_{\ast}$ is the Borel measure on $(0, \infty) \times S^{n - 1}$ defined by
\begin{equation*} 
	m_{\ast}(E) \vcentcolon= m(\Phi^{-1}(E)).
\end{equation*}
We define the measure $\rho = \rho_{n}$ on $(0, \infty)$ by $\rho(E) \vcentcolon= \int_{E} r^{n - 1} \,{\mathrm d} r$.

\begin{thm}
	There is a unique measure $\sigma = \sigma_{n - 1}$ on $S^{n - 1}$ such that $m_{\ast} = \rho \times \sigma$. If $f$ is Borel measurable on $\mathbb{R}^{n}$ and $f \ge 0$ or $f \in L^{1}(m)$, then
	\begin{equation*} 
		\int_{\mathbb{R}^{n}} f(x) \,{\mathrm{d}}x = \int_{0}^{\infty} \int_{S^{n - 1}} f(r x') r^{n - 1} \,{\mathrm{d}}\sigma(x') \,{\mathrm{d}}r.
	\end{equation*}
\end{thm}

$\sigma$ above is defined as follows: given a Borel set $E \subset S^{n - 1}$, define $E' = \Phi^{-1}((0, 1] \times E) = \{rx' : 0 < r \le 1, x' \in E\}$, and set $\sigma(E) \vcentcolon= n \cdot m(E')$.

\begin{cor}
	If $f$ is a measurable function on $\mathbb{R}^{n}$ such that $f \in L^{+} \cup L^{1}$ and $f(x) = g(\|x\|)$ for some function $g$ on $(0, \infty)$, then
	\begin{equation*} 
		\int_{\mathbb{R}^{n}} f(x) \,{\mathrm{d}}x = \sigma(S^{n - 1}) \int_{0}^{\infty} g(r) r^{n - 1} \,{\mathrm{d}}r.
	\end{equation*}
\end{cor}

\begin{prop}
	For $a > 0$, we have
	\begin{equation*} 
		\int_{\mathbb{R}^{n}}^{} \exp(-a \|x\|^{2}) \,{\mathrm{d}}x = \left(\frac{\pi}{a}\right)^{n/2}.
	\end{equation*}
	Moreover,
	\begin{equation*} 
		\sigma(S^{n - 1}) = \frac{2 \pi^{n/2}}{\Gamma(n/2)}.
	\end{equation*}
	If $B^{n} = \{x \in \mathbb{R}^{n} : \|x\| < 1\}$, then
	\begin{equation*} 
		m(B^{n}) = \frac{\pi^{n/2}}{\Gamma\left(\frac{1}{2}n + 1\right)}.
	\end{equation*}
\end{prop}

\section{Signed Measures and Differentiation}

\subsection{Signed measures}

\begin{defn}
	Let $(X, \mathcal{M})$ be a measurable space. A \deff{signed measure} on $(X, \mathcal{M})$ is a function $\nu : X \to [-\infty, \infty]$ such that
	\begin{enumerate}
		\item $\nu(\emptyset) = 0$;
		\item $\nu$ assumes at most one of the values $\pm \infty$;
		\item if $(E_{j})_{j}$ is a sequence of disjoint sets in $\mathcal{M}$, then $\nu(\bigcup_{j} E_{j}) = \sum_{j} \nu(E_{j})$, where the sum converges absolutely if it is finite.
	\end{enumerate}
\end{defn}
Measures as defined earlier are examples of signed measures. For emphasis, we may use the term \deff{positive measure} for the usual measures.

\begin{ex}
	Here are two examples, which are essentially the only examples of signed measures.
	\begin{enumerate}
		\item If $\mu_{1}, \mu_{2}$ are positive measures on $\mathcal{M}$ and at least one of them is finite, then $\nu = \mu_{1} - \mu_{2}$ is a signed measure.
		\item If $\mu$ is a positive measure on $\mathcal{M}$ and $f : X \to [-\infty, \infty]$ is a measurable function such that at least one of $\int f^{+} \,{\mathrm d}\mu$ or $\int f^{-} \,{\mathrm d}\mu$ is finite (in which case we call $f$ an \deff{extended $\mu$-integrable} function), then the function $\nu$ defined on $\mathcal{M}$ by
		\begin{equation*} 
			\nu(E) \vcentcolon= \int_{E} f \,{\mathrm{d}}\mu
		\end{equation*}
		is a signed measure.

		We denote the above relationship by
		\begin{equation} \label{eq:002}
			{\mathrm d}\nu = f \,{\mathrm d}\mu.
		\end{equation}
		By abuse, we may even refer to $\nu$ by $f \,{\mathrm d}\mu$.
	\end{enumerate}
\end{ex}
\begin{rem}
	Note that monotonicity is not a property of a signed measure. (In fact, monotonicity is a property iff the measure is positive.)
\end{rem}

\begin{prop}
	Let $\nu$ be a signed measure on $(X, \mathcal{M})$. If $(E_{j})_{j}$ is an increasing sequence in $\mathcal{M}$, then $\nu(\bigcup_{j} E_{j}) = \lim_{j} \nu(E_{j})$. If $(E_{j})_{j}$ is a decreasing sequence with some $\nu(E_{j})$ finite, then $\nu(\bigcap_{j} E_{j}) = \lim_{j} \nu(E_{j})$.
\end{prop}

\begin{defn}
	If $\nu$ is a signed measure on $(X, \mathcal{M})$, a set $E \in \mathcal{M}$ is called \deff{positive} (resp. \deff{negative}, \deff{null}) for $\nu$ if $\nu(F) \ge 0$ (resp. $\nu(F) \le 0$, $\nu(F) = 0$) for all $F \in \mathcal{M}$ such that $F \subset E$.
\end{defn}
\begin{ex}
	In the earlier example of $\nu(E) = \int_{E} f \,{\mathrm{d}}\mu$, we have that $E$ is positive, negative, or null precisely when $f \ge 0$, $f \le 0$, $f = 0$ $\mu$-a.e. on $E$.
\end{ex}

\begin{rem}
	Note that $\nu(E) = 0$ is not enough for $E$ to be null. (Similar comments for positive and negative.)
\end{rem}

\begin{prop}
	Any measurable subset of a positive set is positive, and the union of any countable family of positive sets is positive.
\end{prop}
The statement is true for ``positive'' replaced with ``negative'' and ``null'' as well.

\begin{thm}[The Hahn Decomposition Theorem]
	If $\nu$ is a signed measure on $(X, \mathcal{M})$, there exists a positive set $P$ and a negative set $N$ for $\nu$ such that $X = P \sqcup N$ (and $P \cap N = \emptyset$). \newline
	If $P'$, $N'$ is another such pair, then $P \Delta P'$ ($= N \Delta N'$) is null for $\nu$.
\end{thm}

The decomposition $X = P \sqcup N$ of $X$ as a disjoint union of a positive set and a negative set is called a \deff{Hahn decomposition} for $\nu$.

\begin{defn}
	Two signed measures $\mu$ and $\nu$ on $(X, \mathcal{M})$ are \deff{mutually singular}, or that \deff{$\nu$ is singular with respect to $\mu$}, or vice-versa, if there exist disjoint sets $E, F \in \mathcal{M}$ such that
	\begin{enumerate}
		\item $X = E \sqcup F$,
		\item $E$ is null for $\mu$,
		\item $F$ is null for $\nu$.
	\end{enumerate}
	This is denoted by $\mu \perp \nu$.
\end{defn}

\begin{thm}[The Jordan Decomposition Theorem]
	If $\nu$ is a signed measure on $(X, \mathcal{M})$, there exist unique positive measures $\nu^{+}$ and $\nu^{-}$ such that $\nu = \nu^{+} - \nu^{-}$ and $\nu^{+} \perp \nu^{-}$.

	Given a Hahn decomposition $X = P \sqcup N$, we have $\nu^{+}(E) = \nu(E \cap P)$ and $\nu^{-}(E) = -\nu(E \cap N)$ for all $E \in \mathcal{M}$.
\end{thm}

The measures $\nu^{+}$ and $\nu^{-}$ are called the \deff{positive} and \deff{negative variations} of $\nu$, and $\nu = \nu^{+} - \nu^{-}$ is called the \deff{Jordan decomposition} of $\nu$. The \deff{total variation} of $\nu$ is the positive measure $\md{\nu}$ defined by $\md{\nu} = \nu^{+} + \nu^{-}$.

\begin{exe}
	$E \in \mathcal{M}$ is $\nu$-null iff $\md{\nu}(E) = 0$.

	$\nu \perp \mu$ iff $\md{\nu} \perp \mu$ iff $\nu^{+} \perp \mu$ and $\nu^{-} \perp \mu$.
\end{exe}

\begin{obs}
	Note that in general, $\nu$ is not bounded by $\nu(X)$. However, $\nu$ \emph{is} bounded by $\nu^{+}(X) = \nu(P)$. In particular, if $\nu$ omits the value $\infty$, then $\nu^{+}(X) < \infty$. Similarly for $-\infty$. \newline
	Consequently, if the range of $\nu$ is contained in $\mathbb{R}$, then $\nu$ is finite.
\end{obs}

\begin{obs}
	Let $\nu$ be a signed measure on $(X, \mathcal{M})$, and $X = P \sqcup N$ be a Hahn decomposition, and set $f \vcentcolon= \chi_{P} - \chi_{N}$. If we set $\mu = \md{\nu}$, then $\mu$ is a positive measure and we have
	\begin{equation*} 
		\nu(E) = \int_{E} f \,{\mathrm{d}}\mu.
	\end{equation*}
\end{obs}

Integration with respect to a signed measure $\nu$ is defined as follows:
\begin{align*} 
	L^{1}(\nu) &\vcentcolon= L^{1}(\nu^{+}) \cap L^{1}(\nu^{-}), \\
	\int f \,{\mathrm{d}}\nu &\vcentcolon= \int f \,{\mathrm{d}}\nu^{+} - \int f \,{\mathrm{d}}\nu^{-} \quad (f \in L^{1}(\nu)).
\end{align*}

A signed measure $\nu$ is called \deff{finite} (resp. \deff{$\sigma$-finite}) if $\md{\nu}$ is so.

\begin{prop}
	Let $\nu$ be a signed measure on $(X, \mathcal{M})$, and $E \in \mathcal{M}$. Then,
	\begin{align*} 
		\nu^{+}(E) &= \sup\{\nu(F) : F \subset E,\, F \in \mathcal{M}\}, \\
		\nu^{-}(E) &= -\inf\{\nu(F) : F \subset E,\, F \in \mathcal{M}\}, \\
		\md{\nu}(E) &= \sup\left\{\sum_{j = 1}^{n} \md{\nu(E_{j})} : n \in \mathbb{N},\, E = \bigsqcup_{j = 1}^{n} E_{j},\, E_{1}, \ldots, E_{n} \in \mathcal{M}\right\}.
	\end{align*}
\end{prop}

\subsection{The Lebesgue-Radon-Nikodym Theorem}

\begin{defn}
	Let $\nu$ be a signed measure and $\mu$ a positive measure on $(X, \mathcal{M})$. We say that $\nu$ is \deff{absolutely continuous} with respect to $\mu$, denoted $\nu \ll \mu$, if
	\begin{equation*} 
		\mu(E) \Rightarrow \nu(E)
	\end{equation*}
	for all $E \in \mathcal{M}$.
\end{defn}

\begin{exe}
	The following are equivalent:
	\begin{enumerate}
		\item $\nu \ll \mu$,
		\item $\md{\nu} \ll \mu$,
		\item $\nu^{+} \ll \mu$ and $\nu^{-} \ll \mu$.
	\end{enumerate}
\end{exe}
\begin{exe}
	$\nu \perp \mu$ and $\nu \ll \mu$ implies $\nu = 0$.
\end{exe}

\begin{thm}
	Let $\nu$ be a \underline{finite} signed measure and $\mu$ a positive measure on $(X, \mathcal{M})$. The following are equivalent:
	\begin{enumerate}
	 	\item $\nu \ll \mu$,
	 	\item for every $\epsilon > 0$, there exists $\delta > 0$ such that $\md{\nu(E)} < \epsilon$ whenever $\mu(E) < \delta$.
	 \end{enumerate} 
\end{thm}
Note that $\nu \ll \mu$ iff $\md{\nu} \ll \mu$ and hence, the ``$\md{\nu(E)} < \epsilon$'' in the second statement can also be replaced with ``$\md{\nu}(E) < \epsilon$''.

\begin{rem}
	Given a positive measure $\mu$ and an extended $\mu$-integrable function $f$, the signed measure $\nu$ defined by $\nu(E) = \int_{E} f \,{\mathrm{d}}\mu$ is absolutely continuous with respect to $\mu$. (That is, ${\mathrm d}\nu = f \,{\mathrm d}\mu$.)

	Moreover, $\nu$ is finite iff $f \in L^{1}(\mu)$.
\end{rem}
\begin{exe}
	$\nu$ being finite cannot be dropped. Check that in the following two examples that $\nu \ll \mu$ but the $\epsilon$-$\delta$ condition is not satisfied. (Note that $\nu$ is $\sigma$-finite in both cases.)
	\begin{enumerate}
		\item ${\mathrm d} \nu(x) = \,{\mathrm d}x/x$ and ${\mathrm d}\mu = \,{\mathrm d} x$ on $(0, 1)$.
		\item $\nu$ is the counting measure and $\mu(E) = \sum_{n \in E} 2^{-n}$ on $\mathbb{N}$.
	\end{enumerate}
\end{exe}

\begin{cor}
	If $f \in L^{1}(\mu)$, for every $\epsilon > 0$, there exists $\delta > 0$ such that $\md{\int_{E} f \,{\mathrm{d}}\mu} < \epsilon$ whenever $\mu(E) < \delta$.
\end{cor}

\begin{prop}
	Suppose that $\nu$ and $\mu$ are finite positive measures on $(X, \mathcal{M})$. Either $\nu \perp \mu$, or there exists $\epsilon > 0$ and $E \in \mathcal{M}$ such that $\mu(E) > 0$ and $E$ is a positive set for $\nu - \epsilon \mu$.
\end{prop}

\begin{thm}[The Lebesgue-Radon-Nikodym Theorem]
	Let $\nu$ be a $\sigma$-finite signed measure and $\mu$ a $\sigma$-finite positive measure on $(X, \mathcal{M})$. There exist unique $\sigma$-finite signed measures $\lambda$, $\rho$ on $(X, \mathcal{M})$ such that
	\begin{align}
		\lambda \perp \mu, \quad \rho \ll \mu, \label{eq:003} \\
		\nu = \lambda + \rho. \label{eq:004}
	\end{align}
	Moreover, there is an extended $\mu$-integrable function $f : X \to \mathbb{R}$ such that ${\mathrm d}\rho = f \,{\mathrm d}\mu$, and any two such functions are equal a.e.
\end{thm}
(Recall \Cref{eq:002} for the last notation.) The decomposition $\nu = \lambda + \rho$ satisfying \Cref{eq:003} is called the \deff{Lebesgue decomposition} of $\nu$ with respect to $\mu$.
\begin{cor}[Radon-Nikodym theorem]
	(Continuing the same hypothesis.) In particular, if $\nu \ll \mu$, then ${\mathrm d} \nu = f \,{\mathrm d} \mu$ for some $f$.
\end{cor}
$f$ above is called the \deff{Radon-Nikodym derivative} of $\nu$ with respect to $\mu$ and is denoted by ${\mathrm d} \nu/{\mathrm d} \mu$. (Technically, this is a class of functions equal to $f$ a.e.)

\begin{exe}
	$\sigma$-finiteness is necessary. Let $X = [0, 1]$, $\mathcal{M} = \mathcal{B}([0, 1])$, $m = \text{Lebesgue measure}$, and $\mu = \text{counting measure}$ on $\mathcal{M}$. Show that
	\begin{enumerate}
		\item $m \ll \mu$ but ${\mathrm d} m \neq f \,{\mathrm d} \mu$ for any $f$,
		\item $\mu$ has no Lebesgue decomposition with respect to $m$.
	\end{enumerate}
\end{exe}

\begin{prop}
	Suppose that $\nu$ is a $\sigma$-finite signed measure and $\mu$, $\lambda$ are $\sigma$-finite signed measures on $(X, \mathcal{M})$ such that $\nu \ll \mu \ll \lambda$.
	\begin{enumerate}
		\item If $g \in L^{1}(\nu)$, then $g \cdot \frac{{\mathrm d} \nu}{{\mathrm d} \mu} \in L^{1}(\mu)$ and
		\begin{equation*} 
			\int g \,{\mathrm{d}}\nu = \int g \frac{{\mathrm d} \nu}{{\mathrm d} \mu} \,{\mathrm{d}}\mu.
		\end{equation*}
		%
		\item We have $\nu \ll \lambda$, and
		\begin{equation*} 
			\frac{{\mathrm d} \nu}{{\mathrm d} \lambda} = \frac{{\mathrm d} \nu}{{\mathrm d} \mu}\frac{{\mathrm d} \mu}{{\mathrm d} \lambda} \quad \lambda \text{-a.e.}
		\end{equation*}
	\end{enumerate}
\end{prop}

\begin{cor}
	If $\mu \ll \lambda$ and $\lambda \ll \mu$, then $(\frac{{\mathrm d} \lambda}{{\mathrm d} \mu})(\frac{{\mathrm d} \mu}{{\mathrm d} \lambda}) = 1$ a.e. (with respect to either $\mu$ or $\lambda$).
\end{cor}

\begin{obs}
	If $\mu_{1}, \ldots, \mu_{n}$ are positive measures on $(X, \mathcal{M})$, then $\mu \vcentcolon= \sum_{j} \mu_{j}$ is a positive measure such that $\mu_{j} \ll \mu$ for all $j$.
\end{obs}

\subsection{Complex measures}

\begin{defn}
	A \deff{complex measure} on a measurable space $(X, \mathcal{M})$ is a map $\nu : \mathcal{M} \to \mathbb{C}$ such that
	\begin{enumerate}
		\item $\nu(\emptyset) = 0$,
		\item if $(E_{j})_{j}$ is a sequence if disjoint sets in $\mathcal{M}$, then $\nu(\bigcup_{j} E_{j}) = \sum_{j} \nu(E_{j})$, where the sum converges absolutely.
	\end{enumerate}
\end{defn}
Note that $\nu$ cannot take infinite values. So, a usual positive measure is a complex measure only if it is finite. 
\begin{ex}
	If $\mu$ is a positive measure, and $f \in L^{1}(\mu)$, then $f \,{\mathrm d}\mu$ is a complex measure.
\end{ex}

If $\nu$ is a complex measure, we write $\nu_{r}$ and $\nu_{i}$ for the real and imaginary parts of $\nu$. $\nu_{r}$ and $\nu_{i}$ are signed measures which do not take the values $\pm \infty$ and hence, finite. Thus, $\nu$ is a bounded subset of $\mathbb{C}$.

Integration: $L^{1}(\nu) \vcentcolon= L^{1}(\nu_{r}) \cap L^{1}(\nu_{i})$, and for $f \in L^{1}(\nu)$, we define
\begin{equation*} 
	\int f \,{\mathrm{d}}\nu \vcentcolon= \int f \,{\mathrm{d}}\nu_{r} + \iota \int f \,{\mathrm{d}}\nu_{i}.
\end{equation*}

If $\nu$ and $\mu$ are complex measures, we say $\nu \perp \mu$ if $\nu_{a} \perp \mu_{b}$ for all $\{a, b\} \subset \{i, r\}$. If $\lambda$ is a positive measure, we say $\nu \ll \lambda$ if $\nu_{r} \ll \lambda$ and $\nu_{i} \ll \lambda$.

\begin{thm}[The Lebesgue-Radon-Nikodym Theorem]
	If $\nu$ is a complex measure and $\mu$ a $\sigma$-finite positive measure on $(X, \mathcal{M})$, there exist a complex measure $\lambda$ and an $f \in L^{1}(\mu)$ such that $\lambda \perp \mu$ and ${\mathrm d} \nu = {\mathrm d} \lambda + f \, {\mathrm d} \mu$. \newline
	If also $\lambda' \perp \mu$ and ${\mathrm d} \nu = {\mathrm d} \lambda' + f \, {\mathrm d} \mu$, then $\lambda = \lambda'$ and $f = f'$ $\mu$-a.e.
\end{thm}
As before, if $\nu \ll \mu$, we denote $f$ above by ${\mathrm d}\nu/{\mathrm d}\mu$.

Given any complex measure $\nu$, we can write $\nu$ as ${\mathrm d}\nu = f \,{\mathrm d}\mu$ for some positive measure $\mu$ (one candidate is $\mu = \md{\nu_{r}} + \md{\nu_{i}}$). The \deff{total variation} of $\nu$ is the positive measure $\md{\nu}$ determined by 
\begin{equation*} 
	{\mathrm d}\md{\nu} = \md{f} \,{\mathrm d}\mu.
\end{equation*}
One can check that this $\nu$ is independent of $f$ and $\mu$. Moreover, this coincides with the earlier definition for a (finite) signed measure.

\begin{prop}
	Let $\nu$ be a complex measure on $(X, \mathcal{M})$.
	\begin{enumerate}
		\item $\md{\nu(E)} \le \md{\nu}(E)$ for all $E \in \mathcal{M}$.
		\item $\nu \ll \md{\nu}$, and ${\mathrm d}\nu/{\mathrm d}\md{\nu}$ has absolute value $1$ $\md{\nu}$-a.e.
		\item $L^{1}(\nu) = L^{1}(\md{\nu})$, and if $f \in L^{1}(\nu)$, then $\md{\int f \,{\mathrm{d}}\nu} \le \int \md{f} \,{\mathrm{d}}\md{\nu}$.
	\end{enumerate}
\end{prop}

\begin{prop}
	$\md{\nu_{1} + \nu_{2}} \le \md{\nu_{1}} + \md{\nu_{2}}$ for complex measures $\nu_{1}$, $\nu_{2}$ on $(X, \mathcal{M})$.
\end{prop}

\subsection{Differentiation on Euclidean Space}

In this section, we look at the special case of the Lebesgue measure $m$ on $\mathbb{R}^{n}$. The terms ``integrable'' and ``almost everywhere'' will mean with respect to the Lebesgue measure.

\begin{prop}
	Let $\mathcal{C}$ be a collection of open balls in $\mathbb{R}^{n}$, and let $U = \bigcup_{B \in \mathcal{C}} B$. If $c < m(U)$, there exist disjoint $B_{1}, \ldots, B_{k} \in \mathcal{C}$ such that $\sum_{j = 1}^{k} m(B_{j}) > 3^{-n} c$.
\end{prop}

\begin{defn}
	A measurable function $f : \mathbb{R}^{n} \to \mathbb{C}$ is called \deff{locally integrable} if $\int_{K} \md{f} < \infty$ for every bounded measurable set $K \subset \mathbb{R}^{n}$. (Equivalently, for every compact set $K \subset \mathbb{R}^{n}$.)

	The space of locally integrable functions is denoted by $\loc$. If $f \in \loc$, $x \in \mathbb{R}^{n}$, and $r > 0$, we define $A_{r}f(x)$ by
	\begin{equation*} 
		A_{r}f(x) \vcentcolon= \frac{1}{m(B(r, x))} \int_{B(r, x)} f.
	\end{equation*}
\end{defn}

\begin{prop}
	If $f \in \loc$, $A_{r}f(x)$ is jointly continuous in $r$ and $x$ ($r > 0$, $x \in \mathbb{R}^{n}$).
\end{prop}

\begin{defn}
	If $f \in \loc$, we define its \deff{Hardy-Littlewood maximal function} $Hf$ by
	\begin{equation*} 
		Hf(x) \vcentcolon= \sup_{r > 0} A_{r}\md{f}(x) = \sup_{r > 0} \frac{1}{m(B(r, x))} \int_{B(r, x)} \md{f}.
	\end{equation*}
\end{defn}
$Hf$ is a measurable function.

\begin{thm}
	Fix $n$. There is a constant $C > 0$ such that for all $f \in L^{1}(\mathbb{R}^{n})$ and all $\alpha > 0$,
	\begin{equation*} 
		m(\{x : Hf(x) > \alpha\}) \le \frac{C}{\alpha} \int_{\mathbb{R}^{n}} \md{f}.
	\end{equation*}
\end{thm}

\begin{thm}
	If $f \in \loc$, then $\lim_{r \to 0} A_{r}f(x) = f(x)$ for a.e. $x \in \mathbb{R}^{n}$.
\end{thm}

\begin{defn}
	For $f \in \loc$, define the \deff{Lebesgue set} $L_{f}$ of $f$ to be 
	\begin{equation*} 
		L_{f} \vcentcolon= \left\{x \in \mathbb{R}^{n} : \lim_{r \to 0} \frac{1}{m(B(r, x))} \int_{B(r, x)} \md{f(x) - f(y)} \,{\mathrm{d}}y\right\}.
	\end{equation*}
\end{defn}

\begin{thm}
	If $f \in \loc$, then $m((L_{f})^{c}) = 0$.
\end{thm}
Note that this is a strengthening of the previous theorem.

\begin{defn}
	A family $(E_{r})_{r > 0}$ of Borel subsets of $\mathbb{R}^{n}$ is said to \deff{shrink nicely} to $x \in \mathbb{R}^{n}$ if
	\begin{enumerate}
		\item $E_{r} \subset B(r, x)$ for each $r$;
		\item there is a constant $\alpha > 0$, independent of $r$, such that $m(E_{r}) > \alpha m(B(x, r))$ for all $r$.
	\end{enumerate}
\end{defn}
Note that $x \in E_{r}$ is not necessary.

\begin{thm}[The Lebesgue Differentiation Theorem]
	Suppose $f \in \loc$. For every $x \in L_{f}$ -- in particular, for almost every $x$ -- we have
	\begin{align*} 
		\lim_{r \to 0} \frac{1}{m(E_{r})} \int_{E_{r}} \md{f(y) - f(x)} \,{\mathrm{d}}y &= 0 \andd \\
		\lim_{r \to 0} \frac{1}{m(E_{r})} \int_{E_{r}} f &= f(x)
	\end{align*}
	for every family $(E_{r})_{r > 0}$ that shrinks nicely to $x$.
\end{thm}

\begin{defn}
	A Borel measure $\nu$ on $\mathbb{R}^{n}$ will be called \deff{regular} if
	\begin{enumerate}
		\item $\nu(K) < \infty$ for every compact $K$;
		\item $\nu(E) = \inf\{\nu(U) : U \supset E,\, U \text{ open}\}$ for every $E \in \mathcal{B}(\mathbb{R}^{n})$.
	\end{enumerate}
	A signed or complex Borel measure $\nu$ will be called \deff{regular} if $\md{\nu}$ is regular.
\end{defn}
The second condition is actually implied by the first. For $n = 1$, this follows from results in the first section.

Every regular measure is $\sigma$-finite. 

\begin{ex}
	If $f \in L^{+}(\mathbb{R}^{n})$, the measure $f \,{\mathrm d}m$ is regular iff $f \in \loc$.
\end{ex}

\begin{thm}
	Let $\nu$ be a regular signed or complex Borel measure on $\mathbb{R}^{n}$, and let ${\mathrm d}\nu = {\mathrm d}\lambda + f \, {\mathrm d}m$ be its Lebesgue-Radon-Nikodym representation. Then, for $m$-almost every $x \in \mathbb{R}^{n}$,
	\begin{equation*} 
		\lim_{r \to 0} \frac{\nu(E_{r})}{m(E_{r})} = f(x)
	\end{equation*}
	for every family $(E_{r})_{r > 0}$ that shrinks nicely to $x$.
\end{thm}

\subsection{Functions of Bounded Variation}

\textbf{Notations}: For a function $F : \mathbb{R} \to \mathbb{R}$, $F(x+)$ denotes the right limit $\lim_{y \to x^{+}} F(y)$. (This will exist, for example, when $F$ is increasing.) $F(x-)$ is defined similarly. \newline
If $F$ is increasing and right-continuous, $\mu_{F}$ is the Borel measure on $\mathbb{R}$ determined by $\mu_{F}((a, b]) = F(b) - F(a)$.

\begin{thm}
	Let $F : \mathbb{R} \to \mathbb{R}$ be increasing, and let $G(x) = F(x+)$. 
	\begin{enumerate}
		\item The set of discontinuities of $F$ is countable.
		\item $F$ and $G$ are differentiable a.e., and $F' = G'$ a.e.
	\end{enumerate}
\end{thm}

\begin{defn}
	If $F : \mathbb{R} \to \mathbb{C}$ and $x \in \mathbb{R}$, we define
	\begin{equation*} 
		T_{F}(x) \vcentcolon= \sup \left\{\sum_{j = 1}^{n} \md{F(x_{j}) - F(x_{j - 1})} : n \in \mathbb{N},\, -\infty < x_{0} < \cdots < x_{n} = x\right\}.
	\end{equation*}
	$T_{F}$ is called the \deff{total variation} of $F$.
\end{defn}

If $a < b$, we have
\begin{equation} \label{eq:005}
	T_{F}(b) - T_{F}(a) = \sup\left\{\sum_{j = 1}^{n} \md{F(x_{j}) - F(x_{j - 1})} : n \in \mathbb{N},\, a = x_{0} < \cdots < x_{n} = b\right\}.
\end{equation}

$T_{F}$ is an increasing function with values in $[0, \infty]$. 

\begin{defn}
	If $T_{F}(\infty) = \lim_{x \to \infty} T_{F}(x)$ is finite, we say that $F$ is of \deff{bounded variation} on $\mathbb{R}$, and we denote the space of all such $F$ by $\BV$.
\end{defn}

$\BV$ forms a complex vector space.

The supremum on the right in \Cref{eq:005} is called the \deff{total variation} of $F$ on $[a, b]$. The space of functions $F : [a, b] \to \mathbb{C}$ whose total variation on $[a, b]$ is finite is denoted $\BV([a, b])$.

\begin{rem}
	If $F \in \BV$, then $F|[a, b]$ is in $\BV([a, b])$ for all $a, b \in \mathbb{R}$ with $a < b$.

	Conversely, if $F \in \BV([a, b])$ and we set $F(x) = F(a)$ for $x < a$ and $F(x) = F(b)$ for $x > b$, then $F \in \BV$.
\end{rem}
\begin{ex}
	\begin{enumerate}
		\item If $F : \mathbb{R} \to \mathbb{R}$ is bounded and increasing, then $F \in \BV$.
		\item $\sin \in \BV([a, b])$ for all reals $a < b$ but $\sin \notin \BV$.
		\item If $F$ is differentiable and $F'$ is bounded, then $F \in \BV([a, b])$ for all reals $a < b$.
		\item 
	\end{enumerate}
\end{ex}

\begin{prop}
	If $F \in \BV$ is real-valued, then $T_{F} + F$ and $T_{F} - F$ are increasing.
\end{prop}

\begin{thm}
	Let $F : \mathbb{R} \to \mathbb{C}$.
	\begin{enumerate}
		\item $F \in \BV$ iff $\Re(F) \in \BV$ and $\Im(F) \in \BV$.
		\item If $F$ is real valued, then $F \in \BV$ iff $F$ is the difference of two bounded increasing functions; for $F \in \BV$, these functions may be taken to be $\frac{1}{2}(T_{F} \pm F)$.
		\item If $F \in \BV$, then $F(x+)$ and $F(x-)$ exist for all $x \in \mathbb{R}$, as do $F(\pm \infty)$.
		\item If $F \in \BV$, the set of discontinuities of $F$ is countable.
		\item If $F \in \BV$ and $G(x) \vcentcolon= F(x+)$, then $F'$ and $G'$ exist a.e. and are equal a.e.
	\end{enumerate}
\end{thm}

The representation
\begin{equation*} 
	F = \frac{1}{2}(T_{F} + F) - \frac{1}{2}(T_{F} - F)
\end{equation*}
of a real-valued $F \in \BV$ is called a \deff{Jordan decomposition} of $F$, and $\frac{1}{2}(T_{F} + F)$ and $\frac{1}{2}(T_{F} - F)$ are called the \deff{positive} and \deff{negative variations} of $F$. 

For $x \in \mathbb{R}$, define $x^{+} \vcentcolon= \max(x, 0) = \frac{1}{2}(\md{x} + x)$ and $x^{-} \vcentcolon= \max(-x, 0) = \frac{1}{2}(\md{x} - x)$. We then have
\begin{equation*} 
	\frac{1}{2}(T_{F} \pm F)(x) = \sup\left\{\sum_{j = 1}^{n} \left[F(x_{j}) - F(x_{j - 1})\right]^{\pm} : n \in \mathbb{N},\, x_{0} < \cdots < x_{n} = x \right\} \pm \frac{1}{2} F(-\infty).
\end{equation*}

We define the space $\NBV$ (N for ``normalised''):
\begin{equation*} 
	\NBV \vcentcolon= \{F \in \BV : F \text{ is right-continuous and } F(-\infty) = 0\} \subset \BV.
\end{equation*}

If $F \in \BV$, then the function defined by $G(x) \vcentcolon= F(x+) - F(-\infty)$ is in $\NBV$ and $F' = G'$ a.e.

\begin{prop}
	If $F \in \BV$, then $T_{F}(-\infty) = 0$. If $F$ is also right-continuous, then so is $T_{F}$.
\end{prop}

\begin{thm}
	If $\mu$ is a complex Borel measure on $\mathbb{R}$ and $F(x) \vcentcolon= \mu((-\infty, x])$, then $F \in \NBV$. \newline
	Conversely, if $F \in \NBV$, then there is a unique complex Borel measure $\mu_{F}$ such that $F(x) = \mu_{F}((-\infty, x])$; moreover, $\md{\mu_{F}} = \mu_{T_{F}}$.
\end{thm}

\begin{prop}
	If $F \in \NBV$, then $F' \in L^{1}(m)$.
	\begin{enumerate}
		\item $\mu_{F} \perp m$ iff $F' = 0$ a.e.
		\item $\mu_{F} \ll m$ iff $F$ is absolutely continuous iff $F(x) = \int_{-\infty}^{x} F'(t) \,{\mathrm{d}}t$.
	\end{enumerate}
	If $f \in L^{1}(m)$, then the function $F(x) \vcentcolon= \int_{-\infty}^{x} f(t) \,{\mathrm{d}}t$ is in $\NBV$ and is absolutely continuous, and $f = F'$ a.e. 
\end{prop}

Recall that $F : \mathbb{R} \to \mathbb{C}$ is \deff{absolutely continuous} if for every $\epsilon > 0$, there exists $\delta > 0$ such that for any finite set of disjoint intervals $(a_{1}, b_{1}), \ldots, (a_{N}, b_{N})$,
\begin{equation*} 
	\sum_{j} (b_{j} - a_{j}) < \delta \Rightarrow \sum_{j} \md{F(b_{j}) - F(a_{j})} < \epsilon.
\end{equation*}
More generally, $F$ is said to be absolutely continuous on $[a, b]$ if this condition is satisfied whenever $(a_{j}, b_{j})$ all lie in $[a, b]$.

If $F$ is differentiable on $\mathbb{R}$ and $F'$ is bounded, then $F$ is absolutely continuous.

For the following results, $a$ and $b$ are reals with $a < b$.

\begin{prop}
	If $F$ is absolutely continuous on $[a, b]$, then $F \in \BV([a, b])$.
\end{prop}

\begin{thm}[The Fundamental Theorem of Calculus for Lebesgue Integrals]
	For $F : [a, b] \to \mathbb{C}$, the following are equivalent:
	\begin{enumerate}
		\item $F$ is absolutely continuous on $[a, b]$.
		\item $F(x) - F(a) = \int_{x}^{a} f(t) \,{\mathrm{d}}t$ for some $f \in L^{1}([a, b], m)$.
		\item $F$ is differentiable a.e. on $[a, b]$, $F' \in L^{1}([a, b], m)$, and $F(x) - F(a) = \int_{x}^{a} F'(t) \,{\mathrm{d}}t$.
	\end{enumerate}
\end{thm}

\begin{defn}
	A complex measure $\mu$ on $\mathbb{R}^{n}$ is called \deff{discrete} if there is a countable set $\{x_{j}\}_{j \ge 1} \subset \mathbb{R}^{n}$ and complex numbers $(c_{j})_{j \ge 1}$ such that $\sum \md{c_{j}} < \infty$ and $\mu = \sum_{j} c_{j} \delta_{x_{j}}$, where $\delta_{x}$ is the point mass at $x$.

	$\mu$ is called \deff{continuous} if $\mu(\{x\}) = 0$ for all $x \in \mathbb{R}^{n}$.
\end{defn}

Any complex measure $\mu$ can be uniquely written as $\mu = \mu_{d} + \mu_{c}$ where $\mu_{d}$ is discrete and $\mu_{c}$ continuous. \newline
$\mu$ is discrete $\Rightarrow$ $\mu \perp m$. \newline
$\mu \ll m$ $\Rightarrow$ $\mu$ is continuous.

Any (regular) complex Borel measure on $\mathbb{R}^{n}$ can be written uniquely as
\begin{equation*} 
	\mu_{d} + \mu_{ac} + \mu_{sc},
\end{equation*}
where $\mu_{d}$ is discrete, $\mu_{ac}$ is absolutely continuous with respect to $m$, and $\mu_{sc}$ is a ``singular continuous'' measure, that is, $\mu_{sc}$ is continuous but $\mu_{sc} \perp m$.

If $F \in \NBV$, we denote the integral of a function $g$ with respect to $\mu_{F}$by $\int g \,{\mathrm d} F$ or $\int g(x) \,{\mathrm d}F(x)$; such integrals are \deff{Lebesgue-Stieltjes integrals}.

\begin{thm}
	If $F$ and $G$ are in $\NBV$ and at least one of them is continuous, then for $-\infty < a < b < \infty$,
	\begin{equation*} 
		\int_{(a, b]} F \,{\mathrm{d}}G + \int_{(a, b]} G \,{\mathrm{d}}F = F(b) G(b) - F(a) G(a).
	\end{equation*}
	If $F$ and $G$ are absolutely continuous on $[a, b]$, then so is $FG$, and
	\begin{equation*} 
		\int_{a}^{b} (FG' + G'F) = F(b) G(b) - F(a) G(a).
	\end{equation*}
\end{thm}

\end{document}