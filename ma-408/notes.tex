\documentclass[12pt]{article}	% Always compile at least twice.
\usepackage[lmargin=1in,rmargin=1in,tmargin=1in,bmargin=1in]{geometry}
% \usepackage{pdfpages}

% Cover Information	
\def\univname{}
\def\coursenum{MA 408}
\def\coursename{Measure Theory}
\def\professor{}
\def\student{Aryaman Maithani}
\def\semesteryear{Spring 2020-21}
\def\imagename{iitb.png}		    % Replace with University Seal
\def\scalesize{2}					% Scale Logo Size 

% Style Package (Load After Cover Information)
\usepackage{lecture-notes}	% Change to match style file

% -------------------
% Content
% -------------------
\begin{document}


% Cover Page
\coverpage

% Last Updated
\updated{\today}

% Table of Contents
\thispagestyle{empty}
\tableofcontents
\newpage
\pagestyle{fancy}
\setcounter{page}{1}

\section{} % Lecture 1

\begin{thm}[Non existence of ideal measure]
	There is no map $\mu : \mathcal{P}(\mathbb{R}) \to [0, \infty]$ such that
	\begin{enumerate}
		\item $\mu(\emptyset) = 0,$
		\item $\mu(E) = \mu(x + E)$ for all $x \in \mathbb{R}$ and $E \in \mathcal{P}(\mathbb{R}),$\\
		where $x + E \vcentcolon= \{x + y \mid y \in E\},$
		\item for any disjoint countable collection $\{E_i\}_{i}^{\infty}$ of subsets of $\mathbb{R},$ we have
		\begin{equation*} 
			\mu\left(\bigsqcup_{i = 1}^\infty E_i\right) = \sum_{i = 1}^{\infty} \mu(E_i),
		\end{equation*}
		\item $\mu([0, 1]) = 1.$
	\end{enumerate}
\end{thm}
Note that the last is a ``normalisation'' property. Otherwise $\mu \equiv 0$ or $\mu(X) = \begin{cases}
	0 & X = \emptyset,\\
	\infty & \text{otherwise}
\end{cases}$ would also satisfy and give us ``useless'' functions.

Replacing ``countable union'' with ``finite union'' also won't do the trick in general due to the Banach-Tarski ``paradox'' (theorem).

Both the above required a use of the Axiom of Choice.

\begin{defn}[Algebra]
	Let $X$ be a non-empty set.\\
	An \deff{algebra (``field'')} on $X$ is a non-empty collection $\mathcal{F} \subset \mathcal{P}(X)$ satisfying
	\begin{enumerate}
		\item $A \in \mathcal{F} \implies A^c \in \mathcal{F},$
		\item $A_1, \ldots, A_n \in \mathcal{F} \implies \bigcup_{i = 1}^nA_i \in \mathcal{F}.$
	\end{enumerate}
\end{defn}


\begin{defn}[$\sigma$-algebra]
	Let $X$ be a non-empty set.\\
	A \deff{$\sigma$-algebra (``$\sigma$-field'')} on $X$ is a non-empty collection $\mathcal{F} \subset \mathcal{P}(X)$ satisfying
	\begin{enumerate}
		\item $A \in \mathcal{F} \implies A^c \in \mathcal{F},$
		\item $A_1, A_2, \ldots \in \mathcal{F} \implies \bigcup_{i = 1}^\infty A_i \in \mathcal{F}.$
	\end{enumerate}
\end{defn}

\begin{ex}[Countable-cocountable $\sigma$-algebra]
	Let $X \neq \emptyset.$ Then,
	\begin{equation*} 
		\mathcal{F} = \{E \subset X \mid E \text{ or } E^c \text{ is countable}\}
	\end{equation*}
	is a $\sigma$-algebra on $X.$
\end{ex}

\begin{defn}[$\sigma$-algebra generated by a set]
	Let $\mathcal{E} \subset \mathcal{P}(X).$ Then,
	\begin{equation*} 
		\mathcal{M}(\mathcal{E}) \vcentcolon= \bigcap_{\substack{\mathcal{E} \subset \mathcal{B}\\
		\mathcal{B} \text{ is a }\sigma-\text{algebra}}} \mathcal{B}
	\end{equation*}
	is a $\sigma$-algebra. Moreover, it is the smallest $\sigma$-algebra containing $\mathcal{B}.$

	This is called the \deff{$\sigma$-algebra} generated by $\mathcal{B}.$
\end{defn}

\begin{defn}[Borel $\sigma$-algebra]
	Let $(X, \mathcal{T})$ be a topological space. The $\sigma$-algebra generated by $\mathcal{T}$ is called the \deff{Borel $\sigma$-algebra} on $X,$ denoted $\mathcal{B}(X).$
\end{defn}
In other words, $\mathcal{B}(X)$ is the $\sigma$-algebra generated by the open sets of $X.$

\begin{prop}
	All of the following are contained in $\mathcal{B}(\mathbb{R})$:
	\begin{enumerate}
		\item All closed sets.
		\item All open sets.
		\item All $F_\sigma$ and $G_\delta$ sets.
	\end{enumerate}
\end{prop}
Recall that an $F_\sigma$ set is a set which can be written as countable union of closed sets. Similarly, $G_\delta$ as countable intersection of open sets.

\begin{prop}
	$\mathcal{B}(\mathbb{R})$ is generated by any of the following collections.
	\begin{enumerate}
		\item $\{(a, b) \mid a < b\}$ or $\{[a, b] \mid a < b\},$
		\item $\{(a, b] \mid a < b\}$ or $\{[a, b) \mid a < b\},$
		\item $\{(a, \infty) \mid a \in \mathbb{R}\}$ or $\{(-\infty, b) \mid b \in \mathbb{R}\},$
		\item $\{[a, \infty) \mid a \in \mathbb{R}\}$ or $\{(-\infty, b] \mid b \in \mathbb{R}\}.$
	\end{enumerate}
\end{prop}

\begin{defn}[Product of $\sigma$-algebrae]
	Let $\left\{(X_i, \mathcal{M}_i)\right\}_{i = 1}^n$ be a finite collection of sets and $\sigma$-algebrae. 

	Put $X \vcentcolon= \prod_{i = 1}^{n} X_i$ and let $\pi_i : X \to X_i$ denote the projection onto the $i$-th coordinate.

	Let
	\begin{equation*} 
		\mathcal{B} = \{\pi_i^{-1}(E) \mid E \in \mathcal{M}_i,\; i = 1, \ldots, n\}.
	\end{equation*}

	Then, $\mathcal{M} = \mathcal{M}(\mathcal{B})$ is the \deff{product $\sigma$-algebra induced by $\{M_i\}_{i = 1}^n$} which we (misleadingly) denote by $\prod_{i = 1}^{n}\mathcal{M}_i.$
\end{defn}

With the above, we get two (possibly different) $\sigma$-algebrae on $\mathbb{R}^n.$ One is the Borel $\sigma$-algebra on it, by virtue of it being a topological space, i.e., $\mathcal{B}(\mathbb{R}^n)$ and the other is the product of $\sigma$-algebra, i.e., $\prod_{i = 1}^n \mathcal{B}(\mathbb{R}).$ As it turns out, both are equal.

\begin{thm}
	$\mathcal{B}(\mathbb{R}^n) = \prod_{i = 1}^{n}\mathcal{B}(\mathbb{R}).$
\end{thm}

\begin{rem}
	In general, the above can be generalised to a product of separable metric spaces. (Note that the product of metric spaces in the product topology is metrisable.)
\end{rem}

\begin{defn}[Measure]
	Suppose $X$ is a non-empty set and $\mathcal{M}$ a $\sigma$-algebra on $X.$ A \deff{measure} on $X$ is a map 
	\begin{equation*} 
		\mu : X \to [0, \infty]
	\end{equation*}
	satisfying
	\begin{enumerate}
		\item $\mu(\emptyset) = 0,$
		\item if $\{E_i\}_{1}^\infty \subset \mathcal{M}$ are pairwise disjoint, then
		\begin{equation*} 
			\mu\left(\bigsqcup_{i = 1}^\infty E_i\right) = \sum_{i = 1}^{\infty} \mu(E_i).
		\end{equation*}
	\end{enumerate}
	$(X, \mathcal{M}, \mu)$ is called a \deff{measure space}.
\end{defn}
Note that $\mu\left(\bigsqcup E_i\right)$ makes sense because $\mathcal{M}$ is a $\sigma$-algebra and hence $\bigsqcup E_i \in \mathcal{M}.$

\begin{prop}
	Suppose $(X, \mathcal{M}, \mu)$ is a measure space. All sets mentioned below are in $\mathcal{M}.$ Then,
	\begin{enumerate}
		\item $E \subset F \implies \mu(E) \le \mu(F),$
		\item $\mu\left(\bigcup_{1}^\infty E_i\right) \le \sum_{1}^{\infty} \mu(E_i),$
		\item If $E_i \uparrow$ (i.e., $E_1 \subset E_2 \subset \cdots$), then
		\begin{equation*} 
			\mu\left(\bigcup_{i = 1}^\infty E_i\right) = \lim_{n\to \infty} \mu(E_i).
		\end{equation*}
	\end{enumerate}
\end{prop}

\begin{defn}[Null set]
	A \deff{null set} in a measure space $(X, \mathcal{M}, \mu)$ is a set $N$ such that $N \subset F$ for some $F \in \mathcal{M}$ with $\mu(F) = 0.$
\end{defn}
Note that $N$ need not necessarily be in $\mathcal{M}.$ Of course, $F$ in the above is also a null set.

\begin{defn}[Completion]
	Given a measure space $(X, \mathcal{M}, \mu),$ the \deff{completion} of $\mathcal{M},$ denote $\overline{\mathcal{M}}$ is the collection of all subsets of the form $E \cup N$ where $E \in \mathcal{M}$ and $N$ is a null set. 
\end{defn}

Clearly, $\mathcal{M} \subset \overline{\mathcal{M}}$ since $\emptyset$ is a null set.

\begin{prop}[Extension to completion]
	Let $(X, \mathcal{M}, \mu)$ be a measure space.
	\begin{enumerate}
		\item $\overline{\mathcal{M}}$ is a $\sigma$-algebra.
		\item There is a unique measure
		\begin{equation*} 
			\overline{\mu} : \overline{\mathcal{M}} \to [0, \infty]
		\end{equation*}
		such that $\overline{\mu}|\mathcal{M} = \mu.$
	\end{enumerate}
\end{prop}

\section{} % Lecture 2

\begin{defn}[Outer measure]
	An \deff{outer measure} on $X$ is a map
	\begin{equation*} 
		\mu^* : \mathcal{P}(X) \to [0, 1]
	\end{equation*}
	satisfying
	\begin{enumerate}
		\item $\mu^*(\emptyset) = 0,$
		\item $A \subset B \implies \mu*^(A) \le \mu^*(B),$
		\item $\mu\left(\bigcup_{i = 1}^\infty E_i\right) \le \sum_{i = 1}^{\infty} \mu(E_i).$
	\end{enumerate}
\end{defn}
Note that we don't demand equality even if disjoint.

\begin{prop}[A construction of an outer measure] \label{prop:constructoutermeasure}
	Suppose $\mathcal{F} \subset \mathcal{P}(X)$ and $\rho : \mathcal{F} \to [0, \infty]$ is a map such that
	\begin{enumerate}
		\item $\emptyset, X \in \mathcal{F},$
		\item $\rho(\emptyset) = 0.$
	\end{enumerate}
	For $E \in \mathcal{P}(X),$ define
	\begin{equation*} 
		\mu^*(E) \vcentcolon= \inf\left\{\sum_{i = 1}^{\infty}\rho(E_i) \mid E_i \in \mathcal{F},\; E \subset \bigcup_{i = 1}^\infty E_i\right\}.
	\end{equation*}
	Then, $\mu^*$ is an outer measure.
\end{prop}
Note that the above had just the bare minimum requirement for both $\rho$ and $\mathcal{F}$ and still gave us that $\mu^*$ is an outer measure.

\begin{defn}[$\mu^*$-measurable]
	Given an outer measure $\mu^*$ on a set $X,$ a set $A \subset X$ is said to be \deff{$\mu^*$-measurable} if for all $E \subset X,$ we have
	\begin{equation*} 
		\mu^*(E) = \mu^*(E \cap A) + \mu^*(E \cap A^c).
	\end{equation*}
\end{defn}

\begin{defn}[Complete measure]
	A measure $\mu$ on $(X, \mathcal{M})$ is said to be \deff{complete} if $\mathcal{M}$ contains all null sets of $(X, \mathcal{M}, \mu).$
\end{defn}

\begin{thm}[Carathéodory]
	Let $\mu^*$ be an outer measure on $X.$ Let
	\begin{equation*} 
		\mathcal{M} \vcentcolon= \{E \subset X \mid E \text{ is } \mu^*\text{-measurable}\}.
	\end{equation*}
	Then,
	\begin{enumerate}
		\item $\mathcal{M}$ is a $\sigma$-algebra.
		\item $\mu^*|\mathcal{M}$ is a complete measure.
	\end{enumerate} 
\end{thm}

\begin{defn}[Pre-measure]
	Suppose $\mathcal{F}$ is an algebra on $X.$ A map
	\begin{equation*} 
		\mu_0 : \mathcal{F} \to [0, \infty]
	\end{equation*}
	is called a \deff{pre-measure} if
	\begin{enumerate}
		\item $\mu_0(\emptyset) = 0,$
		\item if $\{A_i\}_{i = 1}^\infty \subset \mathcal{F}$ are pairwise disjoint such that $\bigsqcup_{i = 1}^\infty A_i \in \mathcal{F},$ then
		\begin{equation*} 
			\mu_)\left(\bigsqcup_{i = 1}^\infty A_i\right) = \sum_{i = 1}^{\infty} \mu_)(A_i).
		\end{equation*}
	\end{enumerate}
\end{defn}
Note that by putting all but finitely many $A_i = \emptyset,$ the above equality holds for finite unions as well. (The finite union \emph{will} be in $\mathcal{F}$ since it's an algebra.)

\begin{prop}
	Suppose $\mu_0$ is a pre-measure on an algebra $\mathcal{F}.$ Then, if $\mu_*$ is the outer measure as defined in \Cref{prop:constructoutermeasure} (with $\rho = \mu_0$), then
	\begin{enumerate}
		\item $\mu^*|\mathcal{F} = \mu_0,$
		\item every set in $\mathcal{F}$ is $\mu^*$-measurable.
	\end{enumerate}
\end{prop}

\begin{thm}
	Suppose $\mathcal{F} \subset \mathcal{P}(X)$ is an algebra and let $\mathcal{M}$ be the $\sigma$-algebra generated by $\mathcal{F}.$

	Let $\mu_0$ be a pre-measure defined on $\mathcal{F}$ and let $\mu^*$ be the outer measure as before. Then
	\begin{enumerate}
		\item $\mu^*|\mathcal{M}$ is a measure on $(X, \mathcal{M}).$ Put $\mu = \mu^*|\mathcal{M}$ for the next part.
		\item If $\nu$ is any measure extending $\mu_0,$ then
		\begin{equation*} 
			\nu(E) = \mu(E)
		\end{equation*}
		whenever $\mu(E) < \infty.$
	\end{enumerate}
\end{thm}

\begin{defn}
	A \deff{half-interval} is a subset of $\mathbb{R}$ of one of the following forms:
	\begin{enumerate}
		\item $(a, b]$ for $-\infty \le a < b < \infty,$ 
		\item $(a, \infty)$ for $-\infty \le a < \infty,$
		\item $\emptyset.$
	\end{enumerate}
\end{defn}

\begin{prop}
	The collection of all finite unions of half-intervals is an algebra on $\mathbb{R}.$
\end{prop}

\begin{prop} 
	Let $\mathcal{F}$ be the algebra consisting of finite unions of half-intervals. Let $F : \mathbb{R} \to \mathbb{R}$ be an increasing an right continuous function. Define
	\begin{equation*} 
		\mu_0\left(\bigsqcup_{i = 1}^n (a_j, b_j]\right) \vcentcolon= \sum_{i = 1}^{n} [F(b_j) - F(a_j)],
	\end{equation*}
	and let $\mu_0(\emptyset) = 0.$
	
	Then, $\mu_0$ is a well-defined pre-measure on $\mathcal{F}.$
\end{prop}
\end{document}	

