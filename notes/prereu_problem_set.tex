\documentclass{article}
\usepackage{amsmath}
\usepackage{amsthm} % This is for theorems, lemmas, propositions, etc.
\usepackage{amssymb}%Package for symbols
\usepackage{graphicx} % Required for inserting images
\usepackage{hyperref}
\usepackage{enumitem}
\usepackage{mathtools}

\setlength{\hoffset}{-0.66in}
\setlength{\textwidth}{6in}
\DeclareMathOperator{\Aut}{Aut}
\DeclareMathOperator{\End}{End}
\DeclareMathOperator{\Hom}{Hom}
\DeclareMathOperator{\Ext}{Ext}
\DeclareMathOperator{\Tor}{Tor}
\DeclareMathOperator{\spn}{span}
\DeclareMathOperator{\Gal}{Gal}
\DeclareMathOperator{\id}{id}
\DeclareMathOperator{\Int}{int}
\DeclareMathOperator{\INT}{Int}
\DeclareMathOperator{\Max}{Max}
\DeclareMathOperator{\GL}{GL}
\newcommand{\gl}{\mathfrak{gl}}
\DeclareMathOperator{\SL}{SL}
\renewcommand{\sl}{\mathfrak{sl}}
\DeclareMathOperator{\Sp}{Sp}
\DeclareMathOperator{\OO}{O}
\DeclareMathOperator{\SO}{SO}
\DeclareMathOperator{\UU}{U}
\DeclareMathOperator{\SU}{SU}
\newcommand{\su}{\mathfrak{su}}
\DeclareMathOperator{\im}{im}
\DeclareMathOperator{\Spec}{Spec}
\DeclareMathOperator{\trace}{trace}
\DeclareMathOperator{\Tr}{Tr}
\DeclareMathOperator{\rank}{rank}
\DeclareMathOperator{\nullity}{nullity}
\DeclareMathOperator{\diag}{diag}
\DeclareMathOperator{\sign}{sign}
\DeclareMathOperator{\Cl}{Cl}
\DeclareMathOperator{\ord}{ord}
\DeclareMathOperator{\Fix}{Fix}
\DeclareMathOperator{\Res}{Res}
\DeclareMathOperator{\res}{res}
\DeclareMathOperator{\Ind}{Ind}
\DeclareMathOperator{\type}{type}
\DeclareMathOperator{\op}{op}
\DeclareMathOperator{\pre}{pre}
\DeclareMathOperator{\sh}{sh}
\DeclareMathOperator{\coker}{coker}
\DeclareMathOperator{\Mor}{Mor}
\DeclareMathOperator{\germ}{germ}
\DeclareMathOperator{\Supp}{Supp}
\DeclareMathOperator{\Proj}{Proj}
\DeclareMathOperator{\ann}{ann}
\DeclareMathOperator{\Tot}{Tot}
\DeclareMathOperator{\cone}{cone}
\DeclareMathOperator{\cyl}{cyl}
\DeclareMathOperator{\Presheaves}{Presheaves}
\DeclareMathOperator{\Sheaves}{Sheaves}
\DeclareMathOperator*{\colim}{colim}
\DeclareMathOperator{\exx}{ex}
\DeclareMathOperator{\htt}{ht}
\DeclareMathOperator{\chr}{char}
\DeclareMathOperator{\Ass}{Ass}
\DeclareMathOperator{\Frac}{Frac}
\DeclareMathOperator{\gr}{gr}
\DeclareMathOperator{\codim}{codim}
\DeclareMathOperator{\Frob}{Frob}
\DeclareMathOperator{\depth}{depth}
\DeclareMathOperator{\pdim}{pdim}
\DeclareMathOperator{\injdim}{injdim}
\DeclareMathOperator{\rad}{rad}
\DeclareMathOperator{\Stab}{Stab}
\DeclareMathOperator{\Ad}{Ad}
\DeclareMathOperator{\ad}{ad}
\DeclareMathOperator{\eval}{eval}
\DeclareMathOperator{\soc}{soc}
\DeclareMathOperator{\II}{I}
\DeclareMathOperator{\Sym}{Sym}
\DeclareMathOperator{\Isom}{Isom}

\newcommand{\trans}{\mathsf{T}}
\newcommand{\halfitem}{\item[\number\numexpr\value{enumi}+0.5\relax.]}
\newcommand{\staritem}{\refstepcounter{enumi}\item[${}^{L}$\theenumi]}
\newcommand{\smatrix}[1]{\left(\begin{smallmatrix} #1 \end{smallmatrix}\right)}


\begin{document}

\begin{center}
\bf Problem Set 5 - pre-REU 2025
\end{center}

\textbf{Problem set on Invariant Theory}

Problems marked with ${}^{L}$ are problems on linear algebra. 

\begin{enumerate}[label=\arabic*., leftmargin=*]
	\item Show that the following sets of matrices form a group under matrix multiplication. 
	In each case, justify why the matrices are indeed invertible.
	\begin{enumerate}[label=(\alph*)]
		\item The set of $n \times n$ matrices with determinant one. 
		%
		\item The set of $n \times n$ matrices $M$ satisfying 
		$M M^{\trans} = I_{n}$.
		%
		\item The set of $2n \times 2n$ matrices $M$ satisfying 
		$M \Omega M^{\trans} = \Omega$, 
		where $\Omega$ is the $2n \times 2n$ block matrix given as 
		$\Omega = \smatrix{0 & I_{n} \\ -I_{n} & 0}$. 
	\end{enumerate}
	%
	\item Let $R$ be a ring (such as the ring of polynomials). 
	Let $f \colon R \to R$ be a function satisfying 
	\begin{itemize}
		\item $f(1) = 1$, 
		\item $f(x + y) = f(x) + f(y)$ for all $x, y \in R$, 
		\item $f(x y) = f(x) f(y)$ for all $x, y \in R$.
	\end{itemize}
	Let $S$ be the set of fixed points of $R$, i.e., 
	$S \coloneqq \{r \in R : f(r) = r\}$. 
	Show that $S$ is closed under addition, multiplication, and contains $1$. 
	%
	\item From class, we know that any element of the orthogonal group looks like
	\begin{equation*} 
		M = 
		\begin{bmatrix}
			\cos(\theta) & - \varepsilon \sin(\theta) \\
			\sin(\theta) & \varepsilon \cos(\theta)
		\end{bmatrix}
	\end{equation*}
	for some $\theta \in \mathbb{R}$ and some $\varepsilon \in \{1, -1\}$. 
	Using this description, check that for the action of $G = \OO_{2}(\mathbb{R})$ on $R = \mathbb{R}\left[x, y\right]$, 
	we have $x^{2} + y^{2} \in R^{G}$. 
	(As before, the action is via $\left[\begin{smallmatrix} x \\ y\end{smallmatrix}\right] \mapsto M \left[\begin{smallmatrix} x \\ y\end{smallmatrix}\right]$.)
	%
	\item Let $G = \GL_{2}(\mathbb{R})$ act on $R = \mathbb{R}[X_{2 \times m}]$ in the usual way. 
	Show that $R^{G} = \mathbb{R}$, i.e., the only invariant polynomials are the constants. 

	\emph{Hint:} Think about what the matrix $\smatrix{2 & 0 \\ 0 & 2}$ does. \newline
	Start out with $m = 1$ or $2$ to get an idea. 
	%
	\staritem Let $A, B$ be $n \times n$ matrices such that 
	% there exists an invertible matrix 
	$\exists P \in \GL_{n}(\mathbb{R})$ with $P A P^{-1} = B$. 
	Show that $A$ and $B$ have the same characteristic polynomial, i.e., 
	show that 
	$\det(A - \lambda I) = \det(B - \lambda I)$. 
	%
	\halfitem Recall that for the action of $G = \GL_{n}(\mathbb{R})$ on $R = \mathbb{R}[X_{n \times n}]$ given by conjugation, 
	% we had that 
	$R^{G}$ is generated by the coefficients of the characteristic polynomial. 
	Interpret the previous problem in this context.
	%
	\staritem Let $A$ be an $n \times m$ matrix with columns $\mathbf{a}_{1}, \ldots, \mathbf{a}_{m} \in \mathbb{R}^{n}$. 
	Show that the $(i, j)$th entry of $A^{\trans} A$ is $\mathbf{a}_{i} \cdot \mathbf{a}_{j}$. 
	%
	\halfitem Recall that for the action of $G = \OO_{n}(\mathbb{R})$ on $R = \mathbb{R}[X_{n \times m}]$ given by left multiplication, 
	$R^{G}$ is generated by the entries of $X^{\trans} X$. 
	Using the previous problem, how does this description of $R^{G}$ fit in with our geometrical definition of $\OO_{n}(\mathbb{R})$? 
	%
	\item Consider the action of $G = \SL_{2}(\mathbb{R})$ on $R = \mathbb{R}[X_{2 \times 2}]$ by left multiplication. For ease of notation, assume that the variables are denoted and arranged as
	\begin{equation*} 
		\begin{bmatrix}
			x_{1} & x_{2} \\
			y_{1} & y_{2} \\
		\end{bmatrix}.
	\end{equation*}
	Show that $x_{1} y_{2} - x_{2} y_{1} \in R^{G}$. 
	More generally, show that if $\SL_{2}$ is acting on the polynomial with variables
	\begin{equation*} 
		\begin{bmatrix}
			x_{1} & x_{2} & \cdots & x_{m} \\
			y_{1} & y_{2} & \cdots & x_{m} \\
		\end{bmatrix},
	\end{equation*}
	then $x_{i} y_{j} - x_{j} y_{i} \in R^{G}$ for all $1 \le i < j \le m$. 

	\emph{Hint for the last part:} If $A$ and $B$ are matrices of compatible sizes, think about how the columns of $A B$ look. In particular, the $i$-th column of $A B$ is the product of $A$ and the $i$-th column of $B$. 
	%
	\halfitem Generalise the previous to $\SL_{n}(\mathbb{R})$ acting on $\mathbb{R}[X_{n \times m}]$. 
\end{enumerate}
	

\end{document}

