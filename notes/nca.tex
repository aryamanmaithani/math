\documentclass[12pt]{article}
\usepackage[lmargin=1in,rmargin=1in,tmargin=1in,bmargin=1in]{geometry}
\usepackage{aryaman}

% \numberwithin{fakethm}{subsection}
\setcounter{tocdepth}{2}

\title{Noncommutative algebra}
\author{Aryaman Maithani}
\date{\today}

\begin{document}

\maketitle
\tableofcontents

\part*{Introduction}

This document is a compilation of facts about noncommutative rings. 
The motivation is to understand facts about finite algebras over a field, 
e.g., $k[G]$ for a finite group $G$. 

The facts are stated without proofs. 
The sequence of the theorems may seem funny. 
These are \emph{not} in logical order. 
For example, we begin the sections on simple and semisimple rings by stating the structure theorem for these rings. 
The abstract facts that follow later would then be easy consequences of the structure theorem. 
However, in practice, one goes the other way around in proving the structure theorem (possibly even using theorems from later subsections). 
The hope is that the organisation here makes it an easy reference to get to the relevant facts. 

We first begin by stating all the relevant definitions in \Cref{part:definitions}. 
It may benefit the reader to only look these up when referred to in \Cref{part:theorems}, where we list the theorems.

\part{Definitions} \label{part:definitions}

Rings throughout will be unital but not necessarily commutative. 
All modules will also be unital. 
We assume familiarity with basic notions of (left/right) artinian and noetherian rings and modules. 
For example, a left artinian ring is left noetherian. 
A module has finite length iff it is both artinian and noetherian.

For this section, $A$ will denote an arbitrary ring. 

Note that a \emph{left} $A$-module $M$ is (by definition) an abelian group $M$ with a multiplication $A \times M \to M$ satisfying certain properties. 
This is precisely the data of an abelian group $M$ and a ring homomorphism
\begin{equation*} 
	\rho \colon A \to \End_{\mathbb{Z}}(M).
\end{equation*}
For this reason, we may think of $\rho$ (or $M$) as a representation of $A$. 
The \deff{annihilator} of $M$ is defined as $\ker(\rho)$ and denoted as $\ann_{A}(M)$.

We have that $\rho(a)$ is the left-multiplication (or \deff{homothety}) map. We denote this by $a_{M}$, i.e.,
\begin{align*} 
	a_{M} \colon M &\to M\\
	m & \mapsto a \cdot m.
\end{align*}
Note that $a_{M} \in \End_{\mathbb{Z}}(M)$ but this is not necessarily an $A$-module homomorphism.

\begin{defn} \label{defn:commutant-bicommutant}
	If $M$ is a left $A$-module, the \deff{commutant} is defined as $A' \coloneqq \End_{A}(M)$, and the \deff{bicommutant} as $A'' \coloneqq \End_{A'}(M)$.

	There is a (well-defined) ring homomorphism $\lambda_{M} \colon A \to A''$ given by $a \mapsto a_{M}$.
\end{defn}
Note that $M$ is a left $A'$-module with the multiplication $A' \times M \to M$ given by $(\varphi, m) \mapsto \varphi(m)$. 
Thus, the bicommutant is well-defined. 
Moreover, each homothety $a_{M}$ (for $a \in A$) is an element of the bicommutant $\End_{A'}(M)$ since $a_{M} \circ \varphi = \varphi \circ a_{M}$ for $a \in A$ and $\varphi \in A'$.

\begin{defn}
	Let $R$ be a commutative ring. 
	An \deff{$R$-algebra} is a (possibly noncommutative) ring $A$ with a ring homomorphism $\rho \colon R \to \operatorname{center}(A)$.
\end{defn}
In particular, whenever we talk about a $k$-algebra (for a field $k$), we have that $k$ lies in the center of $A$. 
Note that even though we may have a natural ring homomorphism $\mathbb{C} \into \mathbb{H}$, 
the division ring $\mathbb{H}$ is \emph{not} a $\mathbb{C}$-algebra.

\begin{defn}
	A (nonzero) module is \deff{simple} if it has exactly two submodules.
\end{defn}
The two submodules are then zero and the module itself.

\begin{defn}
	An $A$-module $M$ is \deff{faithful} if $\ann_{A}(M)$ is zero. 
	Equivalently, if $\lambda_{M}$ is injective.
\end{defn}

\begin{defn} \label{defn:semisimple}
	A module $M$ is \deff{semisimple} if any of the following equivalent conditions hold: 
	\begin{itemize}
		\item every submodule of $N$ has a direct sum complement, i.e., every inclusion $N \into M$ splits;
		\item every short exact sequence of the form $0 \to N \to M \to L \to 0$ splits;
		\item $M$ is the sum of its simple submodules;
		\item $M$ is the direct sum of a family of simple modules.
	\end{itemize} 

	A ring $A$ is \deff{(left) semisimple} if $A$ is a semisimple as a left module over $A$.
\end{defn}
Semisimplicity is a symmetric notion: $A$ is left semisimple iff $A$ is right semisimple.

\begin{defn} \label{defn:simple}
	A (nonzero) ring $A$ is \deff{simple} if $A$ is is semisimple and has only one isomorphism class of simple left ideals.
\end{defn}

The definition above is more restrictive than found in some books. 
Also note that being simple (or even weakly-simple) as a ring is weaker than being simple as a left-module over itself. 
See \Cref{thm:simple-as-a-module}.

\begin{defn} \label{defn:weakly-simple}
	A (nonzero) ring $A$ is \deff{weakly-simple} if $A$ has exactly two two-sided ideals.
\end{defn}
See \Cref{thm:weakly-simple-not-semisimple,thm:simple-equivalences-with-semisimple} for the implications. 

\begin{defn}
	A (nonzero) ring $A$ is \deff{primitive} if $A$ has a faithful simple module. 
\end{defn}

\begin{defn}
	A ring $A$ is \deff{semiprimitive} if for any nonzero $a \in A$, there is an irreducible representation $\rho$ of $A$ such that $\rho(a) \neq 0$.
\end{defn}
In the language of modules: for every $a \neq 0$, there is a simple $A$-module $M$ such that $aM \neq 0$. \newline
This is equivalent to the existence of a faithful \emph{semisimple} module, see \Cref{thm:semi-primtive-equivalence}.

\begin{defn}
	Let $L$ be a left ideal of $A$. The \deff{idealiser} of $L$ is defined as
	\begin{equation*} 
		(L : A) \coloneqq \{a \in A \colon aA \subset L\} = \ann_{A}(A/L).
	\end{equation*} 
\end{defn}
The idealiser is the unique largest two-sided ideal contained in $L$.

\begin{restatable}[]{defn}{jacobsonradical} \label{defn:jacobson-radical}
	Let $A$ be a ring. 
	The \deff{(Jacobson) radical} is a two-sided ideal of $A$, denoted $\rad(A)$, defined as any of the following equivalent objects.
	\begin{enumerate}[label=(\alph*)]
		\item the intersection $\bigcap_{\rho} \ker(\rho)$, where $\rho$ varies over all irreducible representations of $A$;
		\item the intersection $\bigcap_{M} \ann_{A}(M)$, where $M$ is a simple left $A$-module;
		\item the intersection $\bigcap_{L} L$ over all maximal left ideals $L$ of $A$;
		\item the intersection $\bigcap_{R} R$ over all maximal right ideals $R$ of $A$;
		\item the intersection $\bigcap_{P} P$ over all two-sided ideals $P$ such that $A/P$ is a primitive ring;
		\item the set of elements $z$ such that $1 - az$ has a left inverse for every $a \in A$.
	\end{enumerate}	
\end{restatable}

\begin{defn}
	An element $a \in A$ is called \deff{left quasi-regular} if $1 - a$ has a left inverse. 
	A \deff{right quasi-regular} element is defined analogously. 
	An element is \deff{quasi-regular} if it is both left and right quasi-regular. 

	A left ideal is \deff{quasi-regular} if all its elements are left quasi-regular. Analogously, a right ideal is quasi-regular if all its elements are right quasi-regular.
\end{defn}

\begin{defn}
	An ideal $I$ is \deff{nilpotent} if $I^{k} = 0$ for some $k$.
\end{defn}

\begin{defn}
	Let $N$ be a submodule of the $A$-module $M$. We say that
	\begin{enumerate}[label=(\alph*)]
		\item $M$ is an \deff{essential extension} of $N$ if for every \underline{nonzero} submodule $K \subset M$, we have $N \cap K \neq 0$;
		\item $N$ is a \deff{superfluous submodule} of $M$ if for every \underline{proper} submodule $K \subsetneq M$, we have $N + K \neq M$.
	\end{enumerate}
\end{defn}

\begin{defn}
	Let $M$ be an $A$-module. 

	The \deff{socle} of $M$ is defined as
	\begin{align*} 
		\soc(M) &\coloneqq \sum_{N \subset M \text{ simple}} N\\
		&= \bigcap_{N \subset M \text{ essential}} N.
	\end{align*} 

	The \deff{radical} of $M$ is defined as
	\begin{align*} 
		\rad(M) &\coloneqq \bigcap_{N \subset M \text{ maximal}} N \\
		&= \sum_{N \subset M \text{ superfluous}} N.
	\end{align*}

	The convention is that the empty sum is $0$ and the empty intersection is $M$.
\end{defn}
Note that $N \subset M$ is maximal iff $M/N$ is simple.

\begin{defn}
	Let $M$ be an $A$-module. 

	An \deff{injective hull} of $M$ is an essential extension $\iota \colon M \into E$ with $E$ injective.

	A \deff{projective cover} of $M$ is a surjection $\pi \colon P \onto M$ with $P$ projective and $\ker(\pi) \subset P$ a superfluous submodule.
\end{defn}

\begin{defn}
	$A$ is called
	\begin{itemize}
		\item \deff{(left) perfect} if every left module has a projective cover;
		\item \deff{(left) semiperfect} if every finitely generated left module has a projective cover.
	\end{itemize}
\end{defn}

\begin{defn}
	A left \deff{principal indecomposable module (PIM)} of $A$ is an indecomposable direct summand of $A$. 
	Equivalently, an indecomposable, projective, cyclic module.
\end{defn}

\begin{defn} \label{defn:subdirect-product}
	$A$ is said to be a \deff{subdirect product} of $A_{1}, \ldots, A_{k}$  if 
	there is an injective ring homomorphism $\iota \colon A \to \prod_{i} A_{i}$ such that 
	the each induced map $\pi_{i} \circ \iota \colon A \to A_{i}$ is surjective for each $i$.
\end{defn}

\begin{defn}
	A (nonzero) ring is called \deff{local} if its set of nonunits forms a two-sided ideal.
\end{defn}

\begin{defn}
	An $A$-module $M$ is 
	\begin{itemize}
		\item \deff{primordial} if $\End_{A}(M)$ is a local ring;
		\item \deff{semi-primordial} if it is the direct sum of a family of primordial submodules.
	\end{itemize}
\end{defn}
Injectives over a noetherian ring are semi-primordial.

\part{Theorems} \label{part:theorems}

A summary of the main implications is given below.

\begin{equation*} 
	\begin{tikzcd}
	 & \text{simple} \arrow[ld, Rightarrow] \arrow[rd, Rightarrow] && \\
	\text{weakly-simple} \arrow[ru, Rightarrow, "\text{semisimple}", bend left=49] \arrow[d, Rightarrow] && \text{semisimple} \arrow[ldd, Rightarrow, bend left] \arrow[rd, Rightarrow] & \\
	\text{primitive} \arrow[rd, Rightarrow]  & & & \text{artinian} \arrow[lu, "\text{semiprimitive}"', Rightarrow, bend right=49] \arrow[d, Rightarrow] & \\
 & \text{semiprimitive} \arrow[d, Leftrightarrow]  &  & \text{perfect} \arrow[d, Rightarrow] \\
 & \rad = 0 & \text{local} \arrow[r, Rightarrow] & \text{semiperfect}  &
	\end{tikzcd}
\end{equation*}

The situation becomes much better for artinian rings.

\begin{restatable}[]{fakethm*}{artinianringimplications} \label{thm:artinian-ring-implications}
	For the class of \textbf{{\color{red}left artinian}} rings, the following chain of implications hold.
	\begin{equation*} 
		\begin{tikzcd}[ampersand replacement = \&]
			\text{simple} \arrow[rr, Leftrightarrow] \arrow[d, Rightarrow] \&  \& \text{weakly-simple} \arrow[rr, Leftrightarrow] \&  \& \text{primitive} \\
		\text{semisimple} \arrow[rr, Leftrightarrow]  \&  \& \rad = 0 \arrow[rr, Leftrightarrow] \& \& \text{semiprimitive}
		\end{tikzcd}	
	\end{equation*}
\end{restatable}

\section{Implications between some of the properties}

\begin{thm} \label{thm:simple-as-a-module}
	Let $A$ be a ring. The following are equivalent.
	\begin{enumerate}[label=(\alph*)]
		\item $A$ is simple as a left module over $A$.
		\item $A$ is a division ring.
		\item $A$ is simple as a right module over $A$.
	\end{enumerate}
	In such a case, $A$ is a simple ring, a weakly-simple ring, and a semisimple ring.
\end{thm}

\begin{thm} \label{thm:semismimple-implies-artinian}
	A semisimple ring is left and right artinian.
\end{thm}

\begin{thm} \label{thm:weakly-simple-not-semisimple}
	There exists a weakly-simple ring that is not artinian and hence not semisimple.
\end{thm}

\begin{thm} \label{thm:simple-equivalences-with-semisimple}
	Let $A$ be a ring. The following are equivalent:
	\begin{enumerate}[label=(\alph*)]
		\item $A$ is simple.
		\item $A$ is weakly-simple and left artinian.
		\item $A$ is weakly-simple and semisimple.
		\item $A$ is weakly-simple and has exactly one isomorphism class of minimal left ideals.
	\end{enumerate}
	All ``left''s can be replaced with ``right''s as well.
\end{thm}

\begin{thm}
	A weakly-simple ring is primitive.
\end{thm}

\begin{thm}
	A left artinian primitive ring is simple.
\end{thm}

% \begin{equation*} 
% 	\begin{tikzcd}
% 		& \text{weakly-simple} & \\ \\
% 		\text{simple} \arrow[ruu, Rightarrow] \arrow[rr, Rightarrow] \arrow[rdd, Rightarrow] & & \text{artinian} \\ \\
% 		& \text{semisimple} \arrow[ruu, Rightarrow] &                
% 	\end{tikzcd}
% \end{equation*}

% \begin{equation*} 
% 	\begin{tikzcd}
% 	\text{weakly-simple} \arrow[rr, "\text{semisimple}", Rightarrow, bend left=49] \arrow[rr, "\text{left artinian}"', Rightarrow, bend right=49] &  & \text{simple}
% \end{tikzcd}
% \end{equation*}

% \begin{equation*} 
% 	\begin{tikzcd}
% 	    \text{weakly-simple} \arrow[rr, "\text{semisimple}", Rightarrow, bend left=49] \arrow[rr, "\text{left artinian}"', Rightarrow, bend right=49] &  & \text{simple} \arrow[rr, Rightarrow] \arrow[rdd, Rightarrow] \arrow[ll, Rightarrow] &   & \text{artinian} \\
% 	    &  & &   & \\
% 	    &  & & \text{semisimple} \arrow[ruu, Rightarrow] &
% 	\end{tikzcd}
% \end{equation*}

% \begin{equation*} 
% 	\begin{tikzcd}
% 		\text{weakly-simple} \arrow[rr, "\text{semisimple}", Rightarrow, bend left=49] \arrow[rr, "\text{left artinian}"', Rightarrow, bend right=49] &  & \text{simple} \arrow[ll, Rightarrow] \arrow[r, Rightarrow] & \text{semisimple} \arrow[r, Rightarrow] & \text{artinian}
% 	\end{tikzcd}
% \end{equation*}

% \begin{equation*} 
% 	\begin{tikzcd}
% 		\text{primitive} & \text{weakly-simple} \arrow[rr, "\text{semisimple}", Rightarrow, bend left=49] \arrow[rr, "\text{left artinian}"', Rightarrow, bend right=49] \arrow[l, Rightarrow] &  & \text{simple} \arrow[ll, Rightarrow] \arrow[r, Rightarrow] & \text{semisimple} \arrow[r, Rightarrow] & \text{artinian}
% 	\end{tikzcd}
% \end{equation*}


% \begin{equation*} 
% 	\begin{tikzcd}
% 		\text{primitive} \arrow[rrr, "\text{left artinian}", Rightarrow, bend left=67] & \text{weakly-simple} \arrow[rr, "\text{semisimple}", Rightarrow, bend left=49] \arrow[rr, "\text{left artinian}"', Rightarrow, bend right=49] \arrow[l, Rightarrow] &  & \text{simple} \arrow[ll, Rightarrow] \arrow[r, Rightarrow] & \text{semisimple} \arrow[r, Rightarrow] & \text{artinian}
% 	\end{tikzcd}
% \end{equation*}

\section{Simple rings}

We recall again that our definition of simple (\Cref{defn:simple}) is possibly stricter than the one the reader may have seen before (\Cref{defn:weakly-simple}). See \Cref{thm:simple-equivalences-with-semisimple}.

We begin by stating the main theorems.

\begin{thm} \label{thm:simple-ring-is-matrix-ring}
	Every simple ring $A$ is of the form $\End_{D}(V)$, 
	where $D$ is a division ring and $V$ is a finite-dimensional vector space over $D$. 

	In other words, $A$ is a matrix ring $M_{n}(D)$ over a division ring. 
	This $D$ and $n$ is uniquely determined.
\end{thm}
See \Cref{thm:simple-ring-commutant-isomorphism} for some description of $D$ and $V$.

\begin{thm}
	Let $D$ be a division ring, and $V$ a finite-dimensional vector space over $D$. Set $A \coloneqq \End_{D}(V)$. Then, $A$ is a simple ring, $M$ a simple $A$-module, and $D = \End_{A}(V)$.
\end{thm}

\begin{thm}
	A simple ring has exactly one simple module up to isomorphism.

	Specifically, the unique simple module over $M_{n}(D)$ is $D^{n}$.
\end{thm}

\begin{thm}
	Let $A$ be a simple ring. Then, $A$ is a finite direct sum of simple left ideals.

	If $I$ and $J$ are simple left ideals of $A$, then there exists $a \in A$ such that $Ia = J$.
\end{thm}

\begin{thm}
	Let $A$ be a simple ring, $M$ a simple $A$-module and $L$ a simple left ideal of $A$. 
	Then, $LM = M$ and $M$ is faithful.
\end{thm}

\begin{thm} \label{thm:simple-ring-commutant-isomorphism}
	Let $A$ be a simple ring. Then, there exists a faithful simple $A$-module $M$ such that $D \coloneqq \End_{A}(M)$ is a division ring and $A \cong \End_{D}(M)$ via $\lambda_{M} \colon a \mapsto a_{M}$. 

	The module $M$ can be chosen to be any simple left ideal of $A$. 
\end{thm}
Note that $\lambda_{M}$ appears in \Cref{defn:commutant-bicommutant}. 
See also \mynameref{thm:jacobson-density-theorem} and \Cref{cor:left-artin-semisimple-commutant-division-ring} which tells us that $M$ is a finite $D$-module.

\section{Semisimple rings}

\begin{thm} \label{thm:semisimple-ring-equivalences}
	Let $A$ be a ring. The following conditions are equivalent. Parts of the equivalence constitute the Wedderburn--Artin theorem.
	\begin{enumerate}[label=(\alph*)]
		\item $A$ is semisimple.
		\item $A$ is a finite direct product of rings of the form $\End_{D}(V)$ where $D$ is a division ring, and $V$ is a finite-dimensional $D$-vector space.
		\item $A \cong M_{n_{1}}(D_{1}) \times \cdots \times M_{n_{k}}(D_{k})$, where $D_{i}$ are division rings, and $n_{i}$ are positive integers. \newline
		This decomposition is unique.
		\item $A$ is left artinian and semiprimitive.
		\item $A$ is left artinian and $\rad A = 0$
		\item $A$ is left artinian and contains no nonzero nilpotent ideals.
		\item $A$ is left artinian and the subdirect product of weakly-simple rings.
	\end{enumerate}
\end{thm}
See \Cref{defn:subdirect-product} for the definition of subdirect product.

\begin{rem}
	We describe the recovery of the $k$, $D_{i}$, and $n_{i}$: 
	\begin{itemize}
		\item The number $k$ is the number of isomorphism classes of simple \underline{right} $A$-modules.
		\item Let $S_{1}, \ldots, S_{k}$ be a list of the non-isomorphic simple right modules.
		\item We recover $D_{i}$ as the endomorphism ring of $S_{i}$.
		\item The corresponding $n_{i}$ is the number of right ideals in $A$ that are isomorphic to $S_{i}$.
	\end{itemize}
	The reason for choosing right ideals and right modules is that 
	we have $A \cong \End_{A}(A_{A})$, where $A_{A}$ denotes $A$ viewed as a right module over itself. \newline
	If we had worked with left ideals, we would have to consider $D_{i}^{\op}$.
\end{rem}

\begin{thm}[Artin--Wedderburn for $k$-algebras]
	If $A$ is a finite-dimensional semisimple $k$-algebra, 
	then in the decomposition $A \cong \prod_{i} M_{n_{i}}(D_{i})$, 
	each $D_{i}$ is a finite-dimensional division algebra over $k$.

	If $k$ is algebraically closed, then $D_{i} = k$ for all $i$.
\end{thm}
Note that without the algebraic closure hypothesis, we only have that the center of $D_{i}$ is a finite extension of $k$. 
It could be larger than $k$. 
For example, $\mathbb{C}$ as an algebra over $\mathbb{R}$.

\begin{rem}
	A semisimple ring may very well contain nonzero nilpotent \emph{elements}! 
	Indeed, the matrix rings contain many such elements.
\end{rem}

\begin{thm} \label{thm:modules-over-semisimple-rings}
	Any module over a semisimple ring is a semisimple module.
\end{thm}

\begin{thm} \label{thm:semisimple-finitely-many-simples}
	A semisimple ring $A$ has finitely many isomorphism classes of simple modules. 
	Moreover, every simple left $A$-module is isomorphic to a (simple) left ideal.
\end{thm}

\begin{thm}
	Let $A$ be a semisimple ring, $L$ a simple left ideal, and $M$ a simple left module. 
	If $L \not\cong M$ as left modules, then $LM = 0$.
\end{thm}

\begin{thm}
	A semisimple ring $A$ can be written as a finite product of ``simple subrings'' (which are, in fact, two-sided ideals):
	\begin{align*} 
		A = \prod_{i = 1}^{k} A_{i}.
	\end{align*}
	Moreover, if $e_{i}$ is the unit of $A_{i} \subset A$, then $1_{A} = \sum_{i = 1}^{k} e_{i}$ and $A_{i} = A e_{i}$.
\end{thm}
Some clarification is needed here: $A_{i}$ is not really a subring of $A$ since $1 \notin A_{i}$. \newline
What is true is that $A$ is a product of simple rings $A_{i}$, in which case each $A_{i}$ is naturally a subset of $A$. 
Each $A_{i}$ is further then a two-sided ideal generated by an idempotent $e_{i}$ that acts as identity of $A_{i}$.

\section{Weakly-simple rings}

\begin{thm}
	If $A$ is weakly-simple, then every nonzero $A$-module is faithful.
\end{thm}

\begin{thm}[Rieffel]
	Let $A$ be a weakly-simple ring, and $L$ a nonzero left ideal. Let $A' \coloneqq \End_{A}(L)$, $A'' \coloneqq \End_{A'}(L)$, and $\lambda \colon A \to A''$ be as in \Cref{defn:commutant-bicommutant}.

	Then, $\lambda$ is an isomorphism. In particular, $L$ is a faithful $A$-module.
\end{thm}

\begin{thm}[Wedderburn]
	Let $A$ be a ring and $M$ a faithful simple module over $A$. So, $A' \coloneqq \End_{A}(M)$ is a division ring, and $\lambda \colon A \into A''$ makes $A$ a subring of $A''$.

	If $M$ is finite-dimensional over $A'$, then $\lambda$ is an isomorphism, i.e., $A = A''$.
\end{thm}

\section{(Semi)primitive rings}

\begin{thm} \label{thm:semi-primtive-equivalence}
	Let $A$ be a ring. The following statements are equivalent.
	\begin{enumerate}[label=(\alph*)]
		\item $A$ is left semiprimitive.
		\item $A$ is right semiprimitive.
		\item $A$ admits a faithful semisimple module.
		\item $A$ is a finite subdirect product of primitive rings.
		\item $\rad(A) = 0$.
	\end{enumerate}
\end{thm}
See \Cref{defn:subdirect-product} for the definition of subdirect product. See also \Cref{thm:radical-and-semiprimitivity} for a generalisation of the last result. 

\begin{thm}
	Let $A$ be a ring.

	$A$ is left primitive iff $A$ has a maximal left ideal $L$ that contains no non-zero two-sided ideal of $A$. 

	$A$ is (left) semiprimitive iff $\bigcap_{L}(L : A) = 0$, where the intersection is taken over all maximal left ideals of $A$.  
\end{thm}

\begin{cor}
	Let $R$ be a \underline{commutative} ring.

	$R$ is primitive iff $R$ is a field. 

	$R$ is semiprimitive iff $R$ is a subdirect product of fields. 
\end{cor}

\section{The radical}

Since the definition is really a theorem in itself, we state it again.

\jacobsonradical

\begin{thm}
	Let $A$ be a ring. The two notions of $\rad(A)$ coincide: namely the Jacobson radical of $A$ as a ring, and the radical of $A$ as a left-module over $A$.
\end{thm}

\begin{thm}
	The Jacobson ideal $\rad A$ is quasi-regular as a left ideal. \newline
	Moreover, $\rad A$ contains every quasi-regular left ideal.
\end{thm}

\begin{thm}
	Any nilpotent element is left quasi-regular. \newline
	Thus, any nilpotent left ideal is quasi-regular.
\end{thm}

\begin{thm} \label{thm:radical-and-semiprimitivity}
	Let $A$ be a ring, and $I$ a two-sided ideal.
	\begin{itemize}
		\item $A$ is semiprimitive iff $\rad(A) = 0$.
		\item If $A/I$ is semiprimitive, then $\rad(A) \subset I$.
		\item $A/{\rad(A)}$ is a semiprimitive ring. Hence, $\rad(A/{\rad A}) = 0$.
	\end{itemize}
\end{thm}

\begin{thm}
	Let $M$ be an $A$-module, and $N$ a submodule.
	\begin{itemize}
		\item $\rad(M/{\rad M}) = 0$.
		\item $\rad(M/N) = 0$ implies $N \supset \rad M$.
	\end{itemize}
\end{thm}

\begin{thm}
	For any $A$-linear map $f \colon M \to N$, one has $f(\rad M) \subset \rad N$.
\end{thm}

\begin{thm}[Radical and semisimplicity] \label{thm:radical-and-semisimplicity}
	Let $M$ be an $A$-module.
	\begin{itemize}
		\item If $M$ is semisimple, then $\rad(M) = 0$ and $\rad(\End_{A}(M)) = 0$.
		\item If $M$ is artinian and $\rad(M) = 0$, then $M$ is semisimple.
	\end{itemize}
\end{thm}

\begin{thm}
	Let $A$ be a left artinian ring, and $M$ a left $A$-module. 
	Then,
	\begin{equation*} 
		\rad(M) = \rad(A) \cdot M.
	\end{equation*}
\end{thm}

\section{Local rings}

\begin{thm}
	Let $A$ be a ring. Let $A^{\ast}$ denote its set of units. The following are equivalent.
	\begin{enumerate}[label=(\alph*)]
		\item $A$ is local, i.e., $A \setminus A^{\ast}$ is an ideal.
		\item $A \setminus A^{\ast}$ is closed addition.
		\item $A \setminus A^{\ast} \subset \rad(A)$.
		\item $A \setminus A^{\ast} = \rad(A)$.
		\item $A \setminus \rad(A) = A^{\ast}$.
		\item $A$ has a unique maximal right ideal.
		\item $A$ has a unique maximal left ideal.
	\end{enumerate}
\end{thm}

\begin{thm}
	If $A$ is local, then $A$ is semiperfect.
\end{thm}

\begin{thm}[Kaplansky]
	Any projective module over a local ring is free.
\end{thm}

\begin{thm} \label{thm:invertible-or-nilpotent-is-local}
	Let $A$ be a ring in which every element is either invertible or nilpotent. 
	Then, $A$ is local.
\end{thm}

\section{Indecomposable-esque modules and locality}

Recall that a module is primordial if its endomorphism ring is local.
\begin{thm}
	A simple module is primordial. \newline
	A primordial module is indecomposable. \newline
	An indecomposable module of finite length is primordial.
\end{thm}

\begin{equation*} 
	\begin{tikzcd}
	\text{simple} \arrow[rr, Rightarrow] && \text{primordial} \arrow[rr, Rightarrow, bend left=49] &  & \text{indecomposable}  \arrow[ll, "\text{finite length}", Rightarrow, bend left=49]
\end{tikzcd}
\end{equation*}

\begin{lem}[Fitting] 
	Let $M$ be an $A$-module of finite length. 
	Given $u \in \End_{A}(M)$, we have
	\begin{equation*} 
		M = \ker(u^{\infty}) \oplus \im(u^{\infty}).
	\end{equation*}
	Further, $u$ restricted to $\ker(u^{\infty})$ is nilpotent and $u$ restricted to $\im(u^{\infty})$ is an isomorphism.
\end{lem}
To clarify notation: the sequences of submodules $(\ker(u^{N}))_{N}$ and $(\im(u^{N}))_{N}$ both stabilise. 
The stable value is denoted by $\ker(u^{\infty})$ and $\im(u^{\infty})$ respectively.

\begin{cor}
	If $E$ is an indecomposable $A$-module of finite length, then any map in $\End_{A}(E)$ is either nilpotent or invertible. 
	Thus, $\End_{A}(E)$ is local.
\end{cor}

\begin{thm}[Schur]
	Any nonzero map between simple modules is an isomorphism.

	If $E$ is a simple $A$-module, then $\End_{A}(E)$ is a division ring.
\end{thm}

\begin{thm}[Krull--Remak--Schmidt]
	Let $M$ be an $A$-module of finite length. 
	Then, $M$ can be \emph{uniquely} decomposed as a (necessarily finite) direct sum of indecomposable modules.

	Uniqueness means the following: if we have
	\begin{equation*} 
		M \cong E_{1} \oplus \cdots \oplus E_{r} \cong F_{1} \oplus \cdots \oplus F_{s}
	\end{equation*}
	for indecomposable modules $E_{i}$ and $F_{j}$, 
	then $r = s$, 
	and there exists a permutation $\sigma$ of $[r]$ such that 
	$E_{i} \cong F_{\sigma(i)}$.
\end{thm}
\begin{cor}
	Let $A$ be a left artinian ring, and $M$ a finitely generated left $A$-module. 
	Then, $M$ is the finite direct sum of indecomposable modules (in a unique way). 

	In particular, $A$ is the direct sum of principal indecomposable modules.
\end{cor}

Since we cannot have infinite direct sums in a noetherian or artinian module, we still have an existence theorem:
\begin{thm}
	Let $M$ be an $A$-module that is either artinian or noetherian. 
	Then, $M$ is a finite direct sum of indecomposables, not necessarily in a unique way.
\end{thm}

We also have a uniqueness theorem for semi-primordial modules.

\begin{thm}
	Let $A$ be a ring. Suppose we have an isomorphism
	\begin{equation*} 
		E_{1} \oplus \cdots \oplus E_{r} \cong F_{1} \oplus \cdots \oplus F_{s}
	\end{equation*}
	for primordial $A$-modules $E_{i}$ and $F_{j}$. 
	Then, $r = s$, 
	and there exists a permutation $\sigma$ of $[r]$ such that 
	$E_{i} \cong F_{\sigma(i)}$.
\end{thm}

\section{Jacobson Density Theorem}

\begin{thm}[Jacobson Density Theorem] \label{thm:jacobson-density-theorem}
	Let $M$ be a \underline{semisimple} $A$-module. 
	Let $A'$ denote the commutant of $M$, and $A''$ the bicommutant. 
	Let $\lambda \colon A \to A''$ be the ring homomorphism $a \mapsto a_{M}$.

	For any $\psi \in A''$ and any finite subset $S \subset M$, 
	there exists $a \in A$ such that $\psi$ and $a_{M}$ agree on $S$.

	If $M$ is finitely generated over $A'$, then $\lambda$ is surjective.
\end{thm}
\begin{cor} \label{cor:left-artin-semisimple-commutant-division-ring}
	Let $A$ be a left artinian ring, and $M$ a semisimple $A$-module. 
	If $A' \coloneqq \End_{A}(M)$ is a division ring (e.g., $M$ is simple), 
	then $M$ is a finite-dimensional vector space over $A'$. 
	Hence, $\lambda(A) = A''$.
\end{cor}
\begin{cor}
	Let $V$ be a finite-dimensional vector space over an algebraically closed field $k$, 
	and $A$ a subalgebra of $\End_{k}(V)$. 

	If $V$ is a simple $A$-module, then $A = \End_{k}(V)$.
\end{cor}
\begin{cor}
	Let $V$ be a finite-dimensional vector space over an algebraically closed field $k$, 
	and $G$ a submonoid of $\GL(V)$. 

	If $V$ is a simple $G$-module, then $k[G] = \End_{k}(V)$.
\end{cor}
Note that here $k[G]$ is the subalgebra of $\End_{k}(V)$ generated by $G \subset \GL(V) \subset \End_{k}(V)$.

\section{Projective modules}

Throughout this section, 
$A$ will denote a ring, 
$J \coloneqq \rad(A)$ the Jacobson radical, 
set $\overline{A} \coloneqq A/J$, 
and $\overline{M} \coloneqq M/JM$ for a left $A$-module $M$. 
This makes $\overline{(-)}$ an additive functor in the obvious way. 
All modules are considered to be left modules. 

In particular, given $A$-modules $M$ and $N$, we have a map of abelian groups
\begin{equation*} 
	\Hom_{A}(M, N) \to \Hom_{\overline{A}}(\overline{M}, \overline{N})
\end{equation*}
and a map of rings
\begin{equation*} 
	\End_{A}(M) \to \End_{\overline{A}}(\overline{M}).
\end{equation*}

We study when these maps are injective or surjective.

\begin{thm}
	Let $P$ be a projective $A$-module, and $N$ an arbitrary $A$-module.

	Then, $\overline{P}$ is a projective $\overline{A}$-module, 
	and $\Hom_{A}(P, N) \onto \Hom_{\overline{A}}(\overline{P}, \overline{N})$ is onto.
\end{thm}

\begin{thm}
	Suppose $A$ is (left) artinian, and $P$ is a projective $A$-module. Then, the map $\End_{A}(P) \onto \End_{\overline{A}}(\overline{P})$ induces an isomorphism
	\begin{equation*} 
		\frac{\End_{A}(P)}{\rad(\End_{A}(P))} \cong \End_{\overline{A}}(\overline{P}).
	\end{equation*}
\end{thm}
\begin{cor} \label{cor:artinian-projective-isomorphic-mod-radical}
	Suppose $A$ is artinian, and $P, Q$ are projective $A$-modules. Then,
	\begin{center}
		$P \cong Q$ $\Leftrightarrow$ $\overline{P} \cong \overline{Q}$.
	\end{center}
\end{cor}

\begin{thm} \label{thm:artinian-PIM-criteria}
	Suppose $A$ is artinian, and $P$ is a direct summand of $A$ as a left module. Then,
	\begin{center}
		$P$ is a PIM $\Leftrightarrow$ $P$ is indecomposable $\Leftrightarrow$ $\overline{P}$ is indecomposable $\Leftrightarrow$ $\overline{P}$ is simple.
	\end{center}
\end{thm}
Note that $\overline{A}$ is semisimple in the above situation, giving the last $\Leftrightarrow$. 

\section{Superfluous extensions and (semi)perfect rings}

\begin{thm}
	The zero submodule is always superfluous, and a nonzero module is never a superfluous submodule of itself. 
\end{thm}

\begin{thm}[Nakayama's lemma]
	Let $M$ be a finitely generated left $A$-module. 
	Then, $\rad(A)M$ is a superfluous submodule of $M$.
\end{thm}

\begin{thm}
	Injective hulls and projective covers are unique up to isomorphism.

	Injective hulls always exist, but projective covers may not exist.
\end{thm}

\begin{thm}
	If $M$ is a projective module, then its projective cover is $M$.
\end{thm}

\begin{thm}
	Let $A$ be a semiprimitive ring (i.e., $\rad(A) = 0$), and $M$ an $A$-module. 

	$M$ has a projective cover iff $M$ is projective.
\end{thm}

\begin{thm}
	Any left artinian ring is right-and-left perfect. 

	Any local ring is right-and-left semiperfect.
\end{thm}

\begin{thm}[Characterisation of perfect rings]
	Let $A$ be a ring. The following are equivalent.
	\begin{enumerate}[label=(\alph*)]
		\item $A$ is \underline{left} perfect.
		\item Every left $A$-module has a projective cover. 
		\item $A$ satisfies the descending chain condition on principal \underline{right} ideals.
	\end{enumerate}
\end{thm}

\begin{thm}[Characterisation of semiperfect rings]
	Let $A$ be a ring. The following are equivalent.
	\begin{enumerate}[label=(\alph*)]
		\item $A$ is left semiperfect.
		\item $A$ is right semiperfect.
		\item Every finitely generated left $A$-module has a projective cover. 
		\item Every simple left $A$-module has a projective cover. 
		\item $A/{\rad A}$ is semisimple and idempotents lift modulo $\rad A$.
	\end{enumerate}
\end{thm}

\begin{thm}
	Over a semiperfect ring, every indecomposable projective module is a PIM, 
	and every finitely generated projective module is a direct sum of PIMs.
\end{thm}

\section{Artinian rings}

We restate the theorem from earlier.

\begin{thm}
	\artinianringimplications
\end{thm}

\begin{thm}
	Let $A$ be left artinian. Then, $\rad A$ is nilpotent.
\end{thm}

\begin{thm}
	Let $A$ be left artinian. Then, $A/{\rad A}$ is semisimple.
\end{thm}
\begin{sketch}
Use \Cref{thm:radical-and-semiprimitivity,thm:semisimple-ring-equivalences}.
\end{sketch}

\begin{thm} \label{thm:radical-module-over-artinian}
	Let $A$ be a left artinian ring, and $M$ a left $A$-module. 
	Then,
	\begin{equation*} 
		\rad(M) = \rad(A) \cdot M.
	\end{equation*}
\end{thm}

\begin{thm}
	If $A$ is left artinian, then $A$ is right-and-left perfect.
\end{thm}

\begin{thm}
	If $A$ is left artinian, then any finitely generated left $A$-module is the direct sum of finitely many indecomposables. 

	In particular, $A$ is a finite direct sum of PIMs.
\end{thm}
\begin{thm}
	Over an artinian ring, every indecomposable projective module is a PIM, 
	and every finitely generated projective module is a direct sum of PIMs.
\end{thm}
\begin{sketch}
	Let $P$ be a projective indecomposable. 
	Write $P \oplus Q = F$ for some free module $F$.
	Since the ring is artinian, we have that $F \cong \bigoplus_{i} P_{i}$ for PIMs $P_{i}$. 
	Thus, $P \oplus Q = \bigoplus_{i} P_{i}$. 
	By uniqueness of decomposition, we see that $P \cong P_{i}$ for some $i$, i.e., 
	$P$ is a PIM.
\end{sketch}

\begin{thm}
	Let $A$ be left artinian. 
	Every nonzero homomorphic image of an indecomposable left projective module is again indecomposable.

	That is, if $P \onto N$ with $P$ an indecomposable project (not necessarily a PIM), then $N$ is indecomposable (or zero).
\end{thm}

\begin{thm}
	Let $A$ be left artinian and local.
	Then, (the left module) $A$ is the only PIM.
\end{thm}

\begin{thm}
	If $A$ is a left artinian ring, then $A$ has finitely many simple left $A$-modules (up to isomorphism). 
	More precisely, setting $J \coloneqq \rad(A)$, one has a bijection
	\begin{equation*} 
		\left\{
			\begin{array}{c}
				\text{isomorphism classes of}\\
				\text{simple $A$-modules}
			\end{array}
		\right\}
		\leftrightarrow 
		\left\{
			\begin{array}{c}
				\text{isomorphism classes of}\\
				\text{simple $A/J$-modules}
			\end{array}
		\right\}.
	\end{equation*}
\end{thm}
\begin{sketch}
	If $M$ is a simple $A$-module, then $\rad(A) \subset \ann_{A}(M)$, i.e., $\rad(A) \cdot M = 0$. 
	% Thus, $M$ is a $A/{\rad(A)}$-module. 

	This gives us the isomorphism-preserving bijection. 
	But $A/{\rad A}$ is a semisimple ring, so 
	\Cref{thm:semisimple-finitely-many-simples} gives us the finiteness.
\end{sketch}

\begin{thm}
	If $A$ is a left artinian ring, and $J \coloneqq \rad(A)$ its Jacobson radical, then the map $P \mapsto P/JP$ induces a bijection
	\begin{equation*} 
		\left\{
			\begin{array}{c}
				\text{isomorphism classes of}\\
				\text{principal indecomposable $A$-modules}
			\end{array}
		\right\}
		\leftrightarrow
		\left\{
			\begin{array}{c}
				\text{isomorphism classes of}\\
				\text{simple modules (over $A$ or $A/J$)}
			\end{array}
		\right\}.
	\end{equation*}
	In particular, the left set is finite. 
	The module $P/JP$ is called the \deff{head} of $P$. 
	The other direction is given by mapping a simple $A/J$-module to its projective cover (viewed as an $A$-module).
\end{thm}
\begin{sketch}
	\Cref{thm:artinian-PIM-criteria} tells us that $P/JP$ is indeed simple when $P$ is a PIM. 
	\Cref{cor:artinian-projective-isomorphic-mod-radical} shows that the map $P \mapsto P/JP$ is injective.

	In the other direction, let $S$ be a simple $A$-module, and $\pi \colon P \onto S$ its projective cover. 
	Write $P = \bigoplus_{i} P_{i}$ as a sum of PIMs. 
	The induced maps $P_{i} \into P \onto S$ cannot all be zero. 
	If $P_{1} \to S$ is nonzero, then $P_{1}$ is a projective cover of $S$. 
	By uniqueness, $P = P_{1}$ is a PIM.

	This shows that we do have a well-defined map in the backwards direction.

	Moreover, $P \onto S$ induces $\overline{P} \onto \overline{S} = S$. 
	Since $\overline{P}$ is simple, this map is an isomorphism. 

	This shows that (going mod $J$) is a left inverse for (taking projective cover). 
	Since the former is known to be injective, this finishes the proof.
	% ???
\end{sketch}

\section{Representation theory}

For a finite group $G$ and a field $k$, we let $k[G]$ denote the group algebra. 
This is a $k$-algebra of dimension equal to $\md{G}$. 

Recall that a \deff{representation} of $G$ on a $k$-vector space $V$ is a group homomorphism $\rho \colon G \to \GL(V)$. 
This is precisely the same data as a $k[G]$-module $V$. 

Note that $k[G]$ is (left and right) artinian, being finite-dimensional over a field. 
The \deff{modular} case refers to the situation when $\chr(k) \mid \md{G}$.

Recall that we have the \deff{augmentation map}
\begin{align*} 
	\varepsilon \colon k[G] & \to k \\
	\sum_{g \in G} a_{g} g & \mapsto \sum_{g \in G} a_{g}.
\end{align*}

\begin{thm}
	The group algebra $k[G]$ is semisimple iff $\chr(k) \nmid \md{G}$.
\end{thm}
\begin{sketch}
	\forward Suppose $\chr(k) \mid \md{G}$. 
	Consider $z \coloneqq \sum_{g \in G} g$. 
	Then, $kz$ is an ideal of $k[G]$. 
	We have $z^{2} = \md{G}z = 0$. 
	Thus, $kz$ is a nonzero nilpotent ideal. 
	Now apply \Cref{thm:semisimple-ring-equivalences}.

	\backward Suppose $\chr(k) \nmid \md{G}$. 
	Any inclusion $W \into V$ of $k[G]$-modules has a $k$-linear splitting $\pi \colon V \to W$. 
	This gives rise to the $k[G]$-linear splitting given by 
	\begin{equation*} 
		\frac{1}{\md{G}} \sum_{g \in G} g_{V} \circ \pi \circ g_{V}^{-1}. \qedhere
	\end{equation*}
\end{sketch}

Thus, we get the following.
\begin{thm}
	Let $k$ be a field and $G$ a finite group such that $\chr(k) \nmid \md{G}$. 
	Then, 
	\begin{itemize}
		\item the indecomposable and irreducible representations of $G$ coincide;
		\item there are finitely many irreducible representations of $G$;
		\item every representation decomposes as a direct sum of irreducible representations of $G$.
	\end{itemize}
\end{thm}

Regardless, since $k[G]$ is artinian even in the modular case, we have the following.
\begin{thm}
	Let $k$ be a field, and $G$ be any finite group. 
	Then,
	\begin{itemize}
		\item the ring $k[G]$ has finitely many simple modules;
		\item the group $G$ has finitely many irreducible representations over $k$;
		\item any finite-dimensional representation is (uniquely) the direct sum of finitely many indecomposable representations;
		\item every $k[G]$-module has a projective cover;
		\item the projective cover of each simple module is indecomposable and cyclic, i.e., a PIM;
		\item if $J \coloneqq \rad(k[G])$, then we have a one-to-one correspondence of isomorphism classes
		\begin{align*} 
			\{\text{PIMs}\} & \leftrightarrow \{\text{simple modules}\} \\
			P & \mapsto P/JP,
		\end{align*}
		with the inverse being given by taking projective covers.
	\end{itemize}
\end{thm}
In the modular case, the issue is that there exist representations that cannot be written as a (direct) sum of irreducible representations.

\begin{thm}
	Let $k$ be a field of characteristic $p > 0$, 
	and $G$ a finite group.

	The ring $k[G]$ is local iff $G$ is a $p$-group. 

	In this case, the ideal of nonunits, which is equal to the Jacobson radical, is given by
	\begin{equation*} 
		\rad(k[G]) = \left\{\sum_{g} a_{g} g : \sum_{g} a_{g} = 0\right\} = \ker(\varepsilon).
	\end{equation*}
\end{thm}

\section{Symmetric and asymmetric properties}

We list the properties we have considered in this document and note whether they are symmetric. 
A property $P$ of a ring is \deff{symmetric} if the following is true: $A$ has $P$ iff $A^{\op}$ has $P$. \newline
Equivalently, since our properties are often listed stated as ``$A$ is left $P$'', being symmetric is saying: $A$ is left $P$ iff $A$ is right $P$.

\begin{enumerate}[label=(\alph*)]
	\item Artinian: not symmetric.
	\item Noetherian: not symmetric.
	\item Finite length: not symmetric. \newline
	Moreover, even if $A$ has finite length as a left and right module, these lengths could be different.
	\item Local: symmetric by definition.
	\item Semisimple: symmetric by classification theorem. 
	\item Simple: symmetric by classification theorem. 
	\item Weakly-simple: symmetric by definition.
	\item Semiprimitive: symmetric since characterised by $\rad = 0$.
	\item Primitive: \textbf{not} symmetric! 
	\item Semiperfect: symmetric.
	\item Perfect: \textbf{not} symmetric! 
\end{enumerate}

\end{document}