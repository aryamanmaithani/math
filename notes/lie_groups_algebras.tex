\documentclass[12pt]{article}
\usepackage[lmargin=1in,rmargin=1in,tmargin=1in,bmargin=1in]{geometry}
\usepackage{aryaman}

\setcounter{tocdepth}{2}

\title{Lie Groups and Lie Algebras}
\author{Aryaman Maithani}
\date{Fall 2024}

\begin{document}

\maketitle
\tableofcontents

\section*{Introduction}

Notes I'm making for my Lie groups course. 

\section{Prerequisite definitions, notations, and conventions}

The words ``smooth'' and ``differentiable'' will be used interchangeably, referring to the object being $\mathcal{C}^{\infty}$. 
We assume that one has the notion of what this means when talking about maps between open subsets of $\mathbb{R}^{n}$. 
We recall some notions of manifolds, mainly for the purpose of setting up the notation. \newline
The definition of a submanifold may be different than usual (see \Cref{rem:submanifold}).

\subsection{Manifolds}

Let $M$ be a topological space. A \deff{chart} $c$ on $M$ is a triple $(U, \varphi, n)$ where
\begin{itemize}
	\item $n \in \{0, 1, 2, \ldots\}$,
	\item $U \subset M$ is open,
	\item $\varphi \colon U \to \mathbb{R}^{n}$ is a continuous function that is a homeomorphism onto its image.
\end{itemize}
$U$ is the \deff{domain} of the chart, and $n$ the \deff{dimension}. 

The charts $(U, \varphi, n)$ and $(V, \psi, m)$ are \deff{compatible} if either
\begin{itemize}
	\item $U \cap V = \emptyset$, or
	\item $U \cap V \neq \emptyset$ and $\psi \circ \left(\varphi^{-1}|_{\varphi(U \cap V)}\right)$ is a diffeomorphism from $\varphi(U \cap V)$ onto $\psi(U \cap V)$.
\end{itemize}
The last condition would imply $m = n$. 

An \deff{atlas} on $\mathcal{M}$ is a collection of compatible of charts such that the domains cover $M$. 
This leads to the notion of a maximal (or saturated) atlas, which always exists. 

\begin{defn}
	A \deff{differentiable (or smooth) manifold} is a tuple $(M, \mathcal{A})$ , where $M$ is a Hausdorff topological space and $\mathcal{A}$ is a saturated atlas on $M$.
\end{defn}
By our definition, $\mathbb{R}$ with the discrete topology is a manifold which is zero-dimensional.

As noted earlier, if $x \in M$, then any chart containing $x$ has the same dimension. 
We denote this dimension by $\dim_{x}(M)$. 
One checks that the map $x \mapsto \dim_{x}(M)$ is locally constant. 

If $M$ is a smooth manifold, then every $x \in M$ has a connected neighbourhood. 
Thus, the connected components of $M$ are open. 
On the other hand, the connected components of any topological space are closed. Thus, we get:

\begin{thm} \label{thm:manifold-components-clopen}
	Let $M$ be a smooth manifold. Then, the connected components of $M$ are both open and closed.
\end{thm}
We only needed a topological manifold for the above.

Let $M$, $N$ be smooth manifolds, and $f \colon M \to N$ a function. 
Suppose that for all charts $(V, \psi, m)$ on $N$ and $(U, \varphi, n)$ on $M$ such that $f(U) \subset V$ we have that the composition
\begin{equation*} 
	\varphi(U) \xrightarrow{\varphi^{-1}} U \xrightarrow{f|_{U}} V \xrightarrow{\psi} \psi(V)
\end{equation*}
is a smooth map. 
Then, $f$ is said to be \deff{smooth}.

This gives us the category of (smooth) manifolds. There is a natural product in this category. 

\subsection{Tangent spaces} \label{subsec:tangent-spaces}

For the most part, we only need to know that there is a functor from the category of manifolds with basepoints to the category of vector spaces. 
More precisely: given a manifold $X$ and $x \in X$, there is an associated vector space $T_{x}(X)$ of dimension $\dim_{x}(X)$. 
To each smooth function $\varphi \colon X \to Y$ we associate a linear map
\begin{equation*} 
	T_{x}(\varphi) \colon T_{x}(X) \to T_{\varphi(x)}(Y).
\end{equation*}

This association satisfies 
\begin{itemize}
	\item $T_{x}(\id_{X}) = \id_{T_{x}(X)}$ for all $x \in X$, and
	\item given $X \xrightarrow{\varphi} Y \xrightarrow{\psi} Z$, we have
	\begin{equation*} 
		T_{x}(\psi \circ \varphi) = T_{\varphi(x)}(\psi) \circ T_{x}(\varphi).
	\end{equation*}
\end{itemize}
We have the natural identification of tangent spaces as
\begin{equation*} 
	T_{(x, y)}(X \times Y) \cong T_{x}(X) \oplus T_{y}(Y).
\end{equation*}
The projections $X \times Y \to X, Y$ induce the corresponding projections on tangent spaces under the above identification.

We give one concrete realisation of the tangent spaces: 
Let $\mathcal{C}^{\infty}(X)$ denote the space of all smooth maps $X \to \mathbb{R}$. 
Fix $x \in X$.
A linear map $\partial \colon C^{\infty}(X) \to \mathbb{R}$ is said to be a \deff{derivation at $x$} if 
\begin{equation*} 
	\partial(fg) = \partial(f) g(x) + f(x) \partial(g)
\end{equation*}
for all $f, g \in C^{\infty}(M)$.

Then,
\begin{equation*} 
	T_{x}(X) = \text{space of all derivations at $x$}.
\end{equation*}
The above is naturally an $\mathbb{R}$-vector space. (Maybe not clear why this has the correct dimension.)

\begin{ex}
	Let us check what $T_{1}(\varphi)$ is under this description. 
	Let $\varphi \colon X \to Y$ be smooth, fix $x \in X$, and set $y \coloneqq \varphi(x)$. 

	We wish to describe the linear map
	\begin{equation*} 
		T_{x}(\varphi) \colon T_{x}(X) \to T_{y}(Y).
	\end{equation*}

	Let $\partial \in T_{x}(X)$ be a derivation at $x$. 
	Given $f \in \mathcal{C}^{\infty}(Y)$, we define
	\begin{equation*} 
		T_{x}(\varphi)(\partial)(f) \coloneqq \partial(f \circ \varphi) \in \mathbb{R}.
	\end{equation*}
	The right hand side makes sense since $f \circ \varphi$ is a smooth map $X \to \mathbb{R}$. 

	Thus, we have described a function $T_{x}(\varphi)(\partial) \colon \mathcal{C}^{\infty}(M) \to \mathbb{R}$. 
	One checks that this is a linear function.

	We check that this is a derivation at $y$. To this end, given $f, g \in C^{\infty}(Y)$, we have
	\begin{align*} 
		T_{x}(\varphi)(\partial)(f \cdot g) &= \partial((fg) \circ \varphi) \\
		&= \partial((f \circ \varphi) \cdot (g \circ \varphi)) \\
		&= \partial(f \circ \varphi) \cdot (g \circ \varphi)(x) + (f \circ \varphi)(x) \cdot \partial(g \circ \varphi)) \\ 
		&= \partial(f \circ \varphi) g(y) + f(y) \partial(g \circ \varphi)). 
	\end{align*}
	This shows that we do get a derivation.

	The functoriality is clear.
\end{ex}

\begin{ex}
	The identification
	\begin{equation*} 
		T_{(x, y)}(X \times Y) \cong T_{x}(X) \oplus T_{y}(Y)
	\end{equation*}
	is done as follows:

	Given $\partial_{x} \in T_{x}(X)$ and $\partial_{y} \in T_{y}(Y)$, we define the derivation $\partial \in T_{(x, y)}(X, Y)$ as
	\begin{equation*} 
		\partial(f) \coloneqq \partial_{x}(f(-, y)) + \partial_{y}(\varphi(x, -)),
	\end{equation*}
	for $f \in \mathcal{C}^{\infty}(X \times Y)$.

	This association gives us a map
	\begin{equation*} 
		T_{x}(X) \oplus T_{y}(Y) \to T_{(x, y)}(X, Y).
	\end{equation*}
	One checks that this is an isomorphism.
\end{ex}

\subsection{Submanifold}

\begin{defn} \label{defn:submanifold}
	Suppose $M$ is a manifold, and $N \subset M$ a subset. $N$ is a \deff{submanifold} of $M$ if $N$ has the following property: for every $x \in N$, there exists a chart $(U, \varphi, n)$ with $x \in U$ such that $\varphi(U \cap N) = \varphi(U) \cap \mathbb{R}^{\ell}$. 
\end{defn}
\begin{rem} \label{rem:submanifold}
	In the above, we assume $\mathbb{R}^{\ell} \subset R^{n}$ in the natural canonical way. \newline
	For a chart as above, we get a chart $(U \cap N, \varphi|_{U \cap N}, \ell)$ on $N$. 
	The collection of all such charts give us an atlas on $N$.

	This is a strong definition: this implies that $N$ is locally closed in $M$.
\end{rem}

The above definition implies that the natural inclusion $\iota \colon N \to M$ is smooth since this locally looks like the natural inclusion $\mathbb{R}^{\ell} \into \mathbb{R}^{n}$.
We then have the following criterion then to check smoothness of maps to submanifolds:
\begin{thm} \label{thm:smoothness-into-submanifold}
	Let $\iota \colon N \into M$ be as above, $P$ a smooth manifold, and $f \colon P \to N$ an arbitrary function. \newline
	Let $\widetilde{f} \colon P \to M$ be the composition $P \to N \into M$. Then,
	\begin{center}
		$f$ is smooth $\Leftrightarrow$ $\widetilde{f}$ is smooth.
	\end{center}
\end{thm}

We also note that given $x \in N$, the inclusion $\iota$ induces an inclusion of tangent spaces and we can identify $T_{x}(N)$ as a subspace of $T_{x}(M)$.

\subsection{Vector spaces as manifolds} \label{subsec:vector-spaces-manifolds}

Let $V$ be a (finite-dimensional) vector space (over $\mathbb{R}$) of dimension $n$. 
After fixing a basis for $V$, we have a bijection $V \cong R^{n}$. 
Using this bijection, we can define an atlas on $V$. 
This atlas is independent of the basis chosen. 
Thus, $V$ is an $n$-dimensional manifold. 

Given $v \in V$, the tangent space $T_{v}(V)$ can be naturally identified with $V$.

\begin{rem}
	The identification 
	\begin{equation*} 
		V \cong T_{v}(V)
	\end{equation*}
	is as follows: given $X \in V$, we define the map $\partial_{X} \colon \mathcal{C}^{\infty}(V) \to \mathbb{R}$ as
	\begin{equation*} 
		\partial_{X}(f) \coloneqq \lim_{t \to 0} \frac{f(v + tX) - f(v)}{t}.
	\end{equation*}
	The map $X \mapsto \partial_{X}$ is the identification.
\end{rem}

Since linear maps $\mathbb{R}^{n} \to \mathbb{R}^{m}$ are smooth, we get any linear map $\varphi \colon V \to W$ between vector spaces is smooth. 
Moreover, under the identifications of tangent spaces with the vector space itself, the differential $T_{v}(\varphi)$ is again $\varphi$. 

Indeed, one checks that $T_{v}(\partial_{X}) = \partial_{\varphi(X)} \in T_{\varphi(v)}(Y)$.

Thus, the derivative at $v$ of
\begin{equation*} 
	\varphi \colon V \to W
\end{equation*}
is again itself.

Next, suppose we are given a bilinear map $\varphi \colon V \times V \to W$. 
Taking its differential at $(x, y) \in V \times V$ gives us a \emph{linear} map
\begin{equation*} 
	T_{(x, y)}(\varphi) : T_{(x, y)}(V \times V) \to T_{\varphi(x, y)}(W).
\end{equation*}
Under the two identifications mentioned before, the above is a linear map
\begin{equation*} 
	V \oplus V \to W.
\end{equation*}
Under these identifications the linear map turns out to be
\begin{equation} \label{eq:product-rule-bilinear-map}
	(\xi, \eta) \mapsto \varphi(x, \eta) + \varphi(\xi, y)
\end{equation}
for $\xi \in V \cong T_{v}(X)$ and $\eta \in V \cong T_{y}(V)$.

\section{Lie groups}

We start with the basic definitions.

\subsection{Definition}

\begin{defn}
	A \deff{Lie group} is a smooth manifold $G$ equipped with smooth functions $m \colon G \times G \to G$ and $\iota \colon G \to G$ such that $(G, m, \iota)$ is a group.

	A \deff{morphism} of Lie groups is a smooth group homomorphism. 

	This gives us the category $\LieGrp$.
\end{defn}
Recall that $(G, m, \iota)$ being a group means the obvious thing: $m$ is associative, unital, and every element has an inverse; 
the map $\iota$ is the inversion map $g \mapsto g^{-1}$.

We will often denote the identity of $G$ by $1$ or $e$. We may include a subscript if necessary.

\begin{ex}
	$\GL(n, \mathbb{R})$ is a Lie group: this has a natural smooth structure since this is an open subset of $M_{n}(\mathbb{R}) \cong \mathbb{R}^{n^{2}}$. 
	The usual multiplication map is a polynomial in the coordinates and hence smooth. 
	Inversion is smooth as well, being a rational function.
\end{ex}

\begin{rem}
	We will see (\Cref{cor:inversion-automatically-smooth}) that one need not require $\iota$ to be smooth.
\end{rem}

We also have the following full subcategories of $\LieGrp$:
\begin{itemize}
	\item $\CLieGrp$, the category of \emph{connected} Lie groups;
	\item $\SCLieGrp$, the category of \emph{simply-connected} Lie groups.
\end{itemize}

\subsection{Natural maps}

Given $g \in G$, we get the following natural maps on $G$:
\begin{itemize}
	\item The \emph{left translation} $\gamma(g) \colon G \to G$ defined by $h \mapsto gh$.
	\item The \emph{conjugation} $\INT(g) \colon G \to G$ defined by $h \mapsto g h g^{-1}$.
\end{itemize}

Both of these maps are diffeomorphisms; the second is also a (iso)morphism of Lie groups.

\subsection{Basic properties of Lie groups}

\begin{thm}
	If $G$ is a group, then the function $g \mapsto \dim_{g}(G)$ is constant. 
	In other words, a Lie group has the same dimension at every point.
\end{thm}

\begin{defn}
	The \deff{dimension} of a Lie group is defined as $\dim_{1_{G}}(G)$.
\end{defn}

\begin{defn}
	The connected component of $1$ is denoted by $G_{0}$; this is a Lie group.
\end{defn}
The smooth structure on $G_{0}$ follows from \Cref{thm:manifold-components-clopen}. 
That this is a subgroup follows from the fact that left multiplication by an element is continuous, as is inversion. 
Thus, these maps preserve components. 
By the same reasoning, we get the following:

\begin{thm}
	$G_{0}$ is an open, normal subgroup of $G$.

	If $\varphi \colon G \to H$ is a morphism of Lie groups, then $\varphi(G_{0}) \subset H_{0}$.
\end{thm}

Thus, we get a functor
\begin{equation*} 
	\LieGrp \xrightarrow{(-)_{0}} \CLieGrp.
\end{equation*}
We also have the inclusion/forgetful functor
\begin{equation*} 
	\CLieGrp \xrightarrow{F} \LieGrp.
\end{equation*}

If $G$ is a connected Lie group, and $H$ an arbitrary Lie group, 
then we have a natural equality of sets
\begin{equation*} 
	\Hom_{\LieGrp}\left(F(G), H\right) = \Hom_{\CLieGrp}(G, H_{0})
\end{equation*}
since $G = G_{0}$. 

The above tells us that
\begin{thm}
	The forgetful functor and the connected-component functors are adjoints.
\end{thm}

\begin{defn}
	A neighbourhood $U$ of $1$ is said to be \deff{symmetric} if $\iota(U) \subset U$.
\end{defn}
Every neighbourhood $U$ contains a symmetric neighbourhood: for example, $U \cap U^{-1}$.

\subsection{Differentials of some maps}

We can consider the differential of $m \colon G \times G \to G$. 
Under the natural identification, this gives us a map
\begin{equation*} 
	T_{(1, 1)}(m) \colon T_{1}(G) \oplus T_{1}(G) \to T_{1}(G).
\end{equation*}
One might think that this is a useful map giving information about the Lie group. Alas,
\begin{thm}[The differential of the multiplication map]
	For $X, Y \in T_{1}(G)$, one has
	\begin{equation*} 
		T_{(1, 1)}(m)(X, Y) = X + Y.
	\end{equation*}
\end{thm}
In particular, the above is the ``same'' map for any Lie group.
\begin{sketch}
	Consider the maps $\begin{tikzcd} \iota_{1}, \iota_{2} \colon M \arrow[r, shift left] \arrow[r, shift right] & M \times M \end{tikzcd}$ defined by $m \mapsto (m, 1)$ and $m \mapsto (1, m)$. 

	Then, $T_{1}(\iota_{\ast})$ is the corresponding inclusion map $T_{1}(M) \to T_{1}(M) \oplus T_{1}(M)$. 
	We also have $m \circ i_{\ast} = \id_{M}$. \newline
	Using the chain rule gives us
	\begin{equation*} 
		T_{(1, 1)}(X, 0) = X \andd T_{(1, 1)}(0, Y) = Y. \qedhere
	\end{equation*}
\end{sketch}

\begin{thm}[The differential of the inversion map]
	For $X \in T_{1}(G)$, one has
	\begin{equation*} 
		T_{1}(\iota)(X) = -X.
	\end{equation*}
\end{thm}
\begin{sketch}
	Note that the composition
	\begin{equation*} 
		G \xrightarrow{\id_{G} \times \iota} G \times G \xrightarrow{m} G
	\end{equation*}
	is constant. Thus, its differential is zero. 
	Differentiating using the chain tells us that
	\begin{equation*} 
		T_{1}(G) \xrightarrow{\id \oplus T_{1}(\iota)} T_{1}(G) \oplus T_{1}(G) \xrightarrow{+} T_{1}(G)
	\end{equation*}
	is zero. Thus, $X + T_{1}(\iota)(X) = 0$.
\end{sketch}

Next, let us compute some explicit differentials of matrix maps. Note that $\GL(n, F)$ is an open subset of the vector space $M_{n}(F)$ and thus, its tangent space at identity can be identified with $M_{n}(F)$. \newline
In particular, $T_{1}(F^{\times}) = F$.

\begin{thm} \label{thm:diff-complex-det-is-trace}
	The differential of $\det \colon \GL(n, \mathbb{C}) \to \mathbb{C}^{\times}$ at identity is the trace map
	\begin{equation*} 
		\trace \colon M_{n}(\mathbb{C}) \to \mathbb{C}.
	\end{equation*}
	This has rank two over $\mathbb{R}$.
\end{thm}
\begin{sketch}
	Consider a tangent vector $T \in T_{I}(\GL(n, \mathbb{C})) = M_{n}(\mathbb{C})$. 
	Look at the line $s \mapsto I + sT$. 
	We have
	\begin{align*} 
		\det(I + sT) &
		= \det
		\begin{bmatrix}
			1 + s T_{11} & s T_{12} & \cdots & s T_{1n} \\
			\vdots & \vdots & \ddots & \vdots \\
			s T_{n1} & s T_{n2} & \cdots & 1 + s T_{nn} \\
		\end{bmatrix} \\
		&= 1 + s(T_{11} + \cdots T_{nn}) + s^{2}(\cdots). \qedhere
	\end{align*}
\end{sketch}

\begin{thm}[Lie bracket on GL]
	Consider $G \coloneqq \GL(n, \mathbb{R})$. 
	For $A \in G$, we have the conjugation map $\INT(A) \colon G \to G$. 
	Let $\Ad(A) \coloneqq T_{I}(\INT(A))$ denote its differential at identity. Thus, $\Ad(A)$ is a linear map $M_{n}(\mathbb{R}) \to M_{n}(\mathbb{R})$.

	For $X \in M_{n}(\mathbb{R})$, we have $\Ad(A)(X) = AXA^{-1}$.

	Furthermore, $\Ad$ gives us a map $\Ad \colon G \to \End(M_{n}(\mathbb{R}))$. We may differentiate this at identity to get a linear map
	\begin{equation*} 
		T_{I}(\Ad) \colon T_{1}(G) \to T_{1}(\End(M_{n}(\mathbb{R}))).
	\end{equation*}
	Under the usual identifications, this can be rewritten as a linear map
	\begin{equation*} 
		T_{I}(\Ad) \colon M_{n}(\mathbb{R}) \to \Hom(M_{n}(\mathbb{R}), M_{n}(\mathbb{R})).
	\end{equation*}
	Currying/uncurrying this gives us a \emph{bilinear} map
	\begin{equation*} 
		T_{I}(\Ad) \colon M_{n}(\mathbb{R}) \times M_{n}(\mathbb{R}) \to M_{n}(\mathbb{R}).
	\end{equation*}
	We have
	\begin{equation*} 
		T_{I}(\Ad)(X, Y) = XY - YX
	\end{equation*}
	for $X, Y \in M_{n}(\mathbb{R})$.
\end{thm}
Note that the above steps could be carried out for a general Lie group $G$ to get a \emph{bilinear} map $T_{1}(G) \times T_{1}(G) \to T_{1}(G)$. 
This will give us the Lie algebra structure on $T_{1}(G)$.
\begin{sketch}
	First, let $A \in G$ and $X \in M_{n}(\mathbb{R})$ be arbitrary. 
	For $s \in \mathbb{R}$, we note
	\begin{equation*} 
		\INT(A)(I + sX) = A(I + sX)A^{-1} = I + s(AXA^{-1}).
	\end{equation*}
	This shows $T_{I}(\INT(A))(X) = AXA^{-1}$.

	Now, let $X, Y \in M_{n}(\mathbb{R})$ be arbitrary. 
	For small $s$, the element $I + sX$ is invertible, and the above tells us
	\begin{equation*} 
		\Ad(I + sX)(Y) = (I + sX) Y (I + sX)^{-1}.
	\end{equation*}
	For small $s$, we may expand the above as
	\begin{align*} 
		\Ad(I + sX)(Y) &= (I + sX) Y \left(I - sX + (sX)^{2} - \cdots \right) \\
		&= (Y + sXY) \left(I - sX + (sX)^{2} - \cdots \right) \\
		&= Y + sXY - sYX - s^{2} XYX + \text{higher order} \\
		&= Y + s(XY - YX) + \text{higher}. \qedhere
	\end{align*}
\end{sketch}

\subsection{Creating Lie groups}

We see a criteria here that gives us that many natural groups are Lie groups.
In particular, we will see that smoothness of the multiplication map implies that of the inversion map.

For this subsection, we have the following setup:

\begin{tcolorbox}
	Let $M$ be a ``Lie monoid'', i.e., a smooth manifold equipped with a smooth binary operator $m \colon M \times M \to M$ that is associative and unital.

	Let $G$ be the subset of \deff{regular} elements, i.e.,
	\begin{equation*} 
		G \coloneqq \{m \in M : m \text{ is invertible}\}.
	\end{equation*}
\end{tcolorbox}

Then, one has the following.

\begin{thm} \label{thm:lie-monoid-invertible-subgroup}
	$G$ is an open subset of $M$. (Thus, $G$ has a natural manifold structure.)

	$G$ is closed under multiplication. The inverse of an element of $G$ is again an element of $G$.

	The induced inversion map on $G$ is smooth. 

	Thus, $G$ is a Lie group.
\end{thm}
We sketch a proof at the end of this subsection.

\begin{cor} \label{cor:inversion-automatically-smooth}
	Suppose $G$ is a smooth manifold, and $m \colon G \times G \to G$ is a smooth binary operator that makes $G$ into a group. \newline
	Then, $G$ is a Lie group, i.e., the inversion map $g \to g^{-1}$ is smooth.
\end{cor}

The following is another setup where the above can be applied:

Suppose $A$ is a finite-dimensional associative $\mathbb{R}$-algebra with a unit $1$. 
Using a basis, we can identify $A \cong \mathbb{R}^{n}$; this gives a (basis-independent) smooth structure on $A$. 
The multiplication on $A$ is $\mathbb{R}$-bilinear, hence smooth. Thus, the regular elements form a Lie group $G$.

Note that since $G$ is an open subset of $A$, we get that $\dim(G) = \dim_{\mathbb{R}}(A)$.

\begin{ex} \label{ex:GL-examples}
	For $A$, we may take $M_{n}(F)$ where $F$ is one of $\mathbb{R}$, $\mathbb{C}$, or $\mathbb{H}$. \newline
	This gives us the corresponding Lie group $\GL(n, \mathbb{F})$ with corresponding dimension.
\end{ex}

\begin{proof}[Sketch of \Cref{thm:lie-monoid-invertible-subgroup}]
	Consider $\Phi \colon M \times M \to M$ defined by $(a, b) \mapsto (a, m(a, b))$. \newline
	% Let $p_{1}, p_{2} \colon M \times M \to M$ be the projections. \newline
	% Note $p_{1} \circ \Phi = p_{1}$ and $p_{2} \circ \Phi = m$. 
	Thus,
	\begin{equation*} 
		T_{(1, 1)}(\Phi)(X, Y) = (X, X + Y).
	\end{equation*}
	(The calculation of $T_{(1, 1)}(m)$ done earlier is valid even for Lie monoids.)

	Note that the above is an automorphism of tangent spaces, using the inverse function theorem tells us that $\Phi$ is a diffeomorphism when restricted to open sets. By restrictions, we may assume that 
	\begin{equation*} 
		\Phi|_{U} = \varphi \colon U \to V \times V
	\end{equation*}
	is a diffeomorphism where $U \subset M \times M$ is open, and $V \subset M$ is open. \newline
	Let $\psi$ be its inverse. Check that $\psi$ looks like
	\begin{equation*} 
		\psi(a, b) = (a, m(a, \beta(a, b)))
	\end{equation*}
	for some smooth $\beta$. 
	But then, $m(a, \beta(a, 1)) = 1$ for all $a \in V$. 
	This gives a neighbourhood of $1$ where every element has a right inverse. 
	By symmetry, there is one with left inverses. 
	By intersection, there is a neighbourhood of $1$ on which every element has an inverse and that this inversion map is smooth. By translation, $G$ is open and inversion is smooth.
\end{proof}

\subsection{Lie subgroups}

\begin{defn}
	A \deff{Lie subgroup} of $G$ is a subgroup of $G$ that is a submanifold (\Cref{defn:submanifold}).
\end{defn}
\begin{rem}
	Note that by \Cref{thm:smoothness-into-submanifold}, the restricted multiplication map $H \times H \to H$ is smooth. 
	In turn, \Cref{cor:inversion-automatically-smooth} tells us that $H$ is a Lie group with the induced structure.
\end{rem}

As remarked in \Cref{rem:submanifold}, $H$ is locally closed in $G$. The following tells us more.
\begin{thm}
	Lie subgroups are closed.
\end{thm}

A deeper theorem of Cartan tells us that the purely topological converse is true:
\begin{thm}[Cartan]
	Any closed subgroup of a Lie group is a Lie subgroup.
\end{thm}

\section{Group actions}

We now study properties of group actions. The interesting takeaway is how questions about group homomorphisms can be reduced to the general framework of group actions.

\begin{defn}
	Let $G$ be a Lie group, and $M$ a smooth manifold. 
	A (smooth) \deff{action} of $G$ on $M$ is a smooth map $\mu \colon G \to M$ inducing a usual action on $M$.
\end{defn}
``Usual action'' means that $\mu(1, m) = m$ for all $m$ and that the diagram
\begin{equation*} 
	\begin{tikzcd}
		G \times G \times M \arrow[rr, "\id_{G} \times \mu"] \arrow[d, "m \times \id_{M}"'] && G \times M \arrow[d, "\mu"] \\
		G \times M \arrow[rr, "\mu"] && M
	\end{tikzcd}
\end{equation*}
commutes.

\subsection{Orbits}

Given $m \in M$ and $g \in G$, we get the following maps

\begin{align*} 
	\omega_{m} \colon G & \to M \\
	g & \mapsto g \cdot m,
\end{align*}

and

\begin{align*} 
	\tau(g) \colon M & \to M \\
	m & \mapsto g \cdot m.
\end{align*}

Note that $\tau(g)$ is a diffeomorphism and we have
\begin{equation*} 
	\omega_{m} \circ \gamma(g) = \tau(g) \circ \omega_{m}.
\end{equation*}
Taking differentials at $1$ and noting diffeomorphisms, we get
\begin{equation*} 
	\rank(T_{g}(\omega_{m})) = \rank(T_{1}(\omega_{m})).
\end{equation*}

\begin{cor}
	For a fixed $m$, the orbit map $\omega_{m} \colon G \to M$ has constant rank. \newline
	Thus, $\omega_{m}$ is a \deff{subimmersion}.
\end{cor}
The rank may very well depend on $m$. 
E.g., consider the action of $\GL(n, \mathbb{R})$ on $\mathbb{R}^{n}$. 
The orbit map of $\mathbf{0}$ is different from any other orbit map.

A subimmersion of rank $r$ locally looks like 
\begin{equation*} 
	\mathbb{R}^{d} \ni (x_{1}, \ldots, x_{r}, x_{r + 1} \ldots, x_{d}) \mapsto (x_{1}, \ldots, x_{r}, 0, \ldots, 0) \in \mathbb{R}^{n}.
\end{equation*}

\subsection{Stabilisers}

Given $m \in M$, the \deff{stabiliser} of $m$ is the subgroup
\begin{equation*} 
	\Stab_{m} \coloneqq \{g \in G : g \cdot m = m\}.
\end{equation*}
We may denote the above as $S_{m}$ if the context is clear.

\begin{thm} \label{thm:stabiliser-subgroup-tangent-space}
	The stabiliser $\Stab_{m}$ is a Lie subgroup of $G$ with
	\begin{equation*} 
		T_{1}(\Stab_{m}) = \ker(T_{1}(\omega_{m})).
	\end{equation*}
\end{thm}

\begin{cor} \label{cor:image-kernel-morphism-properties}
	If $\varphi \colon G \to H$ is a Lie group morphism, then $\varphi$ is of constant rank and $\ker(\varphi)$ is a Lie subgroup of $G$ with 
	\begin{equation*} 
		T_{1}(\ker(\varphi)) = \ker(T_{1}(\varphi)).
	\end{equation*}
\end{cor}
\begin{sketch} 
	$G$ acts on $M$ via $(g, h) \mapsto \varphi(g) h$. Then, $\varphi = \omega_{1_{H}}$ and $\ker(\varphi) = \Stab_{1_{H}}$.
\end{sketch}

\begin{cor} \label{cor:local-diffeo-characterisation}
	If $\varphi \colon G \to H$ is a morphism of Lie groups, then the following are equivalent:
	\begin{enumerate}[label=(\alph*)]
		\item $\varphi$ is a local diffeomorphism.
		\item $T_{1}(\varphi)$ is an isomorphism.
		\item $\dim(G) = \dim(H)$ and $\ker(\varphi)$ is zero-dimensional.
	\end{enumerate}
\end{cor}
\begin{sketch}
	(a) $\Rightarrow$ (b) is OK. \newline
	(b)	$\Rightarrow$ (a): constant rank implies $T_{g}(\varphi)$ is an isomorphism for all $g$. \newline
	(b) $\Leftrightarrow$ (c) follows by \Cref{cor:image-kernel-morphism-properties} and rank-nullity.
\end{sketch}

\subsection{Special linear groups} \label{subsec:SL-subgroup}

$\SL(n, F)$ is a subgroup of $\GL(n, F)$ for $F = \mathbb{R}, \mathbb{C}$ since this is the kernel of $\det \colon \GL(n, F) \to F^{\times}$.
This is called the \deff{special linear group}.

Let us consider $F = \mathbb{R}$ first. 
Calculations as in \Cref{thm:diff-complex-det-is-trace} tell us that the differential $T_{I}(\det)$ is
\begin{equation*} 
	\trace \colon M_{n}(\mathbb{R}) \to \mathbb{R}.
\end{equation*}

Thus, by \Cref{cor:image-kernel-morphism-properties}, we have
\begin{equation*} 
	T_{I}(\SL(n, \mathbb{R})) = \{A \in M_{n}(\mathbb{R}) : \trace(A) = 0\} \le T_{I}(\GL(n, \mathbb{R})).
\end{equation*}
Thus,
\begin{equation*} 
	\dim(\SL(n, \mathbb{R})) = n^{2} - 1.
\end{equation*}

Similarly, for $F = \mathbb{C}$, we see that
\begin{equation*} 
	T_{I}(\SL(n, \mathbb{C})) = \{A \in M_{n}(\mathbb{C}) : \trace(A) = 0\} \le T_{I}(\GL(n, \mathbb{C})).
\end{equation*}
Note that since these are vector space dimensions over $\mathbb{R}$ and the $\trace$ is a rank two map in this case. Thus, we have
\begin{equation*} 
	\dim(\SL(n, \mathbb{C})) = 2n^{2} - 2.
\end{equation*}

\section{Involutions on algebras}

We define the notion of an involution on an algebra (the prototypical examples being conjugation and transposition). 
We then show that the elements whose inverse equals their involution form a Lie subgroup of invertible elements, 
and identify the tangent space as a subspace of $A$.

This general framework lets us define and compute the dimensions of the (usual and special) orthogonal-esque groups.

\subsection{Definitions}

Let $A$ be a finite-dimensional associative unital $\mathbb{R}$-algebra. 
As noted earlier, $A$ is a canonically a smooth manifold with the multiplication being smooth, 
and $G$ the subset of invertible elements is a open subset which is a Lie group. 

\begin{defn}
	An \deff{involution} on $A$ is a function $\tau \colon A \to A$ satisfying
	\begin{itemize}
		\item $\tau$ is $\mathbb{R}$-linear,
		\item $\tau(\tau(a)) = a$ for all $a \in A$,
		\item $\tau(ab) = \tau(b) \tau(a)$ for all $a, b \in A$.
	\end{itemize}
\end{defn}
We use the notation $a^{\tau} \coloneqq \tau(a)$.

\begin{ex}
	$A \coloneqq M_{n}(\mathbb{R})$ is an example of such an $\mathbb{R}$-algebra with the usual matrix multiplication. \newline
	The transpose operation is an involution.

	Similarly, conjugate-transpose is an involution on $A \coloneqq M_{n}(\mathbb{C})$. Note that this \emph{is} $\mathbb{R}$-linear (but not $\mathbb{C}$-linear).
\end{ex}

\begin{lem}
	$\tau(1) = 1$.
\end{lem}

Define
\begin{equation*} 
	H \coloneqq \{g \in A : g g^{\tau} = 1 = g^{\tau} g\}.
\end{equation*}
Note that $H \subset G$.

\begin{lem}
	$H$ is a subgroup of $G$.
\end{lem}

\begin{rem}
	Since $A$ is a finite-dimensional $\mathbb{R}$-algebra, the equation $ab = 1$ implies $ba = 1$. 
	We will use this going forth and only write the equality $gg^{\tau} = 1$.
\end{rem}

\begin{thm} \label{thm:involution-subgroup-tangent-space}
	$H$ is a Lie subgroup of $G$ with
	\begin{equation*} 
		T_{1}(H) = \{a \in A : a = -a^{\tau}\}.
	\end{equation*}
\end{thm}
The above gives us the dimension of $H$ as well.

\subsection{The orthogonal and unitary groups}

\begin{ex} \label{ex:orthogonal-unitary-groups-dimensions}
	The \deff{orthogonal matrix group}
	\begin{equation*} 
		\OO(n) \coloneqq \{A \in M_{n}(\mathbb{R}) : A A^{\trans} = I\}
	\end{equation*}
	is a Lie group of dimension $\frac{n(n - 1)}{2}$.

	The \deff{unitary matrix group}
	\begin{equation*} 
		\UU(n) \coloneqq \{A \in M_{n}(\mathbb{C}) : A A^{\ast} = I\}
	\end{equation*}
	is a Lie group of dimension $n^{2}$.

	The dimensions were computed by computing the (real) dimensions of the tangent spaces obtained from \Cref{thm:involution-subgroup-tangent-space}.
\end{ex}

\begin{ex} \label{ex:special-orthogonal-unitary-groups-dimensions}
	Considering the kernels of the $\det$ map on the above groups, 
	we get $\SO(n)$ and $\SU(n)$ as Lie subgroups of $\OO(n)$ 
	and $\UU(n)$ respectively. 
	These are the subgroups of matrices of determinant $1$, called the \deff{special orthogonal group} 
	and \deff{special unitary group} respectively. 
	(These are subgroups by \Cref{cor:image-kernel-morphism-properties}.)

	Note that $\det(\OO(n)) = \{\pm 1\}$. Thus, $\SO(n)$ is an open subset of $\OO(n)$. 
	This gives us $\dim(\SO(n)) = \dim(\OO(n))$.

	Similarly, we have $\det \colon \UU(n) \to S^{1}$. 
	By \Cref{cor:image-kernel-morphism-properties}, we know that
	\begin{equation*} 
		\dim(\SU(n)) = \dim(\ker(\det)) = \dim_{\mathbb{R}}(\ker(T_{1}(\det))) = \nullity(T_{1}(\det)).
	\end{equation*}
	Recall from \Cref{thm:diff-complex-det-is-trace} that $T_{1}(\det) = \trace$. 
	Restricting this to $T_{1}(\UU(n)) = \{A \in M_{n}(\mathbb{C}) : A = -A^{\ast}\}$ is still a nonzero map. 
	Thus, $\det \colon \UU(n) \to S^{1}$ is a rank-one map. This gives us
	\begin{equation*} 
		\dim(\SU(n)) = \nullity(T_{1}(\det)) = \dim(\UU(n)) - 1.
	\end{equation*}
\end{ex}

\subsection{Quaternions and symplectic group} \label{subsec:sympletic-group}

Recall that the algebra of quaternions $\mathbb{H}$ is a four-dimensional associative unital $\mathbb{R}$-algebra. 
The usual representation is $\mathbb{H} = \mathbb{R}(1 \oplus \iota \oplus \jmath \oplus k)$ with the ``cross-product'' relations between $\iota$, $\jmath$, $k$. 
The conjugation map on $\mathbb{H}$ is the linear map fixing $1$ and negating the other three basis elements.

$\mathbb{H}$ can also be identified with the subalgebra $A$ of $M_{2}(\mathbb{C})$ given by
\begin{equation*} 
	A \coloneqq 
	\left\{\two{\alpha}{\beta}{-\overline{\beta}}{\overline{\alpha}} : \alpha, \beta \in \mathbb{C}\right\}.
\end{equation*}
The determinant of the matrix above is $\md{\alpha}^{2} + \md{\beta}^{2} = 1$. 
Moreover, the usual conjugate-transpose involution on $M_{2}(\mathbb{C})$ restricts to an involution on $A$. 
This corresponds to the conjugation on $\mathbb{H}$.

In turn, this leads to an involution on $M_{n}(\mathbb{H})$ given by conjugate-transpose. 
As in \Cref{ex:orthogonal-unitary-groups-dimensions}, we see that the \deff{symplectic group} defined as
\begin{equation*} 
	\Sp(n) \coloneqq \{A \in M_{n}(\mathbb{H}) : AA^{\ast} = I\}
\end{equation*}
is a compact Lie group of dimension $n(2n + 1)$.

Note that $\Sp(1) \cong \SU(2)$.

\section{Connectedness of some groups} \label{sec:connectedness}

\begin{thm}
	$\SO(n)$, $\SU(n)$, $\Sp(n)$, and $\UU(n)$ are connected Lie groups.
\end{thm}
Note that $\OO(n)$ is not connected.

\begin{proof}[Sketch for $\SO(n)$]
	By induction on $n$. The case $n = 1$ is trivial. For $n = 2$, we explicitly have
	\begin{equation*} 
		\SO(2) = \left\{\two{\cos(\theta)}{-\sin(\theta)}{\sin(\theta)}{\cos(\theta)} : \theta \in \mathbb{R}\right\}
	\end{equation*}
	is connected. Assume $n \ge 3$.

	The action of $\GL(n, \mathbb{R})$ on $\mathbb{R}^{n}$ restricts to a smooth action of $\SO(n)$ on $S^{n - 1}$. 
	Consider the north pole $p = (1, 0, \ldots, 0)^{\trans} \in S^{n - 1}$. Note that $Mp$ is the first column of $M$. This gives us that the differential of the orbit map $\omega_{p}$ is
	\begin{align*} 
		T_{1}(\SO(n)) &\to T_{1}(S^{n - 1}) \\
		M &\mapsto \text{first column of $M$}.
	\end{align*}
	We know $T_{1}(\SO(n)) = \{\text{skew-symmetric matrices}\}$ and 
	$T_{1}(S^{n - 1}) \le T_{1}(\mathbb{R}^{n})$ consists of column vectors with first coordinate zero. 
	This description tells us that $T_{1}(\omega_{p})$ is surjective, hence a submersion, hence an open map (using the local description of a submersion). 

	Thus, $\omega_{p}(\SO(n))$ is an open subset of $S^{n - 1}$. But this is also compact. 
	Connectedness of $S^{n - 1}$ tells us that $\omega_{p}(\SO(n)) = S^{n - 1}$.

	Suppose $\SO(n)$ is not connected. 
	Let $G_{0}$ be the component of $I$. 
	The same arguments applied to $G_{0}$ tell us that $\omega_{p}(G_{0}) = S^{n - 1}$ as well.

	Check that
	\begin{equation*} 
		\Stab_{p} = \left\{\two{1}{}{}{U} : U \in \SO(n - 1)\right\}.
	\end{equation*}
	By induction, the above is connected and hence, $\Stab_{p} \le G_{0}$.

	Now, given $T \in \SO(n)$, the calculation earlier gives us $Tp = Sp$ for some $S \in G_{0}$. Thus, $S^{-1} T \in \Stab_{p} \subset G_{0}$.
\end{proof}

\section{A map from SU(2) to SO(3)}

By earlier, we have $\dim(\SU(2)) = 2^{2} - 1 = 3$ and $\dim(\SO(3)) = 3(3 - 1)/2 = 3$. 
Thus,
\begin{equation*} 
	\dim(\SU(2)) = \dim(\SO(3)).
\end{equation*}
We will show that we have a two-fold covering map $\SU(2) \to \SO(3)$. 

First, note that $\SU(2)$ is simply-connected since topologically $\SU(2) \cong S^{3} \subset \mathbb{C}^{2}$ since
\begin{equation*} 
	\SU(2) = 
	\left\{\two{\alpha}{\beta}{-\overline{\beta}}{\overline{\alpha}} : 
	\begin{array}{c}
	\alpha, \beta \in \mathbb{C}, \\ 
	\md{\alpha}^{2} + \md{\beta}^{2} = 1
	\end{array}
	\right\}.
\end{equation*}

Consider the linear map $T \colon \mathbb{R}^{3} \to M_{2}(\mathbb{C})$ given by
\begin{equation*} 
	(x, y, z) \mapsto \two{x}{y + \iota z}{y - \iota z}{-x}.
\end{equation*}
Let $\mathcal{H} \coloneqq \im(T)$. Check that
\begin{itemize}
	\item $\dim(\mathcal{H}) = 3$.
	\item $\mathcal{H}$ is precisely the space of traceless self-adjoint matrices.
	\item $\det(T(x, y, z)) = -(x^{2} + y^{2} + z^{2}) = -\|(x, y, z)\|^{2}$.
\end{itemize}
Thus, $(\mathcal{H}, -\det)$ is isomorphic to $(\mathbb{R}^{3}, \|\cdot\|)$ as normed spaces.

By the description of $\mathcal{H}$, we see that $\SU(2)$ acts on $\mathcal{H}$ by
\begin{equation*} 
	(T, H) \mapsto THT^{\ast}.
\end{equation*}
Moreover, $-\det(THT^{\ast}) = -\det(H)$. 
Thus, $\SU(2)$ acts via norm-preserving isomorphisms, giving us a homomorphism $\SU(2) \to \OO(\mathcal{H}) \cong \OO(\mathbb{R}^{3})$. 
Since $\SU(2)$ is connected, this map has image inside $\SO(3)$.

One can calculate the map to get that the homomorphism $\rho \colon \SU(2) \to \SO(3)$ is
\begin{equation*} 
	\two{\alpha}{\beta}{-\overline{\beta}}{\overline{\alpha}} 
	\mapsto
	\begin{bmatrix}
		\md{\alpha}^{2} - \md{\beta}^{2} & 2 \Re(\alpha \overline{\beta}) & -2 \Im(\alpha \overline{\beta}) \\
		-2 \Re(\alpha \beta) & \Re(\alpha^{2} - \beta^{2}) & -\Im(\alpha^{2} + \beta^{2}) \\
		-2 \Im(\alpha \beta) & \Im(\alpha^{2} - \beta^{2}) & \Re(\alpha^{2} + \beta^{2}) \\
	\end{bmatrix}.
\end{equation*}
The above gives us that $\ker(\rho) = \{\pm I\}$. 
By \Cref{cor:local-diffeo-characterisation}, $\rho$ is a local diffeomorphism. 
We will see later (\Cref{lem:covering-local-diffeo-equivalent}) that this implies that $\rho$ is a covering map. 
We already checked that $\SU(2)$ is simply-connected. 
Thus, we have constructed the universal cover of $\SO(3)$. 
It would also follow (by \Cref{thm:pi-one-kernel-covering}) that $\pi_{1}(\SO(3), I) \cong \mathbb{Z}/2$.

\section{An open mapping theorem}

\textbf{Question.} Suppose $G$ acts transitively on $M$, and $m \in M$. Is the orbit map $\omega_{m} \colon G \to M$ a submersion? \newline
\textbf{Answer.} {\color{red}\textbf{No!}} Consider $G \coloneqq \mathbb{R}_{\text{disc}}$ and $M \coloneqq \mathbb{R}$. 
By our definitions, $G$ is indeed a (zero-dimensional) Lie group. \newline
We have the transitive action of $G$ on $M$ given by $(x, y) \mapsto x + y$. 
The orbit map of $0$ is the ``identity'' map $\mathbb{R}_{\text{disc}} \to \mathbb{R}$. 
However, this map cannot be a submersion since the dimensions are incorrect. \newline
We will see that under mild assumptions (\Cref{thm:open-mapping-group-action}) such a thing cannot happen.

For this section, just a topological action (i.e., a continuous action) is sufficient to prove the theorems.

\begin{defn}
	A topological space is said to be \deff{locally compact} if it is \underline{Hausdorff} and every point has a compact neighbourhood.
\end{defn}

\begin{defn}
	A locally compact space is \deff{countable at $\infty$} if it is the union of countably many compact sets.
\end{defn}
$\mathbb{R}_{\text{disc}}$ is locally compact but not countable at $\infty$. \newline
The Euclidean space $\mathbb{R}^{n}$ is countable at $\infty$.

\begin{lem} 
	Let $G$ be a connected topological group and $U$ a neighbourhood of $1$. Then,
	\begin{equation*} 
		G = \bigcup_{n \ge 1} U^{n},
	\end{equation*}
	where $U^{n}$ is the set of all products of $n$ elements of $U$.

	In particular, any neighbourhood of the identity of a connected Lie group generates the group.
\end{lem}
\begin{sketch}
	We may assume that $U$ is open and symmetric, in which case the union is an open subgroup $H$. 
	But $G$ is then the disjoint union of the cosets of $H$. 
	Since each coset is also open, we must have $H = G$.
\end{sketch}

\begin{cor}
	Let $G$ be a connected topological group. If $G$ is locally compact, then $G$ is countable at $\infty$.
\end{cor}

\begin{lem} 
	Let $G$ be a Lie group. The following are equivalent:
	\begin{enumerate}[label=(\alph*)]
		\item $G$ is countable at $\infty$.
		\item $G$ has countably many components.
	\end{enumerate}
\end{lem}
In the above, we only need that $G$ is a \emph{topological} manifold (with continuous multiplication and inversion).

\begin{thm} \label{thm:open-mapping-group-action}
	Let $G$ be a locally compact topological group with countably many components. 
	Let $X$ be a locally compact topological space. 
	Suppose $G$ acts on $X$ continuously and that this is action is transitive. 
	Let $x \in X$. The orbit map 
	\begin{align*} 
		\omega_{x} \colon G &\to X\\
		g &\mapsto gx
	\end{align*}
	is open.

	In the case that everything is smooth, the map $\omega_{x}$ is an open submersion.
\end{thm}
Note that in the smooth case, we already knew that $\omega_{x}$ is a subimmersion. 
Thus, being open gives us that it is an immersion. 
(Recalling the local description of a subimmersion.)

\begin{sketch}
	Let $U \subset G$ be an open neighbourhood of $1$. 
	Pick a compact symmetric neighbourhood $1 \in V \subset U$ with $V^{2} \subset U$. 
	Since $G$ is countable at $\infty$, 
	there is a countable set $\{g_{n} \in G : n\}$ such that 
	$(g_{n} \cdot \Int(V))_{n}$ is an open cover of $G$. 
	Hence, $(g_{n} \cdot V)_{n}$ is a compact cover of $G$. 
	We have that $U_{n} \coloneqq X \setminus \omega_{x}(g_{n} \cdot V)$ is open in $X$ with $\bigcap_{n} U_{n} = \emptyset$. 
	By the Baire category theorem, some $U_{n}$ is not dense in $X$. 
	Thus, $\omega_{x}(g_{n} \cdot V) = g_{n} \cdot V \cdot x$ has nonempty interior. 
	Thus, $V \cdot x$ is a neighbourhood of $g \cdot x$ for some $g$. 
	But then $U \cdot x$ is a neighbourhood of $x$.
\end{sketch}

\section{Covering maps}

For a smooth map $p \colon X \to Y$ of smooth connected manifolds, we say that $p$ is a \deff{covering} map 
if $p$ is surjective and given any $y \in Y$, 
there exists a connected open neighbourhood $U$ of $y$ such that 
$p^{-1}(U)$ is a disjoint neighbourhood of components such that 
$p$ restricted to any component is a diffeomorphism onto $U$. \newline
In this case, $U$ is said to be an \deff{evenly covered} neighbourhood.

\begin{lem} \label{lem:covering-local-diffeo-equivalent}
	Assume $G$ and $H$ are \emph{connected} Lie groups, and $p \colon G \to H$ is a Lie group morphism. The following are equivalent:
	\begin{enumerate}[label=(\alph*)]
		\item $p$ is a covering map.
		\item $T_{1}(p)$ is a linear isomorphism.
		\item $p$ is a local diffeomorphism.
	\end{enumerate}
	In such a case, $D \coloneqq \ker(p)$ is a discrete subgroup contained in the center of $G$.
\end{lem}
\begin{sketch}
	(b) $\Leftrightarrow$ (c) was \Cref{cor:local-diffeo-characterisation}.

	(c) $\Rightarrow$ (a): $p(G)$ is open and hence, generates $H$. 
	But $p(G)$ is already a subgroup. 
	Thus, $p(G) = H$ and hence, $p$ is surjective. 
	Let $V$ be a neighbourhood of $1_{G}$ on which $p$ is bijective.
	We must have that $D \coloneqq \ker(p)$ does not intersect $V$. 
	In turn, $D$ is discrete. \newline
	Given $g \in G$, the map $\INT(g)$ lands inside $D$ by normality. 
	A continuous map $G \to D$ must be constant and hence, $D$ is central. \newline
	Now, let $W \subset V$ be a small enough open neighbourhood of $1_{G}$ such that $p|_{W}$ is a diffeomorphism, $W$ is connected, $W = W^{-1}$, $W^{2} \subset V$. 
	Then $p(W)$ is evenly covered since
	\begin{equation*} 
		p^{-1}(p(W)) = \bigsqcup_{d \in D} dW. \qedhere
	\end{equation*}
\end{sketch}

\subsection{Universal cover}

Recall that given any connected smooth manifold with basepoint $(X, x_{0})$, 
there exists a connected manifold $(\widetilde{X}, \widetilde{x}_{0})$ 
with a covering map $p \colon (\widetilde{X}, \widetilde{x}_{0}) \to (X, x_{0})$ 
such that any other covering map $q \colon (Y, y_{0}) \to (X, x_{0})$ factors uniquely as
\begin{equation*} 
	\begin{tikzcd}
		{(\widetilde{X}, \widetilde{x}_{0})} \arrow[rd, "p"'] \arrow[rr, "\exists!r", dashed] & & {(Y, y_{0})} \arrow[ld, "q"] \\
		& {(X, x_{0}).} &                             
	\end{tikzcd}
\end{equation*}
This is called the \deff{universal cover} of $(X, x_{0})$ and is unique up to diffeomorphism.

Now, if $G$ is a connected Lie group, 
then there exists a universal covering space $p \colon (\widetilde{G}, \widetilde{1}) \to (G, 1)$ 
using the fact that $G$ has a smooth structure. \newline
To emphasise, $\widetilde{G}$ is only a smooth manifold with a distinguished basepoint $\widetilde{1}$, 
and $p$ is only a smooth map.

\begin{thm}
	There exists a unique smooth binary operator $\widetilde{m}$ on $\widetilde{G}$ that makes $p$ a multiplicative map. 
	This map $\widetilde{m}$ then also makes $\widetilde{G}$ a group with $\widetilde{1}$ as the identity.

	Thus, $\widetilde{G}$ is then called the \deff{universal covering group}.
\end{thm}

The above will follows from the \textsc{Lifting Theorem}: Suppose we have a covering $(Y, y_{0}) \xrightarrow{p} (X, x_{0})$, a simply-connected space $Z$, and a smooth map $(Z, z_{0}) \xrightarrow{F} (X, x_{0})$. Then, there is a unique map $F'$ making the following commute:
\begin{equation*} 
	\begin{tikzcd}
		{(Z, z_{0})} \arrow[rd, "F"'] \arrow[r, "F'", dashed] & {(Y, y_{0})} \arrow[d, "p"] \\
		& {(X, x_{0})}
	\end{tikzcd}
\end{equation*}

\begin{proof}[Sketch of the theorem]
	To make $p$ multiplicative is precisely asking for a dashed arrow in the following diagram to exist (such that the diagram commutes):
	\begin{equation*} 
		\begin{tikzcd}
		{(\widetilde{G} \times \widetilde{G}, (\widetilde{1}, \widetilde{1}))} \arrow[dd, "p \times p"'] \arrow[rr, dashed] \arrow[rrdd, "m \circ (p \times p)"] &  & {(\widetilde{G}, \widetilde{1})} \arrow[dd, "p"] \\ \\
		{(G \times G, (1, 1))} \arrow[rr, "m"']  &  & {(G, 1).} 
		\end{tikzcd}
	\end{equation*}
	The \textsc{Lifting Theorem} gives the existence and uniqueness of such a map, which we call $\widetilde{m}$. 
	Note that this maps satisfies $\widetilde{m}(\widetilde{1}, \widetilde{1}) = \widetilde{1}$.

	It now remains to check that $\widetilde{m}$ as defined above makes $\widetilde{G}$ 
	into a group with $\widetilde{1}$ as the identity. 
	This follows from using the \textsc{Lifting Theorem} in different ways appropriately. 

	The point to note is the following: 
	if $Z$ is simply-connected and $f, g \colon (Z, z_{0}) \to (\widetilde{G}, \widetilde{1})$ are two maps such that 
	$f \circ p = g \circ p$, 
	then we must have $f = g$.

	Thus, for example, checking associativity means that we want the diagram

	\begin{equation*} 
		\begin{tikzcd}
		\widetilde{G} \times \widetilde{G} \times \widetilde{G} \arrow[d, "\id \times \widetilde{m}"'] \arrow[rr, "\widetilde{m} \times \id"] &  & \widetilde{G} \times \widetilde{G} \arrow[d, "\widetilde{m}"] \\
		\widetilde{G} \times \widetilde{G} \arrow[rr, "\widetilde{m}"']                                               &  & \widetilde{G}                        
		\end{tikzcd}
	\end{equation*}	
	to commute. But the diagram does commute after composing with $p$ since we have associativity in $G$. 
	We check that all the maps above do preserve the obvious basepoints.
	% \begin{align*} 
	% 	(p \circ \widetilde{m} \circ (\id \times \widetilde{m}))(a, b, c) &= (p \circ \widetilde{m})(a, \widetilde{m}(b, c)) \\
	% 	&= m(p(a), p(\widetilde{m}(b, c))) \\
	% 	&= m(p(a), m(p(b), p(c))) \\
	% 	&= p(a) \cdot p(b) \cdot p(c),
	% \end{align*}
	% where the last equation
\end{proof}

Same techniques as earlier give us the following lifting theorem as well.

\begin{thm}
	Let $G$ and $H$ be connected Lie groups, and $\varphi \colon G \to H$ a morphism of Lie groups. 
	Then, there is a unique smooth map $\widetilde{\varphi}$ that makes
	\begin{equation*} 
		\begin{tikzcd}
			\widetilde{G} \arrow[r, "\widetilde{\varphi}"] \arrow[d, "p_{G}"'] & \widetilde{H} \arrow[d, "p_{H}"] \\
			G \arrow[r, "\varphi"'] & H
		\end{tikzcd}
	\end{equation*}
	commute and satisfies $\widetilde{\varphi}(1_{\widetilde{G}}) = 1_{\widetilde{H}}$. \newline
	Moreover, this map is also a morphism of Lie groups.
\end{thm}

Thus, we have a new functor
\begin{equation*} 
	\CLieGrp \xrightarrow{\widetilde{(-)}} \SCLieGrp.
\end{equation*}

In all, we have the functors
\begin{equation*} 
	\begin{tikzcd}
	{\SCLieGrp} \arrow[r, hook, shift left] & {\CLieGrp} \arrow[r, hook, shift left] \arrow[l, shift left, "\widetilde{(-)}"] & {\LieGrp} \arrow[l, shift left, "(-)_{0}"],
	\end{tikzcd}
\end{equation*}
where the top functors are the natural inclusion/forgetful functors. 

\begin{thm}
	All sensible possibilities of pairs of functors above are adjoints.
\end{thm}

\subsection{Deck transformations}

Let $p \colon (\widetilde{X}, \widetilde{x}_{0}) \to (X, x_{0})$ be a universal covering. 
A \deff{deck transformation} is a diffeomorphism $T \colon \widetilde{X} \to \widetilde{X}$ such that
\begin{equation*} 
	\begin{tikzcd}
		\widetilde{X} \arrow[rr, "T"] \arrow[rd, "p"'] & & \widetilde{X} \arrow[ld, "p"] \\
		& X &
	\end{tikzcd}
\end{equation*}
commutes.

Note that $T$ will not usually preserve the basepoint $\widetilde{x}_{0}$. 
However, it will permute the fiber $p^{-1}(x_{0})$.

Usual theory of covering spaces tells us that
\begin{equation*} 
	\pi_{1}(X, x_{0}) \cong \{\text{group of deck transformations}\}.
\end{equation*}

Back to our setup of Lie groups, we have:
\begin{equation*} 
	p \colon (\widetilde{G}, \widetilde{1}) \to (G, 1).
\end{equation*}
We showed that $D \coloneqq \ker(p) = p^{-1}(1)$ is central and discrete (\Cref{lem:covering-local-diffeo-equivalent}). 

For $g \in G$, recall the left translation map $\gamma(g) \colon G \to G$ defined by $h \mapsto gh$.

\begin{thm} \label{thm:pi-one-kernel-covering}
	We have an isomorphism
	\begin{equation*} 
		D \cong \{\text{group of deck transformations}\}
	\end{equation*}
	given by $d \mapsto \gamma(d)$. \newline
	Thus, $\pi_{1}(G, 1) \cong D$. 
	In particular, this is abelian.
\end{thm}

\section{Lie algebras}

We first start with a purely algebraic discussion of Lie algebras. 
However, unsurprisingly, we will later associate a Lie algebra to every Lie group. 
The notations along the way will be alluding to what is to follow. 
For example, we will define a Lie algebra $\gl(n, \mathbb{R})$ which will turn out to be the Lie algebra associated to the Lie group $\GL(n, \mathbb{R})$.

As it will turn out, the Lie algebra of a Lie group $G$ is the vector space $T_{1}(G)$ equipped with a binary operator. 
This will make it clear that the Lie algebra is only concerned with the connected component of $1$. 
What we will also see is that the Lie algebra ``does not see covering maps''. 
This is not surprising in view of \Cref{lem:covering-local-diffeo-equivalent} which told us that covering is equivalent to being an isomorphism of tangent spaces. 
Thus, by passing to the universal covering space of the connected component, 
the category of connected Lie groups seems to be the one to focus on. 
As we shall see, on this full subcategory, the Lie algebra functor will give us an equivalence.

\subsection{Definitions}

In what follows, $F$ is either $\mathbb{R}$ or $\mathbb{C}$.
\begin{defn}
	A \deff{Lie algebra} over $F$ is a (finite-dimensional) vector space $\mathfrak{g}$ equipped with an $F$-bilinear map $[-, -] \colon \mathfrak{g} \times \mathfrak{g} \to \mathfrak{g}$ satisfying
	\begin{enumerate}[label=(\alph*)]
		\item (anticommutativity) $[x, x] = 0$ for all $x \in \mathfrak{g}$, and
		\item (Jacobi identity) $[x, [y, z]] + [y, [z, x]] + [z, [x, y]] = 0$ for all $x, y, z \in \mathfrak{g}$.
	\end{enumerate}
	The operator $[-, -]$ is called the \deff{(Lie) bracket} or the \deff{commutator}.
\end{defn}
Since $\chr(F) = 0$, anticommutativity for a bilinear map is equivalent to $[x, y] = -[y, x]$. \newline

\begin{defn}
	A \deff{morphism} of Lie algebras is a liner map $\varphi \colon \mathfrak{g} \to \mathfrak{h}$ satisfying
	\begin{equation*} 
		\varphi([g, g']) = [\varphi(g), \varphi(g')]
	\end{equation*}
	for all $g, g' \in \mathfrak{g}$.

	This gives us the category $\LieAlg$ of Lie algebras.
\end{defn}

\begin{defn}
	A Lie algebra $(\mathfrak{g}, [-, -])$ is \deff{abelian} if $[-, -]$ is the zero map.
\end{defn}

\begin{rem}
	Being a bilinear map, $[-, -]$ is uniquely determined by its values on a basis. 
	In particular, being abelian can be checked on a basis.
\end{rem}

\begin{rem} \label{rem:abelian-lie-algebras-isomorphic}
	If $\mathfrak{g}$ and $\mathfrak{h}$ are Lie algebras of the same dimension, then they are isomorphic.
\end{rem}

\begin{defn}
	Let $\mathfrak{h} \subset \mathfrak{g}$ be a subspace of a Lie algebra.
	\begin{itemize}
		\item $\mathfrak{h}$ is a \deff{subalgebra} if $[\mathfrak{h}, \mathfrak{h}] \subset \mathfrak{h}$, that is, $[h, h'] \in \mathfrak{h}$ for all $h, h' \in \mathfrak{h}$.
		%
		\item $\mathfrak{h}$ is an \deff{ideal} if $[\mathfrak{h}, \mathfrak{g}] \subset \mathfrak{h}$, that is, $[h, g] \in \mathfrak{h}$ for all $h \in \mathfrak{h}$ and $g \in \mathfrak{g}$.
	\end{itemize}
\end{defn}
Note that an ideal is in particular a subalgebra. 
A subalgebra is a Lie algebra with the induced operations.

\begin{defn}
	$[\mathfrak{g}, \mathfrak{g}]$ denotes the vector space generated by all commutators, i.e., the $F$-span of $\{[x, y] : x, y \in \mathfrak{g}\}$. \newline
	This is the called the \deff{derived subalgebra} or \deff{derived ideal} of $\mathfrak{g}$.
\end{defn}

\begin{thm}
	If $\varphi \colon \mathfrak{g} \to \mathfrak{h}$ is a morphism of Lie algebras, then $\ker(\varphi)$ is an ideal of $\mathfrak{g}$.
\end{thm}
\begin{sketch} 
	For $z \in \ker(\varphi), g \in \mathfrak{g}$, we have $\varphi([z, g]) = [\varphi(z), \varphi(g)] = [0, \varphi(g)] = 0$.
\end{sketch}

\begin{thm}
	If $\mathfrak{h}$ is an ideal of a Lie algebra $\mathfrak{g}$, then the vector space $\mathfrak{g}/\mathfrak{h}$ is naturally a Lie algebra with the induced bracket.
\end{thm}

\begin{rem}
	We have the usual isomorphism and factoring theorems. For example, if $\varphi : \mathfrak{g} \to \mathfrak{h}$ is a map of Lie algebras, then, $\varphi$ factors as
	\begin{equation*} 
		\mathfrak{g} \onto \mathfrak{g}/(\ker \varphi) \into \mathfrak{h}.
	\end{equation*}
\end{rem}

\begin{defn}
	The \deff{center} of $\mathfrak{g}$ is 
	\begin{equation*} 
		\mathfrak{z} \coloneqq \{x \in \mathfrak{g} : [x, y] = 0 \text{ for all } y \in \mathfrak{g}\}.
	\end{equation*}
\end{defn}

\begin{thm}
	$\mathfrak{z}$ is an ideal of $\mathfrak{g}$. 
	$\mathfrak{z}$ is also an abelian subalgebra of $\mathfrak{g}$.
\end{thm}
The above follows by a direct computation. Alternately, $\mathfrak{z}$ is the kernel of the map $\ad$ to be defined below.

\subsection{Examples}

\begin{ex}
	$\gl(n, \mathbb{R}) \coloneqq M_{n}(\mathbb{R})$ is a Lie algebra with
	\begin{equation*} 
		[A, B] \coloneqq AB - BA.
	\end{equation*}
	We also define the subspace
	\begin{equation*} 
		\sl(n, \mathbb{R}) \coloneqq \{A \in \gl(n, \mathbb{R}) : \trace(A) = 0\}.
	\end{equation*}
	The above is a subalgebra. \newline
	In fact, something much stronger is true: $[\gl(n, \mathbb{R}), \gl(n, \mathbb{R})] \subset \sl(n, \mathbb{R})$.

	Similarly, the space of upper triangular and strictly upper triangular matrices are also subalgebras.
\end{ex}
Note that for $\gl$, and the (strictly) upper triangular matrices, the algebra $[\mathfrak{g}, \mathfrak{g}]$ was strictly smaller. 
We show that this is not always the case.

\begin{ex}[Derived subalgebra is whole algebra] \label{ex:sl-2-algebra-computation}
	Consider $\mathfrak{g} \coloneqq \sl(2, \mathbb{R})$, the algebra of $2 \times 2$ traceless matrices. 
	Then, $\mathfrak{g}$ is $3$-dimensional with a basis given by
	\begin{equation*} 
		h \coloneqq \two{1}{}{}{-1},\, e \coloneqq \two{}{1}{}{},\, f \coloneqq \two{}{}{1}{}.
	\end{equation*}
	One checks that
	\begin{align*} 
		[h, e] &= 2e, \\
		[h, f] &= -2f, \\
		[e, f] &= h.
	\end{align*}
	Thus, $[\mathfrak{g}, \mathfrak{g}] = \mathfrak{g}$.
\end{ex}

\begin{ex}[Lie algebras of small dimensions]
	Let $\mathfrak{g}$ be a Lie algebra. We classify what $\mathfrak{g}$ is for $\dim(\mathfrak{g}) \le 2$.
	\begin{itemize}
		\item $\dim(\mathfrak{g}) = 0$: $\mathfrak{g}$ must be abelian and thus, there is exactly one Lie algebra of dimension $0$.
		%
		\item $\dim(\mathfrak{g}) = 1$: since $[\alpha v, \beta v] = 0$ for every $v \in \mathfrak{g}$ and $\alpha, \beta \in F$, we see that $\mathfrak{g}$ is abelian. Thus, there is a unique Lie algebra of dimension $1$.
		%
		\item $\dim(\mathfrak{g}) = 2$: as noted before (\Cref{rem:abelian-lie-algebras-isomorphic}), there is exactly one abelian Lie algebra of any given dimension. 

		There is a unique two-dimensional non-abelian Lie algebra: it has a basis $e, f$ such that $[e, f] = f$.

		Thus, there are exactly two Lie algebras of dimension two (up to isomorphism).
	\end{itemize}
\end{ex}

\begin{ex}[Two-dimensional non-abelian Lie algebra as an algebra of matrices]
	The last example can be realised as a subgroup of $\gl(2, \mathbb{R})$ as
	\begin{equation*} 
		\mathfrak{g} \coloneqq \left\{\two{a}{b}{0}{0} : a, b \in \mathbb{R}\right\}.
	\end{equation*}
	The basis 
	\begin{equation*} 
		e \coloneqq \two{1}{0}{0}{0},\, f \coloneqq \two{0}{1}{0}{0}
	\end{equation*}
	satisfies $[e, f] = f$.
\end{ex}

A theorem of Ado tells that this can always be done:
\begin{thm}[Ado]
	Any finite-dimensional Lie algebra over $F$ is isomorphic to a Lie subalgebra of $\gl(n, F)$ for some $n$.
\end{thm}
The above is true for any field of characteristic zero.

\begin{rem}
	The corresponding theorem about Lie groups is not true: Consider $G \coloneqq \SL(2, \mathbb{R})$. 
	Note that we have a topological decomposition as
	\begin{align*} 
		\SL(2, \mathbb{R}) \cong S^{1} \times \mathbb{R}_{> 0} \times \mathbb{R}.
	\end{align*}
	This is because every element of $G$ has a unique representation of the form
	\begin{equation*} 
		\two{\cos(\theta)}{-\sin(\theta)}{\sin(\theta)}{\cos(\theta)} \cdot \two{r}{}{}{1/r} \cdot \two{1}{x}{0}{1}.
	\end{equation*}

	Thus, $\pi_{1}(G, I) \cong \mathbb{Z}$. 
	This means that $\widetilde{G} \to G$ is an infinite covering map. 
	This will tell us that $\widetilde{G}$ is not a group of matrices. ??
\end{rem}

\subsection{Representation of Lie algebras}

Analogous to $\gl(n, F)$, one defines the following.
\begin{defn}
	Given a vector space $V$, the Lie algebra $\gl(V)$ is defined as follows: this is the $F$-vector space $\End_{F}(V)$ with bracket $[S, T] \coloneqq S \circ T - T \circ S$.
\end{defn}

\begin{defn}
	A \deff{representation} of a Lie group $\mathfrak{g}$ on $V$ is a morphism $\pi \colon \mathfrak{g} \to \gl(V)$.
\end{defn}

\subsection{The adjoint representation and center}

Let $\mathfrak{g}$ be a Lie algebra, and $x \in \mathfrak{g}$. 
This defines a linear map 
\begin{align*} 
	\ad(x) \colon \mathfrak{g} & \to \mathfrak{g} \\
	z & \mapsto [x, z].
\end{align*}

This gives us a map $\ad \colon \mathfrak{g} \to \End(\mathfrak{g})$ given by $x \mapsto \ad(x)$. 
This map is linear.

The Jacobi identity tells us that
\begin{equation*} 
	\ad([x, y])(z) = (\ad(x) \circ \ad(y))(z) - (\ad(y) \circ \ad(x))(z).
\end{equation*}

Thus,
\begin{equation*} 
	\ad([x, y]) = [\ad(x), \ad(y)],
\end{equation*}
where the bracket on the right is in $\End(\mathfrak{g})$. 

In other words, we have
\begin{defn}
	The \deff{adjoint representation} of $\mathfrak{g}$ is $\ad \colon \mathfrak{g} \to \gl(\mathfrak{g})$. 
\end{defn}
This is a representation of $\mathfrak{g}$ on $\mathfrak{g}$.

Note that the center $\mathfrak{z}$ is precisely $\ker(ad)$ and hence is an ideal.

\subsection{Complexification}

If $\mathfrak{g}$ is an $\mathbb{R}$-Lie algebra, then $\mathfrak{g} \otimes_{R} \mathbb{C}$ is a $\mathbb{C}$-Lie algebra in a natural way. 
We have $\dim_{\mathbb{R}}(\mathfrak{g}) = \dim_{\mathbb{C}}(\mathfrak{g} \otimes_{\mathbb{R}} \mathbb{C})$.

We now see a curious example. 

Consider the \underline{real} Lie algebras $\su(2, \mathbb{C})$ and $\sl(2, \mathbb{R})$. 
We had already seen the latter; the former is
\begin{equation*} 
	\su(2) \coloneqq \left\{\two{\iota a}{z}{-\overline{z}}{-\iota a} : a \in \mathbb{R},\, z \in \mathbb{C}\right\},
\end{equation*}
with the bracket being the usual commutator of matrices.

Both of these are $3$-dimensional real Lie algebras. One can check by hand that these are not isomorphic: $\su(2) \not\cong \sl(2, \mathbb{R})$. \newline

\begin{blockquote}
	If we believe the introduction of this section, here's a more conceptual reason to see that the Lie algebras are distinct:

	As one may believe, these are the Lie algebras to be associated to $\SU(2)$ and $\SL(2, \mathbb{R})$ respectively. 
	These are non-isomorphic Lie groups. For example, one is compact and the other is not. 
	In fact, we have already seen the topological isomorphisms
	\begin{equation*} 
		\SU(2) \cong S^{3} \andd \SL(2, \mathbb{R}) \cong S^{1} \times \mathbb{R}_{> 0} \times \mathbb{R}.
	\end{equation*}
	This tells us that $\SU(2)$ is a simply-connected group and $\SL(2, \mathbb{R})$ is not. 
	In particular, $\SU(2)$ is its own universal cover and compactness tells us that this does not cover $\SL(2, \mathbb{R})$. 
\end{blockquote}

On the other hand, one has
\begin{equation*} 
	\su(2) \otimes_{R} \mathbb{C} \cong \sl(2, \mathbb{R}) \otimes_{\mathbb{R}} \mathbb{C}
\end{equation*}
as $\mathbb{C}$-Lie algebras.

To see this, note that $\su(2)$ has an $\mathbb{R}$-basis as
\begin{equation*} 
	A \coloneqq \two{\iota}{}{}{-\iota},\, B \coloneqq \two{0}{1}{-1}{0},\, C \coloneqq \two{0}{\iota}{\iota}{0}.
\end{equation*}
Let $h, e, f$ be the basis of $\sl(2, \mathbb{R})$ as in \Cref{ex:sl-2-algebra-computation}. 

We see that we have
\begin{equation*} 
	A = \iota h,\, B = e - f,\, C = \iota(e + f).
\end{equation*}

The above lets one conclude the desired isomorphism after complexification.

This example shows that the classification of Lie algebras may be a simpler task over $\mathbb{C}$.

\subsection{Lie algebras from Lie groups}

We finally define the Lie algebra associated to a Lie group. 
We will see that this gives us an object that captures more information about the group than just the tangent space.

Let $G$ be a Lie group, and $g \in G$. 
We have the conjugation map $\INT(g) \colon G \to G$. 
We define $\Ad(g) \coloneqq T_{1}(\INT(g))$. 
Note that is a linear map
\begin{equation*} 
	\Ad(g) \colon T_{1}(G) \to T_{1}(G).
\end{equation*}
This gives us a smooth map
\begin{equation*} 
	\Ad \colon G \to \End(T_{1}(G)).
\end{equation*}
Differentiating this at $1$ and identifying the tangent space of a vector space with itself gives us
\begin{equation} \label{eq:t1-ad-t1-G-end-t1-G}
	T_{1}(\Ad) \colon T_{1}(G) \to \End(T_{1}(G)).
\end{equation}
Thus, given $\xi, \eta \in T_{1}(G)$, we have a linear map
\begin{equation*} 
	T_{1}(\Ad)(\xi) \colon T_{1}(G) \to T_{1}(G).
\end{equation*}
Thus, we have an element
\begin{equation*} 
	T_{1}(\Ad)(\xi)(\eta) \in T_{1}(G).
\end{equation*}

We define the bracket on $T_{1}(G)$ as
\begin{equation*} 
	[\xi, \eta] \coloneqq T_{1}(\Ad)(\xi)(\eta).
\end{equation*}

Alternately, we could (un)curry the function \Cref{eq:t1-ad-t1-G-end-t1-G} to get a bilinear map
\begin{equation*} 
	T_{1}(\Ad)(-, -) \colon T_{1}(G) \times T_{1}(G) \to T_{1}(G).
\end{equation*}
The above bilinear map is precisely what we defined $[-, -]$ to be.

\begin{thm}
	$(T_{1}(G), [-, -])$ is a real Lie algebra.
\end{thm}
\begin{thm}
	If $\varphi : G \to H$ is a map of Lie groups, then the induced map
	\begin{equation*} 
		T_{1}(\varphi) \colon T_{1}(G) \to T_{1}(H)
	\end{equation*}
	is a map of Lie algebras.
\end{thm}

A useful step in the above proofs is the following observation: 
given $\varphi$ as above, and $g \in G$, the diagram
\begin{equation*} 
	\begin{tikzcd}
		G \arrow[r, "\varphi"] \arrow[d, "\INT_{G}(g)"'] & H \arrow[d, "\INT_{H}(\varphi(g))"] \\
		G \arrow[r, "\varphi"'] & H
	\end{tikzcd}
\end{equation*}
commutes. Taking differentials at $1_{G}$ gives us
\begin{equation*} 
	\Ad_{H}(\varphi(g)) \circ T_{1}(\varphi) = T_{1}(\varphi) \circ \Ad_{G}(g)
\end{equation*}
as maps $T_{1}(G) \to T_{1}(H)$.

The rest of this subsection is proving the above theorems.

\begin{sketch}
	We give a sketch of both the above theorems. 
	The second theorem is proven first (by showing that $T_{1}(\varphi)$ preserves brackets). 
	We will make use of \mynameref{subsec:vector-spaces-manifolds} to make different identifications.

	For $\eta \in T_{1}(G)$ and $g \in G$, we have
	\begin{equation*} 
		\Ad_{H}(\varphi(g))(T_{1}(\varphi)(\eta)) = T_{1}(\varphi)(\Ad_{G}(g)(\eta)).
	\end{equation*}
	Keeping $\eta$ fixed, we may view both sides as functions $G \to T_{1}(H)$.

	The left side is the composition
	\begin{equation*} 
		G \xrightarrow{\varphi} H \xrightarrow{\Ad_{H}} \End(T_{1}(H)) \xrightarrow{\eval(T_{1}(\varphi)(\eta))} T_{1}(H),
	\end{equation*}
	and the right is
	\begin{equation*} 
		G \xrightarrow{\Ad_{G}} \End(T_{1}(G)) \xrightarrow{\eval(\eta)} T_{1}(G) \xrightarrow{T_{1}(\varphi)} T_{1}(H).
	\end{equation*}

	Since the above functions are equal, we may differentiate them both at $1_{G}$. 
	To do so, we use the chain rule and remember that linear maps between vector spaces can be identified with their own differentials. 
	Thus, the two derivatives respectively are
	\begin{equation*} 
		T_{1}(G) \xrightarrow{T_{1}(\varphi)} T_{1}(H) \xrightarrow{T_{1}(\Ad_{H})} \End(T_{1}(H)) \xrightarrow{\eval(T_{1}(\varphi)(\eta))} T_{1}(H),
	\end{equation*}
	and
	\begin{equation*} 
		T_{1}(G) \xrightarrow{T_{1}(\Ad_{G})} \End(T_{1}(G)) \xrightarrow{\eval(\eta)} T_{1}(G) \xrightarrow{T_{1}(\varphi)} T_{1}(H).
	\end{equation*}

	Evaluating both these functions at $\xi \in T_{1}(G)$ gives us the equality
	\begin{equation*} 
		\left(T_{1}(\Ad_{H})(T_{1}(\varphi)(\xi))\right)\left(T_{1}(\varphi)(\eta)\right) = T_{1}(\varphi)((T_{1}(\Ad_{G})(\xi))(\eta)).
	\end{equation*}
	Recalling $T_{1}(\Ad_{G})(-)(-) = [-, -]$ gives us
	\begin{equation*} 
		[T_{1}(\varphi)(\xi), T_{1}(\varphi)(\eta)] = T_{1}(\varphi)([\xi, \eta]).
	\end{equation*}
	This shows that $T_{1}(\varphi)$ preserves brackets.

	Jacobi identity: 

	Fix $g \in G$. Since $\INT(g) \colon G \to G$ is a homomorphism, we now know that
	\begin{equation*} 
		\Ad(g)\left([\xi, \eta]\right) = [\Ad(g)(\xi), \Ad(g)(\eta)]
	\end{equation*}
	for all $\xi, \eta \in T_{1}(G)$.

	As before, fix $\xi$ and $\eta$ and both sides above as functions of $G$ into $T_{1}(G)$. 
	The left side is
	\begin{equation*} 
		G \xrightarrow{\Ad} \End(T_{1}(G)) \xrightarrow{\eval([\xi, \eta])} T_{1}(G)
	\end{equation*}
	and the right is
	\begin{equation*} 
		G \xrightarrow{\Ad} \End(T_{1}(G)) \xrightarrow{(\eval(\xi), \eval(\eta))} T_{1}(G) \times T_{1}(G) \xrightarrow{[-, -]} T_{1}(G).
	\end{equation*}

	As before, we identify and differentiate this functions at $1_{G}$. 
	For the second function, we use \Cref{eq:product-rule-bilinear-map} to differentiate the bilinear map.

	We get the respective derivatives as
	\begin{equation*} 
		T_{1}(G) \xrightarrow{T_{1}(\Ad)} \End(T_{1}(G)) \xrightarrow{\eval([\xi, \eta])} T_{1}(G)
	\end{equation*}
	and
	\begin{equation*} 
		T_{1}(G) \xrightarrow{T_{1}(\Ad)} \End(T_{1}(G)) \xrightarrow{(\eval(\xi), \eval(\eta))} T_{1}(G) \oplus T_{1}(G) \xrightarrow{T_{1}([-, -])} T_{1}(G).
	\end{equation*}
	Evaluating these maps at $\zeta \in T_{1}(G)$ gives us
	\begin{equation*} 
		(T_{1}(\Ad)(\zeta))([\xi, \eta]) = (T_{1}(\Ad)(\zeta))(\xi), \eta) + (\xi, T_{1}(\Ad)(\zeta))(\eta)).
	\end{equation*}
	Rewriting $T_{1}(Ad)(-)(-)$ as $[-, -]$ gives us
	\begin{equation*} 
		[\zeta, [\xi, \eta]] = [[\zeta, \xi], \eta] + [\xi, [\zeta, \eta]].
	\end{equation*}
	This would give us Jacobi identity, if we knew anticommutativity.

	Anticommutativity: 

	We define a function $\ast$ on $T_{1}(G) \times T_{1}(G)$. 
	We will show that 
	\begin{equation*} 
		[\xi, \eta] = \xi \ast \eta - \eta \ast \xi,
	\end{equation*} 
	proving anticommutativity. 
	(Both sides of the equation are to be viewed as linear operators on $\mathcal{C}^{\infty}(G)$.)

	For this, we will consider the realisation of $T_{1}(G)$ described in \Cref{subsec:tangent-spaces}. 
	Let $\xi, \eta \in T_{1}(G)$ be derivations at $1$ and $f \in \mathcal{C}^{\infty}(G)$ be arbitrary. 
	We wish to define $(\xi \ast \eta)(f) \in \mathbb{R}$. 

	Fix $g \in G$. We then get a function $(f \circ m)(g, -) \in \mathcal{C}^{\infty}(G)$. 
	We can apply the derivation $\eta$ to this function. 
	This gives us a real number (dependent on $g$), which we denote by $(f \circ m)^{\eta}(g)$. 
	But now this gives a (smooth!) function $(f \circ m)^{\eta} \colon G \to \mathbb{R}$. 
	Thus, we may apply $\xi$ to this function to get a real number
	\begin{equation*} 
		\xi\left((f \circ m)^{\eta}\right) \in \mathbb{R}.
	\end{equation*}

	Note that we could have also done this in the other order to get something like
	\begin{equation*} 
		\eta\left((f \circ m)_{\xi}\right).
	\end{equation*}

	Equality of mixed partials tells us that both these numbers are the same. 
	We denote this quantity by
	\begin{equation*} 
		(\xi \ast \eta)(f).
	\end{equation*}

	\textbf{Claim.} $[\xi, \eta] = \xi \ast \eta - \eta \ast \xi$. \newline
	Note that this claim proves the result.


\end{sketch}

\section{Matrix Lie groups and properties}

In this section, we compile the list of examples and their properties that we saw across various sections. 
We give references to the earlier sections where we verified the properties mentioned.

We mention which groups are compact. This is an easy check using Heine-Borel.

\begin{enumerate}[label=(\alph*)]
	\item $\GL(n, \mathbb{R})$ is a Lie group of dimension $n^{2}$.
	\item $\GL(n, \mathbb{C})$ is a Lie group of dimension $2n^{2}$.
	\item $\GL(n, \mathbb{H})$ is a Lie group of dimension $4n^{2}$. \newline
	See \Cref{ex:GL-examples} for the above three.
	%
	\item $\SL(n, \mathbb{R})$ is a Lie group of dimension $(n^{2} - 1)$.
	\item $\SL(n, \mathbb{C})$ is a Lie group of dimension $2(n^{2} - 1)$. \newline
	See \Cref{subsec:SL-subgroup} for the above two.
	%
	\item $\OO(n)$ is a compact Lie group of dimension $n(n - 1)/2$.
	\item $\UU(n)$ is a connected, compact Lie group of dimension $n^{2}$. \newline
	See \Cref{ex:orthogonal-unitary-groups-dimensions} for the above two.
	%
	\item $\SO(n)$ is a connected, compact Lie group of dimension $n(n - 1)/2$.
	\item $\SU(n)$ is a connected, compact Lie group of dimension $n^{2} - 1$. \newline
	See \Cref{ex:special-orthogonal-unitary-groups-dimensions} for the above two. For connectedness, see \Cref{sec:connectedness}.
	%
	\item $\Sp(n)$ is a connected, compact Lie group of dimension $n(2n + 1)$. See \Cref{subsec:sympletic-group,sec:connectedness}.
\end{enumerate}

\end{document}