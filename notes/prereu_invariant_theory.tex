\documentclass[12pt]{article}
\usepackage[lmargin=1in,rmargin=1in,tmargin=1in,bmargin=1in]{geometry}

\usepackage{aryaman}
\setcounter{tocdepth}{2}

\DeclareMathOperator{\Isom}{Isom}

\title{Classical Invariant Theory}
\author{Pre-REU Day 11}
\date{\today}

% \usepackage[
% 	hyperref = true,      	% Link to online documents
%   	backend  = bibtex,      % Use bibtex instead of biber
%   	sorting  = nyt,       	% Sorts by (name, year, title)
%   	style  = alphabetic 	% Citations look like [Har77]
% ]{biblatex}
% \addbibresource{source.bib}

\begin{document}

\maketitle
% \tableofcontents

\section*{Introduction} 

	So far, we have seen groups as ``acting'' on the plane $\mathbb{R}^{2}$, 
	i.e., 
	our groups were transformation subgroups of $\Sym(\mathbb{R}^{2})$. 
	Moreover, we focused our attention to groups preserving structure, 
	e.g., 
	by considering only the isometries. 
	In this lecture, we will consider the action of matrix groups on \emph{rings}. 
	We will focus our attention on the so-called \emph{classical groups} with their ``classical representations'' and look at the \emph{invariant} rings. 

\section{The classical matrix groups}
	
	By a classical group, we shall mean one of the following:
	\begin{itemize}
		\item (General linear group) The group $\GL_{n}(\mathbb{R})$ of invertible $n \times n$ matrices.
		%
		\item (Special linear group) The group $\SL_{n}(\mathbb{R})$ of $n \times n$ matrices with determinant one.
		%
		\item (Orthogonal group) The group $\OO_{n}(\mathbb{R})$ of $n \times n$ matrices satisfying $M M^{\trans} = I$.
		%
		\item (Symplectic group) The group $\Sp_{2n}(\mathbb{R})$ of $2n \times 2n$ invertible matrices satisfying $M \Omega M^{\trans} = \Omega$, 
		where $\Omega$ is the $2n \times 2n$ block matrix given as 
		$\Omega = \smatrix{0 & I \\ -I & 0}$. 
	\end{itemize}
	
	The general linear group is the most general group of invertible matrices; 
	each of the other can be thought of as preserving something:
	\begin{itemize}
		\item The special lienar group preserves the ($n$-dimensional) volume.
		%
		\item As we saw in class, the orthogonal group is the group of matrices that preserves the usual dot \underline{product} (and this was equivalent to preserving the lengths, thanks to the parallelogram law). 
		%
		\item The symplectic group arises as the group of matrices that preserves a certain other kind of \underline{product}. 
		Why is this product relevant? 
		As per Wikipedia\footnote{\url{https://en.wikipedia.org/wiki/Symplectic_group}}: ``[The symplectic group] comes up in classical physics as the symmetries of canonical coordinates preserving the Poisson [product].''
	\end{itemize}

\section{Action on rings}
	
	Now, let us consider the set 
	$R = \mathbb{R}[x_{1}, \ldots, x_{n}]$---this is the set of all polynomials in the variables $x_{1}, \ldots, x_{n}$. 
	Furthermore, the set $R$ is equipped with the two natural binary operations of addition and multiplication that satisfy certain axioms such as commutativity, associativity, distributivity, and having the necessary identities and inverses---this makes $R$ a \emph{ring}. 

	Given an $n \times n$ matrix $M$, it defines a function on $R$. 
	Let us consider this by means of an example: say $n = 2$, and we write $R = \mathbb{R}[x, y]$ and consider $M = \smatrix{1 & 3 \\ 0 & 1}$. 
	First, we define the function on the variables $x$ and $y$ using matrix multiplication as following: 
	we have
	\begin{equation*} 
		M 
		\begin{bmatrix}
			x \\
			y \\
		\end{bmatrix}
		=
		\begin{bmatrix}
			x + 3y \\
			y \\
		\end{bmatrix}.
	\end{equation*}
	So, $M(x) = x + 3y$ and $M(y) = y$. 
	We now extend this to all polynomials by extending it ``polynomially''. 
	For example, if we have the polynomial 
	$f = 10 + 2 x + 3 x y - y^{2}$, then we define
	\begin{align*} 
		M(f) &= M(10 + 2 x + 3 x y - y^{2}) \\
		&= 10 + 2 M(x) + 3 M(x) M(y) - M(y)^{2}.
	\end{align*}
	We can now use the values of $M(x)$ and $M(y)$ from before to compute the above. 

	By defining the action in this way, we get that $M$ acts on $R$ via \emph{ring homomorphisms}---the action of $M$ preserves the addition and multiplication operations. 

	We could generalise the above further: say $M$ is as $n \times n$ matrix,
	and $R$ is a polynomial ring in $nm$ variables for some $n, m \ge 1$. 
	Then, we can again define an action of $M$ on $R$ by arranging the variables in an $n \times m$ matrix and proceeding as before. 

	Let us look at an example of this, say 
	$R = \mathbb{R}[x_{1}, x_{2}, y_{1}, y_{2}]$ is a polynomial ring in four variables, and 
	$M = \smatrix{1 & 3 \\ 0 & 1}$ as before. 
	We then have
	\begin{equation*} 
		M
		\begin{bmatrix}
			x_{1} & x_{2} \\
			y_{1} & y_{2} \\
		\end{bmatrix}
		=
		\begin{bmatrix}
			x_{1} + 3 y_{1} & x_{2} + 3 y_{2} \\
			y_{1} & y_{2} \\
		\end{bmatrix}.
	\end{equation*}
	In particular, $M(y_{1}) = y_{1}$ and $M(y_{2}) = y_{2}$, giving us two \emph{fixed points}. 
	Are there others? 

	A first source of fixed points is ``any polynomial in $y_{1}$ and $y_{2}$'', e.g., 
	$y_{1}^{2} y_{2}$ because we have 
	$M(y_{1}^{2} y_{2}) = M(y_{1})^{2} M(y_{2}) = y_{1}^{2} y_{2}$.

	Let us try to come up with fixed points not of this form. 
	We have
	\begin{align*} 
		M(x_{1} y_{2}) = M(x_{1}) M(y_{2}) &= (x_{1} + 3y_{1}) y_{2} = x_{1} y_{2} + 3 y_{1} y_{2}, \\
		M(x_{2} y_{1}) = M(x_{2}) M(y_{1}) &= (x_{2} + 3y_{2}) y_{1} = x_{2} y_{1} + 3 y_{2} y_{1}.
	\end{align*}
	Can we get an invariant from the above two equations? 
	Yes! We have
	\begin{equation*} 
		M(x_{1} y_{2} - x_{2} y_{1}) = M(x_{1} y_{2}) - M(x_{2} y_{1}) 
		= (x_{1} y_{2} + 3 y_{1} y_{2}) - (x_{2} y_{1} + 3 y_{2} y_{1} )
		= x_{1} y_{2} - x_{2} y_{1}.
	\end{equation*}

	Thus, the determinant is a fixed point or an \emph{invariant}. This is not surprising: the matrix $M$ that we started out with, had determinant $1$. 
	Invariant theory is concerned with the study of the ring of invariants. 
	Instead of studying the fixed points of one given function, we will instead consider a group of functions.

	\begin{defn}
		Let $G$ be a group acting on a ring $R$. 
		The \deff{ring of invariants} is defined as
		\begin{equation*} 
			R^{G} \coloneqq \{r \in R : g(r) = r \text{ for all } g \in G\}.
		\end{equation*}
	\end{defn}

	As we will see in the problem set, $R^{G}$ itself turns out to be a ring. 
	The problems in invariant theory can be broadly described as:
	\begin{enumerate}[label=(\roman*)]
		\item Finding a (preferably optimal) set of generators for $R^{G}$.
		%
		\item Finding what the ``relations'' between those generators are.
		\item Finding what good properties of $R$ are inherited by $R^{G}$.
		%
	\end{enumerate}

	Our results today will focus on the first, with some allusion to the other two.

\section{Classical invariant rings}

	We shall now see the invariants for the classical actions of the classical groups. 

	\begin{thm}
		Let $n, m \ge 1$ be integers, 
		$R \coloneqq \mathbb{R}[X_{n \times m}]$ a polynomial ring in $n m$ variables, 
		and $G \coloneqq \SL_{n}(\mathbb{R})$. 
		Consider the action of $G$ on $R$ in the manner described earlier. 
		The fixed subring is then given as
		\begin{equation*} 
			R^{G} = \mathbb{R}[\Delta : \Delta \text{ is an $n \times n$ minor of $X$}].
		\end{equation*}
	\end{thm}

	Let us parse the above with the help of an example:

	\begin{ex}
		Let us consider $n = 2$ and $m = 2$. 
		Let us denote our variables as
		\begin{equation*} 
			\begin{bmatrix}
				x_{1} & x_{2} \\
				y_{1} & y_{2} \\
			\end{bmatrix}.
		\end{equation*}
		Then, there is exactly one $2 \times 2$ minor, 
		namely $\Delta \coloneqq x_{1} y_{2} - x_{2} y_{1}$. 
		The theorem then reads:
		\begin{equation*} 
			R^{G} = \mathbb{R}[x_{1} y_{2} - x_{2} y_{1}].
		\end{equation*}
		This means that any invariant can be written as a polynomial in the $\Delta$. 
		In this case, $R^{G}$ is particularly nice---it is itself again (isomorphic to) a polynomial ring in one variable.
	\end{ex}

	\begin{ex}
		Let us now consider $n = 2$ and $m = 4$. 
		Let us denote our variables as
		\begin{equation*} 
			\begin{bmatrix}
				x_{1} & x_{2} & x_{3} & x_{4} \\
				y_{1} & y_{2} & y_{3} & y_{4} \\
			\end{bmatrix}.
		\end{equation*}
		Now, there are exactly six minors: 
		let $\Delta_{i, j} = x_{i} y_{j} - x_{j} y_{i}$ 
		denote the determinant of the matrix obtained by considering the columns $i$ and $j$. 

		Then, the theorem reads
		\begin{equation*} 
			R^{G} = \mathbb{R}
			[\Delta_{12}, \Delta_{13}, \Delta_{14}, 
			\Delta_{23}, \Delta_{24}, \Delta_{34}].
		\end{equation*}

		This means that any invariant can be written as a polynomial in the $\Delta_{ij}$s. 
		However, there are \emph{relations}, that is to say, some invariant possibly be written as a polynomial in the $\Delta_{ij}$s in two different ways. 
		One may verify that
		\begin{equation*} 
			\Delta_{12} \Delta_{34} + \Delta_{14} \Delta_{23} = \Delta_{13} \Delta_{24}.
		\end{equation*}
		In particular, $R^{G}$ is \emph{not} isomorphic to the polynomial ring in $6$ variables. 
	\end{ex}

	Note that the inclusion $\mathbb{R}[\{\Delta\}] \subset \mathbb{R}^{G}$ is not too difficult: 
	we saw that elements of $\SL_{2}(\mathbb{R})$ will fix the determinant. 
	The remarkable fact is that these determinants give us all the invariants.

	The theme is repeated in the descriptions for the other invariant rings as well: the ``obvious'' invariants give us all.

	\begin{thm}
		Consider the action of $G \coloneqq \OO_{n}(\mathbb{R})$ on $R \coloneqq \mathbb{R}[X_{n \times m}]$. 
		The invariant ring is given as
		\begin{equation*} 
			R^{G} = \mathbb{R}[X^{\trans} X],
		\end{equation*}
		i.e., the generators are the entries of the matrix $X^{\trans} X$.
	\end{thm}

	Once again, let us see a sketch of why the inclusion 
	$\mathbb{R}[X^{\trans} X] \subset R^{G}$ is true: 
	consider a matrix $M \in \OO_{n}(\mathbb{R})$. 
	Then, we have
	\begin{equation*} 
		X^{\trans} X \xmapsto{M} (M X)^{\trans} (M X) 
		= X^{\trans} (M^{\trans} M) X
		= X^{\trans} X.
	\end{equation*}

	\begin{ex}
		Consider the case $n = 2$ and $m = 1$. 
		If we call the variables $x$ and $y$, then the theorem reads $R^{G} = \mathbb{R}[x^{2} + y^{2}]$. 
		This matches our geometric intuition of the orthogonal group preserving the norm. 

		More generally, if $n \ge 1$ is arbitrary and $m = 1$, we get the invariant ring as being generated by $x_{1}^{2} + \cdots + x_{n}^{2}$. 

		In the problem set, we will see how in the general case, the entries of $X^{\trans} X$ can be interpreted as dot products. 
		Thus, we recover the familiar fact about $\OO_{n}(\mathbb{R})$ preserving the dot product.
	\end{ex}

	We get a similar result for its symplectic cousin.

	\begin{thm}
		Consider the action of $G \coloneqq \Sp_{2n}(\mathbb{R})$ on $R \coloneqq \mathbb{R}[X_{2n \times m}]$. 
		The invariant ring is given as
		\begin{equation*} 
			R^{G} = \mathbb{R}[X^{\trans} \Omega X],
		\end{equation*}
		i.e., the generators are the entries of the matrix $X^{\trans} \Omega X$.
	\end{thm}

	For the group $\GL_{n}(\mathbb{R})$, one typically considers a more complicated action. 
	(In the problem set, we will see that for the usual left multiplication action, the ring of invariants is somewhat ``boring''.)

	\begin{thm}
		Consider the action of the general linear group $G \coloneqq \GL_{n}(\mathbb{R})$ on the polynomial ring 
		$R \coloneqq \mathbb{R}[X_{m \times n}, Y_{n \times p}]$, 
		where $M \in G$ acts as
		\begin{equation*} 
			M \colon 
			\begin{cases}
				X \mapsto X M^{-1}, \\
				Y \mapsto M Y.
			\end{cases}
		\end{equation*}
		The invariant ring is given as
		\begin{equation*} 
			R^{G} = \mathbb{R}[X Y],
		\end{equation*}
		i.e., the generators are the entries of the matrix $X Y$.
	\end{thm}

	Here's another interesting group action, though usually not under the umbrella of the classical group actions:
	\begin{thm}
		Consider the group $G \coloneqq \GL_{n}(\mathbb{R})$ acting on $R \coloneqq \mathbb{R}[X_{n \times n}]$ via conjugation, i.e., 
		$M \in G$ acts via $X \mapsto M^{-1} X M$. 
		Then, the invariant ring is given as
		\begin{equation*} 
			R^{G} = \mathbb{R}[\trace(X), \ldots, \det(X)],
		\end{equation*}
		i.e., the generators are the coefficients of the characteristic polynomial of $X$. 
	\end{thm}

	One question of interest to me is: what if we replace $\mathbb{R}$ with a finite field such as $\mathbb{Z}/5$? What are the rings of invariants in that case? 

% \printbibliography
\end{document}