\documentclass[12pt]{article}
\usepackage{amsmath, amssymb, amsfonts, amsthm, mathtools}
\usepackage{thmtools}
\usepackage[utf8]{inputenc}
\usepackage[inline]{enumitem}
\usepackage[colorlinks=true]{hyperref}
\setlength\parindent{0pt}

\theoremstyle{definition}
\newtheorem{thm}{Theorem}
\numberwithin{thm}{section}
\newtheorem{lem}[thm]{Lemma}
\newtheorem{defn}[thm]{Definition}
\newtheorem{prop}[thm]{Proposition}
\newtheorem{cor}[thm]{Corollary}
\newtheorem{ex}{Example}


\let\emptyset\varnothing
\newcommand{\id}{\operatorname{id}}
\newcommand{\hint}[1]{\textbf{HIDDEN:} {\color[rgb]{0.95, 0.95, 0.95}#1}}

\pagestyle{plain}

\usepackage{titlesec}
\titleformat{\section}[block]{\sffamily\Large\filcenter\bfseries}{\S\thesection.}{0.25cm}{\Large}
\titleformat{\subsection}[block]{\large\bfseries\sffamily}{\S\S\thesubsection.}{0.2cm}{\large}


\usepackage[a4paper]{geometry}
\usepackage{lipsum}

\usepackage{cleveref}
\crefname{thm}{Theorem}{Theorems}
\crefname{lem}{Lemma}{Lemmas}
\crefname{defn}{Definition}{Definitions}
\crefname{prop}{Proposition}{Propositions}
\crefname{cor}{Corollary}{Corollaries}
\crefname{equation}{}{}

\usepackage{mdframed}
\newenvironment{blockquote}
{\begin{mdframed}[skipabove=0pt, skipbelow=0pt, innertopmargin=4pt, innerbottommargin=4pt, bottomline=false,topline=false,rightline=false, linewidth=2pt]}
{\end{mdframed}}

\usepackage{fancyhdr}
\pagestyle{fancy}
\fancyhf{}
\fancyhead[L]{\sffamily{\S\textbf{\nouppercase{\leftmark}}}}
\fancyhead[R]{\sffamily{\thepage}}

\renewcommand{\familydefault}{\sfdefault}

\title{$\mathbb{R}$eal Analysis}
\author{Aryaman Maithani\\\url{https://aryamanmaithani.github.io/}}
\date{Autumn Semester 2020-21}

\begin{document}
\maketitle
\setcounter{section}{-1}
\tableofcontents
\newpage\section{Notations} \label{sec:notations}
\begin{enumerate}
	\item $\mathbb{N} = \{1, 2, 3, \ldots\},$ the set of positive integers.
	\item $\mathbb{Z}$ is the set of integers.
	\item $\mathbb{Q}$ is the set of rational numbers.
	\item $\mathbb{R}$ is the set of real numbers.
	\item Whenever I refer to $\mathbb{R}$ or $\mathbb{R}^n$ or any of its subsets as a metric space, I shall always assume the standard metric unless I explicitly state so otherwise.
	\item $S^1 = \{\mathbf{x} \in \mathbb{R}^2 \mid \|x\| = 1\} \subset \mathbb{R}^2.$ 
	% \item $\mathbb{S}_{\ge 0} = \{s \in \mathbb{S} \mid s \ge 0\}.$ (So, that defines $\mathbb{Z}_{\ge 0}, \mathbb{Q}_{\ge 0}, \mathbb{R}_{\ge 0}.$)
	% \item $\mathbb{S}^+ = \mathbb{S}_{> 0} = \{s \in \mathbb{S} \mid s > 0\}.$
	\item $A \subset B$ is read as ``$A$ is a subset of $B.$'' In particular, note that $A \subset A$ is true for any set $A.$
	\item $A \subsetneq B$ is read ``$A$ is a \emph{proper} subset of $B.$''
	\item $\supset$ and $\supsetneq$ are defined similarly.
	\item Given a set $S,$ the set $\mathcal{P}(S)$ is the \emph{power set} of $S,$ i.e., the set of all subsets of $\mathcal{P}(S).$
	\item Given a function $f:X \to Y,$ $A \subset X,$ $B \subset Y,$ we define
	\begin{align*} 
		f(A) &= \{y \in Y \mid y = f(a) \text{ for some } a \in A\} \subset Y,\\
		f^{-1}(B) &= \{x \in X \mid f(x) \in B\} \subset X.
	\end{align*}
	(Note that this $f^{-1}$ is different from the inverse of a function. In particular, this is always defined, even if $f$ is not bijective. However, the $f$ and $f^{-1}$ above need not be ``inverses.'' See \ref{q:funcinvproperties} of \S\ref{sec:sets}.)\\
	\emph{Remark.} The above is essentially an abuse of notation since we are using $f:A \to B$ to get another function $\tilde{f}:\mathcal{P}(A) \to \mathcal{P}(B)$ which we are again denoting with $f.$ 
	\item \label{funcrestrict} If $f:X \to Y$ is a function and $A \subset X,$ then $f|_A$ is a function
	\begin{equation*} 
		f|_A : A \to Y
	\end{equation*}
	defined as
	\begin{equation*} 
		f|_A(a) = f(a), \quad a \in A.
	\end{equation*}
	\item Since Rudin follows a non-usual definition for ``countable,'' I shall use the following, which makes it always clear:
	\begin{enumerate}
		\item At most countable: A set $S$ is at most countable if there exists an injection $i : S \to \mathbb{N}.$
		\item Countably infinite: A set $S$ is countably infinite if it is at most countable and infinite.
		\item Uncountable: A set $S$ is uncountable if it not at most countable.
	\end{enumerate}
	In particular, I will not use the term ``countable'' just by itself since Rudin uses it to mean ``countably infinite'' but usually people mean ``at most countable.''
	%
	\item Given a set $I,$ $\{P_\alpha\}_{\alpha \in I}$ is a shorthand for writing a set of the form $\{P_\alpha \mid \alpha \in I\}.$ ($P_\alpha$ is defined given the context.)
\end{enumerate}
\newpage\section{Sets and stuff}\label{sec:sets}
\begin{enumerate}
	%
	\item Suppose $S$ is an ordered set. Let $E_1 \subset E_2 \subset S$ be such that $E_1$ when considered as a subset of $E_2$ has a least upper bound in $E_2.$\\
	Is it necessary that $E_1$ being considered as a subset of $S$ must also have a least upper bound?\\
	\hint{No. Give a counterexample.}
	%
	\item Let $S$ be an ordered set and $E \subset S$ be a set such that $\sup E$ exists (in $S$).\\
	Let $s \in S$ be such that
	\begin{equation*} 
		e < s \quad \text{ for all } e \in E.
	\end{equation*} 
	Show that
	\begin{equation*} 
		\sup E \le s.
	\end{equation*}
	Give an example to show that $\sup E < s$ is possible. Show that this happens precisely when $\sup E \notin E.$
	%
	\item \label{q:funcinvproperties}% 
	Given a function $f:X \to Y,$ $A \subset X,$ $B \subset Y,$ we define
	\begin{align*} 
		f(A) &= \{y \in Y \mid y = f(a) \text{ for some } a \in A\} \subset Y,\\
		f^{-1}(B) &= \{x \in X \mid f(x) \in B\} \subset X.
	\end{align*}
	\begin{enumerate}
		\item Show that $A \subset f^{-1}(f(A)).$ \\
		Show that this inclusion can be proper. \\
		Thus, it is possible that $A \neq f^{-1}(f(A)).$ \\
		Show that equality holds if $f$ is injective.
		\item Show that $B \supset f(f^{-1}(B)).$ \\
		Show that this inclusion can be proper. \\
		Thus, it is possible that $B \neq f(f^{-1}(B)).$ \\
		Show that equality holds if $f$ is surjective.
	\end{enumerate}
	%
	\item Let $i:A \to B$ and $j:B\to A$ be injections.\\
	Show that there exists a bijection between $A$ and $B.$\\
	\emph{Remark.} This is known as the \href{https://en.wikipedia.org/wiki/Schr%C3%B6der%E2%80%93Bernstein_theorem}{Schröder–Bernstein theorem}. (The link has a proof of it as well.)
	%
	\item Show that if $S$ is infinite, then there is an injection $i:\mathbb{N} \to S.$
	%
	\item Show that if $S$ is infinite and if there exists an injection $j:S \to \mathbb{N},$ then $S$ is at countably infinite.
	%
	\item Let $C$ be a countably infinite set. Show that if $S$ is infinite and if there exists an injection $j:S \to C,$ then $S$ is countably infinite.
	%
	\item Show that $\mathbb{Q}$ is countably infinite.
	%
	\item Show that if $A$ is at most countable, then so is $A \times A.$ Conclude that $A^n$ is at most countable for all $n \ge 1.$
	%
	\item Show that $\mathbb{Q}^n$ is countably infinite for all $n \ge 1.$
	%
	\item Let $\{0, 1\}^\mathbb{N}$ be the set of all sequences with entries from $\{0, 1\}.$\\
	In other words, $\{0, 1\}^\mathbb{N}$ is the set of all functions from $\mathbb{N}$ to $\{0, 1\}.$\\
	Show that $\{0, 1\}^\mathbb{N}$ is uncountable.
	%
	\item Show that $[0, 1]$ is uncountable. (Hence, so is $\mathbb{R}.$)
	%
	\item Show that there exists a bijection between any two of the following sets: 
	\begin{equation*} 
		(0, 1),\; [0, 1],\; (0, 1],\; \mathbb{R},\; \mathbb{R}\setminus\mathbb{Q}.
	\end{equation*}
	%
	\item Show that there exists a bijection between $\mathcal{P}(\mathbb{N})$ and $\mathbb{R},$ where $\mathcal{P}(\mathbb{N})$ is the power set of $\mathbb{N}.$\\
	(You can use properties such as binary/ternary expansions.)
\end{enumerate}
\newpage\section{Topology}
\begin{enumerate}
	\item Show that $\mathbb{Q}$ has empty interior considered as a subset of $\mathbb{R}.$
	%
	\item Given a metric space $(X, d),$ show that $X^\circ = X.$\\~\\
	(Compare this with the earlier question to conclude that the interior of a set is not an ``inherent'' property of a space but depends on the bigger space. This is just like ``$U$ is open/closed'' depends on the bigger set.)
	%
	\item Just to illustrate the above point better, prove the following:
	\begin{enumerate}
		\item $\mathbb{Q}$ is neither open nor closed in $\mathbb{R}.$
		\item $\mathbb{Q}$ is both open and closed in $\mathbb{Q}.$
		\item $[0, 1]$ is closed but not open in $\mathbb{R}.$
		\item $[0, 1]$ is closed in $I$ for any $I$ satisfying $[0, 1] \subset I \subset \mathbb{R}.$
		\item $(0, 1]$ is neither open nor closed in $\mathbb{R}.$
		\item $(0, 1]$ is open but not closed in $(-\infty, 1].$
		\item $(0, 1]$ is closed but not open in $(0, \infty).$
		\item $(0, 1]$ is both open and closed in $(0, 1] \cup [3, 5].$
	\end{enumerate}
	In each case, find its closure and interior in the given space. (Keep in mind that if it's closed (resp., open), then its closure (resp., interior) must equal itself.)
	%
	\item As opposed to the above, being connected/compact does not depend on the above property. Thus, the following questions make sense:
	\begin{enumerate}
		\item Show that $[0, 1]$ is compact and connected.
		\item Show that $[0, 1]\cap \mathbb{Q}$ is neither compact nor connected.
		\item Show that $(0, 1)$ is connected but not compact.
		\item Show that $\{0, 1\}$ is compact but not connected.
		\item Show that $(0, 1)\cap \mathbb{Q}$ is neither compact nor connected.
		\item Give an example of a compact subset of $\mathbb{Q}$ which is not connected.\\
		\hint{One such is $\{0\} \cup \left\{n^{-1} : n \in \mathbb{N}\right\}.$}
		\item Can you give an example of a connected subset of $\mathbb{Q}$ which is not compact?\\
		\hint{The next question might be helpful.}
	\end{enumerate}
	%
	\item Show that the only nonempty connected subsets of $\mathbb{Q}$ are the singletons.
	%
	\item Show that $\mathbb{R}$ can be written as a union of two sets, each of which have empty interior.
	%
	\item Can you find a (countably infinite) collection $\{C_n\}_{n \in \mathbb{N}}$ of closed subsets of $\mathbb{R}$ such that $\displaystyle\bigcup_{n \in \mathbb{N}}C_n = \mathbb{R}?$\\
	(Compare this with the previous question.)\\
	\hint{No, you can't.}
	%
	\item Let $X$ be a metric space and let $U \subset X.$ Define the \emph{boundary} of $U$ as 
	\begin{equation*} 
		\partial U = \bar{U} \cap \overline{(U^c)}.
	\end{equation*}
	Show that $\partial U = \bar{U} \setminus U^\circ.$
	\item Prove or disprove that
	\begin{equation*} 
		(\partial U)^\circ = \emptyset
	\end{equation*}
	for any subset $U$ of any metric space $X.$\\
	\hint{Disprove it. Even in the case that $X = \mathbb{R}^n.$}
	%
	\item Find the boundary of the following subsets (the ambient space is assumed to be $\mathbb{R}^n$, where the $n$ should be clear from the question):
	\begin{enumerate}
		\item $[0, 1],$
		\item $(0, 1),$
		\item $D^2 = \{\mathbf{x} \in \mathbb{R}^2 \mid \left|\mathbf{x}\right| \le 1\},$
		\item $\{0\},$
		\item $\mathbb{Q},$
		\item $\mathbb{R}\setminus\mathbb{Q}.$
	\end{enumerate}
	(The first four should show that boundary agrees with the intuitive idea of a boundary. The last two should show that one cannot just depend on intuition.)
	%
	\item Construct a set $A \subset [0, 1] \times [0, 1]$ such that $A$ contains at most one point on each horizontal and vertical line but $\partial A = [0, 1] \times [0, 1].$\\
	\hint{It suffices to ensure that $A$ contains points in each quarter of the square $[0, 1] \times [0, 1]$ and also in each sixteenth, et cetera.}
	%
	\item Let $(X, d)$ be a metric space and $x \in X.$ Let $\delta > 0.$ Define the following sets:
	\begin{align*} 
	 	B_\delta(x) &\vcentcolon= \{y \in X \mid d(x, y) < \delta\},\\
	 	C_\delta(x) &\vcentcolon= \{y \in X \mid d(x, y) \le \delta\}.
	\end{align*} 
	Show that $\overline{B_\delta(x)} \subset C_\delta(x).$\\
	Can this inclusion be proper?\\
	\hint{Not if you stay in $\mathbb{R}^n.$ Think about other spaces.}
	%
	\item \textbf{Topological Nim}\\
	You and your friend want to play Topological Nim. Here's how it works:\\
	Let $X$ be your favourite compact metric space and $r > 0$ your favourite (positive) real number.\\
	Each player removes an open disk of radius $r$ from the space on their turn (only the center of the disk must not have been removed in a prior move), until one player—the winner—removes what remains of the space on his turn.\\~\\
	Show that no matter what moves are played, the game stops after a finite number of moves. (In other words, there is no infinite sequence of legal moves.)\\~\\
	\textbf{Bonus:} Fix $n \in \mathbb{N}$ and $r > 0.$ Assuming optimal play, who will win the game if
	\begin{equation*} 
		X = S^n = \{\mathbf{x} \in \mathbb{R}^{n+1} \mid \|x\| = 1\}
	\end{equation*}
	with the standard metric?\\
	(The answer will depend on $r$.)\\~\\
	Credits: \url{https://puzzling.stackexchange.com/questions/99859/}
	%
	\item Let $\{I_\alpha\}_{\alpha \in \Lambda}$ be a collection of disjoint nonempty open intervals. \\
	That is, for each $\alpha \in \Lambda,$ $I_\alpha$ is a nonempty open interval. \\
	Moreover, if $\beta \in \Lambda$ with $\beta \neq \alpha,$ then $I_\alpha \cap I_\beta = \emptyset.$\\
	Show that $\Lambda$ is at most countable.\\
	\hint{Each interval contains a rational. Construct an \emph{injection} $\Lambda \to \mathbb{Q}.$}
	%
	\item Let $I \subset \mathbb{R}$ be such that every $x \in I$ is an isolated point.\\
	Show that $I$ is at most countable.\\
	\hint{Try to use the previous result.}
	%
	\item Construct an at most countable set (in $\mathbb{R}$) which has no isolated point.\\
	(That is, the converse to the previous question isn't true.)
	%
	\item Construct an uncountable set (in $\mathbb{R}$) with an isolated point.
	%
	\item Show that every open set $U$ in $\mathbb{R}$ can be written as a disjoint union of nonempty open intervals. Moreover, show that this set of open intervals is at most countable.\\
	\hint{Consider an equivalence relation $\sim$ on $U$ where $x \sim y$ iff $[x, y] \subset U.$}
	%
	\item Let $K$ be a compact subset of $\mathbb{R}^n$. Fix a constant $r>0.$ \\
	Show that there exists a finite collection of points $x_1, \ldots, x_k\in K$ such that the collection of open balls $\{B(x_i,2r)\}_{i=1}^k$ forms an open cover of $K$ while $B(x_i, r)$ are mutually disjoint.
	%
	\item Let $X$ be a metric space. A subspace $Y \subset X$ is said to be \emph{dense in $X$} if $\bar{Y} = X.$\\
	For any subspace $Y \subset X,$ show that the following are equivalent:
	\begin{enumerate}
		\item $Y$ is dense in $X.$
		\item For every $x \in X$ and every $r > 0,$ there exists a point $y \in Y$ such that $d(x, y) < r.$\\
		In other words, $B_r(x) \cap Y \neq \emptyset$ for all $x \in X$ and $r > 0.$
		\item For every nonempty open set $U \subset X,$ we have $U \cap Y \neq \emptyset.$	
	\end{enumerate}
	%
	\item Show that $\mathbb{Q}$ is dense in $\mathbb{R}.$
	%
	\item Show that $\mathbb{R} \setminus \mathbb{Q}$ is dense in $\mathbb{R}.$
	%
	\item Is $\mathbb{Z}$ dense in $\mathbb{R}?$ Is $\mathbb{Z}$ dense in $\mathbb{Z}?$\\
	\hint{No. Yes.}
	%
	\item Is there any open dense subset of $\mathbb{R}?$\\
	\hint{Yes. $\mathbb{R}$ itself.}
	%
	\item Is there any open dense proper subset of $\mathbb{R}?$\\
	\hint{Yes. $\mathbb{R}$ minus any finite amount of points.}
	%
	\item Is there any open dense subset $X$ of $\mathbb{R}$ such that $\mathbb{R} \setminus X$ is uncountable?\\
	\hint{Yes. Take $\mathbb{Q}$ and try to add intervals to make it open but also keeping the complement uncountable.}
	%
	\item Consider any metric space $(X, d).$ \\
	Define a new function $\bar{d}: X\times X \to \mathbb{R}$ as
	\begin{equation*} 
		\bar{d}(x, y) = \min\{d(x, y), 1\}, \quad x, y \in X.
	\end{equation*}
	\begin{enumerate}
		\item Show that $\bar{d}$ defines a metric on $X.$
		\item Show that the metrics $d$ and $\bar{d}$ \emph{induce the same topology on $X.$}\\
		That is, show that any subset $U \subset X$ is open in $(X, d)$ if and only if it is open in $(X, \bar{d}).$\\
		(If you think about it, the metric is pretty much only useful for telling us what sets are open. Almost everything else can be formulated in terms of just open sets. (This is pretty much what you'll see in topology.)\\
		This exercise shows that the actual metric need not matter and some metric-dependent concepts like boundedness aren't very strong. As a more concrete example, see the next sub-questions.)	
		\item Show that every set $S \subset X$ is bounded in $(X, \bar{d}).$
		\item In particular, if we take $X = \mathbb{R}$ and $d$ as the standard metric, we see that every set is bounded in $(\mathbb{R}, \bar{d}).$\\
		However, the open (and hence, closed) sets are the same in both $(\mathbb{R}, d)$ and $(\mathbb{R}, \bar{d}).$ This means that even compactness is the same in both spaces. (Why?)\\
		In particular, $\mathbb{R}$ is \emph{not} compact in $(\mathbb{R}, \bar{d})$ even though it \emph{is} closed and bounded in $(\mathbb{R}, \bar{d}).$
	\end{enumerate}
	%
	\item Let $d$ be any metric on $\mathbb{R}$ which gives $\mathbb{R}$ its usual topology. (That, is a set in open in $(\mathbb{R}, d)$ iff it open under the usual metric.)\\
	Show that $(0, 1)$ must be bounded in this metric. (In fact, show that any set which is bounded under the usual metric must again be bounded.)
\end{enumerate}
\newpage\section{Sequence and series}
\begin{enumerate}
	\item Let $(a_n)$ be a sequence of rational numbers such that $\sum |a_n|$ converges to a rational number.\\
	Is is necessary that $\sum a_n$ converges to a rational number?\\
	\hint{No.}\\
	The above is essentially whether ``absolute convergence $\implies$ convergence'' is true even in $\mathbb{Q}.$
\end{enumerate}
\newpage\section{Continuity}
\begin{enumerate}
	\item \label{q:openmap} Let $\pi_1:\mathbb{R} \times \mathbb{R} \to \mathbb{R}$ be the first projection map, that is,
	\begin{equation*} 
		\pi_1(x, y) = x.
	\end{equation*}
	Show that $\pi_1$ is an \emph{open map,} that is, $\pi_1(U)$ is open in $\mathbb{R}$ if $U$ is open in $\mathbb{R}^2.$\\
	Is it a closed map?\\
	\hint{No.}
	%
	\item \textbf{Pasting lemma.}\\
	Let $X$ be a metric space and $\{U_\alpha\}_{\alpha \in I}$ be an open cover of $X.$\\
	Let $Y$ be an arbitrary metric space. Suppose that for each $\alpha \in I,$ we have a continuous function
	\begin{equation*} 
		f_\alpha:U_\alpha \to Y.
	\end{equation*}
	Moreover, assume that whenever $x \in U_\alpha \cap U_\beta,$ then $f_\alpha(x) = f_\beta(x).$ (That is, the functions agree on their common domains.)\\~\\
	Show the following:
	\begin{enumerate}
		\item There exists a unique function $f:X \to Y$ such that 
		\begin{equation*} 
			f|_{U_\alpha} = f_\alpha \quad \text{for all } \alpha \in I.
		\end{equation*}
		(Recall \ref{funcrestrict} from \S\ref{sec:notations}.)
		\item The above function $f$ is continuous.
	\end{enumerate}
	%
	\item Show that the above is not true if we replace ``open'' with ``closed.'' \\
	(In particular, observe very carefully where you used open-ness of $U_\alpha.$)
	%
	\item Show that the above becomes true once again after replacing ``open'' with ``closed'' if we further impose that $I$ be finite. 
	%
	\item Show that the above is equivalent to the following formulation:\\
	Let $f:X \to Y$ be a function between metric spaces.\\
	Let $X = C_1 \cup \cdots \cup C_n$ where each $C_i$ is closed in $X.$\\
	Assume that $f|_{C_i}:C_i \to Y$ is continuous for all $1 \le i \le n.$\\
	Then, $f$ is continuous.\\
	(Write the above formulation for open sets as well.)\\~\\
	\emph{Remark.} The above lemma for closed sets makes it especially easy to directly verify the continuity of ``piece-wise'' defined functions, if each piece is closed in the ambient space. (cf. \ref{q:nothomeo})
	%
	\item Give a counterexample if we further drop ``closed'' completely, even if $I$ is finite. (In fact, you can give one with $X = \mathbb{R}$ and $|I| = 2.$)
	%
	\item Given an example of a continuous bijection $f:X \to Y$ such that $f^{-1}:Y \to X$ is not continuous.
	%
	\item \label{q:nothomeo} Justify that the following is an example for the above question:\\
	$f:[0, 1] \cup (2, 3] \to [0, 2]$ defined by
	\begin{equation*} 
		f(x) \vcentcolon= \begin{cases}
			x & x \in [0, 1]\\
			x - 1 & x \in (2, 3]
		\end{cases}.
	\end{equation*}
	%
	\item Construct a continuous function $f:\mathbb{Q} \to \mathbb{Q}$ such that
	\begin{equation*} 
		f(0) < 0, \quad f(2) > 0
	\end{equation*}
	but $f(x) \neq 0$ for any $x \in \mathbb{Q}.$\\
	(Thus, we don't have something like the ``intermediate property'' for continuous function $\mathbb{Q} \to \mathbb{Q}.$ The completeness of $\mathbb{R}$ really gives $\mathbb{R}$ its structure.)
	%
	\item Construct a continuous function $f:[0, 1] \cup [2, 3] \to \mathbb{R}$ such that 
	\begin{equation*} 
		f(0) = 0, \quad f(3) = 6
	\end{equation*}
	but $f(x) \neq 3$ for any $x$ in the domain.\\
	(This shows the requirement that $f$ be defined on a \emph{connected} domain.)
	%
	\item Show that $X$ is connected iff the only clopen subsets of $X$ are $\emptyset$ and $X.$\\
	(A subset is said to be clopen if it is both closed and open.)
	%
	\item A metric space $X$ is said to be \emph{discrete} if $\{x\}$ is open in $X$ for all $x \in X.$\\
	Show that $\mathbb{N}$ and $\mathbb{Z}$ are discrete.
	%
	\item Show that any function $f:X \to Y$ is continuous if $X$ is discrete. ($Y$ is any arbitrary metric space.)
	%
	\item Let $X$ be a connected metric space and $Y$ a discrete one.\\
	Show that if $f:X \to Y$ is continuous, then $f$ is constant.\\
	\hint{Note that singletons in $Y$ are both open and closed. Consider the pre-image of any such singleton.}
	%
	\item Let $f:X\to Y$ be a function between metric spaces.
	\begin{enumerate}
		\item $f$ is said to be \emph{open continuous} if $f^{-1}(U)$ is open in $X$ whenever $U$ is open in $Y.$
		\item $f$ is said to be \emph{closed continuous} if $f^{-1}(U)$ is closed in $X$ whenever $U$ is closed in $Y.$
	\end{enumerate}
	Show that $f$ is continuous iff $f$ is open continuous iff $f$ is closed continuous.
	%
	\item Let $K$ be a compact metric space and $Y$ an arbitrary metric space.\\
	Assume that $f:K\to Y$ is a continuous bijection.
	\begin{enumerate}
		\item Let $C \subset K$ be closed. Show that $C$ is compact.
		\item Show that $f(C)$ is compact.
		\item Show that $f(C)$ is closed.
	\end{enumerate}
	Conclude that $f^{-1}:Y \to K$ is continuous.
	%
	\item Let $X$ and $Y$ be metric spaces; let $p: X \to Y$ be a surjection.\\
	The map $p$ is said to be a \emph{quotient map} provided a subset $U$ of $Y$ is open in $Y$ if and only if $p^{-1}(U)$ is open in $X.$
	\begin{enumerate}
		\item Show that a quotient map is a continuous.
		\item Show that if a continuous map is an open map, then it is a quotient map. (cf. \ref{q:openmap} for definition of an open map.)	
		\item Show that if $q:X \to Y$ is surjective, $q$ is continuous, $X$ is compact, then $q$ is a quotient map.
	\end{enumerate}
	You will study more about quotient maps in Topology, the last property makes it easy to show that certain maps like $x \mapsto e^{2\pi ix}$ from $I$ to $S^1$ are quotient maps. (In fact, you can weaken the condition on $Y$ from being a metric space to just being a Hausdorff space and $X$ just being any compact space.)
	%
	\item The following question appeared on a test:\\
	\begin{blockquote}
		Given an example of a continuous bijection $f:X \to Y$ such that \\
		$f^{-1}:Y \to X$ is not continuous. 
	\end{blockquote}
	The lazy TA sees that a student has started their answer as\\
	\begin{blockquote}
		The following is example:\\
		Let $f:S^1 \to S^1$ be defined as...
	\end{blockquote}
	The TA sees that and marks it wrong straight away. Was the TA justified (mathematically, not morally) in doing so? Why?
	%
	\item Let $I \subset \mathbb{R}$ and $f:I \to \mathbb{R}$ be continuous. We know that if $I$ is compact, then $f$ is bounded and it achieves (both) its bounds.\\
	Show that if $I$ is not compact, then one can always construct:
	\begin{enumerate}
		\item a continuous $f$ which is not bounded,
		\item a continuous $f$ which is bounded but fails to achieve one (or both) of its bounds.
	\end{enumerate}
	%
	\item Let $I \subset \mathbb{R}$ and $f:I \to \mathbb{R}$ be continuous. We know that if $I$ is compact, then $f$ is uniformly continuous.\\
	Can we again do something like the previous case?\\
	That is: if $I$ is not compact, then can one always construct a continuous $f$ which is \emph{not} uniformly continuous?\\
	\hint{No. Show that every function $f:\mathbb{Z} \to Y$ is not only continuous but uniformly continuous.}
	%
	\item Let $f:\mathbb{R} \to \mathbb{R}$ be continuous such that
	\begin{equation*} 
		\lim_{x\to \infty}f(x) \text{ and } \lim_{x\to -\infty}f(x)
	\end{equation*}
	both exist and are finite.\\
	Show that $f$ is bounded.
	%
	\item Suppose $f$ is continuous on $[0, 1]$ with $f(0) = f(1) = 0.$ For all $x\in (0, 1)$, there exists $h > 0$ with $0 \le x - h < x < x + h \le 1$ such that 
	\begin{equation*} 
		f(x)=\dfrac{f(x + h) + f(x - h)}{2}.
	\end{equation*} 
	Show that $f(x) = 0$ for all $x \in [0, 1].$\\
	(Note that given any $x,$ the above only says that there's \emph{one} particular $h$ with the given property.)
	%
	\item Prove/disprove:\\
	If $f:\mathbb{R} \to \mathbb{R}$ is continuous and bounded, then $f$ is uniformly continuous.\\
	\hint{Disprove.}
\end{enumerate}
\newpage\section{Derivatives}
\begin{enumerate}
	\item Prove or disprove:\\
	Let $f:\mathbb{R} \to \mathbb{R}$ be continuously differentiable. If $f'(x_0) > 0$ for some $x_0 \in \mathbb{R},$ the there exists an interval $I$ containing $x_0$ such that $f$ is increasing on $I.$\\
	\hint{Prove.}
	%
	\item Prove or disprove:\\
	Let $f:\mathbb{R} \to \mathbb{R}$ be differentiable. If $f'(x_0) > 0$ for some $x_0 \in \mathbb{R},$ the there exists an interval $I$ containing $x_0$ such that $f$ is increasing on $I.$\\
	\hint{Disprove.}
	%
	\item Let $f:\mathbb{R} \to \mathbb{R}$ be a continuously differentiable function such that $\displaystyle\lim_{x\to \infty}f(x)$ exists and is finite.\\
	Prove or disprove:
	\begin{equation*} 
		\lim_{x\to \infty}f'(x) = 0.
	\end{equation*}
	\hint{The limit need not exist.}
	%
	\item Let $f:\mathbb{R} \to \mathbb{R}$ be a differentiable function such that $\displaystyle\lim_{x\to \infty}f(x)$ exists and is finite. Further assume that $f'$ is uniformly continuous. \\
	Prove or disprove:
	\begin{equation*} 
		\lim_{x\to \infty}f'(x) = 0.
	\end{equation*}
	\hint{Prove.}
	%
	\item Let $I$ be an open interval and $f:I\to \mathbb{R}$ be differentiable. Show that $f'$ need not be continuous.\\
	Show that $f'$ has the intermediate value property. That is, if $a, b \in I$ with $f'(a) < r < f'(b),$ then there exists $c \in (\min\{a, b\}, \max\{a, b\})$ such that $f'(c) = r.$\\
	This is known as Darboux's Theorem.
	%
	\item Let $I$ be an open interval and $f:I \to \mathbb{R}$ be differentiable.\\
	Prove that $f'$ is continuous if and only if the inverse image under $f'$ of any point is a closed set.
	%
	\item Let $(X, d)$ be a complete metric space. (That is, every Cauchy sequence in $X$ converges.)\\
	Let $f:X \to X$ be a function with the following property:\\
	There exists $0 < K < 1$ such that
	\begin{equation*} 
		d(f(x), f(y)) \le Kd(f(x), f(y)) \quad \text{for all } x, y \in X.
	\end{equation*}
	Show that:
	\begin{enumerate}
		\item $f$ is (uniformly) continuous.
		\item \label{fixedpoint} $f$ has a fixed point.\\
		(That is, $f(x) = x$ for some $x \in X$.)
		\item $f$ has a unique fixed point.
	\end{enumerate}
	%
	\item Let $f:\mathbb{R} \to \mathbb{R}$ be differentiable such that $|f'(x)| \le K$ for all $x \in \mathbb{R},$ where $K < 1$ is some fixed positive constant.\\
	Show that $\mathbb{R}$ has a unique fixed point.
	%
	\item Give an example of a differentiable function $f:\mathbb{R} \to \mathbb{R}$ with $|f'(x)| < 1$ such that $f$ has no unique fixed point.\\~\\
	Contemplate on how this is different from the earlier question.
	%
	\item Show that $f:\mathbb{R} \to \mathbb{R}$ defined as 
	\begin{equation*} 
		f(x) = \exp\left(-\cos^2(x)\right)
	\end{equation*}
	has a unique fixed point.\\
	(How would you calculate it numerically? Was your proof of \ref{fixedpoint} ``constructive''?)
\end{enumerate}
\newpage\section{Integrals}
\begin{enumerate}
	\item Does there exist a function $f:[0, 1] \to \mathbb{R}$ such that it takes only a finitely many values and is Riemann Integrable on $[0, 1]$ but is not locally constant?\\
	\hint{Yes. Find/show the existence of one.}
\end{enumerate}
\newpage\section{A word on local properties}
Given some property $P,$ we sometimes ask whether an object $X$ is \emph{locally} $P$. This usually means whether $P$ is satisfied on some small enough open sets.\\
The next question formulate this more precisely.\\
Terminology: Given a metric space $X$ and a point $x \in X,$ a \emph{neighbourhood} of $x$ is an open set $U \subset X$ such that $x \in U.$
\begin{enumerate}
	\item A function $f:X\to Y$ is said to be \emph{locally constant} if for every $x \in X,$ there exists a neighbourhood $U$ of $x$ such that $f|_U$ is constant.\\
	Give a example of a function $f:[0, 1] \cup [2, 3] \to \mathbb{R}$ which is locally constant but not constant.
	%
	\item Show that if $f:X \to Y$ is locally constant and if $X$ is connected, then $f$ is constant.
	%
	\item A function $f:X \to \mathbb{R}$ is said to be \emph{locally bounded} if for every $x \in X,$ there exists a neighbourhood $U$ of $x$ such that $f|_U$ is bounded.\\
	Give an example of a function $f:\mathbb{R} \to \mathbb{R}$ which is locally bounded but not bounded. (In fact, it might be tough for you to come up with a non-locally bounded function. Look up Conway's base-13 function as an example of this.)
	%
	\item Show that if $f:X\to\mathbb{R}$ is continuous, then it is locally bounded.
	%
	\item A metric space $X$ is said to be \emph{locally connected at $x$} if for every neighbourhood $U$ of $x,$ there is a connected neighbourhood $V$ of $x$ contained in $U.$\\
	$X$ is said to be \emph{locally connected} if it is locally connected at $x$ for every $x \in X.$\\
	Show that $[0, 1]\cup[2, 3]$ is locally connected but not connected.
	%
	\item A metric space $X$ is said to be \emph{locally compact at $x$} if there is some compact subspace $C$ of $X$ that contains a neighbourhood of $x.$\\
	If $X$ is locally compact at each of its points, $X$ is said to be \emph{locally compact.}\\
	Show that $\mathbb{R}$ is locally compact.
	%
	\item $f:X\to Y$ is said to be \emph{locally continuous} if for every $x \in X,$ there exists a neighbourhood $U$ of $x$ such that $f|_U$ is continuous.\\
	Show that $f$ is locally continuous iff $f$ is continuous.\\
	Due to the above reason, there actually isn't a concept of local continuity since continuity is a ``local property'' to begin with.
	%
	\item Show that the above is true for differentiation as well.
	%
	\item In each of the above cases, show that if the object is locally $P$ if it is $P$ to begin with.\\
	(That is, show that constant functions are locally constant, et cetera.)
\end{enumerate}
\newpage\section{Sequence and series of functions}
\begin{enumerate}
	\item \textbf{(Non-)converse of Weierstrass M-test}\\
	Construct an example of a family $(f_n)_{n \in \mathbb{N}}$ of functions $f_n:\mathbb{R} \to \mathbb{R}$ such that $\sum f_n$ converges uniformly but $\sum M_n$ does not, where $M_n \vcentcolon= \displaystyle\sup_{x \in \mathbb{R}}|f_n(x)|.$\\
	\hint{Consider $f_n$ such that $f_n$ takes value $1/n$ at $n$ and $0$ otherwise.}
	%
	\item Recall that if $f:K \to \mathbb{R}$ is a continuous function and $K$ is compact, then there exists a sequence $(P_n)_{n \in \mathbb{N}}$ of polynomials such that $P_n \to f$ uniformly on $K.$\\
	Show that this need not be true if $K$ is not compact.\\
	\hint{Consider $K = \mathbb{R}$ and $f = \exp.$}
	%
	\item Let $f:\mathbb{R} \to \mathbb{R}$ be continuous. Show that there exists a sequence $(P_n)_{n \in \mathbb{N}}$ of polynomials such that $P_n \to f$ \textbf{pointwise} on $\mathbb{R}.$
	%
	\item Let $f_n:\mathbb{R} \to \mathbb{R}$ be defined as
	\begin{equation*} 
		f_n(x) \vcentcolon= \left(1 + \dfrac{x}{n}\right)^n.
	\end{equation*}
	Show that $(f_n)$ does not converge uniformly.
	%
\end{enumerate}
%
\newpage\section{\texorpdfstring{$\mathbb{R}$ versus $\mathbb{C}$}{R versus C}}
In this section, I point out some differences in how complex differentiable function behave. (How they're so much ``nicer.'')\\
The definition of a complex differentiable function is how one would expect it to be.

\begin{defn}[Complex differentiability]
\begin{blockquote}
	Let $\Omega \subset \mathbb{C}$ be open.\\
	A function $f:\Omega \to \mathbb{C}$ is said to be complex differentiable at $z \in \Omega$ if the limit
	\begin{equation*} 
		\lim_{z\to z_0}\dfrac{f(z) - f(z_0)}{z - z_0}
	\end{equation*}
	exists. In which case, it is denoted by $f'(z).$ \\
	If $f$ is complex differentiable at all points in $\Omega,$ $f$ is simply said to be complex differentiable. In which case, we get a new function $f':\Omega \to \mathbb{C}.$
\end{blockquote}
\end{defn}
Note that I'm saying ``complex differentiable'' in the above only to emphasise that this is different from the definition seen in the course since that was for real functions. Also, note that $z \to z_0$ in the limit is in the complex plane.\\~\\
%
In the next few exercises, I will ask you to construct an example of something going ``bad'' in $\mathbb{R}$ and then follow it with a ``good'' fact from $\mathbb{C}.$\\~\\
In the following $U$ will always be a nonempty open subset of $\mathbb{R}$ and $\Omega$ a nonempty open subset of $\mathbb{C}.$ Moreover, whenever I ask you to construct an example involving $U,$ you are free to choose $U$ as per your liking. (Of course, it should be a nonempty open set.)
\begin{enumerate}
	\item Construct a function $f:U \to \mathbb{R}$ which is differentiable once but $f'$ is not differentiable (anywhere).\\
	In fact, given any $n \in \mathbb{N},$ construct a function $f:U\to \mathbb{R}$ which is differentiable $n$ times but $f^{(n)}$ is not differentiable anywhere.\\
	\begin{blockquote}
		If a function $f:\Omega \to \mathbb{C}$ is complex differentiable, then so is $f'.$\\
		(Which implies so is $f'', f''',$ et cetera. Thus, differentiable once implies differentiable always.)
	\end{blockquote}
	%
	\item Pick $U$ to be connected: Construct a non-constant differentiable function $f:U\to \mathbb{R}$ such that $|f|$ has a local maximum. \\
	\begin{blockquote}
		If $\Omega$ is connected and $f:\Omega \to \mathbb{C}$ is a non-constant complex differentiable, then $|f|$ cannot have a local maximum.
	\end{blockquote}
	%
	\item Construct a non-constant differentiable function $f:\mathbb{R} \to \mathbb{R}$ which is bounded.\\
	\begin{blockquote}
		\textbf{Liouville's Theorem.}\\
		If $f:\mathbb{C} \to \mathbb{C}$ is a bounded complex differentiable function, then $f$ is bounded.
	\end{blockquote}
	In fact, an even stronger fact is true.\\
	\begin{blockquote}
		\textbf{Little Picard's Theorem.}\\
		If $f:\mathbb{C} \to \mathbb{C}$ is a non-constant complex differentiable function, then $\left|\mathbb{C}\setminus f(\mathbb{C})\right| \le 1.$
	\end{blockquote}
	That is, the image of $f$ misses at most one point. (It may or may not miss anything. For example, the identity map misses nothing whereas the exponential map misses $0.$ The above fact tells you that you can't miss two or more points.)\\
	This is stronger because the first fact don't disallow the possibility that the image of $f$ is contained in, say, $\{z \in \mathbb{C} \mid \Im z > 0\},$ the upper half plane.
	%
	\item Pick $U$ to be connected and let $S \subset U$ be such that $S$ has a limit point \textbf{in} $U$: Construct functions differentiable $f, g:U \to \mathbb{R}$ such that $f|_S = g|_S$ but $f \neq g.$\\
	(As an example, you can pick $U = \mathbb{R}$ and $K = \{0\} \cup \{n^{-1} \mid n \in \mathbb{N}\}.$)\\
	\begin{blockquote}
		Let $\Omega$ be connected and $S \subset \Omega$ be such that $S$ has a limit point \textbf{in} $\Omega.$ Let $f, g:\Omega \to \mathbb{C}$ be complex differentiable functions such that $f|_S = g|_S.$\\
		Then, $f = g.$
	\end{blockquote}
	(That is, $f$ and $g$ being equal on $S$ is sufficient for them to be equal everywhere.)\\
	Note that the connectivity is important and so is the fact that $S$ has a limit point \textbf{in} $\Omega.$ For example, taking $\Omega = \{z \in \mathbb{C} \mid \Re z > 0\}$ and $S = \{n^{-1} \mid n \in \mathbb{N}\}$ does not work.
	%
	\item 

	\begin{defn}[Power series representation]
		\begin{blockquote}
		A function $f:U \to \mathbb{R}$ is said to \emph{admit a power series representation at $x_0$} exists a neighbourhood $B_r(x_0) \subset U$ and a sequence of real numbers $(a_n)$ such that 
		\begin{equation*} 
			f(x) = a_0 + a_1(x - x_0) + a_2(x - x_0)^2 + \cdots
		\end{equation*}
		for all $x \in B_r(x_0).$\\~\\
		An entirely analogous definition is there for $f:\Omega \to \mathbb{C}$ as well.
		\end{blockquote}
	\end{defn}
	It can be shown that such a function must be infinitely differentiable at $x_0$ and that $a_n = \dfrac{f^{(n)}(x_0)}{n!}.$\\~\\
	Construct a function $f:U \to \mathbb{R}$ which is infinitely differentiable but there exists $x_0 \in U$ such that $f$ does not admit a power series representation at $x_0.$\\
	(In fact, one can find a function which does not admit a power series representation \emph{anywhere}.)\\
	\begin{blockquote}	
		Let $f:\Omega \to \mathbb{C}$ be complex differentiable. \\
		Then, for every $z_0 \in \Omega,$ $f$ admits a power series representation. Moreover, this representation is valid on the largest open disc centered at $z_0$ contained in $\Omega.$
	\end{blockquote}
	\item Let $(f_n)_{n \in \mathbb{N}}$ be a sequence of differentiable functions $f_n:U \to \mathbb{R}.$\\
	Suppose that $f_n\to f$ uniformly on compact subsets of $U.$\\
	Show that $f$ is continuous.\\
	Show that it is not necessary that $f$ is differentiable (anywhere).\\
	\hint{Consider $U$ to be bounded and $f$ to be a Weierstrass type function.}\\~\\
	\begin{blockquote}
		\textbf{Montel's Theorem.}\\
		Let $\Omega$ be an open set in $\mathbb{C}$ and $(f_n)$ a sequence of (complex) differentiable functions $f_n:\Omega \to \mathbb{C}.$\\
		Suppose that $f_n\to f$ uniformly on compact subsets of $\Omega.$\\
		Then, $f$ is also (complex) differentiable.\\
		Further, $f_n' \to f'$ uniformly on compact subsets of $\Omega.$
	\end{blockquote}
	This is just one example of how much ``better'' things behave in $\mathbb{C}$ Analysis as compared to $\mathbb{R}.$ In $\mathbb{R},$ not only can $f$ fail to be differentiable but it can differentiable \emph{nowhere}.
	%
\end{enumerate}
\end{document}