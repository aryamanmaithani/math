\documentclass[12pt]{article}
\usepackage[lmargin=1in,rmargin=1in,tmargin=1in,bmargin=1in]{geometry}

\usepackage{aryaman}
% \usepackage{sepfootnotes}
\newcommand{\myint}{I\!I}

\setcounter{tocdepth}{2}

\title{Polynomial invariants of \texorpdfstring{$\GL_{2}$}{GL(2)}: Conjugation over finite fields}
\author{Aryaman Maithani}
\date{%
February 7, 2025 at University of Utah \\ 
February 24, 2025 at UC Sand Diego %
}

\usepackage[backend=biber,style=alphabetic,doi=false,isbn=false,url=false,eprint=true,maxbibnames=5,minbibnames=5,mincitenames=5,maxcitenames=5,maxalphanames=5,minalphanames=5,backref=true]{biblatex}
\addbibresource{talks.bib}
\DeclareFieldFormat{extraalpha}{#1}
\DeclareLabelalphaTemplate{%
  \labelelement{%
    \field[final]{shorthand}
    \field{label}
    \field[strwidth=2,strside=left,ifnames=1]{labelname}
    \field[strwidth=1,strside=left]{labelname}
  }
}
\DefineBibliographyStrings{english}{%
  backrefpage={},
  backrefpages={}
}
\renewcommand*{\finentrypunct}{}
\usepackage{xpatch}
\DeclareFieldFormat{backrefparens}{\addperiod#1}
\xpatchbibmacro{pageref}{parens}{backrefparens}{}{}

\begin{document}

\maketitle
% \tableofcontents

\section*{Introduction}

	These are the notes that I made for the talks that I gave on my paper~\Cite{Maithani:Conjugation}. Talk abstract: \newline

	\begin{blockquote}
		Consider the conjugation action of $\GL_{2}(K)$ on the polynomial ring $K[X_{2 \times 2}]$. 
		When $K$ is an infinite field, the ring of invariants is a polynomial ring generated by the trace and the determinant. 
		We describe the ring of invariants when $K$ is a finite field, and show that it is a hypersurface.
	\end{blockquote}

	Let $K$ be a field, $S \coloneqq K[X_{n \times n}]$ the polynomial ring in $n^{2}$ variables, and $G \coloneqq \GL_{n}(K)$ the general linear group. 
	The group $G$ acts on $S$ via \emph{conjugation}, i.e., the element $\sigma \in G$ acts on $S$ via
	\begin{equation*} 
		X \mapsto \sigma X \sigma^{-1};
	\end{equation*}
	if $X$ denotes the square matrix of variables, then the element $\sigma \in G$ acts by mapping $x_{ij}$ to the $(i, j)$-th entry of $\sigma^{-1} X \sigma$. 

	We are interested in the $K$-subalgebra
	\begin{equation*} 
		S^{G} \coloneqq \{f \in S : \sigma(f) = f \text{ for all } \sigma \in G\}.
	\end{equation*}

	\textbf{Question.} Are any of the following matrices similar (over $\mathbb{Q}$)? How would you tell?

	\begin{equation*} 
		\two{1}{2}{3}{4},\, \two{1}{4}{2}{3},\, \two{2}{1}{4}{3}
	\end{equation*}

	How does this relate to question of invariants? 

\section{Over infinite fields}

	\begin{thm}
		If $K$ is an infinite field, then $S^{G} = K[\trace(X), \ldots, \det(X)]$, i.e., $S^{G}$ is generated by the coefficients of the characteristic polynomial of $X$. 
		Moreover, $S^{G}$ is a polynomial ring.
	\end{thm}

	\begin{proof} 
		Because the field is infinite, we may adopt the following point of view:
		\begin{align*} 
			\text{elements of $S$} & \equiv \text{polynomial functions on $K^{n \times n}$} \\
			\text{elements of $S^{G}$} & \equiv \text{polynomial functions constant on orbits,}
		\end{align*}
		where by orbits we are referring to natural conjugation action of $G$ on $K^{n \times n}$.

		Write
		\begin{equation*} 
			\det(tI - X) = t^{n} - f_{1} t^{n - 1} + \cdots + (-1)^{n} f_{n}
		\end{equation*}
		for $f_{i} \in S$. 
		We wish to show that $S^{G} = K[f_{1}, \ldots, f_{n}]$ and that the $f_{i}$ are algebraically independent. 
		The inclusion ($\supset$) is clear. 
		For the converse, let $f \in S^{G}$ be arbitrary. 
		% As the field $K$ is infinite, we may think of $S$ as polynomial functions on the affine space $V = K^{n \times n}$. 
		% The group $G$ acts on $V$ by conjugation and the elements of $S^{G}$ are precisely the functions that are constant on $G$-orbits.

		Consider the subspace of diagonal matrices $D \le V$, and the symmetric group $S_{n}$ as a subgroup of $\GL_{n}(K)$ in the natural way. 
		Then, the action of $G$ restricts to $S_{n}$, and $S_{n}$ acts on $D$ in the `obvious' way: the transposition $(i, j)$ swaps the $i$-th and $j$-th diagonal entries. 
		Let $e_{1}, \ldots, e_{n}$ denote the elementary symmetric polynomials on $x_{11}, x_{22}, \ldots, x_{nn}$.
		The function $f|_{D}$ is $S_{n}$-invariant and thus, we may write
		\begin{equation*} 
			f|_{D} - p(e_{1}, \ldots, e_{n}) \equiv 0
		\end{equation*}
		for some polynomial $p$.
		In particular, this means that we have
		\begin{equation} \tag{$\dagger$} \label{eq:001}
			f - p(f_{1}, \ldots, f_{n}) \equiv 0 \text{ on $D$};
		\end{equation}
		this is because the $f_{i}|_{D} = e_{i}$. 
		This also shows that the $f_{i}$ are algebraically independent. 
		But $f$ and each $f_{i}$ is $G$-invariant. 
		This means that the equation \Cref{eq:001} holds on $G \cdot D$, the set of all diagonalisable matrices. 
		But this set is Zariski-dense in $V$, showing that $f = p(f_{1}, \ldots, f_{n})$ as elements of $S$.
	\end{proof}

	The above cannot hold if $K$ is a finite field, and $n$ is at least $2$. 
	Indeed, $\GL_{n}(K)$ is then a finite group and thus, the inclusion
	\begin{equation*} 
		S^{G} \subset S
	\end{equation*}
	is integral. In particular, both rings must have Krull dimension $n^{2}$. 
	However, the subring $K[\trace(X), \ldots, \det(X)]$ has Krull dimension $n$.

\section{Over finite fields}
	
	From now on, we fix some notations.
	\begin{tcolorbox}
		We have $K \coloneqq \mathbb{F}_{q}$ the finite field on $q$ elements, 
		$G \coloneqq \GL_{2}(K)$ the general linear group, 
		$S \coloneqq K[X_{2 \times 2}] = K\left[\smatrix{a & b \\ c & d}\right]$, 
		and $G$ acts on $S$ by conjugation. 
	\end{tcolorbox}

	The idea is to compute $S^{G}$ as follows: 
	first, construct a \emph{Noetherian normalisation} for $S^{G}$; this amounts to finding a homogeneous system of parameters $f_{1}, \ldots, f_{4} \in S^{G}$ (it suffices to show that they form an hsop for $S$). 
	In that case, the ring $R \coloneqq K[f_{1}, f_{2}, f_{3}, f_{4}]$ is a polynomial ring such that $S^{G}$ is a finite $R$-module. 
	Next, we find $h_{1}, \ldots, h_{n} \in S^{G}$ such that $S^{G} = R h_{1} + \cdots R h_{n}$ as $R$-modules. 
	In particular, $S^{G}$ is generated, as a $K$-algebra, by the $f_{i}$ and $h_{j}$.

	The $f_{i}$ are called \deff{primary invariants}, the $h_{j}$ \deff{secondary invariants}. 
	These are not uniquely determined by any means. 
	However, there are different notions of minimality that one may impose. 
	Experiments on \texttt{Magma}~\Cite{Magma} suggested that the ring of invariants is a hypersurface: more precisely, there exist primary invariants in degrees $1$, $2$, $q+1$, and $q^{2} - q$, such that with these primary invariants, the secondary invariants are in degrees $0$ and $q^{2}$.

\section{Primary invariants}
	
	\begin{tcolorbox}
		Set $f_{1} \coloneqq a + d$, $f_{2} \coloneqq ad - bc$.
	\end{tcolorbox}

	It is clear that the above are invariants. 
	Using \texttt{Magma}, it looked that the third primary invariant took a particularly nice closed form.
	We define
	\begin{tcolorbox}
		\begin{equation*} 
			f_{3} \coloneqq a^{q + 1} + b^{q} c + b c^{q} + d^{q + 1}.
		\end{equation*}
	\end{tcolorbox}

	It is not too difficult to check that the above is $G$-invariant. 
	For example, one may use that 
	\begin{equation} \label{eq:gens}
		\GL_{2}(K) = \left\langle \two{}{1}{1}{},\, \two{\beta}{}{}{1},\, \two{1}{1}{}{1} \right\rangle,
	\end{equation}
	where $K^{\times} = \langle \beta \rangle$. 

	The action of the three elements is respectively given as
	\begin{align*} 
		a \leftrightarrow d, \quad & b \leftrightarrow c, \\
		a \mapsto a, \quad b \mapsto \beta^{-1} b, \quad & c \mapsto \beta c,\quad d \mapsto d, \\
		a \mapsto a - c, \quad b \mapsto a + b - c - d, \quad & c \mapsto c, \quad d \mapsto c + d.
	\end{align*}
	One may then check $\sigma(f_{3}) = f_{3}$ for any of the above generators, noting that $\beta^{q - 1} = 1$. 
	However, there is a more abstract way to see this: we have
	\begin{equation*} 
		f_{3} = (a + d)^{q + 1} - (a^{q} d + a d^{q} - b^{q} c - b c^{q})
	\end{equation*}
	and so, it suffices to show that the last element is an invariant;
	this follows by noting that it is $\mathcal{P}^{1}(ad - bc)$ for a `nice' operation $\mathcal{P}^{1}$, a \emph{Steenrod operation}.

	Things now seemed to be a dead end. 
	\texttt{Magma} suggested that the fourth primary invariant should have degree $q^{2} - q$. 
	But it was not clear what it should be. 
	One way of producing invariants for finite groups is to look at orbit products. 
	We get lucky with the following. 

	\begin{tcolorbox}
		Fix an irreducible quadratic $g(x) \coloneqq x^{2} - \tau x + \delta \in K[x]$. 
	\end{tcolorbox}

	Such a quadratic exists because $K$ is a finite field. Straightforward linear algebra gives us the following fact.

	\begin{thm}
		Let $\Omega \subset V$ be the set of $2 \times 2$ matrices with characteristic polynomial equal to $g(x)$. 
		Then,
		\begin{equation*} 
			\Omega = \left\{\two{A}{B}{-\frac{g(A)}{B}}{\tau - A} : A \in K , B \in K^{\times}\right\}.
		\end{equation*}
		In particular, $\md{\Omega} = q(q - 1) = q^{2} - q$.
	\end{thm}

	Thus, we get a fourth invariant of the \emph{correct degree} defined as

	\begin{tcolorbox}
		\begin{equation*} 
			f_{4} \coloneqq \prod_{\substack{A \in K \\ B \in K^{\times}}} \left(Aa + Bb - \frac{g(A)}{B} c + (\tau - A) d\right).
		\end{equation*}
	\end{tcolorbox}

	\begin{thm}
		The elements $f_{1}, \ldots, f_{4}$ form a homogeneous system of parameters for $S$ and hence, for $S^{G}$.
	\end{thm}
	\begin{sketch}
		It suffices to show that the only solution to $f_{1} = \cdots = f_{4} = 0$ over $\overline{K}^{4}$ is the origin. 
		Let $(a, b, c, d) \in \overline{K}^{4}$ be such a solution. 
		We may discard $f_{1} = 0$ by substituting $d = -a$ in the other equations and then it suffices to show that $a = b = c = 0$. 
		The equation $f_{4} = 0$ gives us the existence of $A \in K$ and $B \in K^{\times}$ such that one the factors in $f_{4}$ is zero. 
		We may solve for $b$ in terms of $a$ and $c$ and substitute this in $f_{2} = 0$. 
		We then get a quadratic equation in $a$ that we may solve as
		\begin{equation*} 
			a = \frac{A + \mu}{B} c
		\end{equation*}
		for some $\mu \in \overline{K}$ such that $g(\mu) = 0$.
		Necessarily $\mu \notin K$. 
		In turn, we get $b$ as
		\begin{equation*} 
			b = -\left(\frac{A + \mu}{B}\right)^{2} c.
		\end{equation*}
		Letting $\gamma \coloneqq (A + \mu)/B \in \overline{K} \setminus K$, we substitute these values in $f_{3} = 0$ to get
		\begin{equation*} 
			-(\gamma^{q} - \gamma)^{q} c^{q + 1} = 0.
		\end{equation*}
		The first term is nonzero because $\gamma \notin K$. Thus, $c = 0$ and in turn, so are the others.
	\end{sketch}
	% \begin{sketch}
	% 	It suffices to show that the only solution to $f_{1} = \cdots = f_{4} = 0$ over $\overline{K}^{4}$ is the origin. 
	% 	Let $(a, b, c, d) \in \overline{K}^{4}$ be such a solution. 
	% 	For convenience, assume that $g(x) = x^{2} + \delta$. 
	% 	We may discard $f_{1} = 0$ by substituting $d = -a$ in the other equations and then it suffices to show that $a = b = c = 0$. 
	% 	The equation $f_{4} = 0$ gives us the existence of $A \in K$ and $B \in K^{\times}$ such that
	% 	\begin{equation*} 
	% 		b = \frac{A^{2} + \delta}{B} c - \frac{2A}{B} a.
	% 	\end{equation*}

	% 	Next, $f_{3} = 0$ gives us
	% 	\begin{equation*} 
	% 		a^{2} + \frac{A^{2} + \delta}{B^{2}} c^{2} - \frac{2A}{B} ac = 0.
	% 	\end{equation*}

	% 	The above can be rearranged as
	% 	\begin{equation*} 
	% 		\left(a - \frac{A}{B} c\right)^{2} = -\delta \left(\frac{c}{B}\right)^{2}.
	% 	\end{equation*}

	% 	As we are working over $\overline{K}$, we may solve the above for $a$ as
	% 	\begin{equation*} 
	% 		a = \left(\frac{A + \gamma}{B}\right) c
	% 	\end{equation*}	
	% 	for some $\gamma \in \overline{K}$ satisfying $\gamma^{2} = -\delta$. 
	% 	The irreducibility of $g(x)$ tells us that $\gamma \notin K$.

	% 	Substituting this back in the expression for $b$, we may rewrite $b$ as
	% 	\begin{equation*} 
	% 		b = -\left(\frac{A + \gamma}{B}\right)^{2} c.
	% 	\end{equation*}

	% 	Set $\mu \coloneqq (A + \gamma)/B \in \overline{K} \setminus K$ to get $b = -\mu^{2} c$ and $a = \mu c$. 
	% 	Substituting this in $f_{4} = 0$ gives us the equation
	% 	\begin{equation*} 
	% 		-(\mu^{q} - \mu)^{2} c^{q + 1} = 0.
	% 	\end{equation*}

	% 	The first term is nonzero because $\mu \notin K$. Thus, $c = 0$ and in turn, so are the others. 
	% \end{sketch}

	Thus, we now have a Noether normalisation for $S^{G}$,
	\begin{tcolorbox}
		\begin{equation*} 
			R \coloneqq K[f_{1}, f_{2}, f_{3}, f_{4}] \subset S^{G}.
		\end{equation*}
	\end{tcolorbox}

	In turn, we have a decomposition of $R$-modules
	\begin{tcolorbox}
		\begin{equation*} 
			S = R h_{1} + R h_{2} + \cdots + R h_{n}.
		\end{equation*}
	\end{tcolorbox}

\section{Determining \texorpdfstring{$n$}{n}}

	We now determine $n$ by first showing that $R$ is Cohen--Macaulay. 
	First, we define $P$ to be the following Sylow-$p$ group of $G$:
	\begin{equation*} 
		V \coloneqq \two{1}{K}{}{1} \le G.
	\end{equation*}

	\begin{lem} 
		We have $\dim(V^{P}) = 2$. Equivalently, $\codim(V^{P}) = 2$.
	\end{lem}
	\begin{sketch}
		The fixed points are precisely the elements that commute with elements of $P$.
		Check that $V^{P} = \left\{\smatrix{a & b \\ 0 & a} : a, b \in K\right\}$.
	\end{sketch}

	\begin{cor}
		$S^{P}$ is Cohen--Macaulay.
	\end{cor}
	\begin{proof} 
		This follows from~\Cite[Theorem 3.9.2]{CampbellWehlau:ModularInvariantTheory} as we have shown $\codim(V^{P}) = 2$.
	\end{proof}

	\begin{cor}
		$S^{G}$ is Cohen--Macaulay.
	\end{cor}
	\begin{proof} 
		The inclusion $S^{G} \into S^{P}$ is split via the splitting $s \mapsto \frac{1}{[G : P]} \sum\limits_{g \in G/P} g(s)$. 
		Because this is a finite extension, we obtain the result.
	\end{proof}

	Thus, we can improve the decomposition to
	\begin{tcolorbox}
		\begin{equation*} 
			S = R h_{1} \oplus R h_{2} \oplus \cdots \oplus R h_{n}.
		\end{equation*}
	\end{tcolorbox}

	% Before we proceed to determine more properties (such as what $n$ is), we are at a stage where we must take care of the fact that our group action is not faithful: 
	We are now at a stage where we must take faith seriously:
	the conjugation is not faithful, action the scalar matrices act trivially. 

	Indeed, the action of $G$ leads to a corresponding homomorphism
	\begin{tcolorbox}
		\begin{equation*} 
			\rho \colon G \to \GL(V).
		\end{equation*}
	\end{tcolorbox}
	The kernel of the above is precisely the subgroup of scalar matrices.

	We let $\widehat{G}$ denote its image, i.e.,
	\begin{tcolorbox}
		\begin{equation*} 
			\rho \colon G \onto \widehat{G} \subset \GL(V).
		\end{equation*}
		Then, $\mdd{\widehat{G}} = q(q^{2} - 1)$.
	\end{tcolorbox}

	The action of $\widehat{G}$ on $V$ (and $S$) is faithful and we have $S^{G} = S^{\widehat{G}}$.

	Now, using~\Cite[Theorem 3.7.1]{DerksenKemper}, we obtain the (minimal) number of secondary invariants as
	\begin{equation*} 
		n = \frac{\prod_{i = 1}^{4} \deg(f_{i})}{\mdd{\widehat{G}}} = \frac{1 \cdot 2 \cdot (q + 1) \cdot (q^{2} - q)}{q (q^{2} - q)} = 2.
	\end{equation*}

	Thus,
	\begin{tcolorbox}
		\begin{equation*} 
			S = R h_{1} \oplus R h_{2}.
		\end{equation*}
	\end{tcolorbox}

	Moreover, we may always take $h_{1} = 1$ as a minimal secondary invariant to obtain the decomposition
	\begin{tcolorbox}
		\begin{equation*} 
			S = R \oplus R h.
		\end{equation*}
		In particular, $S$ is a hypersurface with
		\begin{equation*} 
			S = K[f_{1}, f_{2}, f_{3}, f_{4}, h].
		\end{equation*}
	\end{tcolorbox}

	Consequently, the Hilbert series of $S^{G}$ is then given as
	\begin{equation*} 
		\Hilb(S^{G}, z) = 
		\frac{1 + z^{\deg(h)}}
		{(1 - z)(1 - z^{2})(1 - z^{q + 1})(1 - z^{q^{2} - q})}.
	\end{equation*}

\section{Determining \texorpdfstring{$\deg(h)$}{deg(h)}}
	
	To determine $\deg(h)$, it suffices to determine the degree of the Hilbert series%
	\footnote{The \deff{degree} of a rational function is the difference of the degrees of the numerator and denominator.} 
	$\Hilb(S^{G})$. 
	Because the ring $S^{G}$ is Cohen--Macaulay, this degree is given by the \emph{$a$-invariant}.%
	\footnote{The \deff{$a$-invariant} of a graded ring $R$ is the highest degree in which the local cohomology module $H_{\mathfrak{m}_{R}}^{\dim(R)}(R)$ is nonzero.} 
	We make use of the following theorem to determine the $a$-invariant.

	\begin{thm}[{\Cite[Theorem 4.4]{GoelJeffriesSingh}}]
		If $\widehat{G}$ is a subgroup of $\SL(V)$ and contains no pseudoreflections, then $a(S^{\widehat{G}}) = a(S)$.
	\end{thm}
	We recall that an element $\sigma \in \GL(V)$ is said to be a \deff{pseudoreflection} if $\rank(\sigma - \id) = 1$.

	\begin{prop}
		For the $\widehat{G}$ in our context, the hypothesis of the above theorem holds. 
		In particular, $a(S^{G}) = -4$.
	\end{prop}
	\begin{sketch} 
		To check $\widehat{G} \le \SL(V)$, one checks that $\rho(\sigma) \in \SL(V)$ for each of the three generators $\sigma$ defined in~\Cref{eq:gens}. 
		Alternately: we see that $\rho(\sigma)$ is the composition $L(\sigma) \circ R(\sigma)^{-1}$, where $L(\sigma)$ and $R(\sigma)$ denote the left and right multiplication maps, respectively. 
		Thus, it suffices to show $\det(L(\sigma)) = \det(R(\sigma))$. 
		Simple linear algebra tells us that both of these are indeed equal (and equal to $\det(\sigma)^{2}$).

		To check that $\widehat{G}$ contains no pseudoreflections, it suffices to show that the dimension of the centraliser of any $M \in \GL_{2}(K)$ is not $3$. 
		By considering Jordan forms, one sees that this dimension is either $2$ or $4$.
	\end{sketch}

	Thus,
	\begin{equation*} 
		-4 = \deg(h) - (1 + 2 + (q+1) + (q^{2} - q)),
	\end{equation*}
	giving us $\deg(h) = q^{2}$.

\section{The missing invariant}
	
	We now need to construct a new invariant $h$ of degree $q^{2}$. 
	In fact, it is not difficult to check using normality that \emph{any} homogeneous invariant $h \in S^{G} \setminus R$ of degree $q^{2}$ will do the job.

	We define
	\begin{align} 
		h &\coloneqq \Jac(f_{1}, \ldots, f_{4}) \nonumber \\
		&= \det 
		\begin{bmatrix}
			1 & 0 & 0 & 1 \\
			d & -c & -b & a \\
			a^{q} & c^{q} & b^{q} & d^{q} \\
			\frac{\partial f_{4}}{\partial a} & \frac{\partial f_{4}}{\partial b} & \frac{\partial f_{4}}{\partial c} & \frac{\partial f_{4}}{\partial d} \\
		\end{bmatrix}.
		\label{eq:002}
	\end{align}

	Because the group $\widehat{G}$ is contained in $\SL(V)$, the chain rule gives us that $h \in S^{G}$, see~\Cite[Proposition 1.5.6]{Smith:PolynomialInvariantsBook}. 

	Moreover, the degree of the entries of the $i$-th row is seen to be $\deg(f_{i}) - 1$, and thus,
	\begin{equation*} 
		\deg(h) = \sum_{i = 1}^{4} (\deg(f_{i}) - 1) = \boxed{q^{2}},
	\end{equation*}
	as desired!

	There is only one issue left: is $h \notin R$? As it turns, this fails precisely in characteristic $2$.

	\begin{thm}
		If $\chr(K) \neq 2$, then $h \notin R$.
	\end{thm}
	\begin{proof} 
		Consider the element $\tau_{ad} \in \GL(V)$ that acts on $S$ by fixing $b$ and $c$, and swapping $a \leftrightarrow d$. 
		Then, it is a quick check that all the $f_{i}$ are $\tau_{ad}$-invariant. 
		Thus,
		\begin{equation*} 
			R \subset S^{\langle \widehat{G}, \tau_{ad} \rangle} \subset S^{G}.
		\end{equation*}
		However, the action of $\tau_{ad}$ on the matrix in~\Cref{eq:002} swaps the extreme columns and thus, $\tau_{ad}(h) = -h$. 
		If $\chr(K) \neq 2$, then this shows that $h$ is not $\tau_{ad}$-invariant and hence, $h \notin R$.
	\end{proof}

	\begin{rem}
		For the above argument to work, one needs that $h \neq 0$. 
		This requires a slight calculation (but is true, in any characteristic). 

		Moreover, if $\chr(K) = 2$, then the above calculation shows that $h \in S^{\langle \widehat{G}, \tau_{ad} \rangle}$. 
		It is not too difficult to show that $S^{\langle \widehat{G}, \tau_{ad} \rangle} = R$ and thus, $h \in R$ in characteristic two.
	\end{rem}

\section{Additional results}
	
	Because the $a$-invariant remains the same and the group action is \emph{modular}\footnote{The order of $\mdd{\widehat{G}}$ is divisible by $\chr(K)$}, it follows that the inclusion $S^{G} \into S$ is not split, see~\Cite[Corollary 4.2]{GoelJeffriesSingh}. 
	Thus, $S^{G}$ is not $F$-regular.

	The class group of $S^{G}$ is well-known. Because the group action contains no pseudoreflections, the class group of $S^{G}$ is given by
	\begin{equation*} 
		\operatorname{Class}(S^{G}) \cong \Hom_{\mathbf{Grp}}(\widehat{G}, K^{\times}) \cong \Hom_{\mathbb{Z}}(\widehat{G}/[\widehat{G}, \widehat{G}], K^{\times});
	\end{equation*}
	see \Cite[Theorem 3.9.2]{Benson:PolynomialInvariantsBook} for the first isomorphism.

	In particular, $S^{G}$ is a UFD iff there is no nontrivial homomorphism $\widehat{G} \to K^{\times}$. 
	One notes that $\widehat{G} \cong \PGL_{2}(K)$. 
	Some group theory gives us that
	\begin{equation*} 
		\chr(K) = 2 \Leftrightarrow \operatorname{Class}(S^{G}) = 0 \Leftrightarrow \text{$S^{G}$ is a UFD} 
	\end{equation*}
	and
	\begin{equation*} 
		\chr(K) \neq 2 \Leftrightarrow \operatorname{Class}(S^{G}) \cong \mathbb{Z}/2.
	\end{equation*}

	In fact, these results generalise readily to an arbitrary $n \ge 3$ with similar arguments: 
	if $G \coloneqq \GL_{n}(K)$ acts on $S \coloneqq K[X_{n \times n}]$ via conjugation, then
	\begin{enumerate}[label=(\alph*)]
		\item $a(S^{G}) = a(S) = -n^{2}$ and $S^{G} \into S$ does not split (hence, $S^{G}$ is not $F$-regular), and
		\item $S^{G}$ is a unique factorisation domain iff $n$ and $q - 1$ are coprime; with the class group being $\mathbb{Z}/\gcd(n, q - 1)$ in general.
	\end{enumerate}

\printbibliography
\end{document}