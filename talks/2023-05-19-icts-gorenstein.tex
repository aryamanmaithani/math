\documentclass[12pt]{article}
\usepackage[lmargin=1in,rmargin=1in,tmargin=1in,bmargin=1in]{geometry}

\usepackage{aryaman}
\setcounter{tocdepth}{2}

\DeclareMathOperator{\ty}{type}
\DeclareMathOperator{\soc}{soc}
\DeclareMathOperator{\Perf}{Perf}
\DeclareMathOperator{\Ferp}{InjPerf}
\DeclareMathOperator{\Hilb}{Hilb}

\title{Gorenstein Rings}
\author{Aryaman Maithani}
\date{May 19, 2023}

\begin{document}

\maketitle
% \tableofcontents

\section{Dualising complexes}

Let $R$ be a commutative noetherian ring. A complex $\omega_{R}$ is a \deff{dualising complex} if
\begin{enumerate}[label=(\arabic*)]
	\item $\omega_{R} \in D^{b}(\rmod{R})$,
	\item $\injdim(\omega_{R}) < \infty$, i.e., $\omega_{R}$ is quasi-isomorphic to a bounded complex of injectives,
	\item $R \xrightarrow{\simeq} \RHom_{R}(\omega_{R}, \omega_{R})$.
\end{enumerate}

$\omega_{R}$ is called \deff{semi-dualising} if (1) and (3) hold.

\begin{rem}
	$R$ is always semi-dualising. Thus, $R$ is dualising iff $\injdim R < \infty$. \newline
	Recall that $R$ is said to be \deff{regular} if $\operatorname{gldim}(R_{\mathfrak{p}}) < \infty$ for all $\mathfrak{p} \in \Spec R$. \newline
	In particular, if $R$ is regular local, then the global dimension of $R$ is finite and thus, $\injdim(R) < \infty$. Thus, $R$ is even a dualising complex.
\end{rem}

\begin{rem}[Existence]
	Dualising complexes need not exist. Indeed, if $\omega_{R} \simeq I^{\bullet}$ is a minimal injective resolution, then up to shifts, one has that
	\begin{equation*} 
		I^{-n} \cong \bigoplus_{\substack{\mathfrak{p} \in \Spec(R) \\ \dim(R/\mathfrak{p}) = n}} E_{R}(R/\mathfrak{p}),
	\end{equation*}
	where $E_{R}(-)$ denotes the injective hull. 

	The key point here is that every prime shows up exactly once, at location prescribed by its ``co-height''. In particular, if $\mathfrak{p} \subset \mathfrak{q}$ are primes, then any two saturated chains of prime ideals joining $\mathfrak{p}$ and $\mathfrak{q}$ must have the same length. That is, $R$ must be \deff{catenary}. \newline
	However, as Nagata showed, there exists a noetherian local domain $(R, \mathfrak{m})$ of dimension three that is not catenary. % ?? catenary but no dualising complex?
\end{rem}

\begin{rem}[Uniqueness]
	Even if $\omega_{R}$ exists, it need not be unique. For one, $\Sigma^{n} \omega_{R}$ is another dualising complex, for all $n \in \mathbb{Z}$. \newline
	Slightly less trivial, if $P$ is a projective $R$-module of rank $1$ (i.e., $P_{\mathfrak{p}} \cong R_{\mathfrak{p}}$ for all $\mathfrak{p} \in \Spec R$), then $\omega_{R} \otimes_{R} P$ is again dualising. 

	But this is all: Given any two dualising complexes, one can obtain the other by applying the above operations successively. In particular, if $R$ is local, then $\omega_{R}$ is unique up to shifts. (In fact, one can set $P \vcentcolon= \Sigma^{n} \RHom_{R}(\omega_{R}, \omega_{R}')$ for a suitable shift $n$.) 
\end{rem}

\begin{thm}[Local Duality Theorem]
	Let $R$ be a commutative noetherian ring with a dualising complex $\omega_{R}$. The functor
	\begin{equation*} 
		(-)^{\dagger} \vcentcolon= \RHom_{R}(-, \omega_{R}) : D(R)^{\op} \to D(R)
	\end{equation*}
	restricts to auto-equivalences
	\begin{equation*} 
		\begin{tikzcd} % [row sep=small]
		D^{b}(\rmod{R})^{\op} \arrow[r, "\cong"] & D^{b}(\rmod{R}) \arrow[l] \\
		\Perf(R)^{\op} \arrow[r, "\cong"] \arrow[u, phantom, sloped, "\subset"] & \Ferp(R), \arrow[u, phantom, sloped, "\subset"] \arrow[l]
		\end{tikzcd}
	\end{equation*}
	where $\Perf(R)$ (resp. $\Ferp(R)$) is the subcategory of objects with finite projective (resp. injective) dimension.
	% okay to use Ferp??
\end{thm}
In fact, for $M \in D^{b}(\rmod{R})$, one has that the natural map $M \to M^{\dagger \dagger}$ is a quasi-isomorphism. This is seen by checking that the map factors as
\begin{equation*} 
	M \xrightarrow{\simeq} M \otimes^{\ell}_{R} \RHom_{R}(\omega_{R}, \omega_{R}) \xrightarrow{\simeq} \RHom_{R}(\RHom_{R}(M, \omega_{R}), \omega_{R}).
\end{equation*}

\textbf{Connection to Matlis duality}. If we further assume that $(R, m, k)$ is local and $\ell(M) < \infty$, then we see that $M$ is $\mathfrak{m}$-torsion. Then, letting $I^{\bullet}$ be the minimal injective resolution (as described below), we see that
\begin{equation*} 
	M^{\dagger} = \Hom_{R}(M, I^{\bullet}) = \Hom_{R}(M, E_{R}(k))
\end{equation*}
since every other $\Hom_{R}(M, E_{R}(R/\mathfrak{p}))$ must vanish. In particular, $M^{\dagger} = M^{\vee}$. (Strictly speaking, we first ``normalise'' the complex appropriately.) \newline
Thus, Matlis duality is a special case of local duality.

\section{Gorenstein rings}

% For the sake of correctness, let us now further assume that $\dim R < \infty$. We say that $R$ is \deff{Gorenstein} if $\injdim R < \infty$, i.e., $R$ is self-dualising. If we do not assume $R$ to be 

\begin{defn}
	A noetherian local ring $(R, \mathfrak{m})$ is \deff{Gorenstein} if $\injdim R < \infty$. \newline
	More generally, a noetherian ring $R$ is \deff{Gorenstein} if $R_{\mathfrak{m}}$ is Gorenstein for all $\mathfrak{m} \in \Max R$.
\end{defn}

In the case that $\dim(R) < \infty$, $R$ is Gorenstein iff $\injdim(R) < \infty$. In this case, being Gorenstein is the same as $R$ being self-dualising. \newline
Side note: Recall Nagata's example of a noetherian domain with infinite Krull dimension. This is a regular ring and hence, Gorenstein. However, this ring has infinite injective dimension.

\begin{rem}
	One can show that $R$ being Gorenstein (in our sense) is equivalent to $\RHom_{R}(-, R)$ being an auto-equivalence on $D^{b}(\rmod{R})$.
\end{rem}

For the occasional noncommutative ring $A$, we shall use Gorenstein to mean that ${\injdim(_{A}A) < \infty}$ and $\injdim(A_{A}) < \infty$. This is what was defined as (Iwanaga-)Gorenstein in an earlier talk.

\begin{ex}
	Let $k$ be an arbitrary field, and $G$ a finite group. The group ring $kG$ is Gorenstein since it is self-injective. In the case that $\chr(k) \nmid \md{G}$, this follows since $kG$ is semisimple and thus, every $kG$-module is injective.
\end{ex}

As we remarked earlier, we have
\begin{equation*} 
	\text{regular $\Rightarrow$ Gorenstein}.
\end{equation*}
% ?? don't need dim < \infty here, right?
% (At least for local rings.)

In fact, one can check the following.

\begin{lem} 
	Let $(R, \mathfrak{m})$ be local, and $x \in \mathfrak{m}$ be a nonzerodivisor. Then, 
	\begin{equation*} 
		\text{$R$ is Gorenstein $\Leftrightarrow$ $R/xR$ is Gorenstein}.
	\end{equation*} 
\end{lem}
\begin{cor}
	For local rings, we have: Regular $\Rightarrow$ complete intersection $\Rightarrow$ Gorenstein. \newline
	% (For correctness: One requires the fact that $R$ is Gorenstein iff $\widehat{R}$ is so?)
\end{cor}

\begin{thm}
	For a noetherian local ring $(R, \mathfrak{m}, k)$ of Krull dimension $d$, the following are equivalent:
	\begin{enumerate}
		\item $R$ is Gorenstein.
		\item $\Ext_{R}^{n}(k, R) = 0$ for $n \neq d$ and $\Ext_{R}^{d}(k, R) \cong k$.
	\end{enumerate}
	% ?? what did Greenlees mention about \Sigma^{a + r}?
\end{thm}

For a local ring $(R, \mathfrak{m}, k)$ with $d \vcentcolon= \depth(R)$, we define the \deff{type} of $R$ to be the $d$-th \emph{Bass number}:
\begin{equation*} 
	\ty(R) \vcentcolon= \rank_{k} \Ext_{R}^{\depth(R)}(k, R).
\end{equation*}
The above is simply the number of copies of $E_{R}(k)$ that appear in the $d$-th spot of the minimal injective resolution of $R$.

Then, has the following.
\begin{thm}
	Let $(R, \mathfrak{m}, k)$ be local. Then,
	\begin{equation*} 
		\text{$R$ is Gorenstein $\Leftrightarrow$ $R$ is CM and $\ty(R) = 1$}.
	\end{equation*}
\end{thm}
% In fact, one even has:
% \begin{thm}[Roberts]
% 	$\ty(R) = 1 \Rightarrow R$ is Gorenstein.
% \end{thm}

In particular, if $R$ is artinian, then $R$ is Cohen-Macaulay with $d = 0$. This gives us the following.

\begin{thm}
	Let $(R, \mathfrak{m}, k)$ be local artinian. TFAE:
	\begin{enumerate}
		\item $R$ is Gorenstein.
		\item $\ty(R) = 1$.
		\item $\Hom_{R}(k, R) \cong k$.
		\item $\soc(R)$ is one-dimensional.
	\end{enumerate}
\end{thm}
Note that $\Hom_{R}(k, R) = (0 :_{R} \mathfrak{m}) =\vcentcolon \soc(R)$ is the \deff{socle} of $R$. This is the largest submodule of $R$ which has a $k$-module structure. \newline
Note that in the zero-dimensional case, the type is the dimension of the socle.

\begin{ex}
	The ring $R = k[X, Y]/(X^{2}, Y^{2})$ is an artinian ring. The socle is one-dimensional, being generated by $xy$. This was the example of the group ring $\mathbb{F}_{2} V_{4}$ which we saw yesterday.

	On the other hand, if we quotient further to get $R = k[X, Y]/(X^{2}, XY, Y^{2})$, then the socle is two dimensional: generated by $x$ and $y$.

	Both the rings \emph{are} Cohen-Macaulay, being zero-dimensional.
\end{ex}

\begin{thm}[Watanabe]
	Let $K$ be a field of characteristic zero. Let $G$ be a finite subgroup of $\GL_{n}(K)$ acting on $S \vcentcolon= K[x_{1}, \ldots, x_{n}]$. \newline
	If $G \le \SL_{n}(K)$, then $S^{G}$ is Gorenstein. \newline
	If $G$ contains no pseudoreflections, then the converse holds too.
\end{thm}

Recall that an element $g \in \GL_{n}(k)$ is called a \deff{pseudoreflection} if $\rank(g - I) = 1$.

\begin{ex}
	$\mathbb{C}[x^{2}, xy, y^{2}]$ is Gorenstein, but $\mathbb{C}[x^{3}, x^{2} y, x y^{2}, y^{3}]$ is not. These appear as invariant rings of $\left \langle \two{\zeta}{}{}{\zeta} \right \rangle$ for $\zeta = -1$ and $\exp(2 \pi \iota/3)$. This is in $\SL_{2}(\mathbb{C})$ precisely in the former case. \newline
	More generally, the $d$-th Veronese of $\mathbb{C}[x_{1}, \ldots, x_{n}]$ is Gorenstein iff $d \mid n$.

	$\mathbb{C}[x^{n}, xy, y^{n}]$ is also Gorenstein for all $n \ge 1$. This is the invariant ring corresponding $\left \langle \two{\zeta}{}{}{\zeta^{-1}} \right \rangle$ for $\zeta = \exp(2 \pi \iota/n)$.
\end{ex}

\section{Symmetry}

We now look at some more examples of Gorenstein rings.

\begin{ex}[Poincar\'{e} Algebras]
	Suppose that $R$ is a graded $k$-algebra of the form
	\begin{equation*} 
		R = k \oplus R_{1} \oplus \cdots \oplus R_{d}
	\end{equation*}
	with $\rank_{k} R < \infty$ and $R_{d} \neq 0$. Assume $R$ is either commutative or graded-commutative. $R$ being graded gives us bilinear maps
	\begin{equation*} 
		\langle -, - \rangle_{i} : R_{i} \times R_{d - i} \to R_{d}
	\end{equation*}
	for all $i \in [d]$. Equivalently, we have maps
	\begin{equation*} 
		\rho_{i} : R_{i} \to \Hom_{k}(R_{d - i}, R_{d}).
	\end{equation*}

	Then, the following are equivalent:
	\begin{enumerate}
		\item $R$ is Gorenstein.
		\item $R_{d} \cong k$ and $\langle -, - \rangle_{i}$ is nondegenerate for all $i$.
		\item $\rho_{i}$ is a bijection for all $i$.
		\item $R \cong \Hom_{k}(R, k)[d]$ as $R$-modules. (Note that $A = k$ is a Noether normalisation since $R$ is finite-dimensional over $k$.)
	\end{enumerate}
	Note that in this case, we must necessarily have $\rank_{k}(R_{i}) = \rank_{k}(R_{d - i})$.
\end{ex}

\begin{ex}
	Poincar\'{e} Duality tells us that for a compact manifold $M$, the ring $H^{\ast}(M; \mathbb{F}_{2})$ is Gorenstein. \newline
	For example, $k[x]/(x^{n + 1})$ is Gorenstein, as can be seen using the criteria above. (For $k = \mathbb{F}_{2}$, this ring is the cohomology ring of $\mathbb{R}P^{n}$.)
\end{ex}

\begin{ex}
	As a concrete example of the above, one can create many zero-dimensional Gorenstein rings as follows: Let $V$ be any finite-dimensional $k$-vector space, and let $\langle -, - \rangle$ be any (anti-)symmetric nondegenerate bilinear form on $V$. Then, give
	\begin{equation*} 
		R \vcentcolon= k \oplus V \oplus k
	\end{equation*}
	a graded ring structure in the obvious way. % (Multiplying elements of $V$ with the $k$ in degree $2$ gives zero; multiplication of two elements of $V$ is done using the pairing; multiplying elements of $k$ in degree $2$ gives zero.)

	\hrulefill
	
	In fact, our earlier example $R = k[X, Y]/(X^{2}, Y^{2})$ fits in this form. We have
	\begin{equation*} 
		R = k \oplus k \cdot \{x, y\} \oplus k \cdot \{xy\}
	\end{equation*}
	with the pairing being given by $\smatrix{0 & 1 \\ 1 & 0}$. 

	Similarly, the example $k[X, Y]/(X^{2}, XY, Y^{2})$ can be seen as non-Gorenstein using the above criteria since the top graded piece has dimension $> 1$.

	On the other hand, $k[X, Y]/(X^{3}, XY, Y^{2}) \cong k \oplus k^{2} \oplus k$ is not Gorenstein because the pairing is degenerate. (Note that the dimensions are still palindromic!)
	
	\hrulefill
	
	Let us consider $R = k \oplus k \cdot \{x, y, z\} \oplus k \cdot T$ with pairing given by
	\begin{equation*} 
		z^{2} = xy = yx = T,
	\end{equation*}
	and all other pairs in $\{x, y, z\} \times \{x, y, z\}$ get mapped to zero. This is nondegenerate as this is given by the invertible matrix
	\begin{equation*} 
		\begin{pmatrix}
			0 & 1 & 0 \\
			1 & 0 & 0 \\
			0 & 0 & 1 \\
		\end{pmatrix}.
	\end{equation*}
	Thus, the ring $R$ is Gorenstein. In fact, it is not too difficult to check that we have
	\begin{equation*} 
		R \cong k[X, Y, Z]/(X^{2}, Y^{2}, XZ, YZ, Z^{2} - XY).
	\end{equation*}
	One can also show that the ideal above does not have fewer than $5$ generators. Thus, $R$ gives an example of a Gorenstein ring which is not a complete intersection.
\end{ex}

\begin{ex}
	Suppose $R$ is a noetherian $\mathbb{N}$-graded ring with $K \vcentcolon= R_{0}$ a field. Let $A \subset R$ be a Noether normalisation. Then, $R$ is Gorenstein iff $R$ is Cohen-Macaulay and there exists $q \in \mathbb{Z}$ such that $\Hom_{A}(R, A) \cong R[q]$ as graded $R$-modules.

	Using this, one gets that ``the'' numerator of the Hilbert series is a palindrome. More precisely: consider $A \subset R$, where $A$ is a homogeneous Noether normalisation and $R$ is Cohen-Macaulay. Say $A = K[f_{1}, \ldots, f_{n}]$, $\deg f_{i} = k_{i}$. Then,
	\begin{equation*} 
		\Hilb(A, t) = \frac{1}{\prod (1 - t^{k_{i}})} \andd \Hilb(R, t) = \frac{c_{0} + \cdots + c_{m} t^{m}}{\prod (1 - t^{k_{i}})},
	\end{equation*}
	where $c_{j}$ is the number of basis elements of degree $j$ in a homogeneous $A$-basis for $R$. Then, $\Hom_{A}(R, A) \cong R[q]$ gives us that
	\begin{equation*} 
		(c_{0}, \ldots, c_{m}) = (c_{m}, \ldots, c_{0}),
	\end{equation*}
	since $\Hom_{A}(-, A)$ negates the degrees of the basis elements.

	(We assume $c_{m} \neq 0$.)

	A more inherent symmetry is given as:
	\begin{equation} \label{eq:01}
		\Hilb(R, t^{-1}) = (-1)^{\dim R} t^{\ell} \Hilb(R, t)
	\end{equation}
	for some $\ell \in \mathbb{Z}$.

	In fact, Stanley proved a converse as well: If $R$ is a Cohen-Macaulay graded domain satisfying \Cref{eq:01}, then $R$ is Gorenstein. \newline
	Without the additional hypothesis, \Cref{eq:01} does not suffice: we already have an example from earlier.
\end{ex}

\begin{ex}[Numerical monoids]
	A \deff{numerical monoid} $\Sigma$ is a subset $\Sigma \subset \mathbb{N}_{0}$ such that
	\begin{enumerate}[label=(\arabic*)]
		\item $0 \in \Sigma$,
		\item $\Sigma$ is closed under addition,
		\item $\mathbb{N}_{0} \setminus \Sigma$ is finite.
	\end{enumerate}

	Let $k$ be a field. Corresponding to $\Sigma$ above, we get a ring
	\begin{equation*} 
		k[\Sigma] \vcentcolon= k[x^{i} : i \in \Sigma] \subset k[x].
	\end{equation*}
	That is, $k[\Sigma]$ is the $k$-subalgebra of $k[x]$ generated by $\{x^{i} : i \in \Sigma\}$. 

	By (3), there exists a smallest $c \ge 0$ such that $x^{\ge c} \in k[\Sigma]$. Then, $R$ is Gorenstein iff exactly half the numbers in $[0, c - 1]$ are not in $\Sigma$. Equivalently, the holes are anti-palindromic, i.e., for all $0 \le i \le c - 1$, $i \in \Sigma \Leftrightarrow c - 1 - i \notin \Sigma$.
\end{ex}

\begin{ex}
	As concrete examples, one can check $R = K[x^{3}, x^{5}, x^{7}]$ is not Gorenstein. (Check with $c = 5$.)

	On the other hand, $R = K[x^{4}, x^{5}, x^{6}]$ is Gorenstein. (Check with $c = 8$.)
\end{ex}

\begin{comment}
\section{References}

Two good papers for exposition and examples of Gorenstein rings are:
\begin{itemize}
	\item `On the ubiquity of Gorenstein rings' by \emph{Hyman Bass}.
	\item `Hyman Bass and Ubiquity: Gorenstein Rings' by \emph{Craig Huneke}.
\end{itemize}

The material for this talk was mainly prepared using the material I learnt from a course on commutative algebra that I took under Professor Srikanth Iyengar at the University of Utah. 
\end{comment}
\end{document}