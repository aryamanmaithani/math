\documentclass[dvipsnames]{beamer}
% \mode<presentation>{}
\usepackage[utf8]{inputenc}
\usepackage{amsmath, amssymb, amsfonts, amsthm, mathtools, mathrsfs}

% \usepackage{enumitem}
\setbeamertemplate{theorems}[numbered]
\title[Extending Conformal Mappings]{Extending Conformal Mappings Onto the Unit Disc}
\author{Aryaman Maithani}
\date[16-03-2022]{16th March 2022}
\institute{IIT Bombay}
\usetheme{Madrid}
\usepackage{parskip}
\usepackage{tcolorbox}
\usepackage{tikz-cd}
\usepackage{commands}
\usepackage{graphicx}
% \usepackage{soul}

\usepackage{tikz}
% \usetikzlibrary{topaths,calc}
% \usepackage{caption}
% \usepackage{subcaption}
% \usepackage{pgfplots}  
% \pgfplotsset{compat=1.17}
% \usepackage{cancel}
% \usepgfplotslibrary{polar}   
% \usepgflibrary{shapes.geometric}  
% \usetikzlibrary{calc}  
% \pgfplotsset{posQuad/.append style={grid=both, xlabel={$x$}, ylabel={$y$}, line width=2pt, mark size=3pt,draw=NordWhite}}

\tikzset{
    invisible/.style={opacity=0},
    visible on/.style={alt={#1{}{invisible}}},
    alt/.code args={<#1>#2#3}{%
      \alt<#1>{\pgfkeysalso{#2}}{\pgfkeysalso{#3}}%
  }
}

\newcommand\mynum[1]{%
  \usebeamercolor{enumerate item}%
  \tikzset{beameritem/.style={circle,inner sep=0,minimum size=2ex,text=enumerate item.bg,fill=enumerate item.fg,font=\footnotesize}}%
  \tikz[baseline=(n.base)]\node(n)[beameritem]{#1};%
}

\makeatletter
\newenvironment<>{proofs}[1][\proofname]{%
    \par
    \def\insertproofname{#1\@addpunct{.}}%
    \usebeamertemplate{proof begin}#2}
  {\usebeamertemplate{proof end}}
\makeatother

% \setbeamercolor{footline}{fg=brown}
% \setbeamerfont{footline}{series=\bfseries}
% \addtobeamertemplate{navigation symbols}{}{%
%     \usebeamerfont{footline}%
%     \usebeamercolor[fg]{footline}%
%     \hspace{1em}%
%     [\insertframenumber/\inserttotalframenumber]
% }

\theoremstyle{definition}
\newtheorem{thm}{Theorem}
\newtheorem{defn}[thm]{Definition}
\newtheorem{prop}[thm]{Proposition}
\newtheorem{cor}[thm]{Corollary}
\newtheorem{caution}[thm]{Caution}
\newtheorem*{ques}{Question}
\newtheorem{rem}[thm]{Remark}
% \newtheorem*{fac}{Fact}
% \newtheorem*{ex}{Example}

\let\subset\subseteq
\let\supset\supseteq
\let\ge\geqslant
\let\le\leqslant

\AtBeginSection[]
{
  \begin{frame}
    \frametitle{Table of Contents}
    \tableofcontents[currentsection]
  \end{frame}
}

\begin{document}
\begin{frame}
    \titlepage
\end{frame}

% \begin{frame}
%     \frametitle{Table of Contents}
%     \tableofcontents
% \end{frame}

% \section{Introduction}

\begin{frame}{Notations and Conventions}
    \begin{enumerate}[<+->]
        \item $\mathbb{D}$ will denote the open unit disc. $S^{1} \vcentcolon= \partial \mathbb{D}$ is the unit circle.
        \item $\mathcal{O}^{\infty}$ denotes the set of bounded holomorphic functions on $\mathbb{D}$.
        \item $\Omega$ will always denote a nonempty, open, bounded, and simply-connected subset of $\mathbb{C}$.
        \item Recall that a conformal mapping of $\Omega$ onto $\mathbb{D}$ is simply a biholomorphism $\Omega \to \mathbb{D}$.
        \item A curve shall mean a continuous function with domain $[0, 1]$. \\
        Typically, $\gamma$ will be a curve such that $\gamma([0, 1)) \subset \Omega$ and $\gamma(1) \in \partial \Omega$. \\
        Similarly, $\Gamma$ will be a curve such that $\Gamma([0, 1)) \subset \mathbb{D}$ and $\Gamma(1) \in \partial \mathbb{D}$.
    \end{enumerate}
\end{frame}

\begin{frame}{Introduction}
    Let $\Omega$ be a bounded simply-connected domain in $\mathbb{C}$. \pause By the Riemann Mapping Theorem, we know that there exists a biholomorphism $f : \Omega \to \mathbb{D}$. \pause The following is a natural question. \pause

    \begin{ques}
        Can $f$ be continuously extended up to $\overline{\Omega}$?
    \end{ques}

    \pause The obvious way to extend $f$ is \pause via sequences. \pause In fact, if an extension exists, this \emph{is} how it must be obtained. \pause In particular, this extension is unique \pause and we must have $f(\overline{\Omega}) \subset \overline{\mathbb{D}}$. \pause This must force $f(\overline{\Omega}) = \overline{\mathbb{D}}$. \pause (Why?) \pause 
    
    Thus, it also makes sense to ask whether \emph{the} extension is a homeomorphism onto $\overline{\mathbb{D}}$.
\end{frame}

\begin{frame}{Spoilers}
    What we shall see is the following: \pause By imposing a \emph{simple} topological restriction on $\Omega$, \pause one gets that \emph{any} biholomorphism $\Omega \to \mathbb{D}$ \pause can be extended to a continuous \emph{injective} map $\overline{\Omega} \to \overline{\mathbb{D}}$. \pause Moreover, this will be a \emph{homeomorphism}. \pause

    \begin{rem}
        The last line is not difficult to see. \pause Indeed, once we have continuously extended $f$ to $\widetilde{f} : \overline{\Omega} \to \overline{\mathbb{D}}$, \pause we have
        \begin{equation*} 
            \mathbb{D} \subset \widetilde{f}(\overline{\Omega}) \subset \overline{\mathbb{D}}. 
        \end{equation*}
        \pause 
        As $\widetilde{f}(\overline{\Omega})$ is compact, \pause we have $\widetilde{f}(\overline{\Omega}) = \overline{\mathbb{D}}$. \pause \\~\\
        Furthermore, if $\widetilde{f}$ is an injection, \pause then compactness again tells us that $\widetilde{f}$ is a homeomorphism (as $\widetilde{f}$ is a bijection).
    \end{rem}
\end{frame}

\begin{frame}{Simple Boundary Points}
    \begin{defn} \label{defn:simple} \pause
        A boundary point $\beta$ of $\Omega$ is called a \deff{simple boundary point} if $\beta$ has the following property: \pause \\
        For every sequence $(\alpha_{n})_{n}$ in $\Omega$ \pause such that $\alpha_{n} \to \beta$ as $n \to \infty$, \pause there exists a curve $\gamma$ \pause and a strictly increasing sequence $(t_{n})_{n}$ in $(0, 1)$ \pause such that
        \begin{equation*} 
            t_{n} \to 1, \; \pause \gamma(t_{n}) = \alpha_{n} \; (n = 1, 2, \ldots,), \; \pause \gamma([0, 1)) \in \Omega. \pause 
        \end{equation*}
        $\gamma(1) = \beta$ follows by continuity. \pause
    \end{defn}
    In words: there is a curve in $\Omega$ which passes through $\alpha_{n}$ and ends at $\beta$. 
\end{frame}

\begin{frame}{(Non-)Examples}
    \pause 
    \begin{enumerate}
        \item Every boundary point of $\mathbb{D}$ is a simple boundary point. \pause
        \item Let $\Omega \vcentcolon= \mathbb{D} \setminus [0, 1)$. 
        \begin{center}
            \begin{tikzpicture}
                \filldraw[color=red!60, fill=red!5, very thick](0,0) circle (1.5);
                \draw[red!60, very thick] (0, 0) -- (1.5, 0);
            \end{tikzpicture}
        \end{center}
        \pause 
        The boundary points of $\Omega$ lying on the real axis are not simple. \pause Note that $\Omega$ is indeed bounded and simply-connected and thus, biholomorphic to $\mathbb{D}$. \pause However, $\partial \Omega$ is clearly not homeomorphic to $\partial \mathbb{D}$ and \pause thus, no biholomorphism can be extended to a homeomorphism $\overline{\Omega} \to \overline{\mathbb{D}}$.
    \end{enumerate}
\end{frame}

\begin{frame}{Some constraints}
    \begin{thm}[Helper Theorem] \pause
        Let $\Omega$ be a bounded simply-connected domain, and let $f$ be a conformal mapping of $\Omega$ onto $\mathbb{D}$. \pause
        \begin{enumerate}
            \item If $\beta$ is a simple boundary point of $\Omega$, \pause then $f$ has a continuous extension to $\Omega \cup \{\beta\}$. \pause If $f$ is so extended, then $\md{f(\beta)} = 1$. \pause
            \item If $\beta_{1}$ and $\beta_{2}$ are distinct simple boundary points of $\Omega$ \pause and \emph{if} $f$ is continuously extended to $\Omega \cup \{\beta_{1}, \beta_{2}\}$, \pause then $f(\beta_{1}) \neq f(\beta_{2})$.
        \end{enumerate}
    \end{thm}
    \pause We give the proof after proving the main theorem assuming the above.

    As remarked, the extension in \mynum{1} is unique and would have to be attained as follows: \pause given a sequence $(\alpha_{n})_{n}$ in $\Omega$ converging to $\beta$, \pause we have $f(\beta) \vcentcolon= \lim f(\alpha_{n})$. \pause Once we show that this limit (exists and) is independent of the sequence $(\alpha_{n})$, \pause we would have shown continuity. 
\end{frame}

\begin{frame}{The Main Extension Theorem}
    \begin{thm}
        If $\Omega$ is a bounded simply-connected domain \pause and if every boundary point of $\Omega$ is simple, \pause then every conformal mapping of $\Omega$ onto $\mathbb{D}$ \pause extends to a homeomorphism of $\overline{\Omega}$ onto $\overline{D}$.
    \end{thm}
    \pause 
    \begin{proof} \pause 
        Let $f : \Omega \to \mathbb{D}$ be a biholomorphism. \pause By the Helper Theorem and the remark following it, \pause we see that we may extend $f$ to $\overline{\Omega}$ using sequences. \pause By \mynum{2}, \pause it follows that $f$ so extended is one-one. \pause We now check that it is continuous on $\overline{\Omega}$. \pause As remarked earlier, this would finish the proof. \pause \\~\\
        %
        To this end, let $(z_{n})_{n}$ be an arbitrary sequence in $\overline{\Omega}$ that converges to $z$. \pause We can pick a corresponding sequence $(\alpha_{n})_{n}$ in $\Omega$ such that \pause $\md{\alpha_{n} - z_{n}} < 1/n$ \pause and $\md{f(\alpha_{n}) - f(z_{n})} < 1/n$. \pause Thus, $\alpha_{n} \to z$ and hence, $f(\alpha_{n}) \to f(z)$. \pause In turn, $f(z_{n}) \to f(z)$, as desired. \pause
    \end{proof}
\end{frame}

% \begin{frame}{Proof of the Main Theorem}
%     \begin{proof}[Proof (continued)]
%         Thus, we have shown that $f : \overline{\Omega} \to \overline{D}$ is continuous. Note that
%         \begin{equation*} 
%             \mathbb{D} \subset f(\overline{\Omega}) \subset \overline{\mathbb{D}}. 
%         \end{equation*}
%         As $f(\overline{\Omega})$ is compact, we have $f(\overline{\Omega}) = \overline{\mathbb{D}}$. As noted earlier, (the extended) $f$ is one-one. Since $\overline{\Omega}$ is compact and $\overline{\mathbb{D}}$ Hausdorff, it follows that $f$ is a homeomorphism.
%     \end{proof}
% \end{frame}

\begin{frame}{A Purely Topological Corollary}
    Recall that a Jordan curve is the image of an injective map $S^{1} \to \mathbb{C}$. \pause

    \begin{cor} \pause 
        If every boundary point of a bounded simply-connected region $\Omega$ is simple, \pause then the boundary of $\Omega$ is a Jordan curve, \pause and $\overline{\Omega}$ is homeomorphic to $\overline{\mathbb{D}}$.
    \end{cor}

    \pause In fact, the converse is true too: \pause If the boundary of $\Omega$ is a Jordan curve, then every boundary point of $\Omega$ is simple.
\end{frame}

\begin{frame}{Blackboxes}
    \begin{thm}[Radial Limit Theorem] \pause 
        To every $g \in \mathcal{O}^{\infty}$ corresponds a function $g^{\ast} \in L^{\infty}(S^{1})$, \pause defined almost everywhere by \pause
        \begin{equation*} 
            g^{\ast}(e^{\iota \theta}) = \lim_{r \to 1} g(r e^{\iota \theta}).
        \end{equation*} 
        \pause
        % The equality $\|g\|_{\infty} = \|g^{\ast}\|_{\infty}$ holds. \pause
        %
        If $g^{\ast}(e^{\iota \theta}) = 0$ \pause for almost all $e^{\iota \theta}$ on some arc $J \subset S^{1}$, \pause then $g(z) = 0$ for every $z \in \mathbb{D}$. \pause
    \end{thm}
    \begin{thm}[Lindel\"{o}f's Theorem] \pause
        Suppose $\Gamma$ is a curve such that $\Gamma([0, 1)) \subset \mathbb{D}$ and $\Gamma(1) = 1$. \pause \\
        If $g \in \mathcal{O}^{\infty}$ and \pause
        \begin{equation*} 
            \lim_{t \to 1^{-}} g(\Gamma(t)) = L, \pause
        \end{equation*}
        then $g$ has radial limit $L$ at $1$. 
    \end{thm}
\end{frame}

\begin{frame}{Proof of Helper Theorem}
    Now, we prove the earlier Helper Theorem assuming the earlier two results. \pause 

    \mynum{1} Suppose that $f$ cannot be extended to $\beta$. \pause Then, there exists a sequence $(\alpha_{n})_{n}$ in $\Omega$ \pause and points $w_{1}, w_{2} \in \overline{\mathbb{D}}$ such that \pause 
    \begin{equation*} 
        \alpha_{n} \to \beta, \pause f(\alpha_{2n}) \to w_{1}, \pause f(\alpha_{2n + 1}) \to w_{2}, \pause w_{1} \neq w_{2}. \pause 
    \end{equation*}
    Choose $\gamma$ as given by $\beta$ being a simple boundary point, \pause and put $\Gamma \vcentcolon= f \circ \gamma$. \pause Let $g = f^{-1}$ \pause and put $K_{r} \vcentcolon= g(\overline{D}(0; r))$ for $0 < r < 1$. \pause Then $K_{r}$ is a compact subset of $\Omega$. \pause Since $\gamma(t) \to \beta$ as $t \to 1$, \pause there exists $t^{\ast} < 1$ (depending on $r$) \pause such that $\gamma(t) \notin K_{r}$ if $t^{\ast} < t < 1$. \pause Thus, $\md{\Gamma(t)} \to 1$ as $t \to 1$. \pause In particular, $\md{w_{1}} = \md{w_{2}} = 1$. \pause

    Let $J$ be one of the open arcs of $S^{1} \setminus \{w_{1}, w_{2}\}$ \pause such that every radius of $\mathbb{D}$ which ends at a point of $J$ \pause intersects the range of $\Gamma$ in a set which has a limit point on $S^{1}$. \pause By the Radial Limit Theorem, $g$ has radial limits a.e. on $S^{1}$ \pause since $g \in \mathcal{O}^{\infty}$ \pause (as $\Omega$ is bounded).
\end{frame}

\begin{frame}{Continuing the Proof}
    We have that $g \circ \Gamma = \gamma$ and that $g$ has radial limits a.e. on $S^{1}$. \pause Now, for whichever $e^{it} \in J$ the radial limit does exist, \pause we must have
    \begin{equation*} 
        \lim_{r \to 1} g(r e^{it}) = \pause \beta, \pause
    \end{equation*}
    since $g(\Gamma(t)) = \gamma(t) \to \beta$ as $t \to 1$. \pause Thus, by the Radial Limit Theorem again, \pause applied to $g - \beta$, \pause we see that $g \equiv \beta$ on $\mathbb{D}$, contradicting that \pause $g$ is an injection. \pause 

    Thus, we have shown that $w_{1} = w_{2}$ \pause and $\md{w_{1}} = 1$. 
\end{frame}

\begin{frame}{Continuing the Proof}
    \mynum{2} Now, we need to prove that an extension takes different values at different boundary points. \pause Let $\beta_{1}, \beta_{2}$ be simple boundary points with $f(\beta_{1}) = f(\beta_{2})$. \pause We may assume $f(\beta_{i}) = 1$. \pause Let $\gamma_{i}$ be curves with $\gamma_{i}([0, 1)) \subset \Omega$, \pause and $\gamma_{i}(1) = \beta_{i}$, \pause and let $\Gamma_{i} \vcentcolon= f \circ \gamma_{i}$. \pause Then, each $\Gamma_{i}$ satisfies the condition of Lindel\"{o}f's Theorem with \pause
    \begin{equation*} 
        \lim_{t \to 1} g(\Gamma_{i}(t)) = \lim_{t \to 1} \gamma_{i}(t) = \beta_{i}. \pause
    \end{equation*}
    Thus, the radial limit of $g$ at $1$ is \pause both $\beta_{1}$ and $\beta_{2}$ \pause and hence, $\beta_{1} = \beta_{2}$. \hfill $\qed$
\end{frame}

\end{document}