\section{Group representations} \label{sec:01}

\subsection{Definition and Examples}
\begin{defn}%[Representation]
	A \deff{representation} of a group $G$ is a homomorphism $\varphi:G\to \GL(V)$ for some (finite-dimensional) vector space $V.$ The dimension of $V$ is called the \emph{degree} of $\varphi.$ We write $\varphi_g$ for $\varphi(g)$ and $\varphi_g(v)$ or simply $\varphi_gv,$ for the action of $\varphi_g \in \GL(V)$ on $v \in V.$
\end{defn}

\begin{rem}
	We shall implicitly assume that $V \neq 0$ from hereon, even though the definition doesn't explicitly demand that. Note that $\GL(V)$ would be the trivial group if $V = 0$ and there isn't much to discuss about that.
\end{rem}

\begin{rem}
	Since representation are homomorphisms, if a group $G$ is generated by $X,$ then a representation $\varphi$ of $G$ is determined by its values on $X.$ Of course, one must keep in mind that not every assignment of values to $X$ will actually determine a homomorphism. 
\end{rem}

\begin{rem} \label{rem:groupaction}
	Recall the group $S_{X}$ which is the group of all bijections from $X$ to itself. If we consider a vector space $V,$ we see that $\GL(V)$ is a subgroup of $S_V.$ Recall from basic algebra, the concept of group actions. One may define it ($G$ acting on a set $X$) as a certain map satisfying some properties but one sees that it was simply equivalent to a group homomorphism $\varphi:G \to S_X.$ 

	In this sense, we see that representations are special group actions where we don't just want $\varphi_g$ to be \emph{bijections} but also to be \emph{linear}.
\end{rem}

\begin{ex}
	Recall from \Cref{subsec:linearisation}, the concept of \nameref{defn:linearisation}. Given a set $X,$ we can construct the $\mathbb{C}$-vector space $\mathbb{C}X$ with $X$ as a basis.

	Now given a group $G$ which acts on $X,$ we saw in \Cref{prop:extendingactiontorep} that the action can actually be extended to an action on $\mathbb{C}X.$ Moreover, it has the property that each element acts linearly, i.e., we get a representation.
\end{ex}

Note that if $V$ is a $\mathbb{C}$ vector space of dimension $1,$ then $V \cong \mathbb{C}$ and $\GL(V) \cong \mathbb{C}^*.$ For psychological reasons, we may sometimes use $z$ instead of $\varphi$ for a degree one representation, to remind us that $\varphi_g$ is simply multiplication by a complex number.

\begin{ex}[Trivial representation] \label{ex:trivialrepresentation}
	The trivial representation of a group $G$ is the homomorphism $z:G \to \mathbb{C}^*$ given by $z_g = 1$ for all $g \in G.$ % amnote: shouldn't it be \GL(V) instead of C^* and \varphi(g) = Id for all g?

	This is a degree one representation.
\end{ex}

\begin{ex}[Degree one representations of $\mathbb{Z}/n\mathbb{Z}$] \label{ex:ZnZCstardeg1}
	Given $n \in \mathbb{N},$ note that a homomorphism $z:\mathbb{Z}/n\mathbb{Z} \to \mathbb{C}^*$ is determined by mapping $[1]$ to an element $\zeta \in \mathbb{C}^*$ such that $\zeta^n = 1.$ Thus, we must map $[1]$ to an $n$-th root of unity. One such example of a representation is
	\begin{equation*} 
		z([m]) \vcentcolon= \omega_n^m.
	\end{equation*}
	This is again a degree one representation. Another such is
	\begin{equation*} 
		z([m]) \vcentcolon= \omega_n^{-m}
	\end{equation*}	
	In fact, for each $k = 0, \ldots, n - 1,$ we get a different degree one representation given by
	\begin{equation*} 
		z^{(k)}([m]) = \omega_n^{mk}.
	\end{equation*}
	We will eventually show that the above are the ``only'' representations (in some sense) of a finite cyclic group.

	On the other hand, for $\mathbb{Z},$ we note that giving a homomorphism $z:\mathbb{Z}\to\mathbb{C}^*$ is the same as giving an element of $\mathbb{C}^*.$ Thus, we have uncountably many distinct representations of $\mathbb{Z}.$ Moreover, if $z_1$ is not some $n$-root of unity, then $z$ will be a injection of $\mathbb{Z}$ into $\mathbb{C}^*.$
\end{ex}

Soon, we shall define a concept of ``equivalence'' of representations. We shall then show that all the representations mentioned above are actually inequivalent.

\begin{rem} \label{rem:finrepintoS1}
	Note that if $G$ is a finite group and $z$ a degree one representation, then $z:G\to\mathbb{C}^*$ is actually very restrictive. Note that we must have $z_g^{\md{G}} = 1$ and thus, $\varphi$ actually maps into\footnotemark\ the unit circle. In fact, it further maps into the subgroup which is the $\md{G}$-th roots of unity.
\end{rem}
\footnotetext{``into'' does not mean ``injectively''}

\begin{ex}[Degree one representations of $S_n$] \label{ex:degonerepsSn}
	Note that we already have two obvious degree one representations of $S_n.$ The first is the trivial one and the second is the $\sign$ homomorphism mapping each permutation to its sign. (Recall the definition \nameref{defn:signofperm} and the fact it is a homomorphism, \Cref{cor:signisahomo}.)

	We now show that these are all. Firstly, note that all transpositions are conjugates. (Recall \nameref{thm:descconjclassSn}.) Hence, if the kernel of a representation contains one transposition, it must contain all. (Since kernels are normal.) 

	Also note that transpositions generate $S_n.$ Thus, any homomorphism is completely determined by its values on the transpositions. 

	Now, noting that any transposition has order $2,$ we see that it can only mapped to $\pm 1.$ If a single transposition is mapped to $1,$ then all are; this gives us the trivial representation. Thus, if the representation is non-trivial, then each transposition is mapped to $-1.$ However, then it must agree with $\sign.$ 
\end{ex}

\begin{ex}[Degree one representations of non-abelian groups] \label{ex:deg1factorthrough}
	Let $G$ be a group and let $z : G \to \mathbb{C}^*$ be a degree one representation. Noting that $\mathbb{C}^*$ is abelian, we see that the $\ker z$ must contain the (normal) commutator subgroup $[G, G].$ Thus, it must factor through the quotient as follows:

	\begin{center}
		\begin{tikzcd}
		G \arrow[dd, two heads] \arrow[rr, "z"]           &  & \mathbb{C}^* \\
		                                                        &  &\\
		{G/[G, G]} \arrow[rruu, "\widetilde{z}"', dashed] &  &             
		\end{tikzcd}
	\end{center}

	In other words, it then suffices to study the degree one representations of the abelian group $G/[G, G].$
\end{ex}

\begin{ex}[Determining conjugacy using representations]
	Note that since $\varphi$ is a homomorphism, the (images of the) relations that hold in $G$ must also hold in $\GL(V).$ In particular, some relations which are easier to solve in $\GL(V)$ may help in solving those in $G.$

	To give a specific example, consider the problem of having $x, y \in G$ and wanting to find $g \in G$ such that $gxg^{-1} = y.$ A priori, there may not even be a way of deducing whether such a $g$ exists. However, considering a representation $\varphi,$ we can try to solve
	\begin{equation*} 
		\varphi_g\varphi_x\varphi_g^{-1} = \varphi_y.
	\end{equation*}
	Since similarity of matrices is better solvable (at least, in theory for up to degree four representations, we can find all eigenvalues of $\varphi_x, \varphi_y$) by Jordan form, we can hope to get a better answer. If $\varphi_x$ and $\varphi_y$ are \emph{not} similar (even consideration of trace or determinant could possibly tell us that), we know for a fact that such a $g$ cannot exist.

	Now, if we see that they are similar, we can find a matrix $M$ such that $M\varphi_xM^{-1} = \varphi_y.$ Then, elements in the set $\varphi^{-1}(M)$ are good candidates for $g.$ (Of course, it could be possible that that set is empty or that none of them actually work.)
\end{ex}

We now make the following observation:\\
Let $\varphi : G \to \GL(V)$ be a representation of degree $n.$\\
Now, suppose that we have two bases $B, B'$ of $V.$ Corresponding to these, we get two isomorphisms
\begin{equation*} 
	T : V \to \mathbb{C}^n \andd S : V \to \mathbb{C}^n
\end{equation*}
mapping the basis elements to the standard basis vectors of $\mathbb{C}^n.$ Now, we have two representations
\begin{equation*} 
	\psi : G \to \GL (\mathbb{C}^n) \andd \psi' : G \to \GL (\mathbb{C}^n)
\end{equation*}
obtained by setting
\begin{equation*} 
	\psi_g \vcentcolon= T\varphi_gT^{-1} \andd \psi'_g \vcentcolon= S\varphi_gS^{-1}.
\end{equation*} 
Note that $\psi$ and $\psi'$ are related to each other by
\begin{equation*} 
	\psi_g' = (ST^{-1})\psi_g(ST^{-1})^{-1}.
\end{equation*}
This resembles a ``change of basis'' and we would wish for $\varphi, \psi, \psi'$ to be considered as the ``same'' representation. To this end, we have the following definition.

\begin{defn}%[Equivalence]
	Two representations $\varphi : G \to \GL(V)$ and $\psi : G \to \GL(W)$ are said to be \deff{equivalent} if there exists an isomorphism (called an \deff{equivalence}) $T : V \to W$ such that 
	\begin{equation*} 
		\psi_g = T\varphi_g T^{-1}
	\end{equation*}
	\underline{for all $g \in G.$} In such a case, we write $\varphi\sim\psi.$ Note that the above definition is saying that the following diagram commutes
	\begin{center}
		\begin{tikzcd}
			{V} \arrow[rr, "\varphi_g"]\arrow[dd, "T"'] & & {V}\arrow[dd, "T"]\\
			& & \\
			{W} \arrow[rr, "\psi_g"'] & & {W}
		\end{tikzcd}
	\end{center}
	\underline{for all $g \in G.$}
\end{defn}
The above equivalence can easily checked to be an actual ``equivalence relation.''
\begin{rem}
	Note that in the above, we have not assumed $V = W.$ However, $V$ and $W$ must be \emph{isomorphic}. In particular, $\varphi$ and $\psi$ have the same degree.
\end{rem}

\begin{ex}[Equivalent degree two representations of $\mathbb{Z}/n\mathbb{Z}$] \label{ex:ZnZGL2Cequiv}
	Define $\varphi, \psi:\mathbb{Z}/n\mathbb{Z} \to GL_2(\mathbb{C})$ by
	\begin{equation*} 
		\varphi_{[m]} = \begin{bmatrix}
			\cos\left(\dfrac{2\pi m}{n}\right) & -\sin\left(\dfrac{2\pi m}{n}\right)\\
			\sin\left(\dfrac{2\pi m}{n}\right) & \cos\left(\dfrac{2\pi m}{n}\right)
		\end{bmatrix} 
		\andd \psi_{[m]} = \begin{bmatrix}
			\omega_n^m & 0\\
			0 & \omega_n^{-m}
		\end{bmatrix}.
	\end{equation*}
	Then, we have the automorphism of $\GL_2(\mathbb{C})$ induced by 
	\begin{equation*} 
		A = \two{\iota}{-\iota}{1}{1}.
	\end{equation*}
	We have
	\begin{equation*} 
		A^{-1} = \dfrac{1}{2\iota}\two{1}{\iota}{-1}{\iota}
	\end{equation*}
	and a direct computation shows that
	\begin{equation*} 
		A^{-1}\varphi_{[m]}A = \psi_{[m]}
	\end{equation*}
	for all $[m] \in \mathbb{Z}/n\mathbb{Z}.$ Thus, we have $\varphi\sim\psi.$
\end{ex}

\begin{prop} \label{prop:distinctdeg1inequiv}
	Let $G$ be a group and $z, z':G \to \mathbb{C}^*$ be degree one representations. Then, $z \sim z'$ iff $z = z'.$
\end{prop}
\begin{proof} 
	Clearly, equality implies equivalence. We show the converse.

	Let $T : \mathbb{C} \to \mathbb{C}$ be an isomorphism such that $z'_g = Tz_gT^{-1}$ for all $g \in G.$ Then, for any $v \in \mathbb{C},$ we have
	\[\begin{WithArrows}[displaystyle]
		z'_g(v) &= T\left(z_gT^{-1}(v)\right) \Arrow{$T$ is linear} \\
		&= z_g T\left(T^{-1}(v)\right)\\
		&= z_g v.
	\end{WithArrows}\]
	Thus, $z_g = z'_g$ for all $g \in G$ and hence, $z = z'.$
\end{proof}

\begin{cor}
	All the distinct representations in \Cref{ex:ZnZCstardeg1} are actually inequivalent.
\end{cor}
\begin{cor}
	There are exactly $n$ distinct inequivalent degree one representations of $\mathbb{Z}/n\mathbb{Z}.$
\end{cor}
\begin{proof} 
	We had constructed $n$ distinct (and hence, inequivalent) representations in \Cref{ex:ZnZCstardeg1}. 

	To see that these are all, note that $\varphi$ is completely determined once we define $\varphi_{[1]}.$ Moreover, $\varphi_{[1]}^n$ must be $1,$ since $\varphi$ is a homomorphism. That is, $\varphi_{[1]}$ must be an $n$-th root of unity. Thus, we have at most $n$ homomorphisms.
\end{proof}

In fact, the above can be generalised to all finite abelian groups.
\begin{cor} \label{cor:deg1repsfinabel}
	Let $G$ be a finite abelian group. There are exactly $\md{G}$ inequivalent degree one representations of $G.$
\end{cor}
\begin{proof} 
	By \Cref{cor:GconghatG}, there are exactly $\md{G}$ homomorphisms from $G$ to $\mathbb{C}^*.$ By \Cref{prop:distinctdeg1inequiv}, these are all inequivalent as well and we are done.
\end{proof}

\begin{ex}[Degree one representations of non-abelian groups revisited]
	As noted in \Cref{ex:deg1factorthrough}, if we wish to study degree one representations of $G,$ it suffices to study those of the \emph{abelian} group $G' = G/[G, G].$ Now, if $G/[G, G]$ is finite (which is certainly the case if $G$ is finite), then the above corollary tells us that there are exactly $\md{G'}$ many such representations. In fact, \Cref{cor:GconghatG} actually tells us the description of the representations as well. In fact, there's something more that we know which we isolate as a corollary.
\end{ex}

\begin{cor} \label{cor:numdegoneirrepsdivG}
	Let $G$ be a finite group. Then, the number of (distinct, inequivalent) degree one representations of $G$ is $\md{G/[G, G]}.$ In particular, the number divides $\md{G}.$
\end{cor}
\begin{proof} 
	As noted in \Cref{ex:deg1factorthrough}, any degree one representation factors through $G/[G, G]$ and all degree one representations are obtained in precisely this way.

	Thus, the number of degree one representations of $G$ is that of $G/[G, G].$ Since $G/[G, G]$ is abelian, there are exactly $\md{G/[G, G]}$ many such. This number divides $\md{G},$ by elementary group theory.
\end{proof}

\begin{ex}[Standard representation of $S_n$] \label{ex:standardrepSn}
	We define $\varphi:S_n \to \GL_n(\mathbb{C})$ as follows:\\
	Given $\sigma \in S_n,$ we define $\varphi_\sigma$ to be the matrix representing the linear transformation determined by $e_i \mapsto e_{\sigma(i)}.$

	One checks easily that this is indeed a homomorphism. One can verify that the matrix $\varphi_\sigma$ is explicitly given by permuting the columns of the identity matrix according to $\sigma.$ To be more explicit, the $i$-th columns of $\varphi_\sigma$ will be the $\sigma(i)$-th column of the identity matrix. This is because we wish to map $e_i$ to $e_{\sigma(i)}.$ 

	As an example, for $n = 3,$ we have
	\begin{equation*} 
		\varphi_{(123)} = \begin{bmatrix}
			& & 1 \\
			1 & & \\
			 & 1 &
		\end{bmatrix}.
	\end{equation*}
\end{ex}

For the above example, note that
\[\begin{WithArrows}[displaystyle]
	\varphi_\sigma(e_1 + \cdots + e_n) &= \varphi_\sigma(e_1) + \cdots + \varphi_\sigma(e_n)\\
	&= e_{\sigma(1)} + \cdots + e_{\sigma(n)} \Arrow{since $\sigma$ is a permutation}\\
	&= e_{1} + \cdots + e_{n}
\end{WithArrows}\]

Thus, the subspace $\mathbb{C}(e_1 + \cdots + e_n)$ is $\varphi_\sigma$-invariant \underline{for all $\sigma \in S_n$}. This motivates the following definition.

\begin{defn}%[$G$-invariant subspace]
	Let $\varphi : G \to \GL(V)$ be a representation. A subspace $W \le V$ is said to be \deff{$G$-invariant} if, for all $g \in G$ and $w \in W,$ we have $\varphi_g(w) \in W.$
\end{defn}
% manote: does this mean that we can consider \vaprhi:G \to \GL(W) as some representation?

\begin{rem}
	Note that the invariance depends on $G$ as well the representation being considered. To emphasise on the representation at times, we may add ``with respect to $\varphi$.''
\end{rem}

\begin{prop}
	If $W \le V$ is a $G$-invariant subspace with respect to $\varphi:G\to \GL(V),$ then $\varphi|_W: G \to \GL(W)$ by setting $(\varphi|_W)_g(w) = \varphi_g(w)$ for $w \in W$ is a representation.
\end{prop}
\begin{proof} 
	We first show that the $\varphi|_W$ so defined actually maps into $\GL(W).$ By the hypothesis that $W$ is $G$-invariant, we get that
	\begin{equation*} 
		\varphi_g(w) \in W
	\end{equation*}
	for all $w \in W$ and thus, $\varphi_g$ restricts to a function from $W$ to $W,$ for all $g \in G.$\\
	The fact that this is linear follows from the fact that $\varphi$ was a representation to begin with. Moreover, it is invertible since $\varphi_{g^{-1}}$ also restricts from $W$ to $W$ and thus, we have that $\varphi_g \in \GL(W)$ for all $g \in G.$ This shows that $\varphi|_W$ is actually a function from $G$ to $\GL(W).$

	The fact that it is a homomorphism follows from the fact that $\varphi$ was one to begin with.
\end{proof}

\begin{defn}%[Subrepresentation]
	Let $\varphi:G\to \GL(V)$ be a representation. If $W \le V$ is a $G$-invariant subspace, then $\varphi|_W : G \to \GL(W)$ is again a representation and we call $\varphi|_W$ is a \deff{subrepresentation} of $\varphi.$
\end{defn}

At his point, we quickly recall direct sums: If $V$ is a vector space, and $U, W \le V$ are subspaces, then $U + W$ is defined to be the set of all elements of the form $u + w$, where $u \in U$ and $w \in V$. If it is the case that $U \cap W$, then $U + W$ is denoted as $U \oplus W$. This is the \emph{internal} direct sum of $U$ and $W$. The subspace $U \oplus W$ has the property that every element can be written as $u + w$ for a \emph{unique} choice of $u \in U$ and $w \in W$. \newline
On the other hand, if $V_{1}$ and $V_{2}$ are arbitrary vector spaces, then the Cartesian product $V_{1} \times V_{2}$ is a vector space with pointwise operations. This is referred to as the \emph{external} direct sum of $V_{1}$ and $V_{2}$, and again denoted by $V_{1} \oplus V_{2}$. 

The relation between the two is straightforward: If $U$ and $W$ are as in the earlier paragraph, then there is an isomorphism from the internal direct sum to the external direct sum given by $u + w \mapsto (u, w)$. \newline
Conversely, $V_{1}$ and $V_{2}$ can be considered as subspaces of the (external) direct sum via the inclusions $v_{1} \mapsto (v_{1}, 0)$ and $v_{2} \mapsto (0, v_{2})$. Then, under this identification, the external and internal sums coincide.

Going back to \Cref{ex:ZnZGL2Cequiv}, it is easy to note that $\mathbb{C}e_1$ and $\mathbb{C}e_2$ are $\mathbb{Z}/n\mathbb{Z}$-invariant subspaces with respect to $\psi$ (and not $\varphi$!). Moreover, we have that $\mathbb{C}^2 = \mathbb{C}e_1 \oplus \mathbb{C}e_2.$ This motivates the next definition.

\begin{defn}%[Direct sum of representations]
	Let $\varphi^{(1)}:G \to \GL(V_1)$ and $\varphi^{(2)}:G \to \GL(V_2)$ be representations. Then, their (external) \deff{direct sum} is the representation
	\begin{equation*} 
		\varphi^{(1)}\oplus\varphi^{(2)} : G \to \GL(V_1 \oplus V_2)
	\end{equation*}
	given by
	\begin{equation*} 
		\left(\varphi^{(1)}\oplus\varphi^{(2)}\right)_g(v_1, v_2) = \left(\varphi^{(1)}_g(v_1), \varphi^{(2)}_g(v_2)\right)
	\end{equation*}
	for all $g \in G$ and for all $(v_1, v_2) \in V_1 \oplus V_2.$
\end{defn}
Note that in the above, we are using the tuple notation for representing the (external) direct sum of the vector spaces $V_1$ and $V_2.$ This sum can be visualised naturally in terms of matrices.\\
If $\varphi^{(1)}:G\to \GL_n(\mathbb{C})$ and $\varphi^{(2)}:G \to \GL_m(\mathbb{C})$ are representations, then each $\varphi^{(i)}_g$ is a matrix. Then, the matrix $\left(\varphi^{(1)} \oplus \varphi^{(2)}\right)_g \in \GL_{n + m}(\mathbb{C})$ is given as the block matrix
\begin{equation*} 
	\left(\varphi^{(1)} \oplus \varphi^{(2)}\right)_g = \two{\varphi_g^{(1)}}{}{}{\varphi_g^{(2)}}.
\end{equation*}
(The empty places are $0$ matrices of appropriate sizes.)

\begin{ex}
	The two representations from \Cref{ex:ZnZCstardeg1} have their (external) direct sum as $\psi$ from \Cref{ex:ZnZGL2Cequiv}.

	A slightly less direct (but simple) calculation shows that 
	\begin{equation*} 
		\mathbb{C}\col{\iota}{1} \andd \mathbb{C}\col{-\iota}{1}
	\end{equation*}
	are also $\mathbb{Z}/n\mathbb{Z}$-invariant subspaces with respect to $\varphi.$ (Note that are just the columns of $A$ in the example, multiplied by a scalar.)

	Thus, $\varphi$ can also be written as a sum of subrepresentations. This should not be surprising as one would expect that equivalent representations behave similarly in this aspect. This will be made more precise and proven at the end of this section.
\end{ex}

\begin{ex} \label{ex:directsumoftrivialreps}
	The representation $\rho: G \to \GL_n(\mathbb{C})$ given by $\rho_g = I_n$ for all $g \in G$ is equivalent to the direct sum of $n$ copies of the \nameref{ex:trivialrepresentation}. Note that if $n > 1,$ then it is \emph{not} equivalent to the trivial representation since the degrees are different.
\end{ex}

\begin{ex} \label{ex:S3GL2Crho}
	Let $\rho:S_3 \to \GL_2(\mathbb{C})$ be specified on the generators by
	\begin{equation*} 
		\rho_{(12)} = \two{-1}{-1}{0}{1}, \quad \rho_{(123)} = \two{-1}{-1}{1}{0}.
	\end{equation*}
	(It must be checked that this defines a representation. We do this at the end.) 

	Let $\psi:S_3 \to \mathbb{C}^* \cong \GL_1(\mathbb{C})$ be the trivial representation, i.e., $\psi_g = 1.$ Then, we have the representation $\rho \oplus \psi$ which is specified on the generators by
	\begin{equation*} 
		(\rho \oplus \psi)_{(12)} = \begin{bmatrix}
			-1 & 1 & \\
			0 & 1 & \\
			& & 1
		\end{bmatrix}, \quad (\rho \oplus \psi)_{(123)} = \begin{bmatrix}
			-1 & 1 & \\
			1 & 0 & \\
			& & 1
		\end{bmatrix}.
	\end{equation*}

	We shall see later that $\rho\oplus\psi$ is equivalent to the standard representation as considered in \Cref{ex:standardrepSn}. % amnote: simultaneous diagonalisability
	
	\hrulefill
	
	To check that $\rho$ actually gives us a representation (group homomorphism), we must verify that the matrices satisfy the relations that the generators satisfy. That is,
	\begin{equation*} 
		\rho_{(12)}^2 = I_2,\;\rho_{(123)}^3 = I_2,\; \rho_{(12)}\rho_{(123)} = \rho_{(123)}^2\rho_{(12)}.
	\end{equation*}
	(We are using the fact from group theory that the above relations completely determine $S_3.$)

	One can compute and see that the above relations do hold. % amnote: is this the way to do this?
\end{ex}

\begin{prop} \label{prop:Ginvariantdirectsum}
	If $V_1, V_2 \le V$ are $G$-invariant subspaces with respect to $\varphi$ and $V = V_1 \oplus V_2,$ then $\varphi$ is equivalent to the (external) direct sum $\varphi|_{V_1} \oplus \varphi|_{V_2}.$
\end{prop}
\begin{proof} 
	Let $T:V \to V_1 \oplus V_2$ be the natural map $v_1 + v_2 \mapsto (v_1, v_2).$ (Here we are considering the external direct sum of the vector spaces.)\\
	This map is well-defined and an isomorphism because $V$ is the (internal) direct sum of $V_1$ and $V_2.$

	Now, put $\psi = \varphi|_{V_1} \oplus \varphi|_{V_2}.$ Then, for any $g \in G,$ we have
	\[\begin{WithArrows}[displaystyle]
		\psi_g(v_1, v_2) &= \left(\left(\varphi|_{V_1}\right)_g(v_1), \left(\varphi|_{V_2}\right)_g(v_2)\right)\\
		&= \left(\varphi_g(v_1), \varphi_g(v_2)\right) \Arrow{since $V_1, V_2$ are $G$-invariant} \\
		&= T(\varphi_g(v_1) + \varphi_g(v_2)) \Arrow{$\varphi_g$ is a linear map} \\
		&= T(\varphi_g(v_1 + v_2))\\
		&= T(\varphi_g(T^{-1}(v_1, v_2))),
	\end{WithArrows}\]
	showing that
	\begin{equation*} 
		\psi_g = T\varphi_gT^{-1},
	\end{equation*}
	as desired.	
\end{proof}

The above can also be visualised in terms of matrices. Let $B_i$ be a basis for $V_i.$ Then, $B \vcentcolon= B_1 \cup B_2$ is a basis for $V$ (since $V$ is the internal direct sum of the $V_i$). Since $V_i$ is $G$-invariant, we see that $\varphi_g(B_i) \subset \mathbb{C}B_i.$ Thus, the matrix representation with respect to $B$ is as follows:
\begin{equation*} 
	[\varphi_g]_B = \two{\left[\varphi^{(1)}\right]_{B_1}}{}{}{\left[\varphi^{(2)}\right]_{B_2}}.
\end{equation*}

One who has studied algebra would be familiar with the idea of breaking down structures into simpler ``irreducibles'' (similar to the prime factorisation of an integer). To such a reader, the following definition should not come as a surprise.

\begin{defn}%[Irreducible representation] % amnote: non-degree zero?
	A \underline{non-zero} representation $\varphi:G\to \GL(V)$ of a group $G$ is said to be \deff{irreducible} if the only $G$-invariant subspaces of $V$ are $0$ and $V.$
\end{defn}
In the above, $0$ refers to the $0$ \emph{subspace}, i.e., $\{0\}.$

\begin{ex}
	Any degree one representation is irreducible since there is no non-trivial proper subspace of a dimension $1$ vector space.
\end{ex}

\begin{ex}
	If $G = \{1\},$ the trivial group, then the only irreducible representation is a degree one representation. (In other words, the converse of the previous example holds too.)

	Indeed, the only representation $\varphi:G\to \GL(V)$ is $\varphi_1 = I$ and thus, every subspace of $V$ is a $G$-invariant subspace.
\end{ex}

\begin{rem}
	Note that in the above case, we actually have that the representation is actually a direct sum of subrepresentations. However, irreducibility does not demand that. The reader can see this happening in \Cref{ex:Zredbutnotdecomposable}. % manote

	However, we will soon show that the above is actually true when the group is finite.
\end{rem}

\begin{ex}[Irreducible representations of dihedral type groups]
	Let $G$ be a finite group with generators $a$ and $b.$ (By hypothesis, $a$ and $b$ have finite order.) Suppose further that every element of $G$ can be written as $a^ib^j$ for some non-negative integers $i$ and $j.$ (Note that given any $g \in G,$ $g^{-1}$ can be written as $a^ib^j$ and hence, $g = b^{-j}a^{-i}.$ Using that $a$ and $b$ have finite orders, we can actually write every element of $G$ as $b^{j'}a^{i'}$ for non-negative integers as well.)

	By the parenthetical remark, we can assume without loss of generality that $\md{a} \le \md{b}.$ ($\md{g}$ denotes the order of $g \in G.$)

	Let $n \vcentcolon= \md{a}.$ We show that any irreducible representation of $G$ has degree at most $n.$ 

	To this end, let $\varphi : G \to \GL(V)$ be an irreducible representation and let $v$ be an eigenvector of $\varphi_b.$ Consider the following subspace $W$ of $V$ given by
	\begin{equation*} 
		W \vcentcolon= \langle v, \varphi_av, \ldots, \varphi_{a^{n-1}}v\rangle.
	\end{equation*}
	Clearly, $0 < \dim W \le n.$ We show that $W$ is $G$-invariant. Then, since $\varphi$ is irreducible, it would follow that $V = W,$ proving our claim.

	By hypothesis, an arbitrary element of $G$ can be written as $a^ib^j.$ Pick an arbitrary element of the spanning set given above for $W.$ It is of the form $\varphi_{a^k}v$ for some $0 \le k \le n - 1.$ It suffices to show that 
	\begin{equation*} 
		\varphi_{a^ib^j}\left(\varphi_{a^k}v\right) \in W.
	\end{equation*}
	Note that, by hypothesis, $a^ib^ja^k = a^pb^q$ for some non-negative integers $p$ and $q.$ Since $n = \md{a},$ we may assume that $p < n.$ Since $\varphi$ is a homomorphism, we get that
	\begin{equation*} 
		\varphi_{a^ib^j}\left(\varphi_{a^k}v\right) = \varphi_{a^p}\left(\varphi_{b^q}v\right).
	\end{equation*}
	Since $v$ is an eigenvector of $\varphi_b,$ we see that $\varphi_{b^q}v$ is some linear multiple of $v,$ say $\lambda v.$ Then, the right side of the above equation becomes
	\begin{equation*} 
		\varphi_{a^p}\left(\varphi_{b^q}v\right) = \varphi_{a^p}\left(\lambda v\right) = \lambda \varphi_{a^p}(v) \in W,
	\end{equation*}
	as desired.
\end{ex}

\begin{ex}[Irreducible representations of dihedral groups] \label{ex:irredrepDndegbound}
	Consider the dihedral group $D_n$ with $r$ denoting a rotation and $s$ a reflection. Then, the hypothesis of the previous example applies with $a = s$ and hence, $n = 2.$ This tells us that every irreducible representation of $D_n$ has degree at most two.
\end{ex}

\begin{thm} \label{thm:ontogrouphomogivesirredrep}
	Let $\rho : H \to \GL(V)$ be an irreducible representation of $H$ and $\psi : G \to H$ be an onto group homomorphism. Then,
	\begin{equation*} 
		\rho \circ \psi : G \to \GL(V)
	\end{equation*}
	is an irreducible representation of $G.$
\end{thm}
\begin{proof} 
	Let $\varphi \vcentcolon= \rho \circ \psi.$ It is clear that this is a representation, being the composition of group homomorphisms. We now show that it is irreducible.

	Let $W \le V$ be a $G$-invariant subspace (with respect to $\varphi$). We show that $W$ is also $H$-invariant (with respect to $\rho$) and conclude.\footnote{One should also check that it is non-zero but that is clear.}

	This is simple for if $w \in W$ and $h \in H,$ then we get $h = \psi(g)$ for some $g \in G.$ We then note that
	\begin{equation*} 
		\rho_h(w) = \rho(h)(w) = \rho(\psi(g))(w) = (\rho \circ \psi)(g)(w) = \varphi(g)(w) = \varphi_g(w) \in W. \qedhere
	\end{equation*}
\end{proof}

We now see when are degree two representations irreducible.	

\begin{prop} \label{prop:deg2repirreducible}
	If $\varphi:G\to \GL(V)$ is a degree two representation, then $\varphi$ is irreducible if and only if there is no common eigenvector $v$ to all $\varphi_g$ with $g \in G.$
\end{prop}

% amnote: this will generalise to degree 3 reps also once we know a complementary subspace exists

\begin{proof} 
	One direction is easy. Suppose that $v \in V$ is such that $v$ is an eigenvector of $\varphi_g$ for all $g \in G.$ In that case, $\mathbb{C}v$ is a one dimensional $G$-invariant subspace of $V$ and hence, is proper and non-trivial. (Recall that eigenvectors are non-zero, by definition.)

	Now, suppose the converse. Let $W$ be a proper non-trivial $G$-invariant subspace of $V.$ Then, $W = \mathbb{C}v$ for some $0 \neq v \in V.$ Then, given any $g \in G,$ we have that
	\begin{equation*} 
		\varphi_gv \in W
	\end{equation*}
	and hence, $\varphi_gv = \lambda_g v$ for some $\lambda_g \in \mathbb{C}.$ This shows that $v$ is an eigenvector for all $\varphi_g.$ (Since it was non-zero to begin with.)
\end{proof}

\begin{rem}
	For finite groups, the above proposition can also be generalised to degree three representations, using an almost identical proof. The only extra ingredient required is that if a representation of a finite group is reducible, then we can actually write $V = W \oplus W'$ for non-zero $G$-invariant subspaces.

	For infinite groups, the above proposition does not generalise to degree three representation, as seen in \Cref{ex:deg3repredbutnocommoneigen}.

	The above does not generalise to degree four representations, even in the case of finite groups. This is seen in \Cref{ex:deg4repredbutnocommoneigen}.
\end{rem}

\begin{ex} \label{ex:showingS3GL2Crhoisirred}
	The representation $\rho:S_3 \to \GL_2(\mathbb{C})$ in \Cref{ex:S3GL2Crho} is irreducible. 

	We show this by showing that no eigenvector of $\rho_{(12)}$ is also an eigenvector of $\rho_{(123)}.$ (That is, they have no common eigenvectors.) Then, we are done, by the above proposition.

	To this end, we first compute the eigenvalues of $\rho_{(12)}$ to be $\pm 1.$ Corresponding to these, we get the eigenvectors $\col{1}{0}$ and $\col{-1}{2}.$ (Note that since the eigenvalues are distinct, any other eigenvector must be a scalar multiple (as opposed to a linear combination) of either of these.)

	A direct computation gives us that neither is an eigenvector of $\rho_{(123)}.$ Indeed, we have
	\begin{equation*} 
		\rho_{(123)}\col{1}{0} = \col{-1}{1} \andd \rho_{(123)}\col{-1}{2} = \col{-1}{-1}.
	\end{equation*}
\end{ex}

\begin{ex}[An irreducible representation of $D_4$] \label{ex:D4irreddeg2}
	Consider the group $D_4.$ Let $r$ be rotation by $\pi/2$ and $s$ be a reflection about a perpendicular bisector of a side. We know that
	\begin{equation*} 
		D_4 = \langle r, s \mid r^4 = s^2 = rsrs^{-1} = 1\rangle.
	\end{equation*}
	Using the above, we see that the following is a representation:
	\begin{equation*} 
		\varphi(r) \vcentcolon= \two{\iota}{}{}{-\iota} \andd \varphi(s) \vcentcolon= \two{}{1}{1}{}.
	\end{equation*}
	Clearly, the only eigenvectors of $\varphi(r)$ (up to scaling) are $e_1$ and $e_2,$ neither of which is an eigenvector of $\varphi(s).$ Thus, $\varphi$ is irreducible.
\end{ex}

\begin{ex} \label{ex:deg3repredbutnocommoneigen}
	We now show that \Cref{prop:deg2repirreducible} is not true for degree three representations when the group is infinite. \\
	Let $G \vcentcolon= F(a, b)$ be the free group on two generators $a$ and $b.$ (Recall that a homomorphism from $G$ to any group is defined uniquely by specifying its values on $a$ and $b.$)

	Consider the representation $\varphi : G \to \GL_3(\mathbb{C})$ defined by
	\begin{equation*} 
		\varphi_a \vcentcolon= \begin{bmatrix}
			1 &   &  \\
			  & 2 &  \\
			  &   & 3
		\end{bmatrix} \andd \varphi_b \vcentcolon= \begin{bmatrix}
			  & 1 & 1\\
			1 &   & 1\\
			  &   & 1
		\end{bmatrix}.
	\end{equation*} 
	Note that $\varphi_a$ and $\varphi_b$ are indeed elements of $\GL_3(\mathbb{C})$ as can be checked by noting that they both have nonzero determinant. Thus, the above defines a representation.

	\textbf{Claim 1.} $\varphi_a$ and $\varphi_b$ have no common eigenvector. In particular, there is no $v \in \mathbb{C}^3$ which is a common eigenvector for all $\left\{\varphi_g\right\}_{g \in G}.$

	\begin{proof} 
		This is simple for the only eigenvectors of $\varphi_a$ (up to scaling) are $e_1,$ $e_2,$ and $e_3.$ Clearly, none of them is an eigenvector of $\varphi_b.$
	\end{proof}

	\textbf{Claim 2.} $W = \mathbb{C}\{e_1, e_2\}$ is a $G$-invariant subspace.
	\begin{proof} 
		Clearly, $W$ is $\varphi_a$ and $\varphi_b$-invariant. By \Cref{prop:Tinverseinvariance}, it follows that it is also $\varphi_{a^{-1}} = \left(\varphi_a\right)^{-1}$ and $\varphi_{b^{-1}}$ invariant.

		Since any element $g \in G$ is a product of positive powers of $a, b, a^{-1}, b^{-1},$ it follows that $W$ is $\varphi_g$-invariant, by \Cref{prop:STinvariance}.
	\end{proof}

	Thus, we have an example of a degree three representation which is reducible but there's no common eigenvector.
\end{ex}

Similar to irreducible representations, we define some terms which the reader should find natural.

\begin{defn}%[Completely reducible]
	Let $G$ be a group. A representation $\varphi:G\to \GL(V)$ is said to \deff{completely reducible} if $V = V_1 \oplus \cdots \oplus V_n$ where $V_i$ is $G$-invariant and $\varphi|_{V_i}$ irreducible for each $i = 1, \ldots, n.$
\end{defn}

\begin{rem}
	In view of \Cref{prop:Ginvariantdirectsum}, $\varphi$ is completely reducible is equivalent to saying that $\varphi\sim\varphi^{(1)}\oplus\cdots\oplus\varphi^{(n)}$ for some irreducible representations $\varphi^{(i)}.$
\end{rem}

\begin{rem}
	As funny as it may seem, an irreducible representation \emph{is} completely reducible. We did not demand for the $V_i$s to be proper subspaces of $V.$
\end{rem}

The above is similar to a sort of prime factorisation or diagonalisation. Our eventual goal is to show that any representation of a finite group is completely reducible. Thus, one can then just study irreducible representations.

\begin{defn}%[Decomposable representation]
	A \underline{non-zero} representation $\varphi$ is said to be \deff{decomposable} if $V = V_1 \oplus V_2$ for some \underline{non-zero} $G$-invariant subspaces $V_1, V_2 \le V.$ Otherwise, $V$ is said to be \deff{indecomposable}.
\end{defn}

Note that the above is, a priori, stronger than saying that $\varphi$ is irreducible. However, we shall see later that the two coincide for when $G$ is finite.

We now wish to show irreducible, completely reducible, and decomposability are actually notions that depend on the equivalence class of the representation. To this end, we first prove the following lemma.

\begin{lem} \label{lem:isopreservesinv}
	Let $\varphi:G \to \GL(V)$ and $\psi:G \to \GL(W)$ be equivalent representations and let $T: V \to W$ be an isomorphism such that the desired diagram commutes. If $V_1 \le V$ is $G$-invariant, then so is $W_1 \vcentcolon= T(V_1) \le W.$
\end{lem}
\begin{proof} 
	Let $w \in W_1$ and let $g \in G.$ Then, we have
	\begin{equation*} 
		\psi_g = T\varphi_gT^{-1}.
	\end{equation*}
	Note that $T^{-1}w \in V_1$ and thus, $\varphi_gT^{-1}w \in V_1$ since $V_1$ is $T$-invariant. In turn, we get that
	\begin{equation*} 
		\psi_gw = T\varphi_gT^{-1} \in T(V_1) = W_1,
	\end{equation*}
	as desired.
\end{proof}

For the following three propositions, let $\varphi:G \to \GL(V)$ and $\psi:G \to \GL(W)$ be equivalent representations and let $T: V \to W$ be an isomorphism such that the desired diagram commutes.

\begin{prop} \label{prop:irreducibleequiv}
	$\psi$ is irreducible if $\varphi$ is so.
\end{prop}
\begin{proof} 
	Let $V_1 \le V$ be a $G$-invariant subspace which is non-zero and proper. Then, $W_1 \vcentcolon= T(V_1)$ is non-zero and proper since $T$ is an isomorphism. By \Cref{lem:isopreservesinv}, this is also $G$-invariant and we are done.
\end{proof}

\begin{prop} \label{prop:decomposableequiv}
	$\psi$ is decomposable if $\varphi$ is so.
\end{prop}
\begin{proof} 
	If $V = V_1 \oplus V_2$ for non-zero subspaces, then $W = T(V_1) \oplus T(V_2)$ (with $T(V_1) \neq 0 \neq T(V_2)$) since $T$ is an isomorphism. If $V_1, V_2$ are $G$-invariant, then so are $T(V_1)$ and $T(V_2),$ by \Cref{lem:isopreservesinv}.
\end{proof}

\begin{prop} \label{prop:compreducibleequiv}
	$\psi$ is completely reducible if $\varphi$ is so.
\end{prop}
\begin{proof} 
	By a similar argument as earlier, we see that if
	\begin{equation*} 
		V = V_1 \oplus \cdots \oplus V_n,
	\end{equation*}
	then
	\begin{equation*} 
		W = W_1 \oplus \cdots \oplus W_n,
	\end{equation*}
	where $W_i \vcentcolon= T(V_i)$ and each subspace on the right is $G$-invariant.

	We now wish to show that if $\varphi|_{V_i}$ is irreducible, then $\psi|_{W_i}$ is too. This is simple because we note that the following diagram commutes for all $g \in G:$
	\begin{center}
		\begin{tikzcd}
			{V_i} \arrow[rr, "\varphi_g|_{V_i}"]\arrow[dd, "T|_{V_i}"'] & & {V_i}\arrow[dd, "T|_{V_i}"]\\
			& & \\
			{W_i} \arrow[rr, "\psi_g|_{W_i}"'] & & {W_i}
		\end{tikzcd}
	\end{center}
	and thus, $\varphi|_{V_i} \sim \psi|_{W_i}.$ (Note that $T|_{V_i}$ is indeed an isomorphism.) Thus, by \Cref{prop:irreducibleequiv}, we are done
\end{proof}

\begin{thm}[Irreducible representations of finite cyclic groups] \label{thm:irredcyclicgroup}
	Let $G$ be a finite cyclic group. All irreducible representations of $G$ are of degree one.
\end{thm}
\begin{proof} 
	Without loss of generality, we may assume $G = \mathbb{Z}/n\mathbb{Z}.$ Suppose that $\varphi:G \to \GL_m(\mathbb{C})$ is a representation with $m \ge 2.$ We show that it is reducible. 

	Note that $\varphi_{[1]}^n = I.$ Thus, the minimal polynomial of $\varphi_{[1]}$ is a factor of $X^n - 1$ and hence, has distinct roots. This shows that $\varphi_{[1]}$ is diagonalisable. (\Cref{thm:splitdistinctdiagonalise})

	Let $T \in \GL_m(\mathbb{C})$ be such that
	\begin{equation*} 
		T\varphi_{[1]}T^{-1} = D
	\end{equation*}
	for some diagonal matrix $D.$ Note that raising both sides to the power $k$ yields
	\begin{equation*} 
		T\varphi_{[1]}^kT^{-1} = D^k
	\end{equation*}
	or
	\begin{equation*} 
		T\varphi_{[k]}T^{-1} = D^k.
	\end{equation*}
	In other words, the equivalent representation $\psi:G \to \GL_m(\mathbb{C})$ given by $\psi_{[k]} = T\varphi_{[k]}T^{-1}$ has the property that $\psi_{[k]}$ is diagonal for all $[k] \in G.$

	This shows that $\psi$ can be decomposed as $m$ non-zero proper sub-representations, proving reducibility. As a consequence, $\varphi$ is reducible.
\end{proof}

In the above, we used the fact from Linear Algebra that if the minimal polynomial of a matrix has distinct roots, then it is diagonalisable. In the next section, we shall prove the above theorem again without the fact and in turn, get the above fact as a corollary. (Note that there is no circular reasoning.)

\subsection{Maschke's Theorem and Complete Reducibility}
We recall the following definitions from linear algebra.

\begin{defn}%[Unitary representation]
	Let $V$ be an inner product space. A representation $\varphi:G \to \GL(V)$ is said to be \deff{unitary} if $\varphi_g \in U(V)$ for all $g \in G.$
\end{defn}
In other words, we can view $\varphi$ as a map $\varphi:G \to U(V).$ In yet other words, we have
\begin{equation*} 
	\langle \varphi_gv, \varphi_gw\rangle = \langle v, w\rangle
\end{equation*}
for all $g \in G$ and all $v, w \in V.$

\begin{defn}[Unit circle]
	We define $S^1 = \{z \in \mathbb{C} \mid \md{z} = 1\}.$
\end{defn}

Identifying $\GL_1(\mathbb{C})$ with $\mathbb{C}^*,$ we see that $U_1(\mathbb{C})$ is identified with $S^1.$ Hence, a degree-one unitary representation is a homomorphism $\varphi:G \to S^1.$

\begin{rem} \label{rem:findegoneunitary}
	As noted in \Cref{rem:finrepintoS1}, degree one representations of finite groups actually map into $S^1.$ Thus, they are all unitary.
\end{rem}

\begin{ex}
	$\varphi:\mathbb{R}\to S^1$ given by $t \mapsto \exp(2\pi\iota t)$ is a degree one-unitary representation of the additive group $(\mathbb{R}, +)$ since $\varphi(s + t) = \varphi(s)\varphi(t).$
\end{ex}

As we had noted earlier, decomposability was a stronger statement than reducibility. Now, we show that the two coincide for unitary representations.

\begin{prop} \label{prop:unitirredordecom}
	Let $\varphi:G \to \GL(V)$ be a unitary representation. Then, $\varphi$ is either irreducible or decomposable.
\end{prop}
\begin{proof} 
	Suppose that $\varphi$ is not irreducible. Then, there exists a non-zero proper subspace $W \le V$ which is $G$-invariant. Then, we have $V = W \oplus W^{\perp}$ and $W^{\perp}$ is non-zero proper. Thus, it now suffices to show that $W^\perp$ is $G$-invariant.

	Now, given any $g \in G,$ we know that $\varphi_g$ is unitary and $W$ is $\varphi_g$-invariant. Thus, by \Cref{cor:unitaryinvariance}, we see that $W^\perp$ is $\varphi_g$-invariant. Since this is true for all $g \in G,$ we see that $W^\perp$ is $G$-invariant, as desired.
\end{proof}

Now, we show that for finite groups, every representation is equivalent to a unitary representation and thus, conclude that decomposable and reducible are equivalent for finite groups. To make the final proof simpler, we first state two lemmata.

\begin{lem} \label{lem:newinnerproduct}
	Let $G$ be a \underline{finite group} and $\rho:G \to \GL_n(\mathbb{C})$ be a representation. Let $\langle \cdot, \cdot\rangle$ denote the standard inner product on $\mathbb{C}^n.$ Define the new product $(\cdot, \cdot)$ on $\mathbb{C}^n$ as
	\begin{equation*} 
		(v, w) \vcentcolon= \sum_{g \in G} \langle \rho_gv, \rho_gw\rangle.
	\end{equation*}
	Then, $(\cdot, \cdot)$ is an inner product.
\end{lem}
Note that the finiteness of $G$ tells us that the above sum is well-defined. (Of course, along with the fact that addition is commutative.)
\begin{proof} 
	Let $c_1, c_2 \in \mathbb{C}$ and $v_1, v_2, w, w \in \mathbb{C}^n$ be arbitrary.

	First, note
	\begin{align*} 
		(c_1v_1 + c_2v_2, w) &= \sum_{g \in G} \langle \rho_g(c_1v_1 + c_2v_2), \rho_gw\rangle\\
		&= \sum_{g \in G} \langle c_1\rho_gv_1 + c_2\rho_gv_2, \rho_gw\rangle\\
		&= \sum_{g \in G} \left[c_1\langle \rho_gv_1, \rho_gw\rangle + c_2\langle \rho_gv_2, \rho_gw\rangle\right]\\
		&= c_1\sum_{g \in G} \langle \rho_gv_1, \rho_gw\rangle + c_2\sum_{g \in G} \langle \rho_gv_2, \rho_gw\rangle\\
		&= c_1(v_1, w) + c_2(v_2, w).
	\end{align*}
	Next, 
	\begin{align*} 
		(w, v) &= \sum_{g \in G} \langle \rho_gw, \rho_gv\rangle\\
		&= \sum_{g \in G} \overline{\langle \rho_gv, \rho_gw\rangle}\\
		&= \overline{\sum_{g \in G} \langle \rho_gv, \rho_gw\rangle}\\
		&= \overline{(v, w)}.
	\end{align*}
	Lastly,
	\begin{equation*} 
		(v, v) = \sum_{g \in G} \langle \rho_gv, \rho_gv\rangle \ge 0
	\end{equation*}
	since each term is non-negative and hence,
	\begin{equation*} 
		(v, v) = 0 \implies \langle \rho_gv, \rho_gv\rangle = 0 \text{ for all } g \in G
	\end{equation*}
	and thus, $\rho_gv = 0$ for all $g \in G$ since $\langle \cdot, \cdot\rangle$ is an inner-product.\\
	In particular, $v = \rho_1v = 0,$ as desired.
\end{proof}

\begin{lem} \label{lem:unitarywrtspecial}
	With the same notation as in \Cref{lem:newinnerproduct}, we have that $\rho$ is unitary with respect to the inner product $(\cdot, \cdot).$
\end{lem}
\begin{proof} 
	Let $v, w \in V$ and $g \in G.$ Then,
	\begin{align*} 
		(\rho_gv, \rho_gw) &= \sum_{g' \in G} \langle \rho_{g'}\rho_gv, \rho_{g'}\rho_gw\rangle\\
		&= \sum_{g' \in G} \langle \rho_{g'g}v, \rho_{g'g}w\rangle.
	\end{align*}
	Note that $g' \mapsto g'g$ is a bijection and thus, the above is simplified as
	\begin{equation*} 
		(\rho_gv, \rho_gw) = \sum_{h \in G} \langle \rho_hv, \rho_hw\rangle = (v, w),
	\end{equation*}
	as desired.
\end{proof}

Note that in the previous two lemmata, we worked in $\GL_n(\mathbb{C})$ and used the standard inner product on $\mathbb{C}^n.$ However, this was just for the sake of concreteness. Instead of which, we could've worked with any inner product space $(V, \langle \cdot, \cdot\rangle).$

\begin{prop} \label{prop:repoffingroupisunitary}
	Every representation of a \underline{finite group} $G$ is equivalent to a unitary representation.
\end{prop}
\begin{proof} 
	Let $\varphi:G\to \GL(V)$ be a representation and let $n \vcentcolon= \dim V.$ Fix an isomorphism $T : V \to \mathbb{C}^n$ and put $\rho_g \vcentcolon= T\varphi_g T^{-1}$ for all $g \in G.$ This defines a representation $\rho: G \to \GL_n(\mathbb{C})$ which is equivalent to $\varphi.$ We now show that $\rho$ is unitary.

	Let $(\cdot, \cdot)$ be the inner product as in \Cref{lem:newinnerproduct}. Then, by \Cref{lem:unitarywrtspecial}, we know that $\rho$ is a unitary representation and we are done.
\end{proof}

We now state the corollary alluded all along.

\begin{cor} \label{cor:fingroupirrordec}
	Let $\varphi:G\to \GL(V)$ be a non-zero representation of a \underline{finite group}. Then, $\varphi$ is either irreducible or decomposable.
\end{cor}
\begin{proof} 
	By \Cref{prop:repoffingroupisunitary}, $\varphi \sim \rho$ for some unitary representation $\rho.$ By \Cref{prop:unitirredordecom}, $\rho$ is either irreducible or decomposable. By \Crefrange{prop:irreducibleequiv}{prop:decomposableequiv}, we see that the same is true for $\varphi$ as well.
\end{proof}
\begin{rem}
	For any group, we obviously have that decomposable $\implies$ reducible. The above says that the converse is true for finite groups.

	What the above says that if we have a $G$-invariant subspace $W,$ then we can actually decompose $V$ as $W_1 \oplus W_2$ (with them having the usual properties). In fact, our proof of \Cref{prop:unitirredordecom} actually shows that we can take $W_1 = W$ and $W_2$ is then the orthogonal subspace (after suitably finding an isomorphism which transports the inner product structure).
\end{rem}

With the above remark in mind, we rewrite the previous corollary as follows.

\begin{cor} \label{cor:existenceofcomplimentaryGinvarsubs}
	Let $\varphi:G\to \GL(V)$ be a non-zero representation of a \underline{finite group}. Suppose that $W$ is a non-zero proper $G$-invariant subspace of $V.$ Then, we can write
	\begin{equation*} 
		V = W \oplus W'
	\end{equation*}
	for a $G$-invariant subspace $W'.$ (It follows that $W'$ is also non-zero and proper.)
\end{cor}

With the above, we can strengthen \Cref{prop:deg2repirreducible} to degree three representations as well when $G$ is finite.
\begin{prop} \label{prop:deg3repirreducible}
	If $\varphi:G\to \GL(V)$ is a degree three representation, then $\varphi$ is reducible if and only if there is a common eigenvector $v$ to all $\varphi_g$ with $g \in G.$
\end{prop}
\begin{proof} 
	As before, $\impliedby$ is obvious. (That is true for all groups and all non-zero degree representation, in fact.)

	We show the other direction. Suppose that $\varphi$ is reducible. Then, by \Cref{cor:fingroupirrordec}, $\varphi$ is decomposable and we can write
	\begin{equation*} 
		V = W \oplus W'
	\end{equation*}
	for non-zero $G$-invariant subspaces $W$ and $W'.$ By looking at dimensions, we see that one of $W$ or $W'$ is one-dimensional. Thus, mimicking the proof of \Cref{prop:deg2repirreducible} shows that there is a common eigenvector.
\end{proof}
One can observe that the above proof is similar to how shows that if a three degree polynomial is reducible, then it has a root. However, we really did need the finiteness of $G$ as the following example shows us.

\begin{ex} \label{ex:Zredbutnotdecomposable}
	Let $\varphi:\mathbb{Z} \to \GL_2(\mathbb{C})$ be the representation
	\begin{equation*} 
		\varphi(n) = \two{1}{n}{}{1}.
	\end{equation*}
	Then, $\varphi$ is reducible since $\mathbb{C}e_1$ is a $\mathbb{Z}$-invariant subspace. However, one sees that there is no other common eigenvector to all $\varphi(n)$ and hence, there is no other $\mathbb{Z}$-invariant subspace. Thus, $\varphi$ is not decomposable.

	That is, $\varphi$ is neither irreducible nor decomposable, showing that \Cref{cor:fingroupirrordec} is false for infinite groups. In turn, \Cref{prop:repoffingroupisunitary} is also false for infinite groups. (Note that \Cref{prop:unitirredordecom} had no  assumption of finiteness of group.)
\end{ex}

Moreover, the previous cannot be strengthened to degree four representations (even for finite groups) as the next example shows us.

\begin{ex} \label{ex:deg4repredbutnocommoneigen}
	Let $\varphi:D_4 \to \GL_2(\mathbb{C})$ be the representation as in \Cref{ex:D4irreddeg2}. Put $\psi \vcentcolon= \varphi \oplus \varphi.$ Then, $\psi: G \to \GL_4(\mathbb{C})$ is a degree four representation and we have
	\begin{equation*} 
		\psi(r) = \begin{bmatrix}
			\iota & & & \\
			 & -\iota & & \\
			 & & \iota & \\
			 & & & -\iota \\
		\end{bmatrix} \andd \psi(s) = \begin{bmatrix}
			 & 1 & & \\
			 1 & & & \\
			 & & & 1 \\
			 & & 1 & \\
		\end{bmatrix}.
	\end{equation*}
	Clearly, the eigenvectors of $\psi(r)$ are the standard basis vectors (up to scaling) and none of them is an eigenvector of $\psi(s).$

	Thus, $\psi$ is reducible even though there is no $v \in V$ which is a common eigenvector for all $\psi_g.$
\end{ex}

The next result is again something we had prompted earlier. It is similar to the existence (but not uniqueness yet) to the decomposition of integers into their prime factors.

\begin{thm}[Maschke] \label{thm:maschke}
	Every representation of a \underline{finite} group is completely reducible.
\end{thm}
\begin{proof} 
	We prove this by induction on the degree of the representation. Let $\varphi : G \to \GL(V)$ be a representation.

	If $\dim V = 1,$ then $\varphi$ is irreducible (and hence, completely reducible) and we are done.

	Now, let $n \ge 1$ and assume the statement is degree of representation of degree $\le n.$ Let $\dim V = n + 1.$ If $\varphi$ is irreducible, then we are done. If not, then
	\begin{equation*} 
		V = U \oplus W
	\end{equation*}
	for non-zero $G$-invariant subspaces, by \Cref{cor:fingroupirrordec}. Since $U, W$ both have dimension strictly less than $\dim V,$ the induction hypothesis applies and we can write
	\begin{align*} 
		U &= U_1 \oplus \cdots \oplus U_n\\
		W &= W_1 \oplus \cdots \oplus W_m
	\end{align*}
	for $G$-invariant subspaces such that $\varphi|_{U_i}$ and $\varphi|_{W_j}$ is irreducible for all $1 \le i \le n$ and $1 \le j \le m.$ In turn, we have
	\begin{equation*} 
		V = U_1 \oplus \cdots \oplus U_n \oplus W_1 \oplus \cdots \oplus W_m,
	\end{equation*}
	as desired.
\end{proof}