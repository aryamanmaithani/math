\section{Fourier Analysis on Finite Groups}
\subsection{Periodic Functions on Cyclic Groups}

\begin{defn}
    Let $n \in \mathbb{Z} \setminus \{0\}.$ A function $f : \mathbb{Z} \to \mathbb{C}$ is said to be \deff{periodic} with \deff{period} $n$ if $f(x + n) = f(x)$ for all $x \in \mathbb{Z}.$
\end{defn}

Note that there is no unique period associated to a periodic function. Indeed, if $n$ is a period, then so is any multiple of $n.$

\begin{rem}
    Fix an integer $n > 0.$ The set of functions periodic with period $n$ is in bijection with the set of all complex valued functions on $\mathbb{Z}/n\mathbb{Z}.$
    
    Indeed, given a period function $f : \mathbb{Z} \to \mathbb{C}$ with period $n,$ we get a function $\tilde{f} : \mathbb{Z}/n\mathbb{Z} \to \mathbb{C}$ defined as $\tilde{f}([m]) = f(m).$ (This is well-defined by assumption of periodicity.)
    
    Conversely, given a function $g : \mathbb{Z}/n\mathbb{Z} \to \mathbb{C},$ we get a function $\dot{g} : \mathbb{Z} \to \mathbb{C}$ defined as $\dot{g}(m) = g([m]).$ Clearly, $g$ is periodic with period $n.$
    
    Moreover, the correspondences $f \mapsto \tilde{f}$ and $g \mapsto \dot{g}$ are inverses. Thus, the set of functions periodic with period $n$ is in bijection with $L(\mathbb{Z}/n\mathbb{Z}).$ (Recall that $L(G)$ was the \hyperref[defn:groupalg]{group algebra} of the group $G.$)
\end{rem}

\begin{rem}
    Recall that \Cref{cor:orthnormalbasis} gave us an orthonormal basis of $L(G).$ Since the irreducible representations of $\mathbb{Z}/n\mathbb{Z}$ are all of degree one, we see that the set is simply $\{\chi_k : 0 \le k < n\},$ where $\chi_k([m]) = w_n^{mk}.$
    
    Thus, for $f \in L(\mathbb{Z}/n\mathbb{Z})$ we get 
    \begin{equation*}
        f = \langle f, \chi_0\rangle \chi_0 + \cdots + \langle f, \chi_{n - 1}\rangle \chi_{n - 1}.
    \end{equation*}
\end{rem}

\begin{defn} \label{defn:fourierZnZ}
    Let $f : \mathbb{Z}/n\mathbb{Z} \to \mathbb{C}.$ The \deff{Fourier transform} $\mathcal{F}(f) = \widehat{f} : \mathbb{Z}/n\mathbb{Z} \to \mathbb{C}$ of $f$ is defined as
    \begin{equation*}
        \widehat{f}([m]) = \sum_{k = 0}^{n - 1} f([k])e^{-2\pi\iota mk/n} = \sum_{k = 0}^{n - 1} f([k])\omega_n^{-mk} = n\langle f, \chi_m\rangle.
    \end{equation*}
\end{defn}

The last equality follows by the definition of $\langle \cdot , \cdot\rangle$ and the fact that $n \md{\mathbb{Z}/n\mathbb{Z}}.$ Moreover, note that $\mathcal{F} : L(\mathbb{Z}/n\mathbb{Z}) \to L(\mathbb{Z}/n\mathbb{Z})$ is linear since
$\langle \cdot , \cdot\rangle$ is linear in the first variable.

\begin{prop}
    The Fourier transform is invertible. More precisely, 
    \begin{equation*}
        f = \frac{1}{n}\sum_{k = 0}^{n - 1} \widehat{f}([k])\chi_k.
    \end{equation*}
\end{prop}
\begin{proof}
    We simply compute $\langle f, \chi_k\rangle$ and show that it equals $\widehat{f}([k])/n.$
    
    To this end, note that
    \begin{align*}
        \langle f, \chi_k\rangle &= \frac{1}{n} \sum_{j = 0}^{n - 1}f([j])\overline{\chi_k(j)} \\
        &= \frac{1}{n} \sum_{j = 0}^{n - 1}f([j])e^{-2\pi\iota kj/n} \\
        &= \frac{1}{n} \widehat{f}([k]). \qedhere
    \end{align*}
\end{proof}

\subsection{The Convolution Product}

\begin{defn}
    Let $G$ be a group.
    Given $g \in G,$ we define $\delta_g : G \to \mathbb{C}$ as
    \begin{equation*}
        \delta_g(x) \vcentcolon=
        \begin{cases}
            1 & x = g, \\
            0 & x \neq g.
        \end{cases}
    \end{equation*}
\end{defn}

\begin{defn}
    Let $G$ be a finite group and $a, b \in L(G).$ Then, $a * b \in L(G)$ is defined as
    \begin{equation*}
        (a * b)(x) = \sum_{y \in G} a(xy^{-1})b(y)
    \end{equation*}
    and is called the \deff{convolution} of $a$ with $b.$
\end{defn}

The above is well-defined since $G$ is finite and the sum is taken in $\mathbb{C},$ which is commutative. 

\begin{rem} \label{rem:classfunctionincenter}
    We have
    \begin{equation*}
        (a * b)(x) = \sum_{y \in G} a(xy^{-1})b(y).
    \end{equation*}
    The change of variable $y \mapsto xz^{-1}$ gives
    \begin{equation*}
        (a * b)(x) = \sum_{z \in G} a(xzx^{-1})b(xz^{-1}) = \sum_{z \in G} b(xz^{-1})a(xzx^{-1}).
    \end{equation*}
    Thus, if $a$ is a class function, then we get
    \begin{equation*}
        (a * b)(x) = \sum_{z \in G} b(xz^{-1})a(z) = (b * a)(x)
    \end{equation*}
    for all $x \in G$ and hence, $a * b = b * a.$
    
    In particular, if $G$ is abelian, then $a * b = b * a$ for all $a, b \in L(G).$
    
    However, for a general group, it is not necessary that $*$ is commutative. In fact, after the next proposition, it will be clear that $*$ is commutative iff $G$ is abelian.
\end{rem}

\begin{prop} \label{prop:convdeltaprod}
    Let $G$ be a finite group and $g, h \in G.$ Then, $\delta_g * \delta_h = \delta_{gh}.$
\end{prop}
\begin{proof}
    Let $x \in G.$ Then,
    \begin{equation*}
        (\delta_g * \delta_h)(x) = \sum_{y \in G} \delta_g(xy^{-1})\delta_h(y).
    \end{equation*}
    Thus, $(\delta_g * \delta_h)(x) = 1$ iff $h = y$ and $xy^{-1} = g$ iff $x = gh$ and $0$ otherwise.
    
    In other words, $(\delta_g * \delta_h) = \delta_{gh}.$
\end{proof}

Thus, if $G$ is a non-abelian group, then choose $g, h \in G$ such that $gh \neq hg.$ Then, we get $\delta_g * \delta_h \neq \delta_h * \delta_g.$ 

\begin{prop} \label{prop:convole}
    Let $a \in L(G)$ and $h, g \in G.$ Then, $(a * \delta_h)(g) = a(gh^{-1}).$
\end{prop}

There are many properties that one can verify about the convolution product. We list them below and leave the proof as a computational exercise.

\begin{prop}
    Let $G$ be a finite group. Then,
    \begin{enumerate}
        \item $a * \delta_1 = a = \delta_1 * a$ for all $a \in L(G),$
        \item $a * (b * c) = (a * b) * c$ for all $a, b, c \in L(G),$
        \item $a * (b + c) = (a * b) + (a * c)$ for all $a, b, c \in L(G).$
    \end{enumerate}
    In other words, $(L(G), +, *)$ is a ring with $\delta_1$ as (multiplicative) identity.
\end{prop}

As noted earlier, $L(G)$ is a commutative ring iff $G$ is commutative. The above also justifies why we used the term ``group algebra''. Soon, the usage of $Z(L(G))$ for the set of class functions will become apparent too.

\begin{rem}
    Note that since $\delta_1$ is the identity, \Cref{prop:convdeltaprod} tells us the map $i : G \to L(G)$ defined as $g \mapsto \delta_g$ is a group homomorphism into the group of units $(L(G))^\times.$ (Recall \Cref{lem:determininggrouphomoring}.)
\end{rem}

\begin{defn}
    Let $R$ be a ring. The \deff{center} of $R$ is denoted $Z(R)$ and defined as
    \begin{equation*}
        Z(R) \vcentcolon= \{r \in R : rs = sr \text{ for all } s \in R\}.
    \end{equation*}
    That is, it is the set of all those elements which commute with every element of $R.$
\end{defn}

\begin{rem}
    One can show that the above is actually a \emph{subring} of $R.$ Moreover, it is a commutative ring. However, we do not use this fact.
\end{rem}

Recall that we had already used the notation $Z(L(G))$ to denote the set of class functions of $G.$ We now show that it is indeed the center and thus, the notation is unambiguous.

\begin{prop}
    Let $G$ be a finite group. Then, $a : G \to \mathbb{C}$ is a class function if and only if $a$ is in the center of $L(G).$
\end{prop}
\begin{proof}
    $(\implies)$ Suppose $a : G \to \mathbb{C}$ is a class function. We had observed in \Cref{rem:classfunctionincenter} that $a * b = b * a$ for all $b \in L(G)$ and hence, $a$ is in the center.
    
    $(\impliedby)$ Let $a$ be in the center of $L(G)$ and let $g, h \in G$ be arbitrary. Then,
    \begin{align*}
        a(gh) = \sum_{y \in G} a(gy^{-1})\delta_{h^{-1}}(y) &= (a * \delta_{h^{-1}})(g) \\
        &= (\delta_{h^{-1}} * a)(g) = \sum_{y \in G}\delta_{h^{-1}}(gy^{-1})a(y) = a(hg).
    \end{align*}
    Thus, given any $x, y \in G,$ setting $g = xy$ and $h = x^{-1}$ gives $a(xyx^{-1}) = a(x),$ showing that $a$ is a class function.
\end{proof}

\subsection{Fourier Analysis on Finite Abelian Groups}

Recall that we had defined the \hyperref[defn:dualgroup]{dual of a group}, the group of all homomorphisms $G \to \mathbb{C}^\times$ with point-wise multiplication as operation. In the case that $G$ is finite and abelian, we see that the elements of $\widehat{G}$ are precisely the irreducible characters of $G$ and that $G \cong \widehat{G}.$ (The last was \Cref{cor:GconghatG}.) \\
Moreover, note that $L(G)$ is abelian. 

Recall that we had defined the Fourier transform $\mathcal{F} : L(G) \to L(G)$ earlier for the case that $G = \mathbb{Z}/n\mathbb{Z}.$ Now, we do it in a more general setting, taking inspiration from the earlier. Instead of a map $L(G) \to L(G),$ we define $\mathcal{F} : L(G) \to L(\widehat{G}).$ 

\begin{defn}
    Let $G$ be a finite abelian group and let $f : G \to \mathbb{C}$ be a function. We define the \deff{Fourier transform} $\mathcal{F}(f) = \widehat{f} \in L(\widehat{G})$ by 
    \begin{equation*}
        \widehat{f}(\chi) = \md{G}\langle f, \chi\rangle = \sum_{g \in G}f(g) \overline{\chi(g)}.
    \end{equation*}
    The complex numbers $\md{G}\langle f, \chi\rangle$ are called the \deff{Fourier coefficients} of $f.$
\end{defn}

Note that $\widehat{f}$ is a function $\widehat{f} : \widehat{G} \to \mathbb{C}.$ That is, $\widehat{f}$ takes irreducible characters of $G$ as input.

\begin{rem}
    Recall we had constructed an isomorphism $\mathbb{Z}/n\mathbb{Z} \cong \widehat{\mathbb{Z}/n\mathbb{Z}}$ in \Cref{prop:ZnZconghatZnZ}. The isomorphism was given as $[k] \leftrightarrow \chi_k,$ where $\chi_k \in \widehat{\mathbb{Z}/n\mathbb{Z}},$ as before is
    \begin{equation*} 
        \chi_k([m]) = w_n^{mk}.
    \end{equation*}

    Under this identification, we see that the Fourier transform defined above agrees with the one in \Cref{defn:fourierZnZ}.
\end{rem}

\begin{ex}[Fourier transform of a character] \label{ex:fourtranschar}
    Let $\chi, \theta \in \widehat{G}$ be (irreducible) characters of the abelian group $G.$ Then, $\chi \in L(G)$ and thus, the Fourier transform of $\chi$ makes sense and we have
    \begin{equation*}
        \widehat{\chi}(\theta) = \md{G} \langle \chi, \theta\rangle = 
        \begin{cases}
            \md{G} & \theta = \chi, \\
            0 & \theta \neq \chi.
        \end{cases}
    \end{equation*}
    The last equality follows from \nameref{thm:firstorthorel}.
    
    Thus, $\widehat{\chi} = \md{G} \delta_\chi.$
\end{ex}

As before, we have the Fourier inversion. 

\begin{thm}[Fourier inversion] \label{thm:fourierinv}
    Let $G$ be an abelian group. If $f \in L(G),$ then
    \begin{equation*}
        f = \frac{1}{\md{G}}\sum_{\chi \in \widehat{G}} \widehat{f}(\chi)\chi.
    \end{equation*}
    That is, if $\mathcal{F}(f) = \mathcal{F}(g),$ then $f = g.$ In other words, $\mathcal{F}$ is injective.
\end{thm}
\begin{proof}
    As earlier, we use the fact that the characters form an orthonormal basis along with the computation that
    \begin{equation*}
        f = \sum_{x \in \widehat{G}}\langle f, \chi\rangle\chi = \frac{1}{\md{G}}\sum_{x \in \widehat{G}}\md{G}\langle f, \chi\rangle\chi = \frac{1}{\md{G}}\sum_{x \in \widehat{G}} \widehat{f}(\chi)\chi. \qedhere
    \end{equation*}
\end{proof}

\begin{prop} \label{prop:fourierisovspace}
    The map $\mathcal{F} : L(G) \to L(\widehat{G})$ is an isomorphism of vector spaces.
\end{prop}
\begin{proof}
    Let $\md{G} = n.$ Let $f_1, f_2 \in L(G)$ and $\alpha \in \mathbb{C}$ be arbitrary. For $\chi \in \widehat{G},$ we note that
    \begin{align*}
        \mathcal{F}(\alpha f_1 + f_2)(\chi) &= n\langle \alpha f_1 + f_2, \chi\rangle \\
        &= n\alpha\langle f_1, \chi\rangle + n\langle f_2, \chi\rangle = \alpha\mathcal{F}(f_1)(\chi) + \mathcal{F}(f_2)(\chi).
    \end{align*}
    Thus, $\mathcal{F}$ is linear.
    
    By \nameref{thm:fourierinv}, $\mathcal{F}$ is injective. Now, since $\dim(L(G)) = n = \dim(L(\widehat{G})),$ we see that $\mathcal{F}$ is an isomorphism.
\end{proof}

We now also wish $\mathcal{F}$ to be an isomorphism of rings. However, for this purpose, the convolution product on $L(\widehat{G})$ does not do the trick. Instead, we need the point-wise product on $L(\widehat{G}).$ Clearly, this does make $L(\widehat{G})$ a commutative ring with the constant map $g \mapsto 1$ as identity. As noted earlier, $L(G)$ is also commutative in this case. However, its identity is not the constant function but the function $\delta_1.$

\begin{thm} \label{thm:fourierisorings}
    Let $G$ be an abelian group and let $a, b \in L(G).$

    The Fourier transform satisfies
    \begin{equation*}
        \widehat{a * b} = \widehat{a} \cdot \widehat{b}.
    \end{equation*}
    
    Consequently, the linear map $\mathcal{F} : L(G) \to L(\widehat{G})$ is an isomorphism between the rings $(L(G), +, *)$ and $(L(\widehat{G}), +, \cdot).$
\end{thm}
\begin{proof}
    The only thing that is required to be proven is that $\widehat{a * b} = \widehat{a} \cdot \widehat{b}.$ (Note that both sides are functions $\widehat{G} \to \mathbb{C}.$) Set $n = \md{G}$ and let $\chi \in \widehat{G}$ be arbitrary.
    
    We see that
    \begin{align*}
        \widehat{a * b}(\chi) &= n \langle a * b, \chi\rangle \\
        &= n \cdot \frac{1}{n} \sum_{x \in G} (a * b)(x) \overline{\chi(x)} \\
        &= \sum_{x \in G}\overline{\chi(x)} \sum_{y \in G} a(xy^{-1})b(y) \\
        &= \sum_{y \in G}b(y) \sum_{x \in G}a(xy^{-1})\overline{\chi(x)} \\
        &= \sum_{y \in G}b(y) \sum_{z \in G}a(z)\overline{\chi(zy)} \\
        &= \sum_{y \in G}b(y)\overline{\chi(y)} \sum_{z \in G}a(z)\overline{\chi(z)} \\
        &= n\langle a, \chi\rangle \cdot n\langle b, \chi\rangle \\
        &= \widehat{a}(\chi) \cdot \widehat{b}(\chi). \qedhere
    \end{align*}
\end{proof}

\subsection{An application to Graph Theory}

\begin{defn}%[Graph]
    A \deff{graph} $\Gamma$ is an ordered pair $(V, E)$ where $V$ is a finite \underline{ordered} set and $E$ is a set consisting of unordered \emph{pairs} of elements of $V.$ $V$ is called the \deff{vertex set} of $\Gamma$ and each element of $V$ is called a \deff{vertex}. $E$ is called the \deff{edge set} of $\Gamma$ and each element of $E$ is called an \deff{edge}.

    Given $v, w \in V,$ we say that $v$ and $w$ are \deff{connected by an edge} if $\{v, w\} \in E.$
\end{defn}

\begin{ex}
    $\Gamma = ((1, 2, 3, 4), \{\{1, 3\}, \{2, 3\}, \{2, 4\}, \{3, 4\}\})$ is a graph. This can be depicted as the following diagram.
    

    \begin{center}
        \captionsetup{type=figure}
        \begin{tikzpicture}
            \GraphInit[vstyle=Normal]
            \SetGraphUnit{2}
            \begin{scope}[rotate = 90]
                \Vertices{circle}{2, 3, 4}
            \end{scope}

            \NOWE(3){1}
            \Edges(1, 3, 2, 4, 3)
        \end{tikzpicture}
        \captionof{graph}{A graph} \label{fig:graph1}
    \end{center}
\end{ex}

\begin{rem}
    As in the previous example, we often depict graphs using a diagram as shown above. We draw circles to represent the vertices and draw a line between two vertices $v_i$ and $v_j$ iff $\{v_i, v_j\} \in E.$

    Note that our definition only talks about unordered pairs. Moreover, note that we do not talk about an edge from a vertex to itself.
\end{rem}

\begin{defn}%[Adjacency matrix]
    Let $\Gamma$ be a graph with vertex set $V = (v_1, \ldots, v_n)$ and edge set $E.$ Then, we define the \deff{adjacency matrix} $A = (a_{ij}) \in M_n(\mathbb{Z})$ as
    \begin{equation*} 
        a_{ij} = \begin{cases}
            1 & \{v_i, v_j\} \in E,\\
            0 & \{v_i, v_j\} \notin E. 
        \end{cases}
    \end{equation*}
\end{defn}

\begin{rem}
    By definition, we see that $A$ is symmetric and thus, real diagonalisable, by the spectral theorem. Note that the diagonal entries of $A$ will always be $0.$
\end{rem}

\begin{ex}
    For \Cref{fig:graph1}, the adjacency matrix is given as
    \begin{equation*} 
        A = \begin{bmatrix}
             &   & 1 &   \\
             &   & 1 & 1 \\
           1 & 1 &   & 1 \\
             & 1 & 1 &   \\
        \end{bmatrix}.
    \end{equation*}
\end{ex}

\begin{defn}%[]
    Let $G$ be a finite group written in some fixed order. A subset $S \subset G$ is said to be a \deff{symmetric subset} of $G$ if
    \begin{enumerate}
        \item $1 \notin S,$
        \item $s \in S \implies s^{-1} \in S.$
    \end{enumerate}
    If $S$ is a symmetric subset of $G,$ then the \deff{Cayley graph} of $G$ with respect to $S$ is the graph with vertex set $G$ and with an edge $\{g, h\}$ iff $gh^{-1} \in S.$
\end{defn}

\begin{aside}
    \textbf{Convention}

    Whenever we consider $G = \mathbb{Z}/n\mathbb{Z},$ we shall consider the order to be fixed as $G = ([0], \ldots, [n - 1]).$
\end{aside}

\begin{rem}
    Since $S$ is symmetric, $gh^{-1} \in S$ is equivalent to $hg^{-1} \in S.$ This shows that the above is indeed well-defined. (Since $\{g, h\} = \{h, g\}.$) Note that since $1 \notin S,$ we do not have any singletons in the edge set, which is compatible with our definition.
\end{rem}

\begin{ex}
    Consider the group $G = \mathbb{Z}/4\mathbb{Z}$ (with the decided order). Then, $S = \{\pm [1]\}$ is a symmetric subset of $G.$ (Note that the $1$ of $G$ is $[0].$)

    The Cayley graph is given as

    \begin{center}
        \captionsetup{type=figure}
        \begin{tikzpicture}
            \GraphInit[vstyle=Normal]
            \SetGraphUnit{2}
            \begin{scope}[rotate = 45]
                \Vertices{circle}{[0], [1], [2], [3]}
            \end{scope}
            \Edges([0], [1], [2], [3], [0])
        \end{tikzpicture}
        \captionof{graph}{Cayley graph of $\mathbb{Z}/4\mathbb{Z}$ with respect to $\{\pm [1]\}$} \label{fig:cayleyz4z}
    \end{center}
    and the adjacency matrix as
    \begin{equation*} 
        A = \begin{bmatrix}
            & 1 & & 1\\
             1 &&  1&\\
            & 1 & & 1\\
             1 &&  1&\\
        \end{bmatrix}.
    \end{equation*}
\end{ex}  

\begin{ex}
    If we take $G = \mathbb{Z}/6\mathbb{Z}$ and $S = \{\pm [1], \pm [2]\},$ we get the graph as
    \begin{center}
        \captionsetup{type=figure}
        \begin{tikzpicture}
            \GraphInit[vstyle=Normal]
            \SetGraphUnit{3}
            \begin{scope}%[rotate = 45]
                \Vertices{circle}{[0], [1], [2], [3], [4], [5]}
            \end{scope}
            \Edges([0], [1], [2], [3], [4], [5], [0])
            \Edges([0], [2], [4], [0])
            \Edges([1], [3], [5], [1])
        \end{tikzpicture}
        \captionof{graph}{Cayley graph of $\mathbb{Z}/6\mathbb{Z}$ with respect to $\{\pm [1], \pm [2]\}$} \label{fig:cayleyz6z}
    \end{center}

    and the adjacency matrix as

    \begin{equation*} 
        A = \begin{bmatrix}
              & 1 & 1 &   & 1 & 1 \\
            1 &   & 1 & 1 &   & 1 \\
            1 & 1 &   & 1 & 1 &   \\
              & 1 & 1 &   & 1 & 1 \\
            1 &   & 1 & 1 &   & 1 \\
            1 & 1 &   & 1 & 1 &   \\
        \end{bmatrix}
    \end{equation*}
\end{ex}

\begin{defn}%[Circulant graph]
    A Cayley graph of $\mathbb{Z}/n\mathbb{Z}$ is called a \deff{circulant graph} (on $n$ vertices).
\end{defn}

\begin{defn}%[Circulant matrix]
    An $n \times n$ \deff{circulant matrix} is a matrix of the form
    \begin{equation*} 
        A = \begin{bmatrix}
            a_0 & a_1 & \cdots & a_{n - 2} & a_{n - 1} \\
            a_{n - 1} & a_0 & \cdots & a_{n - 3} & a_{n - 2}\\
            \vdots & \vdots & \ddots & \vdots & \vdots \\
            a_2 & a_3 & \cdots & a_{0} & a_1\\
            a_1 & a_2 & \cdots & a_{n - 1} & a_0
        \end{bmatrix}.
    \end{equation*}
    Equivalently, a matrix $A = (a_{ij})$ is circulant if there exists a function $f : \mathbb{Z}/n\mathbb{Z} \to \mathbb{C}$ such that $a_{ij} = f([j] - [i]).$
\end{defn} 

\begin{ex}
    If $S$ is a symmetric subset of $G \vcentcolon= \mathbb{Z}/n\mathbb{Z},$ then the indicator function $\delta_S : G \to \mathbb{C}$ is defined as
    \begin{equation*} 
        \delta_S(x) = \begin{cases}
            1 & x \in S,\\
            0 & x \notin S. 
        \end{cases}
    \end{equation*} 
    Consider the matrix $A = (a_{ij})$ given by $a_{ij} = f([j] - [i]).$ Then, note that 
    \begin{equation*} 
        a_{ij} = 1 \iff [j] - [i] \in S \iff \{[i], [j]\} \in E,
    \end{equation*}
    where $E$ denotes the edge set of the Cayley graph of $G$ with respect to $S.$

    Thus, we see that the circulant matrix corresponding to $\delta_S$ is the adjacency matrix of the Cayley graph of $G$ with respect to $S.$
\end{ex}

\begin{lem} \label{lem:convolvediagonal}
    Let $G$ be an abelian group and $a \in L(G).$ Define $A : L(G) \to L(G)$ by $A(b) = a * b.$ Then, $A$ is linear and $\chi$ is an eigenvector of $A$ with eigenvalue $\widehat{a}(\chi)$ for all $\chi \in \widehat{G}.$ Consequently, $A$ is diagonalisable.
\end{lem}
\begin{proof} 
    That $A$ is additive follows from the fact that $*$ distributes over $+.$ That $A(\alpha b) = \alpha A(b)$ for $\alpha \in \mathbb{C}$ and $b \in L(G)$ follows easily from the definition of $*.$ Thus, $A$ is linear.

    Put $n \vcentcolon= \md{G}.$ Now, let $\chi \in \widehat{G}$ be arbitrary. Then,
    \begin{equation*} 
        \widehat{a * \chi} = \widehat{a} \cdot \widehat{\chi} = \widehat{a} \cdot n \delta_\chi,
    \end{equation*}
    where the last equality follows from \Cref{ex:fourtranschar}. Now, for $\theta \in \widehat{G},$ we note that
    \begin{equation*} 
        (\widehat{a} \cdot n \delta_\chi)(\theta) = \begin{cases}
            n\widehat{a}(\chi) & \theta = \chi,\\
            0 & \theta \neq \chi.   
        \end{cases}
    \end{equation*}
    Thus, we have $\widehat{a * \chi} = \widehat{a} \cdot n \delta_\chi = n \widehat{a}(\chi) \delta_\chi.$ By \Cref{ex:fourtranschar} again, we get
    \begin{equation*} 
        \widehat{a * \chi} = n \widehat{a}(\chi) \widehat{\chi}.
    \end{equation*}
    Applying the inverse Fourier transform (and using its linearity), we get
    \begin{equation*} 
        a * \chi = \widehat{a}(\chi) \chi.
    \end{equation*}
    (Note that $\widehat{a}(\chi) \in \mathbb{C}$ is a constant.) The above equation is simply
    \begin{equation*} 
        A(\chi) = \widehat{a}(\chi)\chi
    \end{equation*}
    and hence, $\chi$ is an eigenvector with eigenvalue $\widehat{a}(\chi).$

    Note that the irreducible characters of $G$ form a basis of $Z(L(G)),$ by \Cref{thm:onbforzlg}. Since $G$ is abelian, the irreducible characters are precisely the elements of $\widehat{G}$ and $Z(L(G)) = L(G).$ Thus, we see that $\widehat{G}$ forms a basis of $L(G)$ and have just shown that all these elements are eigenvectors of $A.$ Thus, $\widehat{G}$ is an eigenbasis of $L(G)$ corresponding and $A$ is diagonalisable.
\end{proof}

\begin{thm} \label{thm:adjcayleyeigen}
    Let $G = (g_1, \ldots, g_n)$ be an ordered abelian group and let $S \subset G$ be a symmetric subset. Let $\chi_1, \ldots, \chi_n$ be irreducible characters of $G$ and let $A$ be the adjacency matrix of the Cayley graph of $G$ with respect to $S.$ Then,
    \begin{enumerate}
        \item The eigenvalues of $A$ are the real numbers
        \begin{equation*} 
            \lambda_i \vcentcolon= \sum_{s \in S} \chi_i(s)
        \end{equation*}
        for $1 \le i \le n;$
        \item A corresponding orthonormal basis of eigenvectors is given by the vectors $(v_1, \ldots, v_n)$ where
        \begin{equation*} 
            v_i \vcentcolon= \frac{1}{\sqrt{\md{G}}} \begin{bmatrix}
                \chi_i(g_1) & \cdots & \chi_i(g_n)
            \end{bmatrix}^\mathsf{T}.
        \end{equation*}
    \end{enumerate}
\end{thm}

\begin{proof} 
    The indicator function $\delta_S$ of $S$ is given by $\delta_S = \sum_{s \in S} \delta_s.$ Note that $\delta_S \in L(G).$ Define $F : L(G) \to L(G)$ by
    \begin{equation*} 
        F(b) = \delta_S * b.
    \end{equation*}
    Then, by \Cref{lem:convolvediagonal}, $F$ has eigenvectors $\chi_i$ with corresponding eigenvalue
    \[\begin{WithArrows}[displaystyle]
        \widehat{\delta_S}(\chi_i) = n\langle \delta_S, \chi_i\rangle &= \sum_{x \in G} \delta_S(x)\overline{\chi_i(x)}\\
        &= \sum_{x \in S} \overline{\chi_i(x)} \Arrow{\Cref{rem:findegoneunitary}}\\
        &= \sum_{x \in S} \chi_i(x^{-1}) \Arrow{$S$ is symmetric}\\
        &= \sum_{y \in S} \chi_i(y)\\
        &= \lambda_i.
    \end{WithArrows}\]
    This proves that $\lambda_i$ is an eigenvalue for each $1 \le i \le n.$ (Note that $\lambda_i \in \mathbb{R}$ follows since $A$ is symmetric.\footnote{Alternately, one can note that that sum in the definition of $\lambda_i$ is real. Indeed, if $s = s^{-1}$, then $\chi(s) = \chi(s^{-1}) = \overline{\chi(s)} \in \mathbb{R}.$ On the other hand, if $s \neq s^{-1},$ then pairing up $\chi(s)$ and $\chi(s^{-1})$ in the sum gives $\chi(s) + \chi(s^{-1}) = \chi(s) + \overline{\chi(s)} \in \mathbb{R}.$})

    Consider the ordered basis $B = (\delta_{g_1}, \ldots, \delta_{g_n})$ for $L(G).$ Let $[F]_B$ denote the matrix of $F$ with respect to this ordered basis. Note that the coordinate vector of $\chi_i$ with respect to $B$ is precisely $\sqrt{\md{G}}v_i.$ The above shows that it is an eigenvector with eigenvalue $\lambda_i.$ The orthogonality of $v_i$ follows from \Cref{thm:secondorthorel}. (Note that the conjugacy classes are singletons since $G$ is abelian.) \\
    Lastly, the orthonormality follows from noting that
    \begin{equation*} 
        \|v_i\|^2 = \frac{1}{G}\left(\md{\chi_i(g_1)}^2 + \cdots + \md{\chi_i(g_n)}^2\right) = \|\chi_i\|^2 = 1.
    \end{equation*}
    Thus, to complete the proof, it suffices to show that $A = [F]_B.$

    Let $1 \le i, j \le n$ be arbitrary. We show that $A_{ij} = ([F]_B)_{ij}.$ Note that $([F]_B)_{ij}$ is the coefficient of $\delta_{g_i}$ in $F(\delta_{g_j}).$ We note that
    \begin{equation*} 
        F(\delta_{g_j}) = \delta_S * \delta_{g_j} = \sum_{s \in S}\delta_s * \delta_{g_j} = \sum_{s \in S} \delta_{sg_j} = \sum_{g \in Sg_j} \delta_g.
    \end{equation*}
    (We have used $\delta_{gh} = \delta_g * \delta_h,$ by \Cref{prop:convdeltaprod}.)

    Thus, 
    \begin{align*} 
        ([F]_B)_{ij} &= \begin{cases}
            1 & g_i \in Sg_j,\\
            0 & g_i \notin Sg_j,
        \end{cases}\\
        &= \begin{cases}
            1 & g_ig_j^{-1} \in S,\\
            0 & g_ig_j^{-1} \notin S,
        \end{cases}\\
        &= A_{ij}. \qedhere
    \end{align*}
\end{proof}

\begin{cor}
    Let $A$ be a circulant matrix of degree $n,$ which is the adjacency matrix of the Cayley graph of $\mathbb{Z}/n\mathbb{Z}$ with respect to some symmetric subset $S \subset \mathbb{Z}/n\mathbb{Z}.$ Then, the eigenvalues of $A$ are
    \begin{equation*} 
        \lambda_k \vcentcolon= \sum_{[m] \in S} \omega_n^{km},
    \end{equation*}
    for $k = 0, \ldots, n - 1$ with corresponding basis of orthonormal eigenvectors given by
    \begin{equation*} 
        v_k \vcentcolon= \frac{1}{\sqrt{n}}\begin{bmatrix}
            1 & \omega_n^{k} & \omega_n^{2k} & \cdots & \omega_n^{(n - 1)k}
        \end{bmatrix}^\mathsf{T}.
    \end{equation*}
\end{cor}
\begin{proof} 
    The result follows from \Cref{thm:adjcayleyeigen} since we have
    \begin{equation*} 
        \chi_k([m]) = \omega_n^{km}
    \end{equation*}
    for $k = 0, \ldots, n - 1.$ 
\end{proof}

\begin{ex}
    Recall \Cref{fig:cayleyz6z}. We had $n = 6$ and $S = \{\pm [1], \pm [2]\}.$ Thus, the eigenvalues of $A$ are given by
    \begin{equation*} 
        \lambda_k = \omega_6^k + \omega_6^{-k} + \omega_6^{2k} + \omega_6^{-2k} = 2\left(\cos\left(\frac{\pi k}{3}\right) + \cos\left(\frac{2\pi k}{3}\right)\right)
    \end{equation*}
    for $k = 0, \ldots, 5.$
\end{ex}

\subsection{Fourier Analysis on Non-abelian Groups}

If $G$ is a non-abelian group, then $L(G) \neq Z(L(G)).$ Note however that that pointwise product of functions into $\mathbb{C}$ is commutative. Thus, we cannot have a Fourier transform converting convolution into pointwise product while being an isomorphism. To remedy this, we look at matrix multiplication instead of pointwise.

Before that, we look at the case of abelian groups in a different light. First, recall that $\mathbb{C}^n$ is a ring with product given as
\begin{equation*} 
    (w_1, \ldots, w_n) \cdot (z_1, \ldots, z_n) \vcentcolon= (w_1z_1, \ldots, w_nz_n).
\end{equation*}

\begin{prop}
    Let $G$ be a finite abelian group with irreducible characters $\chi_1, \ldots, \chi_n.$ Define $T : L(G) \to \mathbb{C}^n$ by
    \begin{equation*} 
        Tf = (\widehat{f}(\chi_1), \ldots, \widehat{f}(\chi_n)).
    \end{equation*}
    Then, $T$ is an isomorphism of rings.
\end{prop}
\begin{proof} 
    Note that if $f, g \in L(G)$ are such that $Tf = Tg,$ then $\widehat{f} = \widehat{g}.$ Fourier inversion gives $f = g.$ Thus, $T$ is injective. Since $T$ is $\mathbb{C}$-linear (same proof as \Cref{prop:fourierisovspace}) and $\dim_{\mathbb{C}}(L(G)) = n = \dim_{\mathbb{C}}(\mathbb{C}^n),$ it follows that $T$ is a bijection and hence, an isomorphism of $\mathbb{C}$-vector spaces.

    To show that is an isomorphism of rings, all that remains is to show that $T(f * g) = Tf \cdot Tg.$ This follows directly from the fact that $\widehat{f * g}(\chi_i) = \widehat{f}(\chi_i)\widehat{g}(\chi_i)$ for all $i = 1, \ldots, n.$
\end{proof}

Thus, \Cref{thm:fourierisorings} can be stated as follows.
\begin{thm}
    Let $G$ be a finite abelian group of order $n.$ Then, $L(G) \cong \mathbb{C}^n$ as rings.
\end{thm}

Note that all the irreducible representations of $G$ have degree one. The above product can be seen as $\mathbb{C}^n \cong M_1(\mathbb{C}) \times \cdots \times M_1(\mathbb{C}).$ In general, we replace the $1$s with the degrees of the irreducible representations.

\begin{aside}
    \textbf{Setup.}

    $G$ is a finite group of order $n$ and $\varphi^{(1)}, \ldots, \varphi^{(s)}$ is a transversal of irreducible unitary representations of $G.$ Put $d_k = \deg \varphi^{(k)}.$ 

    For $1 \le k \le s$ and $1 \le i, j \le d_k,$ we have the functions $\varphi^{(k)}_{ij} : G \to \mathbb{C}$ such that the matrix $\varphi^{(k)}(g)$ is given as $(\varphi^{(k)}_{ij}(g))$ for all $g \in G.$

    Let $D = \{(i, j, k) : 1 \le k \le s,\; 1 \le i, j \le d_k\}.$

    We recall \Cref{cor:orthnormalbasis} which told us that $B = \left\{\sqrt{d_k}\varphi_{ij}^{(k)} \mid (i, j, k) \in D\right\}$ is an orthonormal basis for $L(G).$
\end{aside}

\begin{defn}%[]
    Define
    \begin{equation*} 
        \mathcal{F} : L(G) \to M_{d_1}(\mathbb{C}) \times \cdots \times M_{d_s}(\mathbb{C})
    \end{equation*}
    by $\mathcal{F}(f) \vcentcolon= (\widehat{f}(\varphi^{(1)}), \ldots, \widehat{f}(\varphi^{(s)}))$ where
    \begin{equation*}
        \widehat{f}(\varphi^{(k)}) \vcentcolon= \sum_{g \in G} f(g)\overline{\varphi^{(k)}_g}.
    \end{equation*}
    We call $\mathcal{F}(f)$ the \deff{Fourier transform} of $f.$
\end{defn}
\begin{rem}
    In terms of the matrix entries, the above can be rewritten as
    \begin{equation} \label{eq:012}
        \widehat{f}(\varphi^{(k)})_{ij} \vcentcolon= \sum_{g \in G} f(g)\overline{\varphi^{(k)}_{ij}(g)} = n\langle f, \varphi^{(k)}_{ij}\rangle.
    \end{equation}
\end{rem}

\begin{rem} \label{rem:dimLGdimmatrix}
    Note that $L(G)$ is a $\mathbb{C}$-vector space of dimension $\md{G}.$ On the other hand, $M_{d_i}(\mathbb{C})$ is a $\mathbb{C}$-vector space with dimension $d_i^2.$ Thus, the product $M_{d_1}(\mathbb{C}) \times \cdots \times M_{d_s}(\mathbb{C})$ is a vector space of dimension $d_1^2 + \cdots + d_s^2 = \md{G}.$

    Thus, we see that $L(G)$ and $M_{d_1}(\mathbb{C}) \times \cdots \times M_{d_s}(\mathbb{C})$ have the same dimension. We shall show that $\mathcal{F}$ is an isomorphism.
\end{rem}

\begin{thm}[Fourier inversion] \label{thm:fourierinvgen}
    Let $f \in L(G).$ Then,
    \begin{equation*} 
        f = \frac{1}{n}\sum_{(i, j, k) \in D}d_k \widehat{f}(\varphi^{(k)})_{ij} \varphi^{(k)}_{ij}.    
    \end{equation*}
    In particular, $\mathcal{F}$ is injective.
\end{thm}
\begin{proof} 
    The proof is as before. We know that $B$ is an orthonormal basis. Thus, it suffices to prove that
    \begin{equation*} 
        \langle f, \sqrt{d_k}\varphi^{(k)}_{ij}\rangle = \frac{1}{n}\sqrt{d_k} \widehat{f}(\varphi^{(k)})_{ij}
    \end{equation*}
    for all $(i, j, k) \in D.$ However, the above is precisely \Cref{eq:012}.
\end{proof}

\begin{thm}
    The Fourier transform $\mathcal{F} : L(G) \to M_{d_1}(\mathbb{C}) \times \cdots \times M_{d_s}(\mathbb{C})$ is an isomorphism of vector spaces.
\end{thm}
\begin{proof} 
    As usual, the check that it is linear is simple and essentially follows from the fact that $\langle \cdot , \cdot \rangle$ is linear in the first variable. \\
    By \Cref{thm:fourierinvgen}, we know that $\mathcal{F}$ is injective. \\
    By \Cref{rem:dimLGdimmatrix}, we know that $\dim_{\mathbb{C}}(L(G)) = \dim_{\mathbb{C}}(M_{d_1}(\mathbb{C}) \times \cdots \times M_{d_s}(\mathbb{C}))$ and thus, $\mathcal{F}$ is an isomorphism.
\end{proof}

Note that $M_{d_1}(\mathbb{C}) \times \cdots \times M_{d_s}(\mathbb{C})$ is a ring as well, with coordinate-wise product.

\begin{thm}[Wedderburn]
    The Fourier transform $\mathcal{F} : L(G) \to M_{d_1}(\mathbb{C}) \times \cdots \times M_{d_s}(\mathbb{C})$ is an isomorphism of rings.
\end{thm}
\begin{proof} 
    All that is required to be proven is that $T(f * g) = Tf \cdot Tg,$ where the latter product is in the ring $M_{d_1}(\mathbb{C}) \times \cdots \times M_{d_s}(\mathbb{C}).$ Let $a, b \in L(G).$

    Since the latter product is coordinate-wise, it suffices to show that
    \begin{equation*} 
        \widehat{(a * b)}(\varphi^{(k)}) = \widehat{a}(\varphi^{(k)})\widehat{b}(\varphi^{(k)})
    \end{equation*}
    for all $1 \le k \le s.$ (The product on the right is matrix multiplication.)

    The computation is the same as before.
    \begin{align*} 
        \widehat{(a * b)}(\varphi^{(k)}) &= \sum_{g \in G} (a * b)(g)\overline{\varphi^{(k)}(g)}\\
        &= \sum_{g \in G}\left(\sum_{h \in G}a(gh^{-1})b(h)\right)\overline{\varphi^{(k)}(g)}\\
        &= \sum_{h \in G}b(h)\sum_{g \in G}a(gh^{-1})\overline{\varphi^{(k)}(g)}\\
        &= \sum_{h \in G}b(h)\sum_{g' \in G}a(g')\overline{\varphi^{(k)}(g'h)}\\
        &= \sum_{h \in G}b(h)\sum_{g' \in G}a(g')\overline{\varphi^{(k)}(g')}\cdot\overline{\varphi^{(k)}(h)}\\
        &= \sum_{g' \in G}a(g')\overline{\varphi^{(k)}(g')}\sum_{h \in G}b(h)\overline{\varphi^{(k)}(h)}\\
        &= \widehat{a}(\varphi^{(k)})\widehat{b}(\varphi^{(k)}). \qedhere
    \end{align*}
\end{proof}
In the above computation, note that $a$ and $b$ took values in $\mathbb{C}$ and so, commuted with the other terms.