\section*{Preface}
\addcontentsline{toc}{section}{Preface}
This report serves as an introduction to the topic of Representation Theory of Finite Groups. It is largely self-contained with only the basic definitions of group theory and linear algebra being assumed. Everything else is either introduced later or in the Preliminaries (\Cref{sec:00}). \Cref{sec:08} is the only exception to this rule where we assume some knowledge of Galois theory and Sylow theorems.

\textbf{How to read}

There is not much point in going through all the preliminaries to start with. The reader should read \nameref{subsec:notations} and \nameref{subsec:ips} and then begin with \Cref{sec:01}. Whenever there is a need to recall something from the preliminaries, it is mentioned at the relevant moment.

Here are the relations between the various sections.

\begin{equation*} 
	\begin{tikzcd}
		    & \S1 \arrow[d]                                              &     &     \\
		    & \S2 \arrow[d] \arrow[ld] \arrow[rd] \arrow[rrd, bend left] &     &     \\
		\S3 & \S4 \arrow[ld] \arrow[d]                                   & \S7 & \S8 \\
		\S5 & \S6                                                        &     &    
	\end{tikzcd}
\end{equation*}

\textbf{References}

Only one reference book has been followed for this, namely:

\begin{itemize}
	\item[{[BS]}] Benjamin Steinberg, \emph{Representation Theory of Finite Groups}.
\end{itemize}

The results and broad structure of presentation is the same as in the book. However, the style of writing is influenced largely by the suggestions of Professor A. Hariharan. There are many more examples and few more results not in book that came out of my discussions with him. However, the credits of the typo s and misteaks go to me.