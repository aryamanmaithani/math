\documentclass[12pt]{article}	% Always compile at least twice.
\usepackage[lmargin=1in,rmargin=1in,tmargin=1in,bmargin=1in]{geometry}
% Cover Information	
\def\univname{}
\def\coursenum{}
\def\coursename{Representation Theory of Finite Groups}
\def\professor{Ananthnarayan Hariharan}
\def\student{Aryaman Maithani}
\def\semesteryear{Winter 2020}
\def\imagename{iitb.png}		    % Replace with University Seal
\def\scalesize{0.20}					% Scale Logo Size 

% Style Package (Load After Cover Information)
\usepackage{aryaman}	% Change to match style file

% -------------------
% Content
% -------------------
\begin{document}


% Cover Page
\coverpage

% Last Updated
\thispagestyle{empty}
\updated{\today}

% Table of Contents
\thispagestyle{empty}
\tableofcontents
% \newpage
\pagestyle{fancy}
\setcounter{section}{-1}
\setcounter{page}{1}

\section{Notations and Preliminaries}
\setcounter{subsection}{-1}
\subsection{Notations}
$\mathbb{C}$ will denote the field of complex numbers and $V, W$ vector spaces over $\mathbb{C}.$ (We will stick to finite dimensional vector spaces.)

\begin{enumerate}
	\item If $V$ is a vector space and $W$ a subspace, then we write $W \le V.$
	\item If $X \subset V,$ then $\mathbb{C}X = \spn X.$ (cf. \Cref{subsec:linearisation} for the definition of $\mathbb{C}X$ when $X$ is an arbitrary set.)
	\item $R^*$ denotes the group of units of a ring $R.$
	\item $M_{m \times n}(\mathbb{C})$ is the vector space $m \times n$ matrices with entries in $\mathbb{C}.$
	\item $M_n(\mathbb{C}) = M_{n \times n}(\mathbb{C}).$
	\item $\Hom_{\mathbb{C}}(V, W)$ is the vector space of linear maps from $V$ to $W.$
	\item $\End(V) = \Hom_{\mathbb{C}}(V, V)$ is the \emph{ring of endomorphisms}. This is isomorphic to $M_{\dim V}(V).$
	\item $\GL(V) = \{A \in \End(V) \mid A \text{ is invertible}\} = \End(V)^*$ is the \emph{general linear group} of $V.$
	\item $\GL_n(\mathbb{C}) = M_n(\mathbb{C})^*.$ This is isomorphic to $\GL(\mathbb{C}^n).$
	\item We have the usual sets $\mathbb{N}, \mathbb{Z}, \mathbb{Q}, \mathbb{R}, S_n, \mathbb{Z}/n\mathbb{Z}.$ For us, $0 \notin \mathbb{N}.$ We note $\mathbb{N}_0 \vcentcolon= \mathbb{N} \cup \{0\}.$
	\item $D_n$ will denote the dihedral group with $2n$ elements. 
	\item Given $\mathbb{C}_n,$ we denote by $e_i,$ the $i$-th standard basis vector.
	\item $\omega_n \vcentcolon= \exp\left(\dfrac{2\pi\iota}{n}\right).$
	\item We use $\sqcup$ to denote disjoint union.
\end{enumerate}

\begin{defn}%[Transversal]
	\label{defn:transversal}
	Let $X$ be a set and $\sim$ an equivalence relation on $X.$ A subset $Y \subset X$ is called a \deff{transversal} if $Y$ intersects each equivalence class in exactly one element.	
\end{defn}

\begin{ex}
	If $G$ is a group and $H$ is a subgroup, then the left cosets of $H$ partition $G.$ In particular, they give rise to an equivalence relation. Assuming that the index $m = [G:H]$ is finite, a transversal in this context is simply a set $\{t_1, \ldots, t_m\}$ of representatives of distinct cosets.

	We shall often denote this by writing ``Let $t_1, \ldots, t_m$ be a transversal of the cosets.''
\end{ex}

%
%
\subsection{Linear Algebra Preliminaries}
\subsubsection{Inner product spaces}
\begin{defn}
	Let $V$ be a vector space and $T \in \End(V).$ If $W \le V$ is such that $Tw \in W$ for all $w \in W,$ then $W$ is said to be \deff{$T$-invariant.}
\end{defn}

\begin{prop} \label{prop:TinvariantiffimageW}
	Let $W \le V$ be vector spaces and $T \in \GL(V).$ Then, $W$ is $T$-invariant iff $T(W) = W.$
\end{prop}
\begin{proof} 
	$\impliedby$ is trivial. We prove the other direction.

	By hypothesis, we know that $T(W) \le W.$ However, since $W$ is finite-dimensional and $T$ an isomorphism, we see that
	\begin{equation*} 
		\dim(T(W)) = \dim(W)
	\end{equation*}
	and hence, $T(W) = W.$ (If a subspace of a finite dimensional vector space has the same dimension, then the subspace must be the whole space.)
\end{proof}

\begin{prop} \label{prop:Tinverseinvariance}
	Let $W \le V$ be vector spaces and $T \in \GL(V)$ be such that $W$ is $T$-invariant. Then, $W$ is also $T^{-1}$-invariant.
\end{prop}
\begin{proof} 
	Using \Cref{prop:TinvariantiffimageW}, we know that $T(W) = W.$ Since $T$ is a bijection, this immediately yields that $W = T^{-1}(W),$ proving the desired result.
\end{proof}

\begin{prop} \label{prop:STinvariance}
	Let $W \le V$ be vector spaces and $T, S \in \GL(V)$ be such that $W$ is $T$-invariant and $S$-invariant. Then, $W$ is also $S \circ T$-invariant.
\end{prop}
\begin{proof} 
	Let $w \in W.$ Then, $Tw \in W$ since $W$ is $T$-invariant. In turn, $S(Tw) \in W$ since $W$ is $S$-invariant. Thus, $(S \circ T)(w) \in W$ for all $w \in W,$ as desired.
\end{proof}

\begin{defn}%[Adjoint of an operator]
	Let $(V, \langle \cdot, \cdot\rangle)$ be a finite dimensional inner product space and $T \in \End(V).$ The \deff{adjoint of $T$} is the unique linear operator $T^*$ such that the following equality holds for all $v, w \in V:$
	\begin{equation*} 
		\langle Tv, w\rangle = \langle v, T^*w\rangle.
	\end{equation*}
\end{defn}

\begin{prop} \label{prop:Tadjointinvariance}
	Let $(V, \langle \cdot, \cdot\rangle)$ be an inner product space and $T \in \End(V).$ Suppose that $W \le V$ is $T$-invariant. Then, $W^\perp$ is $T^*$-invariant.
\end{prop}
\begin{proof} 
	Let $v \in W^\perp$ and $w \in W$ be arbitrary. It suffices to show that $\langle T^*v, w\rangle = 0.$ However, this is immediate since
	\begin{equation*} 
		0 = \langle v, Tw\rangle = \langle T^*v, w\rangle.
	\end{equation*}
	The first equality is true since $Tw \in W$ by $T$-invariance of $W$ and $v \in W^\perp,$ by hypothesis.
\end{proof}

\begin{defn}%[Unitary operator] 
	\label{defn:unitaryoperator}
	Let $V$ be an inner product space and $U \in \GL(V).$ $U$ is said to be \deff{unitary} if 
	\begin{equation*} 
		\langle Uv, Uw\rangle = \langle v, w\rangle
	\end{equation*}
	for all $v, w \in V.$ The subset $U(V) \subset \GL(V)$ of all unitary operators forms a subgroup.
\end{defn}
In other words, one sees that
\begin{equation*} 
	\langle v, U^*Uw\rangle = \langle v, w\rangle
\end{equation*}
for all $v, w \in V.$ In other words, $U^*U$ is the adjoint of the identity map. However, since identity is its own adjoint, we see that $U^*U$ is the identity map. In other words, $U^* = U^{-1}.$

\begin{defn}%[Unitary matrix] 
	\label{defn:unitarymatrix}
	A matrix $U \in \GL_n(\mathbb{C})$ is said to be \deff{unitary} if $UU^* = I.$ A set of all such matrices is denoted by $U_n(\mathbb{C})$ and forms a subgroup of $\GL_n(\mathbb{C}).$
\end{defn}
As usual, $U^*$ denotes the conjugate transpose of $U.$ One can show that the matrix $U$ is unitary (\Cref{defn:unitarymatrix}) iff the corresponding linear operator is unitary (\Cref{defn:unitaryoperator}), with respect to the standard inner product on $\mathbb{C}^n.$

\begin{cor} \label{cor:unitaryinvariance}
	Let $(V, \langle \cdot, \cdot\rangle)$ be an inner product space and $T \in U(V).$ Suppose that $W$ is $T$-invariant. Then, $W^\perp$ is also $T$-invariant.
\end{cor}
\begin{proof} 
	By \Cref{prop:Tadjointinvariance}, we see that $W^\perp$ is $T^*$ invariant and hence, $T^{-1}$-invariant.\\
	(Note that $T^{-1} = T^*$ since $T$ is unitary.)\\
	By \Cref{prop:Tinverseinvariance}, we then see that $W^\perp$ is $T$-invariant. \\
	(We are using that $(T^{-1})^{-1} = T.$)
\end{proof}

\subsubsection{Minimal polynomials and diagonalisation} 

\begin{defn}%[Minimal polynomial]
	Let $T \in \End(V).$ The \deff{minimal polynomial} of $T$ is the unique monic polynomial $m(X) \in \mathbb{C}[X]$ such that $m(T)$ is the zero operator.
\end{defn} 

\begin{defn}%[Diagonalisable]
	Let $T \in \End(V).$ $T$ is said to be \deff{diagonalisable} if there exists a basis $B$ of $V$ consisting of eigenvectors of $T.$
\end{defn}

For the remainder, $T$ will denote an element of $\End(V)$ and $m(X)$ its minimal polynomial.

\begin{prop} \label{prop:minimalpolyroots}
	Let $p(X) \in \mathbb{C}[X]$ be any polynomial such that $p(T) = 0.$ Then $p(\lambda) = 0$ for any eigenvalue $\lambda \in \mathbb{C}$ of $T.$ In particular, all eigenvalues of $T$ (in $\mathbb{C}$) are roots of the minimal polynomial.	
\end{prop}
\begin{proof} 
	Let $\lambda \in \mathbb{C}$ be an eigenvalue of $T.$ Let $v \neq 0$ be an eigenvector corresponding to $\lambda.$ Then, note that
	\begin{equation*} 
		T^k v = \lambda^k v
	\end{equation*}
	for all $k \ge 0.$ In particular, if 
	\begin{equation*} 
		p(X) = a_0 + a_1X + \cdots + a_rX^r,
	\end{equation*}
	then we have
	\begin{align*} 
		0 = p(T)v &= (a_0 + a_1T + \cdots + a_rT^r)v\\
		&= a_0v + a_1Tv + \cdots + a_rT^rv\\
		&= a_0v + a_1\lambda v + \cdots + a_r\lambda^r v\\
		&= p(\lambda)v.
	\end{align*}
	Thus, $p(\lambda)v = 0.$ But since $v \neq 0,$ we get that $p(\lambda) = 0,$ as desired.
\end{proof}

\begin{rem}
	Of course, the eigenvalues of $T$ are precisely the roots of the characteristic polynomial of $T.$ Thus, the above proposition tells us that the minimal polynomial and characteristic polynomial have precisely the same roots. (One way implication is in the above, the other is obvious since the minimal polynomial must divide the characteristic polynomial.)
\end{rem}

\begin{prop}
	If $T$ is diagonalisable, then $m(X)$ has distinct roots.
\end{prop}
\begin{proof} 
	Let $\lambda_1, \ldots, \lambda_r \in \mathbb{C}$ be the distinct eigenvalues of $T.$ Let
	\begin{equation*} 
		p(X) = (X - \lambda_1)\cdots(X - \lambda_r).
	\end{equation*}
	Then, by the previous proposition, we know that $p(X) \mid m(X).$ Since both are monic, it suffices to show that $m(X) \mid p(X)$ to conclude that $m(X) = p(X).$ And to do that, it suffices to show that $p(T)$ is the zero operator. And to do \emph{that}, it suffices to show that $p(T)$ annihilates some basis of $V.$ To this end, let $B$ be an eigenbasis of $V$ with respect to $T$ (which exists since $T$ is diagonalisable). Then, any $v \in B$ is annihilated by some $T - \lambda_i.$ Since all the $T - \lambda_j$ commute, we see that $p(T)v = 0$ and we are done.
\end{proof}

\begin{prop}
	Suppose that $m(X)$ has distinct roots. Then, $T$ is diagonalisable.
\end{prop}
\begin{proof} 
	By hypothesis, $m(X) = (X - \lambda_1)\cdots(X - \lambda_r)$ for some distinct $\lambda_1, \ldots, \lambda_r \in \mathbb{C}.$\\
	Since $m(X)$ divides the characteristic polynomials, it follows that each $\lambda_i$ is an eigenvalue. We wish to show that
	\begin{equation*} 
		V = E_{\lambda_1} \oplus \cdots \oplus E_{\lambda_r}.
	\end{equation*}
	Note that since we know that eigenspaces intersect trivially, it suffices to show that
	\begin{equation*} 
		V = E_{\lambda_1} + \cdots + E_{\lambda_r}.
	\end{equation*}
	Now, consider the polynomials
	\begin{equation*} 
		f_i(X) = \dfrac{m(X)}{X - \lambda_i} = \prod_{j \neq i}(X - \lambda_j)
	\end{equation*}
	for $i = 1, \ldots, r.$ Put
	\begin{equation*} 
		g_i(X) = \dfrac{f_i(X)}{f_i(\lambda_i)}.
	\end{equation*}
	(Note that each $f_i(\lambda_i)$ is non-zero since the roots are distinct.)\\
	Note that $g_i(\lambda_j) = \delta_{i, j}.$

	Now, note that
	\begin{equation*} 
		1 = \sum_{i = 1}^{r} g_i(X).
	\end{equation*}
	(Both sides are polynomials of degree at most $r - 1$ which agree on the $r$ points $\lambda_1, \ldots, \lambda_r.$)\\
	Thus, $g_i(T)$ is the identity operator.

	Thus, given any $v \in V,$ we have
	\begin{equation} \tag{$\sum$} \label{eq:sumeigenvectors}
		v = \sum_{i = 1}^{r} g_i(T)v.
	\end{equation}
	However, note now that
	\begin{align*} 
		(T - \lambda_i)g_i(T)v &= (T - \lambda_i)\dfrac{f_i(T)}{f_i(\lambda_i)}v\\
		&= \dfrac{p(T)}{f_i(\lambda_i)}v\\
		&= 0.
	\end{align*}
	Thus, $g_i(T)v \in E_{\lambda_i}$ for each $i$ and \Cref{eq:sumeigenvectors} shows that $V = \bigoplus E_{\lambda_i}.$ 
\end{proof}	

The above two propositions are summarised in the following theorem.

\begin{thm} \label{thm:splitdistinctdiagonalise}
	Let $V$ be a vector space over $\mathbb{C}.$ Let $T \in \End(V)$ and $m(X) \in \mathbb{C}[X]$ be the minimal polynomial of $T.$ Then, $T$ is diagonalisable if and only if $m(X)$ has distinct roots.	
\end{thm}

\subsubsection{Linearisation} \label{subsec:linearisation}
\begin{defn}[Linearisation] \label{defn:linearisation}
	Given a non-empty \underline{finite} set $X,$ we define a $\mathbb{C}$-vector space $\mathbb{C}X$ whose elements are formal linear combinations
	\begin{equation*} 
		\sum_{x \in X} c_x x
	\end{equation*}
	where $c_x \in \mathbb{C}.$

	The addition is given by adding the corresponding scalar coefficients and scalar multiplication is defined similarly.

	$X$ is identified as a subset of $\mathbb{C}X$ be identifying $x$ with $1x.$ Under this, $X$ is a basis for $\mathbb{C}X.$

	This is an inner product space with the product defined as
	\begin{equation*} 
		\left\langle \sum_{x \in X} a_xx, \sum_{x \in X} b_xx\right\rangle = \sum_{x \in X} a_x \overline{b_x}.
	\end{equation*}
	Under this, $X$ is an \emph{orthonormal} basis for $\mathbb{C}X.$
\end{defn}

Note very carefully that we have assumed that $X$ is finite. This avoids the complication of having to make sure that the sums are finite.

The above construction has the following property.
\begin{prop} \label{prop:linearfuncextend}
	Given non-empty finite sets $X,$ $Y$ and a function $f : X \to Y,$ there exists a unique linear transformation 
	\begin{equation*} 
		\mathbb{C}f : \mathbb{C}X \to \mathbb{C}Y
	\end{equation*}
	such that $\mathbb{C}f|_X = f.$
\end{prop}

Those familiar with category theory can actually verify that the above defines a \emph{functor} from the category of sets (and functions) to that of $\mathbb{C}$-vector spaces (and $\mathbb{C}$-linear function).

We can also note the following construction.

\begin{prop} \label{prop:extendingactiontorep}
	Let $G$ be a group which acts on a set $X.$ Then, extending the action in the natural way gives an action on $\mathbb{C}X.$ In other words, we get a homomorphism $\varphi : G \to S_{\mathbb{C}X}.$ Moreover, we have the property that not only is $\varphi(g)$ is a bijection for each $g \in G$ but also an isomorphism.
\end{prop}
\begin{proof} 
	The ``natural way'' of extension is to define
	\begin{equation*} 
		\cdot : G \times \mathbb{C}X \to \mathbb{C}X
	\end{equation*}
	as
	\begin{equation*} 
		g\cdot\left(\sum_{x \in X} c_x x\right) \vcentcolon= \sum_{x \in X} c_x (g \cdot x).
	\end{equation*}
	The $\cdot$ on the right is the original action.\\
	(The right hand side makes sense because $g \cdot x \in X.$)

	With the above explicit formula, it is clear that the group action axioms are satisfied. We now show that the last part. It suffices to show that $\varphi(g)$ is linear.

	In other words, we need to show that $g \cdot (v_1 + v_2) = g \cdot v_1 + g \cdot v_2$ for all $g \in G$ and $v_1, v_2 \in V.$ This is simple, for we note that
	\begin{align*} 
		g \cdot \left(\sum_{x \in X} c_xx + \sum_{x \in X} d_xx \right) &= g \cdot \left(\sum_{x \in X} (c_x + d_x)x \right)\\
		&= \sum_{x \in X} (c_x + d_x)(g \cdot x)\\
		&= \sum_{x \in X} c_x g \cdot x + \sum_{x \in X} d_x g \cdot x\\
		&= g \cdot \left(\sum_{x \in X} c_x x\right) + g \cdot \left(\sum_{x \in X}d_x x\right). \qedhere
	\end{align*}
\end{proof}

In other words, what we have above is actually a \emph{representation}, the central topic of study in this report.

%
%
\subsection{Group Theory Preliminaries}
\begin{lem} \label{lem:determininggrouphomoring}
	Let $G$ be a group, $R$ a commutative ring with identity and $\varphi : G \to R$ a function. Suppose that $\varphi\left(1_G\right) = 1_R$ and $\varphi(g_1g_2) = \varphi(g_1)\varphi(g_2)$ for all $g_1, g_2 \in G.$ Then, $\varphi$ is a homomorphism into $R^\times,$ the group of units of $R.$
\end{lem}
\begin{proof} 
	We only need to show that $\varphi$ is a function into $R^\times.$ The fact that it is a homomorphism would then follow from the fact that it is multiplicative.

	To see that, we simply note 
	\begin{equation*} 
		\varphi(g)\varphi(g^{-1}) = \varphi(gg^{-1}) = \varphi\left(1_G\right) = 1_R = \varphi(g^{-1})\varphi(g)
	\end{equation*}
	and hence, $\varphi(g)$ is invertible for all $g \in G$ with inverse $\varphi(g^{-1}).$
\end{proof}

\begin{rem} \label{rem:monoidhomo}
	Note that the above proof does not require the complete ring structure of $R.$ The reader familiar with monoids can observe that we could replace $R$ with a monoid $M$ and the above would hold.
\end{rem}

\subsubsection{Group of complex homomorphisms}
\begin{defn} \label{defn:dualgroup}
	Given a group $G,$ let the $\widehat{G}$ denote the set of all group homomorphisms from $G$ to $\mathbb{C}^*.$ This is a group under point-wise operations and is called the \deff{dual group} of $G.$
\end{defn}

\begin{prop}
	Let $G,\;G_1,$ and $G_2$ be (not necessarily abelian) groups. If $G = G_1 \times G_2,$ then $\widehat{G} \cong \widehat{G_1} \times \widehat{G_2}.$ 
\end{prop}
\begin{proof} 
	Given $\varphi \in \widehat{G},$ we define $\varphi_1 : \widehat{G_1} \to \mathbb{C}^*$ by
	\begin{equation*} 
		\varphi_1(g_1) \vcentcolon= \varphi(g_1, 1)
	\end{equation*}
	and similarly, $\varphi_2 : \widehat{G_1} \to \mathbb{C}^*$ by
	\begin{equation*} 
		\varphi_2(g_2) \vcentcolon= \varphi(1, g_2).
	\end{equation*}
	It is easy to see that each $\varphi_i$ is a homomorphism. That is, $\varphi_i \in \widehat{G_i}$ for $i = 1, 2.$ 

	Now, we define $\Phi:\widehat{G} \to \widehat{G_1} \times \widehat{G_2}$ as follows:	
	\begin{equation*} 
		\Phi(\varphi) = (\varphi_1, \varphi_2).
	\end{equation*}

	It is easy to verify that $\Phi$ is a homomorphism using the fact that
	\begin{equation*} 
		(\varphi\varphi')_i = \varphi_i\varphi'_i
	\end{equation*}
	for every $\varphi, \varphi' \in \widehat{G}$ and $i = 1, 2.$

	Moreover, if $\Phi(\varphi) = (1, 1),$ then $\varphi_1(g_1) = 1 = \varphi_2(g_2)$ for all $(g_1, g_2) \in G.$ Thus, $\varphi \equiv 1,$ showing that $\Phi$ is injective.

	To show surjectivity, let $(\rho, \rho') \in \widehat{G_1} \times \widehat{G_2}$ be arbitrary. Then, define $\varphi:G \to \mathbb{C}^*$ by
	\begin{equation*} 
		\varphi(g_1, g_2) = \rho(g_1)\rho'(g_2).
	\end{equation*}
	Note that
	\begin{align*} 
		\varphi(g_1g'_1, g_2g'_2) &= \rho(g_1g'_1)\rho'(g_2g'_2)\\
		&= \rho(g_1)\rho(g'_2)\rho'(g_2)\rho'(g'_2)\\
		&= \rho(g_1)\rho'(g_2)\rho(g'_2)\rho'(g'_2)\\
		&= \varphi(g_1, g_2)\varphi(g'_1, g'_2)
	\end{align*}
	and hence, $\varphi \in \widehat{G}.$ It is now easy to see that
	\begin{equation*} 
		\Phi(\varphi) = (\rho, \rho'),
	\end{equation*}
	proving surjectivity.	
\end{proof}

\begin{prop} \label{prop:ZnZconghatZnZ}
	If $G = \mathbb{Z}/n\mathbb{Z},$ then $G \cong \widehat{G}.$
\end{prop}
\begin{proof} 
	Note that we have the $n$ distinct homomorphisms $\varphi^{(0)}, \ldots, \varphi^{(n-1)}$ given by
	\begin{equation*} 
		\varphi^{(k)}([m]) = \omega_n^{km}.
	\end{equation*}
	(It can be verified easily that this is indeed a well-defined map and a homomorphism by using the fact that $\omega_n^n = 1$ and $\omega^a\omega^b = \omega^{a + b}.$)

	Moreover, these are the only homomorphisms since any homomorphism is uniquely determined once we map $[1]$ to an element, and that element is forced to be an $n$-th root of unity.

	This shows that $\md{\widehat{G}} = n.$ To show that it is cyclic, we simply observe that
	\begin{equation*} 
		\left(\varphi^{(1)}\right)^k = \varphi^{(k)}. \qedhere
	\end{equation*}
\end{proof}

\begin{cor} \label{cor:GconghatG}
	Let $G$ be a finite abelian group, then $G \cong \widehat{G}.$
\end{cor}
\begin{proof} 
	Using the structure theorem of finite abelian groups, we know that
	\begin{equation*} 
		G \cong G_1 \times \cdots \times G_n
	\end{equation*}
	for some finite cyclic groups $G_1, \ldots, G_n.$ From the previous two propositions, the result follows.
\end{proof}

\subsubsection{Sign of a permutation}

In the following, $e_i$ denotes the $i$-th standard basis vector of $\mathbb{C}^n.$

\begin{defn}[Matrix of a permutation]
	We define $M:S_n \to M_n(\mathbb{C})$ as follows:\\
	Given $\sigma \in S_n,$ we define $M(\sigma)$ to be the matrix representing the linear transformation determined by $e_i \mapsto e_{\sigma(i)}.$
\end{defn}

We immediately note that $M$ actually maps into $M_n(\mathbb{Z})$ since the $i$-th column of $M(\sigma)$ is simply $e_{\sigma(i)},$ i.e., all $0$s with a $1$ in the $i$-th place.

\begin{prop}[$M$ is multiplicative]
	Given $\sigma, \tau \in S_n,$ we have
	\begin{equation*} 
		M(\sigma\tau) = M(\sigma)M(\tau).
	\end{equation*}
\end{prop}
\begin{proof} 
	It suffices to those that the matrices on either side of the equation act the same way on each $e_i.$ To this end, note that
	\begin{align*} 
		M(\sigma\tau)e_i &= e_{(\sigma\tau)(i)}\\
		&= e_{\sigma(\tau(i))}\\
		&= M(\sigma)e_{\tau(i)}\\
		&= M(\sigma)M(\tau)e_i. \qedhere
	\end{align*}
\end{proof}

We now state corollaries of the above proposition.

\begin{cor}
	Given any $\sigma \in S_n,$ the matrix $M(\sigma)$ has determinant $\pm 1.$	In particular, each such matrix is invertible.
\end{cor}
\begin{proof} 
	Note $M(\id) = I$ and thus,
	\begin{equation*} 
		M(\sigma)M(\sigma^{-1}) = I,
	\end{equation*}
	by the above proposition.

	Since $M(\sigma)$ and $M(\sigma^{-1})$ have integer entries, their determinants are also integers. Taking $\det$ on both sides above yields the result.
\end{proof}

\begin{defn}[Sign of a permutation] \label{defn:signofperm}
	Define the function $\sign:S_n \to \{-1, 1\}$ as the following composition
	\begin{equation*} 
		S_n \overset{M}{\longrightarrow} M_n(\mathbb{C}) \overset{\det}{\longrightarrow}\{-1, 1\}.
	\end{equation*}
\end{defn}

By the above corollary, the above composition is well-defined.

\begin{cor} \label{cor:signisahomo}
	The $\sign$ function is a homomorphism from $S_n$ to $\{-1, 1\} = \mathbb{Z}^\times.$
\end{cor}
\begin{proof} 
	Follows from the above proposition and the fact that $\det$ is multiplicative.
\end{proof}

\begin{prop}[Sign in terms of transpositions]
	Let $\sigma \in S_n$ and suppose that we can write
	\begin{equation*} 
		\sigma = \tau_1\cdots\tau_n
	\end{equation*}
	for transpositions $\tau_i \in S_n.$

	Then, $\sign\sigma$ is $1$ iff $n$ is even.
\end{prop}
\begin{proof} 
	Let $\tau$ be a transposition, say $(ij).$ Then, $M(\tau)$ is the elementary row matrix that swaps the rows $i$ and $j.$ Thus, 
	\begin{equation*} 
		\sign(\tau) = \det M(\tau) = -1.
	\end{equation*} By the earlier proposition, it follows that
	\begin{equation*} 
		M(\sigma) = M(\tau_1)\cdots M(\tau_n)
	\end{equation*}
	and hence,
	\begin{equation*} 
		\sign(\sigma) = (-1)^n,
	\end{equation*}
	which immediately proves the result.
\end{proof}

\begin{cor}
	Given any decompositions of a permutation into transpositions, the parity of the number of transpositions is fixed.
\end{cor}

\begin{rem}
	The above way seems to have avoided all difficulties of showing that $\sign$ is well-defined by avoiding the definition in terms of transpositions. In fact, we get that as a corollary!\\
	It seems that the work has gone in the fact that $\det$ is multiplicative. Note that we are actually using this result from linear algebra (over fields, that is) and not necessarily that from ring theory.
\end{rem}

\subsubsection{Conjugacy classes of \texorpdfstring{$S_n$}{Sn}} \label{subsubsec:conjclassofSn}

\begin{prop}
	Let $\sigma, \tau \in S_n.$ Suppose that a disjoint cycle decomposition of $\sigma$ is given as
	\begin{equation*} 
		(a_1 \ldots a_{m_1})(a_{m_1 + 1} \ldots a_{m_2})\ldots(a_{m_{k-1} + 1} \ldots a_{m_k}),
	\end{equation*}
	where $\{a_1, \ldots, a_{m_k}\} = \{1, \ldots, n\}$ and $m_k = n.$ Then, the cycle decomposition of $\tau\sigma\tau^{-1}$ is given by
	\begin{equation*} 
		(\tau(a_1) \ldots \tau(a_{m_1}))(\tau(a_{m_1 + 1}) \ldots \tau(a_{m_2}))\ldots(\tau(a_{m_{k-1} + 1}) \ldots \tau(a_{m_k})).
	\end{equation*}
\end{prop}
\begin{proof} 
	Let 
	\begin{equation*} 
		\rho \vcentcolon= (\tau(a_1) \ldots \tau(a_{m_1}))(\tau(a_{m_1 + 1}) \ldots \tau(a_{m_2}))\ldots(\tau(a_{m_{k-1} + 1}) \ldots \tau(a_{m_k})).
	\end{equation*}
	We wish to show that $\tau\sigma\tau^{-1} = \rho.$ It suffices to show that $\tau\sigma = \rho\tau.$ To this end, let $i \in [n].$ Then, $i = a_{j}$ for some $j.$ 

	If $j$ is of the form $m_r,$ then (with $m_0 \vcentcolon= 0$)
	\begin{equation*} 
		\rho(\tau(i)) = \rho(\tau(a_{m_r})) = \tau(a_{m_{r-1} + 1}) = \tau(\sigma(a_{m_r})) = \tau(\sigma(i)).
	\end{equation*}
	Otherwise, we have
	\begin{equation*} 
		\rho(\tau(i)) = \rho(\tau(a_j)) = \tau(a_{j+1}) = \tau(\sigma(a_j)) = \tau(\sigma(i)),
	\end{equation*}
	completing the proof.
\end{proof}

\begin{cor}
	Any two conjugates have the same cycle type.
\end{cor}
\begin{proof} 
	Immediate.
\end{proof}

\begin{cor}
	If two permutations have the same cycle type, then they are conjugates.
\end{cor}
\begin{proof} 
	Let $\sigma$ and $\sigma'$ have the same cycle type. Then, we have write
	\begin{align*} 
		\sigma &= (a_1 \ldots a_{m_1})(a_{m_1 + 1} \ldots a_{m_2})\ldots(a_{m_{k-1} + 1} \ldots a_{m_k})\\
		\sigma' &= (b_1 \ldots b_{m_1})(b_{m_1 + 1} \ldots b_{m_2})\ldots(b_{m_{k-1} + 1} \ldots b_{m_k})
	\end{align*}
	Then, define $\tau:[n] \to [n]$ by
	\begin{equation*} 
		\tau(a_i) = b_i.
	\end{equation*}
	This defines a bijection since both $(a_1, \ldots, a_{m_k})$ and $(b_1, \ldots, b_{m_k})$ are permutations of $[n].$ By the earlier proposition, $\tau$ conjugates $\sigma$ to $\sigma'.$
\end{proof}

The above two corollaries put together gives us:
\begin{thm}[Description of conjugacy classes] \label{thm:descconjclassSn}
	The conjugacy classes of $S_n$ consist precisely of permutations of the same cycle type.
\end{thm}

\begin{rem}
	We have assumed that every permutation does have a (unique, up to ordering) disjoint cycle decomposition.
\end{rem}

\subsubsection{Group actions} \label{subsubsec:groupactions}

\begin{defn}%[Group action]
	An \deff{action} of a group $G$ on a (finite) set $X$ is a homomorphism $\sigma : G \to S_X.$ We often write $\sigma_g$ for $\sigma(g).$ The cardinality of $X$ is called the \deff{degree} of the action.

	For $g \in G$ and $x \in X,$ we often denote $\sigma_g(x)$ by $g \cdot x.$ 
\end{defn}

\begin{rem}
	We shall implicitly assume that $\md{X} \neq 2$ from hereon, even though the definition doesn't explicitly demand that. Note that $S_X$ would be the trivial group if $\md{X} = 0, 1$ and there isn't much to discuss about that.
\end{rem}

\begin{rem}
	In the more suggestive notation $g \cdot x$ for $\sigma_g(x),$ we get the following identities for all $g_1, g_2 \in G$ and $x \in X$:
	\begin{enumerate}
		\item $1 \cdot x = x,$
		\item $(g_1g_2) \cdot x = g_1 \cdot (g_2 \cdot x).$
	\end{enumerate}
	Both follow from the fact that $\sigma$ is a homomorphism and hence, $\sigma_1 = \id_X$ and $\sigma_{g_1g_2} = \sigma_{g_1} \circ \sigma_{g_2}.$
\end{rem}

\begin{rem}
	In fact, an alternate definition group action is a map $\cdot : G \times X \to X$ satisfying the above two properties. Note that a map $G \times X \to X$ can be seen as a map $f : G \to \Hom(X, X),$ where $\Hom(X, X)$ is the set of all \emph{functions} from $X$ to itself.

	This set is actually a monoid under the composition operation. In view of \Cref{rem:monoidhomo} and the properties of action, we actually see that $f$ maps into group of invertible elements. However, this is precisely $S_X.$
\end{rem}

% At this point, it would be educational to recall \Cref{rem:groupaction}.

\begin{defn}%[Orbit]
	Let $\sigma : G \to S_X$ be a group action. Then \deff{orbit} of $x \in X$ under $G$ is the set
	\begin{equation*} 
		G \cdot x = \{\sigma_g(x) \mid g \in G\} = \{g \cdot x \mid g \in G\}.
	\end{equation*}
\end{defn}

\begin{prop} \label{prop:orbitspartitionX}
	The orbits form a partition of $X.$
\end{prop}
\begin{proof} 
	We define the relation $\sim$ on $X$ by $x_1 \sim x_2$ iff there exists $g \in G$ such that $\sigma_g(x_1) = x_2.$

	From the definition, it is clear that $G \cdot x$ is simply the collection of all those $y \in X$ such that $x \sim y.$ Thus, to prove the proposition, it suffices to prove that $\sim$ is an equivalence relation. 

	\begin{enumerate}
		\item (Reflexive) Note that $1 \cdot x = x$ for all $x \in X.$
		\item (Symmetric) Note that $g \cdot x = y \implies x = g^{-1} \cdot y$ for all $x, y \in X$ and $g \in G.$
		\item (Transitive) Let $x, y, z \in X$ and $g_1, g_2 \in G$ be such that
		\begin{equation*} 
			g_1 \cdot x = y \andd g_2 \cdot y = z.
		\end{equation*}
		Then,
		\begin{equation*} 
			(g_2g_1) \cdot x = z. \qedhere
		\end{equation*}
	\end{enumerate}
\end{proof}

\begin{defn}%[Transitive]
	A group action $\sigma : G \to S_X$ is said to be \deff{transitive} if there is a unique orbit.
\end{defn}

\begin{rem}
	By the earlier proposition, it is clear that the above definition is equivalent to saying that given any $x, y \in X,$ there exists $g \in G$ such that $\sigma_g(x) = y.$
\end{rem}

\begin{ex}[Regular action] \label{ex:regularaction}
	Let $G$ be a group and consider $X = G.$ Then, $G$ acts on $X$ by left multiplication. That is, $\lambda : G \to S_X$ defined by
	\begin{equation*} 
		\lambda_g(x) = gx
	\end{equation*}
	for all $g, x \in G$ is a group action.

	This is a transitive action as can be easily verified.
\end{ex}

% \begin{rem}
% 	Recall \Cref{defn:regularrepresentation} from earlier. The ``Regular'' there and in the previous example are indeed related. This will be made more precise in the next subsection.
% \end{rem}

\begin{ex}[Coset action] \label{ex:cosetaction}
	Let $G$ be a group and $H$ a (not necessarily normal) subgroup. Let $G/H$ be the set of all \emph{left} cosets of $H.$ Then, $G$ acts on $G/H$ by left multiplication. That is, $\sigma : G \to S_{G/H}$ defined by
	\begin{equation*} 
		\sigma_g(C) = gC
	\end{equation*}
	for all $g \in G$ and $c \in G/H$ is a group action.

	Let us check this. Note that for $g_1, g_2 \in C,$ we have
	\begin{align*} 
		\sigma_{g_1g_2}(C) &= (g_1g_2)C\\
		&= \{g_1g_2c \mid c \in C\}\\
		&= g_1\{g_2c \mid c \in C\}\\
		&= g_1 (g_2 C)\\
		&= \sigma_{g_1}(\sigma_{g_2}(C)),
	\end{align*}
	proving that
	\begin{equation*} 
		\sigma_{g_1g_2} = \sigma_{g_1} \cdot \sigma_{g_2}.
	\end{equation*}
	Using the fact that $\sigma_1 = \id_{G/H},$ we see that each $\sigma_g$ is a bijection and that $\sigma$ is a homomorphism.

	Moreover, this a transitive. Indeed, given any two cosets $x_1H, x_2H \in G/H,$ we see that $g = x_2x_1^{-1}$ satisfies
	\begin{equation*} 
		\sigma_g(x_1H) = x_2H. 
	\end{equation*}
\end{ex}

\begin{defn}%[2-transitive]
	An action $\sigma : G \to S_X$ on $X$ is \deff{2-transitive} if given any two pairs of \underline{distinct} elements $(x, y) \in X^2$ and $(x', y') \in X^2,$ there exists $g \in G$ such that
	\begin{equation*} 
		\sigma_g(x) = x' \andd \sigma_g(y) = y'.
	\end{equation*}
\end{defn}
Note that the ``distinct'' above means that $x \neq y$ and $x' \neq y'.$ 

\begin{prop} \label{prop:2transistrans}
	A 2-transitive action is transitive.
\end{prop}
\begin{proof} 
	Let $x, y \in X$ be arbitrary. We wish to show that there exists $g \in G$ such that $\sigma_g(x) = y.$ Note that since $\sigma_1(x) = x,$ we may assume $x \neq y.$

	Put $(x', y') = (y, x).$ By 2-transitivity, there exists $g \in G$ such that 
	\begin{equation*} 
		\sigma_g(x) = x' = y,
	\end{equation*}
	as desired.
\end{proof}

\begin{ex}[Action of $D_4$]
	The converse of the above is not true. Consider the action of $D_4$ on the four vertices of a square. It is easy to see that his action is transitive. 

	Label the vertices $1, \ldots, 4.$ Any $g \in D_4$ takes opposite vertices to opposite vertices. Thus, considering the pairs
	\begin{equation*} 
		(x, y) = (1, 3) \andd (x', y') = (2, 3)
	\end{equation*}
	shows that the action is not 2-transitive.
\end{ex}

\begin{ex}[Action of symmetric groups] \label{ex:actsymgroups}
	As before, there's a natural action of $S_n$ on $X \vcentcolon= \{1, \ldots, n\}.$ 

	To be more explicit, we define $\tau \cdot i = \tau(i)$ for all $\tau \in S_n$ and $i \in X.$

	For $n \ge 2,$ this action is 2-transitive. Indeed, let $i \neq j$ and $i' \neq j'$ be elements in $X.$ Define 
	\begin{equation*} 
		Y_1 \vcentcolon= X \setminus \{i, j\} \andd Y_2 \vcentcolon= X \setminus \{i', j'\}.
	\end{equation*}
	Since $\md{Y_1} = \md{Y_2},$ there exists a bijection $\alpha : Y_1 \to Y_2.$ Define $\tau \in S_n$ by
	\begin{equation*} 
		\tau(k) = \begin{cases}
			i' & k = i,\\
			j' & k = j,\\
			\alpha(k) & \text{otherwise}.
		\end{cases}
	\end{equation*}
	The above is an element of $S_n$ precisely because $i \neq j$ and $i' \neq j'.$ Noting that $\tau \cdot i = i'$ and $\tau \cdot j = j'$ establishes that the action is 2-transitive.

	In terms of cycles, we can see that $\tau$ is simply $(ii')(jj'),$ assuming that all four are distinct. One can take different cases considering $i = i'$ and so on to explicitly get a cycle representation in each case.
\end{ex}

\begin{defn}%[Orbital]
	Let $\sigma : G \to S_X$ be a group action. Define $\sigma^2 : G \to S_{X \times X}$ by
	\begin{equation*} 
		\sigma_g^2(x_1, x_2) = (\sigma_g(x_1), \sigma_g(x_2)).
	\end{equation*}
	This is a group action of $G$ on $S \times S.$ An orbit of $\sigma^2$ is called an \deff{orbital} of $\sigma.$ The number of orbitals is called the \deff{rank} of $\sigma.$
\end{defn}

\begin{rem}
	Let $\Delta = \{(x, x) \mid x \in X\}.$ Note that 
	\begin{equation*} 
		\sigma_g^2(x, x) = (\sigma_g(x), \sigma_g(x)) \in \Delta.
	\end{equation*}
	That is $\Delta$ is \emph{closed} under the action of $\sigma^2.$ Moreover, $\Delta$ is an orbital iff $\sigma$ is transitive.
\end{rem}

\begin{rem}
	Note that $\sigma$ being 2-transitive is precisely the same as saying that
	\begin{equation*} 
		X^2 \setminus \Delta = \{(x, y) \in X \times X \mid x \neq y\}
	\end{equation*}
	is an orbital.
\end{rem}

\begin{prop} \label{prop:2transiffrank2}
	Let $\sigma : G \to S_X$ be a group action (with $\md{X} \ge 2$). Then, $\sigma$ is 2-transitive if and only if $\sigma$ is transitive with $\rank(\sigma) = 2.$
\end{prop}

\begin{proof} 
	Assume that $\sigma$ is 2-transitive. By \Cref{prop:2transistrans}, it follows that $\sigma$ is transitive. By the earlier remarks, we see that $\Delta$ and $X^2 \setminus \Delta$ are (distinct) orbitals. Since their union is $X^2,$ it follows that $\rank(\sigma) = 2.$

	Conversely, suppose that $\sigma$ is transitive and $\rank(\sigma) = 2.$ Since $\Delta$ is an orbital and orbitals partition $X \times X$ (\Cref{prop:orbitspartitionX}), it follows that $X^2 \setminus \Delta$ is the other orbital. As before, this is precisely saying that $\sigma$ is 2-transitive.
\end{proof}
	
In the above, we $\md{X} \ge 2$ was implicitly used in asserting that $X^2 \setminus \Delta$ is nonempty.

\begin{rem}
	The proof also shows that the rank is at least $2,$ whenever $\md{X} > 1,$ regardless of $\sigma$ being transitive.	
\end{rem}


\begin{ex}[Rank of $D_4$]
	As noted earlier, the action of $D_4$ on $\{1, \ldots, 4\}$ is not 2-transitive. Let us now compute the rank. Since the action is transitive, we know that $\Delta$ is an orbital.

	We now see how to partition $X^2 \setminus \Delta$ into orbitals. Note that if $(i, j) \in X^2 \setminus \Delta,$ then $i \neq j.$ There are precisely two distinct possibilities: Either $i$ and $j$ are adjacent, or $i$ and $j$ are opposite.

	It is easy to see that if $(i, j)$ and $(i', j')$ are both adjacent (resp. opposite) pairs of vertices, then they are in the same orbit. Moreover, as commented earlier, no opposite pair is in the orbit of any adjacent pair.

	Thus, there are precisely three orbitals:
	\begin{align*} 
		\Delta &= \{(x, y) \in X \times X \mid x = y\},\\
		\mathcal{O}_{\text{opp}} &= \{(x, y) \in X \times X \mid x - y \equiv 2 \bmod 4\},\\
		\mathcal{O}_{\text{adj}} &= \{(x, y) \in X \times X \mid x - y \equiv 1 \bmod 2\}.
	\end{align*}
\end{ex}

\begin{ex}[Rank of $S_n$]
	As noted in \Cref{ex:actsymgroups}, the (natural) action of $S_n$ is 2-transitive if $n \ge 2.$ Thus, it has rank $2.$
\end{ex}

\begin{defn}
	Let $\sigma : G \to S_X$ be a group action. For $g \in G,$ we define
	\begin{equation*} 
		\Fix(g) = \{x \in X \mid \sigma_g(x) = x\}
	\end{equation*}
	to be the set of \deff{fixed points} of $g.$ Let $\Fix^2(g)$ denote the set of fixed points of $g$ for the action $\sigma^2.$
\end{defn}

Note that $\Fix^2(g)$ could also possibly denote the Cartesian product of the set $\Fix(g)$ with itself. The following proposition states that this is unambiguous since the two are equal.

\begin{prop} \label{prop:fix2isfix2}
	Let $\sigma : G \to S_X$ be a group action. Then,
	\begin{equation*} 
		\Fix^2(g) = \Fix(g) \times \Fix(g).
	\end{equation*}
	In particular, $\md{\Fix^2(g)} = \md{\Fix(g)}^2.$
\end{prop}

\begin{proof} 
	Let $(x, y) \in X \times X.$ Then
	\begin{align*} 
		(x, y) \in \Fix^2(g) &\iff \sigma^2_g(x, y) = (x, y)\\
		& \iff (\sigma_g(x), \sigma_g(y)) = (x, y)\\
		& \iff \sigma_g(x) = x \andd \sigma_y(y) = y\\
		& \iff x \in \Fix(g) \andd y \in \Fix(g)\\
		& \iff (x, y) \in \Fix(g) \times \Fix(g). \qedhere
	\end{align*}
\end{proof}
% \subsection{Some more group actions}
% \begin{defn}
% 	Let $G$ act on a set $X$ and $x \in X$
% \end{defn}

\begin{defn}%[$G$-equivalence relation]
	\label{defn:leftcongruence}
	Let $X$ be a set and $\cdot : G \times X \to X$ an action. An equivalence relation $\sim$ on $X$ is said to be a \deff{$G$-equivalence relation} if $x \sim y$ implies $g \cdot x \sim g \cdot y$ for all $x, y \in X$ and $g \in G.$
\end{defn}

\begin{prop} \label{prop:leftcongruence}
	Let $G$ act on a set $X.$ Suppose that $\sim$ is an equivalence relation on $X$ which is a $G$-equivalence relation. Then, $G$ acts on $X/{\sim}$ with the action defined by $g \cdot [x] = [g \cdot x]$ for all $(g, x) \in G \times X.$
\end{prop}
\begin{proof} 
	The definition is well-defined precisely because $\sim$ is a $G$-equivalence relation. To see that it is indeed an action, note that 
	\begin{equation*} 
		1 \cdot [x] = [1 \cdot x] = [x]
	\end{equation*}
	and
	\begin{equation*} 
		g_1 \cdot (g_2 \cdot [x]) = g_1 \cdot [g_2 \cdot x] = [g_1 \cdot (g_2 \cdot x)] = [(g_1 g_2) \cdot x] = (g_1 g_2) \cdot [x]. \qedhere
	\end{equation*}
\end{proof}

\subsubsection{Double cosets} \label{subsubsec:doublecosets}

\begin{defn}%[Double cosets]
	\label{defn:doublecosets}
	Let $G$ be a group and $H, K$ be subgroups of $G.$ Then, the map $\sigma : H \times K \to S_G$ defined as
	\begin{equation*} 
		\sigma_{(h, k)}(g) = hgk^{-1}
	\end{equation*}
	is a group action. The orbit of $g$ under $H \times K$ is then the set
	\begin{equation*} 
		HgK \vcentcolon= \{hgk^{-1} \mid h \in H,\; k \in K\} = \{hgk \mid h \in H,\; k \in K\}
	\end{equation*}
	and is called a \deff{double coset} of $g.$ We write $\dcos{H}{G}{K}$ for the set of all double cosets of $H$ and $K$ in $G.$
\end{defn}

\begin{rem}
	Note that $H \times K$ is indeed a group. For this action, we only consider $G$ to be a \emph{set}. To see that this is a group action, we note that
	\begin{equation*} 
		\sigma_{(1, 1)}(g) = 1g1^{-1} = 1
	\end{equation*}
	and
	\begin{equation*} 
		\sigma_{(hh', kk')}(g) = (hh')g(kk'^{-1}) = h(h'gk'^{-1})k^{-1} = h\sigma_{(h', k')}(g)k^{-1} = \left(\sigma_{(h, k)} \circ \sigma_{(h', k')}\right)(g).
	\end{equation*}
\end{rem}

\begin{prop}
	Distinct double cosets are disjoint and $G$ is the union of all double cosets. In other words, the double cosets partition $G.$
\end{prop}
\begin{proof} 
	The double cosets are just the cosets under the action defined above. Thus, we are done by \Cref{prop:orbitspartitionX}.
\end{proof}

\begin{prop} \label{prop:dcosetsofnormal}
	Suppose $H \unlhd G.$ Then, $\dcos{H}{G}{H} = G/H.$
\end{prop}
\begin{proof} 
	We show that given $g \in G,$ we have $gH = HgH.$ This would prove the proposition.

	Note that $gH \subset HgH$ is clear since $1 \in H.$

	Conversely, note that
	\begin{equation*} 
		h_1gh_2 = g\underbrace{g^{-1}h_1g}_{\in H}h_2 \in gH. \qedhere
	\end{equation*}
\end{proof}

\subsection{Partitions and Tableaux} \label{subsec:partsandtableaux}
\begin{defn}%[Partition]
	\label{defn:partition}
	Let $n \in \mathbb{N}.$ A \deff{partition} of $n$ is a tuple $\lambda = (\lambda_1, \ldots, \lambda_l)$ of positive integers $\lambda_1 \ge \lambda_2 \ge \cdots \ge \lambda_l$ such that $\lambda_1 + \cdots + \lambda_l = n.$ We denote this by $\lambda \vdash n.$
\end{defn}

\begin{ex}
	$(4, 3, 2, 1),\;(5, 5),\;(10)$ are partitions of $10$ but $(1, 2, 3, 4)$ is not and neither is $(4, 2, 4).$
\end{ex}

\begin{defn}%[Cycle type]
	Given any permutation $\sigma \in S_n,$ we define the \deff{cycle type} $\type(\sigma)$ of $\sigma$ as 
	\begin{equation*} 
		\type(\sigma) = (\lambda_1, \ldots, \lambda_l),
	\end{equation*} 
	where $\lambda_1, \ldots, \lambda_l$ are the lengths of the disjoint cycles of $\sigma$ written in decreasing order, with multiplicity. We include the cycles of length $1$ as well.
\end{defn}
\begin{rem}
	Note that given any $\sigma \in S_n,$ we have $\type(\sigma) \vdash n;$ that is, the cycle type of $n$ gives a partition of $n.$ Conversely, given any partition $\lambda \vdash n,$ there exists a permutation $\sigma \in S_n$ such that $\type(\sigma) = \lambda.$
\end{rem}

\begin{ex}
	Let $n = 5$ and consider the cycle types of the following permutations.
	\begin{align*} 
		\type(\id) &= (1, 1, 1, 1, 1)\\
		\type\left((12)(345)\right) &= (3, 2)\\
		\type\left((12)\right) &= (2, 1, 1, 1)\\
		\type\left((12)(13)\right) &= (3, 1, 1)\\
		\type\left((12345)\right) &= (5).
	\end{align*}
	Note that for the fourth one, we must first convert $(12)(13)$ to have a disjoint cycle representation. This is done by noting that $(12)(13) = (132).$
\end{ex}

\begin{defn}%[Young diagram]
	\label{defn:youngdiag}
	If $\lambda = (\lambda_1, \ldots, \lambda_l)$ is a partition of $n,$ then the \deff{Young diagram} (or simply, the \deff{diagram}) of $\lambda$ consists of $n$ boxes placed into $l$ rows where the $i$-th row has $\lambda_i$ boxes.
\end{defn}

\begin{ex}
	For the partition $(3, 1)$ of $4,$ we have the diagram as \ydiagram{3, 1}.
\end{ex}

% \begin{defn}%[Conjugate partition]
% 	\label{defn:conjpartition}
% 	If $\lambda \vdash n,$ then the \deff{conjugate partition} $\lambda^\mathsf{T}$ of $\lambda$ is the partition whose Young diagram is the transpose of the diagram of $\lambda.$
% \end{defn}

% \begin{ex}
% 	For $\lambda = (3, 1)$ as before, the transpose of its diagram is the diagram
% 	\begin{equation*} 
% 		\ydiagram{2, 1, 1,}
% 	\end{equation*}
% 	and thus, $\lambda^\mathsf{T} = (2, 1, 1).$
% \end{ex}

We now put an order on the partitions of $n.$
\begin{defn}%[Domination order]
	\label{defn:domorder}
	Suppose that $\lambda = (\lambda_1, \ldots, \lambda_l)$ and $\mu = (\mu_1, \ldots, \mu_m)$ are partitions of $n.$ Then, $\lambda$ is said to \deff{dominate} $\mu$ if
	\begin{equation*} 
		\lambda_1 + \cdots + \lambda_i \ge \mu_1 + \cdots + \mu_i
	\end{equation*}
	for all $i \ge 1.$ Here, we set $\lambda_i = 0$ if $i > l$ and $\mu_i = 0$ if $i > m.$

	We denote this by $\lambda \unrhd \mu.$
\end{defn}

\begin{rem}
	The above definition is simply saying that for all $i,$ the number of blocks in the first $i$ rows of the diagram of $\lambda$ is at least that in the first $i$ rows of the diagram of $\mu.$
\end{rem}

\begin{ex}
	$(5, 1) \unrhd (3, 3)$ since $5 \ge 3$ and $5 + 1 \ge 3 + 3.$

	However, neither of $(3, 3, 1) \unrhd (4, 1, 1, 1)$ or $(4, 1, 1, 1) \unrhd (3, 3, 1)$ is true. Indeed, note that
	\begin{equation*} 
		\lambda = (3, 3, 1) = \ydiagram{3, 3, 1} \andd \mu = (4, 1, 1, 1) = \ydiagram{4, 1, 1, 1}.
	\end{equation*}
	If we consider the first row, then $\mu$ has more boxes than $\lambda.$ However, if we look at the first two rows, then the situation is reversed.
\end{ex}
\begin{ex}
	\begin{equation*} 
		\ytableausetup{centertableaux}
		\ydiagram{4} \unrhd \ydiagram{3, 1} \unrhd \ydiagram{2, 2} \unrhd \ydiagram{2, 1, 1,} \unrhd \ydiagram{1, 1, 1, 1}
	\end{equation*}
\end{ex}
\begin{ex}
	Note that the partitions $(1, \ldots, 1) \vdash n$ and $(n)$ are the minimum and maximum elements of the poset. That is, given any partition $\lambda \vdash n,$ one has
	\begin{equation*} 
		(n) \unrhd \lambda \unrhd (1, \ldots, 1).
	\end{equation*}
\end{ex}

\begin{prop}
	Let $\lambda, \mu, \rho$ be any partitions of $n.$ Then:
	\begin{enumerate}
		\item (Reflexivity) $\lambda \unrhd \lambda,$
		\item (Anti-symmetry) $\lambda \unrhd \mu$ and $\mu \unrhd \lambda$ implies $\lambda = \mu,$
		\item (Transitivity) $\lambda \unrhd \mu$ and $\mu \unrhd \rho$ implies $\lambda \unrhd \rho.$
	\end{enumerate}
	In other words, the set of all partitions of $n$ along with $\unrhd$ forms a \emph{poset}.
\end{prop}
\begin{proof} 
	Reflexivity and transitivity are obvious. We prove anti-symmetry. Suppose $\lambda = (\lambda_1, \ldots, \lambda_l)$ and $\mu = (\mu_1, \ldots, \mu_m)$ are partitions of $n$ such that $\lambda \unrhd \mu$ and $\mu \unrhd \lambda.$ \\
	Without loss of generality, we may assume that $l \le m.$ Note that we have
	\begin{equation*} 
		\lambda_1 + \cdots + \lambda_i \le \mu_1 + \cdots + \mu_i \le \lambda_1 + \cdots + \lambda_i
	\end{equation*}
	for all $i$ and hence, we have the following \emph{equality} for all $i:$
	\begin{equation*} 
		\lambda_1 + \cdots + \lambda_i = \mu_1 + \cdots + \mu_i.
	\end{equation*}
	Successively putting $i = 1, \ldots, l$ shows that $\lambda_k = \mu_k$ for $k = 1, \ldots, l.$ Now, since
	\begin{equation*} 
		\lambda_1 + \cdots + \lambda_l = n = \mu_1 + \cdots + \mu_l + \cdots + \mu_m,
	\end{equation*}
	we see that $m = l.$ (Since each $m_k$ is positive.) Thus, we have $\lambda = \mu.$
\end{proof}

\begin{defn}%[Young tableau]
	\label{defn:youngtableau}
	If $\lambda \vdash n,$ then a \deff{$\lambda$-tableau}\footnotemark (or \deff{Young tableau of shape $\lambda$}) is an array $t$ of integers by placing $1, \ldots, n$ into the boxes of the Young diagram for $\lambda.$
\end{defn}
\footnotetext{plural: tableaux}
Given a $\lambda \vdash n,$ there are $n!$ $\lambda$-tableaux. In fact, there is a natural action of $S_n$ on the set of all $\lambda$-tableaux.

\begin{defn}%[]
	For $n \in \mathbb{N}$ and $\lambda \vdash n,$ we see that $S_n$ acts transitively on the set of all $\lambda$-tableaux. The action of $\sigma \in S_n$ on $t$ is given by applying $\sigma$ to all the elements of $t.$ This tableau is denoted by $\sigma t.$
\end{defn}

\begin{ex}
	Given $\lambda = (3, 2, 1),$ the following are few (of the $720$ many) $\lambda$-tableaux:
	\begin{equation*} 
		\begin{ytableau}
			1 & 2 & 3 \\
			4 & 5  \\
			6
		\end{ytableau},\;\;
		\begin{ytableau}
			3 & 1 & 2 \\
			6 & 5  \\
			4
		\end{ytableau},\;\;
		\begin{ytableau}
			3 & 2 & 6 \\
			1 & 4  \\
			5
		\end{ytableau}.
	\end{equation*}
\end{ex}

\begin{prop} \label{prop:domprop}
	Let $\lambda = (\lambda_1, \ldots, \lambda_l)$ and $\mu = (\mu_1, \ldots, \mu_m)$ be partitions of $n.$ Suppose that $t^\lambda$ is a $\lambda$-tableau and $s^\mu$ is a $\mu$-tableau such that if two entries are in the same row of $s^\mu,$ then they are in different columns of $t^\lambda.$ In such a case, there exists a $\lambda$-tableau $u^\lambda$ such that:
	\begin{enumerate}[label = (\alph*)]
		\item \label{item:008} The $j$-th columns of $t^\lambda$ and $u^\lambda$ contain the same elements for $1 \le j \le l;$
		\item \label{item:009} The entries of the first $i$ rows of $s^\mu$ belong to the first $i$ rows of $u^\lambda$ for each $1 \le i \le m.$
	\end{enumerate}
	In particular, $l \le m.$
\end{prop}

\begin{ex}
	Let us look at an example of what the proposition is really saying. Consider
	\begin{equation*} 
		t^\lambda = \begin{ytableau}
			8 & 5 & 4 & 2 & 7\\
			1 & 3 \\
			6
		\end{ytableau}
		\andd
		s^\mu = \begin{ytableau}
			1 & 2 & 3 & 4\\
			5 & 6\\
			7 & 8
		\end{ytableau}.
	\end{equation*}
	Note that each of $1, \ldots, 4$ appear in different columns in $t^\lambda.$ The same is true for $5, 6$ and $7, 8.$ Thus, the tableaux satisfy the hypothesis of the proposition.

	Can we take $u^\lambda = t^\lambda?$ No, the problem is that $1, 3$ appear in the first row of $s^\mu$ but not in the first row of $t^\lambda.$ To remedy this, we may swap $(1, 8)$ and $(3, 5)$ in $t^\lambda$ to get:
	\begin{equation*} 
		t_1^\lambda = \begin{ytableau}
			1 & 3 & 4 & 2 & 7\\
			8 & 5 \\
			6
		\end{ytableau}.
	\end{equation*}
	Note that since we only swapped within columns, \ref{item:008} is maintained.\\
	Can we now take $t_1^\lambda$ as $u^\lambda?$ The answer is still ``no.'' The first row is all good but note that $6$ appears in the first two rows of $s^\mu$ but not of $t_1^\lambda.$ Thus, we swap $(6, 8)$ to get

	\begin{equation*} 
		t_2^\lambda = \begin{ytableau}
			1 & 3 & 4 & 2 & 7\\
			6 & 5 \\
			8
		\end{ytableau}.
	\end{equation*}
	Can we now take $t_2^\lambda = u^\lambda?$ The answer is now ``yes.'' \\
	The condition \ref{item:008} is satisfied as can be seen by noting that that each column of $u^\lambda$ is obtained as a permutation of the corresponding column of $t^\lambda.$

	That the condition \ref{item:009} is satisfied can also be observed by simply checking each row.
\end{ex}
In fact, the proof of the proposition is simply giving an algorithm on constructing the $u^\lambda$ following steps similar to the ones above. As an exercise, the reader can try showing that for the same $\mu$ as in the above example and for $\lambda = (3, 3, 2),$ one cannot find $t^\lambda$ and $s^\mu$ which satisfy the hypothesis of the proposition.

\begin{proof} 
	For each $1 \le r \le m,$ we construct a $\lambda$-tableau $t_r^\lambda$ such that:
	\begin{enumerate}[label = (\alph*')]
		\item \label{item:010} The $j$-th column of $t^\lambda$ and $t_r^\lambda$ contain the same elements for $1 \le j \le \lambda_1;$
		\item \label{item:011} The entries of the first $i$ rows of $s^\mu$ belong to the first $i$ rows of $t_r^\lambda$ for each $1 \le i \le r.$
	\end{enumerate}
	Note that taking $u^\lambda = t_m^\lambda$ would then prove the proposition. The construction is by induction on $r.$ Set $t_0^\lambda \vcentcolon= t^\lambda.$ \\
	Suppose that $t_0^\lambda, \ldots, t^r_\lambda$ have been constructed where $0 \le r \le m - 1.$ We define $t_{r+1}^\lambda$ as follows:

	For each $k$ in the $(r + 1)$-th row of $s^\mu,$ let $c(k)$ be the column of $t_r^\lambda$ in which $k$ appears. Note that by the hypothesis and \ref{item:010}, it follows that if $k \neq k',$ then $c(k) \neq c(k').$\footnote{Note that $s^\mu$ is fixed throughout the process.}\\
	Now, if $k$ already appears in the first $r + 1$ rows of $t_r^\lambda,$ then we do nothing. Thus, let us assume that $k$ does not appear in the first $r + 1$ rows of $t_r^\lambda.$ From this, it follows that $c(k)$ intersects row $r + 1$ of $t_r^\lambda.$\\
	
	\begin{blockquote}
		To see why $c(k)$ must intersect row $r + 1:$ note that if $c(k)$ does not intersect row $r + 1,$ then $c(k)$ cannot intersect any later row either. (The sizes of the rows are non-increasing.) This means that all elements of $c(k)$ are in the first $r$ rows. However, we know that $k$ is an element of $c(k)$ not in the first $r$ rows.
	\end{blockquote}

	Now, we simply swap $k$ with the element appearing in the intersection of $c(k)$ with row $r + 1.$ This preserves \ref{item:010} since we are only permuting within the same column. This also preserves \ref{item:011} since the first $r$ rows are left unchanged. \\
	Moreover, note that since different $k$ correspond to different $c(k),$ the order in which we do the swap does not matter and the previous changes are unaffected. Thus, we get \ref{item:011} for $t^\lambda_{r + 1}$ as well.

	This finishes the construction. The final statement follows from the fact that the numbers $1$ through $n$ all appear in the first $m$ rows of $s^\mu$ and hence, of $t^\lambda.$ Thus, $t^\lambda$ cannot have any more rows.
\end{proof}
\begin{rem}
	It is possible that $l < m$ in the above. Indeed, consider
	\begin{equation*} 
		u^\lambda = \ydiagram{2} \andd \mu^s = \ydiagram{1, 1}
	\end{equation*}
	filled in any manner.
\end{rem}

\begin{lem}[Dominance lemma] \label{lem:domlemma}
	Let $\lambda$ and $\mu$ be partitions of $\mu$ and suppose that $t^\lambda$ and $s^\mu$ are tableaux of the respective partitions. Further suppose that integers in the same row of $s^\mu$ are located in different columns of $t^\lambda.$ Then, $\lambda \unrhd \mu.$
\end{lem}
\begin{proof} 
	Let $u^\lambda$ be as in \Cref{prop:domprop}. Let $\lambda = (\lambda_1, \ldots, \lambda_l)$ and $\mu = (\mu_1, \ldots, \mu_m).$ Note that for each $i,$ the number $\lambda_1 + \cdots + \lambda_i$ denotes the number of boxes in the first $i$ rows of $\lambda.$ (The same is true for $\mu$ instead of $\lambda$ as well.)\\
	However, since the numbers in the first $i$ rows of $s^\mu$ appear in $u^\lambda,$ we see that
	\begin{equation*} 
		\lambda_1 + \cdots + \lambda_i \ge \mu_1 + \cdots + \mu_i
	\end{equation*}
	for all $i \ge 1.$ Thus, $\lambda \unrhd \mu.$
\end{proof}

\subsection{Number Theory} \label{subsec:numbertheory}
\begin{defn}%[Algebraic integer]
	\label{defn:alginteger}
	A complex number $\alpha$ is said to be an \deff{algebraic integer} if it is a root of a \emph{monic} polynomial with integer coefficients. In other words, there exists $n > 0$ and integers $a_0, \ldots, a_{n - 1}$ such that
	\begin{equation*} 
		\alpha^n + a_{n - 1}\alpha^{n - 1} + \cdots + a_0 = 0.
	\end{equation*}
	The set of all algebraic integers is denoted by $\mathbb{A}.$
\end{defn}

\begin{ex}[Non-example]
	Note that the ``monic'' makes an important difference. For example, $\frac{1}{2}$ is a root of the polynomial $2z - 1$ but it is in fact not an algebraic integer. (\Cref{prop:rationalalgintareint}.)
\end{ex}

\begin{ex}[Integers]
	Any integer $m$ is trivially an algebraic integer since it is a root of the monic $z - m.$
\end{ex}

\begin{ex}[$n$-th roots] \label{ex:nrootsalgint}
	More generally, given any $m \in \mathbb{Z}$ and $n \in \mathbb{N},$ any $n$-th root of $m$ is an algebraic integer, in view of the polynomial $z^n - m.$ Thus, each $\omega_m^k$ is an algebraic integer and so is $\sqrt[n]{2}.$ Moreover, so is $\sqrt[3]{2}\omega_3.$ As we shall see later, product of two algebraic integers is always an algebraic integer.
\end{ex}

\begin{ex}[Eigenvalues of integer matrices] \label{ex:eigenintmatrix}
	Let $A = (a_{ij})$ be a matrix with integer entries. Then, the eigenvalues of $A$ are precisely the roots of $\det(zI - A).$ Now, $\det(zI - A)$ is a monic polynomial in $z$ and each coefficient is an integer because each $a_{ij}$ is so. Thus, each eigenvalue is an algebraic integer.
\end{ex}

\begin{ex}[Additive inverse of algebraic integers]
	Let $\alpha \in \mathbb{C}$ be an algebraic integer and let $p(z)$ be a monic integer polynomial of degree $n$ such that $p(\alpha) = 0.$ Then, $q(z) \vcentcolon= (-1)^np(-z)$ is again a monic integer polynomial. Moreover, $q(-\alpha) = (-1)^np(\alpha) = 0.$

	Thus, $\mathbb{A}$ is closed under additive inverses.
\end{ex}

\begin{ex}[Conjugate of algebraic integers]
	Let $\alpha \in \mathbb{C}$ be an algebraic integer and let $p(z)$ be a monic integer polynomial such that $p(\alpha) = 0.$ Since $p(z)$ is a real polynomial, we see that $p(\overline{\alpha}) = 0.$

	Thus, $\mathbb{A}$ is closed under conjugation.
\end{ex}

\begin{prop} \label{prop:rationalalgintareint}
	If $\alpha \in \mathbb{Q}$ is an algebraic integer, then $\alpha \in \mathbb{Z}.$ In other words, the rational algebraic integers are precisely the integers.
\end{prop}
\begin{proof} 
	Let $r = p/q \in \mathbb{Q}$ be algebraic with $p \in \mathbb{Z}, q \in \mathbb{N}$ and $\gcd(p, q) = 1.$ Since $r$ is algebraic, there exist $a_0, \ldots, a_{n-1}$ such that
	\begin{equation*} 
		r^n + a_{n - 1}r^{n - 1} + \cdots + r_0 = 0.
	\end{equation*}
	Multiplying the above with $q^n,$ we get
	\begin{equation*} 
		p^n + a_{n - 1}p^{n - 1}q + \cdots + r_0q^n = 0.
	\end{equation*}
	Clearly, every term except for the first is divisible by $q.$ Thus, so is the first term. That is, $q \mid p^n.$ Since $\gcd(p, q) = 1$ and $q > 0,$ we get $q = 1.$ Thus, $r = p \in \mathbb{Z},$ as desired.
\end{proof}

We would like to show that $\mathbb{A}$ is closed under sums and products. Since $0, 1 \in \mathbb{A}$ and $\mathbb{A}$ is closed under inverses, this would show that $\mathbb{A}$ is a subring of $\mathbb{C}.$ For that, we see an alternate characterisation of elements of $\mathbb{A}.$

\begin{prop} \label{prop:characalgint}
	An element $y \in \mathbb{C}$ is an algebraic integer if and only if there exist $y_1, \ldots, y_t \in \mathbb{C}$ not all zero and a $t \times t$ integer matrix $A$ such that
	\begin{equation*} 
		\begin{bmatrix}
			yy_1\\
			yy_2\\
			\vdots\\
			yy_t
		\end{bmatrix} = A\begin{bmatrix}
			y_1\\
			y_2\\
			\vdots\\
			y_t
		\end{bmatrix}.
	\end{equation*}
	In other words, each $yy_i$ is an integral linear combination of the $y_j.$
\end{prop}
\begin{proof} 
	$(\implies)$ Suppose that $y \in \mathbb{A}.$ Let $y$ be a root of 
	\begin{equation*} 
		p(z) = z^t + a_{t - 1}y^{t - 1} + \cdots + a_0.
	\end{equation*}
	Thus, we have $y^t = -a_{t - 1}y^{t - 1} - \cdots - a_0.$ Putting $y_i = y^{i - 1}$ for $1 \le i \le n$ gives
	\begin{equation*} 
		\begin{bmatrix}
			yy_1\\
			yy_2\\
			\vdots\\
			yy_{t - 1}\\
			yy_t
		\end{bmatrix} = \begin{bmatrix}
			y^1\\
			y^2\\
			\vdots\\
			y^{t - 1}\\
			y^t
		\end{bmatrix} = \begin{bmatrix}
			0 & 1 & 0 & \cdots & 0 & 0\\
			0 & 0 & 1 & \cdots & 0 & 0\\
			\vdots & \vdots & \vdots & \ddots & \vdots & \vdots\\
			0 & 0 & 0 & \cdots & 0 & 1\\
			-a_0 & -a_1 & -a_2 & \cdots & -a_{t - 2} & -a_{t - 1}
		\end{bmatrix}\begin{bmatrix}
			1\\
			y^1\\
			\vdots\\
			y^{t - 2}\\
			y^{t - 1}
		\end{bmatrix}
	\end{equation*}

	$(\impliedby)$ Let $y_1, \ldots, y_t$ and $A$ be as in the statement. Define 
	\begin{equation*} 
		Y = \begin{bmatrix}
			y_1\\
			y_2\\
			\vdots\\
			y_t
		\end{bmatrix} \in \mathbb{C}^t.
	\end{equation*}
	Then, we have
	\begin{equation*} 
		AY = yY,
	\end{equation*}
	by assumption. Since $y_1, \ldots, y_t$ are not all zero, we see $Y \neq 0.$ Thus, $Y$ is an eigenvector with eigenvalue $y.$ By \Cref{ex:eigenintmatrix}, $y$ is an algebraic integer.
\end{proof}	

\begin{prop} \label{prop:algintsubring}
	The set $\mathbb{A}$ of algebraic integers is a subring of $\mathbb{C}.$ In other words, $0 \in \mathbb{A}$ and if $\alpha, \beta \in \mathbb{A},$ then $\alpha \pm \beta, \alpha\beta \in \mathbb{A}.$
\end{prop}
\begin{proof} 
	We have already noted that $0 \in \mathbb{A}$ and that it is closed under (additive) inverses. Thus, we only need to show that it is closed under sums and products.

	Let $\alpha, \beta \in \mathbb{A}.$ Corresponding to each, we get $y_1, \ldots, y_t \in \mathbb{C}$ not all zero and $y'_1, \ldots, y'_s \in \mathbb{C}$ not all zero such that
	\begin{equation*} 
		\alpha y_i = \sum_{j = 1}^{t}a_{ij}y_j \andd \beta y'_k = \sum_{j = 1}^{s}b_{kj}y'_j.
	\end{equation*}
	(The above equalities hold for all $1 \le i \le t$ and $1 \le k \le s.$ Each $a_{ij}$ and $b_{kj}$ is an integer.) 

	Now, we consider the elements $\{y_iy'_k : 1 \le i \le t, 1 \le k \le s\}.$ These are not all zero. Moreover, the above equation gives
	\begin{equation*} 
		(\alpha + \beta)y_iy'_k = \alpha y_iy'_k + \beta y'_ky_i = \sum_{j = 1}^{t}a_{ij}y_jy_k + \sum_{j = 1}^{s}b_{kj}y_iy'_j.
	\end{equation*}
	Thus, each $(\alpha + \beta)y_iy'_k$ is an integral linear combination of $y_jy'_l.$ This implies that $\alpha + \beta$ is an algebraic integer.

	Similarly, $\alpha\beta$ is written as an integral linear combination of the $y_jy'_l,$ finishing the proof.
\end{proof}

\section{Group representations} \label{sec:01}

\subsection{Definition and Examples}
\begin{defn}%[Representation]
	A \deff{representation} of a group $G$ is a homomorphism $\varphi:G\to \GL(V)$ for some (finite-dimensional) vector space $V.$ The dimension of $V$ is called the \emph{degree} of $\varphi.$ We write $\varphi_g$ for $\varphi(g)$ and $\varphi_g(v)$ or simply $\varphi_gv,$ for the action of $\varphi_g \in \GL(V)$ on $v \in V.$
\end{defn}

\begin{rem}
	We shall implicitly assume that $V \neq 0$ from hereon, even though the definition doesn't explicitly demand that. Note that $\GL(V)$ would be the trivial group if $V = 0$ and there isn't much to discuss about that.
\end{rem}

\begin{rem}
	Since representation are homomorphisms, if a group $G$ is generated by $X,$ then a representation $\varphi$ of $G$ is determined by its values on $X.$ Of course, one must keep in mind that not every assignment of values to $X$ will actually determine a homomorphism. 
\end{rem}

\begin{rem} \label{rem:groupaction}
	Recall the group $S_{X}$ which is the group of all bijections from $X$ to itself. If we consider a vector space $V,$ we see that $\GL(V)$ is a subgroup of $S_V.$ Recall from basic algebra, the concept of group actions. One may define it ($G$ acting on a set $X$) as a certain map satisfying some properties but one sees that it was simply equivalent to a group homomorphism $\varphi:G \to S_X.$ 

	In this sense, we see that representations are special group actions where we don't just want $\varphi_g$ to be \emph{bijections} but also to be \emph{linear}.
\end{rem}

\begin{ex}
	Recall from \Cref{subsec:linearisation}, the concept of \nameref{defn:linearisation}. Given a set $X,$ we can construct the $\mathbb{C}$-vector space $\mathbb{C}X$ with $X$ as a basis.

	Now given a group $G$ which acts on $X,$ we saw in \Cref{prop:extendingactiontorep} that the action can actually be extended to an action on $\mathbb{C}X.$ Moreover, it has the property that each element acts linearly, i.e., we get a representation.
\end{ex}

Note that if $V$ is a $\mathbb{C}$ vector space of dimension $1,$ then $V \cong \mathbb{C}$ and $\GL(V) \cong \mathbb{C}^*.$ For psychological reasons, we may sometimes use $z$ instead of $\varphi$ for a degree one representation, to remind us that $\varphi_g$ is simply multiplication by a complex number.

\begin{ex}[Trivial representation] \label{ex:trivialrepresentation}
	The trivial representation of a group $G$ is the homomorphism $z:G \to \mathbb{C}^*$ given by $z_g = 1$ for all $g \in G.$ % amnote: shouldn't it be \GL(V) instead of C^* and \varphi(g) = Id for all g?

	This is a degree one representation.
\end{ex}

\begin{ex}[Degree one representations of $\mathbb{Z}/n\mathbb{Z}$] \label{ex:ZnZCstardeg1}
	Given $n \in \mathbb{N},$ note that a homomorphism $z:\mathbb{Z}/n\mathbb{Z} \to \mathbb{C}^*$ is determined by mapping $[1]$ to an element $\zeta \in \mathbb{C}^*$ such that $\zeta^n = 1.$ Thus, we must map $[1]$ to an $n$-th root of unity. One such example of a representation is
	\begin{equation*} 
		z([m]) \vcentcolon= \omega_n^m.
	\end{equation*}
	This is again a degree one representation. Another such is
	\begin{equation*} 
		z([m]) \vcentcolon= \omega_n^{-m}
	\end{equation*}	
	In fact, for each $k = 0, \ldots, n - 1,$ we get a different degree one representation given by
	\begin{equation*} 
		z^{(k)}([m]) = \omega_n^{mk}.
	\end{equation*}
	We will eventually show that the above are the ``only'' representations (in some sense) of a finite cyclic group.

	On the other hand, for $\mathbb{Z},$ we note that giving a homomorphism $z:\mathbb{Z}\to\mathbb{C}^*$ is the same as giving an element of $\mathbb{C}^*.$ Thus, we have uncountably many distinct representations of $\mathbb{Z}.$ Moreover, if $z_1$ is not some $n$-root of unity, then $z$ will be a injection of $\mathbb{Z}$ into $\mathbb{C}^*.$
\end{ex}

Soon, we shall define a concept of ``equivalence'' of representations. We shall then show that all the representations mentioned above are actually inequivalent.

\begin{rem} \label{rem:finrepintoS1}
	Note that if $G$ is a finite group and $z$ a degree one representation, then $z:G\to\mathbb{C}^*$ is actually very restrictive. Note that we must have $z_g^{\md{G}} = 1$ and thus, $\varphi$ actually maps into\footnote{``into'' does not mean ``injectively''} the unit circle. In fact, it further maps into the subgroup which is the $\md{G}$-th roots of unity.
\end{rem}

\begin{ex}[Degree one representations of $S_n$] \label{ex:degonerepsSn}
	Note that we already have two obvious degree one representations of $S_n.$ The first is the trivial one and the second is the $\sign$ homomorphism mapping each permutation to its sign. (Recall the definition \nameref{defn:signofperm} and the fact it is a homomorphism, \Cref{cor:signisahomo}.)

	We now show that these are all. Firstly, note that all transpositions are conjugates. (Recall \nameref{thm:descconjclassSn}.) Hence, if the kernel of a representation contains one transposition, it must contain all. (Since kernels are normal.) 

	Also note that transpositions generate $S_n.$ Thus, any homomorphism is completely determined by its values on the transpositions. 

	Now, noting that any transposition has order $2,$ we see that it can only mapped to $\pm 1.$ If a single transposition is mapped to $1,$ then all are; this gives us the trivial representation. Thus, if the representation is non-trivial, then each transposition is mapped to $-1.$ However, then it must agree with $\sign.$ 
\end{ex}

\begin{ex}[Degree one representations of non-abelian groups] \label{ex:deg1factorthrough}
	Let $G$ be a group and let $z : G \to \mathbb{C}^*$ be a degree one representation. Noting that $\mathbb{C}^*$ is abelian, we see that the $\ker z$ must contain the (normal) commutator subgroup $[G, G].$ Thus, it must factor through the quotient as follows:

	\begin{center}
		\begin{tikzcd}
		G \arrow[dd, two heads] \arrow[rr, "z"]           &  & \mathbb{C}^* \\
		                                                        &  &\\
		{G/[G, G]} \arrow[rruu, "\widetilde{z}"', dashed] &  &             
		\end{tikzcd}
	\end{center}

	In other words, it then suffices to study the degree one representations of the abelian group $G/[G, G].$
\end{ex}

\begin{ex}[Determining conjugacy using representations]
	Note that since $\varphi$ is a homomorphism, the (images of the) relations that hold in $G$ must also hold in $\GL(V).$ In particular, some relations which are easier to solve in $\GL(V)$ may help in solving those in $G.$

	To give a specific example, consider the problem of having $x, y \in G$ and wanting to find $g \in G$ such that $gxg^{-1} = y.$ A priori, there may not even be a way of deducing whether such a $g$ exists. However, considering a representation $\varphi,$ we can try to solve
	\begin{equation*} 
		\varphi_g\varphi_x\varphi_g^{-1} = \varphi_y.
	\end{equation*}
	Since similarity of matrices is better solvable (at least, in theory for up to degree four representations, we can find all eigenvalues of $\varphi_x, \varphi_y$) by Jordan form, we can hope to get a better answer. If $\varphi_x$ and $\varphi_y$ are \emph{not} similar (even consideration of trace or determinant could possibly tell us that), we know for a fact that such a $g$ cannot exist.

	Now, if we see that they are similar, we can find a matrix $M$ such that $M\varphi_xM^{-1} = \varphi_y.$ Then, elements in the set $\varphi^{-1}(M)$ are good candidates for $g.$ (Of course, it could be possible that that set is empty or that none of them actually work.)
\end{ex}

We now make the following observation:\\
Let $\varphi : G \to \GL(V)$ be a representation of degree $n.$\\
Now, suppose that we have two bases $B, B'$ of $V.$ Corresponding to these, we get two isomorphisms
\begin{equation*} 
	T : V \to \mathbb{C}^n \andd S : V \to \mathbb{C}^n
\end{equation*}
mapping the basis elements to the standard basis vectors of $\mathbb{C}^n.$ Now, we have two representations
\begin{equation*} 
	\psi : \GL (V) \to \GL (\mathbb{C}^n) \andd \psi' : \GL (V) \to \GL (\mathbb{C}^n)
\end{equation*}
obtained by setting
\begin{equation*} 
	\psi_g \vcentcolon= T\varphi_gT^{-1} \andd \psi'_g \vcentcolon= S\varphi_gS^{-1}.
\end{equation*} 
Note that $\psi$ and $\psi'$ are related to each other by
\begin{equation*} 
	\psi_g' = (ST^{-1})\psi_g(ST^{-1})^{-1}.
\end{equation*}
This resembles a ``change of basis'' and we would wish for $\varphi, \psi, \psi'$ to be considered as the ``same'' representation. To this end, we have the following definition.

\begin{defn}%[Equivalence]
	Two representations $\varphi : G \to \GL(V)$ and $\psi : G \to \GL(W)$ are said to be \deff{equivalent} if there exists an isomorphism (called an \deff{equivalence}) $T : V \to W$ such that 
	\begin{equation*} 
		\psi_g = T\varphi_g T^{-1}
	\end{equation*}
	\underline{for all $g \in G.$} In such a case, we write $\varphi\sim\psi.$ Note that the above definition is saying that the following diagram commutes
	\begin{center}
		\begin{tikzcd}
			{V} \arrow[rr, "\varphi_g"]\arrow[dd, "T"'] & & {V}\arrow[dd, "T"]\\
			& & \\
			{W} \arrow[rr, "\psi_g"'] & & {W}
		\end{tikzcd}
	\end{center}
	\underline{for all $g \in G.$}
\end{defn}
The above equivalence can easily checked to be an actual ``equivalence relation.''
\begin{rem}
	Note that in the above, we have not assumed $V = W.$ However, $V$ and $W$ must be \emph{isomorphic}. In particular, $\varphi$ and $\psi$ have the same degree.
\end{rem}

\begin{ex}[Equivalent degree two representations of $\mathbb{Z}/n\mathbb{Z}$] \label{ex:ZnZGL2Cequiv}
	Define $\varphi, \psi:\mathbb{Z}/n\mathbb{Z} \to GL_2(\mathbb{C})$ by
	\begin{equation*} 
		\varphi_{[m]} = \begin{bmatrix}
			\cos\left(\dfrac{2\pi m}{n}\right) & -\sin\left(\dfrac{2\pi m}{n}\right)\\
			\sin\left(\dfrac{2\pi m}{n}\right) & \cos\left(\dfrac{2\pi m}{n}\right)
		\end{bmatrix} 
		\andd \psi_{[m]} = \begin{bmatrix}
			\omega_n^m & 0\\
			0 & \omega_n^{-m}
		\end{bmatrix}.
	\end{equation*}
	Then, we have the automorphism of $\GL_2(\mathbb{C})$ induced by 
	\begin{equation*} 
		A = \two{\iota}{-\iota}{1}{1}.
	\end{equation*}
	We have
	\begin{equation*} 
		A^{-1} = \dfrac{1}{2\iota}\two{1}{\iota}{-1}{\iota}
	\end{equation*}
	and a direct computation shows that
	\begin{equation*} 
		A^{-1}\varphi_{[m]}A = \psi_{[m]}
	\end{equation*}
	for all $[m] \in \mathbb{Z}/n\mathbb{Z}.$ Thus, we have $\varphi\sim\psi.$
\end{ex}

\begin{prop} \label{prop:distinctdeg1inequiv}
	Let $G$ be a group and $z, z':G \to \mathbb{C}^*$ be degree one representations. Then, $z \sim z'$ iff $z = z'.$
\end{prop}
\begin{proof} 
	Clearly, equality implies equivalence. We show the converse.

	Let $T : \mathbb{C} \to \mathbb{C}$ be an isomorphism such that $z'_g = Tz_gT^{-1}$ for all $g \in G.$ Then, for any $v \in \mathbb{C},$ we have
	\[\begin{WithArrows}[displaystyle]
		z'_g(v) &= T\left(z_gT^{-1}(v)\right) \Arrow{$T$ is linear} \\
		&= z_g T\left(T^{-1}(v)\right)\\
		&= z_g v.
	\end{WithArrows}\]
	Thus, $z_g = z'_g$ for all $g \in G$ and hence, $z = z'.$
\end{proof}

\begin{cor}
	All the distinct representations in \Cref{ex:ZnZCstardeg1} are actually inequivalent.
\end{cor}
\begin{cor}
	There are exactly $n$ distinct inequivalent degree one representations of $\mathbb{Z}/n\mathbb{Z}.$
\end{cor}
\begin{proof} 
	We had constructed $n$ distinct (and hence, inequivalent) representations in \Cref{ex:ZnZCstardeg1}. 

	To see that these are all, note that $\varphi$ is completely determined once we define $\varphi_{[1]}.$ Moreover, $\varphi_{[1]}^n$ must be $1,$ since $\varphi$ is a homomorphism. That is, $\varphi_{[1]}$ must be an $n$-th root of unity. Thus, we have at most $n$ homomorphisms.
\end{proof}

In fact, the above can be generalised to all finite abelian groups.
\begin{cor} \label{cor:deg1repsfinabel}
	Let $G$ be a finite abelian group. There are exactly $\md{G}$ inequivalent degree one representations of $G.$
\end{cor}
\begin{proof} 
	By \Cref{cor:GconghatG}, there are exactly $\md{G}$ homomorphisms from $G$ to $\mathbb{C}^*.$ By \Cref{prop:distinctdeg1inequiv}, these are all inequivalent as well and we are done.
\end{proof}

\begin{ex}[Degree one representations of non-abelian groups revisited]
	As noted in \Cref{ex:deg1factorthrough}, if we wish to study degree one representations of $G,$ it suffices to study those of the \emph{abelian} group $G' = G/[G, G].$ Now, if $G/[G, G]$ is finite (which is certainly the case if $G$ is finite), then the above corollary tells us that there are exactly $\md{G'}$ many such representations. In fact, \Cref{cor:GconghatG} actually tells us the description of the representations as well. In fact, there's something more that we know which we isolate as a corollary.
\end{ex}

\begin{cor} \label{cor:numdegoneirrepsdivG}
	Let $G$ be a finite group. Then, the number of (distinct, inequivalent) degree one representations of $G$ is $\md{G/[G, G]}.$ In particular, the number divides $\md{G}.$
\end{cor}
\begin{proof} 
	As noted in \Cref{ex:deg1factorthrough}, any degree one representation factors through $G/[G, G]$ and all degree one representations are obtained in precisely this way.

	Thus, the number of degree one representations of $G$ is that of $G/[G, G].$ Since $G/[G, G]$ is abelian, there are exactly $\md{G/[G, G]}$ many such. This number divides $\md{G},$ by elementary group theory.
\end{proof}

\begin{ex}[Standard representation of $S_n$] \label{ex:standardrepSn}
	We define $\varphi:S_n \to \GL_n(\mathbb{C})$ as follows:\\
	Given $\sigma \in S_n,$ we define $\varphi_\sigma$ to be the matrix representing the linear transformation determined by $e_i \mapsto e_{\sigma(i)}.$

	One checks easily that this is indeed a homomorphism. One can verify that the matrix $\varphi_\sigma$ is explicitly given by permuting the columns of the identity matrix according to $\sigma.$ To be more explicit, the $i$-th columns of $\varphi_\sigma$ will be the $\sigma(i)$-th column of the identity matrix. This is because we wish to map $e_i$ to $e_{\sigma(i)}.$ 

	As an example, for $n = 3,$ we have
	\begin{equation*} 
		\varphi_{(123)} = \begin{bmatrix}
			& & 1 \\
			1 & & \\
			 & 1 &
		\end{bmatrix}.
	\end{equation*}
\end{ex}

For the above example, note that
\[\begin{WithArrows}[displaystyle]
	\varphi_\sigma(e_1 + \cdots + e_n) &= \varphi_\sigma(e_1) + \cdots + \varphi_\sigma(e_n)\\
	&= e_{\sigma(1)} + \cdots + e_{\sigma(n)} \Arrow{since $\sigma$ is a permutation}\\
	&= e_{1} + \cdots + e_{n}
\end{WithArrows}\]

Thus, the subspace $\mathbb{C}(e_1 + \cdots + e_n)$ is $\varphi_\sigma$-invariant \underline{for all $\sigma \in S_n$}. This motivates the following definition.

\begin{defn}%[$G$-invariant subspace]
	Let $\varphi : G \to \GL(V)$ be a representation. A subspace $W \le V$ is said to be \deff{$G$-invariant} if, for all $g \in G$ and $w \in W,$ we have $\varphi_g(w) \in W.$
\end{defn}
% manote: does this mean that we can consider \vaprhi:G \to \GL(W) as some representation?

\begin{rem}
	Note that the invariance depends on $G$ as well the representation being considered. To emphasise on the representation at times, we may add ``with respect to $\varphi$.''
\end{rem}

\begin{prop}
	If $W \le V$ is a $G$-invariant subspace with respect to $\varphi:G\to \GL(V),$ then $\varphi|_W: G \to \GL(W)$ by setting $(\varphi|_W)_g(w) = \varphi_g(w)$ for $w \in W$ is a representation.
\end{prop}
\begin{proof} 
	We first show that the $\varphi|_W$ so defined actually maps into $\GL(W).$ By the hypothesis that $W$ is $G$-invariant, we get that
	\begin{equation*} 
		\varphi_g(w) \in W
	\end{equation*}
	for all $w \in W$ and thus, $\varphi_g$ restricts to a function from $W$ to $W,$ for all $g \in G.$\\
	The fact that this is linear follows from the fact that $\varphi$ was a representation to begin with. Moreover, it is invertible since $\varphi_{g^{-1}}$ also restricts from $W$ to $W$ and thus, we have that $\varphi_g \in \GL(W)$ for all $g \in G.$ This shows that $\varphi|_W$ is actually a function from $G$ to $\GL(W).$

	The fact that it is a homomorphism follows from the fact that $\varphi$ was one to begin with.
\end{proof}

\begin{defn}%[Subrepresentation]
	Let $\varphi:G\to \GL(V)$ be a representation. If $W \le V$ is a $G$-invariant subspace, then $\varphi|_W : G \to \GL(W)$ is again a representation and we call $\varphi|_W$ is a \deff{subrepresentation} of $\varphi.$
\end{defn}

Going back to \Cref{ex:ZnZGL2Cequiv}, it is easy to note that $\mathbb{C}e_1$ and $\mathbb{C}e_2$ are $\mathbb{Z}/n\mathbb{Z}$-invariant subspaces with respect to $\psi$ (and not $\varphi$!). Moreover, we have that $\mathbb{C}^2 = \mathbb{C}e_1 \oplus \mathbb{C}e_2.$ This motivates the following.

\begin{defn}%[Direct sum of representations]
	Let $\varphi^{(1)}:G \to \GL(V_1)$ and $\varphi^{(2)}:G \to \GL(V_2)$ be representations. Then, their (external) \deff{direct sum} is the representation
	\begin{equation*} 
		\varphi^{(1)}\oplus\varphi^{(2)} : G \to \GL(V_1 \oplus V_2)
	\end{equation*}
	given by
	\begin{equation*} 
		\left(\varphi^{(1)}\oplus\varphi^{(2)}\right)_g(v_1, v_2) = \left(\varphi^{(1)}_g(v_1), \varphi^{(2)}_g(v_2)\right)
	\end{equation*}
	for all $g \in G$ and for all $(v_1, v_2) \in V_1 \oplus V_2.$
\end{defn}
Note that in the above, we are using the tuple notation for representing the (external) direct sum of the vector spaces $V_1$ and $V_2.$ This sum can be visualised naturally in terms of matrices.\\
If $\varphi^{(1)}:G\to \GL_n(\mathbb{C})$ and $\varphi^{(2)}:G \to \GL_m(\mathbb{C})$ are representations, then each $\varphi^{(i)}_g$ is a matrix. Then, the matrix $\left(\varphi^{(1)} \oplus \varphi^{(2)}\right)_g \in \GL_{n + m}(\mathbb{C})$ is given as the block matrix
\begin{equation*} 
	\left(\varphi^{(1)} \oplus \varphi^{(2)}\right)_g = \two{\varphi_g^{(1)}}{}{}{\varphi_g^{(2)}}.
\end{equation*}
(The empty places are $0$ matrices of appropriate sizes.)

\begin{ex}
	The two representations from \Cref{ex:ZnZCstardeg1} have their (external) direct sum as $\psi$ from \Cref{ex:ZnZGL2Cequiv}.

	A slightly less direct (but simple) calculation shows that 
	\begin{equation*} 
		\mathbb{C}\col{\iota}{1} \andd \mathbb{C}\col{-\iota}{1}
	\end{equation*}
	are also $\mathbb{Z}/n\mathbb{Z}$-invariant subspaces with respect to $\varphi.$ (Note that are just the columns of $A$ in the example, multiplied by a scalar.)

	Thus, $\varphi$ can also be written as a sum of subrepresentations. This should not be surprising as one would expect that equivalent representations behave similarly in this aspect. This will be made more precise and proven at the end of this section.
\end{ex}

\begin{ex} \label{ex:directsumoftrivialreps}
	The representation $\rho: G \to \GL_n(\mathbb{C})$ given by $\rho_g = I_n$ for all $g \in G$ is equivalent to the direct sum of $n$ copies of the \nameref{ex:trivialrepresentation}. Note that if $n > 1,$ then it is \emph{not} equivalent to the trivial representation since the degrees are different.
\end{ex}

\begin{ex} \label{ex:S3GL2Crho}
	Let $\rho:S_3 \to \GL_2(\mathbb{C})$ be specified on the generators by
	\begin{equation*} 
		\rho_{(12)} = \two{-1}{-1}{0}{1}, \quad \rho_{(123)} = \two{-1}{-1}{1}{0}.
	\end{equation*}
	(It must be checked that this defined a representation. We do this at the end.) 

	Let $\psi:S_3 \to \mathbb{C}^* \cong \GL_1(\mathbb{C})$ be the trivial representation, i.e., $\psi_g = 1.$ Then, we have the representation $\rho \oplus \psi$ which is specified on the generators by
	\begin{equation*} 
		(\rho \oplus \psi)_{(12)} = \begin{bmatrix}
			-1 & 1 & \\
			0 & 1 & \\
			& & 1
		\end{bmatrix}, \quad (\rho \oplus \psi)_{(123)} = \begin{bmatrix}
			-1 & 1 & \\
			1 & 0 & \\
			& & 1
		\end{bmatrix}.
	\end{equation*}

	We shall see later that $\rho\oplus\psi$ is equivalent to the standard representation as considered in \Cref{ex:standardrepSn}. % amnote: simultaneous diagonalisability
	
	\hrulefill
	
	To check that $\rho$ actually gives us a representation (group homomorphism), we must verify that the matrices satisfy the relations that the generators satisfy. That is,
	\begin{equation*} 
		\rho_{(12)}^2 = I_2,\;\rho_{(123)}^3 = I_2,\; \rho_{(12)}\rho_{(123)} = \rho_{(123)}^2\rho_{(12)}.
	\end{equation*}
	(We are using the fact from group theory that the above relations completely determine $S_3.$)

	One can compute and see that the above relations do hold. % amnote: is this the way to do this?
\end{ex}

\begin{prop} \label{prop:Ginvariantdirectsum}
	If $V_1, V_2 \le V$ are $G$-invariant subspaces with respect to $\varphi$ and $V = V_1 \oplus V_2,$ then $\varphi$ is equivalent to the (external) direct sum $\varphi|_{V_1} \oplus \varphi|_{V_2}.$
\end{prop}
\begin{proof} 
	Let $T:V \to V_1 \oplus V_2$ be the natural map $v_1 + v_2 \mapsto (v_1, v_2).$ (Here we are considering the external direct sum of the vector spaces.)\\
	This map is well-defined and an isomorphism because $V$ is the (internal) direct sum of $V_1$ and $V_2.$

	Now, put $\psi = \varphi|_{V_1} \oplus \varphi|_{V_2}.$ Then, for any $g \in G,$ we have
	\[\begin{WithArrows}[displaystyle]
		\psi_g(v_1, v_2) &= \left(\left(\varphi|_{V_1}\right)_g(v_1), \left(\varphi|_{V_2}\right)_g(v_2)\right)\\
		&= \left(\varphi_g(v_1), \varphi_g(v_2)\right) \Arrow{since $V_1, V_2$ are $G$-invariant} \\
		&= T(\varphi_g(v_1) + \varphi_g(v_2)) \Arrow{$\varphi_g$ is a linear map} \\
		&= T(\varphi_g(v_1 + v_2))\\
		&= T(\varphi_g(T^{-1}(v_1, v_2))),
	\end{WithArrows}\]
	showing that
	\begin{equation*} 
		\psi_g = T\varphi_gT^{-1},
	\end{equation*}
	as desired.	
\end{proof}

The above can also be visualised in terms of matrices. Let $B_i$ be a basis for $V_i.$ Then, $B \vcentcolon= B_1 \cup B_2$ is a basis for $V$ (since $V$ is the internal direct sum of the $V_i$). Since $V_i$ is $G$-invariant, we see that $\varphi_g(B_i) \subset \mathbb{C}B_i.$ Thus, the matrix representation with respect to $B$ is as follows:
\begin{equation*} 
	[\varphi_g]_B = \two{\left[\varphi^{(1)}\right]_{B_1}}{}{}{\left[\varphi^{(2)}\right]_{B_2}}.
\end{equation*}

One who has studied algebra would be familiar with the idea of breaking down structures into simpler ``irreducibles'' (similar to the prime factorisation of an integer). To such a reader, the following definition should not come as a surprise.

\begin{defn}%[Irreducible representation] % amnote: non-degree zero?
	A \underline{non-zero} representation $\varphi:G\to \GL(V)$ of a group $G$ is said to be \deff{irreducible} if the only $G$-invariant subspaces of $V$ are $0$ and $V.$
\end{defn}
In the above, $0$ refers to the $0$ \emph{subspace}, i.e., $\{0\}.$

\begin{ex}
	Any degree one representation is irreducible since there is no non-trivial proper subspace of a dimension $1$ vector space.
\end{ex}

\begin{ex}
	If $G = \{1\},$ the trivial group, then the only irreducible representation is a degree one representation. (In other words, the converse of the previous example holds too.)

	Indeed, the only representation $\varphi:G\to \GL(V)$ is $\varphi_1 = I$ and thus, every subspace of $V$ is a $G$-invariant subspace.
\end{ex}

\begin{rem}
	Note that in the above case, we actually have that the representation is actually a direct sum of subrepresentations. However, irreducibility does not demand that. The reader can see this happening in \Cref{ex:Zredbutnotdecomposable}. % manote

	However, we will soon show that the above is actually true when the group is finite.
\end{rem}

\begin{ex}[Irreducible representations of dihedral type groups]
	Let $G$ be a finite group with generators $a$ and $b.$ (By hypothesis, $a$ and $b$ have finite order.) Suppose further that every element of $G$ can be written as $a^ib^j$ for some non-negative integers $i$ and $j.$ (Note that given any $g \in G,$ $g^{-1}$ can be written as $a^ib^j$ and hence, $g = b^{-j}a^{-i}.$ Using that $a$ and $b$ have finite orders, we can actually write every element of $G$ as $b^{j'}a^{i'}$ for non-negative integers as well.)

	By the parenthetical remark, we can assume without loss of generality that $\md{a} \le \md{b}.$ ($\md{g}$ denotes the order of $g \in G.$)

	Let $n \vcentcolon= \md{a}.$ We show that any irreducible representation of $G$ has order at most $n.$ 

	To this end, let $\varphi : G \to \GL(V)$ be an irreducible representation and let $v$ be an eigenvector of $\varphi_b.$ Consider the following subspace $W$ of $V$ given by
	\begin{equation*} 
		W \vcentcolon= \langle v, \varphi_av, \ldots, \varphi_{a^{n-1}}v\rangle.
	\end{equation*}
	Clearly, $0 < \dim W \le n.$ We show that $W$ is $G$-invariant. Then, since $\varphi$ is irreducible, it would follow that $V = W,$ proving our claim.

	By hypothesis, an arbitrary element of $G$ can be written as $a^ib^j.$ Pick an arbitrary element of the spanning set given above for $W.$ It is of the form $\varphi_{a^k}v$ for some $0 \le k \le n - 1.$ It suffices to show that 
	\begin{equation*} 
		\varphi_{a^ib^j}\left(\varphi_{a^k}v\right) \in W.
	\end{equation*}
	Note that, by hypothesis, $a^ib^ja^k = a^pb^q$ for some non-negative integers $p$ and $q.$ Since $n = \md{a},$ we may assume that $p < n.$ Since $\varphi$ is a homomorphism, we get that
	\begin{equation*} 
		\varphi_{a^ib^j}\left(\varphi_{a^k}v\right) = \varphi_{a^p}\left(\varphi_{b^q}v\right).
	\end{equation*}
	Since $v$ is an eigenvector of $\varphi_b,$ we see that $\varphi_{b^q}v$ is some linear multiple of $v,$ say $\lambda v.$ Then, the right side of the above equation becomes
	\begin{equation*} 
		\varphi_{a^p}\left(\varphi_{b^q}v\right) = \varphi_{a^p}\left(\lambda v\right) = \lambda \varphi_{a^p}(v) \in W,
	\end{equation*}
	as desired.
\end{ex}

\begin{ex}[Irreducible representations of dihedral groups] \label{ex:irredrepDndegbound}
	Consider the dihedral group $D_n$ with $r$ denoting a rotation and $s$ a reflection. Then, the hypothesis of the previous example applies with $a = s$ and hence, $n = 2.$ This tells us that every irreducible representation of $D_n$ has degree at most two.
\end{ex}

\begin{thm} \label{thm:ontogrouphomogivesirredrep}
	Let $\rho : H \to \GL(V)$ be an irreducible representation of $H$ and $\psi : G \to H$ be an onto group homomorphism. Then,
	\begin{equation*} 
		\rho \circ \psi : G \to \GL(V)
	\end{equation*}
	is an irreducible representation of $G.$
\end{thm}
\begin{proof} 
	Let $\varphi \vcentcolon= \rho \circ \psi.$ It is clear that this is a representation, being the composition of group homomorphisms. We now show that it is irreducible.

	Let $W \le V$ be a $G$-invariant subspace (with respect to $\varphi$). We show that $W$ is also $H$-invariant (with respect to $\rho$) and conclude.\footnote{One should also check that it is non-zero but that is clear.}

	This is simple for if $w \in W$ and $h \in H,$ then we get $h = \psi(g)$ for some $g \in G.$ We then note that
	\begin{equation*} 
		\rho_h(w) = \rho(h)(w) = \rho(\psi(g))(w) = (\rho \circ \psi)(g)(w) = \varphi(g)(w) = \varphi_g(w) \in W. \qedhere
	\end{equation*}
\end{proof}

We now see when are degree two representations irreducible.	

\begin{prop} \label{prop:deg2repirreducible}
	If $\varphi:G\to \GL(V)$ is a degree two representation, then $\varphi$ is irreducible if and only if there is no common eigenvector $v$ to all $\varphi_g$ with $g \in G.$
\end{prop}

% amnote: this will generalise to degree 3 reps also once we know a complementary subspace exists

\begin{proof} 
	One direction is easy. Suppose that $v \in V$ is such that $v$ is an eigenvector of $\varphi_g$ for all $g \in G.$ In that case, $\mathbb{C}v$ is a one dimensional $G$-invariant subspace of $V$ and hence, is proper and non-trivial. (Recall that eigenvectors are non-zero, by definition.)

	Now, suppose the converse. Let $W$ be a proper non-trivial $G$-invariant subspace of $V.$ Then, $W = \mathbb{C}v$ for some $0 \neq v \in V.$ Then, given any $g \in G,$ we have that
	\begin{equation*} 
		\varphi_gv \in W
	\end{equation*}
	and hence, $\varphi_gv = \lambda_g v$ for some $\lambda_g \in \mathbb{C}.$ This shows that $v$ is an eigenvector for all $\varphi_g.$ (Since it was non-zero to begin with.)
\end{proof}

\begin{rem}
	For finite groups, the above proposition can also be generalised to degree three representations, using an almost identical proof. The only extra ingredient required is that if a representation of a finite group is reducible, then we can actually write $V = W \oplus W'$ for non-zero $G$-invariant subspaces.

	For infinite groups, the above proposition does not generalise to degree three representation, as seen in \Cref{ex:deg3repredbutnocommoneigen}.

	The above does not generalise to degree four representations, even in the case of finite groups. This is seen in \Cref{ex:deg4repredbutnocommoneigen}.
\end{rem}

\begin{ex} \label{ex:showingS3GL2Crhoisirred}
	The representation $\rho:S_3 \to \GL_2(\mathbb{C})$ in \Cref{ex:S3GL2Crho} is irreducible. 

	We show this by showing that no eigenvector of $\rho_{(12)}$ is also an eigenvector of $\rho_{(123)}.$ (That is, they have no common eigenvectors.) Then, we are done, by the above proposition.

	To this end, we first compute the eigenvalues of $\rho_{(12)}$ to be $\pm 1.$ Corresponding to these, we get the eigenvectors $\col{1}{0}$ and $\col{-1}{2}.$ (Note that since the eigenvalues are distinct, any other eigenvector must be a scalar multiple (as opposed to a linear combination) of either of these.)

	A direct computation gives us that neither is an eigenvector of $\rho_{(123)}.$ Indeed, we have
	\begin{equation*} 
		\rho_{(123)}\col{1}{0} = \col{-1}{1} \andd \rho_{(123)}\col{-1}{2} = \col{-1}{-1}.
	\end{equation*}
\end{ex}

\begin{ex}[An irreducible representation of $D_4$] \label{ex:D4irreddeg2}
	Consider the group $D_4.$ Let $r$ be rotation by $\pi/2$ and $s$ be a reflection about a perpendicular bisector of a side. We know that
	\begin{equation*} 
		D_4 = \langle r, s \mid r^4 = s^2 = rsrs^{-1} = 1\rangle.
	\end{equation*}
	Using the above, we see that the following is a representation:
	\begin{equation*} 
		\varphi(r) \vcentcolon= \two{\iota}{}{}{-\iota} \andd \varphi(s) \vcentcolon= \two{}{1}{1}{}.
	\end{equation*}
	Clearly, the only eigenvectors of $\varphi(r)$ (up to scaling) are $e_1$ and $e_2,$ neither of which is an eigenvector of $\varphi(s).$ Thus, $\varphi$ is irreducible.
\end{ex}

\begin{ex} \label{ex:deg3repredbutnocommoneigen}
	We now show that \Cref{prop:deg2repirreducible} is not true for degree three representations when the group is infinite. \\
	Let $G \vcentcolon= F(a, b)$ be the free group on two generators $a$ and $b.$ (Recall that a homomorphism from $G$ to any group is defined uniquely by specifying its values on $a$ and $b.$)

	Consider the representation $\varphi : G \to \GL_3(\mathbb{C})$ defined by
	\begin{equation*} 
		\varphi_a \vcentcolon= \begin{bmatrix}
			1 &   &  \\
			  & 2 &  \\
			  &   & 3
		\end{bmatrix} \andd \varphi_b \vcentcolon= \begin{bmatrix}
			  & 1 & 1\\
			1 &   & 1\\
			  &   & 1
		\end{bmatrix}.
	\end{equation*} 
	Note that $\varphi_a$ and $\varphi_b$ are indeed elements of $\GL_3(\mathbb{C})$ as can be checked by noting that they both have nonzero determinant. Thus, the above defines a representation.

	\textbf{Claim 1.} $\varphi_a$ and $\varphi_b$ have no common eigenvector. In particular, there is no $v \in \mathbb{C}^3$ which is a common eigenvector for all $\left\{\varphi_g\right\}_{g \in G}.$

	\begin{proof} 
		This is simple for the only eigenvectors of $\varphi_a$ (up to scaling) are $e_1,$ $e_2,$ and $e_3.$ Clearly, none of them is an eigenvector of $\varphi_b.$
	\end{proof}

	\textbf{Claim 2.} $W = \mathbb{C}\{e_1, e_2\}$ is a $G$-invariant subspace.
	\begin{proof} 
		Clearly, $W$ is $\varphi_a$ and $\varphi_b$-invariant. By \Cref{prop:Tinverseinvariance}, it follows that it is also $\varphi_{a^{-1}} = \left(\varphi_a\right)^{-1}$ and $\varphi_{b^{-1}}$ invariant.

		Since any element $g \in G$ is a product of positive powers of $a, b, a^{-1}, b^{-1},$ it follows that $W$ is $\varphi_g$-invariant, by \Cref{prop:STinvariance}.
	\end{proof}

	Thus, we have an example of a degree three representation which is reducible but there's no common eigenvector.
\end{ex}

Similar to irreducible representations, we define some terms which the reader should find natural.

\begin{defn}%[Completely reducible]
	Let $G$ be a group. A representation $\varphi:G\to \GL(V)$ is said to \deff{completely reducible} if $V = V_1 \oplus \cdots \oplus V_n$ where $V_i$ is $G$-invariant and $\varphi|_{V_i}$ irreducible for each $i = 1, \ldots, n.$
\end{defn}

\begin{rem}
	In view of \Cref{prop:Ginvariantdirectsum}, $\varphi$ is completely reducible is equivalent to saying that $\varphi\sim\varphi^{(1)}\oplus\cdots\oplus\varphi^{(n)}$ for some irreducible representations $\varphi^{(i)}.$
\end{rem}

\begin{rem}
	As funny as it may seem, an irreducible representation \emph{is} completely reducible. We did not demand for the $V_i$s to be proper subspaces of $V.$
\end{rem}

The above is similar to a sort of prime factorisation or diagonalisation. Our eventual goal is to show that any representation of a finite group is completely reducible. Thus, one can then just study irreducible representations.

\begin{defn}%[Decomposable representation]
	A \underline{non-zero} representation $\varphi$ is said to be \deff{decomposable} if $V = V_1 \oplus V_2$ for some \underline{non-zero} $G$-invariant subspaces $V_1, V_2 \le V.$ Otherwise, $V$ is said to be \deff{indecomposable}.
\end{defn}

Note that the above is, a priori, stronger than saying that $\varphi$ is irreducible. However, we shall see later that the two coincide for when $G$ is finite.

We now wish to show irreducible, completely reducible, and decomposability are actually notions that depend on the equivalence class of the representation. To this end, we first prove the following lemma.

\begin{lem} \label{lem:isopreservesinv}
	Let $\varphi:G \to \GL(V)$ and $\psi:G \to \GL(W)$ be equivalent representations and let $T: V \to W$ be an isomorphism such that the desired diagram commutes. If $V_1 \le V$ is $G$-invariant, then so is $W_1 \vcentcolon= T(V_1) \le W.$
\end{lem}
\begin{proof} 
	Let $w \in W_1$ and let $g \in G.$ Then, we have
	\begin{equation*} 
		\psi_g = T\varphi_gT^{-1}.
	\end{equation*}
	Note that $T^{-1}w \in V_1$ and thus, $\varphi_gT^{-1}w \in V_1$ since $V_1$ is $T$-invariant. In turn, we get that
	\begin{equation*} 
		\psi_gw = T\varphi_gT^{-1} \in T(V_1) = W_1,
	\end{equation*}
	as desired.
\end{proof}

For the following three propositions, let $\varphi:G \to \GL(V)$ and $\psi:G \to \GL(W)$ be equivalent representations and let $T: V \to W$ be an isomorphism such that the desired diagram commutes.

\begin{prop} \label{prop:irreducibleequiv}
	$\psi$ is irreducible if $\varphi$ is so.
\end{prop}
\begin{proof} 
	Let $V_1 \le V$ be a $G$-invariant subspace which is non-zero and proper. Then, $W_1 \vcentcolon= T(V_1)$ is non-zero and proper since $T$ is an isomorphism. By \Cref{lem:isopreservesinv}, this is also $G$-invariant and we are done.
\end{proof}

\begin{prop} \label{prop:decomposableequiv}
	$\psi$ is decomposable if $\varphi$ is so.
\end{prop}
\begin{proof} 
	If $V = V_1 \oplus V_2$ for non-zero subspaces, then $W = T(V_1) \oplus T(V_2)$ (with $T(V_1) \neq 0 \neq T(V_2)$) since $T$ is an isomorphism. If $V_1, V_2$ are $G$-invariant, then so are $T(V_1)$ and $T(V_2),$ by \Cref{lem:isopreservesinv}.
\end{proof}

\begin{prop} \label{prop:compreducibleequiv}
	$\psi$ is completely reducible if $\varphi$ is so.
\end{prop}
\begin{proof} 
	By a similar argument as earlier, we see that if
	\begin{equation*} 
		V = V_1 \oplus \cdots \oplus V_n,
	\end{equation*}
	then
	\begin{equation*} 
		W = W_1 \oplus \cdots \oplus W_n,
	\end{equation*}
	where $W_i \vcentcolon= T(V_i)$ and each subspace on the right is $G$-invariant.

	We now wish to show that if $\varphi|_{V_i}$ is irreducible, then $\psi|_{W_i}$ is too. This is simple because we note that the following diagram commutes for all $g \in G:$
	\begin{center}
		\begin{tikzcd}
			{V_i} \arrow[rr, "\varphi_g|_{V_i}"]\arrow[dd, "T|_{V_i}"'] & & {V_i}\arrow[dd, "T|_{V_i}"]\\
			& & \\
			{W_i} \arrow[rr, "\psi_g|_{W_i}"'] & & {W_i}
		\end{tikzcd}
	\end{center}
	and thus, $\varphi|_{V_i} \sim \psi|_{W_i}.$ (Note that $T|_{V_i}$ is indeed an isomorphism.) Thus, by \Cref{prop:irreducibleequiv}, we are done
\end{proof}

\begin{thm}[Irreducible representations of finite cyclic groups] \label{thm:irredcyclicgroup}
	Let $G$ be a finite cyclic group. All irreducible representations of $G$ are of degree one.
\end{thm}
\begin{proof} 
	Without loss of generality, we may assume $G = \mathbb{Z}/n\mathbb{Z}.$ Suppose that $\varphi:G \to \GL_m(\mathbb{C})$ is a representation with $m \ge 2.$ We show that it is reducible. 

	Note that $\varphi_{[1]}^n = I.$ Thus, the minimal polynomial of $\varphi_{[1]}$ is a factor of $X^n - 1$ and hence, has distinct roots. This shows that $\varphi_{[1]}$ is diagonalisable. (\Cref{thm:splitdistinctdiagonalise})

	Let $T \in \GL_m(\mathbb{C})$ be such that
	\begin{equation*} 
		T\varphi_{[1]}T^{-1} = D
	\end{equation*}
	for some diagonal matrix $D.$ Note that raising both sides to the power $k$ yields
	\begin{equation*} 
		T\varphi_{[1]}^kT^{-1} = D^k
	\end{equation*}
	or
	\begin{equation*} 
		T\varphi_{[k]}T^{-1} = D^k.
	\end{equation*}
	In other words, the equivalent representation $\psi:G \to \GL_m(\mathbb{C})$ given by $\psi_{[k]} = T\varphi_{[k]}T^{-1}$ has the property that $\psi_{[k]}$ is diagonal for all $[k] \in G.$

	This shows that $\psi$ can be decomposed as $m$ non-zero proper sub-representations, proving reducibility. As a consequence, $\varphi$ is reducible.
\end{proof}

In the above, we used the fact from Linear Algebra that if the minimal polynomial of a matrix has distinct roots, then it is diagonalisable. In the next section, we shall prove the above theorem again without the fact and in turn, get the above fact as a corollary. (Note that there is no circular reasoning.)

\subsection{Maschke's Theorem and Complete Reducibility}
We recall the following definitions from linear algebra.

\begin{defn}%[Unitary representation]
	Let $V$ be an inner product space. A representation $\varphi:G \to \GL(V)$ is said to be \deff{unitary} if $\varphi_g \in U(V)$ for all $g \in G.$
\end{defn}
In other words, we can view $\varphi$ as a map $\varphi:G \to U(V).$ In yet other words, we have
\begin{equation*} 
	\langle \varphi_gv, \varphi_gw\rangle = \langle v, w\rangle
\end{equation*}
for all $g \in G$ and all $v, w \in V.$

\begin{defn}[Unit circle]
	We define $S^1 = \{z \in \mathbb{C} \mid \md{z} = 1\}.$
\end{defn}

Identifying $\GL_1(\mathbb{C})$ with $\mathbb{C}^*,$ we see that $U_1(\mathbb{C})$ is identified with $S^1.$ Hence, a degree-one unitary representation is a homomorphism $\varphi:G \to S^1.$

\begin{rem} \label{rem:findegoneunitary}
	As noted in \Cref{rem:finrepintoS1}, degree one representations of finite groups actually map into $S^1.$ Thus, they are all unitary.
\end{rem}

\begin{ex}
	$\varphi:\mathbb{R}\to S^1$ given by $t \mapsto \exp(2\pi\iota t)$ is a degree one-unitary representation of the additive group $(\mathbb{R}, +)$ since $\varphi(s + t) = \varphi(s)\varphi(t).$
\end{ex}

As we had noted earlier, decomposability was a stronger statement than reducibility. Now, we show that the two coincide for unitary representations.

\begin{prop} \label{prop:unitirredordecom}
	Let $\varphi:G \to \GL(V)$ be a unitary representation. Then, $\varphi$ is either irreducible or decomposable.
\end{prop}
\begin{proof} 
	Suppose that $\varphi$ is not irreducible. Then, there exists a non-zero proper subspace $W \le V$ which is $G$-invariant. Then, we have $V = W \oplus W^{\perp}$ and $W^{\perp}$ is non-zero proper. Thus, it now suffices to show that $W^\perp$ is $G$-invariant.

	Now, given any $g \in G,$ we know that $\varphi_g$ is unitary and $W$ is $\varphi_g$-invariant. Thus, by \Cref{cor:unitaryinvariance}, we see that $W^\perp$ is $\varphi_g$-invariant. Since this is true for all $g \in G,$ we see that $W^\perp$ is $G$-invariant, as desired.
\end{proof}

Now, we show that for finite groups, every representation is equivalent to a unitary representation and thus, conclude that decomposable and reducible are equivalent for finite groups. To make the final proof simpler, we first state two lemmata.

\begin{lem} \label{lem:newinnerproduct}
	Let $G$ be a \underline{finite group} and $\rho:G \to \GL_n(\mathbb{C})$ be a representation. Let $\langle \cdot, \cdot\rangle$ denote the standard inner product on $\mathbb{C}^n.$ Define the new product $(\cdot, \cdot)$ on $\mathbb{C}^n$ as
	\begin{equation*} 
		(v, w) \vcentcolon= \sum_{g \in G} \langle \rho_gv, \rho_gw\rangle.
	\end{equation*}
	Then, $(\cdot, \cdot)$ is an inner product.
\end{lem}
Note that the finiteness of $G$ tells us that the above sum is well-defined. (Of course, along with the fact that addition is commutative.)
\begin{proof} 
	Let $c_1, c_2 \in \mathbb{C}$ and $v_1, v_2, w, w \in \mathbb{C}^n$ be arbitrary.

	First, note
	\begin{align*} 
		(c_1v_1 + c_2v_2, w) &= \sum_{g \in G} \langle \rho_g(c_1v_1 + c_2v_2), \rho_gw\rangle\\
		&= \sum_{g \in G} \langle c_1\rho_gv_1 + c_2\rho_gv_2, \rho_gw\rangle\\
		&= \sum_{g \in G} \left[c_1\langle \rho_gv_1, \rho_gw\rangle + c_2\langle \rho_gv_2, \rho_gw\rangle\right]\\
		&= c_1\sum_{g \in G} \langle \rho_gv_1, \rho_gw\rangle + c_2\sum_{g \in G} \langle \rho_gv_2, \rho_gw\rangle\\
		&= c_1(v_1, w) + c_2(v_2, w).
	\end{align*}
	Next, 
	\begin{align*} 
		(w, v) &= \sum_{g \in G} \langle \rho_gw, \rho_gv\rangle\\
		&= \sum_{g \in G} \overline{\langle \rho_gv, \rho_gw\rangle}\\
		&= \overline{\sum_{g \in G} \langle \rho_gv, \rho_gw\rangle}\\
		&= \overline{(v, w)}.
	\end{align*}
	Lastly,
	\begin{equation*} 
		(v, v) = \sum_{g \in G} \langle \rho_gv, \rho_gv\rangle \ge 0
	\end{equation*}
	since each term is non-negative and hence,
	\begin{equation*} 
		(v, v) = 0 \implies \langle \rho_gv, \rho_gv\rangle = 0 \text{ for all } g \in G
	\end{equation*}
	and thus, $\rho_gv = 0$ for all $g \in G$ since $\langle \cdot, \cdot\rangle$ is an inner-product.\\
	In particular, $v = \rho_1v = 0,$ as desired.
\end{proof}

\begin{lem} \label{lem:unitarywrtspecial}
	With the same notation as in \Cref{lem:newinnerproduct}, we have that $\rho$ is unitary with respect to the inner product $(\cdot, \cdot).$
\end{lem}
\begin{proof} 
	Let $v, w \in V$ and $g \in G.$ Then,
	\begin{align*} 
		(\rho_gv, \rho_gw) &= \sum_{g' \in G} \langle \rho_{g'}\rho_gv, \rho_{g'}\rho_gw\rangle\\
		&= \sum_{g' \in G} \langle \rho_{g'g}v, \rho_{g'g}w\rangle.
	\end{align*}
	Note that $g' \mapsto g'g$ is a bijection and thus, the above is simplified as
	\begin{equation*} 
		(\rho_gv, \rho_gw) = \sum_{h \in G} \langle \rho_hv, \rho_hw\rangle = (v, w),
	\end{equation*}
	as desired.
\end{proof}

Note that in the previous two lemmata, we worked in $\GL_n(\mathbb{C})$ and used the standard inner product on $\mathbb{C}^n.$ However, this was just for the sake of concreteness. Instead of which, we could've worked with any inner product space $(V, \langle \cdot, \cdot\rangle).$

\begin{prop} \label{prop:repoffingroupisunitary}
	Every representation of a \underline{finite group} $G$ is equivalent to a unitary representation.
\end{prop}
\begin{proof} 
	Let $\varphi:G\to \GL(V)$ be a representation and let $n \vcentcolon= \dim V.$ Fix an isomorphism $T : V \to \mathbb{C}^n$ and put $\rho_g \vcentcolon= T\varphi_g T^{-1}$ for all $g \in G.$ This defines a representation $\rho: G \to \GL_n(\mathbb{C})$ which is equivalent to $\varphi.$ We now show that $\rho$ is unitary.

	Let $(\cdot, \cdot)$ be the inner product as in \Cref{lem:newinnerproduct}. Then, by \Cref{lem:unitarywrtspecial}, we know that $\rho$ is a unitary representation and we are done.
\end{proof}

We now state the corollary alluded all along.

\begin{cor} \label{cor:fingroupirrordec}
	Let $\varphi:G\to \GL(V)$ be a non-zero representation of a \underline{finite group}. Then, $\varphi$ is either irreducible or decomposable.
\end{cor}
\begin{proof} 
	By \Cref{prop:repoffingroupisunitary}, $\varphi \sim \rho$ for some unitary representation $\rho.$ By \Cref{prop:unitirredordecom}, $\rho$ is either irreducible or decomposable. By \Crefrange{prop:irreducibleequiv}{prop:decomposableequiv}, we see that the same is true for $\varphi$ as well.
\end{proof}
\begin{rem}
	For any group, we obviously have that decomposable $\implies$ reducible. The above says that the converse is true for finite groups.

	What the above says that if we have a $G$-invariant subspace $W,$ then we can actually decompose $V$ as $W_1 \oplus W_2$ (with them having the usual properties). In fact, our proof of \Cref{prop:unitirredordecom} actually shows that we can take $W_1 = W$ and $W_2$ is then the orthogonal subspace (after suitably finding an isomorphism which transports the inner product structure).
\end{rem}

With the above remark in mind, we rewrite the previous corollary as follows.

\begin{cor} \label{cor:existenceofcomplimentaryGinvarsubs}
	Let $\varphi:G\to \GL(V)$ be a non-zero representation of a \underline{finite group}. Suppose that $W$ is a non-zero proper $G$-invariant subspace of $V.$ Then, we can write
	\begin{equation*} 
		V = W \oplus W'
	\end{equation*}
	for a $G$-invariant subspace $W'.$ (It follows that $W'$ is also non-zero and proper.)
\end{cor}

With the above, we can strengthen \Cref{prop:deg2repirreducible} to degree three representations as well when $G$ is finite.
\begin{prop} \label{prop:deg3repirreducible}
	If $\varphi:G\to \GL(V)$ is a degree three representation, then $\varphi$ is reducible if and only if there is a common eigenvector $v$ to all $\varphi_g$ with $g \in G.$
\end{prop}
\begin{proof} 
	As before, $\impliedby$ is obvious. (That is true for all groups and all non-zero degree representation, in fact.)

	We show the other direction. Suppose that $\varphi$ is reducible. Then, by \Cref{cor:fingroupirrordec}, $\varphi$ is decomposable and we can write
	\begin{equation*} 
		V = W \oplus W'
	\end{equation*}
	for non-zero $G$-invariant subspaces $W$ and $W'.$ By looking at dimensions, we see that one of $W$ or $W'$ is one-dimensional. Thus, mimicking the proof of \Cref{prop:deg2repirreducible} shows that there is a common eigenvector.
\end{proof}
One can observe that the above proof is similar to how shows that if a three degree polynomial is reducible, then it has a root. However, we really did need the finiteness of $G$ as the following example shows us.

\begin{ex} \label{ex:Zredbutnotdecomposable}
	Let $\varphi:\mathbb{Z} \to \GL_2(\mathbb{C})$ be the representation
	\begin{equation*} 
		\varphi(n) = \two{1}{n}{}{1}.
	\end{equation*}
	Then, $\varphi$ is reducible since $\mathbb{C}e_1$ is a $\mathbb{Z}$-invariant subspace. However, one sees that there is no other common eigenvector to all $\varphi(n)$ and hence, there is no other $\mathbb{Z}$-invariant subspace. Thus, $\varphi$ is not decomposable.

	That is, $\varphi$ is neither neither irreducible nor decomposable, showing that \Cref{cor:fingroupirrordec} is false for infinite groups. In turn, \Cref{prop:repoffingroupisunitary} is also false for infinite groups. (Note that \Cref{prop:unitirredordecom} had no  assumption of finiteness of group.)
\end{ex}

Moreover, the previous cannot be strengthened to degree four representations (even for finite groups) as the next example shows us.

\begin{ex} \label{ex:deg4repredbutnocommoneigen}
	Let $\varphi:D_4 \to \GL_2(\mathbb{C})$ be the representation as in \Cref{ex:D4irreddeg2}. Put $\psi \vcentcolon= \varphi \oplus \varphi.$ Then, $\psi: G \to \GL_4(\mathbb{C})$ is a degree four representation and we have
	\begin{equation*} 
		\psi(r) = \begin{bmatrix}
			\iota & & & \\
			 & -\iota & & \\
			 & & \iota & \\
			 & & & -\iota \\
		\end{bmatrix} \andd \psi(s) = \begin{bmatrix}
			 & 1 & & \\
			 1 & & & \\
			 & & & 1 \\
			 & & 1 & \\
		\end{bmatrix}.
	\end{equation*}
	Clearly, the eigenvectors of $\psi(r)$ are the standard basis vectors (up to scaling) and none of them is an eigenvector of $\psi(s).$

	Thus, $\psi$ is reducible even though there is no $v \in V$ which is a common eigenvector for all $\psi_g.$
\end{ex}

The next result is again something we had prompted earlier. It is similar to the existence (but not uniqueness yet) to the decomposition of integers into their prime factors.

\begin{thm}[Maschke] \label{thm:maschke}
	Every representation of a \underline{finite} group is completely reducible.
\end{thm}
\begin{proof} 
	We prove this by induction on the degree of the representation. Let $\varphi : G \to \GL(V)$ be a representation.

	If $\dim V = 1,$ then $\varphi$ is irreducible (and hence, completely reducible) and we are done.

	Now, let $n \ge 1$ and assume the statement is degree of representation of degree $\le n.$ Let $\dim V = n + 1.$ If $\varphi$ is irreducible, then we are done. If not, then
	\begin{equation*} 
		V = U \oplus W
	\end{equation*}
	for non-zero $G$-invariant subspaces, by \Cref{cor:fingroupirrordec}. Since $U, W$ both have dimension strictly less than $\dim V,$ the induction hypothesis applies and we can write
	\begin{align*} 
		U &= U_1 \oplus \cdots \oplus U_n\\
		W &= W_1 \oplus \cdots \oplus W_m
	\end{align*}
	for $G$-invariant subspaces such that $\varphi|_{U_i}$ and $\varphi|_{W_j}$ is irreducible for all $1 \le i \le n$ and $1 \le j \le m.$ In turn, we have
	\begin{equation*} 
		V = U_1 \oplus \cdots \oplus U_n \oplus W_1 \oplus \cdots \oplus W_m,
	\end{equation*}
	as desired.
\end{proof}
\section{Character Theory and the Orthogonality Relations} \label{sec:02}
\subsection{Morphisms of Representations}

\begin{defn}%[Morphism]
	Let $\varphi:G \to \GL(V)$ and $\rho:G \to \GL(W)$ be representations. A \deff{morphism} from $\varphi$ to $\rho$ is a linear map $T : V \to W$ such that the following diagram commutes
	\begin{center}
		\begin{tikzcd}
			{V} \arrow[rr, "\varphi_g"]\arrow[dd, "T"'] & & {V}\arrow[dd, "T"]\\
			& & \\
			{W} \arrow[rr, "\rho_g"'] & & {W}
		\end{tikzcd}
	\end{center}
	\underline{for all $g \in G.$}

	The set of all morphisms from $\varphi$ to $\rho$ is denoted by $\Hom_{G}(\varphi, \rho).$ 
\end{defn}
Note that $\Hom_{G}(\varphi, \rho) \subset \Hom(V, W).$ % amnote : notation here, should keep \Bbb C?

The above definition can be seen as follows: Recall from \Cref{rem:groupaction} that a representation can be viewed as giving a group action. With this understanding, we may write $gv$ for $\varphi_gv$ and $gw$ for $\rho_gw$ (where $v \in V$ and $w \in W$). Now, under this notation, we see that a morphism from $\varphi$ to $\rho$ is simply a linear transformation $T : V \to W$ such that
\begin{equation*} 
	Tgv = gTv
\end{equation*}
for all $g \in G$ and all $v \in V.$

\begin{rem}
	If $T \in \Hom_{G}(\varphi, \rho)$ is an isomorphism, then $T$ is actually an \emph{equivalence} and $\varphi\sim\rho.$
\end{rem}

\begin{rem} \label{rem:commutingmorphisms}
	$T \in \Hom(V, V)$ is an element of $\Hom_{G}(\varphi, \varphi)$ if and only if $T \circ \varphi_g = \varphi_g \circ T$ for all $g \in G.$ In other words, $T$ commutes with every element of $\varphi(G).$ In particular, the identity map is always an element of $\Hom_{G}(\varphi, \varphi).$
\end{rem}

\begin{prop} \label{prop:Ginvmorphism}
	Let $T : V \to W$ be in $\Hom_{G}(\varphi, \rho).$ Then $\ker T$ and $\im T$ are $G$-invariant subspaces of $V$ and $W$ with respect to $\varphi$ and $\rho,$ respectively.
\end{prop}
\begin{proof} 
	$\ker T:$ Let $v \in \ker T$ and $g \in G$ be arbitrary. Then, 
	\begin{equation*} 
		T(\varphi_gv) = \rho_g(Tv) = \rho_g(0) = 0
	\end{equation*}
	and hence, $\varphi_gv \in \ker T,$ as desired.

	$\im T:$ Let $w \in \im T$ and $g \in G$ be arbitrary. Then, $w = Tv$ for some $v \in V.$ Then,
	\begin{equation*} 
		\rho_gw = \rho_g(Tv) = T(\varphi_gv)
	\end{equation*}
	showing that $\rho_gw \in \im T,$ as desired.
\end{proof}

As we had earlier observed, $\Hom_{G}(\varphi, \rho) \subset \Hom(V, W).$ In fact, more is true as the following proposition shows.

\begin{prop} \label{prop:morphismssubspace}
	Let $G$ be a group and $\varphi:G \to \GL(V),\;\rho:G \to \GL(W)$ be representations. Then, $\Hom_{G}(\varphi, \rho)$ is a subspace of the vector space $\Hom(V, W).$
\end{prop}
\begin{proof} 
	Clearly, the zero operator $0:V \to W$ is an element of $\Hom_{G}(\varphi, \rho).$

	Now, suppose that $S, T \in \Hom_{G}(\varphi, \rho)$ and $\alpha \in \mathbb{C}$ are arbitrary. Let $g \in G$ and $v \in V$ be arbitrary. Then,
	\[\begin{WithArrows}[displaystyle]
		(S + \alpha T)(\varphi_gv) &= S(\varphi_gv) + \alpha T(\varphi_gv) \Arrow{$S, T \in \Hom_{G}(\varphi, \rho)$} \\
		&= \rho_gSv + \alpha \rho_gTv \Arrow{$\rho_g$ is linear}\\
		&= \rho_g(Sv + \alpha Tv).
	\end{WithArrows}\]
	Thus, $S + \alpha T \in \Hom_{G}(\varphi, \rho).$
\end{proof}

\begin{prop} \label{prop:equivrepisoHoms}
	Let $\varphi : G \to \GL(V),\;\varphi' : G \to \GL(V'),\;\rho : G \to \GL(W),\;$ and $\rho' : G \to \GL(W')$ be representations.

	If $\varphi \sim \varphi'$ and $\rho \sim \rho',$ then $\dim\Hom_{G}(\varphi, \rho) = \dim\Hom_{G}(\varphi', \rho').$
\end{prop}
\begin{proof} 
	Let $T : V \to V'$ and $T' : W \to W'$ be isomorphisms showing the equivalences $\varphi\sim\varphi'$ and $\rho\sim\rho',$ respectively. (That is, they make the desired rectangles commute.) 

	Then, define the obvious map $\Phi:\Hom_{G}(\varphi, \rho) \to \Hom_{G}(\varphi', \rho')$ by
	\begin{equation*} 
		\Phi(S) = T' \circ S \circ T^{-1}.
	\end{equation*}
	That is, we wish to make the following diagram commute:
	\begin{center}
		\begin{tikzcd}
			{V} \arrow[rr, "T"]\arrow[dd, "S"'] & & {V'}\arrow[dd, "\Phi(S)"]\\
			& & \\
			{W} \arrow[rr, "T'"'] & & {W'}
		\end{tikzcd}
	\end{center}
	First, we verify that $\Phi$ actually maps into $\Hom_{G}(\varphi', \rho').$ This is simple. Let $g \in G,$ $S \in \Hom_{G}(\varphi, \rho)$ and $v' \in V'$ be arbitrary. We then note
	\[\begin{WithArrows}[displaystyle]
		\Phi(S)(\varphi'_gv') &= (T' \circ S \circ T^{-1})(\varphi'_gv')\\
		&= T'S(T^{-1}(\varphi'_gv')) \Arrow{$T$ and hence, $T^{-1}$ is an equivalence}\\
		&= T'S(\varphi_gT^{-1}v') \Arrow{$S \in \Hom_{G}(\varphi, \rho)$}\\
		&= T'(\rho_gST^{-1}v') \Arrow{$T'$ is an equivalence}\\
		&= \rho'_g(T'ST^{-1}v')\\
		&= \rho'_g(\Phi(S)v'),
	\end{WithArrows}\]
	as desired.

	It is easy to see that $\Phi$ is linear. Indeed, this follows simply because $T$ is linear. Lastly, to see that it is a bijection, note that we have a two-sided inverse for $\Phi$ defined in the similar manner.
\end{proof}

\begin{lem}[Schur's lemma] \label{lem:schur}
	Let $\varphi, \rho$ be irreducible representations of $G,$ and $T \in \Hom_{G}(\varphi, \rho).$ Then either $T$ is invertible or $T = 0.$ Consequently:
	\begin{enumerate}
		\item If $\varphi \not\sim \rho,$ then $\Hom_{G}(\varphi, \rho) = 0$;
		\item If $\varphi = \rho,$ then $T = \lambda I$ with $\lambda \in \mathbb{C}.$ In other words, $T$ is simply multiplication with a scalar. (Here is where we use that the base field is $\mathbb{C}.$)
	\end{enumerate}
\end{lem}
\begin{proof} 
	Let $\varphi : G \to \GL(V)$ and $\rho : G \to \GL(W)$ be irreducible representations.

	If $T = 0,$ then we are done. Thus, assume that $T \neq 0.$ In this case, $\ker T \neq V.$ On the other hand, by \Cref{prop:Ginvmorphism}, we know that $\ker T$ is $G$-invariant. Hence, irreducibility of $\varphi$ forces that $\ker T = 0.$ In other words, $T$ is injective.

	Similarly, we know that $\im T$ is $G$-invariant and hence, $\im T = 0$ or $\im T = W.$ As $T \neq 0,$ the former is not possible. Thus, we see that $\im T	= W,$ i.e., $T$ is onto.

	Thus, we conclude that $T$ is invertible. We now prove the consequences.
	\begin{enumerate}
		\item This is immediate for if $\varphi\not\sim\rho,$ then $T$ cannot be invertible for otherwise it would be an equivalence. Thus, the only possible morphism is the zero map.
		\item Let $\lambda$ be an eigenvalue of $T$ (which exists because the base field is the algebraically closed $\mathbb{C}$).\\
		Now, recall that the identity map $I$ is an element of $\Hom_{G}(\varphi, \varphi).$ (\Cref{rem:commutingmorphisms})\\
		By \Cref{prop:morphismssubspace}, we then see that $T - \lambda I \in \Hom_{G}(\varphi, \varphi).$ Now, by definition of an eigenvalue, $T - \lambda I$ cannot be invertible. Thus, $T - \lambda I = 0$ which establishes the consequence.
	\end{enumerate}
	Thus, we are done.
\end{proof}

\begin{cor} \label{cor:schurdimone}
	If $\varphi$ and $\rho$ are equivalent irreducible representations, then $\dim \Hom_{G}(\varphi, \rho) = 1.$
\end{cor}
\begin{proof} 
	By \Cref{prop:equivrepisoHoms}, it suffices to show that $\dim \Hom_{G}(\varphi, \varphi) = 1.$ By the previous part, we see that $\{I\}$ is a basis for $\Hom_{G}(\varphi, \varphi).$
\end{proof}

\begin{prop} \label{prop:homoplusiso}
	Let $\varphi : G \to \GL(V), \rho_1 \to \GL(W_1), \rho_2 \to \GL(W_2)$ be representations. Then, the isomorphism
	\begin{equation*} 
		\Hom_{G}(\varphi, \rho_1 \oplus \rho_2) \cong \Hom_{G}(\varphi, \rho_1) \oplus \Hom_{G}(\varphi, \rho_2)
	\end{equation*}
	holds and in particular, we have
	\begin{equation*} 
		\dim\Hom_{G}(\varphi, \rho_1 \oplus \rho_2) = \dim\Hom_{G}(\varphi, \rho_1) + \dim\Hom_{G}(\varphi, \rho_2).
	\end{equation*}
\end{prop}
\begin{proof} 
	Let $T \in \Hom_{G}(\varphi, \rho_1 \oplus \rho_2).$ Thus, $T$ is of the form $T : V \to W_1 \oplus W_2.$ Letting $\pi_i$ denote the projection map, we see that $\pi_i \circ T : V \to W_i$ are linear.\\
	Moreover, $\pi_i \circ T$ is a morphism. Indeed, for $g \in G$ and $v \in V,$ we note that
	\begin{align*} 
		((\pi_i \circ T)\circ\varphi_g)(v) &= \pi_i(T(\varphi_g(v)))\\
		&= \pi_i\left((\rho_1 \oplus \rho_2)_g(Tv)\right)\\
		&= \rho_i(g)(T(v)).
	\end{align*}
	Thus, $\pi_i \circ T \in \Hom_{G}(\varphi, \rho_i)$ for $i = 1, 2.$ 

	Conversely, given a morphisms $T_i \in \Hom_{G}(\varphi, \rho_i)$ for $i = 1, 2,$ the function
	\begin{equation*} 
		T : V \to W_1 \oplus W_2
	\end{equation*}
	defined by
	\begin{equation*} 
		T(v) = (T_1(v), T_2(v))
	\end{equation*}
	is again a morphism. The correspondence $(T_1, T_2) \leftrightarrow T$ is $\mathbb{C}$-linear and bijective. This yields the desired isomorphism.
\end{proof}

\begin{cor} \label{cor:extractmultiplicitywithhom}
	Suppose $\varphi^{(1)}, \ldots, \varphi^{(s)}$ are pairwise inequivalent irreducible representations of $G.$ Put
	\begin{equation*} 
		\varphi = \underbrace{\varphi^{(1)} \oplus \cdots \oplus \varphi^{(1)}}_{m_1} \oplus \cdots \oplus \underbrace{\varphi^{(s)} \oplus \cdots \oplus \varphi^{(s)}}_{m_s}
	\end{equation*}
	for positive integers $m_1, \ldots, m_s.$ Then, 
	\begin{equation*} 
		\dim\Hom_{G}(\varphi^{(r)}, \varphi) = m_r
	\end{equation*}
	for $1 \le r \le m.$
\end{cor}
\begin{proof} 
	By \Cref{prop:homoplusiso}, it follows that
	\begin{equation*} 
		\dim\Hom_{G}(\varphi^{(r)}, \varphi) = \sum_{i = 1}^{m}m_i\dim\Hom_{G}(\varphi^{(r)}, \varphi^{(i)}).
	\end{equation*}
	By \Cref{lem:schur} and \Cref{cor:schurdimone}, it follows that only $r = i$ survives in which case the dimension is one.
\end{proof}

We now generalise the result of \Cref{thm:irredcyclicgroup} (in fact, this also gives an alternate proof of \Cref{thm:irredcyclicgroup}).

\begin{thm}[Irreducible representations of abelian groups] \label{thm:irredabelgroup}
	Let $G$ be an abelian group. Then any irreducible representation of $G$ has degree $1.$
\end{thm}
\begin{proof} 
	The idea is simple. We first show that every $\varphi_h$ is a morphism from $\varphi$ to itself. Using that, we construct a dimension one invariant subspace of $V$ forcing $V$ to be one dimensional.

	To this end, fix $h \in H.$ Put $T \vcentcolon= \varphi_h$ and let $g \in G$ be arbitrary. Then, we have
	\begin{equation*} 
		T\varphi_g = \varphi_h\varphi_g = \varphi_{hg} = \varphi_{gh} = \varphi_g\varphi_h = \varphi_gT
	\end{equation*}
	proving that $\varphi_h \in \Hom_{G}(\varphi, \varphi).$ Consequently, \autoref{lem:schur} (which is applicable since $\varphi$ is \underline{irreducible}) tells us that $\varphi_h = \lambda_h I$ for some $\lambda_h \in \mathbb{C}.$ 

	Now, fix a non-zero vector $v \in V.$ Then, $\varphi_hv = \lambda_hv \in \mathbb{C}v.$ This shows that $\mathbb{C}v$ is $\varphi_h$ invariant. Note that $h$ was arbitrary and $v$ did not dependent on $h.$ Thus, $\mathbb{C}v$ is a $G$-invariant subspace and irreducibility forces $V = \mathbb{C}v.$
\end{proof}

\begin{rem}
	By \Cref{cor:deg1repsfinabel}, we already have a description of the degree one representations of the finite abelian groups.
\end{rem}

\begin{cor} \label{cor:finabelrepdiagonal}
	Let $G$ be a finite abelian group and $\varphi : G \to \GL_n(\mathbb{C})$ a representation. Then, there exists an invertible matrix $T$ such that $T^{-1}\varphi_gT$ is diagonal for all $g \in G.$
\end{cor}
Note that the matrix $T$ is independent of $g.$ 
\begin{proof} 
	Since $G$ is finite, $\varphi$ is completely reducible, by \Cref{thm:maschke}. Thus, we can write
	\begin{equation*} 
		\varphi \sim \varphi^{(1)} \oplus \cdots \oplus \varphi^{(m)}
	\end{equation*}
	where each $\varphi^{(i)}$ is irreducible. By the previous corollary, it follows that that each $\varphi^{(i)}$ is of degree $1$ and hence, we also get $m = n.$

	If $T : \mathbb{C}^n \to \mathbb{C}^n$ is an isomorphism giving the above equivalence, then we see that 
	\begin{equation*} 
		T^{-1}\varphi_gT = \diag\left(\varphi^{(1)}_g, \ldots, \varphi^{(n)}_g\right),
	\end{equation*}
	as desired.
\end{proof}

\begin{cor} \label{cor:finorderdiagonalisable}
	Let $A \in \GL_m(\mathbb{C})$ be a matrix of finite order. Then, $A$ is diagonalisable.
\end{cor}
\begin{proof}
	Let $n > 0$ be the order of $A.$ Then, we get a representation $\varphi:\mathbb{Z}/n\mathbb{Z} \to \GL_m(\mathbb{C})$ given by $\varphi\left([k]\right) = A^k.$ Then, by \Cref{cor:finabelrepdiagonal}, $\varphi\left([1]\right) = A$ is diagonalisable. (In fact, the collection $I, \ldots, A^{n-1}$ is \emph{simultaneously} diagonalisable.)
\end{proof}

\subsection{The Orthogonality Relations}

\begin{disc}
	From now on, for the rest of the report, the group $G$ will be assumed to be finite unless otherwise mentioned.
\end{disc}

\begin{defn}%[Group algebra] 
	\label{defn:groupalg}
	Let $G$ be a group and let $L(G)$ denote the set of all functions from $G$ to $\mathbb{C}.$ That is,
	\begin{equation*} 
		L(G) \vcentcolon= \mathbb{C}^G = \{f \mid f : G \to \mathbb{C}\}.
	\end{equation*}
	Then, $L(G)$ is a vector space over $\mathbb{C}$ in the natural way. It is also an inner product space with inner product defined as
	\begin{equation*} 
		\langle f_1, f_2\rangle \vcentcolon= \dfrac{1}{\md{G}}\sum_{g \in G} f_1(g) \overline{f_2(g)}.
	\end{equation*}
	$L(G)$ is called the \deff{group algebra} of the group $G.$
\end{defn}
The last sum makes sense without any convergence issues because $G$ is finite.
\begin{defn}%[Norm]
	\label{defn:norm} 
	Given a group $G$ and $f \in L(G),$ the \deff{norm} of $f$ is defined as
	\begin{equation*} 
		\|f\| \vcentcolon= \sqrt{\langle f, f\rangle}.
	\end{equation*}
\end{defn}

Note that given a representation $\varphi : G \to \GL_n(\mathbb{C}),$ we get $n^2$ functions $\varphi_{ij} : G \to \mathbb{C},$ corresponding to the $n^2$ entries. We now wish to study properties of $\varphi_{ij} \in L(G)$ when $\varphi$ is irreducible and unitary.

Our eventual goal will be to prove \Cref{thm:schurorthorel}.

\begin{prop} \label{prop:Thash}
	Let $\varphi : G \to \GL(V)$ and $\rho : G \to \GL(W)$ be representations and suppose that $T : V \to W$ is a linear transformation. Then,
	\begin{enumerate}
		\item\label{item:001} $T^{\#} = \frac{1}{\md{G}}\sum_{g \in G} \rho_{g^{-1}}T\varphi_g \in \Hom_{G}(\varphi, \rho).$
		\item\label{item:002} If $T \in \Hom_{G}(\varphi, \rho),$ then $T^{\#} = T.$
		\item\label{item:003} The map $P : \Hom_{\mathbb{C}}(V, W) \to \Hom_{G}(\varphi, \rho)$ defined by $T \mapsto T^{\#}$ is an onto linear map.
	\end{enumerate}
\end{prop}
\begin{proof} 
	The proof of (\ref{item:001}) is by direct computation. Let $h \in G$ be arbitrary. Note that
	\begin{equation*} 
		T^{\#}\varphi_h = \frac{1}{\md{G}}\sum_{g \in G} \rho_{g^{-1}}T\varphi_{gh} = \frac{1}{\md{G}}\sum_{g' \in G} \rho_{hg'^{-1}}T\varphi_{g'} = \rho_hT^{\#}.
	\end{equation*}
	The middle inequality follows by the (bijective) change of variable $gh = g'.$ The above then establishes (\ref{item:001}).

	Now, if $T \in \Hom_{G}(\varphi, \rho),$ then we get
	\begin{equation*} 
		T^{\#} =  \frac{1}{\md{G}}\sum_{g \in G} \rho_{g^{-1}}T\varphi_g = \frac{1}{\md{G}}\sum_{g \in G} \rho_{g^{-1}}\rho_gT = \frac{1}{\md{G}}\md{G}T = T,
	\end{equation*}
	which proves (\ref{item:002}).

	Note that the above also proves that $T \mapsto T^{\#}$ is onto. Thus, to prove (\ref{item:003}), we only need to prove linearity of $P.$ This again follows by direct computation. Let $c \in \mathbb{C}$ and $T_1, T_2 \in \Hom_{\mathbb{C}}(V, W)$ be arbitrary.
	\begin{align*} 
		P(cT_1 + T_2) &= \frac{1}{\md{G}}\sum_{g \in G} \rho_{g^{-1}}(cT_1 + T_2)\varphi_g\\
		&=c\frac{1}{\md{G}}\sum_{g \in G} \rho_{g^{-1}}(T_1)\varphi_g + \frac{1}{\md{G}}\sum_{g \in G} \rho_{g^{-1}}(T_2)\varphi_g\\
		&= cP(T_1) + P(T_2),
	\end{align*}
	as desired.
\end{proof}

\begin{prop} \label{prop:hashpropertiesirredrep}
	Let $\varphi : G \to \GL(V)$ and $\rho : G \to \GL(W)$ be \underline{irreducible} representations of $G$ and let $T : V \to W$ be a linear map. Then:
	\begin{enumerate}
		\item \label{item:004} If $\varphi \not\sim \rho,$ then $T^{\#} = 0;$
		\item \label{item:005} If $\varphi = \rho,$ then $T^{\#} = \frac{\trace T}{\deg \varphi}I.$
	\end{enumerate}
\end{prop}
\begin{proof} 
	(\ref{item:004}) is simple for $T^{\#} \in \Hom_{G}(\varphi, \rho) = 0,$ by \nameref{lem:schur}. Now, if $\varphi = \rho,$ then $T^{\#} = \lambda I$ for some $\lambda \in \mathbb{C},$ again by \nameref{lem:schur}. We now wish to determine $\lambda.$

	Note that $\trace T^{\#} = \trace(\lambda I) = \lambda \dim V = \lambda \deg \varphi.$ That is,
	\begin{equation} \tag{$*$} \label{eq:001}
		T^{\#} = \lambda I = \dfrac{\trace T^{\#}}{\deg \varphi}I.
	\end{equation}
	We may also calculate $\trace T^{\#}$ separately, using the definition of $T^{\#}$ and the fact that $\trace(ABC) = \trace(CAB).$ This gives us that
	\begin{align*} 
		\trace(T^{\#}) &= \dfrac{1}{\md{G}}\sum_{g \in G} \trace(\varphi_{g^{-1}}T\varphi_g)\\
		&=\dfrac{1}{\md{G}}\sum_{g \in G} \trace(\varphi_g\varphi_{g^{-1}}T)\\
		&=\dfrac{1}{\md{G}}\sum_{g \in G} \trace(T)\\
		&= \trace(T).
	\end{align*}
	Putting the above back in \Cref{eq:001}, we get
	\begin{equation*} 
		T^{\#} = \dfrac{\trace T}{\deg \varphi}I. \qedhere
	\end{equation*}
\end{proof}

If we consider $V = \mathbb{C}^n$ and $\GL(V) = \GL_n(\mathbb{C})$ (and similarly for $W = \mathbb{C}^m$), then \Cref{prop:Thash} tells us that we can consider $P$ as a linear from $\GL(V, W) = M_{m \times n}(\mathbb{C})$ to itself. It is now natural to ask what is the matrix representation of $P$ with respect to the standard basis $\{E_{ij}\}.$ (Recall that $E_{ij}$ is the $m \times n$ matrix with $1$ in the $(i, j)$-th entry and $0$ everywhere else.)

\begin{lem} \label{lem:matmult}
	Let $A \in M_{r \times m}(\mathbb{C}),$ $E_{ki} \in M_{m \times n}(\mathbb{C}),$ and $B \in M_{n \times s}(\mathbb{C}).$ Then, we have 
	\begin{equation*} 
		(AE_{ki}B)_{lj} = a_{lk}b_{ij},
	\end{equation*}
	where $A = (a_{ij})$ and $B = (b_{ij}).$
\end{lem}
\begin{proof} 
	By definition, we have
	\begin{equation*} 
		(AE_{ki}B)_{lj} = \sum_{x, y} a_{lx}(E_{ki})_{xy}b_{yj}.
	\end{equation*}
	The only (possibly) non-zero term appearing in the summation is when $(x, y) = (k, i)$ which proves the result since $(E_{ki})_{ki} = 1.$
\end{proof}

\begin{lem} \label{lem:Ahashinnerprod}
	Let $\varphi : G \to U_n(\mathbb{C})$ and $\rho : G \to U_m(\mathbb{C})$ be unitary representations of $G.$ Let $A = E_{ki} \in M_{m \times n}(\mathbb{C}).$ Then, $A^{\#}_{lj} = \langle \varphi_{ij}, \rho_{kl}\rangle.$
\end{lem}
Note that we had remarked earlier that given a function $\varphi:G \to U_n(\mathbb{C}),$ we actually get $n^2$ $\mathbb{C}$-valued functions. The inner product appearing in the above lemma is the one defined in \Cref{defn:groupalg}. 
\begin{proof} 
	Let $g \in G.$ Then $\rho_g \in U_n(\mathbb{C}).$ Note that we have
	\begin{equation*} 
		\rho_{g^{-1}} = \left(\rho_g\right)^{-1} = \rho_g^*
	\end{equation*}
	because $\rho_g$ is unitary.

	Thus, we see that
	\begin{equation*} 
		\rho_{lk}(g^{-1}) = \overline{\rho_{kl}(g)}.
	\end{equation*}

	With the above, we note that
	\[\begin{WithArrows}[displaystyle]
		A^{\#}_{lj} &= \dfrac{1}{\md{G}}\sum_{g \in G} (\rho_{g^{-1}}E_{ki}\varphi_g)_{lj} \Arrow{\Cref{lem:matmult}}\\
		&= \dfrac{1}{\md{G}}\sum_{g \in G} \rho_{lk}(g^{-1})\varphi_{ij}(g)\\
		&= \dfrac{1}{\md{G}}\sum_{g \in G} \overline{\rho_{kl}(g)}\varphi_{ij}(g) \Arrow{\Cref{defn:groupalg}}\\
		&= \langle \varphi_{ij}, \rho_{kl}\rangle,
	\end{WithArrows}\]
	as desired.
\end{proof}

We now prove the desired theorem.

\begin{thm}[Schur's orthogonality relations] \label{thm:schurorthorel}
	Let $G$ be a finite group.\\
	Suppose that $\varphi : G \to U_n(\mathbb{C})$ and $\rho : G \to U_m(\mathbb{C})$ are inequivalent \underline{irreducible} unitary representations. Then:
	\begin{enumerate}
		\item \label{item:006} $\langle \varphi_{ij}, \rho_{kl}\rangle = 0,$
		\item \label{item:007} $\langle \varphi_{ij}, \varphi_{kl}\rangle = \begin{cases}
			1/n & \text{if } (i, j) = (k, l),\\
			0 & \text{otherwise}.
		\end{cases}$
	\end{enumerate}
	In particular, $\{\varphi_{ij} \mid 1 \le i, j \le n\} \cup \{\rho_{kl} \mid 1 \le k, l \le m\}$ is a linearly independent set.
\end{thm}
The last part follows since the theorem tells us that the above set of functions form an orthogonal set of non-zero vectors.
\begin{proof} 
	Letting $A = E_{ki} \in M_{m \times n}(\mathbb{C}),$ we see that $A^{\#} = 0$ by \Cref{item:004} of \Cref{prop:hashpropertiesirredrep}. On the other hand, $\langle \varphi_{ij}, \rho_{kl}\rangle = (A^{\#})_{lj},$ by \Cref{lem:Ahashinnerprod}. This proves (\ref{item:006}).

	Now, we put $\rho = \varphi.$ We apply the same proposition and lemma again. Letting $A = E_{ki} \in M_{n}(\mathbb{C}),$ we see that 
	\begin{equation*} 
		A^{\#} = \frac{\trace A}{n}I
	\end{equation*} 
	by \Cref{item:005} of \Cref{prop:hashpropertiesirredrep}. By \Cref{lem:Ahashinnerprod}, we see that
	\begin{equation*} 
		\langle \varphi_{ij}, \varphi_{kl}\rangle = (A^{\#})_{lj} = \frac{\trace A}{n}I_{lj}.
	\end{equation*}
	Now if $i \neq k,$ then $\trace A = 0.$ On the other hand, if $l \neq j,$ then $I_{lj} = 0.$ Now, if $(i, j) = (k, l),$ then $\trace A = 1$ and $I_{lj} = 1.$ These three cases put together prove (\ref{item:007}).
\end{proof}

\begin{cor} \label{cor:schurorthonormal}
	Let $\varphi$ be an \underline{irreducible} unitary representation of $G$ of degree $n.$ Then, the following set of $n^2$ functions
	\begin{equation*} 
		\left\{\sqrt{n}\varphi_{ij} \mid 1 \le i, j \le n\right\}
	\end{equation*}
	forms an ortho\underline{normal} set.
\end{cor}
\begin{proof} 
	By the previous theorem, we already know that any two distinct elements of the set are orthogonal. The multiplication by $\sqrt{n}$ simply makes all the functions have unit norm.
\end{proof}

\begin{prop} \label{prop:fingroupirredbounds}
	Let $G$ be a finite group. Then, the following hold.
	\begin{enumerate}
		\item There are only finitely many equivalence classes of irreducible representations of $G.$

		\item Let $\varphi^{(1)}, \ldots, \varphi^{(s)}$ be a transversal of unitary representatives of irreducible representations of $G.$ Set $d_i \vcentcolon= \deg \varphi^{(i)}.$ Then, the set of functions
		\begin{equation*} 
			\left\{\sqrt{d_k}\varphi_{ij}^{(k)} \mid 1 \le k \le s,\;1 \le i, j \le d_k\right\}
		\end{equation*}
		forms an orthonormal set in $L(G).$	
		\item In particular, $s \le d_1^2 + \cdots + d_s^2 \le \md{G}.$
	\end{enumerate}
\end{prop}
\begin{proof} All of these follow from \Cref{cor:schurorthonormal}.
	\begin{enumerate}
		\item Note that given any set of equivalence classes of (not necessarily irreducible) representations, each class contains a unitary representation, by \Cref{prop:repoffingroupisunitary}. Now, since $\dim L(G) = \md{G},$ no linearly independent set of vectors from $L(G)$ can contain more than $\md{G}$ many elements. Since orthonormal sets are linearly independent, \Cref{cor:schurorthonormal} shows that there can only be finitely many classes of \underline{irreducible} representations.
		\item This part again follows mainly from \Cref{cor:schurorthonormal}. The orthogonality of two functions of representations of different degrees follows from \nameref{thm:schurorthorel} since the representations $\varphi^{(i)}$ and $\varphi^{(j)}$ are inequivalent if $d_i \neq d_j.$
		\item $s \le d_1^2 + \cdots + d_s^2$ is clear since each $d_i^2$ is at least $1.$ On the other hand, the orthonormal set given has $d_1^2 + \cdots + d_s^2$ elements in a vector space of dimension $\md{G},$ proving the second inequality. \qedhere
	\end{enumerate}
\end{proof}

\begin{rem}
	We shall later see that we actually have the equality 
	\begin{equation*} 
		\md{G} = d_1^2 + \cdots + d_s^2.
	\end{equation*}
\end{rem}

\subsection{Some Examples}
\begin{ex}[Degree one representations of $D_n$] \label{ex:degonerepsDn}
	Recall that $D_n$ has the following presentation
	\begin{equation*} 
		D_n = \langle r, s \mid r^n = s^2 = rsrs = 1\rangle.
	\end{equation*}
	In other words, to define a representation $z : G \to \mathbb{C}^*,$ we only need to specify $z_r$ and $z_s$ which satisfy the above relations. (In the sense that this gives all the representations and that every representation is obtained this way.)

	Note that since $\mathbb{C}^*$ is commutative, for the last relation, we only need
	\begin{equation*} 
		z_r^2z_s^2 = 1.
	\end{equation*}
	However, the second relation already tells us that $z_s^2 = 1.$ Thus, we now have the equivalent job of finding $z_r, z_s \in \mathbb{C}^*$ satisfying
	\begin{equation*} 
		z_r^n = 1,\;z_r^2 = 1,\; z_s^2 = 1.
	\end{equation*}
	Note that, in the above, we have separated the relations into those for $z_r$ and $z_s$ separately. Thus, we have precisely two choices for $z_s$ (namely, $\pm 1$) for every choice for $z_r.$

	We now turn to the case of determining $z_r.$ There are two cases.

	\textbf{Case 1.} $n$ is even. In this case, the relation $z_r^n = 1$ is implied by $z_r^2 = 1.$ Thus, we get precisely two choices for $z_r:$ $\pm 1.$

	\textbf{Case 2.} $n$ is odd. Then, since $\gcd(n, 2) = 1,$ one can conclude that $z_r^{1} = 1$ and thus, we have only one choice.

	Thus, we get the number of degree one representations of $D_n$ as:
	\begin{enumerate}
		\item $4,$ if $n$ is even,
		\item $2,$ if $n$ is odd.
	\end{enumerate}

	Note that all of these are inequivalent since distinct degree one representations are inequivalent. (\Cref{prop:distinctdeg1inequiv}.)
\end{ex}

\begin{ex}[An irreducible representation of $D_n$] \label{ex:anirredrepDn}
	Consider the regular $n$-polygon as a subset of $\mathbb{C}$ with vertices as the $n$-th roots of unity. We can think of its set of symmetries as $D_n.$ This gives us an embedding as follows
	\begin{equation*} 
		\varphi : D_n \to \GL_2(\mathbb{C})
	\end{equation*}
	defined as
	\begin{equation*} 
		\varphi_r \vcentcolon= \two{\cos \theta_n}{\sin \theta_n}{-\sin \theta_n}{\cos \theta_n} \andd \varphi_s \vcentcolon= \two{1}{0}{0}{-1},
	\end{equation*}
	where $\theta_n = \frac{2\pi}{n}.$ 

	(Alternately, one can verify that $\varphi_r^n = \varphi_s^2 = (\varphi_r\varphi_s)^2 = 1.$)

	Now, to see that it is irreducible, we note that the eigenvectors of $\varphi_s$ (up to scaling) are $e_1$ and $e_2.$ Thus, $\varphi_r$ and $\varphi_s$ have no common eigenvectors (note that $\sin\theta_n \neq 0$) and hence, $\varphi$ is irreducible.
\end{ex}

\begin{ex}[All irreducible representations of $D_3$ and $D_4$]
	Note that by \Cref{ex:degonerepsDn}, we already know that there are $2$ (inequivalent irreducible) degree one representations of $D_3$ and $4$ of $D_4.$

	By \Cref{ex:anirredrepDn}, we also have $1$ irreducible degree two representation of both.

	On the other hand, note that
	\begin{align*} 
		1^1 + 1^1 + 2^2 &= 6 = \md{D_3},\\
		1^1 + 1^1 + 1^1 + 1^1 + 2^2 &= 8 = \md{D_4}.
	\end{align*}

	Thus, by \Cref{prop:fingroupirredbounds}, we see that we have actually found all irreducible representations of $D_3$ and $D_4$! Note that \Cref{ex:anirredrepDn} and \Cref{ex:D4irreddeg2} are two distinct degree two representations of $D_4.$ The above analysis however tells us that the two are equivalent, even without us explicitly constructing any equivalence.
\end{ex}

\subsection{Characters and Class Functions}
In this subsection, we will prove the uniqueness of decompositions. (That is, the uniqueness of the decomposition given in \nameref{thm:maschke}.)

We start by introducing the character of a representation. Recall that given a endomorphism of a (finite dimensional) vector space, we can talk about its trace. This is defined as the trace of any matrix representation obtained after fixing an ordered basis. It is easy to see that this is basis invariant.

\begin{defn}%[Character]
	Let $\varphi : G \to \GL(V)$ be a representation. The \deff{character} $\chi_\varphi : G \to \mathbb{C}$ of $\varphi$ is defined by $\chi_\varphi(g) = \trace \varphi_g.$ The character of an irreducible representation is called an \deff{irreducible character}.
\end{defn}

As remarked earlier, the computation of character is independent of the basis we choose. For this reason, we may assume without loss of generality that we are talking about matrix representations. (In the cases where the general case is as simple, we need not do so.)

If $\varphi : G \to \GL_n(\mathbb{C})$ is a representation given by $\varphi_g = (\varphi_{ij}(g)),$ then
\begin{equation*} 
	\chi_\varphi(g) = \sum_{i = 1}^{n}\varphi_{ii}(g).
\end{equation*}

\begin{rem}
	If $z : G \to \mathbb{C}^* \hookrightarrow \mathbb{C}$ is a degree one representation, then $\chi_z = z.$ From now on, we shall treat degree one representations and their characters as the same.
\end{rem}

\begin{prop} \label{prop:charatidisdeg}
	If $\varphi : G \to \GL(V)$ is a representation, then $\chi_\varphi(1) = \deg \varphi.$
\end{prop}
\begin{proof} 
	$\chi_\varphi(1) = \trace \varphi_1 = \trace \id_V = \dim V = \deg \varphi.$
\end{proof}

\begin{prop} \label{prop:equivrepssamechar}
	If $\varphi$ and $\rho$ are equivalent representations, then $\chi_\varphi = \chi_\rho.$
\end{prop}
\begin{proof} 
	As remarked earlier, we may assume the representations in the form
	\begin{equation*} 
		\varphi, \rho : G \to \GL_n(\mathbb{C}).
	\end{equation*}
	Since the representations are equivalent, there exists an invertible matrix $T \in \GL_n(\mathbb{C})$ such that
	\begin{equation*} 
		\varphi_g = T\rho_gT^{-1}
	\end{equation*}
	for all $g \in G.$ Since the traces of similar matrices are the same, we are done.
\end{proof}

To recall why the last statement is true, note that $\trace(ABC) = \trace(CAB)$ and thus, if $C = A^{-1}$, we are done.

\begin{cor} \label{cor:chargrouprootsunity}
	Let $G$ be a group of order $n$ and $\chi$ a character of degree $m$ of $G.$ Then, $\chi(g)$ is a sum of $m$ $n$-th roots of unity, for each $g \in G.$
\end{cor}
\begin{proof} 
	Since characters only depend up to equivalence, we may assume that the representation is of the form $\varphi : G \to \GL_m(\mathbb{C})$ with character $\chi.$ Fix $g \in G.$ Then, $\varphi_g^n = I$ and thus, $\varphi_g$ is diagonalisable, by \Cref{cor:finorderdiagonalisable}. Hence, as before, we may assume that $\varphi_g$ is diagonal. It has eigenvalues $\lambda_1, \ldots, \lambda_m$ where each $\lambda_i$ is an $n$-th root of unity.\\
	The character or trace is now simply the sum of all $\lambda_i.$
\end{proof}

The same proof as earlier also tells us that the function $\chi_\varphi : G \to \mathbb{C}$ is constant on the conjugacy classes of $G.$ More precisely:
\begin{prop} \label{prop:charconstonconjclasses}
	Let $\varphi$ be a representation of $G.$ Then, for all $g, h \in G,$ we have that $\chi_\varphi(g) = \chi_\varphi(hgh^{-1}).$
\end{prop}
\begin{proof} 
	Let $g, h \in G$ and note
	\begin{align*} 
		\chi_\varphi(g) &= \trace \varphi_g\\
		&= \trace (\varphi_{h^{-1}}\varphi_h\varphi_g)\\
		&= \trace (\varphi_h\varphi_g\varphi_{h^{-1}})\\
		&= \trace \varphi_{hgh^{-1}}\\
		&= \chi_\varphi(hgh^{-1}). \qedhere
	\end{align*}
\end{proof}

Functions with the above property have a special name.
\begin{defn}%[Class functions]
	A function $f : G \to \mathbb{C}$ is called a \deff{class function} if $f(g) = f(hgh^{-1})$ for all $g, h \in G.$ The space of all class functions is denoted $Z(L(G)).$
\end{defn}

Thus, we have shown that characters are class functions. Given a conjugacy class $C \subset G$ and a class function $f : G \to \mathbb{C},$ $f(C) \in \mathbb{C}$ will denote the (constant) value taken by elements of $C.$

\begin{prop}
	$Z(L(G))$ is a subspace of the vector space $L(G).$
\end{prop}
\begin{proof} 
	Let $c \in \mathbb{C},$ $f_1, f_2 \in L(G),$ and $h, g \in G$ be arbitrary. Then,
	\begin{align*} 
		(cf_1 + f_2)(hgh^{-1}) &= cf_1(hgh^{-1}) + f_2(hgh^{-1})\\
		&= cf_1(g) + f_2(g)\\
		&= (cf_1 + f_2)(g),
	\end{align*}
	showing that $Z(L(G))$ is closed under linear combinations. Also, note that the zero map is an element of $Z(L(G))$ proving that $Z(L(G)) \le L(G).$
\end{proof}

\begin{defn}
	Given a group $G,$ the set of conjugacy classes of $G$ is denoted $\Cl(G).$ For $C \in \Cl(G),$ we define $\delta_C : G \to \mathbb{C}$ as
	\begin{equation*} 
		\delta_C(g) = \begin{cases}
			1 & g \in C,\\
			0 & g \notin C.	
		\end{cases}
	\end{equation*}
\end{defn}
In other words, $\delta_C$ is just the indicator function of $C \subset G.$

\begin{prop} \label{prop:dimZLGCLG}
	The set $B = \{\delta_C \mid C \in \Cl(G)\}$ is a basis for $Z(L(G)).$ In particular, $\dim Z(L(G)) = \md{\Cl(G)}.$
\end{prop}
\begin{proof} 
	It is clear $\delta_C \in Z(L(G))$ for each $C \in \Cl(G).$ (Note that conjugacy classes partition $G$ and thus, distinct conjugacy classes have empty intersection.)

	\textbf{Spanning.} Let $f \in Z(L(G)).$ One verifies
	\begin{equation*} 
		f = \sum_{C \in \Cl(G)} f(C)\delta_C
	\end{equation*}
	by computing each side at an arbitrary $g \in G.$ This proves that $\spn B = Z(L(G)).$

	\textbf{Linear independence.} Note that
	\begin{equation*} 
		\langle \delta_C, \delta_{C'}\rangle = \frac{1}{\md{G}}\sum_{g \in G} \delta_C(g)\overline{\delta_{C'}(g)} = \begin{cases}
			0 & C \neq C',\\
			\frac{\md{C}}{\md{G}} & C = C'.
		\end{cases}
	\end{equation*}
	Thus, $B$ is a set of orthogonal non-zero vectors and hence, is linearly independent.

	Lastly, note that $\md{B} = \md{\Cl(G)}$ since $C \neq C' \implies \delta_C \neq \delta_{C'}.$ Thus, $\dim Z(L(G)) = \md{B} = \md{\Cl(G)}.$
\end{proof}

\begin{thm}[First orthogonality relations] \label{thm:firstorthorel}
	Let $\varphi, \rho$ be \underline{irreducible} representations of $G.$ Then
	\begin{equation*} 
		\langle \chi_\varphi, \chi_\rho\rangle = \begin{cases}
			1 & \varphi \sim \rho,\\
			0 & \varphi \not\sim \rho.
		\end{cases}
	\end{equation*}
	Thus, the irreducible characters of $G$ form an orthonormal set of class functions. In particular, they are linearly independent.	
\end{thm}
Note that technically, we should have said ``distinct irreducible characters'' in the last line but \Cref{prop:equivrepssamechar} tells us that equivalent representations have equal characters.

\begin{proof} 
	By \Cref{prop:repoffingroupisunitary} and \Cref{prop:equivrepssamechar}, we may assume that $\varphi : G \to U_n(\mathbb{C})$ and $\rho : G \to U_m(\mathbb{C}).$ Now, note that
	\begin{align*} 
		\langle \chi_\varphi, \chi_\rho\rangle &= \frac{1}{\md{G}}\sum_{g \in G} \chi_\varphi(g) \overline{\chi_\rho(g)}\\
		&= \frac{1}{\md{G}}\sum_{g \in G}\left[\left(\sum_{i = 1}^{n}\varphi_{ii}(g)\right)\left(\sum_{j = 1}^{m}\overline{\rho_{jj}(g)}\right)\right]\\
		&= \sum_{\substack{1 \le i \le n\\1 \le j \le m}}\left[\dfrac{1}{\md{G}}\sum_{g \in G}\varphi_{ii}(g)\overline{\rho_{jj}(g)}\right]\\
		&= \sum_{\substack{1 \le i \le n\\1 \le j \le m}}\langle \varphi_{ii}, \rho_{jj}\rangle.
	\end{align*}
	Now, if $\varphi \not\sim \rho,$ then all the terms in the summation are $0,$ by \nameref{thm:schurorthorel}. Now, if $\rho \sim \varphi,$ then we may assume $\rho = \varphi,$ by \Cref{prop:equivrepssamechar}. (Since we are making a statement about the characters only.)

	In this case, \nameref{thm:schurorthorel} tell us that the only non-zero terms in the summation are when $i = j,$ in which case
	\begin{equation*} 
		\langle \varphi_{ii}, \rho_{jj}\rangle = \langle \varphi_{ii}, \varphi_{ii}\rangle = \frac{1}{n}
	\end{equation*}
	and so,
	\begin{equation*} 
		\langle \chi_\varphi, \chi_\rho\rangle = n\cdot\frac{1}{n} = 1.	 \qedhere
	\end{equation*}
\end{proof}

\begin{cor}
	Given two irreducible inequivalent representations $\varphi$ and $\rho$ of $G,$ we have $\chi_\varphi \neq \chi_\rho.$
\end{cor}
\begin{proof} 
	Note that $\langle \chi_\varphi, \chi_\rho\rangle = 0.$ If $\chi_\varphi = \chi_\rho,$ this would force $\chi_\varphi = 0.$ However, this is not possible since $\langle \chi_\varphi, \chi_\varphi\rangle = 1 \neq 0.$
\end{proof}
Note that \Cref{prop:equivrepssamechar} already told us that equivalent characters have the same character. We have now proven the converse for irreducible representations. Thus, we have the following.

\begin{thm} \label{thm:irredrepsequiviffsamechar}
	Two irreducible representations are equivalent if and only they have the same character.
\end{thm}

\begin{cor}
	There are at most $\md{\Cl(G)}$ equivalence classes of irreducible representations of $G.$
\end{cor}
\begin{proof} 
	We have already shown that distinct equivalence classes will have distinct characters. Moreover, we have shown that picking a character from each set gives us a orthonormal (and hence, linearly independent) subset of $Z(L(G))$ and in turn, there can be at most $\dim Z(L(G)) = \md{\Cl(G)}$ many such.
\end{proof}

We now introduce some notation for ease of writing.

\begin{defn}
	If $V$ is a vector space, $\varphi$ a representation, and $m \in \mathbb{N},$ then
	\begin{equation*} 
		mV \vcentcolon= \underbrace{V \oplus \cdots \oplus V}_{m} \andd m\varphi \vcentcolon= \underbrace{\varphi \oplus \cdots \oplus \varphi}_{m}.
	\end{equation*}
	If $m = 0,$ then we define $0V$ to be the zero vector space and $0\varphi$ to be the degree zero representation.
\end{defn}

\begin{rem}
	Note that we had said that we won't consider degree zero representations and we shall continue to do so. The only reason for considering $m = 0$ above is so that when we write an expression as
	\begin{equation*} 
		\rho \sim m_1\varphi^{(1)} \oplus \cdots \oplus m_s\varphi^{(s)},
	\end{equation*}
	then we allow that possibility for some $m_i$ to be $0.$ In that case, we simply ignore $\varphi^{(i)}.$ It will never be the case that each $m_i$ is $0.$
\end{rem}

Our immediate goal now is to prove the uniqueness of decomposition. More precisely, if we are given a transversal of irreducible representatives $\varphi^{(1)}, \ldots, \varphi^{(s)}$ and have
\begin{equation*} 
	\rho \sim m_1\varphi^{(1)} \oplus \cdots \oplus m_s \varphi^{(s)},
\end{equation*}
we want to show that each $m_i$ is uniquely determined. We shall see that this information can be extracted from just the character of $\rho.$

\begin{lem} \label{lem:charactersadd}
	Let $\varphi = \rho \oplus \psi.$ Then $\chi_\varphi = \chi_\rho + \chi_\psi.$
\end{lem}
\begin{proof} 
	We may assume that $\rho : G \to \GL_n(\mathbb{C})$ and $\psi : G \to \GL_m(\mathbb{C}).$ Then, we have the block matrix form for $\varphi : G \to \GL_{n + m}(\mathbb{C})$ with
	\begin{equation*} 
		\varphi_g = \two{\rho_g}{}{}{\psi_g}
	\end{equation*}
	for all $g \in G.$ From the above, it follows that
	\begin{equation*} 
		\trace \varphi_g = \trace \rho_g + \trace \psi_g
	\end{equation*}
	for all $g \in G,$ as desired.
\end{proof}

\begin{thm} \label{thm:innerprodwithchi}
	Let $\varphi^{(1)}, \ldots, \varphi^{(s)}$ be transversal of irreducible representations of $G.$ Suppose $\rho$ is a representation such that
	\begin{equation*} 
		\rho \sim m_1\varphi^{(1)} \oplus \cdots \oplus m_s\varphi^{(s)}.
	\end{equation*}
	Then, $m_i = \langle \chi_\rho, \chi_{\varphi^{(i)}}\rangle.$
\end{thm}
\begin{proof} 
	Note that by definition, $\varphi^{(i)} \not\sim \varphi^{(j)}$ if $i \neq j.$ Thus, by \Cref{thm:firstorthorel}, it follows that 
	\begin{equation} \tag{$*$} \label{eq:002}
		\langle \chi_{\varphi^{(i)}}, \chi_{\varphi^{(j)}}\rangle = \begin{cases}
			0 & i \neq j,\\
			1 & i = j.
		\end{cases}
	\end{equation}

	From the previous lemma, it follows that
	\begin{equation*} 
		\chi_\rho = m_1\chi_{\varphi^{(1)}} + \cdots + m_s\chi_{\varphi^{(s)}}.
	\end{equation*}
	Taking the inner product with $\chi_{\varphi^{(i)}}$ and using \Cref{eq:002} prove the result.
\end{proof}	

\begin{cor}
	The composition of $\rho$ into irreducible characters is unique.
\end{cor}
This is immediate for the ``unique'' just means that $m_i$ is uniquely determined. This actually tells us that we can make sense of something as the ``multiplicity'' of an irreducible representation. This leads to \Cref{defn:multiplicity}.

\begin{cor} \label{cor:characbyequiv}
	$\rho$ is determined, up to equivalence by its character. In particular, \Cref{thm:irredrepsequiviffsamechar} is true in general; that is, two representations are equivalent if and only if their characters are equal.
\end{cor}
\begin{proof} 
	Let $f \vcentcolon= \chi_\rho.$ We show that we can construct a representation equivalent to $\rho$ just in terms of $f.$

	To this end, define $n_i \vcentcolon= \langle f, \chi_{\varphi^{(i)}}\rangle$ and set
	\begin{equation*} 
		\varphi \vcentcolon= n_1\varphi^{(1)} \oplus \cdots \oplus n_s\varphi^{(s)}.
	\end{equation*}

	We claim that $\varphi \sim \rho.$ To see this, note that by \nameref{thm:maschke}, $\rho$ is completely reducible and there exists a decomposition of $\rho$ as 
	\begin{equation*} 
		\rho \sim \rho^{(1)} \oplus \cdots \oplus \rho^{(s')}.
	\end{equation*}
	By construction, $\varphi^{(1)}, \ldots, \varphi^{(s)}$ are the only irreducible representations, up to equivalence. Thus, each $\rho^{(j)}$ is equivalent to some $\varphi^{(i)}.$ By clubbing the representations in the same equivalence class together, we get
	\begin{equation*} 
		\rho \sim m_1\varphi^{(1)} \oplus \cdots \oplus m_s\varphi^{(s)}.
	\end{equation*}
	However, $m_i = n_i$ for each $i,$ by the previous theorem and hence, $\rho \sim \varphi.$
\end{proof}

\begin{cor} \label{cor:irrediffnormone}
	A representation $\rho$ is irreducible if and only if $\langle \chi_\rho, \chi_\rho\rangle = 1.$
\end{cor}
\begin{proof} 
	As before, write $\rho \sim m_1\varphi^{(1)} \oplus \cdots \oplus m_s\varphi^{(s)}$ and note that 
	\begin{equation*} 
		\langle \chi_\rho, \chi_\rho\rangle = m_1^2 + \cdots + m_s^2.
	\end{equation*}
	Thus, $\langle \chi_\rho, \chi_\rho\rangle = 1$ iff there exists $j$ such that $m_j = 1$ and $m_i = 0$ for all $i \neq j$ iff $\rho \sim \varphi^{(j)}$ for some $j$ iff $\rho$ is irreducible.
\end{proof}

\begin{rem} \label{rem:normcharacisnatural}
	The above calculation also shows us that $\|\chi\|$ is always a positive integer.
\end{rem}
\begin{cor} \label{cor:innerprodproperties}
	In fact, we have the following observations:
	\begin{enumerate}
		\item $\|\chi\| \in \mathbb{N}$ with $\|\chi\| = 1$ iff $\chi$ is irreducible.
		\item $\langle \chi_1, \chi_2\rangle \in \mathbb{N}_0.$ In particular, the inner product is always real. Note that the characters themselves are complex valued and not necessarily real.
	\end{enumerate}
\end{cor}

\begin{cor} \label{cor:multiplyingdegonerep}
	Let $z : G \to \mathbb{C}^*$ be a degree one representation and $\rho : G \to \GL_n(\mathbb{C})$ a representation. Then, $\varphi : G \to \GL_n(\mathbb{C})$ defined as
	\begin{equation*} 
		\varphi_g = z_g \rho_g
	\end{equation*}
	is a representation. Furthermore, the equalities
	\begin{equation*} 
		\chi_\varphi = z \chi_\rho \andd \langle \chi_\varphi, \chi_\varphi\rangle = \langle \chi_\rho, \chi_\rho\rangle
	\end{equation*}
	hold.

	In particular,
	\begin{enumerate}
		\item $\varphi$ is irreducible if and only if $\rho$ is;
		\item if there exists $g_0 \in G$ such that $z_{g_0} \neq 1$ and $\chi_\varphi(g_0) \neq 0,$ then $\rho \not\sim \varphi.$
	\end{enumerate} 
\end{cor}
\begin{proof} 
	First we show that $\varphi$ is indeed a representation. This simple for $z_{g_2}\rho_{g_1} = \rho_{g_1}z_{g_2}$ for any $g_1, g_2 \in G$ which gives
	\begin{equation*} 
		\varphi_{g_1g_2} = \varphi_{g_1} \varphi_{g_2},
	\end{equation*}
	as desired.

	Moreover, we also note that
	\begin{equation*} 
		\trace \varphi_g = \trace (z_g \rho_g) = z_g \trace \rho_g
	\end{equation*}
	or
	\begin{equation} \tag{$\star$} \label{eq:004}
		\chi_\varphi(g) = z_g \chi_\rho(g).
	\end{equation}
	This proves the first equality. The above also yields
	\begin{equation*} 
		\md{\chi_\varphi(g)}^2 = \md{z_g}^2 \md{\chi_\rho(g)}^2.
	\end{equation*}
	Recall that since $G$ is finite, $z_g^{\md{G}} = 1$ and hence, $\md{z_g} = 1,$ which gives us
	\begin{equation} \tag{$*$} \label{eq:003}
		\md{\chi_\varphi(g)}^2 = \md{\chi_\rho(g)}^2.
	\end{equation}

	To see that the second equality, note that
	\[\begin{WithArrows}[displaystyle]
		\langle \chi_\varphi, \chi_\varphi\rangle &= \frac{1}{\md{G}}\sum_{g \in G} \chi_\varphi(g)\overline{\chi_\varphi(g)}\\
		&= \frac{1}{\md{G}} \sum_{g \in G} \md{\chi_\varphi(g)}^2 \Arrow{\Cref{eq:003}}\\
		&= \frac{1}{\md{G}} \sum_{g \in G} \md{\chi_\rho(g)}^2\\
		&= \langle \chi_\rho, \chi_\rho\rangle
	\end{WithArrows}\]
	By \Cref{cor:irrediffnormone}, irreducibility is equivalent to the above inner product being $1.$

	We now prove the last statement. For this, we will use \Cref{prop:equivrepssamechar}. \\
	Let $g_0$ be as in the theorem; then, by \Cref{eq:004}, we see that
	\begin{equation*} 
		\chi_\rho(g_0) = z_{g_0}^{-1} \chi_\varphi (g_0) \neq \chi_\varphi (g_0).
	\end{equation*}
	Thus, by \Cref{prop:equivrepssamechar}, we have $\rho \not\sim \varphi.$
\end{proof}

\begin{rem}
	Note that the last part of the theorem is really just asking us to look at the characters of $\rho$ and $\varphi$ and conclude inequivalence.

	Also, note that \Cref{eq:004} tells us that the character of $\varphi$ is obtained by multiplying $\chi_z$ and $\chi_\rho.$ (Recall that character of a degree one representation is the representation itself.)
\end{rem}

\begin{ex}
	Let us use the above corollary to show that the representation $\rho$ of $S_3$ in \Cref{ex:S3GL2Crho} is irreducible. (We had already done this earlier in \Cref{ex:showingS3GL2Crhoisirred}.)

	Recalling \nameref{thm:descconjclassSn}, we see that there are exactly three conjugacy classes in $S_3,$ namely, $[1],\;[(12)],\;[(123)].$ These have cardinalities $1,\;3,\;2,$ respectively. 

	Note that $\chi_\rho(1) = 2,$ $\chi_\rho\left((12)\right) = 0,$ and $\chi_\rho\left((123)\right) = -1.$

	Since characters are class functions, we see that
	\begin{align*} 
		\langle \chi_\rho, \chi_\rho\rangle &= \frac{1}{6}\sum_{\sigma \in S_3} \chi_p(\sigma)\overline{\chi_p(\sigma)}\\
		&= \frac{1}{6}(1\cdot2^2 + 3\cdot0^2 + 2\cdot(-1)^2)\\
		&= \frac{1}{6}(6) = 1.
	\end{align*}
\end{ex}
\begin{ex}[Character table of $S_3$]
	The previous example gives us an irreducible degree two representation of $S_3.$ \Cref{ex:degonerepsSn} had given us two degree one (inequivalent and irreducible) representations. Since the number of conjugacy classes of $S_3$ is $3,$ these are all. (Of course, using that $S_3 \cong D_3,$ we knew this already.)

	Let $\chi_1$ denote the character of the the trivial representation, $\chi_2$ of the $\sign$ representation, and $\chi_3$ of the representation from the previous example.

	Each of these are class functions, that is, constant on the conjugacy classes. Thus, we can construct something called the ``character table.''

	\captionsetup{type=figure}
	\[\begin{array}{rrrr}
		 & [1] & [(12)] & [(123)]\\
		\thiccline
		\chi_1 & 1 & 1 & 1\\
		\chi_2 & 1 & -1 & 1\\
		\chi_3 & 1 & 0 & -1	
	\end{array}\]
	\captionof{table}{Character table of $S_3$} \label{tab:charS3}
	
\end{ex}

\begin{ex}[Revisiting a representation of $S_3$]
	Let us again turn back to \Cref{ex:S3GL2Crho}. We had remarked that we shall show that $\rho \oplus \psi$ is equivalent to the standard representation from \Cref{ex:standardrepSn}.

	To see this, now we simply compute the character of the standard representation $\varphi.$

	Computing it at $1,\;(12),\;(123),$ we see that the table is as follows.

	\[\begin{array}{rrrr}
		 & [1] & [(12)] & [(123)]\\
		\thiccline
		\chi_\varphi & 3 & 1 & 0
	\end{array}\]

	From the above table, it is evident that
	\begin{equation*} 
		\chi_\varphi = \chi_1 + \chi_3,
	\end{equation*}
	where we have retained the notation from the previous example. In turn, this establishes the desired equivalence.
\end{ex}

\begin{defn}%[Multiplicity]
	\label{defn:multiplicity}
	Let $G$ be a finite group and $\varphi^{(1)}, \ldots, \varphi^{(s)}$ be a transversal of irreducible unitary representations of $G.$ Set $d_i \vcentcolon= \deg \varphi^{(i)}.$

	If $\rho \sim m_1\varphi^{(1)} \oplus \cdots \oplus m_s\varphi^{(s)},$ then $m_i$ is called the \deff{multiplicity} of $\varphi^{(i)}$ in $\rho.$ If $m_i > 0,$ then we say that $\varphi^{(i)}$ is an \deff{irreducible constituent} of $\rho.$
\end{defn}

\begin{rem}
	With the same notation, we have
	\begin{equation*} 
		\deg \rho = m_1d_1 + \cdots + m_sd_s.
	\end{equation*}
\end{rem}

The result in the proof of \Cref{cor:characbyequiv} is important and so, we isolate it below.

\begin{thm} \label{thm:calcequivfromchar}
	Let $G$ be a finite group and $\rho$ a representation. Let $\varphi^{(1)}, \ldots, \varphi^{(s)}$ be as earlier. Define, $m_i \vcentcolon= \langle \chi_\rho, \chi_{\varphi^{(i)}}\rangle.$ Then,
	\begin{equation*} 
		\rho \sim m_1\varphi^{(1)} \oplus \cdots \oplus m_s\varphi^{(s)}.
	\end{equation*}
\end{thm}

Note the similarity with inner product spaces where the coefficients of a vector with respect to an orthonormal basis is given by the inner product. The similarity is not surprising since the theorems and corollaries above actually tell us how the above equivalence of representations translates to equality of characters in the inner product space $Z(L(G)).$

\subsection{The Regular Representation}
Recall from \Cref{subsec:linearisation}, the concept of \nameref{defn:linearisation}.

\begin{defn}%[Regular representation]
	\label{defn:regularrepresentation}
	Let $G$ be a finite group. The \deff{regular representation} of $G$ is the homomorphism $L : G \to \GL(\mathbb{C}G)$ defined by
	\begin{equation*} 
		L_g\left(\sum_{h \in G} c_h h\right) = \sum_{h \in G} c_hgh = \sum_{x \in G} c_{g^{-1} x}x
	\end{equation*}
	for all $g \in G.$
\end{defn}

\begin{rem} \label{rem:degofregrep}
	Note that since $G$ is a basis for $\mathbb{C}G,$ we have that $\deg L = \md{G}.$
\end{rem}
\begin{rem}
	Of course, one must now verify that $L_g$ is actually an element of $\GL(\mathbb{C}G)$ and that $L$ is a homomorphism.
	
	The above can be seen permuting the coefficients of a given element of $\mathbb{C}G.$ Its action on the (natural) basis vectors can be seen as follows:
	\begin{equation*} 
		L_gh = gh.
	\end{equation*}
	In other words, $L_g$ acts on basis vectors by (left) multiplication by $g$ and the (unique) map obtained by extending it linearly to all of $\mathbb{C}G$ gives us the map $L_g.$ (cf. \Cref{prop:linearfuncextend}.) The $L$ stands for ``left.''
\end{rem}

\begin{prop}
	The regular representation is a unitary representation of $G.$ In particular, it is indeed a representation.
\end{prop}
\begin{proof} 
	% As remarked earlier, $L_g$ is indeed linear for all $g \in G.$ Now, let $g_1, g_2 \in G$ and let $h \in G \hookrightarrow \mathbb{C}G$ be a basis element. Note that
	% \begin{equation*} 
	% 	(L_{g_1} \circ L_{g_2})(h) = L_{g_1}(g_2h) = g_1g_2h = L_{g_1g_2}(h).
	% \end{equation*}
	% Thus, $L_{g_1} \circ L_{g_2}$ and $L_{g_1g_2}$ are linear transformations which agree on all basis elements. Thus, they must be equal. This proves the invertibility of $L_g$ (for all $g \in G$) as well as the fact that $L$ is a homomorphism.

	% Thus, $L$ is a representation. 

	The fact that $L$ is a representation follows from \Cref{prop:extendingactiontorep}.

	To see that it is unitary, note that
	\[\begin{WithArrows}[displaystyle]
		\left\langle L_g\sum_{h \in G} c_hh, L_g\sum_{h \in G} k_hh\right\rangle &= \left\langle \sum_{x \in G} c_{g^{-1}x} x, \sum_{x \in G} k_{g^{-1}x} x\right\rangle\\
		&= \sum_{x \in G} c_{g^{-1}x}\overline{k_{g^{-1}x}} \Arrow{$x \mapsto gy$}\\
		&= \sum_{y \in G} c_y\overline{k_y}\\
		&= \left\langle \sum_{h \in G} c_hh, \sum_{h \in G} k_hh \right\rangle,
	\end{WithArrows}\]
	as desired.
\end{proof}

\begin{prop} \label{prop:charofregrep}
	The character of the regular representation $L$ is given as
	\begin{equation*} 
		\chi_L(g) = \begin{cases}
			\md{G} & g = 1,\\
			0 & g \neq 1.
		\end{cases}
	\end{equation*}
\end{prop}
\begin{proof} 
	For $g = 1,$ note that $\chi_L(1) = \deg L,$ by \Cref{prop:charatidisdeg} and $\deg L = \md{G},$ by \Cref{rem:degofregrep}.

	We now compute the character for $g \neq 1.$ Let $n \vcentcolon= \md{G}$ and write
	\begin{equation*} 
		G = (g_1, \ldots, g_n).
	\end{equation*}
	(We are using tuple notation to denote that we have fixed an order.) 

	Now, we look at the matrix representation $[L_g]$ of $L_g$ with respect to this ordered basis $G.$

	We contend that all the diagonal entries of $[L_g]$ are $0.$\\
	Indeed, for any $g_i \in G,$ we have $gg_i = g_j \neq g_i.$ (Since $g \neq 1.$) \\
	Thus, the $i$-the entry in the $i$-th column will be $0.$ It follows at once that 
	\begin{equation*} 
		\chi_L(g) = \trace L_g = \trace [L_g] = 0,
	\end{equation*} 
	as desired.
\end{proof}

\begin{rem}
	Note that from the above, we can conclude the following.
	\begin{align*} 
		\langle \chi_L, \chi_L\rangle &= \frac{1}{\md{G}}\sum_{g \in G} \chi_L(g)\overline{\chi_L(g)}\\
		&= \dfrac{1}{\md{G}}\md{G}^2\\
		&= \md{G}.
	\end{align*}
	In particular, if $G$ is non-trivial, then $L$ is \emph{not} irreducible. In fact, the next proposition shows us exactly the description of $L$ in terms of its decomposition.
\end{rem}

We shall now fix the following notation: $G$ is a finite group and $\left\{\varphi^{(1)}, \ldots, \varphi^{(s)}\right\}$ is a transversal of inequivalent irreducible \textbf{unitary} representatives of $G.$ As usual, $d_i \vcentcolon= \deg \varphi^{(i)}.$ Moreover, $\chi_i \vcentcolon= \chi_{\varphi^{(i)}}.$

\begin{prop} \label{prop:descripofL}
	Let $L$ denote the regular representation of $G.$ Then,
	\begin{equation*} 
		L \sim d_1 \varphi^{(1)} \oplus \cdots \oplus d_s \varphi^{(s)}.
	\end{equation*}
	In particular, the equality $\md{G} = d_1^2 + \cdots + d_s^2$ holds.
\end{prop}
\begin{proof} 
	By \Cref{thm:calcequivfromchar}, it suffices to show that $\langle \chi_L, \chi_i\rangle = d_i$ holds. To that end, note that
	\begin{align*} 
		\langle \chi_L, \chi_i\rangle &= \dfrac{1}{\md{G}}\sum_{g \in G} \chi_L(g)\overline{\chi_i(g)}\\
		&= \dfrac{1}{\md{G}}\chi_L(1)\overline{\chi_i(1)}\\
		&= \dfrac{1}{\md{G}}\md{G}\deg \varphi^{(i)}\\
		&= \deg \varphi^{(i)} = d_i,
	\end{align*}
	as desired.

	Now, note that by the \Cref{prop:descripofL} and \Cref{lem:charactersadd}, we see that
	\begin{equation*} 
		\chi_L = d_1\chi_1 + \cdots + d_s\chi_s.
	\end{equation*}
	Evaluating both sides at $1$ finishes the proof.
\end{proof}

\begin{rem} \label{rem:regrepcontainsallreps}
	The above shows that every irreducible representation of $G$ appears as a constituent in its regular representation.
\end{rem}

\begin{cor} \label{cor:orthnormalbasis}
	The set $B = \left\{\sqrt{d_k}\varphi_{ij}^{(k)} \mid 1 \le k \le s,\;1 \le i, j \le d_k\right\}$ is an orthonormal basis of $L(G).$
\end{cor}
\begin{proof}
	By \Cref{prop:fingroupirredbounds}, we already know that it is orthonormal and hence, linearly independent. On the other hand, note that
	\begin{equation*} 
		\md{B} = d_1^2 + \cdots + d_s^2 = \md{G} = \dim L(G). \qedhere
	\end{equation*}
\end{proof}

\begin{ex}[Number of irreducible representations of $D_n$] \label{ex:numirredrepsDn}
	Note that by \Cref{ex:degonerepsDn}, we know the exact number of degree one representations of $D_n.$ By \Cref{ex:irredrepDndegbound}, we know that all other irreducible representations must have degree two.

	Now, let $t_n$ denote the number of inequivalent irreducible degree two representations of $D_n.$ We shall now calculate $t_n,$ using \Cref{prop:descripofL}.

	\textbf{Case 1.} $n = 2k + 1.$\\
	In this case, there are $2$ inequivalent degree one representations. Thus, we see that
	\begin{equation*} 
		2 \cdot 1^2 + t_n \cdot 2^2 = \md{D_n} = 4k + 2
	\end{equation*}
	which gives us
	\begin{equation*} 
		t_n = k = \frac{n - 1}{2}.
	\end{equation*}

	\textbf{Case 2.} $n = 2k.$\\
	In this case, there are $4$ inequivalent degree one representations. Thus, we see that
	\begin{equation*} 
		4 \cdot 1^2 + t_n \cdot 2^2 = \md{D_n} = 4k
	\end{equation*}
	which gives us
	\begin{equation*} 
		t_n = k - 1 = \frac{n}{2} - 1.
	\end{equation*}

	Thus, we get the total number of inequivalent irreducible representations as
	\begin{align*} 
		\frac{n + 3}{2} &\quad \text{if }n \text{ is odd,}\\
		\frac{n}{2} + 3 &\quad \text{if }n \text{ is even.}
	\end{align*}
\end{ex}

\begin{ex}[Finishing off $D_n$] \label{ex:finishingDn}
	With the above calculations, we now finish the study of irreducible representations of $D_n.$ Fix $n \ge 3.$

	Let us first set up the notation as follows: $\theta \vcentcolon= \frac{2\pi}{n}$ and
	\begin{equation*} 
		A_k \vcentcolon= \two{\cos k\theta}{\sin k\theta}{-\sin k\theta}{\cos k\theta}
	\end{equation*}
	for $k \in \{0, \ldots, n - 1\}.$

	Also, let
	\begin{equation*} 
		A \vcentcolon= \two{1}{0}{0}{-1}.
	\end{equation*}

	As the reader might have guessed, the above matrices do indeed satisfy the following relations:
	\begin{equation*} 
		A_k^n = A^2 = (A_kA)^2 = I_n
	\end{equation*}
	and hence $r \mapsto A_k,\; s \mapsto A$ defines a two dimensional representation $\varphi_k$ of $D_n.$

	Our goal is now to identify as many irreducible and pairwise inequivalent representations as possible. We shall end up showing that we get precisely $t_n$ many. ($t_n$ being as in \Cref{ex:numirredrepsDn}.)

	First, we note that the eigenvector of $A$ (up to scaling) are $e_1$ and $e_2.$ Thus, if $\sin k\theta \neq 0,$ then $\varphi_k$ is irreducible. (\Cref{prop:deg2repirreducible}.) Thus, we need to ensure that $k\theta \neq 0, \pi.$

	Second, we need to see when two irreducible representations above are actually inequivalent. The answer is actually quite simple, in view of \Cref{thm:irredrepsequiviffsamechar}. One notes that
	\begin{equation*} 
		\chi_{\varphi_k}(r) = \trace \varphi_k(r) = 2\cos k\theta
	\end{equation*}
	and hence, $\varphi_k \not\sim \varphi_{k'}$ if $\cos k\theta \neq \cos k'\theta.$ Noting that $k\theta, k'\theta \in [0, 2\pi),$ simple trigonometry tells us that
	\begin{equation*} 
		\cos k\theta = \cos k'\theta \iff k = k', \frac{2\pi}{\theta} - k \iff k = k', n - k.
	\end{equation*}

	Thus, if we looks at $k \in \{1, \ldots, n - 1\}$ such that $k\theta < \pi,$ we see that all the $\varphi_k$ are pairwise inequivalent.

	If $n$ is even, then there are $\frac{n}{2} - 1$ such $k$ and if $n$ is odd, then there are $\frac{n-1}{2}$ many such. However, by \Cref{ex:numirredrepsDn}, there are no more and we are done!
\end{ex}

\begin{thm} \label{thm:onbforzlg}
	The set $B = \{\chi_1, \ldots, \chi_s\}$ is an orthonormal basis for $Z(L(G)).$
\end{thm}
\begin{proof} 
	We shall assume that $\varphi^{(i)} : G \to U_{d_i}(\mathbb{C})$ since we wish to use \Cref{prop:hashpropertiesirredrep}. Since our statement is about characters, which is unaffected by equivalence, our claim follows.

	Note that we know that $B \subset Z(L(G))$ since characters are indeed class functions. Moreover, we know that $B$ is an orthonormal set, by \nameref{thm:firstorthorel}. Thus, only spanning needs to be shown.

	To this end, let $f \in Z(L(G)) \le L(G)$ be given. By the previous corollary, we see that
	\begin{equation*} 
		f = \sum_{i, j, k} c_{ij}^{(k)}\varphi_{ij}^{(k)},
	\end{equation*}
	for some $c_{ij}^{(k)} \in \mathbb{C},$ $1 \le k \le s,$ $1 \le i, j \le d_k.$ Let $x \in G$ be arbitrary. Note that

	\[\begin{WithArrows}[displaystyle]
		f(x) &= \dfrac{1}{\md{G}}\sum_{g \in G} f(x) \Arrow{$f \in Z(L(G))$}\\
		&=\dfrac{1}{\md{G}}\sum_{g \in G} f(g^{-1}xg)\\
		&=\dfrac{1}{\md{G}}\sum_{g \in G} \sum_{i, j, k} c_{ij}^{(k)}\varphi_{ij}^{(k)}(g^{-1}xg)\\
		&= \sum_{i, j, k}\dfrac{1}{\md{G}}\sum_{g \in G} c_{ij}^{(k)}\varphi_{ij}^{(k)}(g^{-1}xg)\\
		&= \sum_{i, j, k} c_{ij}^{(k)}\dfrac{1}{\md{G}}\sum_{g \in G}\varphi_{ij}^{(k)}(g^{-1}xg)\\
		&= \sum_{i, j, k} c_{ij}^{(k)}\dfrac{1}{\md{G}}\sum_{g \in G}\left[\varphi^{(k)}(g^{-1}xg)\right]_{ij}\\
		&= \sum_{i, j, k} c_{ij}^{(k)}\left[\dfrac{1}{\md{G}}\sum_{g \in G}\varphi^{(k)}(g^{-1}xg)\right]_{ij} \Arrow{$\varphi^{(k)}$ is a representation}\\
		&= \sum_{i, j, k} c_{ij}^{(k)}\left[\dfrac{1}{\md{G}}\sum_{g \in G}\varphi_{g^{-1}}^{(k)}\varphi_{x}^{(k)}\varphi_{g}^{(k)}\right]_{ij} \Arrow{$\#$ with respect to $(\varphi, \varphi)$}\\
		&= \sum_{i, j, k} c_{ij}^{(k)}\left[(\varphi_x^{(k)})^{\#}\right]_{ij} \Arrow{\Cref{item:005} of \Cref{prop:hashpropertiesirredrep}}\\
		&= \sum_{i, j, k} c_{ij}^{(k)}\frac{\trace \varphi_x^{(k)}}{\deg \varphi^{(k)}}I_{ij}\Arrow{$I_{ij} = 0$ if $i \neq j$ and $I_{ii} = 1$}\\
		&= \sum_{i, k} c_{ii}^{(k)}\frac{\trace \varphi_x^{(k)}}{\deg \varphi^{(k)}} \Arrow{definition of $d_k$ and $\chi$}\\
		&= \sum_{i, k} c_{ii}^{(k)}\frac{\chi_k(x)}{d_k}.
	\end{WithArrows}\]

	This shows that
	\begin{equation*} 
		f = \sum_{1 \le k \le s} \left[\sum_{1 \le i \le d_k}\frac{c_{ii}^{(k)}}{d_k}\right]\chi_k. \qedhere
	\end{equation*}
\end{proof}

\begin{cor} \label{cor:numirredrepsconjclass}
	The number of equivalence classes of irreducible representations of $G$ is number of conjugacy classes of $G.$
\end{cor}
\begin{proof} 
	By the above theorem, we have $s = \dim Z(L(G)).$ By \Cref{prop:dimZLGCLG}, we have $\dim Z(L(G)) = \md{\Cl(G)},$ as desired.
\end{proof}

\begin{ex}[Number of conjugacy classes of $D_n$]
	By \Cref{ex:numirredrepsDn}, we know the number of inequivalent irreducible representations of $D_n.$ By the previous corollary, this is also the number of conjugacy classes of $D_n.$
\end{ex}

\begin{cor} \label{cor:numberofirredrepsofG}
	Let $G$ be a finite group. Then, $G$ has $\md{G}$ equivalence classes of irreducible representations if and only if $G$ is abelian.
\end{cor}
\begin{proof} 
	$\md{G} = \md{\Cl(G)}$ holds if and only if $G$ is abelian.
\end{proof}

\begin{cor}
	Let $G$ be a finite group. Then, $G$ is abelian if and only if all the irreducible representations of $G$ have degree one.
\end{cor}
\begin{proof} 
	The ``only if'' was proven in \Cref{thm:irredabelgroup}.

	To prove the ``if'' part, note that if $G$ is not abelian, then $s < \md{G}.$ On the other hand
	\begin{equation*} 
		d_1^2 + \cdots + d_s^2 = \md{G}.
	\end{equation*}
	Thus, at least one $d_i$ is at least $2.$ In other words, there is a non-degree-one irreducible representation of $G.$
\end{proof}

\begin{defn}%[Character table]
	\label{defn:charactertable}
	Let $G$ be a finite group with irreducible $\chi_1, \ldots, \chi_s$ and conjugacy classes $C_1, \ldots, C_s.$ The \deff{character table} of $G$ is the $s \times s$ matrix $\mathsf{X}$ with $\mathsf{X}_{ij} = \chi_i(C_j).$ In other words, the rows of $\mathsf{X}$ are indexed by the characters of $G$ and columns by the conjugacy classes; the $(ij)$-th entry of $\mathsf{X}$ denotes the value of the $i$-th character on the $j$-th conjugacy class.
\end{defn}

Note that the fact that the above table is square (that is, the number of irreducible characters equals the number of conjugacy classes) is due to \Cref{cor:numirredrepsconjclass}. We had seen an example of the character table of $S_3.$ (Recall \Cref{tab:charS3}.)

\begin{ex}[Character table of $\mathbb{Z}/n\mathbb{Z}$]
	As noted earlier, the character of a degree one representation is simply the representation itself. Thus, we get the table as follows. To make the table look more natural, we shall consider $\mathbb{Z}/n\mathbb{Z}$ as the $n$-th roots of unity.

	Recall the $n$ representations $\varphi^{(0)}, \ldots, \varphi^{(n-1)}$ from \Cref{ex:ZnZCstardeg1}. Letting $\chi_k \vcentcolon= \chi_{\varphi^{(k)}},$ we get the following character table.

	\captionsetup{type=figure}
	\[\begin{array}{rrrrr}
		 & [1] & [\omega_n] & \cdots & [\omega_n^{n-1}]\\
		\thiccline
		\chi_0 & 1 & 1 & \cdots & 1\\
		\chi_1 & 1 & \omega_n & \cdots & \omega_n^{n-1} \\
		\chi_2 & 1 & \omega_n^2 & \cdots & \omega_n^{2(n - 1)}\\
		\vdots & \vdots & \vdots & \ddots & \vdots\\
		\chi_{n-1} & 1 & \omega_n^{n-1} & \cdots & \omega_n^{(n-1)^2}
	\end{array}\]
	\captionof{table}{Character table of $\mathbb{Z}/n\mathbb{Z}$} \label{tab:charZnZ}

	The astute reader might have noticed that the columns are orthogonal. To make things more concrete, let us consider $n = 4,$ in which case the table becomes as follows.

	\[\begin{array}{rrrrr}
		 & [1] & [\iota] & [-1] & [-\iota]\\
		\thiccline
		\chi_0 & 1 & 1 & 1 & 1\\
		\chi_1 & 1 & \iota & -1 & -\iota\\
		\chi_2 & 1 & -1 & 1 & -1\\
		\chi_3 & 1 & -\iota & -1 & \iota
	\end{array}\]

	Note that this was the case in \Cref{tab:charS3}. In could do a computation for two general columns in \Cref{tab:charZnZ} and conclude the same. Instead of doing that, we now prove that this is always the case.	
\end{ex}

To do that, we first note that if $C$ and $C'$ are conjugacy classes of $G,$ then the inner product of their columns is given by
\begin{equation*} 
	\sum_{i = 1}^{s}\chi_i(g)\overline{\chi_i(h)},
\end{equation*}
where $g$ (resp. $h$) is any element of $C$ (resp. $C'$).

Retaining the notation as in \Cref{defn:charactertable}, we get the following theorem.

\begin{thm}[Second orthogonality relations] \label{thm:secondorthorel}
	Let $C, C'$ be conjugacy classes of $G$ and let $g \in C$ and $h \in C'.$ Then
	\begin{equation*} 
		\sum_{i = 1}^{s} \chi_i(g)\overline{\chi_i(h)} = \begin{cases}
			\md{G}/\md{C} & C = C',\\
			0 & C \neq C'.
		\end{cases}
	\end{equation*}
	Consequently, the columns of the character table are orthogonal and the matrix $\mathsf{X}$ is invertible.
\end{thm}

\begin{proof} 
	Note that since $\{\chi_i\}$ form an orthonormal basis for $Z(L(G))$ and $\delta_{C'} \in Z(L(G)),$ we get that
	\begin{equation*} 
		\delta_{C'} = \sum_{i = 1}^{s} \langle \delta_{C'}, \chi_i\rangle \chi_i.
	\end{equation*}
	Thus, (where $g$ is as in the theorem) we get
	\begin{align*} 
		\delta_{C'}(g) &= \sum_{i = 1}^{s} \langle \delta_{C'}, \chi_i\rangle \chi_i(g)\\
		&= \sum_{i = 1}^{s} \dfrac{1}{\md{G}}\sum_{x \in G} \delta_{C'}(x)\overline{\chi_i(x)} \chi_i(g)\\
		&= \sum_{i = 1}^{s} \frac{1}{\md{G}} \sum_{x \in C'} \delta_{C'}(x)\overline{\chi_i(x)} \chi_i(g)\\
		&= \frac{1}{\md{G}}\sum_{i = 1}^{s} \sum_{x \in C'}\overline{\chi_i(x)} \chi_i(g).
	\end{align*}
	Noting that $\chi_i$ is a class function and that $h \in C',$ the above simplifies as following.
	\begin{align*} 
		\delta_{C'}(g) &= \frac{1}{\md{G}}\sum_{i = 1}^{s} \sum_{x \in C'}\overline{\chi_i(h)} \chi_i(g)\\
		&= \frac{1}{\md{G}}\sum_{i = 1}^{s} \md{C'}\overline{\chi_i(h)} \chi_i(g)\\
		&= \frac{\md{C'}}{\md{G}}\sum_{i = 1}^{s} \chi_i(g)\overline{\chi_i(h)}.
	\end{align*}
	Rearranging gives us
	\begin{equation*} 
		\sum_{i = 1}^{s} \chi_i(g)\overline{\chi_i(h)} = \frac{\md{G}}{\md{C'}}\delta_{C'}(g).
	\end{equation*}
	Noting that $\delta_{C'}(g) \neq 0 \iff \delta_{C'}(g) = 1 \iff g \in C' \iff C = C'$ yields the result.
\end{proof}

\subsection{Representations of Abelian Groups}
We now conclude this section with completing our discussion of finite abelian groups. By \Cref{thm:irredabelgroup}, we know that every degree one representation of $G$ has degree one. Moreover, by \Cref{cor:numberofirredrepsofG}, we know that there are $\md{G}$ many such. We now explicitly calculate all of these.

Note that the structure theorem of finite abelian groups tells us that every such group is a direct product of cyclic groups. Since we already know explicitly these representations (and their character tables) by \Cref{ex:ZnZCstardeg1}, we would get a complete description for all abelian groups.

\begin{prop}
	Let $G_1,$ $G_2$ be finite abelian groups with $m = \md{G_1}$ and $n = \md{G_2}.$ Suppose that $\rho_1, \ldots, \rho_m$ and $\varphi_1, \ldots, \varphi_n$ are all the irreducible representations of $G_1$ and $G_2,$ respectively. The functions $\alpha_{ij} : G_1 \times G_2 \to \mathbb{C}$ with $1 \le i \le m$ and $1 \le j \le n$ given by
	\begin{equation*} 
		\alpha_{ij}(g_1, g_2) = \rho_i(g_1)\varphi_j(g_2)
	\end{equation*}
	form a complete set of irreducible representations of $G_1 \times G_2.$
\end{prop}
\begin{proof} 
	Note that it suffices to show that each $\alpha_{ij}$ is a homomorphism. Indeed, the fact that each $\alpha_{ij}$ irreducible follows from the fact that it is degree one. Moreover, the fact that $\{\alpha_{ij}\}_{1 \le i \le m}^{1 \le j \le n}$ forms a complete set will follow once we show that all the $mn$ $\alpha_{ij}$s are distinct.

	\textbf{Homomorphism.} Note that a degree one representation is simply a map into $\mathbb{C}^*$ and thus, commutativity gives us that

	\begin{align*} 
		\alpha_{ij}\left((g_1, g_2)(g_1', g_2')\right) &= \alpha_{ij}(g_1g_1', g_2g_2')\\
		&= \rho_i(g_1g_1')\varphi_j(g_2g_2')\\
		&= \rho_i(g_1)\rho_i(g_1')\varphi_j(g_2)\varphi_j(g_2')\\
		&= \rho_i(g_1)\varphi_j(g_2)\rho_i(g_1')\varphi_j(g_2')\\
		&= \alpha_{ij}(g_1, g_2)\alpha_{ij}(g_1', g_2').
	\end{align*}

	\textbf{Distinctness.} Suppose that $\alpha_{ij} = \alpha_{kl}.$ Then, note that
	\begin{equation*} 
		\rho_i(g_1) = \alpha_{ij}(g_1, 1) = \alpha_{kl}(g_1, 1) = \rho_k(g_1),
	\end{equation*}
	for all $g_1 \in G_1.$ Thus, $i = k.$ Similarly, analysing $\alpha_{ij}(1, g_2)$ for $g_2 \in G_2$ yields $j = l,$ as desired.	
\end{proof}

Note that character of a degree one representation is the representation itself. The above proposition easily gives us the character table of the products now.

\begin{ex}[Character table of the Klein group]
	Note that we have the following character table for $\mathbb{Z}/2\mathbb{Z}.$

	\[\begin{array}{rrr}
		 & [0] & [1]\\
		\thiccline
		\chi_1 & 1 & 1\\
		\chi_2 & 1 & -1
	\end{array}\]

	Looking at the products, we get the following table for $\mathbb{Z}/2\mathbb{Z} \times \mathbb{Z}/2\mathbb{Z}.$

	\captionsetup{type=figure}
	\[\begin{array}{rrrrr}
		 & [(0, 0)] & [(0, 1)] & [(1, 0)] & [(1, 1)]\\
		\thiccline
		\chi_{11} & 1 & 1 & 1 & 1\\
		\chi_{12} & 1 & -1 & 1 & -1\\
		\chi_{21} & 1 & 1 & -1 & -1 \\
		\chi_{22} & 1 & -1 & -1 & 1
	\end{array}\]
	\captionof{table}{Character table of Klein group} \label{tab:charklein}
\end{ex}
\section{The Dimension Theorem} \label{sec:03}
In this section, we establish the result that the degree of any irreducible representation of a group divides the order of the group. For this, we require result about algebraic integers from number theory and the reader is encouraged to read \Cref{subsec:numbertheory}.

\begin{prop} \label{prop:charisalgint}
	Let $\chi$ be a character of $G.$ Then, $\chi(g)$ is an algebraic integer for all $g \in G.$
\end{prop}
\begin{proof} 
	Follows immediately from \Cref{cor:chargrouprootsunity}, \Cref{ex:nrootsalgint}, and \Cref{prop:algintsubring}.
\end{proof}

We now set up some notation for the next few results and proofs.

\begin{aside}
	\textbf{Setup.}

	$G$ is a finite group with conjugacy classes $\{1\} = C_1, \ldots, C_s.$ For $i \in \{1, \ldots, s\},$ we define $h_i = \md{C_i}.$

	$\varphi : G \to \GL(V)$ will denote a representation of degree $d$ and $\chi_i$ the value of $\chi_\varphi$ on $C_i.$ (Recall that characters are constant on conjugacy classes, \Cref{prop:charconstonconjclasses}.)

	We define the operators $T_1, \ldots, T_s$ by
	\begin{equation*} 
		T_i = \sum_{x \in C_i} \varphi_x.
	\end{equation*}
\end{aside}

\begin{lem}
	If $\varphi$ is irreducible, then $T_i = \frac{h_i}{d}\chi_i \cdot I.$
\end{lem}
\begin{proof} 
	We first show that $T_i \in \Hom_{G}(\varphi, \varphi).$ Indeed, let $g \in G$ be arbitrary. Then, we have
	\begin{equation*} 
		\varphi_gT_i\varphi_g^{-1} = \sum_{x \in C_i} \varphi_{gxg^{-1}} = \sum_{y \in C_i} \varphi_y = T_i.
	\end{equation*}
	Thus, by \nameref{lem:schur}, we see that $T_i = \lambda_iI$ for some $\lambda_i \in \mathbb{C}.$ We now wish to show that $\lambda_i = h_i\chi_i/d.$ By considering the $\trace$ of both operators, we see that
	\begin{equation*} 
		d\lambda_i = \trace(T_i) = \sum_{x \in C_i} \trace(\varphi_x) = \sum_{x \in C_i} \chi_\varphi(x) = \sum_{x \in C_i} \chi_i = \md{C_i}\chi_i = h_i\chi_i
	\end{equation*}
	and thus, $\lambda_i = h_i\chi_i/d,$ as desired.
\end{proof}

We now show that the $T_i$ satisfy a relation like in \Cref{prop:characalgint}.

\begin{lem} 
	Let $\varphi$ be a (not necessarily irreducible) representation. \\
	Then, $T_i \circ T_j = \displaystyle\sum_{k = 1}^{s}a_{ijk}T_k$ for some $\{a_{ijk}\}_{1 \le i, j, k \le s} \subset \mathbb{Z}.$
\end{lem}
\begin{proof} 
	We note that
	\begin{equation*} 
		T_iT_j = \sum_{x \in C_i}\varphi_x\sum_{y \in C_j} \varphi_y = \sum_{x \in C_i, y \in C_j}\varphi_{xy} = \sum_{g \in G}a_{ijg}\varphi_g,
	\end{equation*}
	where $a_{ijg}$ denotes the cardinality of $X_{ijg} = \{(x, y) \in C_i \times C_j : xy = g\}.$

	Assume for the moment that $a_{ijg}$ depends only on the conjugacy class of $g$ (along with $i$ and $j$). Then, we let $a_{ijk}$ denote the common value of $a_{ijg}$ for $g \in C_k.$ We get
	\begin{equation*} 
		T_iT_j = \sum_{g \in G}a_{ijg}\varphi_g = \sum_{k = 1}^{s}\sum_{g \in C_k} a_{ijg}\varphi_g = \sum_{k = 1}^{s}a_{ijk} \sum_{g \in C_k}\varphi_g = \sum_{k = 1}^{s}a_{ijk}T_k,
	\end{equation*}
	as desired.

	Now, we prove that $a_{ijg}$ depends only the conjugacy class of $g.$ Let $g'$ be in the conjugacy class of $g.$ It suffices to construct a bijection $\psi : X_{ijg} \to X_{ijg'}.$ Write $g' = kgk^{-1}$ and define $\psi$ as
	\begin{equation*} 
		\psi(x, y) = (kxk^{-1}, kyk^{-1}).
	\end{equation*}
	Clearly, $\psi(x, y) \in X_{ijg'}$ since the product of the two elements in the tuple above is indeed $g'$ and both the coordinates are elements of the desired conjugacy class. Moreover, $\psi$ is indeed a bijection as it has inverse $(x' , y') \mapsto (k^{-1}x'k, k^{-1}y'k).$
\end{proof}

\begin{cor}
	With the same notations as earlier, we have
	\begin{equation*} 
		\left(\frac{h_i}{d}\chi_i\right)\left(\frac{h_j}{d}\chi_j\right) = \sum_{k = 1}^{s}a_{ijk}\frac{h_k}{d}\chi_k.
	\end{equation*}
\end{cor}

\begin{thm} \label{thm:hchidisalgint}
	Let $\varphi : G \to \GL(V)$ be an irreducible representation of a finite group $G$ of degree $d.$ Let $g \in G$ and let $h$ be the size of the conjugacy class of $g.$ Then, $h\chi_\varphi(g)/d$ is an algebraic integer.
\end{thm}
\begin{proof}
	In our earlier notation, we wish to show that $h_i\chi_i/d_i$ is an algebraic integer for all $i = 1, \ldots, s.$

	This follows at once from the previous corollary and \Cref{prop:characalgint}. (Note that $\chi_1 \neq 0.$)
\end{proof}

\begin{thm}[Dimension Theorem] \label{thm:dimthm}
	Let $\varphi$ be an irreducible representation $G$ of degree $d.$ Then, $d$ divides $\md{G}.$
\end{thm}
\begin{proof} 
	By \Cref{cor:irrediffnormone}, we know that $\langle \chi_\varphi, \chi_\varphi\rangle = 1.$ Thus, we get
	\begin{equation*} 
		1 = \langle \chi_\varphi, \chi_\varphi\rangle = \frac{1}{\md{G}} \sum_{g \in G} \chi_\varphi(g)\overline{\chi_\varphi(g)}
	\end{equation*}
	and thus,
	\begin{equation*} 
		\frac{\md{G}}{d} =  \sum_{g \in G} \frac{\chi_\varphi(g)}{d}\overline{\chi_\varphi(g)} = \sum_{i = 1}^{s}\sum_{g \in C_i} \frac{\chi_i}{d}\overline{\chi_i} = \sum_{i = 1}^{s}\left(h_i\frac{\chi_i}{d}\right)\overline{\chi_i}.
	\end{equation*}
	Note the expression on the right. Each $\chi_i$ is an algebraic integer, by \Cref{prop:charisalgint} and so is each $h_i\frac{\chi_i}{d},$ by \Cref{thm:hchidisalgint}. Since $\mathbb{A}$ is closed under products, conjugates, and sums, we see that $\frac{\md{G}}{d}$ is an algebraic integer. However, this is clearly rational. Thus, by \Cref{prop:rationalalgintareint}, it follows that $\frac{\md{G}}{d}$ is an integer or equivalently, $d \mid \md{G}.$
\end{proof}

% \begin{cor}
% 	Let $p$ be a prime and let $G$ be a group of order $p^2.$ Then, $G$ is abelian.
% \end{cor}
% \begin{proof} 
% 	Let $d_1, \ldots, d_s$ be the degrees of the irreducible representations of $G.$ Due to the trivial representation, we know that one of the degrees is $1.$ Without loss of generality, $d_1 = 1.$ Thus, we get
% 	\begin{equation*} 
% 		p^2 = \md{G} = 1 + d_2^2 + \cdots + d_s^2.
% 	\end{equation*}
% 	Now, each $d_i \neq 1$ is either $p$ or $p^2.$ Clearly, neither is possible since then the right side would exceed the left. Thus, each $d_i$ is $1$ and thus, $G$ is abelian, by \Cref{cor:numberofirredrepsofG}.
% \end{proof}

\begin{cor}
	Let $p, q$ be primes with $p \le q$ and $q \not\equiv 1 \bmod p.$ Then, any group $G$ of order $pq$ is abelian. In particular, so are groups of order $p^2.$
\end{cor}
\begin{proof} 
	Let $d_1, \ldots, d_s$ be the degrees of the irreducible representations of $G.$ Our aim is to show that $d_i = 1$ for all $i.$ Then, the result will follow, in view of \Cref{cor:numberofirredrepsofG}. 

	Without loss of generality, we may assume $d_1 = 1.$ (Since we always have the trivial representation.) We have
	\begin{equation*} 
		pq = 1 + d_2^2 + \cdots + d_s^2.
	\end{equation*}
	Now, we know that $d_i \in \{1, p, q, pq\}$ for each $i,$ by the \Cref{thm:dimthm}. Clearly, $(pq)^2 > q^2 \ge pq$ and thus, $d_i = q$ or $pq$ is not possible. (If $q = p$, then we are done at this stage.) 

	Now, let $m$ be the number of degree $1$ representations and $n$ of degree $p.$ Thus, we wish to show that $m = \md{G} = pq.$ We have
	\begin{equation*} 
		pq = m + np^2.
	\end{equation*}
	The above shows that $p \mid m.$ Writing $m = pm'$ gives
	\begin{equation} \tag{$*$} \label{eq:011}
		q = m' + np.
	\end{equation}
	By \Cref{cor:numdegoneirrepsdivG}, we know that $m \mid pq$ and hence, $m' \mid q.$ Thus, $m' = 1$ or $q.$ If $m' = 1,$ then \Cref{eq:011} contradicts that $q \not\equiv 1 \mod p.$ Therefore, $m' = q$ and hence, $m = pq.$
\end{proof}

\begin{rem}
	Note that the above corollary is a basic fact from group theory that is usually proven using class equations and Sylow theorems.
\end{rem}

In fact, the proof of the above corollary also gave us the following result.

\begin{cor}
	Let $G$ be a group of order $pq$ with $p < q.$ Then, all irreducible representations of $G$ have degree either $1$ or $p.$ Moreover, $G$ has an irreducible representation of degree $p$ iff $G$ is non-abelian.
\end{cor}

\begin{cor}
	Let $G$ be a group of order $pq$ with $p < q.$ Then, the index of the normal subgroup $[G, G]$ in $G$ is a multiple of $p,$ i.e., it is either $p$ or $pq.$ The former happens iff $G$ is non-abelian.
\end{cor}
\begin{proof} 
	The proof is similar to the previous case. Let $d_1, \ldots, d_s$ be the degrees of the irreducible representations of $G.$ Then, $d_i = 1$ or $p.$ Let $m$ denote the number of degree one representations and $n$ the number of degree $p$ representations. Note that $m$ is precisely the index of $[G, G]$ in $G,$ by \Cref{cor:numdegoneirrepsdivG}.

	We have
	\begin{equation*} 
		pq = m + np^2.
	\end{equation*}
	Thus, $p \mid m,$ as desired. The later parts of the result follow easily.
\end{proof}

\begin{rem}
	Once again, the above can be proved using just group theory as well. 

	Note that $G$ must have a subgroup $H$ of order $q$ (by Sylow theorems or even Cauchy's theorem). \\
	This subgroup has index $p,$ the smallest prime dividing $\md{G}.$ \\
	Thus, $H$ is normal and $G/H$ is a group of order $p$ and hence, abelian. This implies $[G, G] \le H.$ \\
	Since $H$ is of prime order, either $[G, G] = H$ or $[G, G] = \{1\}.$ In either case, the index is a multiple of $p.$ As before, the former happens iff $G$ is non-abelian.
\end{rem}

\section{Permutation Representations} \label{sec:04}

The reader is advised to recall \nameref{subsubsec:groupactions}. We shall continue with the notation established in that section.

\begin{defn}%[Permutation representation]
	\label{defn:permrep}
	Let $\sigma : G \to S_X$ be a group action. Define a representation $\sigma : G \to \GL(\mathbb{C}X)$ by setting
	\begin{equation*} 
		\widetilde{\sigma}_g\left(\sum_{x \in X} c_x x\right) = \sum_{x \in X} c_x \sigma_g(x).
	\end{equation*}
	$\widetilde{\sigma}$ is called the \deff{permutation representation} associated to $\sigma.$
\end{defn}

\begin{rem}
	Note that $\widetilde{\sigma}$ is a representation by \Cref{prop:extendingactiontorep}. Note that $\widetilde{\sigma}_g$ is the linear map defined by extending the map $x \mapsto \sigma_g(x).$ This can be done since $X$ is a basis for $\mathbb{C}X.$

	In more suggestive notation, the above representation can also be written as
	\begin{align*} 
		\widetilde{\sigma}_g\left(\sum_{x \in X} c_x x\right) &= \sum_{x \in X} c_x \sigma_g(x)\\
		&= \sum_{x \in X} c_x(g \cdot x)\\
		&= \sum_{y \in X} c_{g^{-1} \cdot y} y.
	\end{align*}
\end{rem}

\begin{rem}
	Recall \Cref{ex:regularaction} which was the action $\lambda$ of $G$ on $G$ by left multiplication. Then, we have $\widetilde{\lambda} = L,$ the \Cref{defn:regularrepresentation}.	Note that the degree of the action and of the representation coincide.
\end{rem}

\begin{rem} \label{rem:stdrepSnispermrepofstdact}
	Recall the action $\sigma$ of $S_n$ on $\{1, \ldots, n\}$ as in \Cref{ex:actsymgroups}. The corresponding $\widetilde{\sigma}$ is precisely the standard representation of $S_n$ as in \Cref{ex:standardrepSn}.
\end{rem}

\begin{prop} \label{prop:permrepisunitary}
	Let $\sigma : G \to S_X$ be a group action. Then, the representation $\widetilde{\sigma} : G \to \GL(\mathbb{C}X)$ is unitary.
\end{prop}

\begin{proof} 
	Let $g \in G,$ $x, y \in X$ be arbitrary. Note that
	\[\begin{WithArrows}[displaystyle]
		\left\langle \widetilde{\sigma}_g\sum_{x \in X} c_xx, \widetilde{\sigma}_g\sum_{x \in X} k_xx\right\rangle &= \left\langle \sum_{x \in X} c_{g^{-1} \cdot x} x, \sum_{x \in X} k_{g^{-1} \cdot x} x\right\rangle\\
		&= \sum_{x \in X} c_{g^{-1} \cdot x}\overline{k_{g^{-1} \cdot x}} \Arrow{$x \mapsto g \cdot y$}\\
		&= \sum_{y \in X} c_y\overline{k_y}\\
		&= \left\langle \sum_{x \in X} c_x, \sum_{x \in X} k_xx \right\rangle,
	\end{WithArrows}\]
	as desired.
\end{proof}

As before, we now wish to compute the character of such representations. As with the regular representation, we have a simple formula.

\begin{prop} \label{prop:charofpermrep}
	Let $\sigma : G \to S_X$ be a group action. Then,
	\begin{equation*} 
		\chi_{\widetilde{\sigma}}(g) = \md{\Fix(g)}.
	\end{equation*}
\end{prop}

\begin{proof} 
	The proof is again almost identical to that of \Cref{prop:charofregrep}. Note that $X$ acts as a basis for $\mathbb{C}X.$ Fix an ordering $X = \{x_1, \ldots, x_n\}.$ Let $g \in G$ be arbitrary. Note that the matrix $[\widetilde{\sigma}_g]$ with respect to this basis $X$ will consists of columns with exactly with $1$ and rest $0$s.

	More precisely, the $i$-th column will consist of all $0$s and a $1$ at the $j$-th position with $j$ satisfies $x_j = g \cdot x_i.$ In particular, $[\widetilde{\sigma}_g]_{ii} = 1$ iff $g \cdot x_i = x_i$ and $0$ otherwise. The statement now follows at once.
\end{proof}

\begin{cor} \label{cor:normchisigmatilde}
	Retaining the same notation, we have
	\begin{equation*} 
		\langle \chi_{\widetilde{\sigma}}, \chi_{\widetilde{\sigma}}\rangle = \frac{1}{\md{G}}\sum_{g \in G} \md{\Fix(g)}^2 = \frac{\md{X}^2}{\md{G}} + \frac{1}{\md{G}}\sum_{1 \neq g \in G} \md{\Fix(g)}^2.
	\end{equation*}
\end{cor}

\begin{cor}
	Let $\sigma : G \to S_X$ be an action. If $\md{G} \nmid \md{X}^2,$ then there exists $g \in G \setminus \{1\}$ and $x \in X$ such that $g \cdot x = x.$
\end{cor}

\begin{proof} 
	Note that the statement is precisely saying that $\md{\Fix(g)} \neq 0$ for some $1 \neq g \in G.$ Suppose not, that is, suppose that $\md{\Fix(g)} = 0$ for all $g \in G \setminus \{1\}.$ Then, by the earlier corollary, we get that
	\begin{equation*} 
		\langle \chi_{\widetilde{\sigma}}, \chi_{\widetilde{\sigma}}\rangle = \frac{\md{X}^2}{\md{G}} \notin \mathbb{Z}.
	\end{equation*}
	However, this is a contradiction. (\Cref{rem:normcharacisnatural}.)
\end{proof}

% \begin{rem}
% 	As with the case of regular representation, the permutation representation is not irreducible if $\md{X} > 1.$ Note that the above generalises our earlier result (\Cref{prop:charofregrep}). 
% \end{rem}

\begin{defn}%[Fixed subspace]
	Let $\varphi : G \to \GL(V)$ be a representation. Then,
	\begin{equation*} 
		V^G \vcentcolon= \{v \in V \mid \varphi_g(v) = v \text{ for all } g \in G\}
	\end{equation*}
	is a subspace of $V,$ called the \deff{fixed subspace} of $G.$
\end{defn}

The check that $V^G$ is a subspace is simple. We now show that it has some better properties.

\begin{prop}
	$V^G$ is a $G$-invariant subspace.
\end{prop}
\begin{proof} 
	Let $v \in V^G$ and $g \in G.$ Then, $\varphi_gv = v$ by definition of $V^G.$ Thus, $\varphi_gv \in V^G.$
\end{proof}

\begin{rem} \label{rem:identityactiononfixedspace}
	The above proof also shows that the subrepresentation $\varphi|_{V^G}$ is the trivial one. By \Cref{ex:directsumoftrivialreps}, we know that this can be written as a direct sum of $\dim V^G$ many trivial representations.

	The next proposition shows that there are no more trivial representations in $\varphi.$ To be more precise, given the decomposition
	\begin{equation*} 
		\varphi \sim m_1 \varphi^{(1)} \oplus \cdots \oplus m_s \varphi^{(s)},
	\end{equation*}
	the coefficient of the trivial representation is $\dim V^G.$
\end{rem}

\begin{prop} \label{prop:innerprodvarphichifixeddim}
	Let $\varphi : G \to \GL(V)$ be a representation and let $\chi_1$ be the (character of the) trivial representation of $G.$ Then, $\langle \varphi, \chi_1\rangle = \dim V^G.$
\end{prop}

As remarked earlier, shall use $\chi_1$ for both the character as well as the representation.

\begin{proof} 
	Since $V^G$ is a $G$-invariant subspace, there exists a $G$-invariant subspace $W$ such that
	\begin{equation*} 
		V = V^G \oplus W,
	\end{equation*}
	by \Cref{cor:existenceofcomplimentaryGinvarsubs}. (The above is an \emph{internal} direct sum. In particular, $V^G \cap W = 0.$)

	Let $\psi$ and $\rho$ denote the subrepresentations obtained by restricting $\varphi$ to $V^G$ and $W,$ respectively. Then $\varphi \sim \psi \oplus \rho,$ by \Cref{prop:Ginvariantdirectsum}. 

	% Let $\varphi^{(1)}$ denote the trivial representation. 

	\textbf{Claim.} The multiplicity of $\chi_1$ in $\rho$ is $0.$
	\begin{proof} 
		Assume not. Let $W' \le W$ be a subspace such that $\rho|_{W'} \sim \chi_1.$ \\
		In particular, $W'$ has dimension $1.$ \\
		Choose a nonzero $w \in W' \le W.$ Then, $\rho_g(w) = w$ for all $g \in G.$\footnote{We are using the fact that if a representation is equivalent to the trivial representation, then it acts as identity.} Thus, $w \in V^G,$ a contradiction since $W \cap V^G = 0.$
	\end{proof}

	Note that we know
	\begin{equation*} 
		\psi \sim m_1 \chi_1
	\end{equation*}
	where $m_1 = \dim V^G.$ (\Cref{rem:identityactiononfixedspace}.) The above claim shows that
	\begin{equation*} 
		\langle \rho, \chi_1\rangle = 0.
	\end{equation*}
	Thus, we get
	\begin{equation*} 
		\langle \varphi, \chi_1\rangle = \langle \psi, \chi_1\rangle + \langle \rho, \chi_1\rangle = m_1 + 0 = \dim V^G. \qedhere
	\end{equation*}
\end{proof}
	% Let
	% \begin{equation*} 
	% 	\varphi \sim m_1 \varphi^{(1)} \oplus \cdots \oplus m_s \varphi^{(s)}
	% \end{equation*}
	% be the decomposition of $\varphi$ into irreducible representation. Since characters only depend on equivalence classes, we may assume the above to be equality. We may assume that $\varphi^{(1)}$ is the trivial representation. 

	% Since $\langle \varphi, \chi_1\rangle = m_1,$ our job now is to show that $m_1 = \dim V^G.$

	% Let $V_1, \ldots, V_s$ denote the vector spaces corresponding to the different representations, that is, $\varphi^{(i)} : G \to \GL(V_i)$ and $V = \bigoplus_{i = 1}^s m_iV_i.$

	% Now, if $v \in V,$ then
	% \begin{equation*} 
	% 	v = (v_1, \ldots, v_s)
	% \end{equation*}
	% for $v_i \in m_iV_i.$ Applying $\varphi_g$ for an arbitrary $g \in G$ gives
	% \begin{align*} 
	% 	\varphi_g(v) &= \left(\left(m_1\varphi^{(1)}\right)_g(v_1), \ldots, \left(m_s\varphi^{(s)}\right)_g(v_s)\right)\\
	% 	&= \left(v, \left(m_2\varphi^{(2)}\right)_g(v_2), \ldots, \left(m_s\varphi^{(s)}\right)_g(v_s)\right).
	% \end{align*}
	% Thus, $v \in V^G$ iff $\left(m_i\varphi^{(i)}\right)_g(v_i) = v_i$ for all $2 \le i \le s.$ In other words,
	% \begin{equation*} 
	% 	V^G = m_1V_1 \oplus m_2V_2^G \oplus \cdots \oplus m_sV_s^G.
	% \end{equation*}
	% For $i \ge 2,$ we make the following observation: we know that $V_i^G \le V_i$ is $G$-invariant. Irreducibility of $\varphi^{(i)}$ tells us that $V_i^G = 0$ or $V.$ However, we also know that $\varphi^{(i)}$ acts as identity on $V_i^G$ (\Cref{rem:identityactiononfixedspace}) and thus, $V_i^G = V_i$ would imply that $\varphi^{(i)}$ can be decomposed as a sum of $\varphi^{(1)}$s. Thus, we must have $V_i^G = 0$ for all $i \ge 2.$

	% Thus, $V^G = m_1V_1.$ Since $V_1$ is one dimensional, we see that $\dim V^G = m_1,$ as desired.

% The above is a proposition which could have been proven in the previous section as well. In particular, we did not use any theory of group actions in the above. However, we now connect the above in the next proposition, by computing $\mathbb{C}X^G$ for a permutation representation.

\begin{prop} \label{prop:fixedspacetransaction}
	Let $\sigma : G \to S_X$ be a transitive group action. Define
	\begin{equation*} 
		v_0 \vcentcolon= \sum_{x \in X} x \in \mathbb{C}X.
	\end{equation*}
	Then, $\mathbb{C}X^G = \mathbb{C}v_0.$ In, particular, $\mathbb{C}X^G$ is one-dimensional.
\end{prop}

Note that this is a special case of the immediate next proposition.

\begin{proof} 
	It is clear that $\mathbb{C}v_0 \le \mathbb{C}X^G$ since every $\sigma_g$ is simply a permutation of $X.$ Thus, it suffices to show that $v_0$ spans $\mathbb{C}X^G.$ The idea is simple. Consider $v \in \mathbb{C}X^G.$ Then, it can represented in the standard basis as
	\begin{equation*} 
		v = \sum_{x \in X} c_x x.
	\end{equation*}
	We assert that $c_x$ is independent of $x.$ In other words, we show that $c_y = c_z$ for all $y, z \in X.$

	Indeed, given $y, z \in X,$ choose $g \in G$ such that $g \cdot y = z.$ (We can do so since the action is transitive.) \\
	Now, note that
	\begin{align*} 
		v &= \widetilde{\sigma}_g(v)\\
		\iff \sum_{x \in X} c_x x &= \sum_{x \in X} c_x g \cdot x.
	\end{align*}
	The coefficient of $z$ is $c_z$ on the left and $c_y$ on the right and thus, $c_y = c_z.$

	Thus, each $c_x = c$ for some $c \in \mathbb{C}$ and we get
	\begin{equation*} 
		v = \sum_{x \in X} c_x x = \sum_{x \in X} c x = c\sum_{x \in X} x = cv_0,
	\end{equation*}
	as desired.
\end{proof}

\begin{prop} \label{prop:numberoforbitsisfixeddim}
	Let $\sigma : G \to S_X$ be a group action. Let $\mathcal{O}_1, \ldots, \mathcal{O}_m$ be orbits of $G$ on $X.$ Define
	\begin{equation*} 
		v_i \vcentcolon= \sum_{x \in \mathcal{O}_i} x \in \mathbb{C}X
	\end{equation*} 
	for $i = 1, \ldots, m.$ Then, $B = \{v_1, \ldots, v_m\}$ is a basis for $\mathbb{C}X^G.$ In particular, $\dim \mathbb{C}X^G = m,$ the number of orbits.
\end{prop}

\begin{proof} 
	First, we show that $B$ is indeed a subset of $\mathbb{C}X^G.$ This is simple for if $1 \le i \le m$ and $g \in G$ are arbitrary, then
	\[\begin{WithArrows}[displaystyle]
		\widetilde{\sigma}_gv_i &= \widetilde{\sigma}_g\left(\sum_{x \in \mathcal{O}_i} x\right)\\
		& = \sum_{x \in \mathcal{O}_i} \sigma_g(x) \Arrow{$\sigma_g|_{\mathcal{O}_i}$ is a bijection}\\
		& =  \sum_{x \in \mathcal{O}_i} x\\
		&= v_i
	\end{WithArrows}\]

	Second, we show that $B$ is linearly independent. We do the usual by computing the inner product of elements of $B.$ However, recall that the inner product on $\mathbb{C}X$ is essentially the ``usual'' dot product, just indexed by $X.$ Since distinct orbits are disjoint, we get the following
	\begin{equation*} 
		\langle v_i, v_j\rangle = \begin{cases}
			\md{\mathcal{O}_i} & i = j,\\
			0 & i \neq j.
		\end{cases}
	\end{equation*}
	That is, $B$ consists of non-zero orthogonal vectors and thus, is linearly independent.

	Third, we show that $B$ is spanning. Let $v \in \mathbb{C}X^G$ be an arbitrary vector given by
	\begin{equation*} 
		v = \sum_{x \in X} c_x x
	\end{equation*}
	for some scalars $c_x \in \mathbb{C}.$ Note that $G$ acts transitively on each orbit. Thus, by a similar argument as in the previous proof, we get that $c_z = c_y$ for all $y, z \in X$ if $z \in G \cdot y.$ 

	% We now wish to show that the coefficients of two elements of $X$ in the above is equal if the two are in the same orbit. More precisely, if $y, z \in X$ with $z \in G \cdot y,$ then $c_z = c_y.$ To see this, choose $g \in G$ such that $z = g \cdot y.$ Then, since $v \in \mathbb{C}X^G,$ we note that
	% \begin{align*} 
	% 	v &= \widetilde{\sigma}_g(v)\\
	% 	\iff \sum_{x \in X} c_x x &= \sum_{x \in X} c_x g \cdot x.
	% \end{align*}
	% The coefficient of $z$ is $c_z$ on the left and $c_y$ on the right and thus, $c_y = c_z.$

	Thus, for each $i = 1, \ldots, m,$ there exists $c_i \in \mathbb{C}$ such that $c_x = c_i$ for all $x \in \mathcal{O}_i.$ Hence, we may write $v$ as
	\begin{align*} 
		v &= \sum_{x \in X} c_x x\\
		&= \sum_{i = 1}^{m} \sum_{x \in \mathcal{O}_i} c_i x\\
		&= \sum_{i = 1}^{m} c_i \sum_{x \in \mathcal{O}_i} x\\
		&= \sum_{i = 1}^{m} c_i v_i \in \spn B. \qedhere
	\end{align*}
\end{proof}

\begin{cor} \label{cor:nontrivialpermred}
	Suppose $\sigma : G \to S_X$ is a group action and $\md{X} > 1.$ Then, $\widetilde{\sigma}$ is reducible.
\end{cor}

\begin{proof} 
	Note that the degree of $\widetilde{\sigma}$ is $\md{X} > 1.$ However, since $X$ has at least one orbit, the previous proposition shows that the fixed subspace of $G$ has dimension at least one. Thus, the trivial representation appears as a proper constituent in the decomposition of $\widetilde{\sigma}.$
\end{proof}

\begin{cor}[Burnside's lemma] \label{cor:burnsideslemma}
	Let $\sigma : G \to S_X$ be a group action and let $m$ be the number of orbits of $G$ on $X.$ Then,
	\begin{equation*} 
		m = \frac{1}{\md{G}}\sum_{g \in G} \md{\Fix(g)}.
	\end{equation*}
\end{cor}

That is, the number of orbits equals the average number of fixed points.

\begin{proof} 
	Let $\chi_1$ denote the trivial character of $G.$ Then, we note
	\[\begin{WithArrows}[displaystyle]
		& m \Arrow{\Cref{prop:numberoforbitsisfixeddim}} \\
		&= \dim \mathbb{C}X^G \Arrow{\Cref{prop:innerprodvarphichifixeddim}} \\
		&= \langle \chi_{\widetilde{\sigma}}, \chi_1\rangle \\
		&= \frac{1}{\md{G}} \sum_{g \in G} \chi_{\widetilde{\sigma}}(g) \overline{\chi_1(g)} \Arrow{$\chi_1 \equiv 1$}\\
		&= \frac{1}{\md{G}} \sum_{g \in G} \chi_{\widetilde{\sigma}}(g) \Arrow{\Cref{prop:charofpermrep}}\\
		&= \frac{1}{\md{G}}\sum_{g \in G} \md{\Fix(g)},
	\end{WithArrows}\]
	finishing the proof.
\end{proof}

\begin{cor} \label{cor:transactionranknorm}
	Let $\sigma : G \to S_X$ be a group action. Then, the equalities
	\begin{equation*} 
		\rank(\sigma) = \frac{1}{\md{G}}\sum_{g \in G} \md{\Fix(g)}^2 = \langle \chi_{\widetilde{\sigma}}, \chi_{\widetilde{\sigma}}\rangle
	\end{equation*}
	hold.
\end{cor}

\begin{proof} 
	The left equality follows by recalling that the definition of $\rank$ is the number of orbits of $\sigma^2.$ Thus, applying \nameref{cor:burnsideslemma} to $\sigma^2$ yields the equality since $\md{\Fix^2(g)} = \md{\Fix(g)}^2,$ by \Cref{prop:fix2isfix2}.

	The right equality is simply \Cref{cor:normchisigmatilde}.
\end{proof}

\begin{defn}
	Let $\sigma : G \to S_X$ be a \underline{transitive} action. Let $v_0 \vcentcolon= \sum_{x \in X} x \in \mathbb{C}X.$

	$\mathbb{C}v_0 = \mathbb{C}X^G$ is a $G$-invariant subspace. $V_0 \vcentcolon= \mathbb{C}v_0^\perp$ is $G$-invariant. Let $\widetilde{\sigma}'$ denote the restriction of $\widetilde{\sigma}$ to $V_0.$

	$\mathbb{C}v_0$ is called the \deff{trace} of $\sigma,$ $V_0$ the \deff{augmentation} of $\sigma,$ and $\widetilde{\sigma}'$ the \deff{augmentation representation} associated to $\sigma.$
\end{defn}

\begin{rem}
	Let us justify the various statements in the definition.
	
	$\mathbb{C}v_0 = \mathbb{C}X^G$ followed from \Cref{prop:numberoforbitsisfixeddim}.

	Since $\widetilde{\sigma}$ is a unitary representation, $V_0 \vcentcolon= \mathbb{C}v_0^\perp$ is $G$-invariant, by the proof of \Cref{prop:unitirredordecom}.
\end{rem}

\begin{thm} \label{thm:augmenirrediff2trans}
	Let $\sigma : G \to S_X$ be a transitive group action. Then, the augmentation representation $\widetilde{\sigma}'$ is irreducible if and only if $G$ is 2-transitive.
\end{thm}

\begin{proof} 
	Given that $\sigma$ is transitive, we see that $\sigma$ is 2-transitive if and only if $\rank(\sigma) = 2,$ by \Cref{prop:2transiffrank2}. Also note that \Cref{lem:charactersadd} gives us
	\begin{equation*} 
		\chi_{\widetilde{\sigma}'} = \chi_{\widetilde{\sigma}} - \chi_1.
	\end{equation*}
	Thus, we get
	\[\begin{WithArrows}[displaystyle]
		\langle \chi_{\widetilde{\sigma}'}, \chi_{\widetilde{\sigma}'}\rangle &= \langle \chi_{\widetilde{\sigma}} - \chi_1, \chi_{\widetilde{\sigma}} - \chi_1\rangle\\
		&= \langle \chi_{\widetilde{\sigma}}, \chi_{\widetilde{\sigma}}\rangle - \langle \chi_{\widetilde{\sigma}}, \chi_1\rangle - \langle \chi_1, \chi_{\widetilde{\sigma}}\rangle + \langle \chi_1, \chi_1\rangle\\
		&= \langle \chi_{\widetilde{\sigma}}, \chi_{\widetilde{\sigma}}\rangle - \langle \chi_{\widetilde{\sigma}}, \chi_1\rangle - \langle \chi_1, \chi_{\widetilde{\sigma}}\rangle + 1 \Arrow{\Cref{prop:innerprodvarphichifixeddim}}\\
		&= \langle \chi_{\widetilde{\sigma}}, \chi_{\widetilde{\sigma}}\rangle - \dim \mathbb{C}X^G - \overline{\dim \mathbb{C}X^G} + 1 \Arrow{$\dim \mathbb{C}X^G = 1,$ \Cref{prop:numberoforbitsisfixeddim}}\\
		&= \langle \chi_{\widetilde{\sigma}}, \chi_{\widetilde{\sigma}}\rangle - 1 \Arrow{\Cref{cor:transactionranknorm}} \\
		&= \rank(\sigma) - 1.
	\end{WithArrows}\]
	By \Cref{cor:irrediffnormone}, $\widetilde{\sigma}'$ is irreducible iff $\langle \chi_{\widetilde{\sigma}'}, \chi_{\widetilde{\sigma}'}\rangle = 1$ iff $\rank(\sigma) - 1 = 1$ iff $\rank(\sigma) = 2$ iff $\sigma$ is 2-transitive.
\end{proof}

\begin{ex}[Character table of $S_4$]
	Note that we have five conjugacy classes in $S_4.$ (Recall \nameref{thm:descconjclassSn}.) One set of representatives is
	\begin{equation*} 
		1, (12), (12)(34), (123), (1234).
	\end{equation*}

	We already know it has exactly two degree one representations. (\Cref{ex:degonerepsSn}.) Let $\chi_1$ denote the character of the trivial representation and $\chi_2$ of the sign representation.

	Let $\rho$ denote the standard representation of $S_4.$ (\Cref{ex:standardrepSn}.) Recall that this is the permutation permutation corresponding to the natural action of $S_4$ on $\{1, \ldots, 4\}.$ (\Cref{rem:stdrepSnispermrepofstdact}.) Also, recall that this action is 2-transitive. (\Cref{ex:actsymgroups}.) Thus, by \Cref{thm:augmenirrediff2trans}, the augmentation representation is a degree three irreducible representation of $S_4.$ Let us denote its character by $\chi_4.$ We know that $\chi_4 = \chi_\rho - \chi_1.$

	Thus, there are two more left. By \Cref{prop:descripofL}, we see the sums of squares of their degrees is $13.$ Thus, the degrees are two and three. (This is the reason we used $\chi_4$ and not $\chi_3$ for the earlier representation.)

	Let $\chi_3$ and $\chi_5$ denote the characters of the unknown degree two and three representations, respectively. Thus, so far, we have the following table.

	\[\begin{array}{rrrrrr}
		 & [1] & [(12)] & [(12)(34)] & [(123)] & [(1234)]\\
		\thiccline
		\chi_1 & 1 &  1 & 1  &  1 &  1 \\
		\chi_2 & 1 & -1 & 1  &  1 & -1 \\
		\chi_3 & 2 &    &    &    &    \\
		\chi_4 & 3 &  1 & -1 &  0 & -1 \\
		\chi_5 & 3 &    &    &    &    \\
	\end{array}\]

	(Note that $\chi_4 = \chi_\rho - \chi_1$ is easy to calculate because $\chi_\rho(\tau)$ is the number of elements fixed by $\tau.$ Thus, we just look at the number of elements fixed by $\tau$ and subtract $1$ to get $\chi_4.$)

	From the above table, we note that
	\begin{equation*} 
		\chi_2\left((12)\right) \neq 1 \andd \chi_4\left((12)\right) \neq 0.
	\end{equation*}
	Thus, by \Cref{cor:multiplyingdegonerep}, we see that multiplying the representations corresponding to $\chi_2$ and $\chi_4$ gives us a new inequivalent irreducible degree three representation. Thus, the character $\chi_5$ is obtained by multiplying the corresponding characters to get the following.

	\[\begin{array}{rrrrrr}
		 & [1] & [(12)] & [(12)(34)] & [(123)] & [(1234)]\\
		\thiccline
		\chi_1 & 1 &  1 &  1 &  1 &  1 \\
		\chi_2 & 1 & -1 &  1 &  1 & -1 \\
		\chi_3 & 2 &    &    &    &    \\
		\chi_4 & 3 &  1 & -1 &  0 & -1 \\
		\chi_5 & 3 & -1 & -1 &  0 &  1 \\
	\end{array}\]

	The remaining entries of $\chi_2$ are now easy to fill since the columns are orthogonal, by \Cref{thm:secondorthorel}. Since we do know the first column completely, the other columns can be filled.

	Computing the inner product for $g \neq 1$ with the first column, we get
	\begin{equation*} 
		1 \chi_1(g) + 1 \chi_2(g) + 2 \chi_3(g) + 3 \chi_4(g) + 3 \chi_5(g) = 0
	\end{equation*}
	or
	\begin{equation*} 
		\chi_3(g) = -\frac{1}{2}(\chi_1(g) + \chi_2(g) + 3\chi_4(g) + 3\chi_5(g)).
	\end{equation*}
	(We have dropped the conjugate since everything is real.)

	Thus, we fill the last row to obtain the table as follows.

	\captionsetup{type=figure}
	\[\begin{array}{rrrrrr}
		 & [1] & [(12)] & [(12)(34)] & [(123)] & [(1234)]\\
		\thiccline
		\chi_1 & 1 &  1 &  1 &  1 &  1 \\
		\chi_2 & 1 & -1 &  1 &  1 & -1 \\
		\chi_3 & 2 &  0 &  2 & -1 &  0 \\
		\chi_4 & 3 &  1 & -1 &  0 & -1 \\
		\chi_5 & 3 & -1 & -1 &  0 &  1 \\
	\end{array}\]
	\captionof{table}{Character table of $S_4$} \label{tab:charS4}
\end{ex}
\section{Induced Representations} \label{sec:05}
\subsection{Induced Characters and Frobenius Reciprocity}

Note that if we are given a representation $\rho : H \to \GL(V)$ of a group $H$ and a group homomorphism $\psi : G \to H,$ then we get a representation of $G,$ namely $\varphi = \rho \circ \psi.$ Moreover, if $\rho$ is irreducible and $\psi$ onto, then we know that $\varphi$ is also irreducible. (\Cref{thm:ontogrouphomogivesirredrep}.) 

In particular, the above shows how one can get (irreducible) representations of $G$ if we are given (irreducible) representations of a quotient of $G.$ We would now like to examine the case when $H$ is a subgroup of $G.$ Given a representation of $H,$ can we get an induced representation for $G?$

\begin{defn}%[Restriction]
	Let $H \le G.$ If $f : G \to \mathbb{C}$ is a function, then we can restrict $f$ to $H$ and get a function $f|_H : H \to \mathbb{C}.$ We denote this \deff{restriction} by $\Res^{G}_{H}f.$
\end{defn}

Recall that we had defined the group algebra $L(G)$ of a group $G.$ (\Cref{defn:groupalg}.) Thus, the above is a function
\begin{equation*} 
	\Res^{G}_{H} : L(G) \to L(H).
\end{equation*}
Recall that we had also defined the subspace $Z(L(G))$ of class functions. We now show that the restriction of a class function is again a class function.

\begin{prop}
	Let $H \le G.$ Then, $\Res^{G}_{H} : Z(L(G)) \to Z(L(H))$ is a linear map.
\end{prop}
\begin{proof} 
	First, we show that if $f \in Z(L(G)),$ then $\Res^{G}_{H}f \in Z(L(H)).$ This is simple for if $x, h \in H,$ then $x, h \in G$ as well and we have
	\begin{equation*} 
		\Res^{G}_{H}f(x^{-1}hx) = f(x^{-1}hx) = f(h) = \Res^{G}_{H}f(h),
	\end{equation*}
	where the middle equality follows since $f$ was a class function.

	Second, we show that $\Res^{G}_{H}$ is linear. This too is simple; let $c \in \mathbb{C},$ $f_1, f_2 \in Z(L(G))$ be arbitrary. Then,
	\begin{align*} 
		\Res^{G}_{H}(cf_1 + f_2)(h) &= (cf_1 + f_2)(h)\\
		&= cf_1(h) + f_2(h)\\
		&= c\Res^{G}_{H}f_1(h) + \Res^{G}_{H}f_2(h)\\
		&= (c\Res^{G}_{H}f_1 + \Res^{G}_{H}f_2)(h)
	\end{align*}
	for all $h \in H$ and hence,
	\begin{equation*} 
		\Res^{G}_{H}(cf_1 + f_2) = c\Res^{G}_{H}f_1 + \Res^{G}_{H}f_2. \qedhere
	\end{equation*}
\end{proof}

Thus, the restriction of a class function (on $G$) is again a class function (on $H$). The same is true for characters as well, as we shall see later. (In fact, the obvious candidate works.)

We now wish to construct a map $Z(L(H)) \to Z(L(G)).$

\begin{defn}
	If $H \le G$ and $f : H \to \mathbb{C},$ we define $\hat{f} : G \to \mathbb{C}$ by
	\begin{equation*} 
		\hat{f}(x) \vcentcolon= \begin{cases}
			f(x) & x \in H,\\
			0 & x \notin H.
		\end{cases}
	\end{equation*}
\end{defn}
\begin{rem}
	Note that the notation does not explicitly mention $H$ or $G.$ However, from context it would be clear what $G$ is. ($H$ is recovered as the domain of $f.$)
\end{rem}

\begin{lem} \label{lem:overdotislinear}
	$f \mapsto \hat{f}$ is a linear map from $L(H)$ to $L(H).$
\end{lem}
\begin{proof} 
	Let $c \in \mathbb{C}$ and $f, g \in L(H)$ be arbitrary. Then,
	\begin{align*} 
		\widehat{(cf + g)}(x) &= \begin{cases}
			(cf + g)(x) & x \in H,\\
			0 & x \notin G
		\end{cases}\\
		&= \begin{cases}
			cf(x) + g(x) & x \in H,\\
			c0 + 0 & x \notin G
		\end{cases}\\
		&= c\hat{f}(x) + \hat{g}(x)
	\end{align*}
	for all $x \in G.$
\end{proof}

\begin{defn}%[Induction]
	Let $H \le G.$ We define the \deff{induction} map $\Ind^{G}_{H} : Z(L(H)) \to Z(L(G))$ by the formula
	\begin{equation*} 
		\Ind^{G}_{H}f(g) = \frac{1}{\md{H}}\sum_{x \in G} \hat{f}(x^{-1}gx).
	\end{equation*}
	If $\chi$ is a character of $H,$ $\Ind^{G}_{H}\chi$ is called the \deff{induced character} of $\chi$ on $G.$
\end{defn}

We shall show later that the \emph{induced character} is indeed a \emph{character}.	As earlier, we have check that $\Ind^{G}_{H}f$ is indeed a class function if $f$ is.

\begin{prop}
	Let $H \le G.$ Then, $\Ind^{G}_{H} : Z(L(H)) \to Z(L(G))$ is a linear map.
\end{prop}
\begin{proof} 
	First, we show that if $f \in Z(L(H)),$ then $\Ind^{G}_{H}f \in Z(L(G)).$ This is simple for if $y, g \in G,$ then we have
	\[\begin{WithArrows}[displaystyle]
		\Ind^{G}_{H}f(y^{-1}gy) &= \frac{1}{\md{H}}\sum_{x \in G} \hat{f}(x^{-1}y^{-1}gyx)\\
		&= \frac{1}{\md{H}}\sum_{x \in G} \hat{f}\left((yx)^{-1} g (yx)\right) \Arrow{$x \mapsto y^{-1}z$}\\
		&= \frac{1}{\md{H}}\sum_{z \in G} \hat{f}(z^{-1}gz)\\
		&= \Ind^{G}_{H}f(g).
	\end{WithArrows}\]
	
	Second, we show that the map is linear. We use the fact that $f \mapsto \hat{f}$ is linear. (\Cref{lem:overdotislinear}.) Let $c \in \mathbb{C}$ and $f_1, f_2 \in Z(L(G))$ be arbitrary. Then, we have
	\begin{align*} 
		\Ind^{G}_{H}(cf_1 + f_2)(g) &= \frac{1}{\md{H}}\sum_{x \in G}\widehat{(cf_1 + f_2)}(x^{-1}gx)\\
		&= \frac{1}{\md{H}}\sum_{x \in G}\left[c\hat{f_1}(x^{-1}gx) + \hat{f_2}(x^{-1}gx)\right]\\
		&= c\frac{1}{\md{H}}\sum_{x \in G}\hat{f_1}(x^{-1}gx) + \frac{1}{\md{H}}\sum_{x \in G}\hat{f_2}(x^{-1}gx)\\
		&= c\Ind^{G}_{H}f_1(g) + \Ind^{G}_{H}f_2(g),
	\end{align*}
	for all $g \in G.$ Thus,
	\begin{equation*} 
		\Ind^{G}_{H}(cf_1 + f_2) = c\Ind^{G}_{H}f_1 + \Ind^{G}_{H}f_2. \qedhere
	\end{equation*}
	
\end{proof}

\begin{thm}[Frobenius reciprocity] \label{thm:frobeniusreciprocity}
	Suppose that $H$ is a subgroup of $G.$ Let $a \in Z(L(H))$ and $b \in Z(L(G)).$ Then, the equality
	\begin{equation*} 
		\langle \Ind^{G}_{H}a, b\rangle = \langle a, \Res^{G}_{H}b\rangle
	\end{equation*}
	holds.
\end{thm}
Note that $\Ind^{G}_{H}a, b \in Z(L(G)) \le L(G)$ and hence, the left side is the inner product in the space $L(G).$ On the other hand, the product on the right is in the space $L(H).$ The above is saying $\Ind^{G}_{H}$ and $\Res^{G}_{H}$ act as \emph{adjoints} of each other.

The above can be interpreted as follows: Suppose $\chi$ is an irreducible character of $G$ and $\theta$ of $H.$ Then, the multiplicity of $\chi$ in the induced character $\Ind^{G}_{H}\theta$ is exactly the same as the multiplicity of $\theta$ in $\Res^{G}_{H}\chi.$

\begin{proof} 
	The proof is a simple computation.
	\[\begin{WithArrows}[displaystyle]
		\langle \Ind^{G}_{H}a, b\rangle &= \frac{1}{\md{G}}\sum_{g \in G} \Ind^{G}_{H}a(g) \overline{b(g)}\\
		&= \frac{1}{\md{G}}\sum_{g \in G} \frac{1}{\md{H}}\sum_{x \in G} \hat{a}(x^{-1}gx)\overline{b(g)}\\
		&= \frac{1}{\md{G}\md{H}} \sum_{x \in G} \sum_{g \in G} \hat{a}(x^{-1}gx)\overline{b(g)} \Arrow{$g \mapsto x^{-1}gx$ is a bijection}\\
		&= \frac{1}{\md{G}\md{H}} \sum_{x \in G} \sum_{g \in G} \hat{a}(g)\overline{b(xgx^{-1})} \Arrow{$b \in Z(L(G))$}\\
		&= \frac{1}{\md{G}\md{H}} \sum_{x \in G} \sum_{g \in G} \hat{a}(g)\overline{b(g)} \Arrow{$\hat{a}$ vanishes outisde $H$}\\
		&= \frac{1}{\md{G}\md{H}} \sum_{x \in G} \sum_{h \in H} a(h)\overline{b(h)}\\
		&= \frac{1}{\md{G}\md{H}} \sum_{x \in G} \sum_{h \in H} a(h)\overline{\Res^{G}_{H}b(h)}\\
		&= \frac{1}{\md{G}} \sum_{x \in G} \langle a, \Res^{G}_{H}b\rangle\\
		&= \langle a, \Res^{G}_{H}b\rangle,
	\end{WithArrows}\]
	as desired.
\end{proof}

We now give an alternate way of computed the induction, in terms of coset representatives.

\begin{prop} \label{prop:shortindformula}
	Let $G$ be a group and $H$ a subgroup of $G.$ Let $T = \{t_1, \ldots, t_m\}$ be a transversal of the cosets of $H$ in $G.$ Then, the formula
	\begin{equation*} 
		\Ind^{G}_{H}f(g) = \sum_{i = 1}^{m} \hat{f}(t_i^{-1}gt_i)
	\end{equation*}
	holds for any $f \in Z(L(H))$ and $g \in G.$

	Equivalently,
	\begin{equation*} 
		\Ind^{G}_{H}f(g) = \sum_{\substack{t \in T \\ t^{-1}gt \in H}} f(t^{-1}gt).
	\end{equation*}
\end{prop}
\begin{proof} 
	Fix $g \in G$ and $f \in Z(L(H)).$

	Note that we have the disjoint union 
	\begin{equation*} 
		G = t_1H \sqcup \cdots \sqcup t_mH
	\end{equation*}
	and hence,
	\begin{align*} 
		\Ind^{G}_{H}f(g) &= \frac{1}{\md{H}} \sum_{x \in G} \hat{f}(x^{-1}gx)\\
		&= \frac{1}{\md{H}} \sum_{i = 1}^{m} \sum_{x \in t_iH} \hat{f}(x^{-1}gx)\\
		&= \frac{1}{\md{H}} \sum_{i = 1}^{m} \sum_{h \in H} \hat{f}(h^{-1}t_i^{-1}gt_ih).
	\end{align*}
	Thus, we have
	\begin{equation} \tag{$*$} \label{eq:005}
		\Ind^{G}_{H}f(g) = \frac{1}{\md{H}} \sum_{i = 1}^{m} \sum_{h \in H} \hat{f}(h^{-1}t_i^{-1}gt_ih).
	\end{equation}

	Now, note that if $t_i^{-1}gt_i \notin H,$ then $h^{-1}t_i^{-1}gt_ih \notin H$ and hence,
	\begin{equation*} 
		\hat{f}(h^{-1}t_i^{-1}gt_ih) = 0 = \hat{f}(t_i^{-1}gt_i).
	\end{equation*}
	On the other hand, if $t_i^{-1}gt_i \in H,$ then $h^{-1}t_i^{-1}gt_ih \in H$ and hence,
	\begin{equation*} 
		\hat{f}(h^{-1}t_i^{-1}gt_ih) = f(h^{-1}t_i^{-1}gt_ih) = f(t_i^{-1}gt_i) = \hat{f}(t_i^{-1}gt_i),
	\end{equation*}
	where the middle inequality follows since $f$ is class function \underline{on $H$}.

	Thus, we have shown that
	\begin{equation*} 
		\hat{f}(h^{-1}t_i^{-1}gt_ih) = \hat{f}(t_i^{-1}gt_i)
	\end{equation*}
	for all $h \in H$ and $i = 1, \ldots, m.$

	Substituting this in \Cref{eq:005} gives us that
	\begin{align*} 
		\Ind^{G}_{H}f(g) &= \frac{1}{\md{H}} \sum_{i = 1}^{m} \sum_{h \in H} \hat{f}(h^{-1}t_i^{-1}gt_ih)\\
		&=  \frac{1}{\md{H}} \sum_{i = 1}^{m} \sum_{h \in H} \hat{f}(t_i^{-1}gt_i)\\
		&= \sum_{i = 1}^{m} \hat{f}(t_i^{-1}gt_i). \qedhere
	\end{align*}
\end{proof}

\subsection{Induced Representations}

In the previous subsection, we saw that induction and restriction of class functions give class functions. We now wish to show that the same is true for characters as well. The restriction part is easy.

\begin{prop}
	If $\varphi : G \to \GL(V)$ is a representation and $H \le G,$ then we can restrict $\varphi$ to $H$ to obtain a representation
	\begin{equation*} 
		\Res^{G}_{H}\varphi : H \to \GL(V).
	\end{equation*}
	Then
	\begin{equation*} 
		\chi_{\Res^{G}_{H}\varphi} = \Res^{G}_{H}\chi_\varphi.
	\end{equation*}
\end{prop}
Note we have not formally defined $\Res^{G}_{H}\varphi$ since we had defined $\Res^{G}_{H}$ only on $L(G),$ but it has its natural meaning.

\begin{proof} 
	We need to show equality of two functions of $H.$ To this end, let $h \in H$ be arbitrary. Then,
	\begin{equation*} 
		\chi_{\Res^{G}_{H}\varphi}(h) = \trace \Res^{G}_{H}\varphi(h) = \trace \varphi(h) = \chi_\varphi(h) = \Res^{G}_{H}\chi_\varphi(h). \qedhere
	\end{equation*}
\end{proof}

Thus, restriction of a character is again a character. We would like to show the same for induction but that direction is not as easy. \\
Let us first look at some examples where we show that the induction of a character is actually the character of some known representation. (Note that unlike the case of restriction, we are not actually constructing a representation \emph{yet}. We are simply observing some specific examples and comparing it with \emph{known} characters.)

\begin{ex}[Regular representation]\label{ex:regrepinduct}
	Let $G$ be a group and consider $H = \{1\} \le G.$ Let $\chi_1$ denote the trivial character of $H.$ Computing the induced character, we write
	\begin{equation} \tag{$*$} \label{eq:006}
		\Ind^{G}_{H}\chi_1(g) = \sum_{x \in G} \hat{\chi}_1(x^{-1}gx).
	\end{equation}
	However, note that $x^{-1}gx \in \{1\} \iff g = 1.$ Thus, we see that the right side of \Cref{eq:006} is $0$ if $g \neq 1.$ On the other hand, if $g = 1,$ then we have
	\begin{equation*} 
		\Ind^{G}_{H}\chi_1(g) = \sum_{x \in G} \hat{\chi}_1(x^{-1}1x) = \sum_{x \in G} \hat{\chi}_1(1) = \md{G}.
	\end{equation*}
	Thus, we get that 
	\begin{equation*} 
		\Ind^{G}_{\{1\}}\chi_1(g) = \begin{cases}
			\md{G} & g = 1,\\
			0 & g \neq 1.
		\end{cases}
	\end{equation*}
	The above is precisely the character of the regular representation of $G.$ (\Cref{prop:charofregrep}.)
\end{ex}

\begin{ex}[Permutation representation]\label{ex:permrepinduct}
	We can generalise the above example by taking a general subgroup $H$ of $G.$ As earlier, let $\chi_1$ be the trivial character \underline{on $H$}. We wish to identify $\Ind^{G}_{H}\chi_1$ as a character \underline{on $G$}.

	Recall the action from \Cref{ex:cosetaction}. We had
	\begin{equation*} 
		\sigma : G \to S_{G/H}
	\end{equation*}
	given by 
	\begin{equation*} 
		\sigma_g(xH) = gxH.
	\end{equation*}
	Note that $xH \in \Fix(g)$ iff $gxH = xH$ iff $x^{-1}gx \in H.$

	Now, note that there $\md{H}$ many elements $x \in G$ which give the same coset $xH.$ Thus, $\md{\Fix(g)}$ is $1/\md{H}$ times the number of elements $x \in G$ such that $x^{-1}gx \in H.$

	Observe that we have
	\begin{equation*} 
		\hat{\chi}_1(x^{-1}gx) = \begin{cases}
			1 & x^{-1}gx \in H,\\
			0 & x^{-1}gx \notin H.
		\end{cases}
	\end{equation*}

	Thus, we see that
	\begin{equation*} 
		\Ind^{G}_{H}\chi_1(g) = \frac{1}{\md{H}}\sum_{x \in G} \hat{\chi}_1(x^{-1}gx) = \md{\Fix(g)}.
	\end{equation*}

	The above is precisely the character $\chi_{\widetilde{\sigma}}$ on $G.$ (\Cref{prop:charofpermrep}.)
\end{ex}

Let us now construct the induced representation. Fix a group $G$ and a subgroup $H \le G.$ Let $m = [G : H]$ be the index of $H$ in $G$ and let
\begin{equation*} 
	t_1, \ldots, t_m
\end{equation*}
be a transversal of cosets of $H$ in $G.$ 

\begin{defn}
	Given a vector space $V,$ we define the vector space
	\begin{equation*} 
		W = \bigoplus_{i = 1}^m t_iV
	\end{equation*}
	as the direct sum of $m$ isomorphic copies of $V.$ Instead of using the tuple notation, we denote the elements of $W$ as formal sums of the form
	\begin{equation*} 
		\sum_{i = 1}^{m} t_i v_i,
	\end{equation*}
	where $v_i \in V$ for $i = 1, \ldots, m.$
\end{defn}

\begin{defn}
	\label{defn:gttghg}
	Note that since $t_1, \ldots, t_m$ is a transversal, for each $g \in G$ and $i \in \{1, \ldots, m\},$ there exists a unique $g(i) \in \{1, \ldots, m\}$ and $h_g(i) \in H$ such that
	\begin{equation*} 
		gt_i = t_{g(i)}h_g(i).
	\end{equation*}
\end{defn}

\begin{defn}%[Induced representation abstract]
	\label{defn:inducedrep2}
	Let $\varphi : H \to \GL(V)$ be a representation.\\
	Put $W = \bigoplus_{i = 1}^m t_iV_i.$
	The \deff{induced representation} $\Ind^{G}_{H}\varphi : G \to \GL(W)$ is defined as:
	\begin{equation} \label{eq:inducedrepabs}
		\Ind^{G}_{H}\varphi(g)\left(\sum_{i = 1}^{m}t_iv_i\right) = \sum_{i = 1}^{m}t_{g(i)}\varphi(h_g(i))(v_i).
	\end{equation}
\end{defn}

\begin{rem}
	Note that we are yet to show that $\Ind^{G}_{H}\varphi(g)$ is actually linear and that $\Ind^{G}_{H}\varphi$ is a representation. We shall do both these things in a bit.

	Also note that the definition of the induction above is dependent on the set of coset representatives fixed. However, \Cref{thm:inducedreps} shows that the character of $\Ind^{G}_{H}\varphi$ does not depend on the transversal picked and thus, the character is actually defined uniquely up to equivalence. (Due to \Cref{cor:characbyequiv}.)
\end{rem}

% We now show that the two definitions ``coincide'' in the following sense.

We now construct a matrix representation for the above. Since constructing a matrix representation only depends on the basis, we may as well assume that $V = \mathbb{C}^d.$ We first set up some notation.

\begin{defn}
	Suppose
	\begin{equation*} 
		\varphi : H \to \GL_d(\mathbb{C})
	\end{equation*}
	is a representation of $H.$ We define $\widehat{\varphi} : G \to \GL_d(\mathbb{C})$ by
	\begin{equation*} 
		\widehat{\varphi}_x = \begin{cases}
			\varphi_x & x \in H,\\
			0 & x \notin H,
		\end{cases}
	\end{equation*}
	where the $0$ is the $d \times d$ zero matrix. 
\end{defn}

\begin{rem}
	The above definition also makes sense for a general representation $\varphi : G \to \GL(V)$ after we have fixed a basis $B$ for $V$ to represent the linear transformations. It is in this way that we interpret \Cref{thm:inducedmatrixandabs}.

	To be more precise, by $\varphi_x,$ we shall mean the matrix $[\varphi_x].$
\end{rem}

\begin{defn}%[Induced representation matrix]
	\label{defn:inducedrep1}
	Let $\varphi : H \to \GL_d(\mathbb{C})$ be a representation.\\
	The \deff{induced representation matrix} $\Ind^{G}_{H}\varphi : G \to \GL_{md}(\mathbb{C})$ is defined as following block matrix:
	\begin{equation} \label{eq:inducedrepmatrix}
		\Ind^{G}_{H}\varphi(g) = \begin{bmatrix}
			\widehat{\varphi}_{t_1^{-1}gt_1} & \widehat{\varphi}_{t_1^{-1}gt_2} & \cdots & \widehat{\varphi}_{t_1^{-1}gt_m}\\
			\widehat{\varphi}_{t_2^{-1}gt_1} & \widehat{\varphi}_{t_2^{-1}gt_2} & \cdots & \widehat{\varphi}_{t_2^{-1}gt_m}\\
			\vdots & \vdots & \ddots & \vdots\\
			\widehat{\varphi}_{t_m^{-1}gt_1} & \widehat{\varphi}_{t_m^{-1}gt_2} & \cdots & \widehat{\varphi}_{t_m^{-1}gt_m}\\
		\end{bmatrix}.
	\end{equation}
	We denote $\Ind^{G}_{H}\varphi$ as $\varphi^G,$ for ease of notation. Thus, $\varphi^G_g$ is an $m \times m$ block matrix with $d \times d$ blocks defined as
	\begin{equation*} 
		[\varphi^G_g]_{ij} = \widehat{\varphi}_{t_i^{-1}gt_j}
	\end{equation*}
	for $1 \le i, j \le m.$
\end{defn}

% \begin{rem}
%Note already that \Cref{defn:inducedrep2} is more general in that we don't assume the vector spaces to particularly be $\mathbb{C}^d.$
% \end{rem}

\begin{thm} \label{thm:inducedmatrixandabs}
	Let $\varphi : H \to \GL(V)$ be a representation. 
	Fix an ordered basis $B = (e_1, \ldots, e_d)$ of $V.$ Then,
	\begin{equation*} 
		B' = (t_1e_1, \ldots, t_1e_d, t_2e_1, \ldots, t_2e_d, \ldots, t_me_1, \ldots, t_me_d)
	\end{equation*}
	is an ordered basis of $\bigoplus_{i = 1}^m t_iV_i.$ \\
	Moreover, $\Ind^{G}_{H}\varphi(g)$ as defined in \Cref{eq:inducedrepabs} is a linear transformation whose matrix representation with respect to $B'$ is given by \Cref{eq:inducedrepmatrix}.
\end{thm}
\begin{proof} 
	That $B'$ is a basis is an easy check. The linearity of $\Ind^{G}_{H}\varphi(g)$ follows from the linearity of $\varphi(h_g(i))$ for all $i.$ Only the matrix representation is left to be shown.

	For the sake of clarity, we show that the first $d$ columns of \Cref{eq:inducedrepmatrix} are indeed what we should get. The general argument is identical. (Note that the first $d$ columns would mean the first column of blocks that appears in the matrix written.)

	For ease of notation, we denote $\Ind^{G}_{H}\varphi$ as defined in \Cref{defn:inducedrep1} by $\varphi^G.$

	To determine the $i$-th column for $1 \le i \le d,$ we need to look at the image of $t_1e_i$ under $\varphi^G_g.$ We have
	\begin{equation*} 
		gt_1 = t_{g(1)}h_g(1) \quad\text{or}\quad t_{g(1)}^{-1}gt_1 = h_g(1).
	\end{equation*}
	Thus, we have
	\begin{equation*} 
		\varphi^G_g(t_1e_i) = t_{g(1)}\varphi_{h_g(1)}e_i \in t_{g(1)}V \le \bigoplus_{j = 1}^m t_jV_j.
	\end{equation*}

	Note that if $j \neq g(1),$ then $t_{g(1)}^{-1}gt_1 \notin H$ and hence, the $j$-th block in the first (block) column of \Cref{eq:inducedrepmatrix} will be $0.$ This is consistent with the equation above.

	On the other hand, for $j = g(1),$ the above equation tells us that the $j$-th block should be the matrix representation of 
	\begin{equation*} 
		\varphi_{h(g_1)} = \varphi_{t_{g(1)}^{-1}gt_1}.
	\end{equation*}
	Again, this is consistent with \Cref{eq:inducedrepmatrix} (since $t_{g(1)}^{-1}gt_1 \in H$) and we are done.
\end{proof}

\begin{rem}
	Thus, by this, we may use the two definitions interchangeably after fixing a basis. We will also use the notation $\varphi^G$ when there is no confusion of $H.$

	Note that we are yet to show that $\Ind^{G}_{H}\varphi$ is actually a representation. Before that, we do some computations.
\end{rem}

\begin{lem} \label{lem:gttghg}
	As in the notation of \Cref{defn:gttghg}, we have
	\begin{equation*} 
		g'(g(i)) = (g'g)(i) \andd h_{g'g}(i) = h_{g'}(g(i))h_g(i).
	\end{equation*}
\end{lem}
\begin{proof} 
	Note that
	\begin{equation*} 
		gt_i = t_{g(i)}h_g(i),
	\end{equation*}
	by definition and hence,
	\begin{equation*} 
		g'gt_i = g't_{g(i)}h_g(i).
	\end{equation*}
	On the other hand, applying \Cref{defn:gttghg} to the left hand side gives
	\begin{equation*} 
		t_{(g'g)(i)}h_{g'g}(i) = g't_{g(i)}h_g(i)
	\end{equation*}
	or
	\begin{equation*} 
		g't_{g(i)} = t_{(g'g)(i)}h_{g'g}(i)h_g(i)^{-1}.
	\end{equation*}
	Comparing with \Cref{defn:gttghg} gives
	\begin{equation*} 
		g'(g(i)) = (g'g(i)) \andd h_{g'}(g(i)) = h_{g'g}(i)h_g(i)^{-1}.
	\end{equation*}
	This yields the desired equalities.
\end{proof}

\begin{cor} \label{cor:gttghg}
	For all $g, g' \in G,$ $i \in \{1, \ldots, m\}$ and $v \in V,$ the equality
	\begin{equation*} 
		\varphi^G_{g'}(\varphi^G_g(t_iv)) = \varphi^G_{g'g}(t_iv)
	\end{equation*}
	holds.
\end{cor}
\begin{proof} 
	For ease of notation, we shall denote $\varphi_h(v)$ by $h \cdot v$ for $h \in H.$ Note that this actually is an action of $H$ on $V.$

	Observe that
	\[\begin{WithArrows}[displaystyle] 	
		\varphi^G_{g'}(\varphi^G_g(t_iv)) &= \varphi^G_{g'}\left(t_{g(i)}h_g(i) \cdot v\right)\\
		&= t_{g'(g(i))} h_{g'}(g(i))\cdot(h_g(i) \cdot v)\\
		&= t_{g'(g(i))} (h_{g'}(g(i))h_g(i)) \cdot v \Arrow{\Cref{lem:gttghg}}\\
		&= t_{(g'g)(i)} h_{g'g}(i) \cdot v\\
		&= \varphi^G_{g'g}(t_iv).
	\end{WithArrows}\] 
	Thus, we are done.
\end{proof}

% \begin{rem}
% 	Note that the above definition is in terms of the coset representatives that we have chosen. However, the next theorem will show that the representation obtained is determined up to equivalence, independent of the representatives.
% \end{rem}

\begin{thm}[Induced representations] \label{thm:inducedreps}
	Let $H$ be a subgroup of $G$ of index $m$ and suppose that $\varphi : H \to \GL(V)$ is a representation of $H.$ Then, $\Ind^{G}_{H}\varphi : G \to \GL\left(\bigoplus_{1}^m t_iV_i\right)$ is a representation and $\chi_{\Ind^{G}_{H}\varphi} = \Ind^{G}_{H}\chi_\varphi.$ In particular, $\Ind^{G}_{H}$ maps characters to characters.
\end{thm}
\begin{proof} 
	To show that it is a representation, we work with \Cref{defn:inducedrep2}. By \Cref{lem:determininggrouphomoring}, it suffices to show that $\varphi^G_1$ is identity and $\varphi^G_{g'g} = \varphi^G_{g'} \circ \varphi^G_{g}.$ Put $W \vcentcolon= \bigoplus_{1}^m t_iV_i$

	For the first, we note that $1 t_i = t_i \cdot 1$ and hence, $1(i) = i$ and $h_1(i) = 1$ for all $i.$ From this, it follows that $\varphi^G_1(w) = w$ for all $w \in W.$

	The fact that it is multiplicative follows from \Cref{cor:gttghg} since $\{t_i v \mid 1 \le i \le m,\; v \in V\}$ forms a spanning set for $W.$

	We now show that $\chi_{\Ind^{G}_{H}\varphi} = \Ind^{G}_{H}\chi_\varphi.$ For this, we work with \Cref{defn:inducedrep1} (after fixing a basis for $V$ and getting the natural basis for $W$). We then have
	\[\begin{WithArrows}[displaystyle]
		\chi_{\varphi^G}(g) &= \sum_{i = 1}^{m} \trace(\widehat{\varphi}_{t_1^{-1}gt_i})\\
		&= \sum_{i = 1}^{m}\widehat{\chi_\varphi}(t_i^{-1}gt_i) \Arrow{\Cref{prop:shortindformula}}\\
		&= \Ind^{G}_{H}\chi_\varphi.
	\end{WithArrows}\]
	The equality from the first line to the second is verified by considering the cases $t_i^{-1}gt_i \notin H$ and $\in H.$
\end{proof}

Let us now look at some examples.
\begin{ex}[Induction on dihedral groups] \label{ex:inductdihedral}
	As usual, let $D_n$ denote the dihedral group with $r$ denoting rotation by $2\pi/n$ and $s$ denoting a reflection. Let $G = D_n$ and $H = \langle r\rangle.$ $H$ is a cyclic group of order $n$ and index $2.$ Recall that all the degree one representations of cyclic groups (\Cref{ex:ZnZCstardeg1}) are as follows.

	For $0 \le k \le n - 1,$ define $\chi_k : H \to \mathbb{C}^*$ as $\chi_k(r^m) = \omega_n^{km}.$ We now compute the induced representations $\varphi^{(k)} = \Ind^{G}_{H}\chi_k.$ We choose the coset representatives $t_1 = 1$ and $t_2 = s.$

	We now construct the matrix \Cref{eq:inducedrepmatrix}. For this, we need to compute $t_i^{-1}gt_j$ for all $g \in G$ and $i, j \in \{1, 2\}.$ Note that the elements of $G$ are either of the form $r^m$ or $sr^m$ for $m = 0, \ldots, n- 1.$ \\
	Thus, we have the following:
	\begin{align*} 
		\begin{array}{lll}
			t_1^{-1}r^mt_1 = r^m & \phantom{hi}\hspace{1.5cm}\phantom{hi} & t_1^{-1}sr^mt_1 = sr^m,\\
			t_1^{-1}r^mt_2 = sr^{-m} & & t_1^{-1}sr^mt_2 = r^{-m},\\
			t_2^{-1}r^mt_1 = sr^m & & t_2^{-1}sr^mt_1 = r^m,\\
			t_2^{-1}r^mt_2 = r^{-m} & & t_2^{-1}sr^mt_2 = sr^{-m}.
		\end{array}
	\end{align*}
	Note that $r^m, r^{-m} \in H$ and $sr^m, sr^{-m} \notin H.$ Thus, we have
	\begin{align*} 
		\varphi^{(k)}(r^m) &= \two{\widehat{\chi}_k(r^m)}{\widehat{\chi}_k(sr^{-m})}{\widehat{\chi}_k(sr^{m})}{\widehat{\chi}_k(r^{-m})} = \two{\omega_n^m}{}{}{\omega_n^{-m}},\\
		\varphi^{(k)}(sr^m) &= \two{\widehat{\chi}_k(sr^m)}{\widehat{\chi}_k(r^{-m})}{\widehat{\chi}_k(r^{m})}{\widehat{\chi}_k(sr^{-m})} = \two{}{\omega_n^{-m}}{\omega_n^m}{}.
	\end{align*}
	From the above, we note that $\Ind^{G}_{H}\chi_k(r^m) = 2\cos(2\pi km/n)$ and $\Ind^{G}_{H}\chi_k(sr^m) = 0.$

	The astute reader might have seen the resemblance with \Cref{ex:finishingDn}. Indeed, $\varphi^{(k)}$ in this example is precisely $\varphi_k$ from that example. This shows that all the degree two irreducible representations are actually obtained from degree one representations of $H.$ (Of course, not all give inequivalent ones. For that, we restrict $k$ to satisfy $1 \le k < \frac{n}{2},$ which is consistent with our earlier observation as well.)
\end{ex}

\begin{ex}[Induction on quaternions] \label{ex:inductquaternion}
	Let $Q = \{\pm 1, \pm \I, \pm \J, \pm \K\}$ denote the group of quaternions. Note that the center is $Z(Q) = \{\pm 1\}$ and $Q/Z(Q) \cong \mathbb{Z}/2\mathbb{Z} \times \mathbb{Z}/2\mathbb{Z}$ is abelian. Since $Q$ was not abelian, we see that $Z(Q)$ is the commutator subgroup.\footnotemark

	As noted earlier, this means that there are exactly four irreducible degree one representations of $Q.$ Now, we know that
	\begin{equation*} 
		d_1^2 + \cdots + d_s^2 = 8.
	\end{equation*}
	Thus, for the remaining, we have
	\begin{equation*} 
		d_5^2 + \cdots + d_s^2 = 4.
	\end{equation*}
	However, $d_5 \ge 2$ and thus, we see that $s = 5$ and $d_5 = 2.$

	In other words, there is only one remaining irreducible representation, which is of degree two. Let us now obtain that.

	Defining $H = \langle \I\rangle,$ we see that $\md{H} = 4$ and $[Q : H] = 2.$ Choose the coset representatives $t_1 = 1$ and $t_2 = \J.$ Consider the representation $\chi : H \to \mathbb{C}^*$ given by $\varphi(\I^k) = \iota^k.$ Then, we have
	\begin{align*} 
		\begin{array}{lll}
		\varphi^Q_{\pm 1} = \pm \two{1}{}{}{1}, && \varphi^Q_{\pm \I} = \pm \two{\iota}{}{}{-\iota},\\ && \\
		\varphi^Q_{\pm \J} = \pm \two{}{-1}{1}{}, && \varphi^Q_{\pm \K} = \pm \two{}{-\iota}{-\iota}{}.
		\end{array}
	\end{align*}
	Note that $\varphi^Q_{\I}$ and $\varphi^Q_{\K}$ have no common eigenvector and hence, $\varphi^Q$ is irreducible.

	The character table of $Q$ is given as follows.

	\captionsetup{type=figure}
	\[\begin{array}{rrrrrr}
		 & [1] & [-1] & [\I] & [\J] & [\K]\\
		\thiccline
		\chi_{1} & 1 & 1 & 1 & 1 & 1\\
		\chi_{2} & 1 & 1 & -1 & 1 & -1\\
		\chi_{3} & 1 & 1 & 1 & -1 & -1 \\
		\chi_{4} & 1 & 1 & -1 & -1 & 1\\
		\chi_{5} & 2 & -2 & 0 & 0 & 0
	\end{array}\]
	\captionof{table}{Character table of $Q$} \label{tab:charquat}

	The first four rows are obtained using the Character table of Klein group. (\Cref{tab:charklein})
\end{ex}
\footnotetext{Recall that the commutator subgroup is the smallest normal subgroup such that the quotient is abelian.}

\subsection{Mackey's Irreducibility Criterion}
We now wish to see if the induction of an irreducible representation is again irreducible. This was not the case in \Crefrange{ex:regrepinduct}{ex:permrepinduct} but was the case in \Crefrange{ex:inductdihedral}{ex:inductquaternion}. \\
Note that by \Cref{cor:irrediffnormone}, this can be answered by computation of
\begin{equation*} 
	\langle \Ind^{G}_{H}\varphi, \Ind^{G}_{H}\varphi\rangle.
\end{equation*}
Now, by \nameref{thm:frobeniusreciprocity}, we have
\begin{equation*} 
	\langle \Ind^{G}_{H}\varphi, \Ind^{G}_{H}\varphi\rangle = \langle \varphi, \Res^{G}_{H}\Ind^{G}_{H}\varphi\rangle.
\end{equation*}
Thus, our problem reduces to studying $\Res^{G}_{H}\Ind^{G}_{H}\varphi.$

\begin{defn}%[Disjoint representations]
	Two representations $\varphi$ and $\rho$ of $G$ are said to be \deff{disjoint} if they have no common irreducible constituent. Equivalently, $\langle \chi_\varphi, \chi_\rho\rangle = 0.$
\end{defn}

\begin{rem}
	To see the equivalence, note that if
	\begin{align*} 
		\rho &\sim n_1\varphi^{(1)} \oplus \cdots \oplus n_s\varphi^{(s)},\\
		\varphi &\sim m_1\varphi^{(1)} \oplus \cdots \oplus m_s\varphi^{(s)},
	\end{align*}
	then they are disjoint iff $n_im_i = 0$ for all $1 \le i \le s.$

	On the other hand, since $n_im_i \ge 0,$ we have that
	\begin{equation*} 
		\langle \chi_\varphi, \chi_r\rangle = 0 \iff \sum_{i = 1}^{s}m_in_i = 0 \iff m_in_i = 0 \; \forall i.
	\end{equation*}
\end{rem}

We had noted that we wished to study $\Res^{G}_{H}\Ind^{G}_{H}\varphi.$ As it turns out, studying that it not more difficult than studying $\Res^{G}_{H}\Ind^{G}_{K}\varphi,$ where $H, K \le G.$ 

\begin{defn}
	Let $K \le G$ and $s \in G.$ Then, for $f \in Z(L(K)),$ $f^s \in Z(L(sKs^{-1}))$ is defined by
	\begin{equation*} 
		f^s(x) = f(s^{-1}xs)
	\end{equation*}
	for all $x \in K.$
\end{defn}

Note that $Z(L(sKs^{-1}))$ makes sense since $sKs^{-1}$ is again a subgroup of $G.$ Moreover, $f^s(x)$ makes sense for $x \in sKs^{-1}$ since $s^{-1}xs \in K$ then. We must check that $f^s$ is indeed a class function.

\begin{proof} 
	Let $x, y \in sKs^{-1}.$ Then, there exist $k, k' \in K$ such that $x = sks^{-1}$ and $y = sk's^{-1}.$ We then have
	\begin{align*} 
		f^s(yxy^{-1}) &= f(s^{-1}yxy^{-1}s)\\
		&= f(s^{-1}sk's^{-1}sks^{-1}sk'^{-1}s^{-1}s)\\
		&= f(k'kk'^{-1})\\
		&= f(k),
	\end{align*}
	where the last equality follows since $f$ is a class function on $K.$

	The proof is complete upon noting that
	\begin{equation*} 
		f(k) = f(s^{-1}xs) = f^s(x). \qedhere
	\end{equation*}
\end{proof}

\begin{defn}
	Let $H$ be a subgroup of $G$ and let $\varphi : H \to \GL_d(\mathbb{C})$ be a representation. For $s \in G,$ we define the representation $\varphi^s : sHs^{-1} \to \GL_d(\mathbb{C})$ by
	\begin{equation*} 
		\varphi^s(x) = \varphi(s^{-1}xs).
	\end{equation*}
\end{defn}

As earlier, the above definition makes sense. We only need to check that $\varphi^s$ is indeed a representation.

\begin{proof} 
	Only that $\varphi^s$ is a homomorphism needs to be checked. Let $x, x' \in sHs^{-1}.$ Let $h, h' \in H$ be such that $x = shs^{-1}$ and $x' = sh's^{-1}.$ Then,
	\begin{equation*} 
		\varphi^s(xx') = \varphi(s^{-1}xx's) = \varphi(s^{-1}xss^{-1}x's) = \varphi(hh') = \varphi(h)\varphi(h') = \varphi^s(x)\varphi^s(x').
	\end{equation*}
	The second last equality follows since $\varphi$ was a homomorphism to begin with.
\end{proof}

Before the next theorem, one must recall the notion of \nameref{subsubsec:doublecosets}.

\begin{thm}[Mackey] \label{thm:mackey}
	Let $H, K \le G$ and let $S$ be a transversal of double coset representatives for $\dcos{H}{G}{K}.$ Then, for $f \in Z(L(K)),$ the equality
	\begin{equation*} 
		\Res^{G}_{H}\Ind^{G}_{K}f = \sum_{s \in S} \Ind^{H}_{H \cap sKs^{-1}}\Res^{sKs^{-1}}_{H \cap sKs^{-1}}f^s
	\end{equation*}
	holds.
\end{thm}
Before the proof, one can visualise the left and right terms as ``transferring'' a function from one domain to another in terms of the following triangles:
\begin{center}
	\begin{tikzcd}
	  & G \arrow[ld, "\Res"'] &                                          \\
	H &                       & K \arrow[lu, "\Ind"'] \arrow[ll, dashed]
	\end{tikzcd}
	\begin{tikzcd}
	H &                                    & sKs^{-1} \arrow[ll, dashed] \arrow[ld, "\Res"] \\
	  & H \cap sKs^{-1} \arrow[lu, "\Ind"] &                                               
	\end{tikzcd}
\end{center}
\begin{proof} 
	The main idea behind the proof is selecting a ``correct'' set of (left) coset representatives for $K$ in $G.$ 

	\textbf{Step 1.} For each $s \in S,$ fix a complete set $V_s \subset H$ of coset representatives of $H \cap sKs^{-1}$ in $H.$ (Note that this makes sense because $H \cap sKs^{-1}$ is indeed a subgroup of $H.$)\\
	Thus, we have
	\begin{equation} \tag{$*$} \label{eq:007}
		H = \bigsqcup_{v \in V_s}v(H \cap sKs^{-1}).
	\end{equation}

	\textbf{Step 2.} Now, fix $s \in S.$ We show that
	\begin{equation} \tag{$**$} \label{eq:008}
		HsK = \bigsqcup_{v \in V_s}vsK.
	\end{equation}

	$(\subset)$ Let $h \in H, k \in K$ be arbitrary. By \Cref{eq:007}, we can write $h = vh'$ for some $v \in V_s$ and $h' \in H \cap sKs^{-1}.$ Thus, we have
	\begin{equation*} 
		hsk = vh'sk = vs\underbrace{s^{-1}hs}_{\in K}k \in vsK.
	\end{equation*}
	Since a typical element of $HsK$ is of the form $hsk,$ we are done.

	$(\supset)$ This is clear since $V_s \subset H$ by construction.

	We now show that the union on the right is actually disjoint. Suppose that $vsK = v'sK$ for $v, v' \in V_s.$ Then, 
	\begin{equation*} 
		s^{-1}v'^{-1}vs \in K \quad\text{or}\quad v'^{-1}v \in sKs^{-1}.
	\end{equation*}
	Since $v$ and $v'$ are elements of $H$ to begin with, we see that
	\begin{equation*} 
		v'^{-1}v \in H \cap sKs^{-1}.
	\end{equation*}
	Since $v$ and $v'$ are from a transversal of cosets, we must have $v = v'.$

	\textbf{Step 3.} For $s \in S,$ define $T_s = \{vs \mid v \in V_s\}.$ We show that $T_s$ and $T_{s'}$ are disjoint if $s \neq s'.$

	Suppose there is an element in the intersection; then, there exist $v \in V_s$ and $v' \in V_{s'}$ such that $vs = v's'.$ By \Cref{eq:008}, we see that
	\begin{equation*} 
		HsK \supset vsK = v's'K \subset Hs'K
	\end{equation*}
	and hence, $HsK \cap Hs'K \neq \emptyset.$ Since $s, s'$ belong to a transversal of double cosets, it follows that $s = s'.$

	\textbf{Step 4.} Let $T = \bigsqcup_{s \in S}T_s.$ Note that
	\begin{equation*} 
		G = \bigsqcup_{s \in S}HsK = \bigsqcup_{s \in S}\bigsqcup_{v \in V_s}vsK = \bigsqcup_{s \in S}\bigsqcup_{t \in T_s}tK = \bigsqcup_{t \in T}tK.
	\end{equation*}
	Thus, $T$ is a complete set of coset representatives of $K$ in $G.$

	\textbf{Step 5.} For $h \in H,$ we use \Cref{prop:shortindformula} to note that
	\begin{align*} 
		\Ind^{G}_{K}f(h) &= \sum_{t \in T} \hat{f}(t^{-1}ht)\\
		&= \sum_{s \in S}\sum_{t \in T_s} \hat{f}(t^{-1}ht)\\
		&= \sum_{s \in S}\sum_{v \in V_s} \hat{f}(s^{-1}v^{-1}hvs)\\
		&= \sum_{s \in S}\sum_{\substack{v \in V_s \\ v^{-1}hv \in sKs^{-1}}} f(s^{-1}v^{-1}hvs)\\	
		&= \sum_{s \in S}\sum_{\substack{v \in V_s \\ v^{-1}hv \in sKs^{-1}}} f^s(v^{-1}hv)
	\end{align*}
	As noted earlier, $v \in H$ and thus, $v^{-1}hv \in H.$ Since the summation above is only over those $v$ such that $v^{-1}hv \in sKs^{-1},$ we see that $v^{-1}hv \in H \cap sKs^{-1}.$ Thus, we get
	\begin{align*} 
		\Ind^{G}_{K}f(h) &= \sum_{s \in S}\sum_{\substack{v \in V_s \\ v^{-1}hv \in H \cap sKs^{-1}}} \Res^{sKs^{-1}}_{H \cap sKs^{-1}}f^s(v^{-1}hv).
	\end{align*}
	Using the second form of equality given in \Cref{prop:shortindformula}, we see that 
	\begin{align*} 
		\Ind^{G}_{K}f(h) &= \sum_{s \in S}\Ind^{H}_{H \cap sKs^{-1}} \Res^{sKs^{-1}}_{H \cap sKs^{-1}}f^s(v^{-1}hv). \qedhere
	\end{align*}
\end{proof}

We now deduce Mackey's irreducibility criterion for when the induction of an irreducible representation is again irreducible. Before that, we isolate a calculation.

\begin{cor}
	Let $H$ be a subgroup of $G$ and let $\varphi : H \to \GL_d(\mathbb{C})$ be a representation with character $\chi.$ Then,
	\begin{equation*} 
		\|\Ind^{G}_{H}\chi\| \ge \|\chi\|.
	\end{equation*}
	More precisely, if $S$ is any set of double coset representatives of $\dcos{H}{G}{H},$ then we have
	\begin{equation*} 
		\|\Ind^{G}_{H}\chi\|^2 = \|\chi\|^2 + \sum_{s \in S \setminus H} \langle \Res^{sHs^{-1}}_{H \cap sHs^{-1}}\chi^s, \Res^{H}_{H \cap sHs^{-1}}\chi\rangle.
	\end{equation*}
\end{cor}
\begin{proof} 
	We first replace the coset representative of $H$ by $1.$ Note that then $S \setminus \{1\} = S \setminus H.$

	For $s = 1,$ note that $H \cap sHs^{-1} = H$ and $\chi^s = \chi.$ In particular, for $s = 1,$ we have
	\begin{equation*} 
		\Ind^{H}_{H \cap sHs^{-1}}\Res^{sHs^{-1}}_{H \cap sHs^{-1}}\chi^s = \Ind^{H}_{H}\Res^{H}_{H}\chi = \chi.
	\end{equation*} 
	Now, let $S^* = S \setminus \{1\}.$ By \Cref{thm:mackey}, we get
	\begin{equation*} 
		\Res^{G}_{H}\Ind^{G}_{H}\chi = \chi + \sum_{s \in S^*} \Ind^{H}_{H \cap sHs^{-1}}\Res^{sHs^{-1}}_{H \cap sHs^{-1}}\chi^s.
	\end{equation*}
	By Frobenius reciprocity, we have
	\[\begin{WithArrows}[displaystyle]
		\langle \Ind^{G}_{H}\chi, \Ind^{G}_{H}\chi\rangle &= \langle \chi, \Res^{G}_{H}\Ind^{G}_{H}\chi\rangle \Arrow[i]{\Cref{cor:innerprodproperties} shows this quantity is real}\\
		&= \langle \Res^{G}_{H}\Ind^{G}_{H}\chi, \chi\rangle\\
		&= \langle \chi, \chi\rangle + \sum_{s \in S^*} \langle \Ind^{H}_{H \cap sHs^{-1}}\Res^{sHs^{-1}}_{H \cap sHs^{-1}}\chi^s, \chi\rangle\\
		&= \langle \chi, \chi\rangle + \sum_{s \in S^*} \langle \Res^{sHs^{-1}}_{H \cap sHs^{-1}}\chi^s, \Res^{H}_{H \cap sHs^{-1}}\chi\rangle
	\end{WithArrows}\]
	
	and hence,
	\begin{equation*} 
		\|\Ind^{G}_{H}\chi\|^2 = \|\chi\|^2 + \sum_{s \in S \setminus \{1\}} \langle \Res^{sHs^{-1}}_{H \cap sHs^{-1}}\chi^s, \Res^{H}_{H \cap sHs^{-1}}\chi\rangle.
	\end{equation*}
	From the above, the first inequality follows since the inner product on the left is of characters (restriction of characters is again a character) and hence, is non-negative, by \Cref{cor:innerprodproperties}.
\end{proof}

\begin{thm}[Mackey's irreducibility criterion] \label{thm:mackeyirredcrit}
	Let $H$ be a subgroup of $G$ and let $\varphi : H \to \GL_d(\mathbb{C})$ be a representation. Then, $\Ind^{G}_{H}\varphi$ is irreducible if and only if
	\begin{enumerate}
		\item $\varphi$ is irreducible;
		\item the representations $\Res^{H}_{H \cap sHs^{-1}}\varphi$ and $\Res^{sHs^{-1}}_{H \cap sHs^{-1}}\varphi^s$ are disjoint for all $s \notin H;$ that is, 
		\begin{equation*} 
			\langle \Res^{H}_{H \cap sHs^{-1}}\chi, \Res^{sHs^{-1}}_{H \cap sHs^{-1}}\chi^s\rangle = 0,
		\end{equation*}
		for all $s \notin H.$
	\end{enumerate}
\end{thm}

Note that it makes sense to talk about those representations being disjoint since both are representations of the same group $H \cap sHs^{-1}.$

\begin{proof} 
	Let $\chi = \chi_\varphi.$ Let $S$ be a set of double coset representatives of $\dcos{H}{G}{H}.$ Assume without loss of generality, $1 \in S.$ By the preceding theorem, we have
	\begin{equation*} 
		\|\Ind^{G}_{H}\chi\|^2 = \|\chi\|^2 + \sum_{s \in S \setminus \{1\}} \langle \Res^{sHs^{-1}}_{H \cap sHs^{-1}}\chi^s, \Res^{H}_{H \cap sHs^{-1}}\chi\rangle.
	\end{equation*}

	Thus, by \Cref{cor:irrediffnormone}, $\Ind^{G}_{H}\varphi$ is irreducible iff the above quantity is $1.$ Using \Cref{cor:innerprodproperties}, we see that $\Ind^{G}_{H}\varphi$ is irreducible iff 
	\begin{equation*} 
		\langle \chi, \chi\rangle = 1 \andd \langle \Res^{sHs^{-1}}_{H \cap sHs^{-1}}\chi^s, \Res^{H}_{H \cap sHs^{-1}}\chi\rangle = 0
	\end{equation*}
	for all $s \in S^* \setminus \{1\}.$ 

	Thus, we get that $\chi$ is irreducible and the representations $\Res^{sHs^{-1}}_{H \cap sHs^{-1}}\varphi^s$ and $\Res^{H}_{H \cap sHs^{-1}}\varphi$ are disjoint. We are using the fact that $\Res^{sHs^{-1}}_{H \cap sHs^{-1}}\chi^s$ is indeed the character of $\Res^{G}_{H}\varphi^s.$

	Now, note that given any $s \notin H,$ we can choose $S$ to include $s.$ The theorem follows.
\end{proof}

\begin{rem}
	As the proof the theorem shows, instead of checking the disjointness for \emph{all} $s \notin H,$ we only need to check for a set of double coset representatives. 
\end{rem}

\begin{prop}
	Let $N \unlhd G$ and $\varphi : N \to \GL_d(\mathbb{C})$ be a representation. Then, for every $s \in G,$ $\varphi^s$ is irreducible if and only if $\varphi$ is.
\end{prop}
\begin{proof} 
	Fix $s \in G.$ Note that $sNs^{-1} = N.$ Let $W \le \mathbb{C}^n$ be an $N$-invariant subspace with respect to $\varphi^s.$ We now show that it $N$-invariant with respect to $\varphi.$

	Indeed, for $w \in W$ and $n \in N,$ we note that
	\begin{equation*} 
		\varphi(n)(w) = \varphi^s(sns^{-1})(w) \in W.
	\end{equation*}

	A similar computation shows that the converse is also true. Thus, the $N$-invariant subspaces of $\mathbb{C}^n$ with respect to $\varphi$ and $\varphi^s$ coincide and the result follows.
\end{proof}

\begin{rem}
	With the above proposition, we see that \nameref{thm:mackeyirredcrit} works very well if $H$ is a normal subgroup of $G$ and $\varphi$ an irreducible representation of $H.$ By \Cref{prop:dcosetsofnormal}, we know that $G/H = \dcos{H}{G}{H}.$ By the above proposition, $\varphi^s$ is irreducible for every $s \in G.$ So the criterion reduces to checking that $\varphi$ and $\varphi^s$ are inequivalent, as $s$ ranges over a set of coset representatives.
\end{rem}

\begin{ex}[Checking the dihedral induction]
	As an example, let us apply the criterion on \Cref{ex:inductdihedral}. We already saw that the induction is indeed irreducible. We now verify using Mackey. Fix some $k$ such that $1 \le k < \frac{n}{2}$ and let $\varphi = \chi_k.$ (Where $\chi_k$ is as in \Cref{ex:inductdihedral}.)

	As remarked, it suffices to only check $\varphi$ and $\varphi^s$ are inequivalent. (Here $s$ denotes the reflection element of $D_n.$) Now, since $H = \langle r\rangle$ is normal, we have $sHs^{-1} = H.$ Thus, the restrictions are again $\varphi$ and $\varphi^s.$ Since these are degree one representations, it suffices to show that they are distinct to conclude inequivalence.

	Note that $\varphi(r) = \omega_n^k$ and
	\begin{equation*} 
		\varphi^s(r) = \varphi(s^{-1}rs) = \varphi(r^{-1}) = \omega_n^{-k}.
	\end{equation*}
	Now, since $1 \le k < \frac{n}{2},$ we have $\omega_n^k \neq \omega_n^{-k}$ and hence, the representations are inequivalent.
\end{ex}

\begin{ex}
	Let $p$ be a prime and let
	\begin{equation*} 
		G = \left\{\two{a}{b}{}{[1]} \mid a \in (\mathbb{Z}/p\mathbb{Z})^*,\; b \in \mathbb{Z}/p\mathbb{Z}\right\} \le \GL_2(\mathbb{Z}/p\mathbb{Z}).
	\end{equation*}
	One can note that $\md{G} = p(p - 1).$

	We observe the multiplication in $G$ to be
	\begin{equation*} 
		\two{a}{b}{}{[1]}\two{a'}{b'}{}{[1]} = \two{aa'}{ab' + b}{}{[1]}.
	\end{equation*}
	Thus, $\psi : G \to (\mathbb{Z}/p\mathbb{Z})^*$ defined by
	\begin{equation*} 
		\two{a}{b}{}{[1]} \mapsto a
	\end{equation*}
	is an onto group homomorphism with kernel
	\begin{equation*} 
		H = \left\{\two{[1]}{b}{}{[1]} \mid b \in \mathbb{Z}/p\mathbb{Z}\right\}.
	\end{equation*}
	In particular, we have $H \unlhd G$ and $G/H \cong (\mathbb{Z}/p\mathbb{Z})^*.$ It is easy to see that a complete set of coset representatives of $H$ in $G$ is 
	\begin{equation*} 
		S = \left\{\two{a}{}{}{[1]} \mid a \in (\mathbb{Z}/p\mathbb{Z})^*\right\}.
	\end{equation*}

	Now, consider the representation $\varphi : H \to \mathbb{C}^*$ defined as
	\begin{equation*} 
		\varphi\left(\two{[1]}{[b]}{}{[1]}\right) = \omega_p^b.
	\end{equation*}
	Now, if $s = \two{[a]}{}{}{[1]}$ with $[a] \neq [1],$ we have
	\begin{equation*} 
		\varphi^s\left(\two{[1]}{[b]}{}{[1]}\right) = \varphi\left(\two{[1]}{[ab]}{}{[1]}\right) = \omega_p^{ab}
	\end{equation*}
	and hence, $\varphi$ and $\varphi^s$ are inequivalent (since they are distinct degree one representations). By Mackey's criterion, we see that $\Ind^{G}_{H}\varphi$ is an irreducible representation. Note that this has degree $1\cdot[G : H] = p - 1.$

	On the other hand, we can lift the $p - 1$ inequivalent irreducible degree one representations of $G/H$ to get $p - 1$ irreducible degree one representations of $G.$ Now, note that
	\begin{equation*} 
		(p - 1)\cdot1^2 + 1\cdot(p - 1)^2 = p(p - 1) = \md{G}.
	\end{equation*}
	Thus, we have found all the irreducible representations of $G.$
\end{ex}
\section{Representation Theory of the Symmetric Groups} \label{sec:06}

At this point, we suggest the reader to recall \Cref{subsec:partsandtableaux}.

There will be a lot of new notation involved in this part. A table can be found on Page \pageref{bookkeeping} to keep track of the various objects involved.

\begin{rem} \label{rem:irredrepsconjclassSn}
	Recall that we had seen \Cref{thm:descconjclassSn} which said that $\sigma$ and $\sigma'$ are conjugates in $S_n$ iff $\type(\sigma) = \type(\sigma').$ Thus, the number of irreducible representations of $S_n$ is precisely the number of partitions of $n.$ (\Cref{cor:numirredrepsconjclass}.) We wish to give an explicit bijection.
\end{rem}

\begin{defn}%[]
	If $X \subset \{1, \ldots, n\},$ we identity $S_X$ with the subgroup of $S_n$ consisting of permutations that fix elements outside $X.$ Note that $\md{S_{X}} = \md{X}!.$
\end{defn}
Note that the above involves a bit of abuse of notation since the same $X$ can be a subset of $\{1, \ldots, n\}$ for different $n.$ However, the ambient $n$ will be clear from context.

\begin{defn}%[Column stabiliser]
	Let $t$ be a Young tableau. Then, the \deff{column stabiliser} of $t$ is the subgroup $C_t$ of $S_n$ preserving the columns of $t.$ That is, $\sigma \in C_t$ if and only if $\sigma(i)$ and $i$ are in the same column for each $i \in \{1, \ldots, n\}.$
\end{defn}

With the above definition, we could have written \Cref{item:008} of \Cref{prop:domprop} as ``There exists $\sigma \in C_{[T^\lambda]}$ such that $\sigma [T^\lambda] = u^\lambda$.''

\begin{ex}
	Consider the tableau
	\begin{equation*} 
		t = \begin{ytableau}
			1 & 3 & 7\\
			4 & 5\\
			2 & 6
		\end{ytableau}.
	\end{equation*}
	Then, $C_t \le S_7$ is given by $S_{\{1, 2, 4\}}S_{\{3, 5, 6\}}S_{\{7\}}.$ (Recall that given a group $G$ and subsets $S, T \subset G$, the subset $ST \subset G$ is defined by $ST = \{st \mid s \in S, t \in S\}.$ This extends to any finite product of subsets.)\\
	We have
	\begin{equation*} 
		C_t = S_{\{1, 2, 4\}}S_{\{3, 5, 6\}}S_{\{7\}} \cong S_{\{1, 2, 4\}} \times S_{\{3, 5, 6\}} \times S_{\{7\}},
	\end{equation*}
	where the last isomorphism follows because the sets appearing are subgroups such that the pairwise intersection is trivial and that the elements from any two subgroups commute with each other.

	Thus, we have $\md{C_t} = 3!\cdot3!\cdot1! = 36.$
\end{ex}

\begin{lem} \label{lem:colstabrelation}
	Let $t$ be a $\lambda$-tableau and $\sigma \in S_n.$ Then, $C_{\sigma t} = \sigma C_t \sigma^{-1}.$
\end{lem}
\begin{proof} 
	Let $\tau \in S_n.$ If $X_i$ is the set of entries in the $i$-th column of $t,$ then $\sigma(X_i)$ is that of the $i$-th column of $\sigma t.$ Thus, $\tau$ stabilises $X_i$ iff $\tau\sigma^{-1}$ is a bijection from $\sigma(X_i)$ to $X_i$ iff $\sigma\tau\sigma^{-1}$ stabilises $\sigma(X_i).$
\end{proof}

\begin{ex}
	If $t = \begin{ytableau}
		1 & 2 & 3\\
		4
	\end{ytableau}$ and $\sigma = (24),$ then $\sigma t =\begin{ytableau}
		1 & 4 & 3\\
		2
	\end{ytableau}.$

	We have $C_{\sigma t} = \{1, (12)\} = (24)\{1, (14)\}(24)^{-1}.$
\end{ex}

\begin{defn}%[]
	Fix an $n$ and a partition $\lambda$ of $n.$ The relation $\sim$ is defined on set of $\lambda$-tableaux by putting $t_1 \sim t_2$ if they have the same entries in each row.
\end{defn}
\begin{ex}
	For $n = 6$ and $\lambda = (3, 3),$ one example is
	\begin{equation*} 
		\begin{ytableau}
			1 & 2 & 3\\
			4 & 5 & 6
		\end{ytableau} \sim 
		\begin{ytableau}
			3 & 2 & 1\\
			5 & 6 & 4
		\end{ytableau}.
	\end{equation*}
\end{ex}

\begin{defn}%[Tabloid]
	An equivalence class of $\sim$ is called a \deff{$\lambda$-tabloid} (or a \deff{tabloid of shape $\lambda$}). The tabloid of a tableau $t$ is denoted by $[t].$ The set of all tabloids of shape $\lambda$ is denoted $[T^\lambda].$
\end{defn}
Note that we had not given any notation for the set of all $\lambda$-tableaux. The $[\cdot]$ is to remind ourselves that elements of $[T^\lambda]$ are equivalence classes of tableaux.

% \begin{defn}%[]
% 	Given $\lambda = (\lambda_1, \ldots, \lambda_l),$ the tabloid $T_\lambda$ is defined to be the equivalence class of the $\lambda$-tableau which has $j$ in the $j$-th box. \\
% 	In other words, it is the equivalence class consisting of those tableaux which have $1, \ldots, \lambda_1$ in row $1$ and 
% 	\begin{equation*} 
% 		\lambda_1 + \cdots + \lambda_{i - 1} + 1, \ldots, \lambda_1 + \cdots + \lambda_{i}
% 	\end{equation*}
% 	in row $i$ for $i \ge 2.$
% \end{defn}
% \begin{ex}
% 	$T_{(3, 2)}$ is the equivalence class of $\begin{ytableau}
% 		1 & 2 & 3\\
% 		4 & 5
% 	\end{ytableau}.$ Five other elements in the tabloid are
% 	\begin{align*} 
% 		\begin{ytableau}
% 		1 & 3 & 2\\
% 		4 & 5
% 	\end{ytableau},\;\; \begin{ytableau}
% 		2 & 1 & 3\\
% 		4 & 5
% 	\end{ytableau},\;\; \begin{ytableau}
% 		2 & 3 & 1\\
% 		4 & 5
% 	\end{ytableau},\\
% 	\begin{ytableau}
% 		3 & 1 & 2\\
% 		4 & 5
% 	\end{ytableau},\;\; \begin{ytableau}
% 		3 & 2 & 1\\
% 		4 & 5
% 	\end{ytableau}.
% 	\end{align*}
% 	The tabloid consists of twelve elements in total. The remaining six are obtained from the above six by swapping $(4, 5).$
% \end{ex}

Recall we had defined what a \hyperref[defn:leftcongruence]{$G$-equivalence relation} was.

\begin{prop}%[]
	The equivalence relation $\sim$ defined above is a $G$-equivalence relation.
\end{prop}
\begin{proof} 
	Let $t_1 \sim t_2$ and $\sigma \in S_n.$ We need to show that $\sigma t_1 \sim \sigma t_2.$\\
	Let $i$ and $j$ be two elements in the same row of $\sigma t_1.$ This is possible iff $\sigma^{-1}(i)$ and $\sigma^{-1}(j)$ are in the same row of $t_1.$ In turn, that happens iff $\sigma^{-1}(i)$ and $\sigma^{-1}(j)$ are in the same row of $t_2$ which is iff $i$ and $j$ are in the same row of $\sigma t_2.$
\end{proof}

\begin{cor}
	$S_n$ acts transitively on $[T^\lambda]$ by $\sigma [t] \vcentcolon= [\sigma t].$
\end{cor}
\begin{proof} 
	That the above defines a well-defined action follows from \Cref{prop:leftcongruence}. That it is transitive follows from the fact that $S_n$ acted transitively on the set of $\lambda$-tableaux to begin with.
\end{proof}

% \begin{defn}%[Young subgroup]
% 	Let $\lambda$ be a partition of $n.$ The stabiliser of $T_\lambda$ is denoted $S_\lambda$ and called the \deff{Young subgroup} associate to the partition $\lambda.$
% \end{defn}

% \begin{prop}
% 	For $\lambda = (\lambda_1, \ldots, \lambda_l),$ the isomorphism 
% 	\begin{equation*} 
% 		S_\lambda \cong S_{\{1, \ldots, \lambda_1\}} \times S_{\{\lambda_1 + 1, \ldots, \lambda_1 + \lambda_2\}} \times \cdots \times S_{\{\lambda_1 + \cdots \lambda_{l-1}, \ldots, n\}}
% 	\end{equation*}
% 	holds. In particular, $\md{[T^\lambda]} = n!/(\lambda_1!\cdots\lambda_l!).$
% \end{prop}
% \begin{proof} 
% 	We first show that
% 	\begin{equation*} 
% 		S_\lambda = S_{\{1, \ldots, \lambda_1\}} S_{\{\lambda_1 + 1, \ldots, \lambda_1 + \lambda_2\}} \cdots S_{\{\lambda_1 + \cdots \lambda_{l-1}, \ldots, n\}}.
% 	\end{equation*}
% 	$(\supset)$ is clear. To see the reverse containment, let $\sigma \in S_\lambda.$ Considering the disjoint cycle representation of $\sigma,$ we see that each cycle must belong to one the subgroups listed on the right. (For example, $1$ and $\lambda_1 + 1$ cannot appear in the same cycle since the rows must be preserved.)\\
% 	Noting that any two elements picked from distinct subgroups on the right commute finishes the argument.

% 	The commuting fact along with the fact that intersection of two of the above subgroups is trivial also proves the isomorphism in the theorem. From that, it follows that 
% 	\begin{equation*} 
% 		\md{S_\lambda} = \lambda_1!\cdots\lambda_l!
% 	\end{equation*}
% 	and thus,
% 	\begin{equation*} 
% 		\md{T_\lambda} = \frac{\md{S_n}}{\md{S_\lambda}} = \frac{n!}{\lambda_1!\cdots\lambda_l!},
% 	\end{equation*}
% 	by the orbit-stabiliser theorem.
% \end{proof}

\begin{defn}%[]
	For a partition $\lambda \vdash n,$ set $M^\lambda = \mathbb{C}[T^\lambda]$ and let $\varphi^\lambda : S_n \to \GL(M^\lambda)$ be the associated \hyperref[defn:permrep]{permutation representation}.
\end{defn}
% Note the superscript $^\lambda$ above. To recall, $[T^\lambda]$ was the set of all $\lambda$-tabloids and $S_n$ acted on it. Thus, we get a representation $\varphi^\lambda : S_n \to \GL(\mathbb{C}[T^\lambda])$ as in \Cref{defn:permrep}.

\begin{ex} \label{ex:trivialtabloidrep}
	Suppose $\lambda = (n).$ In this case, there is only one $\lambda$-tabloid and thus, $M^\lambda$ is one-dimensional and the representation is the trivial one.
\end{ex}

\begin{ex} \label{ex:standardtabloidrep}
	Suppose $\lambda = (n-1, 1).$ In this case, two tableaux are equivalent iff they have the same entry in the second row. Thus, there are $n$ $\lambda$-tabloids, which we denote by $[1], \ldots, [n];$ here $[k]$ denotes the equivalence class consisting of the tableaux with $k$ in the lower row.

	Thus, $[T^\lambda] = \{[1], \ldots, [n]\}$ forms a basis for $M^\lambda.$ Moreover, the action (representation) is the natural one with
	\begin{equation*} 
		\varphi^\lambda_{\sigma} [k] = [\sigma(k)].
	\end{equation*}
	Thus, $\varphi^\lambda$ is just the \hyperref[ex:standardrepSn]{standard representation of $S_n$}.
\end{ex}

\begin{ex} \label{ex:alttabloidrep1}
	Suppose $\lambda = (1, \ldots, 1).$ Then, each row has exactly one element and hence, each $\lambda$-tabloid consists of only one tableau. Moreover, each $\lambda$-tableau (and hence, $\lambda$-tabloid) can be identified with a representation. (Consider the element in the $i$-th box.) This gives a one-to-one correspondence between $[T^\lambda]$ and $S_n.$ Under this identification, we see that $\varphi^\lambda$ is just the \hyperref[defn:regularrepresentation]{regular representation}.

	Recalling \Cref{rem:regrepcontainsallreps}, we note that all the irreducible representations of $S_n$ are contained in $\varphi^\lambda.$
\end{ex}

Recall that we had seen that permutation representations are not irreducible unless the set being acted upon is a singleton (\Cref{cor:nontrivialpermred}) and hence, unless $\lambda = (n),$ $\varphi^\lambda$ is not irreducible. However, it contains a special irreducible constituent that we now wish to isolate.

\begin{defn}%[Polytabloid]
	Let $\lambda, \mu \vdash n.$ Let $t$ be a $\lambda$-tableau and define the linear operator $A^\mu_t : M^\mu \to M^\mu$ by
	\begin{equation*} 
		A^\mu_t = \sum_{\pi \in C_t} \sign(\pi)\varphi^\mu_\pi.
	\end{equation*}
	If $\mu = \lambda,$ then we write $A^\lambda_t = A_t$ and the element
	\begin{equation*} 
		e_t = A_t[t] = \sum_{\pi \in C_t} \sign(\pi)\varphi^\lambda_\pi[t] = \sum_{\pi \in C_t} \sign(\pi)\pi[t] \in M^\lambda
	\end{equation*}
	is called the \deff{polytabloid} associated to $t.$
\end{defn}	
\begin{rem}
	It is easy to see that any polytabloid is non-zero. To see this, it suffices to show that if $1 \neq \pi \in C_t,$ then $\pi[t] = [\pi t] \neq [t].$ From this, it would follow that the coefficient of $[t]$ in $e_t$ is $1.$

	To see why the claim is true, note that if $[\pi t] = [t],$ then $\pi$ stabilises the rows of $t.$ On the other hand, we assumed $\pi \in C_t.$ Thus, $\pi$ also stabilises the columns of $t.$ From this, it follows that every element is fixed. (Indeed, it can neither change its row nor its column.) Thus, $\pi$ is the identity permutation.
\end{rem}

There is a lot to absorb in the above definition. It is best to do it with an example.
\begin{ex}
	Let $n = 5.$ Consider $\lambda = (3, 2)$ and $\mu = (4, 1).$ Let 
	\begin{equation*} 
		t = \begin{ytableau}
			1 & 2 & 3\\
			4 & 5
		\end{ytableau}.
	\end{equation*}
	Recall that $C_t$ is the column stabiliser of $t.$ In this case, we have $C_t = S_{\{1, 4\}}S_{\{2, 5\}}S_{\{3\}}.$ More explicitly, we have
	\begin{equation*} 
		C_t = \{1, (14), (25), (14)(25)\},
	\end{equation*}
	where $1$ denotes the identity element. 

	As noted earlier in \Cref{ex:standardtabloidrep}, $M^\mu$ is a five dimensional vector space with basis $\{[1], \ldots, [5]\}.$ (The notation is the same as in the example.) As an example, we may note that
	\begin{align*} 
		A^\mu_t([1]) &= \sum_{\tau \in C_t}\sign(\pi)\varphi^\mu_\pi([1])\\
		&= \sum_{\tau \in C_t}\sign(\pi)[\pi(1)]\\
		&= 1\cdot[1] + (-1)\cdot[4] + (-1)\cdot[1] + 1\cdot[4]\\
		&= 0.
	\end{align*}
	Similarly, we have
	\begin{align*} 
		A^\mu_t([2]) &= 1\cdot[2] + (-1)\cdot[2] + (-1)\cdot[5] + 1\cdot[5] = 0,\\
		A^\mu_t([3]) &= 1\cdot[3] + (-1)\cdot[3] + (-1)\cdot[3] + 1\cdot[3] = 0.
	\end{align*}
	By symmetry, it follows that $A^\mu_t([4]) = A^\mu_t([5]) = 0.$ Thus, we see that $A^\mu_t$ is the zero operator in this case.

	Let us compute the polytabloid now. For ease of notation, we define the $\lambda$-tableaux $t_1, t_2, t_3$ as
	\begin{align*} 
	 	t_1 &= \begin{ytableau}
			4 & 2 & 3\\
			1 & 5
		\end{ytableau},\\
		t_2 &= \begin{ytableau}
			1 & 5 & 3\\
			4 & 2
		\end{ytableau},\\
		t_3 &= \begin{ytableau}
			4 & 5 & 3\\
			1 & 2
		\end{ytableau}.
	\end{align*} 
	Note that the above are simply the tableaux obtained by acting the elements of $C_t$ on $t.$ Moreover, note that each tableaux is in a different equivalence class.

	Now, we have
	\begin{align*} 
		e_t = A_t[t] &= \sum_{\pi \in C_t} \sign(\pi)\pi[t]\\
		&= [t] - [t_1] - [t_2] + [t_3].
	\end{align*}
\end{ex}

\begin{ex}
	Let $t$ be as in the previous example. Consider $\sigma = (123).$ Then, we have
	\begin{align*} 
		\varphi^\lambda_\sigma e_t &= [\sigma t] - [\sigma t_1] - [\sigma t_2] + [\sigma t_3]\\
		&= \left[\begin{ytableau}
			2 & 3 & 1\\
			4 & 5
		\end{ytableau}\right] - \left[\begin{ytableau}
			4 & 3 & 1\\
			2 & 5
		\end{ytableau}\right] - \left[\begin{ytableau}
			2 & 5 & 1\\
			4 & 3
		\end{ytableau}\right] + \left[\begin{ytableau}
			4 & 5 & 1\\
			2 & 3
		\end{ytableau}\right].
	\end{align*}

	On the other hand, let us compute $e_{\sigma t}.$ First, we note that
	\begin{equation*} 
		\sigma t = \begin{ytableau}
			2 & 3 & 1\\
			4 & 5
		\end{ytableau}.
	\end{equation*}
	Secondly, we note that
	\begin{equation*} 
		C_{\sigma t} = \{1, (24), (35), (24)(35)\}.
	\end{equation*}
	Thus, we have
	\begin{equation*} 
		e_{\sigma t} = \left[\begin{ytableau}
			2 & 3 & 1\\
			4 & 5
		\end{ytableau}\right] - \left[\begin{ytableau}
			4 & 3 & 1\\
			2 & 5
		\end{ytableau}\right] - \left[\begin{ytableau}
			2 & 5 & 1\\
			4 & 3
		\end{ytableau}\right] + \left[\begin{ytableau}
			4 & 5 & 1\\
			2 & 3
		\end{ytableau}\right].
	\end{equation*}
	Thus, we see that $\varphi^\lambda_\sigma e_t = e_{\sigma t}.$ We shall now see that this is the case in general.
\end{ex}

\begin{lem}
	If $\sigma \in S_n$ and $t$ is a $\lambda$-tableau, then $\varphi^\lambda_\sigma \circ A_t = A_{\sigma t} \circ \varphi^\lambda_\sigma.$
\end{lem}
\begin{proof} 
	From \Cref{lem:colstabrelation}, we know that $C_{\sigma t} = \sigma C_{t}\sigma^{-1}.$ Now, note that
	\begin{align*} 
		\varphi^\lambda_\sigma \circ A_t &= \sum_{\pi \in C_t} \sign(\pi)\varphi^\lambda_\sigma \circ \varphi^\lambda_\pi\\
		&= \sum_{\pi \in C_t} \sign(\pi)\varphi^\lambda_{\sigma\pi}\\
		&= \sum_{\tau \in \sigma C_t \sigma^{-1}} \sign(\sigma^{-1}\tau\sigma)\varphi^\lambda_{\tau\sigma}\\
		&= \sum_{\tau \in C_{\sigma t}} \sign(\tau)\varphi^\lambda_{\tau\sigma}\\
		&= \sum_{\tau \in C_{\sigma t}} \sign(\tau)\varphi^\lambda_{\tau}\circ\varphi^\lambda_{\sigma}\\
		&= A_{\sigma t} \circ \varphi^\lambda_\sigma. \qedhere
	\end{align*}
\end{proof}

\begin{cor}
	If $\sigma \in S_n$ and $t$ is a $\lambda$-tableau, then $\varphi^\lambda_\sigma e_t = e_{\sigma t}.$
\end{cor}
\begin{proof} 
	\begin{equation*} 
		\varphi^\lambda_\sigma e_t = \varphi^\lambda_\sigma(A_t[t]) = A_{\sigma t}(\varphi^\lambda_\sigma)[t] = A_{\sigma t}[\sigma t] = e_{\sigma t}. \qedhere
	\end{equation*}
\end{proof}

\begin{cor}
	The subspace $S^\lambda = \mathbb{C}\{e_t \mid t \text{ is a } \lambda\text{-tableau}\} \le M^\lambda$ is $S_n$-invariant with respect to $\varphi^\lambda.$
\end{cor}
\begin{proof} 
	It suffices to show that $\varphi^\lambda_\sigma(e_t) \in S^\lambda$ for all $\lambda$-tableaux $t$ and all $\sigma \in S_n.$ By the previous, we have that $\varphi^\lambda_\sigma e_t = e_{\sigma t}.$ Since $\sigma t$ is again a $\lambda$-tableau, we are done.
\end{proof}

\begin{ex} \label{ex:alttabloidrep2}
	Let us revisit \Cref{ex:alttabloidrep1} where we had $\lambda = (1, \ldots, 1).$ Fix a $\lambda$-tableau $t.$ Note that in this case, $C_t = S_n.$ Thus,
	\begin{equation*} 
		e_t = \sum_{\pi \in S_n} \sign(\pi)\pi[t].
	\end{equation*}
	On one hand, we know that $\varphi^\lambda_\sigma e_t = e_{\sigma t}.$ Let us now compute it more explicitly. Applying $\varphi^\lambda_\sigma$ on both sides, we note
	\begin{align*} 
		\varphi^\lambda_\sigma e_t &= \sum_{\pi \in S_n} \sign(\pi)\varphi^\lambda_\sigma[\pi t]\\
		&= \sum_{\pi \in S_n} \sign(\pi)[\sigma\pi t]\\
		&= \sum_{\tau \in S_n} \sign(\sigma^{-1}\tau)[\tau t]\\
		&= \sign(\sigma^{-1})\sum_{\tau \in S_n} \sign(\tau)[\tau t]\\
		&= \sign(\sigma^{-1})e_t.
	\end{align*}
	Since $\sign(\sigma) = \sign(\sigma^{-1}),$ we see that
	\begin{equation*} 
		\varphi^\lambda_\sigma e_t = e_{\sigma t} = \sign(\sigma) e_t.
	\end{equation*}
	In particular, note that each $e_{\sigma t}$ is simply a scalar multiple of $e_t$ and hence, $\{e_{\sigma t}\}_{\sigma \in S_n}$ is a linearly dependent set if $n > 1.$
\end{ex}

\begin{defn}%[Sprecht representation]
	\label{defn:sprechtrepresentation}
	Let $\lambda$ be a partition of $n.$ Define $S^\lambda$ to be the subspace of $M^\lambda$ spanned by the polytabloids
	\begin{equation*} 
		\{e_t \mid t \text{ is a } \lambda\text{-tableau}\}.
	\end{equation*}
	$S^\lambda$ is $S_n$-invariant. Let $\psi^\lambda : S_n \to \GL(S^\lambda)$ be the corresponding subrepresentation. This is called the \deff{Sprecht representation} associated to $\lambda.$
\end{defn}

\begin{rem}
	Note that $\{e_t \mid t \text{ is a } \lambda\text{-tableau}\}$ is simply a \emph{spanning} set for $S^\lambda.$ As \Cref{ex:alttabloidrep2} shows, this set need not be linearly independent.
\end{rem}

\begin{ex}
	Let $\lambda = (1, \ldots, 1).$ 
	By our calculations in \Cref{ex:alttabloidrep2}, we saw that $e_{\sigma t} = \sign(\sigma)e_t.$ Thus, $S^\lambda$ is a one-dimensional subspace of $M^\lambda.$ (Recall that no polytabloid $e_t$ is zero.)

	Now, fix a $\lambda$-tableau $t.$ We have $S^\lambda = \mathbb{C}e_t.$

	We note that
	\begin{equation*} 
		\psi^\lambda_\sigma(e_t) = \varphi^\lambda_\sigma(e_t) = \sign(\sigma)e_t
	\end{equation*}
	and thus, $\psi^\lambda$ is equivalent to the $\sign$ representation of $S_n.$
\end{ex}

We now wish to show that all Sprecht representations are irreducible. We do this via a series of lemmata.

\begin{lem} \label{lem:Amutsnonzero}
	Let $\lambda, \mu \vdash n$ and suppose that $[T^\lambda]$ is a $\lambda$-tableau and $s^\mu$ a $\mu$-tableau such that $A^\mu_{[T^\lambda]}[s^\mu] \neq 0.$ Then, elements in the same row of $s^\mu$ appear in different columns of $[T^\lambda].$\\
	In particular, $\lambda \unrhd \mu.$
\end{lem}
\begin{proof} 
	Note that the last statement follows from \nameref{lem:domlemma}. Thus, we simply need to show that no two elements in the same row of $s^\mu$ are in the same column of $[T^\lambda].$

	To this end, suppose that $i, j$ are distinct elements in the same row of $s^\mu$ and same column of $[T^\lambda].$ Consider the transposition $\rho = (i\;j).$ \\
	By the definition of equivalence relation on tableaux, we see that $[s^\mu] = [\rho s^\mu]$ and thus, 
	\begin{equation} \tag{$*$} \label{eq:009}
		\varphi^\mu_1[s^\mu] - \varphi^\mu_{\rho}[s^\mu] = 0.
	\end{equation}
	(As usual, $1$ denotes the identity permutation.)

	On the other hand, by definition of $C_{[T^\lambda]},$ we see that $H = \{1, \rho\}$ is a subgroup of $C_{[T^\lambda]}.$ Let $S$ be a transversal of left coset representatives of $H$ in $C_{[T^\lambda]}.$ We then see
	\[\begin{WithArrows}[displaystyle]
		A^\mu_{[T^\lambda]}[s^\mu] &= \sum_{\pi \in C_{[T^\lambda]}} \sign(\pi)\varphi^\mu_\pi[s^\mu]\\
		&= \sum_{\sigma \in S}\left(\sign(\sigma)\varphi^\mu_\sigma[s^\mu] + \sign(\sigma\rho)\varphi^\mu_{\sigma\rho}[s^\mu]\right)\\
		&= \sum_{\sigma \in S}\sign(\sigma)\varphi^\mu_\sigma\left(\varphi^\mu_1[s^\mu] - \varphi^\mu_{\rho}[s^\mu]\right) \Arrow{\Cref{eq:009}}\\
		&= 0,
	\end{WithArrows}\]
	a contradiction since we assumed that $A^\mu_{[T^\lambda]}[s^\mu] \neq 0.$
\end{proof}

\begin{lem}
	Let $\lambda \vdash n$ and $t, s$ be $\lambda$-tableaux such that $A_t[s] \neq 0.$ Then, $A_t[s] \in \{\pm e_t\}.$
\end{lem}
\begin{proof} 
	Let $u = u^\lambda$ be as given by \Cref{prop:domprop}. (By the previous lemma, it follows that the hypothesis is indeed followed.)\\
	Let $\sigma$ be the unique permutation such that $u = \sigma t.$ Note that since $u$ and $t$ have the same entries in each column, it follows that $\sigma \in C_{t}.$\\
	Moreover, note that $s$ and $u$ have the same element in each row. (As per the proposition, the elements in the first $i$ rows of $s$ were in the first $i$ rows of $u.$ Since both the tableaux are of shape $\lambda,$ it follows that the row-wise entries are all same.)\\
	In other words, $[u] = [s].$

	Thus, we get
	\[\begin{WithArrows}[displaystyle]
		A_{t}[s] &= A_{t}[u]\\
		&= \sum_{\pi \in C_t} \sign(\pi)\varphi^\lambda_\pi[u]\\
		&= \sum_{\pi \in C_t} \sign(\pi)[\pi u] \Arrow{$\pi \mapsto \tau\sigma^{-1}$ (note $\sigma \in C_t$)}\\
		&= \sum_{\tau \in C_t} \sign(\tau\sigma^{-1})[\tau\sigma^{-1}u]\\
		&= \sign(\sigma^{-1})\sum_{\tau \in C_t}\sign(\tau)[\tau \lambda]\\
		&= \sign(\sigma^{-1})e_t
	\end{WithArrows}\]
	and hence, $A_t[s] \in \{\pm e_t\}.$
\end{proof}

\begin{lem} \label{lem:imageofAt}
	Let $t$ be a $\lambda$-tableau. Then, the image of the operator $A_t$ is $\mathbb{C}e_t.$
\end{lem}
\begin{proof} 
	Note that $A_t[t] = e_t,$ by definition and thus, only $\im A_t \subset \mathbb{C}e_t$ needs to be shown.

	It suffices to show that $A_t[s] \in \mathbb{C}e_t$ for every $\lambda$-tableau $s,$ since the set of all (distinct) $[s]$ form a basis for $M^\lambda.$

	However, this is simple for the previous lemma tells us that $A_t[s] \in \{0, \pm e_t\} \subset \mathbb{C}e_t.$
\end{proof}

Recall that $M = \mathbb{C}[T^\lambda]$ comes with an inner product (\Cref{defn:linearisation}) such that $[T^\lambda]$ is an orthonormal basis. Moreover, the permutation representation $\varphi^\lambda$ is unitary with respect to this. (\Cref{prop:permrepisunitary}.)

\begin{lem} 
	If $t$ is a $\lambda$-tableau, then $A_t = A_t^*.$ That is, $A_t$ is self-adjoint.
\end{lem}
\begin{proof} 
	We have
	\[\begin{WithArrows}[displaystyle]
		A_t^* &= \sum_{\pi \in C_t}\sign(\pi)(\varphi^\lambda_\pi)^* \Arrow{$\varphi^\lambda$ is unitary}\\
		&= \sum_{\pi \in C_t}\sign(\pi)(\varphi^\lambda_\pi)^{-1}\\
		&= \sum_{\pi \in C_t}\sign(\pi)\varphi^\lambda_{\pi^{-1}}\\
		&= \sum_{\pi \in C_t}\sign(\pi^{-1})\varphi^\lambda_{\pi^{-1}} \Arrow{$\pi \mapsto \pi^{-1}$ is a bijection}\\
		&= A_t,
	\end{WithArrows}\]
	as desired.
\end{proof}

\begin{thm}[Subrepresentation theorem] \label{thm:subrepresentation}
	Let $\lambda$ be a partition of $n$ and suppose that $V$ is an $S_n$-invariant subspace of $M^\lambda$ with respect to $\varphi^\lambda.$ Then, either $S^\lambda \subset V$ or $V \subset (S^\lambda)^\perp.$
\end{thm}

Said differently, the above says that either $S^\lambda \subset V$ or $S^\lambda \subset V^\perp.$

\begin{proof} The proof splits into two cases.

	\textbf{Claim 1.} If there exists a $\lambda$-tableau $t$ and $v \in V$ such that $A_tv \neq 0,$ then $S^\lambda \subset V.$

	\begin{proof} 
		By \Cref{lem:imageofAt}, it follows that $A_tv \in \mathbb{C}e_t.$ Since $V$ is $S_n$-invariant, we also have $A_tv \in V.$\footnote{$A_t$ is a linear combination of $\varphi^\lambda_\sigma$s and $V$ is invariant for each.} Since $A_tv \neq 0,$ we see that $\mathbb{C}e_t \cap V$ is not trivial.\\
		Since $\mathbb{C}e_t$ is one-dimensional, it follows that $\mathbb{C}e_t \subset V$ or $e_t \in V.$ From this, it follows that
		\begin{equation*} 
			e_{\sigma t} = \varphi^\lambda_\sigma e_t \in V
		\end{equation*}
		for all $\sigma \in S_n.$ Since $S_n$ acts transitively on $[T^\lambda],$ it follows that $S = \mathbb{C}\{e_s\} \subset V.$
	\end{proof}

	\textbf{Claim 2.} If $A_tv = 0$ for all $\lambda$-tableaux $t$ and all $v \in V,$ then $V \subset (S^\lambda)^\perp.$

	\begin{proof}
		Let $v \in V$ and $[t] \in [T^\lambda]$ be arbitrary. Then, 
		\begin{equation*} 
			\langle v, e_t\rangle = \langle v, A_t[t]\rangle = \langle A_tv, [t]\rangle = \langle 0, [t]\rangle = 0
		\end{equation*}
		and hence, $V \subset (S^\lambda)^\perp.$ (The second equality above follows from $A_t$ being self-adjoint.)
	\end{proof}

	From the above claims, the theorem follows at once.
\end{proof}

\begin{cor} \label{cor:psilambdaisirred}
	Let $\lambda \vdash n.$ Then, $\psi^\lambda : S_n \to \GL(S^\lambda)$ is irreducible.
\end{cor}
\begin{proof} 
	Let $V$ be a proper $S_n$-invariant subspace of $S^\lambda$ (with respect to $\psi^\lambda$). We show that $V = \{0\}.$\\
	Note that $\psi^\lambda$ is the restriction of $\varphi^\lambda.$ Thus, we get that $V$ is an $S_n$-invariant subspace of $M^\lambda$ with respect to $\varphi^\lambda.$ Thus, by \Cref{thm:subrepresentation}, it follows that $V \subset (S^\lambda)^\perp$ since $V$ is assumed to be a proper subspace of $S^\lambda.$

	However, this means that $V \subset S^\lambda \cap (S^\lambda)^\perp = \{0\},$ as desired.
\end{proof}

\begin{aside}
	Let us recap what we have done so far.

	For each partition $\lambda \vdash n,$ we defined a vector space $M^\lambda = \mathbb{C}[T^\lambda]$ and a representation $\varphi^\lambda : S_n \to \GL(M^\lambda).$ We noted that $\varphi^\lambda$ is not irreducible but then we defined $S^\lambda \le M^\lambda$ which turned out to be $S_n$-invariant. Moreover, we have shown that the subrepresentation $\psi^\lambda : S_n \to \GL(S^\lambda)$ is irreducible.

	Thus, we have a function
	\begin{align*} 
		\{\text{partitions of } n\} &\to \{\text{irreducible representations of } n \text{ modulo equivalence}\}\\
		\lambda &\mapsto \psi^\lambda.
	\end{align*}
	By our earlier remark (\Cref{rem:irredrepsconjclassSn}), we know that both the sets above have the same (finite) cardinality. Thus, if we can show that the above function is injective, then we would have shown that the above is a bijection.

	Our goal now is to show precisely that. In other words, we show that if $\lambda \neq \mu$ are partitions of $n,$ then $\psi^\lambda \not\sim \psi^\mu.$
\end{aside}

% We are now close to proving what we had wished to do, in \Cref{rem:irredrepsconjclassSn}. Indeed, for each partition $\lambda \vdash n,$ we have constructed an irreducible representation $\psi^\lambda$ of $S_n.$ All that remains is to show that these are pairwise inequivalent. We would then have found all the representations of $S_n.$

\begin{lem} \label{lem:morphismkerdom}
	Suppose that $\lambda, \mu \vdash n$ and let $T \in \Hom_{S_n}(\varphi^\lambda, \varphi^\mu).$ If $S^\lambda \not\subset \ker T,$ then $\lambda \unrhd \mu.$ \\
	Moreover, if $\lambda = \mu,$ then $T|_{S^\lambda}$ restricts to a linear operator on $S^\lambda$ and is a scalar multiple of the identity map $\id_{S^\lambda}.$
\end{lem}
Note that by ``restricts to a linear operator,'' we mean that $\im(T|_{S^\lambda}) \subset S^\lambda.$
\begin{proof} 
	Note that by \Cref{prop:Ginvmorphism}, $\ker T$ is $S_n$-invariant with respect to $\varphi^\lambda.$ By \nameref{thm:subrepresentation}, it follows that $\ker T \subset (S^\lambda)^\perp.$ Thus, $\ker(T) \cap S^\lambda = \{0\}.$ In particular, $Te_t \neq 0$ for any $t.$

	Now, we note that
	\begin{align*} 
		A^\mu_tT[t] &= \sum_{\pi \in C_t}\sign(\pi)\varphi^\mu_\pi T[t]\\
		&= \sum_{\pi \in C_t}\sign(\pi)T\varphi^\lambda_\pi[t]\\
		&= T\sum_{\pi \in C_t}\sign(\pi)\varphi^\lambda_\pi[t]\\
		&= Te_t \neq 0
	\end{align*}
	and thus, $A^\mu_tT[t] \neq 0.$ However, $T[t]$ is a linear combination of $\mu$-tabloids $[s].$ Thus, there is some $[s]$ such that $A^\mu_t[s] \neq 0.$ By \Cref{lem:Amutsnonzero}, it follows that $\lambda \unrhd \mu.$ This proves the first part.

	Now, suppose that $\mu = \lambda.$ By \Cref{lem:imageofAt}, it follows that
	\begin{equation*} 
		Te_t = A_tT[t] \in \mathbb{C}e_t \subset S^\lambda.
	\end{equation*}
	Since the polytabloids $e_t$ span $S^\lambda,$ the above tells us that $S^\lambda$ is $T$-invariant. Thus, $T$ restricts to a linear operator on $S^\lambda$ and hence, we see that
	\begin{equation*} 
		T|_{S^\lambda} \in \Hom_{S_n}(\psi^\lambda, \psi^\lambda)
	\end{equation*}
	and hence, $T|_{S^\lambda}$ is a multiple of $\id_{S^\lambda},$ by \nameref{lem:schur}. (We knew that $\psi^\lambda$ is irreducible, by \Cref{cor:psilambdaisirred}.)
\end{proof}

\begin{lem} 
	If $\Hom_{S_n}(\psi^\lambda, \varphi^\mu) \neq 0,$ then $\lambda \unrhd \mu.$ Moreover, if $\lambda = \mu,$ then $\dim \Hom_{S_n}(\psi^\lambda, \varphi^\mu) = 1.$
\end{lem}
(Note in the above that it is $\psi^\lambda,$ not $\varphi^\lambda.$)

\begin{proof} 
	Let $0 \neq T \in \Hom_{S_n}(\psi^\lambda, \varphi^\mu).$

	Thus, $T : S^\lambda \to M^\lambda$ is a linear map such that
	\begin{center}
		\begin{tikzcd}
			{S^\lambda} \arrow[rr, "\psi^\lambda_\sigma"]\arrow[dd, "T"'] & & {S^\lambda}\arrow[dd, "T"]\\
			& & \\
			{M^\lambda} \arrow[rr, "\varphi^\mu_\sigma"'] & & {M^\lambda}
		\end{tikzcd}
	\end{center}
	commutes for all $\sigma \in S_n.$

	We wish to extend $T$ to a map $\widetilde{T} : M^\lambda \to M^\lambda.$ Note that $M^\lambda = S^\lambda \oplus (S^\lambda)^\perp.$ \\
	Let $\pi_1 : M^\lambda \to S^\lambda \hookrightarrow M^\lambda$ and $\pi_2 : M^\lambda \to (S^\lambda)^\perp \hookrightarrow M^\lambda$ denote the projection maps.

	We define $\widetilde{T} : M^\lambda \to M^\lambda$ by 
	\begin{equation*} 
		\widetilde{T} = T \circ \pi_1.
	\end{equation*}

	We now show that $\widetilde{T} \in \Hom_{S_n}(\varphi^\lambda, \varphi^\mu).$ (Note that now we have $\varphi^\lambda.$) Note that $(S^\lambda)^\perp$ is again $S_n$-invariant with respect to $\varphi^\lambda$ since $\varphi^\lambda$ is unitary. Thus, 
	\begin{equation} \tag{$*$} \label{eq:010}
		\pi_1 \circ \varphi^\lambda_\sigma \circ \pi_2 = 0
	\end{equation} 
	for all $\sigma \in S_n.$

	Hence, for an arbitrary $\sigma \in S_n,$ we obtain
	\[\begin{WithArrows}[displaystyle]
		\widetilde{T} \circ \varphi^\lambda &= T \circ \pi_1 \circ \varphi^\lambda_\sigma \Arrow{$\id_{M^\lambda} = \pi_1 + \pi_2$} \\
		&= T \circ \pi_1 \circ \varphi^\lambda_\sigma \circ [\pi_1 + \pi_2] \Arrow{\Cref{eq:010}}\\
		&= T \circ \pi_1 \circ \varphi^\lambda_\sigma \circ \pi_1\\
		&= T \circ \psi^\lambda_\sigma \circ \pi_1\Arrow{$T \in \Hom_{S_n}(\psi^\lambda, \varphi^\mu)$}\\
		&= \varphi^\mu_\sigma \circ T \circ \pi_1 \\
		&= \varphi^\mu_\sigma \circ \widetilde{T},
	\end{WithArrows}\]
	as desired.

	% \[\begin{WithArrows}[displaystyle]
	% 	\widetilde{T}(\varphi^\lambda_\sigma(v + w)) &= \widetilde{T}(\varphi^\lambda_\sigma v + \varphi^\lambda_\sigma w) \Arrow{$\varphi^\lambda_\sigma w \in (S^\lambda)^\perp$}\\
	% 	&= T(\varphi^\lambda_\sigma v)\\
	% 	&= T(\psi^\lambda_\sigma v) \Arrow{$T \in \Hom_{S_n}(\psi^\lambda, \varphi^\mu_\sigma)$}\\
	% 	&= \varphi^\mu_\sigma(Tv)\\
	% 	&= \varphi^\mu_\sigma(\widetilde{T}(v + w)).
	% \end{WithArrows}\]

	Now, since $T \neq 0,$ it follows that $S^\lambda \not\subset \ker \widetilde{T}$ and thus, $\lambda \unrhd \mu,$ by \Cref{lem:morphismkerdom}.

	Moreover, if $\mu = \lambda,$ then $T = \widetilde{T}|_{S^\lambda}$ be a must scalar multiple of inclusion map, by \Cref{lem:morphismkerdom} again. Since this is a non-zero map, we get that $\dim \Hom_{S_n}(\psi^\lambda, \varphi^\mu) = 1.$
\end{proof}

\begin{cor}
	Suppose $\mu \vdash n.$ Then $\psi^\mu$ appears with multiplicity one as an irreducible constituent of $\varphi^\mu.$ Any other irreducible constituent $\psi^\lambda$ of $\varphi^\mu$ satisfies $\lambda \unrhd \mu.$
\end{cor}
\begin{proof} 
	Both parts follow from \Cref{cor:extractmultiplicitywithhom} along with the previous lemma.
\end{proof}

With that, we now conclude the final result.

\begin{thm}
	The Sprecht representations $\psi^\lambda$ with $\lambda \vdash n$ form a complete set of inequivalent irreducible representations of $S_n.$
\end{thm}
\begin{proof} 
	It just remains to show that $\psi^\lambda \sim \psi^\mu$ implies $\lambda = \mu.$ To this end, assume that $\psi^\lambda \sim \psi^\mu.$ Then, there is a non-zero (iso)morphism $T \in \Hom_{S_n}(\psi^\lambda, \psi^\mu).$ However, we have the inclusion
	\begin{equation*} 
		\Hom_{S_n}(\psi^\lambda, \psi^\mu) \subset \Hom_{S_n}(\psi^\lambda, \varphi^\mu)
	\end{equation*}
	and thus, $\Hom_{S_n}(\psi^\lambda, \varphi^\mu) \neq 0.$ In turn, the previous lemma implies that $\lambda \unrhd \mu.$ 

	By symmetry, we also see that $\mu \unrhd \lambda.$ It follows that $\lambda = \mu,$ as desired.
\end{proof}

% \begin{ex}[Character table of $S_5$]
% 	Let us now compute the character table of $S_5.$ Note that we have the following partitions of $5:$
% 	\begin{enumerate}
% 		\item $\lambda_1 =(5)$
% 		\item $\lambda_2 =(4, 1)$
% 		\item $\lambda_3 =(3, 2)$
% 		\item $\lambda_4 =(3, 1, 1)$
% 		\item $\lambda_5 =(2, 2, 1)$
% 		\item $\lambda_6 =(2, 1, 1, 1)$
% 		\item $\lambda_7 =(1, 1, 1, 1, 1)$
% 	\end{enumerate}
% 	(How do we know that we have gotten all? Simply compute the number of permutations having that cycle type and see that the sum is $5! = 120.$)
% 	% amnote: This always works!

% 	Thus, we shall get seven characters. 

% 	$\bullet\; \lambda_1$ This is is trivial character, as observed in \Cref{ex:trivialtabloidrep}.

% 	$\bullet\; \lambda_2$ 
% \end{ex}

\newpage
\textbf{Book-keeping} 

\begin{tabular}{|l|l|}
	\hline
	$\lambda$ & A partition of $n$\\
	$t,$ a $\lambda$-tableau & Fill the Young diagram of $\lambda$ from $1$ to $n$\\
	$\sigma t$ & The $\lambda$-tableau obtained by acting $\sigma$ on each box of $t.$ \\
	$C_t$ & Subgroup of $S_n$ stabilising columns of $t.$\\
	$[t],$ a $\lambda$-tabloid & Equivalence class of $t$ of tableaux having same row elements.\\
	$[T^\lambda]$ & Set of all $\lambda$-tabloids.\\
	$M^\lambda = \mathbb{C}[T^\lambda]$ & $\mathbb{C}$-vector space with $\lambda$-tabloids as basis.\\
	$\varphi^\lambda$ & Natural representation $S_n \to \GL(M^\lambda).$\\
	$\varphi^\lambda_\sigma = \varphi^\lambda(\sigma)$ & Linear transform $M^\lambda \to M^\lambda$ given as $\varphi^\lambda_\sigma[t] = [\sigma t]$ on basis.\\
	$A^\mu_t$ & $t$ is a $\lambda$-tableau and there's another partition $\mu.$ $A_t : M^\mu \to M^\mu$ is linear.\\
	$A_t$ & $A^\lambda_t,$ i.e., take $\mu = \lambda$ above.\\
	$e_t$ & $e_t = A_t[t] \in M^\lambda$ is a $\mathbb{C}$-linear combination of tabloids, called a polytabloid.\\
	$S^\lambda$ & Subspace of $M^\lambda$ spanned by $\{e_t \mid t \text{ a } \lambda\text{-tableau}\}$. It is $S_n$-invariant.\\
	$\psi^\lambda$ & The subrepresentation of $\varphi^\lambda$ corresponding to $S^\lambda.$\\
	\hline
\end{tabular}
\label{bookkeeping}
\section{Fourier Analysis on Finite Groups} \label{sec:07}
\subsection{Periodic Functions on Cyclic Groups}

\begin{defn}
    Let $n \in \mathbb{Z} \setminus \{0\}.$ A function $f : \mathbb{Z} \to \mathbb{C}$ is said to be \deff{periodic} with \deff{period} $n$ if $f(x + n) = f(x)$ for all $x \in \mathbb{Z}.$
\end{defn}

Note that there is no unique period associated to a periodic function. Indeed, if $n$ is a period, then so is any multiple of $n.$

\begin{rem}
    Fix an integer $n > 0.$ The set of functions periodic with period $n$ is in bijection with the set of all complex valued functions on $\mathbb{Z}/n\mathbb{Z}.$
    
    Indeed, given a period function $f : \mathbb{Z} \to \mathbb{C}$ with period $n,$ we get a function $\tilde{f} : \mathbb{Z}/n\mathbb{Z} \to \mathbb{C}$ defined as $\tilde{f}([m]) = f(m).$ (This is well-defined by assumption of periodicity.)
    
    Conversely, given a function $g : \mathbb{Z}/n\mathbb{Z} \to \mathbb{C},$ we get a function $\dot{g} : \mathbb{Z} \to \mathbb{C}$ defined as $\dot{g}(m) = g([m]).$ Clearly, $g$ is periodic with period $n.$
    
    Moreover, the correspondences $f \mapsto \tilde{f}$ and $g \mapsto \dot{g}$ are inverses. Thus, the set of functions periodic with period $n$ is in bijection with $L(\mathbb{Z}/n\mathbb{Z}).$ (Recall that $L(G)$ was the \hyperref[defn:groupalg]{group algebra} of the group $G.$)
\end{rem}

\begin{rem}
    Recall that \Cref{cor:orthnormalbasis} gave us an orthonormal basis of $L(G).$ Since the irreducible representations of $\mathbb{Z}/n\mathbb{Z}$ are all of degree one, we see that the set is simply $\{\chi_k : 0 \le k < n\},$ where $\chi_k([m]) = w_n^{mk}.$
    
    Thus, for $f \in L(\mathbb{Z}/n\mathbb{Z})$ we get 
    \begin{equation*}
        f = \langle f, \chi_0\rangle \chi_0 + \cdots + \langle f, \chi_{n - 1}\rangle \chi_{n - 1}.
    \end{equation*}
\end{rem}

\begin{defn} \label{defn:fourierZnZ}
    Let $f : \mathbb{Z}/n\mathbb{Z} \to \mathbb{C}.$ The \deff{Fourier transform} $\mathcal{F}(f) = \widehat{f} : \mathbb{Z}/n\mathbb{Z} \to \mathbb{C}$ of $f$ is defined as
    \begin{equation*}
        \widehat{f}([m]) = \sum_{k = 0}^{n - 1} f([k])e^{-2\pi\iota mk/n} = \sum_{k = 0}^{n - 1} f([k])\omega_n^{-mk} = n\langle f, \chi_m\rangle.
    \end{equation*}
\end{defn}

The last equality follows by the definition of $\langle \cdot , \cdot\rangle$ and the fact that $n \md{\mathbb{Z}/n\mathbb{Z}}.$ Moreover, note that $\mathcal{F} : L(\mathbb{Z}/n\mathbb{Z}) \to L(\mathbb{Z}/n\mathbb{Z})$ is linear since
$\langle \cdot , \cdot\rangle$ is linear in the first variable.

\begin{prop}
    The Fourier transform is invertible. More precisely, 
    \begin{equation*}
        f = \frac{1}{n}\sum_{k = 0}^{n - 1} \widehat{f}([k])\chi_k.
    \end{equation*}
\end{prop}
\begin{proof}
    We simply compute $\langle f, \chi_k\rangle$ and show that it equals $\widehat{f}([k])/n.$
    
    To this end, note that
    \begin{align*}
        \langle f, \chi_k\rangle &= \frac{1}{n} \sum_{j = 0}^{n - 1}f([j])\overline{\chi_k(j)} \\
        &= \frac{1}{n} \sum_{j = 0}^{n - 1}f([j])e^{-2\pi\iota kj/n} \\
        &= \frac{1}{n} \widehat{f}([k]). \qedhere
    \end{align*}
\end{proof}

\subsection{The Convolution Product}

\begin{defn}
    Let $G$ be a group.
    Given $g \in G,$ we define $\delta_g : G \to \mathbb{C}$ as
    \begin{equation*}
        \delta_g(x) \vcentcolon=
        \begin{cases}
            1 & x = g, \\
            0 & x \neq g.
        \end{cases}
    \end{equation*}
\end{defn}

\begin{defn}
    Let $G$ be a finite group and $a, b \in L(G).$ Then, $a * b \in L(G)$ is defined as
    \begin{equation*}
        (a * b)(x) = \sum_{y \in G} a(xy^{-1})b(y)
    \end{equation*}
    and is called the \deff{convolution} of $a$ with $b.$
\end{defn}

The above is well-defined since $G$ is finite and the sum is taken in $\mathbb{C},$ which is commutative. 

\begin{rem} \label{rem:classfunctionincenter}
    We have
    \begin{equation*}
        (a * b)(x) = \sum_{y \in G} a(xy^{-1})b(y).
    \end{equation*}
    The change of variable $y \mapsto xz^{-1}$ gives
    \begin{equation*}
        (a * b)(x) = \sum_{z \in G} a(xzx^{-1})b(xz^{-1}) = \sum_{z \in G} b(xz^{-1})a(xzx^{-1}).
    \end{equation*}
    Thus, if $a$ is a class function, then we get
    \begin{equation*}
        (a * b)(x) = \sum_{z \in G} b(xz^{-1})a(z) = (b * a)(x)
    \end{equation*}
    for all $x \in G$ and hence, $a * b = b * a.$
    
    In particular, if $G$ is abelian, then $a * b = b * a$ for all $a, b \in L(G).$
    
    However, for a general group, it is not necessary that $*$ is commutative. In fact, after the next proposition, it will be clear that $*$ is commutative iff $G$ is abelian.
\end{rem}

\begin{prop} \label{prop:convdeltaprod}
    Let $G$ be a finite group and $g, h \in G.$ Then, $\delta_g * \delta_h = \delta_{gh}.$
\end{prop}
\begin{proof}
    Let $x \in G.$ Then,
    \begin{equation*}
        (\delta_g * \delta_h)(x) = \sum_{y \in G} \delta_g(xy^{-1})\delta_h(y).
    \end{equation*}
    Thus, $(\delta_g * \delta_h)(x) = 1$ iff $h = y$ and $xy^{-1} = g$ iff $x = gh$ and $0$ otherwise.
    
    In other words, $(\delta_g * \delta_h) = \delta_{gh}.$
\end{proof}

Thus, if $G$ is a non-abelian group, then choose $g, h \in G$ such that $gh \neq hg.$ Then, we get $\delta_g * \delta_h \neq \delta_h * \delta_g.$ 

\begin{prop} \label{prop:convole}
    Let $a \in L(G)$ and $h, g \in G.$ Then, $(a * \delta_h)(g) = a(gh^{-1}).$
\end{prop}

There are many properties that one can verify about the convolution product. We list them below and leave the proof as a computational exercise.

\begin{prop}
    Let $G$ be a finite group. Then,
    \begin{enumerate}
        \item $a * \delta_1 = a = \delta_1 * a$ for all $a \in L(G),$
        \item $a * (b * c) = (a * b) * c$ for all $a, b, c \in L(G),$
        \item $a * (b + c) = (a * b) + (a * c)$ for all $a, b, c \in L(G).$
    \end{enumerate}
    In other words, $(L(G), +, *)$ is a ring with $\delta_1$ as (multiplicative) identity.
\end{prop}

As noted earlier, $L(G)$ is a commutative ring iff $G$ is commutative. The above also justifies why we used the term ``group algebra''. Soon, the usage of $Z(L(G))$ for the set of class functions will become apparent too.

\begin{rem}
    Note that since $\delta_1$ is the identity, \Cref{prop:convdeltaprod} tells us the map $i : G \to L(G)$ defined as $g \mapsto \delta_g$ is a group homomorphism into the group of units $(L(G))^\times.$ (Recall \Cref{lem:determininggrouphomoring}.)
\end{rem}

\begin{defn}
    Let $R$ be a ring. The \deff{center} of $R$ is denoted $Z(R)$ and defined as
    \begin{equation*}
        Z(R) \vcentcolon= \{r \in R : rs = sr \text{ for all } s \in R\}.
    \end{equation*}
    That is, it is the set of all those elements which commute with every element of $R.$
\end{defn}

\begin{rem}
    One can show that the above is actually a \emph{subring} of $R.$ Moreover, it is a commutative ring. However, we do not use this fact.
\end{rem}

Recall that we had already used the notation $Z(L(G))$ to denote the set of class functions of $G.$ We now show that it is indeed the center and thus, the notation is unambiguous.

\begin{prop}
    Let $G$ be a finite group. Then, $a : G \to \mathbb{C}$ is a class function if and only if $a$ is in the center of $L(G).$
\end{prop}
\begin{proof}
    $(\implies)$ Suppose $a : G \to \mathbb{C}$ is a class function. We had observed in \Cref{rem:classfunctionincenter} that $a * b = b * a$ for all $b \in L(G)$ and hence, $a$ is in the center.
    
    $(\impliedby)$ Let $a$ be in the center of $L(G)$ and let $g, h \in G$ be arbitrary. Then,
    \begin{align*}
        a(gh) = \sum_{y \in G} a(gy^{-1})\delta_{h^{-1}}(y) &= (a * \delta_{h^{-1}})(g) \\
        &= (\delta_{h^{-1}} * a)(g) = \sum_{y \in G}\delta_{h^{-1}}(gy^{-1})a(y) = a(hg).
    \end{align*}
    Thus, given any $x, y \in G,$ setting $g = xy$ and $h = x^{-1}$ gives $a(xyx^{-1}) = a(x),$ showing that $a$ is a class function.
\end{proof}

\subsection{Fourier Analysis on Finite Abelian Groups}

Recall that we had defined the \hyperref[defn:dualgroup]{dual of a group}, the group of all homomorphisms $G \to \mathbb{C}^\times$ with point-wise multiplication as operation. In the case that $G$ is finite and abelian, we see that the elements of $\widehat{G}$ are precisely the irreducible characters of $G$ and that $G \cong \widehat{G}.$ (The last was \Cref{cor:GconghatG}.) \\
Moreover, note that $L(G)$ is abelian. 

Recall that we had defined the Fourier transform $\mathcal{F} : L(G) \to L(G)$ earlier for the case that $G = \mathbb{Z}/n\mathbb{Z}.$ Now, we do it in a more general setting, taking inspiration from the earlier. Instead of a map $L(G) \to L(G),$ we define $\mathcal{F} : L(G) \to L(\widehat{G}).$ 

\begin{defn}
    Let $G$ be a finite abelian group and let $f : G \to \mathbb{C}$ be a function. We define the \deff{Fourier transform} $\mathcal{F}(f) = \widehat{f} \in L(\widehat{G})$ by 
    \begin{equation*}
        \widehat{f}(\chi) = \md{G}\langle f, \chi\rangle = \sum_{g \in G}f(g) \overline{\chi(g)}.
    \end{equation*}
    The complex numbers $\md{G}\langle f, \chi\rangle$ are called the \deff{Fourier coefficients} of $f.$
\end{defn}

Note that $\widehat{f}$ is a function $\widehat{f} : \widehat{G} \to \mathbb{C}.$ That is, $\widehat{f}$ takes irreducible characters of $G$ as input.

\begin{rem}
    Recall we had constructed an isomorphism $\mathbb{Z}/n\mathbb{Z} \cong \widehat{\mathbb{Z}/n\mathbb{Z}}$ in \Cref{prop:ZnZconghatZnZ}. The isomorphism was given as $[k] \leftrightarrow \chi_k,$ where $\chi_k \in \widehat{\mathbb{Z}/n\mathbb{Z}},$ as before is
    \begin{equation*} 
        \chi_k([m]) = w_n^{mk}.
    \end{equation*}

    Under this identification, we see that the Fourier transform defined above agrees with the one in \Cref{defn:fourierZnZ}.
\end{rem}

\begin{ex}[Fourier transform of a character] \label{ex:fourtranschar}
    Let $\chi, \theta \in \widehat{G}$ be (irreducible) characters of the abelian group $G.$ Then, $\chi \in L(G)$ and thus, the Fourier transform of $\chi$ makes sense and we have
    \begin{equation*}
        \widehat{\chi}(\theta) = \md{G} \langle \chi, \theta\rangle = 
        \begin{cases}
            \md{G} & \theta = \chi, \\
            0 & \theta \neq \chi.
        \end{cases}
    \end{equation*}
    The last equality follows from \nameref{thm:firstorthorel}.
    
    Thus, $\widehat{\chi} = \md{G} \delta_\chi.$
\end{ex}

As before, we have the Fourier inversion. 

\begin{thm}[Fourier inversion] \label{thm:fourierinv}
    Let $G$ be an abelian group. If $f \in L(G),$ then
    \begin{equation*}
        f = \frac{1}{\md{G}}\sum_{\chi \in \widehat{G}} \widehat{f}(\chi)\chi.
    \end{equation*}
    That is, if $\mathcal{F}(f) = \mathcal{F}(g),$ then $f = g.$ In other words, $\mathcal{F}$ is injective.
\end{thm}
\begin{proof}
    As earlier, we use the fact that the characters form an orthonormal basis along with the computation that
    \begin{equation*}
        f = \sum_{x \in \widehat{G}}\langle f, \chi\rangle\chi = \frac{1}{\md{G}}\sum_{x \in \widehat{G}}\md{G}\langle f, \chi\rangle\chi = \frac{1}{\md{G}}\sum_{x \in \widehat{G}} \widehat{f}(\chi)\chi. \qedhere
    \end{equation*}
\end{proof}

\begin{prop} \label{prop:fourierisovspace}
    The map $\mathcal{F} : L(G) \to L(\widehat{G})$ is an isomorphism of vector spaces.
\end{prop}
\begin{proof}
    Let $\md{G} = n.$ Let $f_1, f_2 \in L(G)$ and $\alpha \in \mathbb{C}$ be arbitrary. For $\chi \in \widehat{G},$ we note that
    \begin{align*}
        \mathcal{F}(\alpha f_1 + f_2)(\chi) &= n\langle \alpha f_1 + f_2, \chi\rangle \\
        &= n\alpha\langle f_1, \chi\rangle + n\langle f_2, \chi\rangle = \alpha\mathcal{F}(f_1)(\chi) + \mathcal{F}(f_2)(\chi).
    \end{align*}
    Thus, $\mathcal{F}$ is linear.
    
    By \nameref{thm:fourierinv}, $\mathcal{F}$ is injective. Now, since $\dim(L(G)) = n = \dim(L(\widehat{G})),$ we see that $\mathcal{F}$ is an isomorphism.
\end{proof}

We now also wish $\mathcal{F}$ to be an isomorphism of rings. However, for this purpose, the convolution product on $L(\widehat{G})$ does not do the trick. Instead, we need the point-wise product on $L(\widehat{G}).$ Clearly, this does make $L(\widehat{G})$ a commutative ring with the constant map $g \mapsto 1$ as identity. As noted earlier, $L(G)$ is also commutative in this case. However, its identity is not the constant function but the function $\delta_1.$

\begin{thm} \label{thm:fourierisorings}
    Let $G$ be an abelian group and let $a, b \in L(G).$

    The Fourier transform satisfies
    \begin{equation*}
        \widehat{a * b} = \widehat{a} \cdot \widehat{b}.
    \end{equation*}
    
    Consequently, the linear map $\mathcal{F} : L(G) \to L(\widehat{G})$ is an isomorphism between the rings $(L(G), +, *)$ and $(L(\widehat{G}), +, \cdot).$
\end{thm}
\begin{proof}
    The only thing that is required to be proven is that $\widehat{a * b} = \widehat{a} \cdot \widehat{b}.$ (Note that both sides are functions $\widehat{G} \to \mathbb{C}.$) Set $n = \md{G}$ and let $\chi \in \widehat{G}$ be arbitrary.
    
    We see that
    \begin{align*}
        \widehat{a * b}(\chi) &= n \langle a * b, \chi\rangle \\
        &= n \cdot \frac{1}{n} \sum_{x \in G} (a * b)(x) \overline{\chi(x)} \\
        &= \sum_{x \in G}\overline{\chi(x)} \sum_{y \in G} a(xy^{-1})b(y) \\
        &= \sum_{y \in G}b(y) \sum_{x \in G}a(xy^{-1})\overline{\chi(x)} \\
        &= \sum_{y \in G}b(y) \sum_{z \in G}a(z)\overline{\chi(zy)} \\
        &= \sum_{y \in G}b(y)\overline{\chi(y)} \sum_{z \in G}a(z)\overline{\chi(z)} \\
        &= n\langle a, \chi\rangle \cdot n\langle b, \chi\rangle \\
        &= \widehat{a}(\chi) \cdot \widehat{b}(\chi). \qedhere
    \end{align*}
\end{proof}

\subsection{An application to Graph Theory}

\begin{defn}%[Graph]
    A \deff{graph} $\Gamma$ is an ordered pair $(V, E)$ where $V$ is a finite \underline{ordered} set and $E$ is a set consisting of unordered \emph{pairs} of elements of $V.$ $V$ is called the \deff{vertex set} of $\Gamma$ and each element of $V$ is called a \deff{vertex}. $E$ is called the \deff{edge set} of $\Gamma$ and each element of $E$ is called an \deff{edge}.

    Given $v, w \in V,$ we say that $v$ and $w$ are \deff{connected by an edge} if $\{v, w\} \in E.$
\end{defn}

\begin{ex}
    $\Gamma = ((1, 2, 3, 4), \{\{1, 3\}, \{2, 3\}, \{2, 4\}, \{3, 4\}\})$ is a graph. This can be depicted as the following diagram.
    

    \begin{center}
        \captionsetup{type=figure}
        \begin{tikzpicture}
            \GraphInit[vstyle=Normal]
            \SetGraphUnit{2}
            \begin{scope}[rotate = 90]
                \Vertices{circle}{2, 3, 4}
            \end{scope}

            \NOWE(3){1}
            \Edges(1, 3, 2, 4, 3)
        \end{tikzpicture}
        \captionof{graph}{A graph} \label{fig:graph1}
    \end{center}
\end{ex}

\begin{rem}
    As in the previous example, we often depict graphs using a diagram as shown above. We draw circles to represent the vertices and draw a line between two vertices $v_i$ and $v_j$ iff $\{v_i, v_j\} \in E.$

    Note that our definition only talks about unordered pairs. Moreover, note that we do not talk about an edge from a vertex to itself.
\end{rem}

\begin{defn}%[Adjacency matrix]
    Let $\Gamma$ be a graph with vertex set $V = (v_1, \ldots, v_n)$ and edge set $E.$ Then, we define the \deff{adjacency matrix} $A = (a_{ij}) \in M_n(\mathbb{Z})$ as
    \begin{equation*} 
        a_{ij} = \begin{cases}
            1 & \{v_i, v_j\} \in E,\\
            0 & \{v_i, v_j\} \notin E. 
        \end{cases}
    \end{equation*}
\end{defn}

\begin{rem}
    By definition, we see that $A$ is symmetric and thus, real diagonalisable, by the spectral theorem. Note that the diagonal entries of $A$ will always be $0.$
\end{rem}

\begin{ex}
    For \Cref{fig:graph1}, the adjacency matrix is given as
    \begin{equation*} 
        A = \begin{bmatrix}
             &   & 1 &   \\
             &   & 1 & 1 \\
           1 & 1 &   & 1 \\
             & 1 & 1 &   \\
        \end{bmatrix}.
    \end{equation*}
\end{ex}

\begin{defn}%[]
    Let $G$ be a finite group written in some fixed order. A subset $S \subset G$ is said to be a \deff{symmetric subset} of $G$ if
    \begin{enumerate}
        \item $1 \notin S,$
        \item $s \in S \implies s^{-1} \in S.$
    \end{enumerate}
    If $S$ is a symmetric subset of $G,$ then the \deff{Cayley graph} of $G$ with respect to $S$ is the graph with vertex set $G$ and with an edge $\{g, h\}$ iff $gh^{-1} \in S.$
\end{defn}

\begin{aside}
    \textbf{Convention}

    Whenever we consider $G = \mathbb{Z}/n\mathbb{Z},$ we shall consider the order to be fixed as $G = ([0], \ldots, [n - 1]).$
\end{aside}

\begin{rem}
    Since $S$ is symmetric, $gh^{-1} \in S$ is equivalent to $hg^{-1} \in S.$ This shows that the above is indeed well-defined. (Since $\{g, h\} = \{h, g\}.$) Note that since $1 \notin S,$ we do not have any singletons in the edge set, which is compatible with our definition.
\end{rem}

\begin{ex}
    Consider the group $G = \mathbb{Z}/4\mathbb{Z}$ (with the decided order). Then, $S = \{\pm [1]\}$ is a symmetric subset of $G.$ (Note that the $1$ of $G$ is $[0].$)

    The Cayley graph is given as

    \begin{center}
        \captionsetup{type=figure}
        \begin{tikzpicture}
            \GraphInit[vstyle=Normal]
            \SetGraphUnit{2}
            \begin{scope}[rotate = 45]
                \Vertices{circle}{[0], [1], [2], [3]}
            \end{scope}
            \Edges([0], [1], [2], [3], [0])
        \end{tikzpicture}
        \captionof{graph}{Cayley graph of $\mathbb{Z}/4\mathbb{Z}$ with respect to $\{\pm [1]\}$} \label{fig:cayleyz4z}
    \end{center}
    and the adjacency matrix as
    \begin{equation*} 
        A = \begin{bmatrix}
            & 1 & & 1\\
             1 &&  1&\\
            & 1 & & 1\\
             1 &&  1&\\
        \end{bmatrix}.
    \end{equation*}
\end{ex}  

\begin{ex}
    If we take $G = \mathbb{Z}/6\mathbb{Z}$ and $S = \{\pm [1], \pm [2]\},$ we get the graph as
    \begin{center}
        \captionsetup{type=figure}
        \begin{tikzpicture}
            \GraphInit[vstyle=Normal]
            \SetGraphUnit{3}
            \begin{scope}%[rotate = 45]
                \Vertices{circle}{[0], [1], [2], [3], [4], [5]}
            \end{scope}
            \Edges([0], [1], [2], [3], [4], [5], [0])
            \Edges([0], [2], [4], [0])
            \Edges([1], [3], [5], [1])
        \end{tikzpicture}
        \captionof{graph}{Cayley graph of $\mathbb{Z}/6\mathbb{Z}$ with respect to $\{\pm [1], \pm [2]\}$} \label{fig:cayleyz6z}
    \end{center}

    and the adjacency matrix as

    \begin{equation*} 
        A = \begin{bmatrix}
              & 1 & 1 &   & 1 & 1 \\
            1 &   & 1 & 1 &   & 1 \\
            1 & 1 &   & 1 & 1 &   \\
              & 1 & 1 &   & 1 & 1 \\
            1 &   & 1 & 1 &   & 1 \\
            1 & 1 &   & 1 & 1 &   \\
        \end{bmatrix}
    \end{equation*}
\end{ex}

\begin{defn}%[Circulant graph]
    A Cayley graph of $\mathbb{Z}/n\mathbb{Z}$ is called a \deff{circulant graph} (on $n$ vertices).
\end{defn}

\begin{defn}%[Circulant matrix]
    An $n \times n$ \deff{circulant matrix} is a matrix of the form
    \begin{equation*} 
        A = \begin{bmatrix}
            a_0 & a_1 & \cdots & a_{n - 2} & a_{n - 1} \\
            a_{n - 1} & a_0 & \cdots & a_{n - 3} & a_{n - 2}\\
            \vdots & \vdots & \ddots & \vdots & \vdots \\
            a_2 & a_3 & \cdots & a_{0} & a_1\\
            a_1 & a_2 & \cdots & a_{n - 1} & a_0
        \end{bmatrix}.
    \end{equation*}
    Equivalently, a matrix $A = (a_{ij})$ is circulant if there exists a function $f : \mathbb{Z}/n\mathbb{Z} \to \mathbb{C}$ such that $a_{ij} = f([j] - [i]).$
\end{defn} 

\begin{ex}
    If $S$ is a symmetric subset of $G \vcentcolon= \mathbb{Z}/n\mathbb{Z},$ then the indicator function $\delta_S : G \to \mathbb{C}$ is defined as
    \begin{equation*} 
        \delta_S(x) = \begin{cases}
            1 & x \in S,\\
            0 & x \notin S. 
        \end{cases}
    \end{equation*} 
    Consider the matrix $A = (a_{ij})$ given by $a_{ij} = f([j] - [i]).$ Then, note that 
    \begin{equation*} 
        a_{ij} = 1 \iff [j] - [i] \in S \iff \{[i], [j]\} \in E,
    \end{equation*}
    where $E$ denotes the edge set of the Cayley graph of $G$ with respect to $S.$

    Thus, we see that the circulant matrix corresponding to $\delta_S$ is the adjacency matrix of the Cayley graph of $G$ with respect to $S.$
\end{ex}

\begin{lem} \label{lem:convolvediagonal}
    Let $G$ be an abelian group and $a \in L(G).$ Define $A : L(G) \to L(G)$ by $A(b) = a * b.$ Then, $A$ is linear and $\chi$ is an eigenvector of $A$ with eigenvalue $\widehat{a}(\chi)$ for all $\chi \in \widehat{G}.$ Consequently, $A$ is diagonalisable.
\end{lem}
\begin{proof} 
    That $A$ is additive follows from the fact that $*$ distributes over $+.$ That $A(\alpha b) = \alpha A(b)$ for $\alpha \in \mathbb{C}$ and $b \in L(G)$ follows easily from the definition of $*.$ Thus, $A$ is linear.

    Put $n \vcentcolon= \md{G}.$ Now, let $\chi \in \widehat{G}$ be arbitrary. Then,
    \begin{equation*} 
        \widehat{a * \chi} = \widehat{a} \cdot \widehat{\chi} = \widehat{a} \cdot n \delta_\chi,
    \end{equation*}
    where the last equality follows from \Cref{ex:fourtranschar}. Now, for $\theta \in \widehat{G},$ we note that
    \begin{equation*} 
        (\widehat{a} \cdot n \delta_\chi)(\theta) = \begin{cases}
            n\widehat{a}(\chi) & \theta = \chi,\\
            0 & \theta \neq \chi.   
        \end{cases}
    \end{equation*}
    Thus, we have $\widehat{a * \chi} = \widehat{a} \cdot n \delta_\chi = n \widehat{a}(\chi) \delta_\chi.$ By \Cref{ex:fourtranschar} again, we get
    \begin{equation*} 
        \widehat{a * \chi} = n \widehat{a}(\chi) \widehat{\chi}.
    \end{equation*}
    Applying the inverse Fourier transform (and using its linearity), we get
    \begin{equation*} 
        a * \chi = \widehat{a}(\chi) \chi.
    \end{equation*}
    (Note that $\widehat{a}(\chi) \in \mathbb{C}$ is a constant.) The above equation is simply
    \begin{equation*} 
        A(\chi) = \widehat{a}(\chi)\chi
    \end{equation*}
    and hence, $\chi$ is an eigenvector with eigenvalue $\widehat{a}(\chi).$

    Note that the irreducible characters of $G$ form a basis of $Z(L(G)),$ by \Cref{thm:onbforzlg}. Since $G$ is abelian, the irreducible characters are precisely the elements of $\widehat{G}$ and $Z(L(G)) = L(G).$ Thus, we see that $\widehat{G}$ forms a basis of $L(G)$ and have just shown that all these elements are eigenvectors of $A.$ Thus, $\widehat{G}$ is an eigenbasis of $L(G)$ corresponding and $A$ is diagonalisable.
\end{proof}

\begin{thm} \label{thm:adjcayleyeigen}
    Let $G = (g_1, \ldots, g_n)$ be an ordered abelian group and let $S \subset G$ be a symmetric subset. Let $\chi_1, \ldots, \chi_n$ be irreducible characters of $G$ and let $A$ be the adjacency matrix of the Cayley graph of $G$ with respect to $S.$ Then,
    \begin{enumerate}
        \item The eigenvalues of $A$ are the real numbers
        \begin{equation*} 
            \lambda_i \vcentcolon= \sum_{s \in S} \chi_i(s)
        \end{equation*}
        for $1 \le i \le n;$
        \item A corresponding orthonormal basis of eigenvectors is given by the vectors $(v_1, \ldots, v_n)$ where
        \begin{equation*} 
            v_i \vcentcolon= \frac{1}{\sqrt{\md{G}}} \begin{bmatrix}
                \chi_i(g_1) & \cdots & \chi_i(g_n)
            \end{bmatrix}^\mathsf{T}.
        \end{equation*}
    \end{enumerate}
\end{thm}

\begin{proof} 
    The indicator function $\delta_S$ of $S$ is given by $\delta_S = \sum_{s \in S} \delta_s.$ Note that $\delta_S \in L(G).$ Define $F : L(G) \to L(G)$ by
    \begin{equation*} 
        F(b) = \delta_S * b.
    \end{equation*}
    Then, by \Cref{lem:convolvediagonal}, $F$ has eigenvectors $\chi_i$ with corresponding eigenvalue
    \[\begin{WithArrows}[displaystyle]
        \widehat{\delta_S}(\chi_i) = n\langle \delta_S, \chi_i\rangle &= \sum_{x \in G} \delta_S(x)\overline{\chi_i(x)}\\
        &= \sum_{x \in S} \overline{\chi_i(x)} \Arrow{\Cref{rem:findegoneunitary}}\\
        &= \sum_{x \in S} \chi_i(x^{-1}) \Arrow{$S$ is symmetric}\\
        &= \sum_{y \in S} \chi_i(y)\\
        &= \lambda_i.
    \end{WithArrows}\]
    This proves that $\lambda_i$ is an eigenvalue for each $1 \le i \le n.$ (Note that $\lambda_i \in \mathbb{R}$ follows since $A$ is symmetric.\footnote{Alternately, one can note that that sum in the definition of $\lambda_i$ is real. Indeed, if $s = s^{-1}$, then $\chi(s) = \chi(s^{-1}) = \overline{\chi(s)} \in \mathbb{R}.$ On the other hand, if $s \neq s^{-1},$ then pairing up $\chi(s)$ and $\chi(s^{-1})$ in the sum gives $\chi(s) + \chi(s^{-1}) = \chi(s) + \overline{\chi(s)} \in \mathbb{R}.$})

    Consider the ordered basis $B = (\delta_{g_1}, \ldots, \delta_{g_n})$ for $L(G).$ Let $[F]_B$ denote the matrix of $F$ with respect to this ordered basis. Note that the coordinate vector of $\chi_i$ with respect to $B$ is precisely $\sqrt{\md{G}}v_i.$ The above shows that it is an eigenvector with eigenvalue $\lambda_i.$ The orthogonality of $v_i$ follows from \Cref{thm:secondorthorel}. (Note that the conjugacy classes are singletons since $G$ is abelian.) \\
    Lastly, the orthonormality follows from noting that
    \begin{equation*} 
        \|v_i\|^2 = \frac{1}{G}\left(\md{\chi_i(g_1)}^2 + \cdots + \md{\chi_i(g_n)}^2\right) = \|\chi_i\|^2 = 1.
    \end{equation*}
    Thus, to complete the proof, it suffices to show that $A = [F]_B.$

    Let $1 \le i, j \le n$ be arbitrary. We show that $A_{ij} = ([F]_B)_{ij}.$ Note that $([F]_B)_{ij}$ is the coefficient of $\delta_{g_i}$ in $F(\delta_{g_j}).$ We note that
    \begin{equation*} 
        F(\delta_{g_j}) = \delta_S * \delta_{g_j} = \sum_{s \in S}\delta_s * \delta_{g_j} = \sum_{s \in S} \delta_{sg_j} = \sum_{g \in Sg_j} \delta_g.
    \end{equation*}
    (We have used $\delta_{gh} = \delta_g * \delta_h,$ by \Cref{prop:convdeltaprod}.)

    Thus, 
    \begin{align*} 
        ([F]_B)_{ij} &= \begin{cases}
            1 & g_i \in Sg_j,\\
            0 & g_i \notin Sg_j,
        \end{cases}\\
        &= \begin{cases}
            1 & g_ig_j^{-1} \in S,\\
            0 & g_ig_j^{-1} \notin S,
        \end{cases}\\
        &= A_{ij}. \qedhere
    \end{align*}
\end{proof}

\begin{cor}
    Let $A$ be a circulant matrix of degree $n,$ which is the adjacency matrix of the Cayley graph of $\mathbb{Z}/n\mathbb{Z}$ with respect to some symmetric subset $S \subset \mathbb{Z}/n\mathbb{Z}.$ Then, the eigenvalues of $A$ are
    \begin{equation*} 
        \lambda_k \vcentcolon= \sum_{[m] \in S} \omega_n^{km},
    \end{equation*}
    for $k = 0, \ldots, n - 1$ with corresponding basis of orthonormal eigenvectors given by
    \begin{equation*} 
        v_k \vcentcolon= \frac{1}{\sqrt{n}}\begin{bmatrix}
            1 & \omega_n^{k} & \omega_n^{2k} & \cdots & \omega_n^{(n - 1)k}
        \end{bmatrix}^\mathsf{T}.
    \end{equation*}
\end{cor}
\begin{proof} 
    The result follows from \Cref{thm:adjcayleyeigen} since we have
    \begin{equation*} 
        \chi_k([m]) = \omega_n^{km}
    \end{equation*}
    for $k = 0, \ldots, n - 1.$ 
\end{proof}

\begin{ex}
    Recall \Cref{fig:cayleyz6z}. We had $n = 6$ and $S = \{\pm [1], \pm [2]\}.$ Thus, the eigenvalues of $A$ are given by
    \begin{equation*} 
        \lambda_k = \omega_6^k + \omega_6^{-k} + \omega_6^{2k} + \omega_6^{-2k} = 2\left(\cos\left(\frac{\pi k}{3}\right) + \cos\left(\frac{2\pi k}{3}\right)\right)
    \end{equation*}
    for $k = 0, \ldots, 5.$
\end{ex}

\subsection{Fourier Analysis on Non-abelian Groups}

If $G$ is a non-abelian group, then $L(G) \neq Z(L(G)).$ Note however that that pointwise product of functions into $\mathbb{C}$ is commutative. Thus, we cannot have a Fourier transform converting convolution into pointwise product while being an isomorphism. To remedy this, we look at matrix multiplication instead of pointwise.

Before that, we look at the case of abelian groups in a different light. First, recall that $\mathbb{C}^n$ is a ring with product given as
\begin{equation*} 
    (w_1, \ldots, w_n) \cdot (z_1, \ldots, z_n) \vcentcolon= (w_1z_1, \ldots, w_nz_n).
\end{equation*}

\begin{prop}
    Let $G$ be a finite abelian group with irreducible characters $\chi_1, \ldots, \chi_n.$ Define $T : L(G) \to \mathbb{C}^n$ by
    \begin{equation*} 
        Tf = (\widehat{f}(\chi_1), \ldots, \widehat{f}(\chi_n)).
    \end{equation*}
    Then, $T$ is an isomorphism of rings.
\end{prop}
\begin{proof} 
    Note that if $f, g \in L(G)$ are such that $Tf = Tg,$ then $\widehat{f} = \widehat{g}.$ Fourier inversion gives $f = g.$ Thus, $T$ is injective. Since $T$ is $\mathbb{C}$-linear (same proof as \Cref{prop:fourierisovspace}) and $\dim_{\mathbb{C}}(L(G)) = n = \dim_{\mathbb{C}}(\mathbb{C}^n),$ it follows that $T$ is a bijection and hence, an isomorphism of $\mathbb{C}$-vector spaces.

    To show that is an isomorphism of rings, all that remains is to show that $T(f * g) = Tf \cdot Tg.$ This follows directly from the fact that $\widehat{f * g}(\chi_i) = \widehat{f}(\chi_i)\widehat{g}(\chi_i)$ for all $i = 1, \ldots, n.$
\end{proof}

Thus, \Cref{thm:fourierisorings} can be stated as follows.
\begin{thm}
    Let $G$ be a finite abelian group of order $n.$ Then, $L(G) \cong \mathbb{C}^n$ as rings.
\end{thm}

Note that all the irreducible representations of $G$ have degree one. The above product can be seen as $\mathbb{C}^n \cong M_1(\mathbb{C}) \times \cdots \times M_1(\mathbb{C}).$ In general, we replace the $1$s with the degrees of the irreducible representations.

\begin{aside}
    \textbf{Setup.}

    $G$ is a finite group of order $n$ and $\varphi^{(1)}, \ldots, \varphi^{(s)}$ is a transversal of irreducible unitary representations of $G.$ Put $d_k = \deg \varphi^{(k)}.$ 

    For $1 \le k \le s$ and $1 \le i, j \le d_k,$ we have the functions $\varphi^{(k)}_{ij} : G \to \mathbb{C}$ such that the matrix $\varphi^{(k)}(g)$ is given as $(\varphi^{(k)}_{ij}(g))$ for all $g \in G.$

    Let $D = \{(i, j, k) : 1 \le k \le s,\; 1 \le i, j \le d_k\}.$

    We recall \Cref{cor:orthnormalbasis} which told us that $B = \left\{\sqrt{d_k}\varphi_{ij}^{(k)} \mid (i, j, k) \in D\right\}$ is an orthonormal basis for $L(G).$
\end{aside}

\begin{defn}%[]
    Define
    \begin{equation*} 
        \mathcal{F} : L(G) \to M_{d_1}(\mathbb{C}) \times \cdots \times M_{d_s}(\mathbb{C})
    \end{equation*}
    by $\mathcal{F}(f) \vcentcolon= (\widehat{f}(\varphi^{(1)}), \ldots, \widehat{f}(\varphi^{(s)}))$ where
    \begin{equation*}
        \widehat{f}(\varphi^{(k)}) \vcentcolon= \sum_{g \in G} f(g)\overline{\varphi^{(k)}_g}.
    \end{equation*}
    We call $\mathcal{F}(f)$ the \deff{Fourier transform} of $f.$
\end{defn}
\begin{rem}
    In terms of the matrix entries, the above can be rewritten as
    \begin{equation} \label{eq:012}
        \widehat{f}(\varphi^{(k)})_{ij} \vcentcolon= \sum_{g \in G} f(g)\overline{\varphi^{(k)}_{ij}(g)} = n\langle f, \varphi^{(k)}_{ij}\rangle.
    \end{equation}
\end{rem}

\begin{rem} \label{rem:dimLGdimmatrix}
    Note that $L(G)$ is a $\mathbb{C}$-vector space of dimension $\md{G}.$ On the other hand, $M_{d_i}(\mathbb{C})$ is a $\mathbb{C}$-vector space with dimension $d_i^2.$ Thus, the product $M_{d_1}(\mathbb{C}) \times \cdots \times M_{d_s}(\mathbb{C})$ is a vector space of dimension $d_1^2 + \cdots + d_s^2 = \md{G}.$

    Thus, we see that $L(G)$ and $M_{d_1}(\mathbb{C}) \times \cdots \times M_{d_s}(\mathbb{C})$ have the same dimension. We shall show that $\mathcal{F}$ is an isomorphism.
\end{rem}

\begin{thm}[Fourier inversion] \label{thm:fourierinvgen}
    Let $f \in L(G).$ Then,
    \begin{equation*} 
        f = \frac{1}{n}\sum_{(i, j, k) \in D}d_k \widehat{f}(\varphi^{(k)})_{ij} \varphi^{(k)}_{ij}.    
    \end{equation*}
    In particular, $\mathcal{F}$ is injective.
\end{thm}
\begin{proof} 
    The proof is as before. We know that $B$ is an orthonormal basis. Thus, it suffices to prove that
    \begin{equation*} 
        \langle f, \sqrt{d_k}\varphi^{(k)}_{ij}\rangle = \frac{1}{n}\sqrt{d_k} \widehat{f}(\varphi^{(k)})_{ij}
    \end{equation*}
    for all $(i, j, k) \in D.$ However, the above is precisely \Cref{eq:012}.
\end{proof}

\begin{thm}
    The Fourier transform $\mathcal{F} : L(G) \to M_{d_1}(\mathbb{C}) \times \cdots \times M_{d_s}(\mathbb{C})$ is an isomorphism of vector spaces.
\end{thm}
\begin{proof} 
    As usual, the check that it is linear is simple and essentially follows from the fact that $\langle \cdot , \cdot \rangle$ is linear in the first variable. \\
    By \Cref{thm:fourierinvgen}, we know that $\mathcal{F}$ is injective. \\
    By \Cref{rem:dimLGdimmatrix}, we know that $\dim_{\mathbb{C}}(L(G)) = \dim_{\mathbb{C}}(M_{d_1}(\mathbb{C}) \times \cdots \times M_{d_s}(\mathbb{C}))$ and thus, $\mathcal{F}$ is an isomorphism.
\end{proof}

Note that $M_{d_1}(\mathbb{C}) \times \cdots \times M_{d_s}(\mathbb{C})$ is a ring as well, with coordinate-wise product.

\begin{thm}[Wedderburn]
    The Fourier transform $\mathcal{F} : L(G) \to M_{d_1}(\mathbb{C}) \times \cdots \times M_{d_s}(\mathbb{C})$ is an isomorphism of rings.
\end{thm}
\begin{proof} 
    All that is required to be proven is that $T(f * g) = Tf \cdot Tg,$ where the latter product is in the ring $M_{d_1}(\mathbb{C}) \times \cdots \times M_{d_s}(\mathbb{C}).$ Let $a, b \in L(G).$

    Since the latter product is coordinate-wise, it suffices to show that
    \begin{equation*} 
        \widehat{(a * b)}(\varphi^{(k)}) = \widehat{a}(\varphi^{(k)})\widehat{b}(\varphi^{(k)})
    \end{equation*}
    for all $1 \le k \le s.$ (The product on the right is matrix multiplication.)

    The computation is the same as before.
    \begin{align*} 
        \widehat{(a * b)}(\varphi^{(k)}) &= \sum_{g \in G} (a * b)(g)\overline{\varphi^{(k)}(g)}\\
        &= \sum_{g \in G}\left(\sum_{h \in G}a(gh^{-1})b(h)\right)\overline{\varphi^{(k)}(g)}\\
        &= \sum_{h \in G}b(h)\sum_{g \in G}a(gh^{-1})\overline{\varphi^{(k)}(g)}\\
        &= \sum_{h \in G}b(h)\sum_{g' \in G}a(g')\overline{\varphi^{(k)}(g'h)}\\
        &= \sum_{h \in G}b(h)\sum_{g' \in G}a(g')\overline{\varphi^{(k)}(g')}\cdot\overline{\varphi^{(k)}(h)}\\
        &= \sum_{g' \in G}a(g')\overline{\varphi^{(k)}(g')}\sum_{h \in G}b(h)\overline{\varphi^{(k)}(h)}\\
        &= \widehat{a}(\varphi^{(k)})\widehat{b}(\varphi^{(k)}). \qedhere
    \end{align*}
\end{proof}
In the above computation, note that $a$ and $b$ took values in $\mathbb{C}$ and so, commuted with the other terms.

\section{Fourier Analysis on Finite Groups}
\subsection{Periodic Functions on Cyclic Groups}

\begin{defn}
    Let $n \in \mathbb{Z} \setminus \{0\}.$ A function $f : \mathbb{Z} \to \mathbb{C}$ is said to be \deff{periodic} with \deff{period} $n$ if $f(x + n) = f(x)$ for all $x \in \mathbb{Z}.$
\end{defn}

Note that there is no unique period associated to a periodic function. Indeed, if $n$ is a period, then so is any multiple of $n.$

\begin{rem}
    Fix an integer $n > 0.$ The set of functions periodic with period $n$ is in bijection with the set of all complex valued functions on $\mathbb{Z}/n\mathbb{Z}.$
    
    Indeed, given a period function $f : \mathbb{Z} \to \mathbb{C}$ with period $n,$ we get a function $\tilde{f} : \mathbb{Z}/n\mathbb{Z} \to \mathbb{C}$ defined as $\tilde{f}([m]) = f(m).$ (This is well-defined by assumption of periodicity.)
    
    Conversely, given a function $g : \mathbb{Z}/n\mathbb{Z} \to \mathbb{C},$ we get a function $\dot{g} : \mathbb{Z} \to \mathbb{C}$ defined as $\dot{g}(m) = g([m]).$ Clearly, $g$ is periodic with period $n.$
    
    Moreover, the correspondences $f \mapsto \tilde{f}$ and $g \mapsto \dot{g}$ are inverses. Thus, the set of functions periodic with period $n$ is in bijection with $L(\mathbb{Z}/n\mathbb{Z}).$ (Recall that $L(G)$ was the \hyperref[defn:groupalg]{group algebra} of the group $G.$)
\end{rem}

\begin{rem}
    Recall that \Cref{cor:orthnormalbasis} gave us an orthonormal basis of $L(G).$ Since the irreducible representations of $\mathbb{Z}/n\mathbb{Z}$ are all of degree one, we see that the set is simply $\{\chi_k : 0 \le k < n\},$ where $\chi_k([m]) = w_n^{mk}.$
    
    Thus, for $f \in L(\mathbb{Z}/n\mathbb{Z})$ we get 
    \begin{equation*}
        f = \langle f, \chi_0\rangle \chi_0 + \cdots + \langle f, \chi_{n - 1}\rangle \chi_{n - 1}.
    \end{equation*}
\end{rem}

\begin{defn} \label{defn:fourierZnZ}
    Let $f : \mathbb{Z}/n\mathbb{Z} \to \mathbb{C}.$ The \deff{Fourier transform} $\mathcal{F}(f) = \widehat{f} : \mathbb{Z}/n\mathbb{Z} \to \mathbb{C}$ of $f$ is defined as
    \begin{equation*}
        \widehat{f}([m]) = \sum_{k = 0}^{n - 1} f([k])e^{-2\pi\iota mk/n} = \sum_{k = 0}^{n - 1} f([k])\omega_n^{-mk} = n\langle f, \chi_m\rangle.
    \end{equation*}
\end{defn}

The last equality follows by the definition of $\langle \cdot , \cdot\rangle$ and the fact that $n \md{\mathbb{Z}/n\mathbb{Z}}.$ Moreover, note that $\mathcal{F} : L(\mathbb{Z}/n\mathbb{Z}) \to L(\mathbb{Z}/n\mathbb{Z})$ is linear since
$\langle \cdot , \cdot\rangle$ is linear in the first variable.

\begin{prop}
    The Fourier transform is invertible. More precisely, 
    \begin{equation*}
        f = \frac{1}{n}\sum_{k = 0}^{n - 1} \widehat{f}([k])\chi_k.
    \end{equation*}
\end{prop}
\begin{proof}
    We simply compute $\langle f, \chi_k\rangle$ and show that it equals $\widehat{f}([k])/n.$
    
    To this end, note that
    \begin{align*}
        \langle f, \chi_k\rangle &= \frac{1}{n} \sum_{j = 0}^{n - 1}f([j])\overline{\chi_k(j)} \\
        &= \frac{1}{n} \sum_{j = 0}^{n - 1}f([j])e^{-2\pi\iota kj/n} \\
        &= \frac{1}{n} \widehat{f}([k]). \qedhere
    \end{align*}
\end{proof}

\subsection{The Convolution Product}

\begin{defn}
    Let $G$ be a group.
    Given $g \in G,$ we define $\delta_g : G \to \mathbb{C}$ as
    \begin{equation*}
        \delta_g(x) \vcentcolon=
        \begin{cases}
            1 & x = g, \\
            0 & x \neq g.
        \end{cases}
    \end{equation*}
\end{defn}

\begin{defn}
    Let $G$ be a finite group and $a, b \in L(G).$ Then, $a * b \in L(G)$ is defined as
    \begin{equation*}
        (a * b)(x) = \sum_{y \in G} a(xy^{-1})b(y)
    \end{equation*}
    and is called the \deff{convolution} of $a$ with $b.$
\end{defn}

The above is well-defined since $G$ is finite and the sum is taken in $\mathbb{C},$ which is commutative. 

\begin{rem} \label{rem:classfunctionincenter}
    We have
    \begin{equation*}
        (a * b)(x) = \sum_{y \in G} a(xy^{-1})b(y).
    \end{equation*}
    The change of variable $y \mapsto xz^{-1}$ gives
    \begin{equation*}
        (a * b)(x) = \sum_{z \in G} a(xzx^{-1})b(xz^{-1}) = \sum_{z \in G} b(xz^{-1})a(xzx^{-1}).
    \end{equation*}
    Thus, if $a$ is a class function, then we get
    \begin{equation*}
        (a * b)(x) = \sum_{z \in G} b(xz^{-1})a(z) = (b * a)(x)
    \end{equation*}
    for all $x \in G$ and hence, $a * b = b * a.$
    
    In particular, if $G$ is abelian, then $a * b = b * a$ for all $a, b \in L(G).$
    
    However, for a general group, it is not necessary that $*$ is commutative. In fact, after the next proposition, it will be clear that $*$ is commutative iff $G$ is abelian.
\end{rem}

\begin{prop} \label{prop:convdeltaprod}
    Let $G$ be a finite group and $g, h \in G.$ Then, $\delta_g * \delta_h = \delta_{gh}.$
\end{prop}
\begin{proof}
    Let $x \in G.$ Then,
    \begin{equation*}
        (\delta_g * \delta_h)(x) = \sum_{y \in G} \delta_g(xy^{-1})\delta_h(y).
    \end{equation*}
    Thus, $(\delta_g * \delta_h)(x) = 1$ iff $h = y$ and $xy^{-1} = g$ iff $x = gh$ and $0$ otherwise.
    
    In other words, $(\delta_g * \delta_h) = \delta_{gh}.$
\end{proof}

Thus, if $G$ is a non-abelian group, then choose $g, h \in G$ such that $gh \neq hg.$ Then, we get $\delta_g * \delta_h \neq \delta_h * \delta_g.$ 

\begin{prop} \label{prop:convole}
    Let $a \in L(G)$ and $h, g \in G.$ Then, $(a * \delta_h)(g) = a(gh^{-1}).$
\end{prop}

There are many properties that one can verify about the convolution product. We list them below and leave the proof as a computational exercise.

\begin{prop}
    Let $G$ be a finite group. Then,
    \begin{enumerate}
        \item $a * \delta_1 = a = \delta_1 * a$ for all $a \in L(G),$
        \item $a * (b * c) = (a * b) * c$ for all $a, b, c \in L(G),$
        \item $a * (b + c) = (a * b) + (a * c)$ for all $a, b, c \in L(G).$
    \end{enumerate}
    In other words, $(L(G), +, *)$ is a ring with $\delta_1$ as (multiplicative) identity.
\end{prop}

As noted earlier, $L(G)$ is a commutative ring iff $G$ is commutative. The above also justifies why we used the term ``group algebra''. Soon, the usage of $Z(L(G))$ for the set of class functions will become apparent too.

\begin{rem}
    Note that since $\delta_1$ is the identity, \Cref{prop:convdeltaprod} tells us the map $i : G \to L(G)$ defined as $g \mapsto \delta_g$ is a group homomorphism into the group of units $(L(G))^\times.$ (Recall \Cref{lem:determininggrouphomoring}.)
\end{rem}

\begin{defn}
    Let $R$ be a ring. The \deff{center} of $R$ is denoted $Z(R)$ and defined as
    \begin{equation*}
        Z(R) \vcentcolon= \{r \in R : rs = sr \text{ for all } s \in R\}.
    \end{equation*}
    That is, it is the set of all those elements which commute with every element of $R.$
\end{defn}

\begin{rem}
    One can show that the above is actually a \emph{subring} of $R.$ Moreover, it is a commutative ring. However, we do not use this fact.
\end{rem}

Recall that we had already used the notation $Z(L(G))$ to denote the set of class functions of $G.$ We now show that it is indeed the center and thus, the notation is unambiguous.

\begin{prop}
    Let $G$ be a finite group. Then, $a : G \to \mathbb{C}$ is a class function if and only if $a$ is in the center of $L(G).$
\end{prop}
\begin{proof}
    $(\implies)$ Suppose $a : G \to \mathbb{C}$ is a class function. We had observed in \Cref{rem:classfunctionincenter} that $a * b = b * a$ for all $b \in L(G)$ and hence, $a$ is in the center.
    
    $(\impliedby)$ Let $a$ be in the center of $L(G)$ and let $g, h \in G$ be arbitrary. Then,
    \begin{align*}
        a(gh) = \sum_{y \in G} a(gy^{-1})\delta_{h^{-1}}(y) &= (a * \delta_{h^{-1}})(g) \\
        &= (\delta_{h^{-1}} * a)(g) = \sum_{y \in G}\delta_{h^{-1}}(gy^{-1})a(y) = a(hg).
    \end{align*}
    Thus, given any $x, y \in G,$ setting $g = xy$ and $h = x^{-1}$ gives $a(xyx^{-1}) = a(x),$ showing that $a$ is a class function.
\end{proof}

\subsection{Fourier Analysis on Finite Abelian Groups}

Recall that we had defined the \hyperref[defn:dualgroup]{dual of a group}, the group of all homomorphisms $G \to \mathbb{C}^\times$ with point-wise multiplication as operation. In the case that $G$ is finite and abelian, we see that the elements of $\widehat{G}$ are precisely the irreducible characters of $G$ and that $G \cong \widehat{G}.$ (The last was \Cref{cor:GconghatG}.) \\
Moreover, note that $L(G)$ is abelian. 

Recall that we had defined the Fourier transform $\mathcal{F} : L(G) \to L(G)$ earlier for the case that $G = \mathbb{Z}/n\mathbb{Z}.$ Now, we do it in a more general setting, taking inspiration from the earlier. Instead of a map $L(G) \to L(G),$ we define $\mathcal{F} : L(G) \to L(\widehat{G}).$ 

\begin{defn}
    Let $G$ be a finite abelian group and let $f : G \to \mathbb{C}$ be a function. We define the \deff{Fourier transform} $\mathcal{F}(f) = \widehat{f} \in L(\widehat{G})$ by 
    \begin{equation*}
        \widehat{f}(\chi) = \md{G}\langle f, \chi\rangle = \sum_{g \in G}f(g) \overline{\chi(g)}.
    \end{equation*}
    The complex numbers $\md{G}\langle f, \chi\rangle$ are called the \deff{Fourier coefficients} of $f.$
\end{defn}

Note that $\widehat{f}$ is a function $\widehat{f} : \widehat{G} \to \mathbb{C}.$ That is, $\widehat{f}$ takes irreducible characters of $G$ as input.

\begin{rem}
    Recall we had constructed an isomorphism $\mathbb{Z}/n\mathbb{Z} \cong \widehat{\mathbb{Z}/n\mathbb{Z}}$ in \Cref{prop:ZnZconghatZnZ}. The isomorphism was given as $[k] \leftrightarrow \chi_k,$ where $\chi_k \in \widehat{\mathbb{Z}/n\mathbb{Z}},$ as before is
    \begin{equation*} 
        \chi_k([m]) = w_n^{mk}.
    \end{equation*}

    Under this identification, we see that the Fourier transform defined above agrees with the one in \Cref{defn:fourierZnZ}.
\end{rem}

\begin{ex}[Fourier transform of a character] \label{ex:fourtranschar}
    Let $\chi, \theta \in \widehat{G}$ be (irreducible) characters of the abelian group $G.$ Then, $\chi \in L(G)$ and thus, the Fourier transform of $\chi$ makes sense and we have
    \begin{equation*}
        \widehat{\chi}(\theta) = \md{G} \langle \chi, \theta\rangle = 
        \begin{cases}
            \md{G} & \theta = \chi, \\
            0 & \theta \neq \chi.
        \end{cases}
    \end{equation*}
    The last equality follows from \nameref{thm:firstorthorel}.
    
    Thus, $\widehat{\chi} = \md{G} \delta_\chi.$
\end{ex}

As before, we have the Fourier inversion. 

\begin{thm}[Fourier inversion] \label{thm:fourierinv}
    Let $G$ be an abelian group. If $f \in L(G),$ then
    \begin{equation*}
        f = \frac{1}{\md{G}}\sum_{\chi \in \widehat{G}} \widehat{f}(\chi)\chi.
    \end{equation*}
    That is, if $\mathcal{F}(f) = \mathcal{F}(g),$ then $f = g.$ In other words, $\mathcal{F}$ is injective.
\end{thm}
\begin{proof}
    As earlier, we use the fact that the characters form an orthonormal basis along with the computation that
    \begin{equation*}
        f = \sum_{x \in \widehat{G}}\langle f, \chi\rangle\chi = \frac{1}{\md{G}}\sum_{x \in \widehat{G}}\md{G}\langle f, \chi\rangle\chi = \frac{1}{\md{G}}\sum_{x \in \widehat{G}} \widehat{f}(\chi)\chi. \qedhere
    \end{equation*}
\end{proof}

\begin{prop} \label{prop:fourierisovspace}
    The map $\mathcal{F} : L(G) \to L(\widehat{G})$ is an isomorphism of vector spaces.
\end{prop}
\begin{proof}
    Let $\md{G} = n.$ Let $f_1, f_2 \in L(G)$ and $\alpha \in \mathbb{C}$ be arbitrary. For $\chi \in \widehat{G},$ we note that
    \begin{align*}
        \mathcal{F}(\alpha f_1 + f_2)(\chi) &= n\langle \alpha f_1 + f_2, \chi\rangle \\
        &= n\alpha\langle f_1, \chi\rangle + n\langle f_2, \chi\rangle = \alpha\mathcal{F}(f_1)(\chi) + \mathcal{F}(f_2)(\chi).
    \end{align*}
    Thus, $\mathcal{F}$ is linear.
    
    By \nameref{thm:fourierinv}, $\mathcal{F}$ is injective. Now, since $\dim(L(G)) = n = \dim(L(\widehat{G})),$ we see that $\mathcal{F}$ is an isomorphism.
\end{proof}

We now also wish $\mathcal{F}$ to be an isomorphism of rings. However, for this purpose, the convolution product on $L(\widehat{G})$ does not do the trick. Instead, we need the point-wise product on $L(\widehat{G}).$ Clearly, this does make $L(\widehat{G})$ a commutative ring with the constant map $g \mapsto 1$ as identity. As noted earlier, $L(G)$ is also commutative in this case. However, its identity is not the constant function but the function $\delta_1.$

\begin{thm} \label{thm:fourierisorings}
    Let $G$ be an abelian group and let $a, b \in L(G).$

    The Fourier transform satisfies
    \begin{equation*}
        \widehat{a * b} = \widehat{a} \cdot \widehat{b}.
    \end{equation*}
    
    Consequently, the linear map $\mathcal{F} : L(G) \to L(\widehat{G})$ is an isomorphism between the rings $(L(G), +, *)$ and $(L(\widehat{G}), +, \cdot).$
\end{thm}
\begin{proof}
    The only thing that is required to be proven is that $\widehat{a * b} = \widehat{a} \cdot \widehat{b}.$ (Note that both sides are functions $\widehat{G} \to \mathbb{C}.$) Set $n = \md{G}$ and let $\chi \in \widehat{G}$ be arbitrary.
    
    We see that
    \begin{align*}
        \widehat{a * b}(\chi) &= n \langle a * b, \chi\rangle \\
        &= n \cdot \frac{1}{n} \sum_{x \in G} (a * b)(x) \overline{\chi(x)} \\
        &= \sum_{x \in G}\overline{\chi(x)} \sum_{y \in G} a(xy^{-1})b(y) \\
        &= \sum_{y \in G}b(y) \sum_{x \in G}a(xy^{-1})\overline{\chi(x)} \\
        &= \sum_{y \in G}b(y) \sum_{z \in G}a(z)\overline{\chi(zy)} \\
        &= \sum_{y \in G}b(y)\overline{\chi(y)} \sum_{z \in G}a(z)\overline{\chi(z)} \\
        &= n\langle a, \chi\rangle \cdot n\langle b, \chi\rangle \\
        &= \widehat{a}(\chi) \cdot \widehat{b}(\chi). \qedhere
    \end{align*}
\end{proof}

\subsection{An application to Graph Theory}

\begin{defn}%[Graph]
    A \deff{graph} $\Gamma$ is an ordered pair $(V, E)$ where $V$ is a finite \underline{ordered} set and $E$ is a set consisting of unordered \emph{pairs} of elements of $V.$ $V$ is called the \deff{vertex set} of $\Gamma$ and each element of $V$ is called a \deff{vertex}. $E$ is called the \deff{edge set} of $\Gamma$ and each element of $E$ is called an \deff{edge}.

    Given $v, w \in V,$ we say that $v$ and $w$ are \deff{connected by an edge} if $\{v, w\} \in E.$
\end{defn}

\begin{ex}
    $\Gamma = ((1, 2, 3, 4), \{\{1, 3\}, \{2, 3\}, \{2, 4\}, \{3, 4\}\})$ is a graph. This can be depicted as the following diagram.
    

    \begin{center}
        \captionsetup{type=figure}
        \begin{tikzpicture}
            \GraphInit[vstyle=Normal]
            \SetGraphUnit{2}
            \begin{scope}[rotate = 90]
                \Vertices{circle}{2, 3, 4}
            \end{scope}

            \NOWE(3){1}
            \Edges(1, 3, 2, 4, 3)
        \end{tikzpicture}
        \captionof{graph}{A graph} \label{fig:graph1}
    \end{center}
\end{ex}

\begin{rem}
    As in the previous example, we often depict graphs using a diagram as shown above. We draw circles to represent the vertices and draw a line between two vertices $v_i$ and $v_j$ iff $\{v_i, v_j\} \in E.$

    Note that our definition only talks about unordered pairs. Moreover, note that we do not talk about an edge from a vertex to itself.
\end{rem}

\begin{defn}%[Adjacency matrix]
    Let $\Gamma$ be a graph with vertex set $V = (v_1, \ldots, v_n)$ and edge set $E.$ Then, we define the \deff{adjacency matrix} $A = (a_{ij}) \in M_n(\mathbb{Z})$ as
    \begin{equation*} 
        a_{ij} = \begin{cases}
            1 & \{v_i, v_j\} \in E,\\
            0 & \{v_i, v_j\} \notin E. 
        \end{cases}
    \end{equation*}
\end{defn}

\begin{rem}
    By definition, we see that $A$ is symmetric and thus, real diagonalisable, by the spectral theorem. Note that the diagonal entries of $A$ will always be $0.$
\end{rem}

\begin{ex}
    For \Cref{fig:graph1}, the adjacency matrix is given as
    \begin{equation*} 
        A = \begin{bmatrix}
             &   & 1 &   \\
             &   & 1 & 1 \\
           1 & 1 &   & 1 \\
             & 1 & 1 &   \\
        \end{bmatrix}.
    \end{equation*}
\end{ex}

\begin{defn}%[]
    Let $G$ be a finite group written in some fixed order. A subset $S \subset G$ is said to be a \deff{symmetric subset} of $G$ if
    \begin{enumerate}
        \item $1 \notin S,$
        \item $s \in S \implies s^{-1} \in S.$
    \end{enumerate}
    If $S$ is a symmetric subset of $G,$ then the \deff{Cayley graph} of $G$ with respect to $S$ is the graph with vertex set $G$ and with an edge $\{g, h\}$ iff $gh^{-1} \in S.$
\end{defn}

\begin{aside}
    \textbf{Convention}

    Whenever we consider $G = \mathbb{Z}/n\mathbb{Z},$ we shall consider the order to be fixed as $G = ([0], \ldots, [n - 1]).$
\end{aside}

\begin{rem}
    Since $S$ is symmetric, $gh^{-1} \in S$ is equivalent to $hg^{-1} \in S.$ This shows that the above is indeed well-defined. (Since $\{g, h\} = \{h, g\}.$) Note that since $1 \notin S,$ we do not have any singletons in the edge set, which is compatible with our definition.
\end{rem}

\begin{ex}
    Consider the group $G = \mathbb{Z}/4\mathbb{Z}$ (with the decided order). Then, $S = \{\pm [1]\}$ is a symmetric subset of $G.$ (Note that the $1$ of $G$ is $[0].$)

    The Cayley graph is given as

    \begin{center}
        \captionsetup{type=figure}
        \begin{tikzpicture}
            \GraphInit[vstyle=Normal]
            \SetGraphUnit{2}
            \begin{scope}[rotate = 45]
                \Vertices{circle}{[0], [1], [2], [3]}
            \end{scope}
            \Edges([0], [1], [2], [3], [0])
        \end{tikzpicture}
        \captionof{graph}{Cayley graph of $\mathbb{Z}/4\mathbb{Z}$ with respect to $\{\pm [1]\}$} \label{fig:cayleyz4z}
    \end{center}
    and the adjacency matrix as
    \begin{equation*} 
        A = \begin{bmatrix}
            & 1 & & 1\\
             1 &&  1&\\
            & 1 & & 1\\
             1 &&  1&\\
        \end{bmatrix}.
    \end{equation*}
\end{ex}  

\begin{ex}
    If we take $G = \mathbb{Z}/6\mathbb{Z}$ and $S = \{\pm [1], \pm [2]\},$ we get the graph as
    \begin{center}
        \captionsetup{type=figure}
        \begin{tikzpicture}
            \GraphInit[vstyle=Normal]
            \SetGraphUnit{3}
            \begin{scope}%[rotate = 45]
                \Vertices{circle}{[0], [1], [2], [3], [4], [5]}
            \end{scope}
            \Edges([0], [1], [2], [3], [4], [5], [0])
            \Edges([0], [2], [4], [0])
            \Edges([1], [3], [5], [1])
        \end{tikzpicture}
        \captionof{graph}{Cayley graph of $\mathbb{Z}/6\mathbb{Z}$ with respect to $\{\pm [1], \pm [2]\}$} \label{fig:cayleyz6z}
    \end{center}

    and the adjacency matrix as

    \begin{equation*} 
        A = \begin{bmatrix}
              & 1 & 1 &   & 1 & 1 \\
            1 &   & 1 & 1 &   & 1 \\
            1 & 1 &   & 1 & 1 &   \\
              & 1 & 1 &   & 1 & 1 \\
            1 &   & 1 & 1 &   & 1 \\
            1 & 1 &   & 1 & 1 &   \\
        \end{bmatrix}
    \end{equation*}
\end{ex}

\begin{defn}%[Circulant graph]
    A Cayley graph of $\mathbb{Z}/n\mathbb{Z}$ is called a \deff{circulant graph} (on $n$ vertices).
\end{defn}

\begin{defn}%[Circulant matrix]
    An $n \times n$ \deff{circulant matrix} is a matrix of the form
    \begin{equation*} 
        A = \begin{bmatrix}
            a_0 & a_1 & \cdots & a_{n - 2} & a_{n - 1} \\
            a_{n - 1} & a_0 & \cdots & a_{n - 3} & a_{n - 2}\\
            \vdots & \vdots & \ddots & \vdots & \vdots \\
            a_2 & a_3 & \cdots & a_{0} & a_1\\
            a_1 & a_2 & \cdots & a_{n - 1} & a_0
        \end{bmatrix}.
    \end{equation*}
    Equivalently, a matrix $A = (a_{ij})$ is circulant if there exists a function $f : \mathbb{Z}/n\mathbb{Z} \to \mathbb{C}$ such that $a_{ij} = f([j] - [i]).$
\end{defn} 

\begin{ex}
    If $S$ is a symmetric subset of $G \vcentcolon= \mathbb{Z}/n\mathbb{Z},$ then the indicator function $\delta_S : G \to \mathbb{C}$ is defined as
    \begin{equation*} 
        \delta_S(x) = \begin{cases}
            1 & x \in S,\\
            0 & x \notin S. 
        \end{cases}
    \end{equation*} 
    Consider the matrix $A = (a_{ij})$ given by $a_{ij} = f([j] - [i]).$ Then, note that 
    \begin{equation*} 
        a_{ij} = 1 \iff [j] - [i] \in S \iff \{[i], [j]\} \in E,
    \end{equation*}
    where $E$ denotes the edge set of the Cayley graph of $G$ with respect to $S.$

    Thus, we see that the circulant matrix corresponding to $\delta_S$ is the adjacency matrix of the Cayley graph of $G$ with respect to $S.$
\end{ex}

\begin{lem} \label{lem:convolvediagonal}
    Let $G$ be an abelian group and $a \in L(G).$ Define $A : L(G) \to L(G)$ by $A(b) = a * b.$ Then, $A$ is linear and $\chi$ is an eigenvector of $A$ with eigenvalue $\widehat{a}(\chi)$ for all $\chi \in \widehat{G}.$ Consequently, $A$ is diagonalisable.
\end{lem}
\begin{proof} 
    That $A$ is additive follows from the fact that $*$ distributes over $+.$ That $A(\alpha b) = \alpha A(b)$ for $\alpha \in \mathbb{C}$ and $b \in L(G)$ follows easily from the definition of $*.$ Thus, $A$ is linear.

    Put $n \vcentcolon= \md{G}.$ Now, let $\chi \in \widehat{G}$ be arbitrary. Then,
    \begin{equation*} 
        \widehat{a * \chi} = \widehat{a} \cdot \widehat{\chi} = \widehat{a} \cdot n \delta_\chi,
    \end{equation*}
    where the last equality follows from \Cref{ex:fourtranschar}. Now, for $\theta \in \widehat{G},$ we note that
    \begin{equation*} 
        (\widehat{a} \cdot n \delta_\chi)(\theta) = \begin{cases}
            n\widehat{a}(\chi) & \theta = \chi,\\
            0 & \theta \neq \chi.   
        \end{cases}
    \end{equation*}
    Thus, we have $\widehat{a * \chi} = \widehat{a} \cdot n \delta_\chi = n \widehat{a}(\chi) \delta_\chi.$ By \Cref{ex:fourtranschar} again, we get
    \begin{equation*} 
        \widehat{a * \chi} = n \widehat{a}(\chi) \widehat{\chi}.
    \end{equation*}
    Applying the inverse Fourier transform (and using its linearity), we get
    \begin{equation*} 
        a * \chi = \widehat{a}(\chi) \chi.
    \end{equation*}
    (Note that $\widehat{a}(\chi) \in \mathbb{C}$ is a constant.) The above equation is simply
    \begin{equation*} 
        A(\chi) = \widehat{a}(\chi)\chi
    \end{equation*}
    and hence, $\chi$ is an eigenvector with eigenvalue $\widehat{a}(\chi).$

    Note that the irreducible characters of $G$ form a basis of $Z(L(G)),$ by \Cref{thm:onbforzlg}. Since $G$ is abelian, the irreducible characters are precisely the elements of $\widehat{G}$ and $Z(L(G)) = L(G).$ Thus, we see that $\widehat{G}$ forms a basis of $L(G)$ and have just shown that all these elements are eigenvectors of $A.$ Thus, $\widehat{G}$ is an eigenbasis of $L(G)$ corresponding and $A$ is diagonalisable.
\end{proof}

\begin{thm} \label{thm:adjcayleyeigen}
    Let $G = (g_1, \ldots, g_n)$ be an ordered abelian group and let $S \subset G$ be a symmetric subset. Let $\chi_1, \ldots, \chi_n$ be irreducible characters of $G$ and let $A$ be the adjacency matrix of the Cayley graph of $G$ with respect to $S.$ Then,
    \begin{enumerate}
        \item The eigenvalues of $A$ are the real numbers
        \begin{equation*} 
            \lambda_i \vcentcolon= \sum_{s \in S} \chi_i(s)
        \end{equation*}
        for $1 \le i \le n;$
        \item A corresponding orthonormal basis of eigenvectors is given by the vectors $(v_1, \ldots, v_n)$ where
        \begin{equation*} 
            v_i \vcentcolon= \frac{1}{\sqrt{\md{G}}} \begin{bmatrix}
                \chi_i(g_1) & \cdots & \chi_i(g_n)
            \end{bmatrix}^\mathsf{T}.
        \end{equation*}
    \end{enumerate}
\end{thm}

\begin{proof} 
    The indicator function $\delta_S$ of $S$ is given by $\delta_S = \sum_{s \in S} \delta_s.$ Note that $\delta_S \in L(G).$ Define $F : L(G) \to L(G)$ by
    \begin{equation*} 
        F(b) = \delta_S * b.
    \end{equation*}
    Then, by \Cref{lem:convolvediagonal}, $F$ has eigenvectors $\chi_i$ with corresponding eigenvalue
    \[\begin{WithArrows}[displaystyle]
        \widehat{\delta_S}(\chi_i) = n\langle \delta_S, \chi_i\rangle &= \sum_{x \in G} \delta_S(x)\overline{\chi_i(x)}\\
        &= \sum_{x \in S} \overline{\chi_i(x)} \Arrow{\Cref{rem:findegoneunitary}}\\
        &= \sum_{x \in S} \chi_i(x^{-1}) \Arrow{$S$ is symmetric}\\
        &= \sum_{y \in S} \chi_i(y)\\
        &= \lambda_i.
    \end{WithArrows}\]
    This proves that $\lambda_i$ is an eigenvalue for each $1 \le i \le n.$ (Note that $\lambda_i \in \mathbb{R}$ follows since $A$ is symmetric.\footnote{Alternately, one can note that that sum in the definition of $\lambda_i$ is real. Indeed, if $s = s^{-1}$, then $\chi(s) = \chi(s^{-1}) = \overline{\chi(s)} \in \mathbb{R}.$ On the other hand, if $s \neq s^{-1},$ then pairing up $\chi(s)$ and $\chi(s^{-1})$ in the sum gives $\chi(s) + \chi(s^{-1}) = \chi(s) + \overline{\chi(s)} \in \mathbb{R}.$})

    Consider the ordered basis $B = (\delta_{g_1}, \ldots, \delta_{g_n})$ for $L(G).$ Let $[F]_B$ denote the matrix of $F$ with respect to this ordered basis. Note that the coordinate vector of $\chi_i$ with respect to $B$ is precisely $\sqrt{\md{G}}v_i.$ The above shows that it is an eigenvector with eigenvalue $\lambda_i.$ The orthogonality of $v_i$ follows from \Cref{thm:secondorthorel}. (Note that the conjugacy classes are singletons since $G$ is abelian.) \\
    Lastly, the orthonormality follows from noting that
    \begin{equation*} 
        \|v_i\|^2 = \frac{1}{G}\left(\md{\chi_i(g_1)}^2 + \cdots + \md{\chi_i(g_n)}^2\right) = \|\chi_i\|^2 = 1.
    \end{equation*}
    Thus, to complete the proof, it suffices to show that $A = [F]_B.$

    Let $1 \le i, j \le n$ be arbitrary. We show that $A_{ij} = ([F]_B)_{ij}.$ Note that $([F]_B)_{ij}$ is the coefficient of $\delta_{g_i}$ in $F(\delta_{g_j}).$ We note that
    \begin{equation*} 
        F(\delta_{g_j}) = \delta_S * \delta_{g_j} = \sum_{s \in S}\delta_s * \delta_{g_j} = \sum_{s \in S} \delta_{sg_j} = \sum_{g \in Sg_j} \delta_g.
    \end{equation*}
    (We have used $\delta_{gh} = \delta_g * \delta_h,$ by \Cref{prop:convdeltaprod}.)

    Thus, 
    \begin{align*} 
        ([F]_B)_{ij} &= \begin{cases}
            1 & g_i \in Sg_j,\\
            0 & g_i \notin Sg_j,
        \end{cases}\\
        &= \begin{cases}
            1 & g_ig_j^{-1} \in S,\\
            0 & g_ig_j^{-1} \notin S,
        \end{cases}\\
        &= A_{ij}. \qedhere
    \end{align*}
\end{proof}

\begin{cor}
    Let $A$ be a circulant matrix of degree $n,$ which is the adjacency matrix of the Cayley graph of $\mathbb{Z}/n\mathbb{Z}$ with respect to some symmetric subset $S \subset \mathbb{Z}/n\mathbb{Z}.$ Then, the eigenvalues of $A$ are
    \begin{equation*} 
        \lambda_k \vcentcolon= \sum_{[m] \in S} \omega_n^{km},
    \end{equation*}
    for $k = 0, \ldots, n - 1$ with corresponding basis of orthonormal eigenvectors given by
    \begin{equation*} 
        v_k \vcentcolon= \frac{1}{\sqrt{n}}\begin{bmatrix}
            1 & \omega_n^{k} & \omega_n^{2k} & \cdots & \omega_n^{(n - 1)k}
        \end{bmatrix}^\mathsf{T}.
    \end{equation*}
\end{cor}
\begin{proof} 
    The result follows from \Cref{thm:adjcayleyeigen} since we have
    \begin{equation*} 
        \chi_k([m]) = \omega_n^{km}
    \end{equation*}
    for $k = 0, \ldots, n - 1.$ 
\end{proof}

\begin{ex}
    Recall \Cref{fig:cayleyz6z}. We had $n = 6$ and $S = \{\pm [1], \pm [2]\}.$ Thus, the eigenvalues of $A$ are given by
    \begin{equation*} 
        \lambda_k = \omega_6^k + \omega_6^{-k} + \omega_6^{2k} + \omega_6^{-2k} = 2\left(\cos\left(\frac{\pi k}{3}\right) + \cos\left(\frac{2\pi k}{3}\right)\right)
    \end{equation*}
    for $k = 0, \ldots, 5.$
\end{ex}

\subsection{Fourier Analysis on Non-abelian Groups}

If $G$ is a non-abelian group, then $L(G) \neq Z(L(G)).$ Note however that that pointwise product of functions into $\mathbb{C}$ is commutative. Thus, we cannot have a Fourier transform converting convolution into pointwise product while being an isomorphism. To remedy this, we look at matrix multiplication instead of pointwise.

Before that, we look at the case of abelian groups in a different light. First, recall that $\mathbb{C}^n$ is a ring with product given as
\begin{equation*} 
    (w_1, \ldots, w_n) \cdot (z_1, \ldots, z_n) \vcentcolon= (w_1z_1, \ldots, w_nz_n).
\end{equation*}

\begin{prop}
    Let $G$ be a finite abelian group with irreducible characters $\chi_1, \ldots, \chi_n.$ Define $T : L(G) \to \mathbb{C}^n$ by
    \begin{equation*} 
        Tf = (\widehat{f}(\chi_1), \ldots, \widehat{f}(\chi_n)).
    \end{equation*}
    Then, $T$ is an isomorphism of rings.
\end{prop}
\begin{proof} 
    Note that if $f, g \in L(G)$ are such that $Tf = Tg,$ then $\widehat{f} = \widehat{g}.$ Fourier inversion gives $f = g.$ Thus, $T$ is injective. Since $T$ is $\mathbb{C}$-linear (same proof as \Cref{prop:fourierisovspace}) and $\dim_{\mathbb{C}}(L(G)) = n = \dim_{\mathbb{C}}(\mathbb{C}^n),$ it follows that $T$ is a bijection and hence, an isomorphism of $\mathbb{C}$-vector spaces.

    To show that is an isomorphism of rings, all that remains is to show that $T(f * g) = Tf \cdot Tg.$ This follows directly from the fact that $\widehat{f * g}(\chi_i) = \widehat{f}(\chi_i)\widehat{g}(\chi_i)$ for all $i = 1, \ldots, n.$
\end{proof}

Thus, \Cref{thm:fourierisorings} can be stated as follows.
\begin{thm}
    Let $G$ be a finite abelian group of order $n.$ Then, $L(G) \cong \mathbb{C}^n$ as rings.
\end{thm}

Note that all the irreducible representations of $G$ have degree one. The above product can be seen as $\mathbb{C}^n \cong M_1(\mathbb{C}) \times \cdots \times M_1(\mathbb{C}).$ In general, we replace the $1$s with the degrees of the irreducible representations.

\begin{aside}
    \textbf{Setup.}

    $G$ is a finite group of order $n$ and $\varphi^{(1)}, \ldots, \varphi^{(s)}$ is a transversal of irreducible unitary representations of $G.$ Put $d_k = \deg \varphi^{(k)}.$ 

    For $1 \le k \le s$ and $1 \le i, j \le d_k,$ we have the functions $\varphi^{(k)}_{ij} : G \to \mathbb{C}$ such that the matrix $\varphi^{(k)}(g)$ is given as $(\varphi^{(k)}_{ij}(g))$ for all $g \in G.$

    Let $D = \{(i, j, k) : 1 \le k \le s,\; 1 \le i, j \le d_k\}.$

    We recall \Cref{cor:orthnormalbasis} which told us that $B = \left\{\sqrt{d_k}\varphi_{ij}^{(k)} \mid (i, j, k) \in D\right\}$ is an orthonormal basis for $L(G).$
\end{aside}

\begin{defn}%[]
    Define
    \begin{equation*} 
        \mathcal{F} : L(G) \to M_{d_1}(\mathbb{C}) \times \cdots \times M_{d_s}(\mathbb{C})
    \end{equation*}
    by $\mathcal{F}(f) \vcentcolon= (\widehat{f}(\varphi^{(1)}), \ldots, \widehat{f}(\varphi^{(s)}))$ where
    \begin{equation*}
        \widehat{f}(\varphi^{(k)}) \vcentcolon= \sum_{g \in G} f(g)\overline{\varphi^{(k)}_g}.
    \end{equation*}
    We call $\mathcal{F}(f)$ the \deff{Fourier transform} of $f.$
\end{defn}
\begin{rem}
    In terms of the matrix entries, the above can be rewritten as
    \begin{equation} \label{eq:012}
        \widehat{f}(\varphi^{(k)})_{ij} \vcentcolon= \sum_{g \in G} f(g)\overline{\varphi^{(k)}_{ij}(g)} = n\langle f, \varphi^{(k)}_{ij}\rangle.
    \end{equation}
\end{rem}

\begin{rem} \label{rem:dimLGdimmatrix}
    Note that $L(G)$ is a $\mathbb{C}$-vector space of dimension $\md{G}.$ On the other hand, $M_{d_i}(\mathbb{C})$ is a $\mathbb{C}$-vector space with dimension $d_i^2.$ Thus, the product $M_{d_1}(\mathbb{C}) \times \cdots \times M_{d_s}(\mathbb{C})$ is a vector space of dimension $d_1^2 + \cdots + d_s^2 = \md{G}.$

    Thus, we see that $L(G)$ and $M_{d_1}(\mathbb{C}) \times \cdots \times M_{d_s}(\mathbb{C})$ have the same dimension. We shall show that $\mathcal{F}$ is an isomorphism.
\end{rem}

\begin{thm}[Fourier inversion] \label{thm:fourierinvgen}
    Let $f \in L(G).$ Then,
    \begin{equation*} 
        f = \frac{1}{n}\sum_{(i, j, k) \in D}d_k \widehat{f}(\varphi^{(k)})_{ij} \varphi^{(k)}_{ij}.    
    \end{equation*}
    In particular, $\mathcal{F}$ is injective.
\end{thm}
\begin{proof} 
    The proof is as before. We know that $B$ is an orthonormal basis. Thus, it suffices to prove that
    \begin{equation*} 
        \langle f, \sqrt{d_k}\varphi^{(k)}_{ij}\rangle = \frac{1}{n}\sqrt{d_k} \widehat{f}(\varphi^{(k)})_{ij}
    \end{equation*}
    for all $(i, j, k) \in D.$ However, the above is precisely \Cref{eq:012}.
\end{proof}

\begin{thm}
    The Fourier transform $\mathcal{F} : L(G) \to M_{d_1}(\mathbb{C}) \times \cdots \times M_{d_s}(\mathbb{C})$ is an isomorphism of vector spaces.
\end{thm}
\begin{proof} 
    As usual, the check that it is linear is simple and essentially follows from the fact that $\langle \cdot , \cdot \rangle$ is linear in the first variable. \\
    By \Cref{thm:fourierinvgen}, we know that $\mathcal{F}$ is injective. \\
    By \Cref{rem:dimLGdimmatrix}, we know that $\dim_{\mathbb{C}}(L(G)) = \dim_{\mathbb{C}}(M_{d_1}(\mathbb{C}) \times \cdots \times M_{d_s}(\mathbb{C}))$ and thus, $\mathcal{F}$ is an isomorphism.
\end{proof}

Note that $M_{d_1}(\mathbb{C}) \times \cdots \times M_{d_s}(\mathbb{C})$ is a ring as well, with coordinate-wise product.

\begin{thm}[Wedderburn]
    The Fourier transform $\mathcal{F} : L(G) \to M_{d_1}(\mathbb{C}) \times \cdots \times M_{d_s}(\mathbb{C})$ is an isomorphism of rings.
\end{thm}
\begin{proof} 
    All that is required to be proven is that $T(f * g) = Tf \cdot Tg,$ where the latter product is in the ring $M_{d_1}(\mathbb{C}) \times \cdots \times M_{d_s}(\mathbb{C}).$ Let $a, b \in L(G).$

    Since the latter product is coordinate-wise, it suffices to show that
    \begin{equation*} 
        \widehat{(a * b)}(\varphi^{(k)}) = \widehat{a}(\varphi^{(k)})\widehat{b}(\varphi^{(k)})
    \end{equation*}
    for all $1 \le k \le s.$ (The product on the right is matrix multiplication.)

    The computation is the same as before.
    \begin{align*} 
        \widehat{(a * b)}(\varphi^{(k)}) &= \sum_{g \in G} (a * b)(g)\overline{\varphi^{(k)}(g)}\\
        &= \sum_{g \in G}\left(\sum_{h \in G}a(gh^{-1})b(h)\right)\overline{\varphi^{(k)}(g)}\\
        &= \sum_{h \in G}b(h)\sum_{g \in G}a(gh^{-1})\overline{\varphi^{(k)}(g)}\\
        &= \sum_{h \in G}b(h)\sum_{g' \in G}a(g')\overline{\varphi^{(k)}(g'h)}\\
        &= \sum_{h \in G}b(h)\sum_{g' \in G}a(g')\overline{\varphi^{(k)}(g')}\cdot\overline{\varphi^{(k)}(h)}\\
        &= \sum_{g' \in G}a(g')\overline{\varphi^{(k)}(g')}\sum_{h \in G}b(h)\overline{\varphi^{(k)}(h)}\\
        &= \widehat{a}(\varphi^{(k)})\widehat{b}(\varphi^{(k)}). \qedhere
    \end{align*}
\end{proof}
In the above computation, note that $a$ and $b$ took values in $\mathbb{C}$ and so, commuted with the other terms.

\newpage
\renewcommand*{\listtheoremname}{List of Examples}
\listoftheorems[ignoreall,show={fakeex}]
\end{document}	