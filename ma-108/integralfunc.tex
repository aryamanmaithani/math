\sec{Some integral functions}
Here we study some \emph{integral}\footnote{pun intended} functions that will help us later.\\
The reader may skip this section for now and return to this when referenced later.\\
We skip the formal discussion about convergence of improper Riemann integrals. (These are different from the integrals done in MA 105.) A discussion can be found in Professor GKS' notes mentioned.
\subsection{The Gamma Function}
\begin{defn}[The Gamma Function]
	The Gamma function is defined for $a > 0$ as follows:
	\begin{equation*} 
		\Gamma(a) := \int_{0}^{\infty} t^{a-1}e^{-t} dt.
	\end{equation*}
\end{defn}
\begin{mdframed}[style=boxstyle, frametitle={Some results}]
	\begin{enumerate}[leftmargin=*]
		\item $\Gamma(n) = (n - 1)!$ for all $n \in \mathbb{Z}^+ = \{1, 2, \ldots\}.$
		\item $\Gamma(x+1) = x\Gamma(x)$ for all $x > 0.$
		\item $\Gamma\left(\dfrac{1}{2}\right) = \sqrt{\pi}.$
		\item The last two relations now let us calculate $\Gamma$ for all half-integers.
		\item And that's as far as we can get. Other values of Gamma are not known in such elementary terms. 
		\item Note that we can still calculate ratios such as
		\begin{equation*} 
			\Gamma\left(\dfrac{5}{4}\right)\bigg/\Gamma\left(\dfrac{1}{4}\right).
		\end{equation*}
	\end{enumerate}
\end{mdframed}

\subsection{The Beta Function}
\begin{defn}[The Beta Function]
	This is a function of two variables defined for $a > 0, b > 0$ as 
	\begin{equation*} 
		B(a, b) := \int_{0}^{1} x^{a-1}(1 - x)^{b-1} dx
	\end{equation*}
\end{defn}
\begin{mdframed}[style=boxstyle, frametitle={Some identities}]
	\begin{enumerate}[leftmargin=*]
		\item $B(a, b) = B(b, a).$
		\item $B(a + 1, b) = \dfrac{a}{b}B(a, b+1).$
		\item $B(a + 1, b) + B(a, b+1) = B(a, b).$
		\item $B\left(\dfrac{1}{2}, \dfrac{1}{2}\right) = \pi.$
	\end{enumerate}
\end{mdframed}
\begin{thm}[The Beta-Gamma Relation] 
	\begin{equation*} 
		\Gamma(a)\Gamma(b) = \Gamma(a + b)B(a, b) \quad \forall\; a, b > 0.
	\end{equation*}
\end{thm}
\begin{thm}[Euler's Reflection Formula]
	\begin{equation*} 
		\Gamma(a)\Gamma(1 - a) = \dfrac{\pi}{\sin(\pi a)} \quad \forall\;0 < a < 1
	\end{equation*}
\end{thm}

\begin{mdframed}[style=boxstyle2, frametitle={An integral computation}]
	\begin{thm} \label{thm:randomintegral}
		\begin{equation*} 
			\int_{0}^{\infty} \dfrac{x^{a-1}}{1 + x} dx = \dfrac{\pi}{\sin \pi a} \quad \text{ for } 0 < a < 1
		\end{equation*}
	\end{thm}
	\begin{proof} 
		Put $t = \dfrac{1}{1 + x}.$ The integral transforms as
		\begin{align*} 
			& \int_{0}^{1} (1 - t)^{a - 1}t^{-a} dt\\
			& = B(a, 1 - a)\\
			& = \Gamma(a)\Gamma(1 - a)\\
			& = \dfrac{\pi}{\sin \pi a}.
		\end{align*}
	\end{proof}
\end{mdframed}

