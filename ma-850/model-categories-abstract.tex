\documentclass[12pt]{article}
\usepackage[lmargin=1in,rmargin=1in]{geometry}

\title{Model Categories}
\author{Aryaman Maithani}
\date{April 7, 2022}

\usepackage{euler}
\usepackage{mathpazo}

\usepackage{amsmath, amssymb, amsfonts, amsthm, mathtools}
\usepackage[colorlinks=true]{hyperref}

\begin{document}

\maketitle

This talk is an introduction to the theory of \emph{model categories}, which was introduced by Quillen in 1967. We present the first few sections from the paper \emph{Homotopy theories and model categories} by W. G. Dwyer and J. Spalinski. By definition, a model category is just an ordinary category with three specified classes of morphisms, called fibrations, cofibrations and weak equivalences, which satisfy a few simple axioms that are deliberately reminiscent of properties of topological spaces. Our aim is to use these axioms and construct the basic machinery of homotopy theory.

The talk is broken into three sections. In the first section, we define what a model category is and look at some basic examples and ways to construct more model categories (including the opposite category). The axioms being self-dual also mean that half the results follow by duality. Two of the concrete examples are topological, and is one algebraic. After defining what (co)fibrant objects are, we see that having a nondegenerate basepoint is for a space what being projective is for a chain complex. 

In the second section, we introduce the notion of cylinder and path objects as well as \emph{left} and \emph{right} homotopy. The notations and definitions are again influenced by what one sees in topology. Unlike the previous section, we shall see some proofs in this one. We also see how the two homotopies are well-behaved under composition under fibrant/cofibrant assumptions. Moreover, these coincide is the good situation -- when the domain is cofibrant and the codomain is fibrant. This leads to the definition of two paths being \emph{homotopic}. We end this section by stating and proving the main result -- a map in the good situation is a weak equivalence if and only if it has a homotopy inverse.

In the final section, we define what the homotopy category is and wrap things up. We describe some properties of the homotopy category and define what localisation of a category is. We end with a statement connecting the homotopy category of chain complexes with the $\operatorname{Ext}$ functor.

\end{document}