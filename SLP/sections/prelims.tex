\section{Preliminaries}

\subsection{Conventions, notations, basic definitions}

\begin{enumerate}
	\item $\mathbb{N}$ denotes the set of positive integers, $\mathbb{N}_{0}$ denotes the set of nonnegative integers. $\mathbb{Z}, \mathbb{Q}, \mathbb{R}, \mathbb{C}$ have their usual meanings.
	%
	\item For $n \in \mathbb{N}$, $[n]$ denotes the set $\{1, \ldots, n\}$.
	%
	\item Rings will always be commutative and with multiplicative identity. $R$ will denote a ring. Modules will be unital and typically denoted by $N, M$. We will usually consider the ring as acting on the left. \newline
	If $a \in R$, then $(a)$ denotes the ideal generated by $R$ and the notation extends to more general cases in the obvious manner.
	%
	\item $\mathcal{J}(R)$ denotes the Jacobson radical, $\mathcal{N}(R)$ denotes the nilradical.
	%
	\item A \deff{local ring} is a ring $R$ with a unique maximal ideal $\mathfrak{m}$. We shall denote this by $(R, \mathfrak{m})$ or even $(R, \mathfrak{m}, \kk)$, where $\kk \vcentcolon= R/\mathfrak{m}$ is the \deff{residue field}.
	%
	\item Categories will always be typically denoted by $\C{A}$, $\C{B}$, et cetera. Categories will always be locally small, i.e., given two objects $A$ and $B$ in $\C{A}$, the collection of morphisms from $A$ to $B$ is a set. \newline
	We shall use the notation $A \in \C{A}$ to mean that $A$ is an object of $\C{A}$. \newline
	Given $A$, $B \in \C{A}$, we use $\Hom_{\C{A}}(A, B)$ to denote the set of morphisms from $A$ to $B$.
	%
	\item $N \le M$ indicates that $N$ is a submodule of $M$. \newline
	$N \mid M$ indicates that $N$ is a direct summand of $M$, i.e., $M \cong N \oplus N'$ for some $N'$.
	%
	\item $I \unlhd R$ indicates that $I$ is an ideal of $R$.
	%
	\item A subset $S \subset R$ is said to be \deff{multiplicative} if $S$ satisfies the following:
	\begin{enumerate}
		\item $1 \in S$,
		% \item $0 \notin S$,
		\item $s, t \in S \Rightarrow st \in S$.
	\end{enumerate}
	Localisation of $M$ with respect to $S$ is denoted by $S^{-1}M$. 

	Special cases: 
	\begin{enumerate}
		\item $S = \{1, a, a^{2}, \ldots\}$ for some $a \in R$. In this case, $S^{-1}M = M_{a}$.
		\item $S = R \setminus \mathfrak{p}$ for some prime $\mathfrak{p} \unlhd R$. In this case, $S^{-1}M = M_{\mathfrak{p}}$.
		\item $R$ is an integral domain and $S = R \setminus \{0\}$. In this case, $S^{-1}R = \Frac(R)$. This is the \deff{field of fractions} of $R$.
	\end{enumerate}
	%
	\item $\lmod{R}$ denotes the category of (left) $R$-modules. On rare occasions, we will use $\rmod{R}$ to indicate that we are considering the modules as right $R$-modules and write the multiplication by $R$ on the right. However, note that our rings will always be commutative.
	%
	\item $\Ch{R}$ denotes the category of chain complexes of $R$-modules.
	%
	\item An \deff{$R$-algebra} $S$ is a ring along with a ring homomorphism $f : R \to S$. \newline
	In this case, every module $M$ over $S$ can be considered an $R$-module via $r \cdot x = f(r) \cdot x$ for $r \in R$ and $x \in M$. (In particular, $S$ itself is an $R$-module.)
	%
	\item $a \in R$ is said to be a \deff{zerodivisor} if there exists $b \in R \setminus \{0\}$ such that $ab = 0$. \\
	$\mathcal{Z}(R)$ denotes the set of zerodivisors of $R$. 

	More generally, given an $R$-module $M$, $a \in R$ is said to be a \deff{zerodivisor on $M$} if there exists $x \in M \setminus \{0\}$ such that $a \cdot x = 0$. \newline
	$\mathcal{Z}(R)$ denotes the set of zerodivisors on $M$. 	
	%
	\item Given submodules $K, L$ of $M$, we define
	\begin{equation*} 
		(L :_{R} N) \vcentcolon= \{r \in R : rN \subset L\}.
	\end{equation*}
	The above is an ideal of $R$. 

	In particular, the annihilator is defined as
	\begin{equation*} 
		\ann_{R}(N) \vcentcolon= (0 :_{R} N) = \{r \in R : rN = 0\}.
	\end{equation*}
	\item $\Spec(R)$ denotes the set of prime ideals of $R$.
\end{enumerate}

\subsection{Localisation}

We assume basic familiarity with the notion of localisation. We start by making some elementary observations. Unless otherwise stated, $S$ will denote a multiplicative subset of a ring $R$.

\begin{obs} \label{obs:R-linear-map-between-localised-modules}
	Let $M$ and $N$ be $S^{-1}R$-modules. If $f : M \to N$ is an $R$-linear map, then $f$ is also $S^{-1}R$-linear.

	Indeed, note that
	\begin{equation*} 
		s f\left(\frac{r}{s} m\right) = f\left(s \frac{r}{s} m\right) = f(rm) = r f(m)
	\end{equation*}
	and thus, $f(\frac{r}{s} m) = \frac{r}{s} f(m)$.
\end{obs}

\begin{exe} \label{exe:direct-summand-localised-modules}
	If $M \le E$ are $S^{-1}R$-modules such that $M$ is a direct summand of $E$ as an $R$-module, then $M$ is a direct summand of $E$ as an $S^{-1}R$-module. 
\end{exe}
Here are two outlines to do the above: The first is to write $E = M \oplus N$ as an internal direct sum of $R$-modules and use the fact that $E$ and $M$ are $S^{-1}R$-modules to show that $N$ is also an $S^{-1}R$-module. \newline
Another is to use the fact that $M$ being a direct summand means that there exists and $R$-linear map $p : E \to M$ which restricts to the identity map on $M$. Then use \Cref{obs:R-linear-map-between-localised-modules}.

Let $S \subset R$ be a multiplicative subset, and $M, N$ be $R$-modules. There is a map 
\begin{align*} 
	\Phi : S^{-1}\Hom_{R}(M, N) &\to \Hom_{S^{-1}R}(S^{-1}M, S^{-1}N), \\
	\frac{\varphi}{s} &\mapsto \left(\frac{a}{s'} \mapsto \frac{\varphi(a)}{ss'} \right)
\end{align*}
of $S^{-1}R$-modules. In general, $\Phi$ is not an isomorphism. However, in the special case that $M = R$, we note that the modules are both isomorphic to $S^{-1}N$ and under this identification, $\Phi$ is simply the identity map. This gives us the following.

\begin{prop} \label{prop:Hom-localise-first-coordinate-R}
	Let $S \subset R$ be a multiplicative subset, and $N$ be an $R$-module. Then,
	\begin{equation*} 
		S^{-1}\Hom_{R}(R, N) \cong \Hom_{S^{-1}R}(S^{-1}R, S^{-1}N)
	\end{equation*}
	as $S^{-1}R$-modules.
\end{prop}

Recall that $M$ is said to be \deff{finitely presented} if there exists an exact sequence of the form $R^{m} \to R^{n} \to M \to 0$ for finite natural numbers $n, m \ge 1$. 

Note that if $R$ is Noetherian and $M$ is finitely generated, then we can find a a surjection $R^{n} \onto M$. As $R$ is Noetherian, the kernel will be finitely generated and we can map $R^{m}$ onto it. Thus, a finitely generated module over a Noetherian ring is finitely presented.

\begin{prop} \label{prop:finitely-presented-Hom-localise}
	If $M$ is a finitely presented $R$-module, then for every multiplicative set $S \subset R$ and $R$-module $N$, $\Phi$ is an isomorphism of $S^{-1}R$-modules:
	\begin{equation*} 
		\Phi : S^{-1}\Hom_{R}(M, N) \cong \Hom_{S^{-1}R}(S^{-1}M, S^{-1}N).
	\end{equation*}
	In particular, the above isomorphism holds if $R$ is Noetherian and $M$ is finitely generated.
\end{prop}

\begin{proof} 
	As noted, $\Phi$ is an isomorphism when $M = R$; additivity of $\Hom$ ensures that the above is true when $M = R^{n}$. For the general case, finite presentation of $M$ gives us the usual sequence 
	\begin{equation*} \label{eq:04}
		R^{m} \to R^{n} \to M \to 0.
	\end{equation*} 
	The result follows by the use of the Five Lemma on the following commutative diagram:
	\begin{equation*} 
		\begin{tikzcd}[column sep=3mm]
			0 \arrow[r] & S^{-1}\Hom_{R}(M, N) \arrow[d, "\Phi"] \arrow[r] & S^{-1}\Hom_{R}(R^{n}, N) \arrow[r] \arrow[d, "\cong"] & S^{-1}\Hom_{R}(R^{m}, N) \arrow[d, "\cong"] \\
			0 \arrow[r] & \Hom_{S^{-1}R}(S^{-1}M, S^{-1}N) \arrow[r] & \Hom_{S^{-1}R}(S^{-1}R^{n}, S^{-1}N) \arrow[r] & \Hom_{S^{-1}R}(S^{-1}R^{m}, S^{-1}N)
		\end{tikzcd}
	\end{equation*}
	The first row above is exact since it is obtained by first applying $\Hom_{R}(-, N)$ (which is left exact) and then applying $S^{-1}(-)$ (which is exact) to \Cref{eq:04}. \newline
	The bottom row is similarly obtained by applying the functors in the opposite order.
\end{proof}

\subsection{Associated Primes of a Module} \label{subsec:associated-primes}

Throughout this section, $R$ will denote a ring, and $M$ an $R$-module. For $a \in R$, we denote by $a_{M}$ the $R$-endomorphism of $M$ given by $x \mapsto ax$.\footnote{The advantage of using this notation over $\mu_{a}$ is that the module is also clear.} We shall be as judicious as possible and state results involving ``Noetherian'' and ``finitely generated'' hypotheses only when necessary. Of course, our main interest eventually is in Noetherian rings, but many of the modules seen later will be non finitely generated.

\begin{defn}
	A proper submodule $Q \lneq M$ is said to be primary if for every $a \in R$, the map $a_{M/Q}$ is either injective or nilpotent.
\end{defn}

Note that $a_{M/Q}$ cannot be both injective and nilpotent since $M/Q \neq 0$ by hypothesis.

If $Q$ is primary, then we define the ideal
\begin{equation*} 
	\mathfrak{p} \vcentcolon= \{a \in R : a_{M/Q} \text{ is nilpotent}\}.
\end{equation*}
Note that $R \setminus \mathfrak{p}$ is closed under multiplication since the composition of two injective maps is injective and hence, not nilpotent. Thus, $\mathfrak{p}$ is a prime ideal. We call $\mathfrak{p}$ the \deff{prime belonging to $Q$}, and we say that $Q$ is $\mathfrak{p}$-primary.

By considering $M = R$, we get the notion of a \deff{primary ideal}: $\mathfrak{q} \subsetneq R$ is a primary ideal if $ab \in \mathfrak{q}$ $\Rightarrow$ $a \in \mathfrak{q}$ or $b^{n} \in \mathfrak{q}$ for some $n \ge 1$. In this case, the prime belonging to $\mathfrak{q}$ is simply $\sqrt{\mathfrak{q}}$.

\begin{prop}
	Let $\mathfrak{a} \unlhd R$ be a proper ideal. Let $\mathfrak{m}$ be a maximal ideal such that $\mathfrak{m}^{n} \subset \mathfrak{a}$ for some $n \ge 1$. Then, $\mathfrak{a}$ is an $\mathfrak{m}$-primary ideal.
\end{prop}
\begin{proof} 
	Let $\mathfrak{m}'$ be a maximal ideal containing $\mathfrak{a}$. Then, we have $\mathfrak{m}^{n} \subset \mathfrak{m}'$. Since $\mathfrak{m}'$ is prime, this forces $\mathfrak{m} \subset \mathfrak{m}'$ and maximality forces $\mathfrak{m} = \mathfrak{m}'$. Thus, $\mathfrak{m}$ is the unique maximal ideal containing $\mathfrak{a}$ and hence, $R/\mathfrak{a}$ is a local ring. Suppose $ab \in \mathfrak{a}$. If $b \in \mathfrak{m}$, then $b^{n} \in \mathfrak{a}$. If $b \notin \mathfrak{m}$, then $b$ is a unit in $R/\mathfrak{a}$. But then $ab = 0$ in $R/\mathfrak{a}$ implies $a = 0$ in $R/\mathfrak{a}$, i.e., $a \in \mathfrak{a}$. Thus, $\mathfrak{a}$ is primary.

	Taking radicals on $\mathfrak{m}^{n} \subset \mathfrak{a} \subset \mathfrak{m}$ shows that $\mathfrak{m} = \sqrt{\mathfrak{a}}$.
\end{proof}

\begin{defn}
	Let $R$ be a ring, and $M$ an $R$-module. $\mathfrak{p} \in \Spec(R)$ is said to be an \deff{associated prime} of $M$ if there exists $x \in M$ such that $\mathfrak{p} = \ann_{R}(x)$. The set of all associated primes of $M$ is denoted by $\Ass_{R}(M)$.
\end{defn}
In the notation of the above definition, we must have $x \neq 0$ since a prime ideal is a proper ideal of $R$. We will often drop the subscript if no confusion arises.

\begin{prop} \label{prop:associated-quotient-embeds}
	Let $M$ be a $R$-module, and $\mathfrak{p} \in \Spec(R)$. 

	$\mathfrak{p} \in \Ass_{R}(M)$ iff $R/\mathfrak{p}$ is isomorphic to a submodule of $M$.
\end{prop}
\begin{proof} 
	\forward Write $\mathfrak{p} = \ann_{R}(x)$. Define $\varphi : R \to M$ by $1 \mapsto x$. Then, $\ker(\varphi) = \mathfrak{p}$ and thus, $R/\mathfrak{p} \cong \im(\varphi) \le M$. 

	\backward Let $f : R/\mathfrak{p} \into M$ be an injection. Composing with $R \to R/\mathfrak{p}$ gives a map $\varphi : R/\mathfrak{p} \to M$. Let $x \vcentcolon= \varphi(1)$. Then, $\mathfrak{p} = \ann_{R}(x)$.
\end{proof}

\begin{cor} \label{cor:ass-of-submodule}
	If $N \le M$, then $\Ass_{R}(N) \subset \Ass_{R}(M)$.
\end{cor}

\begin{prop}
	If $N \le M$, then
	\begin{equation*} 
		\Ass_{R}(M) \subset \Ass_{R}(N) \cup \Ass_{R}(M/N).
	\end{equation*}
\end{prop}
\begin{proof} 
	Let $\mathfrak{p} = \ann_{R}(x) \in \Ass(M)$. If $Rx \cap N = 0$, then $Rx$ can be viewed as a submodule of $M/N$ and the result follows. Else, pick $ax \in Rx \cap N$ with $ax \neq 0$. Clearly, $\mathfrak{p} \subset \ann_{R}(ax)$. Conversely, suppose that $b \in \ann_{R}(ax)$. Then, $bax = 0$ and thus, $ba \in \mathfrak{p}$. Since $ax \neq 0$, we have $a \notin \mathfrak{p}$ and thus, $b \in \mathfrak{p}$, showing $\mathfrak{p} = \ann_{R}(ax)$. Since $ax \in N$, we see that $\mathfrak{p} \in \Ass(N)$.
\end{proof}

\begin{cor}
	If $M = \bigoplus_{i = 1}^{n} N_{i}$, then 
	\begin{equation*} 
		\Ass_{R}(M) = \bigcup_{i = 1}^{n} \Ass_{R}(N_{i}).
	\end{equation*}
\end{cor}
\begin{proof} 
	By induction, it suffices to assume $n = 2$. As $N_{i}$ are isomorphic to submodules of $M$, we have the inclusion $\supset$. For the other, note that we have an exact sequence $0 \to N_{1} \to M \to N_{2} \to 0$.
\end{proof}

\begin{prop} \label{prop:Ass-is-nonempty}
	Let $R$ be Noetherian, and $M$ a \underline{nonzero} $R$-module. Then, $\Ass_{R}(M) \neq \emptyset$. Moreover,
	\begin{equation*} 
		\mathcal{Z}(M) = \bigcup_{\mathfrak{p} \in \Ass_{R}(M)} \mathfrak{p}.
	\end{equation*}
	Equivalently, $a_{M}$ is injective iff $a$ does not lie in any associated prime.
\end{prop}
\begin{proof} 
	Define $\mathcal{E} \vcentcolon= \{\ann_{R}(x) : x \in M \setminus \{0\}\}$. As $M \neq 0$, $\mathcal{E}$ is a nonempty set of proper ideals of $R$. Since $R$ is Noetherian, $\mathcal{E}$ has a maximal element $I = \ann_{R}(x)$. We now show that $I$ is prime.

	Suppose $ab \in I$ but $a \notin I$. We must show that $b \in I$. By hypothesis, $bax = 0$. Thus, $b \in \ann_{R}(ax)$. Since $a \in I$, we have $ax \neq 0$ and thus, $\ann_{R}(ax) \in \mathcal{E}$. But clearly, $I \subset \ann_{R}(ax)$. By maximality, $I = \ann_{R}(ax)$ and hence, $b \in I$. This shows that $I$ is a prime and hence, $\Ass(M) \neq \emptyset$.

	We now prove the final statement. Note that if $a \in \mathfrak{p} \in \Ass(M)$, then $a \cdot x = 0$ for some $x \neq 0$ and thus, $a \in \mathcal{Z}(M)$. \newline
	Conversely, assume that $a \in \mathcal{Z}(R)$. Then, $a \in \ann_{R}(x)$ for some $x \neq 0$. As in the first part, we may pick a maximal element $\mathfrak{p} \in \mathcal{E}$ containing $\ann_{R}(x)$. As seen, $\mathfrak{p}$ is prime and thus, $a \in \mathfrak{p} \in \Ass(M)$.
\end{proof}

The above proof also shows us the following.
\begin{por} \label{por:annihilator-contained-in-associated-prime}
	Let $R$ be Noetherian, and $M$ be a $R$-module. If $x \in M \setminus \{0\}$, then $\ann_{R}(x)$ is contained in some associated prime, i.e., $\ann_{R}(x) \subset \mathfrak{p}$ for some $\mathfrak{p} \in \Ass_{R}(M)$.
\end{por}

\begin{defn}
	An endomorphism $f \in \End_{R}(M)$ is said to be \deff{pointwise nilpotent} if for every $x \in M$, there exists $n \ge 1$ such that $f^{n}(x) = 0$.
\end{defn}

\begin{prop} \label{prop:pointwise-nilpotent-every-associated-prime}
	If $a_{M}$ is pointwise nilpotent, $a$ lies in every associated prime of $M$, i.e.,
	\begin{equation*} 
		\{a \in R : a_{M} \text{ is pointwise nilpotent}\} \subset \bigcap_{\mathfrak{p} \in \Ass_{R}(M)} \mathfrak{p}.
	\end{equation*}
\end{prop}
If $\Ass(M) = \emptyset$, then the intersection is assumed to be $R$, in which case the proposition has no content.
\begin{proof} 
	Suppose $a_{M}$ is pointwise nilpotent, and $\mathfrak{p} = \ann_{R}(x) \in \Ass(M)$ is arbitrary. By assumption, there exists $n$ such that $a^{n} x = 0$, i.e., $a^{n} \in \mathfrak{p}$. Since $\mathfrak{p}$ is prime, we get $a \in \mathfrak{p}$.
\end{proof}

\begin{prop}
	Let $S \subset R$ be a multiplicative subset with $0 \notin S$. Let $\mathfrak{p}$ be maximal among all the ideals $\mathfrak{q}$ such that $\mathfrak{q} \cap S = \emptyset$. Then, $\mathfrak{p}$ is a prime ideal.
\end{prop}
\begin{proof} 
	Suppose not. Pick $a, b \in R \setminus \mathfrak{p}$ with $ab \in \mathfrak{p}$. Then, $(a, \mathfrak{p})$ and $(b, \mathfrak{p})$ must intersect $S$ by maximality of $\mathfrak{p}$. Thus, there exist $s_{1}, s_{2} \in S$ with
	\begin{equation*} 
		s_{1} = \lambda_{1} a + p_{1} \andd s_{2} = \lambda_{2} b + p_{2}
	\end{equation*}
	for $\lambda_{i} \in R$ and $p_{i} \in \mathfrak{p}$. Multiplying the elements shows that $s_{1} s_{2} \in S \cap \mathfrak{p}$, a contradiction.
\end{proof}

\begin{defn}
	For an $R$-module $M$, the \deff{support of $M$} is defined as
	\begin{equation*} 
		\Supp_{R}(M) \vcentcolon= \{\mathfrak{p} \in \Spec(R) : M_{\mathfrak{p}} \neq 0\}.
	\end{equation*}
\end{defn}

We may drop the superscript $R$ if no confusion arises.

\begin{prop} \label{prop:pointwise-nilpotent-every-support-prime}
	Let $M$ be an $R$-module.
	\begin{equation*} 
		\{a \in R : a_{M} \text{ is pointwise nilpotent}\} = \bigcap_{\mathfrak{p} \in \Supp(M)} \mathfrak{p}.
	\end{equation*}
\end{prop}
Note that there is no Noetherian hypothesis anywhere.
\begin{proof} 
	($\subset$) Let $a_{M}$ be pointwise nilpotent, and $\mathfrak{p} \in \Supp(M)$ be arbitrary. Then, there exists $x \in M$ such that $(Rx)_{\mathfrak{p}} \neq 0$. Thus, $\ann_{R}(x) \subset \mathfrak{p}$. Moreover, $a^{n} x = 0$ for some $n \ge 1$, by assumption. Thus, $a^{n} \in \ann_{R}(x) \subset \mathfrak{p}$ and hence, $a \in \mathfrak{p}$.

	($\supset$) Suppose $a_{M}$ is not pointwise nilpotent. Let $x \in M$ be such that $a^{n} x \neq 0$ for all $n \ge 1$. In particular, $S = \{1, a, a^{2}, \ldots\}$ does not intersect $\ann_{R}(x)$. Usual application of Zorn's lemma tells us that there is a maximal ideal $\mathfrak{p}$ not intersecting $S$ and containing $\ann_{R}(x)$. By the earlier result, $\mathfrak{p}$ is prime and by construction, $(Rx)_{\mathfrak{p}} \neq 0$. Thus, $M_{\mathfrak{p}} \neq 0$ and $\mathfrak{p} \in \Supp(M)$. As $\mathfrak{p} \cap S = \emptyset$, $a \notin \mathfrak{p}$ and we are done.
\end{proof}

\begin{prop} \label{prop:every-support-prime-every-associated-prime}
	Let $R$ be a Noetherian ring. Then,
	\begin{equation*} 
		\bigcap_{\mathfrak{p} \in \Ass_{R}(M)} \mathfrak{p} \subset \bigcap_{\mathfrak{p} \in \Supp_{R}(M)} \mathfrak{p}.
	\end{equation*}
\end{prop}
\begin{proof} 
	Let $a \in R$ be an element belonging to every associated prime. Let $\mathfrak{p} \in \Supp(M)$. We need to show that $a \in \mathfrak{p}$. To do that, it suffices to show that $\mathfrak{p}$ contains an associated prime $\mathfrak{q}$.

	Let $x \in M$ be such that $(Rx)_{\mathfrak{p}} \neq 0$. Pick $\mathfrak{Q} \in \Ass_{R_{\mathfrak{p}}}((Rx)_{\mathfrak{p}})$ (this set is nonempty since $R_{\mathfrak{p}}$ is Noetherian). Write $\mathfrak{Q} = \ann_{R_{\mathfrak{p}}}(s^{-1} y)$ for some $y \in R \setminus \{0\}$ and $s \notin \mathfrak{p}$. \newline
	Define $\mathfrak{q} \vcentcolon= \mathfrak{Q} \cap R$. $\mathfrak{q}$ is prime, since it is the contraction of a prime. Moreover, $\mathfrak{q}$ cannot intersect $R \setminus \mathfrak{p}$, i.e., $\mathfrak{q} \subset \mathfrak{p}$. To prove the result, we just need to show that $\mathfrak{q}$ is associated. 

	Let $q_{1}, \ldots, q_{k}$ be generators of $\mathfrak{q}$ (note that $R$ is Noetherian). Then, $q_{i} s^{-1} y = 0$ for all $i \in \{1, \ldots, k\}$. Thus, there exist $s_{1}, \ldots, s_{k} \in R \setminus \mathfrak{p}$ such that $s_{i} q_{i} y = 0$ for all $i$. Let $t \vcentcolon= \prod s_{i} \in R \setminus \mathfrak{p}$. 

	We claim that $\mathfrak{q} = \ann_{R}(ty)$ and hence, $\mathfrak{q}$ is associated. \newline
	($\subset$) Suffices to show that each $q_{i}$ annihilates $ty$. But this is true because $t$ is a multiple of $s_{i}$ and $q_{i}$ annihilates $s_{i} y$. \newline
	($\supset$) Suppose $a \in \ann_{R}(ty)$. Since $t$ is a unit in $R_{\mathfrak{p}}$, we see that $a \in \ann_{R_{\mathfrak{p}}}(y) = \mathfrak{Q}$. Thus, $a \in \mathfrak{Q} \cap R = \mathfrak{q}$.
\end{proof}

\begin{cor} \label{cor:noetherian-pointwise-nilpotent-iff-every-associated-prime}
	Let $R$ be a Noetherian ring, and $M$ an $R$-module. Then, $a_{M}$ is pointwise nilpotent iff $a$ lies in every associated prime of $M$, i.e.,
	\begin{equation*} 
		\{a \in R : a_{M} \text{ is pointwise nilpotent}\} = \bigcap_{\mathfrak{p} \in \Ass_{R}(M)} \mathfrak{p}.
	\end{equation*}
\end{cor}
\begin{proof} 
	Combine \Cref{prop:pointwise-nilpotent-every-associated-prime}, \Cref{prop:pointwise-nilpotent-every-support-prime}, \Cref{prop:every-support-prime-every-associated-prime}.
\end{proof}

\begin{cor}
	Let $R$ be a Noetherian ring, and $M$ be a \underline{finitely generated} $R$-module (i.e., $M$ is also Noetherian). Then, $a_{M}$ is nilpotent iff $a_{M}$ lies in every associated prime of $M$, i.e.,
	\begin{equation*} 
		\{a \in R : a_{M} \text{ is pointwise nilpotent}\} = \bigcap_{\mathfrak{p} \in \Ass_{R}(M)} \mathfrak{p}.
	\end{equation*}
\end{cor}
\begin{proof} 
	For a finitely generated module, pointwise nilpotent is same as nilpotent.
\end{proof}

\begin{cor} \label{cor:noetherian-unique-associated-prime}
	If $R$ is Noetherian, and $M$ an $R$-module, then the following are equivalent:
	\begin{enumerate}
		\item $\Ass_{R}(M)$ is a singleton, i.e., $M$ has exactly one associated prime,
		\item $M \neq 0$ and for each $a \in R$, $a_{M}$ is either injective or pointwise nilpotent.
	\end{enumerate}
	In this case, $\mathfrak{p} \in \Ass_{R}(M)$ is given by $\mathfrak{p} = \{a \in R : a_{M} \text{ is pointwise nilpotent}\}$. 

	If $M$ is also Noetherian, then ``pointwise'' can be omitted from the above. Thus, $\mathfrak{p} = \sqrt{\ann_{R}(M)}$.
\end{cor}

\begin{cor}
	Assume that $R$ and $M$ are both Noetherian. A submodule $Q \le M$ is primary if and only if $M/Q$ has exactly one associated prime $\mathfrak{p}$, and in that case, $Q$ is $\mathfrak{p}$-primary.
\end{cor}

\begin{ex} \label{ex:associated-primes-primary-quotient}
	Let $R$ be a Noetherian ring. If $\mathfrak{q} \subset R$ is a $\mathfrak{p}$-primary ideal, then $\Ass_{R}(R/\mathfrak{q}) = \{\mathfrak{p}\}$. In particular, $\Ass_{R}(R/\mathfrak{p}) = \{\mathfrak{p}\}$.
\end{ex}

\subsection{Topology on a module} \label{subsec:topology-module}
\subsubsection{Definition}

\begin{defn}
	A \deff{topological abelian group} is an abelian group $(M, +)$ with a topology such that the maps
	\begin{enumerate}
		\item $(x, y) \mapsto x + y$, and
		\item $x \mapsto -x$
	\end{enumerate}
	are continuous. Consequently, $(x, y) \mapsto x - y$ is also continuous.
\end{defn}

By a \deff{neighbourhood} of $a \in M$, we shall mean a subset $U \subset M$ such that $a$ is contained in the interior of $U$. Equivalently, there is an open set $V$ such that $a \in V \subset U$.

If $a \in M$ is a fixed element of a topological abelian group, then the map $T_{a} : M \to M$ defined by $x \mapsto x + a$ is continuous. Moreover, it is invertible with $T_{-a}$ as inverse and hence, is a homeomorphism. Hence, if $U$ is a neighbourhood of $0$, then $U + a$ is a neighbourhood of $a$; moreover, every neighbourhood is of this. Thus, it suffices to consider neighbourhoods of $0$.

Similarly, the inversion map $x \mapsto -x$ is a homeomorphism and thus, $U$ is a neighbourhood of $0$ iff $-U$ is a neighbourhood of $0$.

\begin{lem}
	Let $M$ be a topological abelian group. If $\{0\}$ is closed in $M$, then $M$ is a Hausdorff space.
\end{lem}
\begin{proof} 
	Note that the diagonal $\Delta \subset M \times M$ is the inverse image of $\{0\}$ under the continuous map $(x, y) \mapsto x - y$. 
\end{proof}

\begin{lem} \label{lem:abelian-group-hausdorff}
	Let $N$ be the intersection of all neighbourhoods of $0$. Then,
	\begin{enumerate}
		\item $N$ is a subgroup.
		\item $N = \overline{\{0\}}$, i.e., $N$ is the closure of $\{0\}$.
		\item $M/N$ is Hausdorff.
		\item $M$ is Hausdorff $\Leftrightarrow$ $N = 0$.
	\end{enumerate}
\end{lem}
\begin{proof} 
	\phantom{hi}
	\begin{enumerate}
		\item Clearly, $0 \in N$. Suffices to show that whenever $a, b \in N$, then $a - b \in N$. From the continuity of group operations, it follows the map $M \times M \to M$ defined by $(x, y) \mapsto x - y$ is continuous. Thus, given an arbitrary neighbourhood $U$ of $0$, there exist neighbourhoods $V_{1}$ and $V_{2}$ of $0$ such that
		\begin{equation*} 
			f(V_{1} \times V_{2}) \subset U.
		\end{equation*}
		Thus, given arbitrary $a, b \in N$ and an arbitrary neighbourhood $U$ of $0$, pick $V_{1}$ and $V_{2}$ as above. Then, $a \in V_{1}$ and $b \in V_{2}$ by assumption and hence, $a - b = f((a, b)) \in U$.
		%
		\item Suppose $a \in N$. An arbitrary neighbourhood of $a$ is of the form $U + a$ for some neighbourhood $U$ of $0$. By the earlier part, $-a \in N$ and hence, $a \in U$. Thus, $0 \in U + a$ and hence, $a \in \overline{\{0\}}$. 

		Conversely, if $a \in \overline{\{0\}}$ and $U$ is an arbitrary neighbourhood of $0$, then $a \in U$ and hence, $a \in N$.
		%
		\item By the earlier part, $N$ is closed and thus, $\{0\}$ is closed in $G/H$. The result follows from the earlier lemma.
		%
		\item Trivial. \qedhere
	\end{enumerate}
\end{proof}

\subsubsection{Completion via Cauchy sequences} \label{subsubsec:completion-cauchy}

In this subsubsection, $M$ will denote a topological abelian group. 

\begin{defn}
	A sequence $(x_{n})_{n \ge 1}$ in $M$ is said to be a \deff{Cauchy sequence} if for every neighbourhood $U$ of $0$, there exists $n_{0} \ge 1$ such that
	\begin{equation*} 
		x_{n} - x_{m} \in U
	\end{equation*} 
	for all $n, m \ge n_{0}$.

	Two Cauchy sequences $(x_{n})_{n}$ and $(y_{n})_{n}$ are said to be \deff{equivalent} if $x_{n} - y_{n} \to 0$ in $M$. The set of all equivalence classes of Cauchy sequences is denoted by $\widehat{M}$.
\end{defn}

Note that we have a well-defined addition on $\widehat{M}$ given by
\begin{equation*} 
	[(x_{n})_{n}] + [(y_{n})_{n}] = [(x_{n} + y_{n})_{n}].
\end{equation*}

It is easily checked that $\widehat{M}$ is an abelian group with the above operation. We have a map $\varphi : M \to \widehat{M}$ given by $x \mapsto [(x)_{n}]$, i.e., $x$ is sent to the equivalence class of the constant sequence $x$.

\begin{prop}
	$\ker(\varphi) = \bigcap U$, where $U$ runs over all the neighbourhoods of $0$.
\end{prop}
\begin{proof} 
	Let $x \in M$. Note that $x \in \ker(\varphi)$ iff $(x)_{n} \sim (0)_{n}$ iff $x - 0 \to 0$ in $M$ iff $x$ is in every neighbourhood of $U$.
\end{proof}

Using the above and \Cref{lem:abelian-group-hausdorff}, we get the following.
\begin{cor}
	$\varphi$ is injective iff $M$ is Hausdorff.
\end{cor}

\begin{rem}
	$\widehat{M}$ defined above has not been given a topology, it is only an abelian group. While it is possible to define a topology at this stage, we postpone this further and look at it in a special case in \Cref{sec:completions}.
\end{rem}

\subsubsection{Fundamental system of neighbourhoods}

\begin{defn}
	We say that $0$ has a \deff{fundamental system of neighbourhoods consisting of subgroups} if there is a sequence of subgroups $(M_{n})_{n \ge 0}$ with
	\begin{enumerate}
		\item $M_{0} = M$,
		\item $M_{n} \supset M_{n + 1}$ for all $n \ge 0$
	\end{enumerate}
	such that $U \subset M$ is a neighbourhood of $0$ if and only if it contains some $M_{n}$.
\end{defn}

Conversely, suppose that $M$ is an abelian group (without any topology) and we are given a sequence
\begin{equation*} 
	M = M_{0} \supset M_{1} \supset M_{2} \supset \cdots
\end{equation*}
of subgroups. We can define a topology on $M$ as follows: $U \subset M$ is open iff for every $a \in U$, there exists $n \ge 0$ such that $a + M_{n} \subset U$. \newline
In other words, we are considering the basis
\begin{equation*} 
	\mathcal{B} \vcentcolon= \{a + M_{n} : a \in M, n \ge 0\}
\end{equation*}
and looking at the topology it generates. 

Note that there is a check to be done, namely that what we have defined is actually a topology. Equivalently, we must check that $\mathcal{B}$ is actually a basis. 

\begin{proof}
	Note that $M \in \mathcal{B}$ and thus, the union of all elements of $\mathcal{B}$ is indeed $M$. Now, suppose that
	\begin{equation*} 
		x \in (a + M_{n}) \cap (b + M_{m})
	\end{equation*}
	for some $a, b \in M$ and $n, m \ge 0$. Without loss of generality, we may assume $m \ge n$. Thus, $M_{m} \subset M_{n}$. \newline
	Note that $x + M_{m}$ is an element of $\mathcal{B}$ containing $x$. We claim that
	\begin{equation*} 
		(x + M_{m}) \subset (a + M_{n}) \cap (b + M_{m}).
	\end{equation*}
	Indeed, if $y \in x + M_{m}$, then
	\begin{equation*} 
		y - a = (y - x) + (x - a) \in M_{m} + M_{n} = M_{n}
	\end{equation*}
	and similarly, $y - b \in M_{m}$, as desired.
\end{proof}

Thus, we have shown that this gives $M$ a topology. As one would desire, more is true, namely that $M$ is a topological abelian group with this topology. 

\begin{proof}
	First, we consider the map $f : M \to M$ given by $x \mapsto -x$. Let $U \subset M$ be open. We must show that $f^{-1}(U) = -U$ is open. But this is true simply because if $B \in \mathcal{B}$, then $-B \in \mathcal{B}$ as well. 

	Now, consider the map $g : M \to M$ given by $(x, y) \mapsto x + y$. Let $U \subset M$ be open and let $(x, y) \in f^{-1}(U)$. Put $a \vcentcolon= f(x, y)$. Then, $U$ contains a set of the form $a + M_{n}$. We claim that
	\begin{equation*} 
		(x + M_{n}) \times (y + M_{n}) \subset f^{-1}(U).
	\end{equation*}
	Indeed, if $(x', y') \in (x + M_{n}) \times (y + M_{n})$, then
	\begin{equation*} 
		f(x', y') = x' + y' = a + (x' - x) + (y' - y) \in a + M_{n} \subset f^{-1}(U). \qedhere
	\end{equation*}
\end{proof}

Lastly, one checks that $(M_{n})_{n \ge 0}$ is a fundamental system of neighbourhoods. We leave this to the reader.

\begin{rem}
	Note that each $M_{n}$ is clearly open in this topology. Moreover, each $M_{n}$ is also closed, being the complement of the open set $\bigcup_{x \notin M_{n}} (x + M_{n})$.
\end{rem}

% Similarly, one defines a topological ring and a topological module -- the theme is the same, these are algebraic as well as topological objects such that the algebraic operations are continuous.