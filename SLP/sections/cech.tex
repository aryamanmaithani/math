\section{\Cech cohomology}

% \textbf{Notation.} For $a \in R$, we let $M_{a}$ denote the localisation of $M$ with respect to the multiplicative subset $\{1, a, a^{2}, \ldots\}$.

\begin{defn}
	Let $R$ be a ring, and $a \in R$. Define the cocomplex $\CC^{\bullet}(a)$ by
	\begin{equation*} 
		0 \to R \to R_{a} \to 0,
	\end{equation*}
	concentrated in degrees $0$ and $1$.
\end{defn}



The map $R \to R_{a}$ is the natural localisation map $r \mapsto r/1$. 

Recall that given cocomplexes $C^{\bullet}$ and $D^{\bullet}$, the tensor product $C^{\bullet} \otimes_{R} D^{\bullet}$ is the cocomplex whose $n$-th term is
\begin{equation*} 
	[C^{\bullet} \otimes_{R} D^{\bullet}]^{n} \vcentcolon= \bigoplus_{p + q = n} C^{p} \otimes_{R} D^{q},
\end{equation*}
with the coboundary map given on simple tensors $c^{p} \otimes d^{q} \in C^{p} \otimes_{R} D^{q}$ by
\begin{equation*} 
	\partial(c^{p} \otimes d^{q}) \vcentcolon= \partial(c^{p}) \otimes d^{q} + (-1)^{p} c^{p} \otimes \partial(d^{q}).
\end{equation*}

\begin{defn}
	Given elements $a_{1}, \ldots, a_{n}$, define the \deff{\Cech cocomplex} $\CC^{\bullet}(a_{1}, \ldots, a_{n})$ by
	\begin{equation*} 
		\CC^{\bullet}(a_{1}) \otimes_{R} \cdots \otimes_{R} \CC^{\bullet}(a_{n}).
	\end{equation*}
\end{defn}

Note that we have natural isomorphisms 
\begin{equation*} 
	R_{a} \otimes_{R} R_{b} \cong (R_{a})_{b} \cong R_{ab}
\end{equation*} 
of $R$-modules. 

For notational sake, let $e_{a}$ denote $1/1$ in $R_{a}$. Similarly, let $e_{ab}$ denote $1/1$ in $R_{ab}$, et cetera.

\begin{ex}
	Let us write out $\CC(a, b)$ and $\CC(a, b, c)$. The modules are given as:
	\begin{align*} 
		0 \to R \to R_{a} \oplus R_{b} &\to R_{ab} \to 0, \\
		0 \to R \to R_{a} \oplus R_{b} \oplus R_{c} &\to R_{ab} \oplus R_{bc} \oplus R_{ac} \to R_{abc} \to 0.
	\end{align*}
	This makes the general picture of $\CC(a_{1}, \ldots, a_{n})$ clear, as far as the modules are concerned. It looks something like
	\begin{equation} \label{eq:cech}
		0 \to R \xrightarrow{\partial^{0}} \bigoplus_{1 \le i \le n} R_{a_{i}} \xrightarrow{\partial^{1}} \bigoplus_{1 \le i_{1} < i_{2} \le n} R_{a_{i_{1}} a_{i_{2}}} \xrightarrow{\partial^{2}} \cdots \to \bigoplus_{1 \le i \le n} R_{a_{1} \cdots \widehat{a_{i}} \cdots a_{n}} \xrightarrow{\partial^{n - 1}} R_{a_{1} \cdots a_{n}} \to 0.
	\end{equation}

	$\partial_{0}$ is given by $1 \mapsto (e_{a_{1}}, \ldots, e_{a_{n}})$. $\partial^{n - 1}$ is ``given by''
	\begin{equation*} 
		e_{a_{1} \cdots \widehat{a_{i}} \cdots a_{n}} \mapsto \pm e_{a_{1} \cdots a_{n}},
	\end{equation*}
	where the sign depends on $i$ and $n$.
\end{ex}

\begin{rem}
	Note that $e_{a}$ is not an $R$-basis for $R_{a}$. Indeed, $1/a$ is not generally in the $R$-submodule generated by $\{e_{a}\}$.
\end{rem}

\begin{prop}
	The zeroth cohomology $H^{0}(\CC(a_{1}, \ldots, a_{n}))$ is given as 
	\begin{equation*} 
		H^{0}(\CC(a_{1}, \ldots, a_{n})) = \Gamma_{a_{1}}(R) \cap \cdots \cap \Gamma_{a_{n}}(R) = \Gamma_{(a_{1}, \ldots, a_{n})}(R).
	\end{equation*}
\end{prop}
\begin{proof} 
	Follows from \Cref{prop:cech-zeroth-cohomology} and \Cref{prop:product-sum-local-cohomology}. 
\end{proof}

\begin{cor}
	If $R$ is a domain, and $a_{1}, \ldots, a_{n} \in R$ are nonzero, then $H^{0}(\CC(a_{1}, \ldots, a_{n})) = 0$. More generally, if $a_{1}, \ldots, a_{n}$ are nonzerodivisors, then $H^{0}(\CC(a_{1}, \ldots, a_{n})) = 0$. 

	Similarly, if $(a_{1}, \ldots, a_{n}) = R$, then $H^{0}(\CC(a_{1}, \ldots, a_{n})) = 0$.
\end{cor}

\begin{prop}
	Let $a_{1}, \ldots, a_{n} \in R$. For each $i \in \{1, \ldots, n\}$, let $N_{i}$ denote the image of $R_{a_{1} \cdots \widehat{a_{i}} \cdots a_{n}}$ in $R_{a_{1} \cdots a_{n}}$. Then,
	\begin{equation*} 
		H^{n}(\CC(a_{1}, \ldots, a_{n})) = R_{a_{1} \cdots a_{n}}/(N_{1} + \cdots + N_{n}).
	\end{equation*}
\end{prop}

Of course, the above proposition is not saying anything since the above is quite literally the definition of (co)homology. However, we now note a particular example.

\begin{ex}
	Consider $R = \kk[X]$ and $a = X$. In this case, note that the \Cech cocomplex $\CC(X)$ is given as
	\begin{equation*} 
		0 \to \kk[X] \into \kk\left[X, \frac{1}{X}\right] \to 0.
	\end{equation*}
	As $\kk[X]$ is an integral domain and $X \neq 0$, it follows that the localisation map above is an inclusion, i.e., $H^{0}(\CC(X)) = 0$. For the first cohomology, note that
	\begin{equation*} 
		H^{1}(\CC(X)) = \frac{\kk\left[X, \frac{1}{X}\right]}{\kk[X]}.
	\end{equation*}

	Let us analyse the above quotient. Elements of $\kk[X, \frac{1}{X}]$ are elements of the form
	\begin{equation*} 
		\frac{a_{-m}}{X^{m}} + \cdots + \frac{a_{-1}}{X} + a_{0} + \cdots + a_{n} X^{n}.
	\end{equation*}
	Quotienting by $\kk[X]$ will only leave us with the ``principal'' (or ``negative'' part). More precisely, each equivalence class of the quotient will contain a unique representative of the form $\frac{a_{-m}}{X^{m}} + \cdots + \frac{a_{-1}}{X}$. Furthermore, adding two representatives gives us the representative of the sum. Finally, looking at the action of $X^{k}$ on such an element, we see that
	\begin{equation*} 
		X^{k} \cdot \left(\frac{a_{-m}}{X^{m}} + \cdots + \frac{a_{-1}}{X}\right) = \sum_{k < i \le m} \frac{a_{-i}}{X^{i - k}}.
	\end{equation*}
	This resembles the familiar module $E_{1} = \kk[X^{-1}]$ from \Cref{defn:En}! More precisely, this is $\kk[X^{-1}]/\kk$.

	In fact, this generalises quite easily to case of $R = R_{n}$. In that case, we have
	\begin{equation*} 
		H_{n}(\CC(X_{1}, \ldots, X_{n})) \cong X_{1}^{-1} \cdots X_{n}^{-1}E_{n}.% = \kk[X_{1}^{-1}, \ldots, X_{n}^{-1}]/.
	\end{equation*}

	Let us try to see the above in the case of $n = 2$. Write $R = \kk[X, Y]$ and take $(a_{1}, a_{2}) = (X, Y)$. Note that $R_{XY}$ are sums of the form
	\begin{equation*} 
		\sum_{a, b \in \mathbb{Z}} ? X^{a} Y^{b},
	\end{equation*}
	where the coefficients are in $\kk$. \newline
	Thus, we have elements like $\frac{1}{XY} + \frac{1}{X} + \frac{1}{Y} + X$. 

	On the other hand, the image of $R_{X} \oplus R_{Y}$ only consists of polynomials of the form
	\begin{equation*} 
		\sum_{a, b \in \mathbb{N}_{0}} ? X^{a} Y^{b} + \sum_{a < 0} ? X^{a} + \sum_{b < 0} ? Y^{b}.
	\end{equation*}
	Thus, there are no polynomials involving terms with both $X$ and $Y$ having negative exponent. Again, it is easy to see that the quotient is $E_{2}/\kk$.

	For the general case, we may define a map $R_{X_{1} \cdots X_{n}} \to E_{n}$ by
	\begin{equation*} 
		\sum_{\alpha \in \mathbb{Z}^{n}} a_{\alpha} X^{\alpha} \mapsto \sum_{\alpha \in (-\mathbb{N}_{0})^{n}} a_{\alpha} X^{\alpha}.
	\end{equation*}
	Composing this with the quotient map $E_{n} \to E_{n}/\kk$ gives us an onto map $R_{X_{1} \cdots X_{n}} \to E_{n}/\kk$. The kernel of this contains precisely of those polynomials which have monomials of the form $X^{\alpha}$ with at least one $\alpha_{i}$ nonnegative. This is exactly the image of $\partial^{n - 1}$, in the notation of \Cref{eq:cech}.
\end{ex}

\begin{ex}
	Let us now compute all cohomologies for polynomials rings in few variables. The rings appearing below will be integral domains. The advantage in that case is that we can think of the localisations as submodules of the quotient field of $R$ (and all the localisation maps are actually inclusions).

	$n = 1:$ $R = \kk[X]$. $\CC(X)$ is given as
	\begin{equation*} 
		0 \to \kk[X] \to \kk\left[X, \frac{1}{X}\right] \to 0.
	\end{equation*}
	As discussed, we have
	\begin{equation*} 
		H^{0}(\CC(X)) = 0 \andd H^{1}(\CC(X)) = \kk[X^{-1}]/\kk.
	\end{equation*}
	
	\hrulefill

	$n = 2:$ $R = \kk[X]$. $\CC(X, Y)$ is given as
	\begin{equation*} 
		0 \to \kk[X, Y] \xrightarrow{\partial^{0}} \kk\left[X, Y, \frac{1}{X}\right] \oplus \kk\left[X, Y, \frac{1}{Y}\right] \xrightarrow{\partial^{1}} \kk\left[X, Y, \frac{1}{XY}\right] \to 0.
	\end{equation*}
	We already know that $H^{0} = 0$ and $H^{2} = \kk[X^{-1}, Y^{-1}]/\kk$ for the above. We claim that $H^{1} = 0$. Indeed, note that
	\begin{equation*} 
		\partial^{1}(f, g) = -f + g.
	\end{equation*}
	Thus, if $\partial^{1}(f, g) = 0$, then 
	\begin{equation*} 
		f = g \in \kk\left[X, Y, \frac{1}{X}\right] \cap \kk\left[X, Y, \frac{1}{Y}\right] = \kk[X, Y].
	\end{equation*}
	Thus, $(f, g) = \partial^{0}(f) \in \im(\partial^{0})$, as desired.
	
	\hrulefill
	
	$n = 3:$ $R = \kk[X, Y, Z]$ and the complex is ``given by''
	\begin{equation*} 
		0 \to R \xrightarrow{\smatrix{1 \\ 1 \\ 1}} R_{X} \oplus R_{Y} \oplus R_{Z} 
		\xrightarrow{\smatrix{-1 & 1 & 0 \\
							  -1 & 0 & 1 \\
							  0 & -1 & 1}} 
		R_{XY} \oplus R_{XZ} \oplus R_{YZ} \xrightarrow{\smatrix{1 & -1 & 1}} R_{XYZ} \to 0.
	\end{equation*}

	Note that the modules appearing above are not free $R$-modules, even though we are using matrices to denote the maps. The above says that for $(f, g, h) \in R_{X} \oplus R_{Y} \oplus R_{Z}$, we have
	\begin{equation*} 
		\partial^{1}(f, g, h) = (-f + g, -f + h, -g + h).
	\end{equation*}
	Thus, if $(f, g, h) \in \ker(\partial^{1})$, then
	\begin{equation*} 
		f = g = h \in R_{X} \cap R_{Y} \cap R_{Z} = R.
	\end{equation*}
	As before, we have $(f, g, h) = \partial^{0}(f) \in \im(\partial^{0})$. Thus, $H^{1}(\CC(X, Y, Z)) = 0$.

	Now, let $(f, g, h) \in R_{XY} \oplus R_{XZ} \oplus R_{YZ}$ be in the kernel of $\partial^{2}$. Then,
	\begin{equation*} 
		f = g - h \in R_{XY} \cap (R_{XZ} + R_{YZ}) = R_{X} + R_{Y}.
	\end{equation*}
	(The sum and intersection above makes sense by working in the field $\kk(X, Y, Z)$, as remarked earlier.)

	Similarly, $g \in R_{X} + R_{Z}$ and $h \in R_{Y} + R_{Z}$. Thus, we can write
	\begin{align*} 
		f &= -f_{1} + f_{2}, \\
		g &= -g_{1} + g_{2}, \\
		h &= -h_{1} + h_{2},
	\end{align*}
	for $f_{1}, g_{1} \in R_{X}$, $f_{2}, h_{1} \in R_{Y}$, and $g_{2}, h_{2} \in R_{3}$.

	Now, using the fact that $f - g + h = 0$, we get
	\begin{equation*} 
		(-f_{1} + g_{1}) + (f_{2} - h_{1}) + (-g_{2} + h_{2}) = 0.
	\end{equation*}
	The terms are grouped so that the terms are in $R_{X}$, $R_{Y}$, and $R_{Z}$, respectively. This shows that all the terms are actually polynomials. (Since $R_{X} \cap (R_{Y} + R_{Z}) = R$, etc.) 

	Note that we have freedom in choosing $f_{1}, g_{1}, h_{1}$ in the sense that we may change the value by an element of $R$ and accordingly fix $f_{2}, g_{2}, h_{2}$. Since $-f_{1} + g_{1}$ is a polynomial, we may as well adjust the terms so that $-f_{1} + g_{1} = 0$. Now, $f_{2}$ is fixed. But $f_{2} - h_{1}$ is a polynomial and thus we may fix $h_{1}$ such that $f_{2} - h_{1} = 0$. This also then forces $g_{2} = h_{2}$. Thus, we are left with
	\begin{equation*} 
		(f, g, h) = (-f_{1} + h_{1}, -f_{1} + g_{2}, -h_{1} + g_{2}) = \partial^{1}(f_{1}, h_{1}, g_{2}) \in \im(\partial^{1}).
	\end{equation*}
	This shows that $H^{2}(\CC(X, Y, Z)) = 0$ as well.
\end{ex}

% \begin{rem}
% 	Assume $\mathfrak{m} = (X_{1}, \ldots, X_{n})$ and $I$ is $\mathfrak{m}$-primary. Then, $R/I$ is Artinian. Nonzero socle. Can find $f \in R$ such that $X_{i} \cdot f \in I$ for all $i$. Can extend to $(I, f)$.
% \end{rem}
