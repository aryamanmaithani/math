\assign{25-03-2019}
\textbf{Notation:} For $c, a \in \mathbb{Z},$ ``$c$ is a divisor of $a$'' (or ``$a$ is a multiple of c'') is denoted by $c|a.$
\begin{enumerate}[label=(\arabic*)]
	\item Let $a, b, c, d \in \mathbb{N}.$ State the following mathematically and write their negations:
	\begin{enumerate}[nosep] 
		\item $c$ divides $a.$ (Easier to think of: $a$ is a multiple of $c$).
		\item $c$ is a common divisor of $a$ and $b.$
		\item $d$ is the greatest common divisor of $a$ and $b.$
	\end{enumerate}
	\item Let $a, b, c \in \mathbb{Z}.$ Prove the following:
	\begin{enumerate}[nosep] 
		\item $c|0.$
		\item If $a|b$ and $c|a,$ then $c|b.$
		\item If $c|a$ and $c|b,$ then $\forall m, n \in \mathbb{Z}, c|(ma + nb).$ In particular, $c|(a+b)$ and $c|(a-b).$
		\item Suppose $c|a.$ If $a \neq 0,$ then $|a| \ge |c|.$
		\item If $a|c$ and $c|a,$ then $a = \pm c.$
	\end{enumerate}
	\item Let $S\subset\mathbb{N}$ be such that
	\begin{enumerate*}[label = (\arabic*)] 
		\item $1 \in S$ and
		\item For $k \in \mathbb{N},$ if $\{1, 2, \cdots, k\} \subset S,$ then $k + 1 \in S.$\\
	\end{enumerate*}
	Show that $S = \mathbb{N}.$
	\item Prove the following statement by (i) induction and (ii) well-ordering principle:\\
	Given $n \in \mathbb{N}\setminus\{1\},$ there is a prime number $p \in \mathbb{N}$ such that $p|n.$
	\item Let $(a_n), (b_n)$ be sequences of real numbers which converge and $c \in \mathbb{R}.$ Prove the following statements.
	\begin{enumerate}[nosep] 
		\item The sequence $(a_n + b_n)$ converges.
		\item The sequence $(ca_n)$ converges.
		\item The sequence $(a_nb_n)$ converges.
	\end{enumerate}
	If you do not assume that $(b_n)$ converges, what can you say about the convergence in each of the above cases?
	\item Let $(a_n)$ be a sequence of real numbers. If $(a_n)$ is bounded (converges), every subsequence is bounded (converges). If we assume that all the subsequences excluding the original sequence are bounded (convergent), then is the converse true?
	\item Show that every convergent sequence of real numbers is bounded. Is the converse true? Justify your answer.
	\item Let $(a_n), (b_n)$ be sequences of real numbers such that $a_n \le b_n$ for all $n \in \mathbb{N}.$ If they converge to $a$ and $b$ respectively, then, $a \le b.$
	\item Let $(a_n)$ be a sequence of non-negative real numbers converging to $a \in \mathbb{R}.$ Show that $(\sqrt{a_n})$ converges to $\sqrt{a}.$
\end{enumerate}