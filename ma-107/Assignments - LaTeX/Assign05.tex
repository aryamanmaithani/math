\assign{20-02-2019}
\begin{enumerate}[label=(\arabic*)]
\item Let $x, y \in \mathbb{R}$ be such that $x, y > 0$ and $n \in \mathbb{N}$. Show that if $x^n\le y^n$, then $x \le y$.
%
\item Show that if $x \in (0, 1),$ then $x \not\in \mathbb{Z}.$
%
\item If $r \in \mathbb{R}\setminus\mathbb{Q}$ and $x\in\mathbb{Q}\setminus\{0\}$, show that $rx$ and $r+x$ are elements of $\mathbb{R}\setminus\mathbb{Q}.$
%
\item Show that there is no rational number $x$ such that $x^2=3.$
%
\item For all $0 < x \in \mathbb{R}$ and $m \in\mathbb{N}$, define $x^{1/m}$ to the unique real number $y > 0$ such that $y^m = x$. Show the following:
\begin{enumerate}[nosep]
    \item For all $0 < x\in\mathbb{R}, m, n \in\mathbb{N}, (x^m)^{1/n}=(x^{1/n})^m.$
    \item For $m, n, l, k \in \mathbb{N}$, if $m/n=l/k$, then show that $(x^m)^{1/n}=(x^l)^{1/k}.$
\end{enumerate}
%
\item Let $f:\mathbb{R}\to\mathbb{R}$ be given by $f(x)=x^2.$ Find $f(A)$ for $A=$\\
\begin{enumerate*}[label=(\roman*)]
	\item $\{1, 1/2, 1/3, -1/2\}$
	\item $[0, 2]$
	\item $(1, 2]$
	\item $[-2, 1)$
	\item $[-2, -1)$
\end{enumerate*}
%
\item Is the function $f(x) = x^2$ one-one or onto as a function from\\
\begin{enumerate*}[label=(\roman*)]
	\item $\mathbb{R}$ to $\mathbb{R}$?
	\item $\mathbb{R}$ to $[0, \infty)$?
	\item $(0, \infty)$ to $(0, \infty)$?
	\item $(0, 1)$ to $(0, 1)$?
\end{enumerate*}
Can you identify properties of the graph that give the one-one or onto conditions?
%
\item Let $f:X\to Y$ and $g:Y\to Z$ be functions.
\begin{enumerate}[nosep]
	\item If $f$ and $g$  are one-one, show that $g\circ f$ is one-one.
	\item Is the converse true?
	\item Answer (a) and (b) with ``one-one" being replaced by ``onto".
\end{enumerate}
%
\item Find a bijection from $(0, 1)$ to $A$, where $A=$\\
\begin{enumerate*}[label=(\roman*)]
	\item $(1, 2)$
	\item $(0, 2 )$
	\item $(1, 3)$
	\item Can you find a bijection from $(0, 1)$ to $\mathbb{R}$?
\end{enumerate*}
%
\item Show that $A$ is countable, where $A=$\\
\begin{enumerate*}[label=(\roman*)]
	\item $\{2, 3, 4, 5, \dots\}$
	\item $\{2, 4, 6, 8, \dots\}$
	\item $\{1, 3, 5, 7, \dots\}$
	\item $2\mathbb{Z}$
	\item $2\mathbb{Z}+1$
	\item $\mathbb{Z}$
	\item $\mathbb{N}\times\mathbb{N}$
\end{enumerate*}
\end{enumerate}