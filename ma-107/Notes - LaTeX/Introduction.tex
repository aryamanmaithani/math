\iffalse Proper introduction to be finalised at the end, main points:
\begin{enumerate}
    \itemsep0em
    \item Not a formal book, supposed to be notes.
    \item Every chapter corresponds to a different lecture.
    \item Two topics were taught parallel-y. The beginning of every chapter mentions which professor has taught that so that the reader knows the flow. (The name of the professor is mentioned in the header of every following page of that chapter as well.)
    \item As the classes sort of complement each other, do read them in chronological order.
    \item {\color{mylinkcolor}{Text written in this shade of blue represents a hyperlink (except for this text itself) and depending on your file viewer, you can click on it to jump to the appropriate section. (Do note your position before clicking on it, though.)}}
    \item There are other colours also used in this document which you would understand yourself after you start reading.
    \item Appreciate the course!
\end{enumerate}
\fi
Hello, if you're reading these notes, you are most likely a BS Math student.\\
These are the notes that I made when I was taking this course in Spring 2019. I hope that you find them helpful and clear.\\
Here are some things that I'd like to point out
\begin{enumerate}[nosep] 
	\item This not meant to be a substitute for a formal book, these are just notes. However, I have tried to write them in as consistent and formal a way as possible. As you'd see, the nature of the course makes it a bit difficult to do so as it's somewhat a course on \emph{How to think Mathematically.}
    \item Every chapter corresponds to a different lecture.
    \item Two topics were taught parallel-y. The beginning of every chapter mentions which professor has taught that so that the reader knows the flow. (The name of the professor is mentioned in the header of every following page of that chapter as well.)
    \item As the classes sort of complement each other, do read them in chronological order.
    \item {\color{mylinkcolor}{Text written in this shade of blue represents a hyperlink (except for this text itself) and depending on your file viewer, you can click on it to jump to the appropriate section. (Do note your position before clicking on it, though.)}}
    \item There are other colours also used in this document which you would understand yourself after you start reading.
\end{enumerate}
Lastly, take this course with an open mind and appreciate the rigour and formalism to which you'll be introduced.\\
You might often come across topics which just seem very \emph{absurd} to prove. However, you must take a step back to see what is it really that you're proving.\\\\
As an example, you'll prove $1 > 0$ at some point of time. This is also the point of time that many lose their minds and feel like this may have been a wrong choice of major. However, what you're \emph{really} proving is that the \emph{multiplicative identity} is greater than the \emph{additive identity.} If you take one more step back, you'll also see that being ``greater'' just means that $1$ is related to $0$ in a certain manner such that the relation follows some certain axioms.\\
Once you realise that `$1$', `$0$' and `$>$' are just some arbitrary symbols which follow certain rules, it would (hopefully) feel less absurd that you actually \emph{need} to prove $1 > 0,$ you might even \emph{appreciate} that this is something that can indeed be proven.\\
Most likely, you'll also prove a result along the way using which you can also show that $\mathbb{C}$ can never be given a ``nice enough'' order. How neat is that?

\hrulefill

These notes are also supplemented with the assignments that were given; which have been placed at the appropriate location. It would be nice practice to solve the assignments alongside.