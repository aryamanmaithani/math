\prs{10-01-2019}
\section{Some Proofs}\label{sec:uniquenessproofs}
Let us now prove some claims working on the axioms stated in the previous lecture.\\
\claim{Additive identity is unique.}{%
Assume $\exists 0' \in \mathbb{F}$ which is also an additive identity. If I can show that $0' = 0$, then I would've proven my statement.\\
\begin{tabular}{l r}
    $\forall a \in \mathbb{F}: a + 0' = a = 0' + a$  & 
(By definition of additive identity) \\
    Also, $\forall b \in \mathbb{F}: b + 0 = b = 0 + b$     & (By \hyperref[ax:A3]{A3})\\
    $a = 0: 0 + 0' = 0 = 0' + 0$ & --- (I)\\
    $b = 0: 0' + 0 = 0' = 0 + 0'$ & ---(II)\\
    By (I) and (II) & \\
    $0+0' = 0 = 0'$ & \\
    $\therefore 0' = 0$ & \\
    $\implies$ Additive identity is unique. &
\end{tabular}}\\~\\
%
\claim{Multiplicative identity is unique.}{Similar as above.}\\
%
\claim{Additive inverse is unique.}{%
By definition, $\forall a\big( \exists b_a \in \mathbb{F}\big)$ such that:\\
$a + b_a = 0 = b_a + a$\\
Assume $\exists b_a' \in \mathbb{F}$ such that:\\
$a + b_a' = 0 = b_a' + a$\\
Now, I would like to prove that $b_a = b_a'$\\
\begin{tabular}{r l r}
    $b_a$ & $=b_a + 0$ & (By \refaxiom{3})  \\
     & $=b_a + (a + b_a')$ & (By definition stated above) \\
     & $= (b_a + a) + b_a'$ & (By \refaxiom{1})\\
     & $= 0 + b_a'$ & (By definition stated above)\\
     & $=b_a'$ & (By \refaxiom{3})  \\
     \hline
\end{tabular}\\~\\
$\therefore b_a = b_a' \implies$ Additive inverse is unique.
}\\
Now that we have shown that the additive inverse is unique, we can denote it by something. The additive inverse of $a$ is denoted by $-a$.\\~\\
\claim{Multiplicative inverse is unique.}{Similar as above.}
Now that we have ``shown'' that the multiplicative inverse is unique, we can denote it by something. The multiplicative inverse of $a$ is denoted by $\dfrac{1}{a}$ or $a^{-1}$.\\
\newpage

\hrulefill
\exercise{For a field $(\mathbb{F}, +, \cdot)$, prove that:%
\begin{enumerate}
    \itemsep0em
    \item $\forall a, b, c \in \mathbb{F}: a + b = a + c\implies b = c$
    \item $\forall a, b, c \in \mathbb{F}, a\neq0: a\cdot b = a\cdot c \implies b=c$
\end{enumerate}
}

\hrulefill
\section{Ordered Field Axioms}\label{sec:ofield}
A field $(\mathbb{F}, +, \cdot)$ is said to be an ordered field if it is an ordered set with an order `$<$' that follows:
\axiom{$\forall x, y, z \in \mathbb{F}$:\\
        $x < y \implies x + z < y + z$}\\
\axiom{$\forall x, y \in \mathbb{F}$:\\
        if $0 < x, 0 < y$; then $0 < x\cdot y$}
\\~\\
$\mathbb{R}$ is an ordered field.\\
Note that yet again, $\mathbb{Q}$ satisfies all the above properties.

\hrulefill

It had been mentioned before that $(\{0\}, +, \cdot)$ would constitute a field. This goes to show that from the field axioms alone, we cannot prove that $1\neq0$.\\
At this point, we add the ``axiom" that the multiplicative identity of $\mathbb{R}$ is not equal to its additive identity. This axiom is not being numbered but from now on, we'll use the fact that $1\neq0$.

\hrulefill
\exercise{After this lecture, \hyperref[assign:1]{this assignment} had been given where we proved many properties about real numbers using the 9 axioms. These properties will now be taken for granted. So, go through the assignment and prove them yourself.}
\exercise{What is the smallest field you can construct which has more than one element?}
\exercise{The ordered field axioms are not stated the same way everywhere. Find another notion of an ordered field and then show that that is equivalent to what we did.}

\hrulefill