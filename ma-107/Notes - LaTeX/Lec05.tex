\prs{16-01-2019}
Let us now try to construct some subsets of $\mathbb{R}$.\\
$\mathbb{N} = \{x \in \mathbb{R} | \text{where x can be written as a non-trivial sum of 1s}\}$\\
This gives us that $\mathbb{N}$ has elements: 1, 1+1, 1+1+1, $\dots$\\
As we had shown that $0<1$, we also have that: $0<1<1+1<1+1+1<\dots$\\
$\mathbb{Z} = \{x \in \mathbb{R} | x \in \mathbb{N} \text{ or } (-x) \in \mathbb{N} \text{ or } x=0\}$\\
$\mathbb{Q} = \{x \in \mathbb{R} | \exists p, q \in \mathbb{Z}, q \neq 0 \text{ such that } \frac{p}{q} = x \}$ %
$\left(\frac{p}{q} \text{ is notation for } p\cdot q^{-1}\right)$\\

Mentioning this once again for the sake of emphasis- till now, all the axioms which we had written (A1-A9) were satisfied by $\mathbb{Q}$ as well. We would now like to find a way to show that $\mathbb{Q} \neq \mathbb{R}$.

\hrulefill

Question: Is there a rational number x such that $x^2=2?$\\
Answer: Suppose $\exists x \in \mathbb{Q}$ such that $x^2 = 2$.
By our definition of rationals, this would mean that x can written as $p/q$ where $p$ and $q$ are integers. \\
To avoid equivalent fractions, let us put extra constraints saying: $p$ and $q$ must have no common factor and $q > 0$.\\
Now, by our assumption:\\~\\
$\left(p\cdot \dfrac{1}{q}\right)^2 = 2$\\~\\
$\implies p^2\cdot\left(\dfrac{1}{q}\right)^2 = 2$\\~\\
$\implies p^2=2q^2$ \hfill (Steps have been skipped, make sure that you can fill in the gaps.)\\~\\
This shows that $p^2$ is an even number.\\
$\implies$ p is an even number. \hfill (Show that the square of an odd number is odd and then, justify.)\\
$\implies p = 2r$ for some $r \in \mathbb{Z}$\\
$\implies p^2=(2r)^2=4r^2$\\
$\therefore 4r^2 = 2q^2 \iff q^2 = 2r^2$
$\implies q^2$ is even. $\implies q$ is even.\\
If $p$ and $q$ are both even, this means that they have a common factor, i.e., 2.\\
This contradicts our original assumption that p and q have no common factors.\\
Therefore, there is no $x \in \mathbb{Q}$ such that $x^2 = 2$. 

\hrulefill

The above shows that $\mathbb{Q}$ is not enough to solve all sorts of equations. Note that I haven't shown that any element of $\mathbb{R}$ \textit{can} solve that equation, so I can't yet say that $\mathbb{Q} \neq \mathbb{R}$.

\hrulefill

Let us look at another question:
$S = \{x \in \mathbb{Q} | x^2 < 2\}$\\
Is $S\neq \emptyset$? Yes. $0\in S, 1 \in S, 4/3 \in S.$\\
If you have any $r \in S$, would there always exist an $r' \in S$ such that $r < r'?$ In other words, does $S$ have a maximum?%
\footnote{Recall that maximum of a set requires that the maximal element belongs in the set as well. So, something like 5 would not be a correct answer for maximum.}\\
$T = \{x \in \mathbb{Q} | x > 0, 2 < x^2\}$\\
Does T have a minimum?
%
\section{Upper bound}\label{sec:ub}
\highlight{\textbf{Definition: }Let $S$ be an ordered set with relation $<$.\\
Then a subset $A \subset S$ is said to be bounded above in $S$ if:\\
$\exists z \in S\big(\forall x \in A (x \le z)\big)$\\
Further, $z$ is said to be an upper bound of $A$.}

\example{Is $\mathbb{N}$ bounded above in $\mathbb{R}?$\\%
Suppose I want to show that $\mathbb{N}$ is not bounded. What would I have to do? I would have to have the negate the statement given above in the definition, that is, I would have to show that-\\
$\forall z \in \mathbb{R}\big(\exists n \in \mathbb{N}(z < n)\big)$\\~\\
So, let us suppose that a given $z \in \mathbb{R}$ is an upper bound.\\
\phantom{ } \hfill(Now I want to show the existence of an $n \in \mathbb{N}$ such that $a < n$)\\
Then: $n \le a$\hspace{1 cm}$\forall n \in \mathbb{N}$\\
Now... it seems that there's nothing that we can do which will lead to any sort of contradiction.\\~\\
$\dots$ To be \textbf{\textit{completed.}}\footnote{pun intended.}
}

\hrulefill

If you do feel that you have a valid proof based only the axioms stated until now, ask a professor about its validity. If your reasoning is anywhere involving ``if $\mathbb{N}$ is bounded, then it must have a largest element", note that that reasoning is not correct. (Why?)

\hrulefill

\exercise{%
Let $S$ be an ordered set.\\
Define what it means for $A \subset S$ to be bounded below.\\
Define what a lower bound for $A$ in $S$ means.\\
(Note that the aim here is for these definitions to be analogous to the upper bound ones, so don't just come up with \textit{any} definition.)}

\hrulefill

\example{%
$A = \{5, 25, 391, 1667, 200899\}$\\
What is the set of all upper bounds ($U$) of $A$ in $\mathbb{R}$?\\
Ans: $U = \{x \in \mathbb{R} | 200899 \le x\}$
}

\hrulefill

\exercise{%
$A = \left\{x\in\mathbb{Q}\bigg|x < \dfrac{1}{2}\right\}$\\
$U = \left\{x\in\mathbb{R}\bigg|\dfrac{1}{2} \le x \right\}$\\
Show that $U$ is set of all upper bounds of $A$ in $\mathbb{R}$.
}

\exercise{\label{ex:unboundedN}Show that $\mathbb{N}$ is not bounded above in $\mathbb{N}$.}

\exercise{Show that $\mathbb{N}$ is not bounded above in $\mathbb{Q}$.}

\hrulefill
\newpage
\section{Completeness Axiom}\label{sec:complete}
$\mathbb{R}$ satisfies the completeness axiom which states that-
\axiom{Any nonempty subset of $\mathbb{R}$ that is bounded above has a least upper bound.}

\hrulefill

With the above, we have now concluded all the axioms of $\mathbb{R}$.