\prs{18-02-2019}
In Lecture 13, we showed the existence of the n$^\text{th}$ roots of positive real numbers. In particular this means that there exists a real number $x$ such that $x^2 = 2$. We had previously shown that no such rational number existed. So, we have finally shown the existence of a real number which is \textit{not} rational. This means that $\mathbb{Q}$ is indeed a proper subset of $\mathbb{R}$, that is: $\mathbb{Q}\subsetneq\mathbb{R}$.\\
We have also shown that $\mathbb{Q}$ is \hyperref[sec:ratdense]{dense} in $\mathbb{R}$. This means that even though there are real numbers which are not rational numbers, there are still \textit{lots} of rationals.\\
Let us now define a new subset of $\mathbb{R}$, irrational numbers:\\
$\mathbb{Q}^c:=\mathbb{R}\setminus\mathbb{Q}$ = set of irrational numbers\\
We have shown that $\mathbb{Q}^c\neq\emptyset$ as $\sqrt{2}\in\mathbb{Q}^c$.\\~\\
Note regarding square root: We had proven that that for every positive real number $\alpha$, there exists a real number $\beta$ which is positive such that $\beta^2=\alpha$. It is not tough to show that such a $\beta$ will be the \textit{unique} positive real number satisfying $x^2 = \alpha$. Therefore, it makes sense to denote $\beta$ as $\sqrt{\alpha}$. Also, $\sqrt{0} = 0$.
\section{Density of irrationals}\label{sec:irratdense}
\highlight{\textbf{Theorem:} Between any two (distinct) real numbers, there exists an irrational numbers.}\\
Before we prove the theorem, here are some lemmas:\\
Lemma 1. If $x\in\mathbb{Q}\setminus\{0\}$, then $x^{-1}\in\mathbb{Q}\setminus\{0\}$\\
Lemma 2. If $x\in\mathbb{Q}\setminus\{0\}$ and $r\in\mathbb{Q}^c$, then $rx\in\mathbb{Q}^c$.\\
\textit{Proofs are trivial and left as an exercise for the reader.}\\~\\
Proof of theorem:
Let $x, y \in \mathbb{R}, x < y$\\~\\
$\implies \dfrac{x}{\sqrt{2}} < \dfrac{y}{\sqrt{2}}$\hfill$[\because\sqrt{2}>0]$\\~\\
By density of rationals, $\exists q\in\mathbb{Q}$ such that:\\~\\
$\implies \dfrac{x}{\sqrt{2}} < q < \dfrac{y}{\sqrt{2}}$\\~\\
Once again, by density of rationals, $\exists q'\in\mathbb{Q}$ such that:\\~\\
$\implies \dfrac{x}{\sqrt{2}} < q < q' < \dfrac{y}{\sqrt{2}}$\\~\\
$\implies x < \sqrt{2}q < \sqrt{2}q' < y$\\
As $q < q'$, at least one of $q$ or $q'$ will be non-zero.\\
As $\sqrt{2}\in\mathbb{Q}^c$, at least one of $\sqrt{2}q$ or $\sqrt{2}q'$ will be irrational. \hfill \qed\\~\\
What this shows is that not only is $\mathbb{R}$ a proper superset of $\mathbb{Q}$, there are \textit{lots} of real numbers which are not rational.
\section{Intervals}\label{sec:intervals}
A closed interval in $\mathbb{R}$ is a set of the form $\{x\in\mathbb{R}|a\le x\le b\}$ denoted as $[a, b]$ where $a\le b$.\\
An open interval in $\mathbb{R}$ is a set of the form $\{x\in\mathbb{R}|a < x < b\}$ denoted as $(a, b)$ where $a\le b$.\\
Let us look at the following intersections:
\begin{enumerate}[nosep]
    \item $\displaystyle\bigcap_{n=1}^\infty\left(a-\dfrac{1}{n},a+\dfrac{1}{n}\right)$
    \item $\displaystyle\bigcap_{n=1}^\infty\left[a-\dfrac{1}{n},a+\dfrac{1}{n}\right]$
    \item $\displaystyle\bigcap_{n=1}^\infty\left(a,a+\dfrac{1}{n}\right)$
    \item $\displaystyle\bigcap_{n=1}^\infty\left[a,a+\dfrac{1}{n}\right]$
\end{enumerate}
The notation $\displaystyle\bigcap_{n=1}^\infty$ refers to $\displaystyle\bigcap_{n\in\mathbb{N}}$ which has been previously defined.\\
Try to think what the above intersections could be.\\
A helpful way to think could be to draw a number line and see how those sets look. That would also lead to the observation that in each case, every interval is a subset of the previous one. Such a collection of intervals is known as a sequence of nested intervals.\\
To put it more clearly: a sequence of nested intervals is a collection of sets of real numbers $I_n$ such that $\forall n\in \mathbb{N}$, $I_n$ is an interval and $I_{n+1}\subset I_n$.\\~\\
To answer the above questions, we would have to think of element(s) that would be present in every single interval mentioned. It might be natural to come up with the following answers:
\begin{enumerate}[nosep]
    \item $\{a\}$
    \item $\{a\}$
    \item $\emptyset$
    \item $\{a\}$
\end{enumerate}
Let us try to prove $\displaystyle \bigcap_{n=1}^\infty \left[a - \dfrac{1}{n}, a+\dfrac{1}{n}\right]=\{a\}$\\
\textit{Proof.} It is easy to see that $a\in\displaystyle \bigcap_{n=1}^\infty \left[a - \dfrac{1}{n}, a+\dfrac{1}{n}\right]$ as $a-\dfrac{1}{n} \le a \le a+\dfrac{1}{n}$ $\forall n \in \mathbb{N}$\\
To show that nothing else belongs to the intersection:\\
Let $b (\neq a) \in \mathbb{R}$.\\
Define $\epsilon := |b-a|$.\\
$b\neq a\implies \epsilon > 0$\\
$\exists n_0 \in\mathbb{N}$ such that $\dfrac{1}{n_0} < \epsilon$ \hfill [Archimedean Property]\\~\\
$\implies \dfrac{1}{n_0} < |b-a|$\\~\\
$\implies \sim\left(|b-a| \le \dfrac{1}{n_0}\right)$\\~\\
$\implies \sim\left(-\dfrac{1}{n_0} \le b-a \le \dfrac{1}{n_0}\right)$\\~\\
$\implies \sim\left(a-\dfrac{1}{n_0}\le b \le \dfrac{1}{n_0}\right)$\\~\\
$\implies \sim\left(b\in\left[a-\dfrac{1}{n_0}, a + \dfrac{1}{n_0}\right]\right)$\\~\\
$\implies b \not \in \left[a-\dfrac{1}{n_0}, a+\dfrac{1}{n_0}\right]$\\~\\
$\therefore b\not\in\displaystyle \bigcap_{n=1}^\infty \left[a - \dfrac{1}{n}, a+\dfrac{1}{n}\right]$\hfill\qed\\
The other examples can be proven similarly.\\
From the examples, it can be observed that in both the cases of closed intervals, the intersection was non-empty. However, this was not the case with open intervals. We shall now state and prove the following theorem.
\section{Nested Interval Theorem}\label{sec:nestedint}
Let $\{I_n\}_{n\in\mathbb{N}}$ be a collection of closed intervals in $\mathbb{R}$ such that $\forall n\in\mathbb{N}\big(I_{n+1}\subset I_n\big)$.\\
Then, $\displaystyle\bigcap_{n\in\mathbb{N}}I_n\neq\emptyset$\\~\\
\textit{Proof.}\\
Given: $\{I_n\}_{n\in\mathbb{N}}$ is a collection of closed intervals.\\
$\therefore I_n = [a_n, b_n]$ where $a_n, b_n\in\mathbb{R}, a_n\le b_n$\\~\\
Given: $[a_{n+1}, b_{n+1}]\subset[a_n, b_n]$\\
$\implies a_n \le a_{n+1} \le b_{n+1} \le b_n$\\~\\
Define $A:= \{a_n|n\in\mathbb{N}\}$\\
$A\neq\emptyset$ as $a_1\in A$.\\
$A$ is bounded above as $\forall n \in \mathbb{N}\big(a_n < b_1\big)$\\~\\
$\therefore \exists \lub(A) = \alpha \in \mathbb{R}$.\\
$\therefore \forall n \in \mathbb{N}\big(a_n \le \alpha\big)$\\~\\
Claim: $\alpha \in I_k$ $\forall k\in \mathbb{N}$\\
Proof: Let $k\in\mathbb{N}$\\~\\
\(\begin{array}{lrr}
J_n \subset J_k &\forall n \ge k &\\
\implies a_k \le a_n \le b_n \le b_k &\forall n\ge k &-(1)\\
J_k \subset J_l & \forall l \le k&\\
\implies a_l \le a_k \le b_k \le b_l &\forall l\le k &-(2)\\
\end{array}\)\\~\\
(1) and (2) $\implies a_m \le b_k$ $\forall m \in \mathbb{N}$
$\implies b_k$ is an upper bound of $A$.\\
As $\alpha$ is the $\lub$ of $A$, $\alpha \le b_k$ \hfill [By definition of $\lub$]\\
$\implies a_k \le \alpha  \le b_k$\\
$\implies \alpha \in [a_k, b_k]$\\
$\implies \alpha \in I_k$\\~\\
$\forall k \in \mathbb{N}\big(\alpha \in I_k\big) \implies \alpha \in \displaystyle\bigcap_{k\in\mathbb{N}}I_k$\\
$\therefore \displaystyle\bigcap_{k\in\mathbb{N}}I_k \neq \emptyset$ \hfill \qed\\
(Where does the proof break for open intervals?)