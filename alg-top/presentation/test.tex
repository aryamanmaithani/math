\documentclass[handout, dvipsnames]{beamer}
\mode<presentation>{}
\usepackage[utf8]{inputenc}
\usepackage{amsmath, amssymb, amsfonts, amsthm, mathtools, mathrsfs}
\setbeamertemplate{theorems}[numbered]
\title{Galois correspondence in Algebraic Topology}
\author{Aryaman Maithani}
\date[16-10-2020]{16th October 2020}
\institute[IITB]{IIT Bombay}
\usetheme{Warsaw}
\usepackage{parskip}
\usepackage{tcolorbox}
\usepackage{tikz-cd}
\tikzset{
    invisible/.style={opacity=0},
    visible on/.style={alt={#1{}{invisible}}},
    alt/.code args={<#1>#2#3}{%
      \alt<#1>{\pgfkeysalso{#2}}{\pgfkeysalso{#3}}%
  }
}
\setbeamercolor{footline}{fg=blue}
\setbeamerfont{footline}{series=\bfseries}
\addtobeamertemplate{navigation symbols}{}{%
    \usebeamerfont{footline}%
    \usebeamercolor[fg]{footline}%
    \hspace{1em}%
    \insertframenumber/\inserttotalframenumber
}
% \usepackage{tikz}
% \usecolortheme{beetle}
% \usepackage{graphicx}
\let\emptyset\varnothing

\newcommand{\id}{\operatorname{id}}
% \renewcommand{\exp}{\operatorname{exp}}

\theoremstyle{definition}
\newtheorem{defn}{Definition}
\newtheorem{thm}{Theorem}
\newtheorem{prop}[thm]{Proposition}
\newtheorem{cor}[thm]{Corollary}
\begin{document}
\begin{frame}
    \titlepage
\end{frame}

\begin{frame}{Galois correspondence: Proof (contd.)}
    \textbf{Existence.} \uncover<2->{Let $(\tilde{X}, \widetilde{x_0})$ be the universal covering space. }\uncover<3->{Let $G$ be the group of covering transformations.}\\
    \uncover<4->{Recall the isomorphism ${\color{ForestGreen}\Phi}:\pi_1(X, x_0) \to G.$ }\uncover<5->{Let $H' = \Phi(H).$ }\uncover<6->{$H'$ acts evenly on $\tilde{X}.$ }

    \uncover<7->{Consider $(E, e_0) \vcentcolon= (\tilde{X}/H', H'\widetilde{x_0}).$ }\uncover<8->{Now, we get an induced covering map $p$ as follows. }

    \uncover<9->{\begin{center}
        \begin{tikzcd}[ampersand replacement=\&]
            {(\tilde{X}, \widetilde{x_0})} \arrow[dd, "\pi"'] \arrow[rrdd, "q"] \& \&\\
            \&  \&\\
            {(E, e_0)} \arrow[rr, "p"'] \& \& {(X, x_0)}
        \end{tikzcd}
    \end{center} }
\end{frame}
\begin{frame}{Galois correspondence: Proof (contd.)}
    We now show that $p_*\pi_1(E, e_0) = H.$

    \uncover<2->{$(\supset)$} \uncover<3->{Let $[\sigma] \in H.$ }\uncover<4->{$\Phi([\sigma]) \in H'.$ }

    \uncover<5->{Let $\tilde{\sigma}$ be a lift to $\tilde{X}$ with $\tilde{\sigma}(0) = \widetilde{x_0}.$ }\uncover<6->{Put $\widetilde{x_1} \vcentcolon= \tilde{\sigma}(1).$ }

    \uncover<7->{Then, there is a homeomorphism $h \in H'$ such that $h(\widetilde{x_0}) = \widetilde{x_1}.$ }\uncover<8->{Thus, $H'\widetilde{x_0} = H'\widetilde{x_1}$ and hence, $\pi\circ\tilde{\sigma}$ is a loop. }\uncover<9->{But we have $p\circ\pi\circ\tilde{\sigma} = q\circ\tilde{\sigma} = \sigma.$ }

    \uncover<10->{Thus, $[\sigma] = p_*([\pi\circ\tilde{\sigma}]) \in p_*\pi_1(E, e_0).$ }
\end{frame}
\begin{frame}{Galois correspondence: Proof (contd.)}
    $(\subset)$ \uncover<2->{Consider a loop $\tau$ in $E$ at $e_0.$ }\uncover<3->{We wish to show $[p\circ\tau] \in H.$ }

    \uncover<4->{Let $\sigma = p\circ\tau.$ }\uncover<5->{This is a loop in $X$ at $x_0.$ Consider a lift $\tilde{\sigma}$ in $\tilde{X}$ with $\tilde{\sigma}(0) = \widetilde{x_0}.$ }

    \uncover<6->{Now, note that $\tau = \pi\circ\tilde{\sigma}$ is a loop in $E.$ }\uncover<7->{Thus, $\pi(\widetilde{x_0}) = \pi(\widetilde{x_1}).$ }\uncover<8->{This tells us that $H'\widetilde{x_0} = H'\widetilde{x_1}.$ }

    \uncover<9->{In other words, there exists $h \in H'$ such that $h(\widetilde{x_0}) = \widetilde{x_1}.$ }\uncover<10->{Thus, $\Phi([\sigma]) \in H'$ or $[\sigma] \in H,$ }\uncover<11->{as desired. }\hfill\uncover<12->{$\qed$ }
\end{frame}
\begin{frame}{Galois correspondence: Proof (contd.)}
    We now show that $p_*\pi_1(E, e_0) = H.$

    Let $[\sigma]$ be an arbitrary element of $\pi_1(X, x_0).$ 

    Consider the lift $\tilde{\sigma}$ in $\tilde{X}$ with $\tilde{\sigma}(0) = \widetilde{x_0}.$ Put $\widetilde{x_1} \vcentcolon= \tilde{\sigma}(1).$

    Let $\tau = \pi\circ\tilde{\sigma}.$ This is a path in $E$ starting at $e_0.$ Moreover, it is the unique lift of $\sigma$ starting at $e_0.$ \\
    Thus, if $\tau$ is a loop, then $[\sigma] = p_*([\tau]) \in p_*\pi_1(E, e_0).$ \\
    More importantly, if $\tau$ is \emph{not} a loop, then $[\sigma] \notin p_*\pi_1(E, e_0).$

    Note that $\tau$ is a loop iff $\pi_1(\widetilde{x_0}) = \pi_1(\widetilde{x_1})$ iff there is a homeomorphism $h \in H'$ such that $h(\widetilde{x_0}) = \widetilde{x_1}$ iff $\Phi([\sigma]) \in H'$ iff $[\sigma] \in H.$

    Thus, $[\sigma] \in H \iff [\sigma] \in p_*\pi_1(E, e_0).$
\end{frame}
\end{document}